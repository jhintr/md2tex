\chapter{Lesson 9}

\section*{Reading 1}

Ekaṃ samayaṃ Bhagavā Bhoganagare viharati Ānandacetiye. Tatra kho Bhagavā bhikkhū āmantesi “Bhikkhavo” ti. “Bhadante” ti te bhikkhū Bhagavato paccassosuṃ. Bhagavā etadavoca “Cattāro’me, bhikkhave, mahāpadese desessāmi, taṃ suṇātha, sādhukaṃ manasikarotha, bhāsissāmī” ti. “Evaṃ, bhante” ti kho te bhikkhū Bhagavato paccassosuṃ. Bhagavā etadavoca

“Katame, bhikkhave, cattāro mahāpadesā? Idha, bhikkhave, bhikkhu evaṃ vadeyya ‘Sammukhā m’etaṃ, āvuso, Bhagavato sutaṃ, sammukhā paṭiggahitaṃ ayaṃ dhammo, ayaṃ vinayo, idaṃ satthusāsanan’ ti. Tassa, bhikkhave, bhikkhuno bhāsitaṃ n’eva abhinanditabbaṃ nappaṭikkositabbaṃ. Anabhinanditvā appaṭikkositvā tāni padabyañjanāni sādhukaṃ uggahetvā sutte otāretabbāni, vinaye sandassetabbāni. Tāni ce sutte otāriyamānāni vinaye sandassiyamānāni na c’eva sutte otaranti na vinaye sandissanti, niṭṭham ettha gantabbaṃ ‘Addhā, idaṃ na c’eva tassa Bhagavato vacanaṃ Arahato Sammāsambuddhassa… ’ ti. Iti h’etaṃ, bhikkhave, chaḍḍeyyātha.”

“Idha pana, bhikkhave, bhikkhu evaṃ vadeyya ‘Sammukhā m’etaṃ, āvuso, Bhagavato sutaṃ, sammukhā paṭiggahitaṃ ayaṃ dhammo, ayaṃ vinayo, idaṃ satthusāsanan’ ti. Tassa, bhikkhave, bhikkhuno bhāsitaṃ n’eva abhinanditabbaṃ nappaṭikkositabbaṃ. Anabhinanditvā appaṭikkositvā tāni padabyañjanāni sādhukaṃ uggahetvā sutte otāretabbāni, vinaye sandassetabbāni. Tāni ce sutte otāriyamānāni vinaye sandassiyamānāni sutte c’eva otaranti vinaye ca sandissanti, niṭṭham ettha gantabbaṃ ‘Addhā, idaṃ tassa Bhagavato vacanaṃ Arahato Sammāsambuddhassa… ’ ti. Idaṃ, bhikkhave, paṭhamaṃ mahāpadesaṃ dhāreyyātha.” \hfill(A 4.18.10)

\section*{Reading 2}

Ahaṃ kho, bhikkhave, ekāsanabhojanaṃ bhuñjāmi ekāsanabhojanaṃ kho, ahaṃ, bhikkhave, bhuñjamāno appābādhataṃ ca sañjānāmi appātaṅkataṃ ca lahuṭṭhānaṃ ca balaṃ ca phāsuvihāraṃ ca. Etha, tumhe pi, bhikkhave, ekāsanabhojanaṃ bhuñjatha, ekāsanabhojanaṃ kho, bhikkhave, tumhe pi bhuñjamānā appābādhataṃ ca sañjānissatha appātaṅkataṃ ca lahuṭṭhānaṃ ca balaṃ ca phāsuvihārañcā ti. \hfill(M 2.2.5)

\section*{Reading 3}

Pāpañ ce puriso kayirā,\\
na naṃ kayirā punappunaṃ,\\
na tamhi chandaṃ kayirātha,\\
dukkho pāpassa uccayo.

Puññaṃ ce puriso kayirā,\\
kayirā naṃ punappunaṃ,\\
tamhi chandaṃ kayirātha,\\
sukho puññassa uccayo.

Pāpo pi passati bhadraṃ,\\
yāva pāpaṃ na paccati,\\
yadā ca paccati pāpaṃ,\\
atha pāpo pāpāni passati.

Bhadro pi passati pāpaṃ,\\
yāva bhadraṃ na paccati,\\
yadā ca paccati bhadraṃ,\\
atha bhadro bhadrāni passati.

Pāṇimhi ce vaṇo nāssa,\\
hareyya pāṇinā visaṃ,\\
nābbaṇaṃ visam anveti,\\
natthi pāpaṃ akubbato.

Gabbhaṃ eke uppajjanti,\\
nirayaṃ pāpakammino,\\
saggaṃ sugatino yanti,\\
parinibbanti anāsavā. \hfill(Dhp 9)

\section*{Further Reading 1}

Evaṃ me sutaṃ. Ekaṃ samayaṃ Bhagavā Rājagahe viharati Veḷuvane Kalandakanivāpe. Tena kho pana samayena Sigālako gahapatiputto kālass’eva uṭṭhāya Rājagahā nikkhamitvā allavattho allakeso pañjaliko puthudisā namassati, puratthimaṃ disaṃ dakkhiṇaṃ disaṃ pacchimaṃ disaṃ uttaraṃ disaṃ heṭṭhimaṃ disaṃ uparimaṃ disaṃ.

Atha kho Bhagavā pubbaṇhasamayaṃ nivāsetvā pattacīvaram ādāya Rājagahaṃ piṇḍāya pāvisi. Addasā kho Bhagavā Sigālakaṃ gahapatiputtaṃ kālass’eva vuṭṭhāya Rājagahā nikkhamitvā allavatthaṃ allakesaṃ pañjalikaṃ puthudisā namassantaṃ puratthimaṃ disaṃ dakkhiṇaṃ disaṃ pacchimaṃ disaṃ uttaraṃ disaṃ heṭṭhimaṃ disaṃ uparimaṃ disaṃ. Disvā Sigālakaṃ gahapatiputtaṃ etadavoca “kinnu kho tvaṃ, gahapatiputta, kālass’eva uṭṭhāya Rājagahā nikkhamitvā allavattho allakeso pañjaliko puthudisā namassasi puratthimaṃ disaṃ dakkhiṇaṃ disaṃ pacchimaṃ disaṃ uttaraṃ disaṃ heṭṭhimaṃ disaṃ uparimaṃ disan” ti?

“Pitā maṃ, bhante, kālaṃ karonto evaṃ avaca ‘disā, tāta, namasseyyāsī’ ti. So kho ahaṃ, bhante, pitu vacanaṃ sakkaronto garuṃ karonto mānento pūjento kālass’eva uṭṭhāya Rājagahā nikkhamitvā allavattho allakeso pañjaliko puthudisā namassāmi puratthimaṃ disaṃ dakkhiṇaṃ disaṃ pacchimaṃ disaṃ uttaraṃ disaṃ heṭṭhimaṃ disaṃ uparimaṃ disan” ti.

“Na kho, gahapatiputta, ariyassa vinaye evaṃ cha disā namassitabbā” ti.

“Yathā kathaṃ pana, bhante, ariyassa vinaye cha disā namassitabbā? Sādhu me, bhante, Bhagavā tathā dhammaṃ desetu, yathā ariyassa vinaye cha disā namassitabbā” ti.

“Tena hi, gahapatiputta, suṇohi sādhukaṃ manasikarohi bhāsissāmī” ti. “Evaṃ, bhante” ti kho Sigālako gahapatiputto Bhagavato paccassosi. Bhagavā etadavoca

“Yato kho, gahapatiputta, ariyasāvakassa cattāro kammakilesā pahīnā honti, catūhi ca ṭhānehi pāpakammaṃ na karoti, cha ca bhogānaṃ apāyamukhāni na sevati, so evaṃ cuddasa pāpakāpagato chaddisā paṭicchādī ubholokavijayāya paṭipanno hoti. Tassa ayañ c’eva loko āraddho hoti paro ca loko. So kāyassa bhedā paraṃ maraṇā sugatiṃ saggaṃ lokaṃ upapajjati.” \hfill(D 3.8 Siṅgālakasuttaṃ)

\section*{Further Reading 2}

Atha kho, bhikkhave, Vipassissa Bhagavato arahato Sammāsambuddhassa etadahosi “yan nūnāhaṃ dhammaṃ deseyyan” ti. Atha kho, bhikkhave, Vipassissa Bhagavato arahato Sammāsambuddhassa etadahosi “adhigato kho myāyaṃ dhammo gambhīro duddaso duranubodho santo paṇīto atakkāvacaro nipuṇo paṇḍitavedanīyo. Ālayarāmā kho panāyaṃ pajā ālayaratā ālayasammuditā. Ālayarāmāya kho pana pajāya ālayaratāya ālayasammuditāya duddasaṃ idaṃ ṭhānaṃ yadidaṃ idappaccayatāpaṭiccasamuppādo. Idam pi kho ṭhānaṃ duddasaṃ yadidaṃ sabbasaṅkhārasamatho sabbūpadhipaṭinissaggo taṇhākkhayo virāgo nirodho nibbānaṃ. Ahañ c’eva kho pana dhammaṃ deseyyaṃ, pare ca me na ājāneyyuṃ, so mam assa kilamatho, sā mam assa vihesā” ti. \hfill(D 2.1 Mahāpadānasuttaṃ)

\section*{Further Reading 3}

Ko imaṃ pathaviṃ vijessati,\\
yamalokañ ca imaṃ sadevakaṃ?\\
ko dhammapadaṃ sudesitaṃ,\\
kusalo puppham iva pacessati?

Sekho pathaviṃ vijessati,\\
yamalokañ ca imaṃ sadevakaṃ,\\
sekho dhammapadaṃ sudesitaṃ,\\
kusalo puppham iva pacessati.

Pheṇūpamaṃ kāyamimaṃ viditvā,\\
marīcidhammaṃ abhisambudhāno,\\
chetvāna mārassa papupphakāni,\\
adassanaṃ maccurājassa gacche. \hfill(Dhp 4)

Yo bālo maññati bālyaṃ,\\
paṇḍito vāpi tena so,\\
bālo ca paṇḍitamānī,\\
sa ve “bālo” ti vuccati. \hfill(Dhp 5)