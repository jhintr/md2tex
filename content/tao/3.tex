\chapter{卷之三\hspace{1ex}詩五言}

\section{始作鎮軍參軍經曲阿}

\textbf{弱齡寄事外,委懷在琴書。被褐欣自得,屢空\footnote{論語先進:回也其庶乎,屢空。}常晏如。時來茍冥會\footnote{李善:富貴榮寵,時之暫來也。},踠轡憩通衢\footnote{李善:言屈長往之駕,息於通衢之中,通衢,仕路也。}。投策命晨裝,暫與園田疏。眇眇孤舟逝,綿綿歸思紆。我行豈不遙,登降千里餘。目倦川塗異,心念山澤居。望雲慚高鳥,臨水愧游魚。真想初在襟,誰謂形跡拘。聊且憑化遷,終返班生廬。}

\section{庚子歲五月中從都還阻風於規林二首}

\textbf{行行循歸路,計日望舊居。一欣侍溫顏,再喜見友于。鼓棹路崎曲,指景限西隅。江山豈不險,歸子念前塗。凱風負我心,戢枻\footnote{枻 \texttt{yì}:船槳。}守窮湖。高莽眇無界,夏木獨森疏。誰言客舟遠,近瞻百里餘。延目識南嶺\footnote{南嶺:指廬山。},空歎將焉如。}

\textbf{自古歎行役,我今始知之。山川一何曠,巽坎難與期。崩浪聒天響,長風無息時。久遊戀所生,如何淹在茲。靜念園林好,人間良可辭。當年\footnote{當年:晉人多以壯年為當年,張華賦「惟幼眇之當年」是也。}詎有幾,縱心復何疑。}

\section{辛丑歲七月赴假還江陵夜行塗口}

\textbf{閑居三十載\footnote{三十載:當作三二載,即六年。},遂與塵事冥。詩書敦宿好,林園無世情。如何捨此去,遙遙至南荊。叩枻新秋月,臨流別友生。涼風起將夕,夜景湛虛明。昭昭天宇闊,皛皛\footnote{皛皛 \texttt{xiǎo}:皎潔明亮。}川上平。懷役不遑寐,中宵尚孤征。商歌\footnote{淮南子道應訓:甯戚商歌車下,而桓公慨然而悟。言干祿也。}非吾事,依依在耦耕。投冠旋舊墟,不為好爵縈。養真衡茅下,庶以善自名。}

\section{癸卯歲始春懷古田舍二首}

\textbf{在昔聞南畝,當年竟未踐。屢空既有人,春興豈自免。夙晨裝吾駕,啟塗情已緬。鳥哢歡新節,泠風送餘善。寒草被荒蹊,地為罕人遠。是以植杖翁,悠然不復返。即理愧通識\footnote{魏晉之際,所謂通字,從後論之,每不為佳號,晉書傅玄傳「魏文慕通達而天下賤守節」,陶公所謂通識,殆即此流耳。},所保\footnote{後漢書逸民傳:龐公者,劉表就候之,曰「夫保全一身,孰若保全天下乎」,龐公笑曰云云。}詎乃淺。}

\textbf{先師有遺訓,憂道不憂貧。瞻望邈難逮,轉欲患長勤。秉耒歡時務,解顏勸農人。平疇交遠風,良苗亦懷新。雖未量歲功,即事多所欣。耕種有時息,行者無問津\footnote{問津:案用孔子使子路問津典,言時無賢人過從也。}。日入相與歸,壺漿勞近鄰。長吟掩柴門,聊為隴畝民。}

\section{癸卯歲十二月中作與從弟敬遠}

\textbf{寢迹衡門下,邈與世相絕。顧眄莫誰知,荊扉晝常閉。淒淒歲暮風,翳翳經日雪。傾耳無希聲,在目皓已潔。勁氣侵襟袖,簞瓢謝屢設。蕭索空宇中,了無一可悅。歷覽千載書,時時見遺烈。高操非所攀,謬得固窮節。平津\footnote{平津:大道。}苟不由,棲遲詎為拙。寄意一言外,茲契誰能別。}

\section{乙巳歲三月為建威參軍使都經錢溪}

\textbf{我不踐斯境,歲月好已積。晨夕看山川,事事悉如昔。微雨洗高林,清飇矯雲翮。眷彼品物存,義風都未隔\footnote{趙泉山曰:此詩大旨,慶遇安帝光復大業,不失舊物也。}。伊余何為者,勉勵從茲役。一形似有制,素襟不可易。園田日夢想,安得久離析。終懷在壑舟,諒哉宜霜柏。}

\section{還舊居}

\textbf{疇昔家上京,六載去還歸。今日始復來,惻愴多所悲。阡陌不移舊,邑屋或時非。履歷周故居,鄰老罕復遺。步步尋往迹,有處特依依。流幻百年中,寒暑日相推。常恐大化盡,氣力不及衰\footnote{禮記王制:五十始衰。}。撥置且莫念,一觴聊可揮。}

\section{戊申歲六月中遇火}

\textbf{草廬寄窮巷,甘以辭華軒。正夏長風急,林室頓燒燔。一宅無遺宇,舫舟蔭門前。迢迢新秋夕,亭亭月將圓。果菜始復生,驚鳥尚未還。中宵竚遙念,一盼周九天。總髮抱孤介,奄出四十年。形迹憑化往,靈府長獨閑。貞剛自有質,玉石乃非堅。仰想東戶\footnote{困學紀聞引子思子曰:東戶季子之時,道上雁行而不拾遺,餘糧棲諸畝首。}時,餘糧宿中田。鼓腹無所思,朝起暮歸眠。既已不遇茲,且遂灌我園。}

\section{己酉歲九月九日}

\textbf{靡靡秋已夕,淒淒風露交。蔓草不復榮,園木空自凋。清氣澄餘滓,杳然天界高。哀蟬無留響,叢雁鳴雲霄。萬化相尋繹\footnote{尋繹:往復、更替。},人生豈不勞。從古皆有沒,念之中心焦。何以稱我情,濁酒且自陶。千載非所知,聊以永今朝。}

\section{庚戌歲九月中於西田穫早稻}

\textbf{人生歸有道,衣食固其端。孰是都不營,而以求自安。開春理常業,歲功聊可觀。晨出肆微勤\footnote{後漢書周燮傳:下有陂田,常肆勤以自給。},日入負禾還。山中饒霜露,風氣亦先寒。田家豈不苦,弗獲辭此難。四體誠乃疲,庶無異患干。盥濯息簷下,斗酒散㦗顏。遙遙沮溺心,千載乃相關。但願長如此,躬耕非所歎。}

\section{丙辰歲八月中於下潠田舍穫}

\textbf{貧居依稼穡,戮力東林隈。不言春作苦,常恐負所懷。司田眷有秋,寄聲與我諧。飢者歡初飽,束帶候鳴雞。揚檝越平湖,汎隨清壑迴。皭皭\footnote{皭:白貌。}荒山裏,猿聲閑且哀。悲風愛靜夜,林鳥喜晨開。曰余作此來,三四星火頹\footnote{星火:火星,言十二年矣。}。姿年逝已老,其事未云乖。遙謝荷蓧翁,聊得從君栖。}

\section{飲酒二十首\hspace{1ex}{\footnotesize 并序}}

\begin{quoting}余閑居寡歡,兼比夜已長,偶有名酒,無夕不飲,顧影獨盡,忽焉復醉,既醉之後,輒題數句自娛,紙墨遂多,辭無詮次,聊命故人書之,以為歡笑爾。\end{quoting}

\begin{quoting}\textbf{其一}\end{quoting}

\textbf{衰榮無定在,彼此更共之。邵生\footnote{史記蕭相國世家:召平,故秦東陵侯,秦破為布衣,貧,種瓜長安城東,瓜美,故世俗謂之東陵瓜。}瓜田中,寧似東陵時。寒暑有代謝,人道每如茲。達人解其會,逝將不復疑。忽與一觴酒,日夕歡相持。}

\begin{quoting}\textbf{其二}\end{quoting}

\textbf{積善云有報,夷叔在西山。善惡茍不應,何事空立言。九十行帶索\footnote{九十行帶索:榮啟期行年九十,鹿皮帶索,鼓琴而歌。},飢寒況當年\footnote{當年:壯年。}。不賴固窮節,百世當誰傳。}

\begin{quoting}\textbf{其三}\end{quoting}

\textbf{道喪向千載,人人惜其情。有酒不肯飲,但顧世間名。所以貴我身,豈不在一生。一生復能幾,倏如流電驚。鼎鼎\footnote{禮記檀弓上:鼎鼎爾,則小人。注:鼎鼎,謂大舒。}百年內,持此欲何成。}

\begin{quoting}\textbf{其四}\end{quoting}

\textbf{栖栖失群鳥,日暮猶獨飛。徘徊無定止,夜夜聲轉悲。厲響思清晨,遠去何所依。因值孤生松,斂翮遙來歸。勁風無榮木,此蔭獨不衰。託身已得所,千載不相違。}

\begin{quoting}\textbf{其五}\end{quoting}

\textbf{結廬在人境,而無車馬喧。問君何能爾,心遠地自偏。採菊東籬下,悠然見南山。山氣日夕佳,飛鳥相與還。此中有真意,欲辨已忘言。}

\begin{quoting}\textbf{其六}\end{quoting}

\textbf{行止千萬端,誰知非與是。是非茍相形,雷同共譽毀。三季多此事,達士似不爾。咄咄俗中愚,且當從黃綺。}

\begin{quoting}\textbf{其七}\end{quoting}

\textbf{秋菊有佳色,裛露掇其英。汎此忘憂物,遠我遺世情。一觴雖獨進,杯盡壺自傾。日入群動息,歸鳥趨林鳴。嘯傲東軒下,聊復得此生。}

\begin{quoting}\textbf{其八}\end{quoting}

\textbf{青松在東園,眾草沒其姿。凝霜殄異類,卓然見高枝。連林人不覺,獨樹眾乃奇。提壺撫寒柯,遠望時復為。吾生夢幻間,何事紲塵羈。}

\begin{quoting}\textbf{其九}\end{quoting}

\textbf{清晨聞叩門,倒裳往自開。問子為誰歟,田父有好懷。壺漿遠見候,疑我與時乖。襤縷茅簷下,未足為高栖。一世皆尚同,願君汩其泥。深感父老言,禀氣寡所諧。紆轡誠可學,違己詎非迷。且共歡此飲,吾駕不可回。}

\begin{quoting}\textbf{其十}\end{quoting}

\textbf{在昔曾遠游,直至東海隅。道路迥且長,風波阻中塗。此行誰使然,似為飢所驅。傾身營一飽,少許便有餘。恐此非名計,息駕歸閑居。}

\begin{quoting}\textbf{其十一}\end{quoting}

\textbf{顏生稱為仁,榮公言有道。屢空不獲年,長飢至於老。雖留身後名,一生亦枯槁。死去何所知,稱心固為好。客養千金軀,臨化消其寶。裸葬\footnote{漢書楊王孫傳:及病且終,先令其子曰「吾欲臝葬,以反吾真,死則為布囊盛尸,入地七尺,既下,從足引脫其囊,以身親土」。}何必惡,人當解其表。}

\begin{quoting}\textbf{其十二}\end{quoting}

\textbf{長公\footnote{史記張釋之傳:其子曰張摯,字長公,官至大夫,免,以不能取容當世,故終身不仕。}曾一仕,壯節忽失時。杜門不復出,終身與世辭。仲理\footnote{後漢書儒林傳:楊倫,字仲理,為郡文學掾,志乖於時,遂去職,不復應州郡命,講授於大澤中,弟子至千餘人。}歸大澤,高風始在茲。一往便當已,何為復狐疑。去去當奚道,世俗久相欺。擺落悠悠談\footnote{晉書王導傳:悠悠之談,宜絕智者之口。},請從余所之。}

\begin{quoting}\textbf{其十三}\end{quoting}

\textbf{有客常同止,趣舍邈異境。一士長獨醉,一夫終年醒。醒醉還相笑,發言各不領。規規\footnote{莊子秋水:規規然自失也。又荀子非十二子注:瞡瞡,小見之貌。}一何愚,兀傲差若穎。寄言酣中客,日沒燭當炳。}

\begin{quoting}\textbf{其十四}\end{quoting}

\textbf{故人賞我趣,挈壺相與至。班荊\footnote{左傳襄二十六年:班荊相與食,而言復故。}坐松下,數斟已復醉。父老雜亂言,觴酌失行次。不覺知有我,安知物為貴。悠悠迷所留,酒中有深味。}

\begin{quoting}\textbf{其十五}\end{quoting}

\textbf{貧居乏人工,灌木荒余宅。班班有翔鳥,寂寂無行跡。宇宙一何悠,人生少至百。歲月相催逼,鬢邊早已白。若不委窮達,素抱深可惜。}

\begin{quoting}\textbf{其十六}\end{quoting}

\textbf{少年罕人事,遊好在六經。行行向不惑,淹留遂無成。竟抱固窮節,飢寒飽所更。弊廬交悲風,荒草沒前庭。披褐守長夜,晨雞不肯鳴。孟公\footnote{劉龔,字孟公。高士傳:張仲蔚,平陵人,好詩賦,常居貧素,所處蓬蒿沒人,時人莫識,惟劉龔知之。}不在茲,終以翳吾情。}

\begin{quoting}\textbf{其十七}\end{quoting}

\textbf{幽蘭生前庭\footnote{晉書謝玄傳:安嘗戒約子姪,因曰「子弟亦何豫人事,而正欲使其佳」,玄曰「譬如芝蘭玉樹,欲使其生於庭階耳」,安悦。},含薰待清風。清風脫然至,見別蕭艾\footnote{離騷:何昔日之芳草兮,今直為此蕭艾也。言惡草也。}中。行行失故路,任道或能通。覺悟當念還,鳥盡廢良弓。}

\begin{quoting}\textbf{其十八}\end{quoting}

\textbf{子雲性嗜酒,家貧無由得。時賴好事人,載醪祛所惑\footnote{漢書楊雄傳:家素貧,嗜酒,人希至其門,時有好事者載酒肴,從遊學。}。觴來為之盡,是諮無不塞。有時不肯言,豈不在伐國\footnote{漢書董仲舒傳:昔者魯公問柳下惠「吾欲伐齊,如何」,柳下惠曰「不可」,歸而有憂色曰「吾聞伐國不問仁人,此言何為至於我哉」?}。仁者用其心,何嘗失顯默。}

\begin{quoting}\textbf{其十九}\end{quoting}

\textbf{疇昔苦長飢,投耒去學仕。將養不得節,凍餒固纏己。是時向立年,志意多所恥。遂盡介然分\footnote{方望辭隗囂書:雖懷介然之節,欲潔去就之分。},終死歸田里。冉冉星氣流,亭亭復一紀。世路廓悠悠,楊朱所以止\footnote{淮南子說林訓:楊子見逵路而哭之,為其可以南可以北。}。雖無揮金\footnote{張協咏二疏:揮金樂當年。言疏廣歸老事。}事,濁酒聊可恃。}

\begin{quoting}\textbf{其二十}\end{quoting}

\textbf{羲農去我久,舉世少復真。汲汲魯中叟,彌縫使其淳。鳳鳥雖不至,禮樂暫得新。洙泗輟微響,漂流逮狂秦。詩書復何罪,一朝成灰塵。區區諸老翁\footnote{湯注:似謂漢初伏生諸人。},為事誠殷勤。如何絕世下,六籍無一親。終日馳車走,不見所問津。若復不快飲,空負頭上巾\footnote{何注:史言先生取頭上葛巾漉酒,還復著之。}。但恨多謬誤,君當恕醉人。}

\section{止酒}

\textbf{居止次城邑,逍遙自閑止。坐止高蔭下,步止蓽門裏。好味止園葵,大歡止稚子。平生不止酒,止酒情無喜。暮止不安寢,晨止不能起。日日欲止之,營衛止不理。徒知止不樂,未知止利己。始覺止為善,今朝真止矣。從此一止去,將止扶桑涘。清顏止宿容,奚止千萬祀。}

\section{述酒\hspace{1ex}{\footnotesize }}

\begin{quoting}儀狄造,杜康潤色之\end{quoting}

\textbf{重離照南陸\footnote{重離照南陸:言孝武帝在位。重離即重日,為昌字,孝武帝小字昌明。此詩原注「儀狄造,杜康潤色之」,蓋喻桓玄篡位在前,劉裕潤色於後也。},鳴鳥聲相聞。秋草雖未黃,融風久已分。素礫皛修渚\footnote{素礫:白石,對玉而言。修渚:以長洲代指江陵。此言桓玄事。},南嶽無餘雲\footnote{無餘雲:則王氣盡矣。}。豫章\footnote{豫章:指劉裕,義熙二年,裕封豫章郡公。}抗高門,重華固靈墳\footnote{重華:恭帝廢為零陵王,舜冢在零陵九疑,故云爾。固,但也,固靈墳,隱言恭帝之死。}。流淚抱中歎,傾耳聽司晨。神州獻嘉粟,西靈\footnote{西靈:當作四靈,劉裕受禪文有「四靈效徵」之語。}為我馴。諸梁董師旅,芊勝喪其身\footnote{諸梁、芊勝:即葉公、白公也,見左傳哀十六年。此指劉裕誅桓玄事,桓玄篡晉稱楚,裕亦楚人,故詩用楚事。}。山陽\footnote{山陽:漢獻帝廢為山陽公。}歸下國,成名猶不勤。卜生善斯牧\footnote{漢書卜式傳:式布衣草蹻而牧羊,上過其羊所,善之,式曰「非獨羊也,治民亦猶是也,以時起居,惡者輒去,毋令敗群」。言劉裕翦滅晉宗室也。},安樂不為君\footnote{用劉賀臣安樂不盡忠言事,託言東晉臣僚不忠晉室。}。平王\footnote{平王:桓玄篡位,以安帝為平固王。}去舊京,峽中納遺薰\footnote{遺薰:用王子搜事,言劉裕弒君而立恭帝,見莊子讓王。}。雙陽\footnote{雙陽:重日,指孝武帝。}甫云育,三趾\footnote{三趾:三足烏,晉初以為代魏祥瑞,今則為宋瑞。}顯奇文。王子愛清吹,日中翔河汾\footnote{用王子晉年十七而死,喻劉裕主政十七年而晉亡。河汾,晉地也。}。朱公練九齒\footnote{朱公:陶自謂也。九齒:長生。},閑居離世紛。峨峨西嶺\footnote{西嶺:即夷齊之西山。}內,偃息常所親。天容自永固,彭殤非等倫。}

\section{責子}

\begin{quoting}舒儼、宣俟、雍份、端佚、通佟凡五人,舒、宣、雍、端、通皆小名也。\end{quoting}

\textbf{白髮被兩鬢,肌膚不復實。雖有五男兒,總不好紙筆。阿舒已二八,懶惰故無匹。阿宣行志學,而不愛文術。雍端年十三,不識六與七。通子垂九齡,但覓梨與栗。天運茍如此,且進杯中物。}

\section{有會而作\hspace{1ex}{\footnotesize 并序}}

\begin{quoting}舊穀既沒,新穀未登,頗為老農,而值年災,日月尚悠,為患未已,登歲之功,既不可希,朝夕所資,煙火裁通,旬日已來,始念飢乏,歲云夕矣,慨然永懷,今我不述,後生何聞哉。\end{quoting}

\textbf{弱年逢家乏,老至更\footnote{更:經也。}長飢。菽麥實所羨,孰敢慕甘肥。惄如亞九飯,當暑厭寒衣。歲月將欲暮,如何辛苦悲。常善粥者心\footnote{禮記檀弓:齊大飢,黔敖為食於路,以待飢者而食之,有飢者蒙袂輯屨貿貿然來,黔敖左奉食、右執飲曰「嗟,來食」,揚其目而視之曰「予惟不食嗟來之食以至於斯也」,從而謝焉,終不食而死。},深念蒙袂非。嗟來何足吝,徒沒空自遺。斯濫\footnote{論語衛靈公:君子固窮,小人窮斯濫矣。}豈攸志,固窮夙所歸。餒也已矣夫,在昔余多師。}

\section{蜡日}

\textbf{風雪送餘運\footnote{餘運:歲暮。},無妨時已和。梅柳夾門植,一條有佳花。我唱爾言得,酒中適何多。未能明多少,章山\footnote{章山:鄣山,即石門山。}有奇歌。}

