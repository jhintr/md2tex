\chapter{八颂品第四}

\section{爱欲经}

\begin{center}Kāma Sutta\end{center}\vspace{1em}

\begin{enumerate}\item 缘起为何?据说,世尊住舍卫国时,某位婆罗门在舍卫国与祇园间阿致罗筏底河的岸边耕田,想「我要播种大麦」。世尊则为比丘僧团随从,在行乞时看到他,经转向便见到「这婆罗门将失去大麦」,而再次转向于近依的成就时,见到其须陀洹果的近依,转向于「何时能圆满」时,见到「在失去收成时,为忧伤压垮,听闻了法的开示后」。随后想道「若我那时去到婆罗门处,他将不想听我的教诫,因为婆罗门们有多种喜好,噫!从现在起我就摄受他,如是,其心于我已得调柔,那时将会听取教诫」,便去到婆罗门处,说:「婆罗门!你在做什么?」
\item 婆罗门想「家族如此高贵的沙门乔达摩来和我打招呼」,便立即对世尊心生净喜,说:「乔达摩君!我在耕田,准备播种大麦。」于是,舍利弗长老想「世尊和婆罗门打了招呼,诸如来不会无因无缘这么做,噫!我也和他打招呼」,便去到婆罗门处,也如是打了招呼。大目犍连长老和其余的八十大弟子也都如此。婆罗门便非常高兴。
\item 于是,在谷物成长时,一天,世尊食事已毕,在从舍卫国前往祇园时,从路上下来,到了婆罗门跟前,便说:「婆罗门!你的麦田长势喜人啊!」「的确长势喜人,乔达摩君!若是收成了,我要分享给你!」于是,四个月后,大麦便成熟了。正当他热心期待着「今天或明天我就收割」时,大云生起,下了整夜的雨。阿致罗筏底河泛滥,淹没了所有大麦。婆罗门整夜不悦,拂晓来到河边,看到所有谷物都已损失,便生起巨大的忧伤:「我完了!现在我该怎么活?」
\item 世尊则在这晚的黎明时分以佛眼观察世间,了知到「今天是对婆罗门开示法的时候」,便因行乞的义务进入舍卫国,站在婆罗门的家门前。婆罗门见到世尊后,想道「沙门乔达摩前来安慰被忧伤压垮的我」,便备了坐具,取了钵,让世尊坐下。世尊已知而问婆罗门:「婆罗门!是否心情不好?」「唯!乔达摩君!我所有的麦田都被水淹了。」于是,世尊说「婆罗门!不应在失去时忧伤,也不应在成就时喜悦,因为爱欲就是有得有失」,了知了这婆罗门的适宜,便因法的开示说了此经。
\item 这里,我们将只简略地解释词义与连结,而其详当知已在「\textbf{义释} \textit{Niddesa}」中所述。且此后的所有经都如此经。\end{enumerate}

\subsection\*{\textbf{773} {\footnotesize 〔PTS 766〕}}

\textbf{对于欲求着爱欲的人,若他于此成功,\\}
\textbf{有死者既得了所希望的,必然有喜意。}

Kāmaṃ kāmayamānassa, tassa ce taṃ samijjhati;\\
addhā pītimano hoti, laddhā macco yad icchati. %\hfill\textcolor{gray}{\footnotesize 1}

\begin{enumerate}\item 这里,\textbf{爱欲},即被称为适意之色等的三界法的物欲。\textbf{欲求着},即希望着。\textbf{若他于此成功},即若这欲求着被称为爱欲的事物的有情于此成功,即是说如果他得到它。\textbf{必然有喜意},即必定心满意足。\textbf{有死者},即有情。\end{enumerate}

\subsection\*{\textbf{774} {\footnotesize 〔PTS 767〕}}

\textbf{若对这欲求着、生起欲望的人,\\}
\textbf{那些爱欲消逝,他便如被箭射穿般恼坏。}

Tassa ce kāmayānassa, chandajātassa jantuno;\\
te kāmā parihāyanti, sallaviddho va ruppati. %\hfill\textcolor{gray}{\footnotesize 2}

\begin{enumerate}\item \textbf{若对这欲求着},即对这希望着爱欲的人,或为爱欲驱使着的人。\textbf{生起欲望},即生起渴爱。\textbf{人},即有情。\textbf{那些爱欲消逝},即如果那些爱欲消逝。\textbf{他便如被箭射穿般恼坏},于是,他便如被铁制等的箭射穿般坏碎。\end{enumerate}

\subsection\*{\textbf{775} {\footnotesize 〔PTS 768〕}}

\textbf{若避开爱欲,如以足避开蛇头,\\}
\textbf{他具念,超越这对世间的爱著。}

Yo kāme parivajjeti, sappasseva padā siro;\\
so’maṃ visattikaṃ loke, sato samativattati. %\hfill\textcolor{gray}{\footnotesize 3}

\begin{enumerate}\item 第三颂的略义为:然而于此,\textbf{若}以镇伏欲贪或以正断\textbf{避开}这些\textbf{爱欲},如以自己的足避开\textbf{蛇头},则此比丘既已\textbf{具念},\textbf{超越}由散布于一切世间且持存而被称为\textbf{对世间的爱著}的渴爱。\end{enumerate}

\subsection\*{\textbf{776} {\footnotesize 〔PTS 769〕}}

\textbf{田地、物品、货币,或牛马、奴仆、\\}
\textbf{妇女、亲眷等种种爱欲,若人贪求,}

Khettaṃ vatthuṃ hiraññaṃ vā, gavāssaṃ dāsaporisaṃ;\\
thiyo bandhū puthu kāme, yo naro anugijjhati. %\hfill\textcolor{gray}{\footnotesize 4}

\begin{enumerate}\item 此后的后三颂之略义为:\textbf{若人贪求}这稻田等的\textbf{田地},或俗家之物等的\textbf{物品},或被称为钱币的\textbf{货币},或牛马等类的\textbf{牛马},或被称为女子的\textbf{妇女},或亲属等的\textbf{亲眷},或其它适意之色等的\textbf{种种爱欲},\end{enumerate}

\subsection\*{\textbf{777} {\footnotesize 〔PTS 770〕}}

\textbf{则诸多无力压制他,诸多危难压迫他,\\}
\textbf{随后,苦追随他,如水之于漏船。}

Abalā naṃ balīyanti, maddante naṃ parissayā;\\
tato naṃ dukkham anveti, nāvaṃ bhinnam ivodakaṃ. %\hfill\textcolor{gray}{\footnotesize 5}

\begin{enumerate}\item 则被称为\textbf{无力}的诸多烦恼\textbf{压制}、征服、压迫这人,或者,无力的烦恼压制这由于无有信力等而无力的人,即由无力而压制之义。然后,当守护、寻求爱欲时,狮子等显明的危难与身恶行等非显明的\textbf{危难压迫}这贪求爱欲者。\textbf{随后},生等\textbf{苦追随}这被非显明的危难征服之人,\textbf{如水之于漏船}。\end{enumerate}

\subsection\*{\textbf{778} {\footnotesize 〔PTS 771〕}}

\textbf{所以,常常具念之人应避开爱欲,\\}
\textbf{舍弃了这些,便能度过暴流,如汲水出船,到达彼岸。}

Tasmā jantu sadā sato, kāmāni parivajjaye;\\
te pahāya tare oghaṃ, nāvaṃ sitvā va pāragū ti. %\hfill\textcolor{gray}{\footnotesize 6}

\begin{enumerate}\item \textbf{所以},由身至念等的修习而\textbf{常常具念之人},\textbf{应}以镇伏、正断等避开色等物欲中的一切品类中的烦恼欲而\textbf{避开爱欲}。如是,\textbf{舍弃了这些}爱欲,\textbf{便能}以令彼舍弃之道\textbf{度过}四种\textbf{暴流},即堪能度过。
\item 随后,好比人\textbf{汲出船}中的重水,能以轻快之船轻易地\textbf{到达彼岸}、去到对岸,如是,汲出了自体之船中的烦恼重水,能以轻快之自体到达彼岸、去到一切法之对岸的涅槃,且能以阿罗汉的成就而往,以无余依涅槃界而般涅槃,即以阿罗汉为顶点完成了开示。当开示终了,婆罗门与婆罗门尼便住于须陀洹果。\end{enumerate}

\begin{center}\vspace{1em}爱欲经第一\\Kāmasuttaṃ paṭhamaṃ.\end{center}