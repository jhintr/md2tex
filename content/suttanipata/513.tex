\section{善器学童问}

\begin{center}Bhadrāvudha Māṇava Pucchā\end{center}\vspace{1em}

\subsection\*{\textbf{1108} {\footnotesize 〔PTS 1101〕}}

\textbf{「我恳求舍弃住处、断除渴爱、不动、」尊者善器说,「舍弃欢喜、度过暴流、解脱、\\}
\textbf{「舍弃想的善慧者,从龙象处听闻后,人们将从此处离开。}

“Okañjahaṃ taṇhacchidaṃ anejaṃ, \textit{(icc āyasmā Bhadrāvudho)} nandiñjahaṃ oghatiṇṇaṃ vimuttaṃ;\\
kappañjahaṃ abhiyāce sumedhaṃ, sutvāna nāgassa apanamissanti ito. %\hfill\textcolor{gray}{\footnotesize 1}

\begin{enumerate}\item \textbf{舍弃住处},即舍弃执著。\textbf{断除渴爱},即断除六爱身。\textbf{不动},即不为世间法所震动。\textbf{舍弃欢喜},即舍弃对未来的色等的希求。(上述)即一渴爱,于此由赞叹故,从种种行相而说。\textbf{舍弃想},即舍弃两种想。\textbf{人们将从此处离开},即众人将从石支提离开。\end{enumerate}

\begin{itemize}\item 案,\textbf{两种想},即爱想、见想。又,译文将第三句的\textbf{我恳求}移动至第一句句首。\end{itemize}

\subsection\*{\textbf{1109} {\footnotesize 〔PTS 1102〕}}

\textbf{「种种人们已从诸多国土聚集,英雄!期待着您的言语,\\}
\textbf{「请您对他们善加解释!因为这法已如是为你所知。」}

Nānājanā janapadehi saṅgatā, tava Vīra vākyaṃ abhikaṅkhamānā;\\
tesaṃ tuvaṃ sādhu viyākarohi, tathā hi te vidito esa dhammo”. %\hfill\textcolor{gray}{\footnotesize 2}

\begin{enumerate}\item \textbf{诸多国土},即从鸯伽等诸多国土。\end{enumerate}

\subsection\*{\textbf{1110} {\footnotesize 〔PTS 1103〕}}

\textbf{「应调伏一切对取著的渴爱!善器!」世尊说,「上方、下方、四方及中间,\\}
\textbf{「凡是他们在世间所执取者,魔罗即以此追随他。}

“Ādānataṇhaṃ vinayetha sabbaṃ, \textit{(Bhadrāvudhā ti Bhagavā)} uddhaṃ adho tiriyañ cāpi majjhe;\\
yaṃ yañ hi lokasmim upādiyanti, ten’eva Māro anveti jantuṃ. %\hfill\textcolor{gray}{\footnotesize 3}

\begin{enumerate}\item 于是,世尊以随顺他的意乐而开示法,说了二颂。\textbf{对取著的渴爱},即取著色等的执取之爱,即爱取之义。\textbf{魔罗},即以取为缘所生的业之行作所生的结生蕴魔罗。\end{enumerate}

\begin{itemize}\item 案,菩提比丘注云,义注认可四种魔罗,这里的\textbf{结生蕴魔罗}即其一种,上述义注即是说,生以有(即业之行作)为缘,而有以取为缘的意思。\end{itemize}

\subsection\*{\textbf{1111} {\footnotesize 〔PTS 1104〕}}

\textbf{「觉察着『于执取的有情中,这人类爱著于死亡境域』,\\}
\textbf{「所以,知晓着,具念的比丘不应执取一切世间的任何。」}

Tasmā pajānaṃ na upādiyetha, bhikkhu sato kiñcanaṃ sabbaloke;\\
ādānasatte iti pekkhamāno, pajaṃ imaṃ maccudheyye visattan” ti. %\hfill\textcolor{gray}{\footnotesize 4}

\begin{enumerate}\item \textbf{所以,知晓着},即所以,了知着这过患,或以无常等了知着诸行。\textbf{觉察着『于执取的有情中,这人类爱著于死亡境域』},即在以可取之义而为执取的色等中取著的一切世间,觉察着这人类固著于死亡境域,或者说,觉察着在执著于执取的补特伽罗中,这人类固著于死亡境域而无能超越,不应执取一切世间的任何。
\item 如是,世尊同样以阿罗汉为顶点开示了此经。当开示终了,与先前一样,而有法的现观。\end{enumerate}

\begin{itemize}\item 案,此颂费解。\textbf{知晓着}本无宾语,二英译补充了 this,义注也作了如上的补充。原文后半颂中的 \textbf{satte},义注给出两种解释,第一种是将其作「取著」解,再将 ādānasatte 与原文第二句末的「一切世间」并列,第二种则将其作「补特伽罗」解。这里的汉译从第二种解释,并调整译文语序为 c-d-a-b。\end{itemize}

\begin{center}\vspace{1em}善器学童问第十二\\Bhadrāvudhamāṇavapucchā dvādasamā.\end{center}