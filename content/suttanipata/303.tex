\section{善说经}

\begin{center}Subhāsita Sutta\end{center}\vspace{1em}

\textbf{如是我闻\footnote{此经旧译见杂阿含经第 1218 经、别译杂阿含经第 253 经。}。一时世尊住舍卫国祇树给孤独园。于此,世尊告诸比丘:「诸比丘!」「大德!」这些比丘答世尊。}

Evaṃ me sutaṃ— eka samayaṃ Bhagavā Sāvatthiyaṃ viharati Jetavane Anāthapiṇḍikassa ārāme. Tatra kho Bhagavā bhikkhū āmantesi: “bhikkhavo” ti. “Bhadante” ti te bhikkhū Bhagavato paccassosuṃ.

\begin{enumerate}\item 此缘起由自身的意乐。因为世尊喜爱善说,他以显示自己善说的习行来制止有情恶说的习行,说了此经。这里,\textbf{如是我闻}等为结集者的话。
\item 这里,「于此,世尊……诸比丘答世尊」先前未见,而其余则已述。所以,为解释先前未见之词,而说:\textbf{于此},即显明时间、地点。因为在此时而住,即显明于此时,且在此园而住,即显明于此园。或者显明与当说者相应的时间、地点。因为世尊不在不相应的时间、地点说法。\begin{quoting}而首先,非时,跋西耶!(自说第 10 经)\end{quoting}等便是证明。\textbf{Kho},即补足语句,或强调等时间之义的不变词。\textbf{世尊},即显明受世间尊重。\textbf{诸比丘},即显明与听闻谈论相关的补特伽罗。\textbf{告},即称呼、说、令觉。
\item \textbf{诸比丘},即显明召唤的方式,且此是由与习于乞施等功德相应的成就而说。以此阐明彼等之行为为贵贱之人所从事,折服了穷通之相。且以「诸比丘」,这遍满悲的柔和之心与目光投射为前导之语,使他们的脸朝向自己,仍以此表露欲谈之语,令彼等生起欲闻,且仍以此旨在令觉之语,敦促彼等善闻、善作意。因为由善闻、善作意故,教法得以成就。
\item 设问:当其他人天在场时,为什么唯独召唤诸比丘?由最长、最胜、近旁、常常安顿之相故。因为这法的开示共通于一切会众,非于特定的补特伽罗,而比丘由最初生起故,为会众之最长,由从现出家相开始效仿大师所行故,以及由领受整个教法故,而为最胜,由坐于大师跟前故,而为近旁,由在大师跟前行止故,而为常常安顿。因此,世尊在开示一切会众共通之法时,便唯独召唤诸比丘。况且,也由彼等为此谈论的受众,能如所教授地行道之善性故,便唯独召唤彼等。
\item \textbf{大德},即尊重之称。\textbf{这些比丘},即世尊所告者,他们如是与世尊交谈,答复世尊。\end{enumerate}

\textbf{世尊说:「诸比丘!具足四支之语为善说,非恶说,无过,不为智者所呵责。哪四者?此处,诸比丘!比丘唯说善语而非恶语,唯说法而非非法,唯说可爱而非不可爱,唯说真实而非虚妄。诸比丘!具足这四支之语为善说,非恶说,无过,不为智者所呵责。」世尊说了这些。善逝说罢,大师进一步说:}

Bhagavā etad avoca: “Catūhi, bhikkhave, aṅgehi samannāgatā vācā subhāsitā hoti, na dubbhāsitā, anavajjā ca ananuvajjā ca viññūnaṃ. Katamehi catūhi? Idha, bhikkhave, bhikkhu subhāsitaṃ yeva bhāsati no dubbhāsitaṃ, dhammaṃ yeva bhāsati no adhammaṃ, piyaṃ yeva bhāsati no appiyaṃ, saccaṃ yeva bhāsati no alikaṃ. Imehi kho, bhikkhave, catūhi aṅgehi samannāgatā vācā subhāsitā hoti, no dubbhāsitā, anavajjā ca ananuvajjā ca viññūnan” ti. Idam avoca Bhagavā. Idaṃ vatvāna Sugato athāparaṃ etad avoca Satthā:

\begin{enumerate}\item \textbf{四支},即四种原因,或四部分。因为戒离妄语等为善说之语的四种原因,真实之语等为四部分。且「支」字作原因义,则「四」为从格,作部分义,则为具格。\textbf{具足},即转起、相应。\textbf{语},即会话之语,即凡是在\begin{quoting}语、话、言路……(法集论第 636 段)\end{quoting}与\begin{quoting}柔和、悦耳……(长部·梵网经第 9 段)\end{quoting}等中所提及者。而「语」在\begin{quoting}若以语所造之业……(法集论义注·身业门)\end{quoting}中为表(色),在\begin{quoting}若回避、戒离四语恶行……这被称为正语。(法集论第 299 段)\end{quoting}中为离(心所),在\begin{quoting}粗恶语,诸比丘!习行、修习、多作,则导向地狱。(增支部第 8:40 经)\end{quoting}中为思(心所),均非此处的意趣。为什么?由未被说故。
\item \textbf{善说},以此显明其带来义利。\textbf{非恶说},以此显明其不带来非义。\textbf{无过},即无有被称为罪过的贪等过失,以此显明其原因的清净与无有所说的过失。\textbf{不为呵责},即免于非难,以此显明其一切行相之成就。\textbf{智者},以此显明在毁誉之中,愚人非量。
\item \textbf{哪四者},即欲说之问。\textbf{此处},即在此教法内。\textbf{诸比丘},即称呼欲对其说者。\textbf{比丘},即说所述品类之语的人例。\textbf{唯说善语},即以基于人的开示,说明四语支中某一支之语。\textbf{非恶语},即遮止此语支的敌对之说,以此禁止「妄语等有时也可说」之见。或者,以「非恶语」显明舍断邪语,以「善语」显明舍断邪语的善人当说之语之相,同于「诸恶莫作,众善奉行\footnote{即\textbf{法句}·第 183 颂。}」,只是为了支的明显,先不说不应说者,而说应说者,于「唯说法」等处亦然。
\item 且此中,以「唯说善语而非恶语」免于两舌的过失,而说致力和合之语,以「唯说法而非非法」免于绮语的过失,而说不离于法的考量之语,以另外二者免于粗恶、虚妄,而说可爱、真实之语。而以「\textbf{具足这……}」等明确显示这些支,总结了谈话。
\item 且特别地,这里在说「诸比丘!具足这四支之语为善说」时,凡他人认为具足宗等部分\footnote{宗等部分:即因明中的宗、因、喻。}、具足名词等的词与具足性、数、格、时态、语态等成就之语的「善说」,皆依法遮止之。因为即便部分等已成就的具足两舌之语,由为自己及他人带来非义故,仍是恶说。而具足这四支者,即便是发于蔑戾车\footnote{蔑戾车 \textit{milakkhu}:即边地。}之语,或发于汲瓶女仆之歌,同样是善说,由带来世、出世间的利乐故。
\item 在僧伽罗岛的路边,当守护收成的僧伽罗女仆正以僧伽罗语唱关于生老死的歌时,走在路上的六十位作观的比丘听后得证阿罗汉,便是此中一例。同样,名为低舍、开始作观的比丘行于莲池畔时,女仆在莲池内剥落朵朵莲花,唱着此歌:\begin{quoting}早上开放的红莲花,被日光炙烤,\\如是,生而为人的有情,被老力催逼。\end{quoting}他听后得证阿罗汉。
\item 还有,在佛的间隔,某男子和七个儿子一起从林中出来,某位女子正以杵舂米,唱着:\begin{quoting}此为衰老所逼,覆以干枯皮囊,\\此为死亡所破,作死王的食味,\\此为蛆虫所聚,填以种种尸体,\\此为不净之器,等同芭蕉茎干。\end{quoting}听到此歌,便与儿子们俱证辟支佛。而其他人也以这样的方法得证圣地,便是例证。
\item 所以,此非异事:听完善巧于意乐、随眠的世尊以「一切行无常」等方法所说的偈颂,五百比丘便证了阿罗汉,以及其他无数人天听完关于蕴、处等之说。
\item 如是,具足这四支之语,即便是发于蔑戾车,或发于汲瓶女仆,当知同样是善说。且唯由善说故,无过,不为智者——即希求义利、依于义利而非依于字句的族姓子——所呵责。
\item \textbf{世尊说了这些},即世尊说了这些善说之相。\textbf{善逝说罢,大师进一步说},即说罢此相,然后,大师又说了其它。现在,结集者为显示将说之颂,说了这一切。这里,\textbf{进一步}是就结颂之语来说的。它有两种:为迟到的会众,或为把握未闻、巩固善闻,仍显明此义,以及以显明先前以某种原因被忽略之义来显明特别之义,如\begin{quoting}对于出生之人,有斧生其口中。(经集·第 663 颂)\end{quoting}等处,而这里为仍显明此义。\end{enumerate}

\subsection\*{\textbf{453} {\footnotesize 〔PTS 450〕}}

\textbf{「善人们说善语为最上,说法而非非法,这是第二,\\}
\textbf{「说可爱而非不可爱,这是第三,说真实而非虚妄,这是第四。」}

“Subhāsitaṃ uttamam āhu santo, dhammaṃ bhaṇe nādhammaṃ taṃ dutiyaṃ;\\
piyaṃ bhaṇe nāppiyaṃ taṃ tatiyaṃ, saccaṃ bhaṇe nālikaṃ taṃ catutthan” ti. %\hfill\textcolor{gray}{\footnotesize 1}

\begin{enumerate}\item 这里,\textbf{善人们},即佛等,而他们解释善语为「最上、最胜」。\textbf{第二、第三、第四}是按先前所示的次序来说的。\end{enumerate}

\textbf{于是,尊者婆耆舍从坐起,把衣偏覆一肩,向世尊合掌,对世尊说:「我明白了,世尊!我明白了,善逝!」「请说明白!婆耆舍!」世尊说。于是,尊者婆耆舍面对世尊,以合适的偈颂赞叹:}

Atha kho āyasmā Vaṅgīso uṭṭhāyāsanā ekaṃsaṃ cīvaraṃ katvā yena Bhagavā ten’añjaliṃ paṇāmetvā Bhagavantaṃ etad avoca: “paṭibhāti maṃ, Bhagavā, paṭibhāti maṃ, Sugatā” ti. “Paṭibhātu taṃ, Vaṅgīsā” ti Bhagavā avoca. Atha kho āyasmā Vaṅgīso Bhagavantaṃ sammukhā sāruppāhi gāthāhi abhitthavi:

\begin{enumerate}\item 而在颂的终了,婆耆舍长老便于世尊的善说净喜。为显示其所作的净喜之状与世尊所说之语,结集者便说了「于是,尊者……」等。这里,\textbf{我明白了}\footnote{这里的「我明白了」,直译或可作「它照耀了我」,「请说明白」则是相应的命令语气:让它照耀你。义注是从语源上解释,用「部分 \textit{bhāga}」来对应 paṭibhāti,但事实上 paṭibhāti < √bhā,即照耀之义。},即我的部分明白了。\textbf{请说明白},即请把你的部分说明白。\textbf{合适},即适合。\textbf{赞叹},即称赏。\end{enumerate}

\subsection\*{\textbf{454} {\footnotesize 〔PTS 451〕}}

\textbf{「应当只说这样的言语,不会因之折磨自己,\\}
\textbf{「也不会伤害到他人,这言语确是善说。}

“Tam eva vācaṃ bhāseyya, yāy’attānaṃ na tāpaye;\\
pare ca na vihiṃseyya, sā ve vācā subhāsitā. %\hfill\textcolor{gray}{\footnotesize 2}

\begin{enumerate}\item \textbf{不会折磨},即不因后悔而折磨。\textbf{不会伤害},即不以分裂彼此而逼恼。\textbf{这言语确是},即这言语定然是善说。至此,他以不两舌称赞世尊。\end{enumerate}

\subsection\*{\textbf{455} {\footnotesize 〔PTS 452〕}}

\textbf{「应当只说可爱的言语,这言语受人欢迎,\\}
\textbf{「若所说的不给他人带去坏恶,便是可爱。}

Piyavācam eva bhāseyya, yā vācā paṭinanditā;\\
yaṃ anādāya pāpāni, paresaṃ bhāsate piyaṃ. %\hfill\textcolor{gray}{\footnotesize 3}

\begin{enumerate}\item \textbf{受人欢迎},即以振奋之心迎面而来,受人欢喜、喜爱。\textbf{若所说的不给他人带去坏恶,便是可爱},即所说的不是给他人带去坏恶、不可爱、厌逆的粗恶之语,唯说义文甜美的可爱之语,即是说应说可爱之语。通过此颂,他以可爱之语称赞世尊。\end{enumerate}

\subsection\*{\textbf{456} {\footnotesize 〔PTS 453〕}}

\textbf{「真实确是甘露的言语,此乃永恒之法,\\}
\textbf{「善人们说,义利与法住立于真实。}

Saccaṃ ve amatā vācā, esa dhammo sanantano;\\
sacce atthe ca dhamme ca, āhu santo patiṭṭhitā. %\hfill\textcolor{gray}{\footnotesize 4}

\begin{enumerate}\item \textbf{甘露},即以甜美之相而与甘露等同,如说:\begin{quoting}味中真实更甜。(经集·第 184 颂)\end{quoting}或者,由涅槃不死之缘故而为不死\footnote{amata 兼有甘露、不死之义,这里随上下文译出。}。\textbf{此乃永恒之法},即此真实语者,乃是古法、传统之所行。因为这唯是古人的习行,他们不曾说虚妄。因此便说:\textbf{善人们说,义利与法住立于真实}。这里,当知唯由住立于真实,能住立于自己及他人的义利,而唯由住立于义利,能住立于法。
\item 或者,当知后两者都是真实的形容词,即住立于真实。什么样的(真实)?义利与法的(真实)。由不离他人的义利而为义利,即是说不造成破坏,而当不造成破坏存在时,由不离法而为法,即是说唯成就如法的义利。通过此颂,他以真实之语称赞世尊。\end{enumerate}

\subsection\*{\textbf{457} {\footnotesize 〔PTS 454〕}}

\textbf{「佛陀所说的安稳言语,为了证得涅槃,\\}
\textbf{「为尽苦的边际,它确是言语中的最上。」}

Yaṃ Buddho bhāsati vācaṃ, khemaṃ nibbānapattiyā;\\
dukkhass’antakiriyāya, sā ve vācānam uttamā” ti. %\hfill\textcolor{gray}{\footnotesize 5}

\begin{enumerate}\item \textbf{安稳},即无怖畏、离灾祸。设问:以什么原因?\textbf{为了证得涅槃,为尽苦的边际},因为令达烦恼的涅槃,且导向流转之苦的边际之义。或者,当如是了知此中之义:佛陀为证得涅槃、尽苦边际这两种涅槃界的义利,为阐明安稳之道而说的安稳言语,\textbf{它确是言语中的最上},这言语是一切言语中的最胜。通过此颂,他以考量之语称赞世尊,以阿罗汉为顶点完成了开示。以上即先前未见之词的解释,其余当知仍如所述。\end{enumerate}

\begin{center}\vspace{1em}善说经第三\\Subhāsitasuttaṃ tatiyaṃ.\end{center}