\section{恶意八颂经}

\begin{center}Duṭṭhaṭṭhaka Sutta\end{center}\vspace{1em}

\subsection\*{\textbf{787} {\footnotesize 〔PTS 780〕}}

\textbf{有些恶意的在说,然后真实意的也在说,\\}
\textbf{但牟尼不参与生起的言论,所以牟尼无任何荒秽。}

Vadanti ve duṭṭhamanā pi eke, atho pi ve saccamanā vadanti;\\
vādañ ca jātaṃ muni no upeti, tasmā munī natthi khilo kuhiñci. %\hfill\textcolor{gray}{\footnotesize 1}

\begin{enumerate}\item 缘起为何?先说初颂的缘起。如牟尼经(第 216 颂)所说,外道们不堪于世尊及比丘僧团所得的利养、恭敬,便派遣了女游行者孙陀利。据说,她是白衣的女游行者,有倾国之貌。她善加沐浴更衣,饰以花鬘、芳香、涂油,在舍卫国的居民听完世尊的法,从祇园离开时,便离开舍卫国,朝祇园走去。人们问「你去哪里」,她说「我去取悦沙门乔达摩和他的弟子们」,在祇园的门廊徘徊,当祇园的门廊关闭时进入城中,晨朝再次去往祇园,在香房附近如采花一般而行。前来给侍佛陀的人们问「你来做什么」,她便随便说了些什么。如是,半个月后,外道们便让人夺了她性命,弃于沟渠,在晨朝发起喧哗「我们没看见孙陀利」,并告知国王,受其允许进入祇园,便假装调查,把她从抛尸处挖出,置于床上,抬入城中,进行谴责,当知一切如在巴利中所说\footnote{即\textbf{自说}第 38 经。}。
\item 世尊在这天黎明时分以佛眼观察世间,得知「外道们今天会制造不名誉」,想「莫让大众信了他们,对我这样的人心生愤怒,趋于苦处」,便关了香房的门,留在香房,不入城乞食。而比丘们见门已关,便如往常一样进城了。人们见了比丘们,以种种方式骂詈。于是,阿难尊者便向世尊告知了本末,说:「尊者!外道们制造了巨大的不名誉,不能在此居住,瞻部洲广大,我们去往别处吧!」「阿难!若在那里也出现不名誉,我们将去哪里呢?」「世尊!另一城。」于是,世尊说「且待!阿难!这声音将只存在七天,过了七天,那些制造不名誉者会忆念起以前」,为了向阿难长老开示法,便说了此颂。
\item 这里,\textbf{说},即指责世尊与比丘僧团。\textbf{有些恶意的,然后真实意的},即有些是恶心的,有些是也这样想的,意即外道们是恶心的,而听闻他们的话并信了的是真实意的。\textbf{生起的言论},即此生起的骂詈的言论。\textbf{牟尼不参与},即佛牟尼以不作为及以不受干扰而不参与。\textbf{所以牟尼无任何荒秽},即以此原因,当知这牟尼无处因贪等荒秽而荒秽。\end{enumerate}

\subsection\*{\textbf{788} {\footnotesize 〔PTS 781〕}}

\textbf{被欲引领、住于喜好者,如何能超越自己的见?\\}
\textbf{从事自身的完整,他只会说他所能了知的。}

Sakañ hi diṭṭhiṃ katham accayeyya, chandānunīto ruciyā niviṭṭho;\\
sayaṃ samattāni pakubbamāno, yathā hi jāneyya tathā vadeyya. %\hfill\textcolor{gray}{\footnotesize 2}

\begin{enumerate}\item 说完此颂,世尊便问阿难长老:「阿难!被如是咒骂、蔑视的比丘们说了什么?」「世尊!什么也没有说。」(世尊)说「阿难!不应于一切处以『我是具戒者』而默然,因为在世间若不说话,人们无由了知混杂于愚人中的智者,阿难!比丘们应如是叱责那些人」,为了开示法,便说了偈颂:\begin{quoting}不实语者进入地狱……\footnote{即\textbf{瞿迦梨经}第 667 颂。}\end{quoting}
\item 长老受持后,便对比丘们说:「你们应以此颂叱责人们。」比丘们便照做。有智的人们便即默然。国王也四处派遣王臣,捉住并指责了外道们给予贿赂、派去谋害她的无赖们,了知经过后,便指责了外道们。人们看见外道们后,也朝他们扔土块、泼尘土:「他们给世尊制造了不名誉。」阿难长老见了,便告知世尊,世尊便对长老说了此颂。
\item 其义为:这外道之人的见,即「让人杀了孙陀利后,再声明沙门释迦子的不光彩,以此方法,我们将享用利养恭敬」,他如何能超越此见?然后,这不名誉又回到无能超越此见的外道之人。或者,常等论的论者\textbf{被}见之\textbf{欲引领},并\textbf{住于}见之\textbf{喜好,如何能超越自己的见}?况且\textbf{从事自身的完整},即唯由自己圆满这些成见,\textbf{他只会说他所能了知的}。\end{enumerate}

\subsection\*{\textbf{789} {\footnotesize 〔PTS 782〕}}

\textbf{若人未被问及,却对别人说自己的戒禁,\\}
\textbf{若唯说自己自身,善人们说这是非圣法。}

Yo attano sīlavatāni jantu, anānupuṭṭho va paresa pāva;\\
anariyadhammaṃ kusalā tam āhu, yo ātumānaṃ sayam eva pāva. %\hfill\textcolor{gray}{\footnotesize 3}

\begin{enumerate}\item 于是,过了七天,国王命人抛了尸骸,哺时去往寺庙,礼敬了世尊后说:「尊者!发生了这样的不名誉,难道你不该告知我吗?」如是说时,世尊说「大王!向他人宣告『我具戒、具德』,对圣者是不合适的」,便对此事由,说了余下的几颂。
\item 这里,\textbf{戒禁},即波罗提木叉等之戒及林野住者等的头陀支之禁。\textbf{若唯说自己自身,善人们说这是非圣法},即对如是唯说自己自身者,善人们便如是评论其说:这是非圣法。\end{enumerate}

\subsection\*{\textbf{790} {\footnotesize 〔PTS 783〕}}

\textbf{而寂静的比丘,内在寂静,于戒不夸耀「我是如此」,\\}
\textbf{若在世间已无任何增盛,善人们说这是圣法。}

Santo ca bhikkhu abhinibbutatto, “iti’han” ti sīlesu akatthamāno;\\
tam ariyadhammaṃ kusalā vadanti, yass’ussadā natthi kuhiñci loke. %\hfill\textcolor{gray}{\footnotesize 4}

\begin{enumerate}\item \textbf{寂静},即以止息贪等烦恼而寂静,\textbf{内在寂静}也如此。\textbf{于戒不夸耀「我是如此」},即于戒不以「我是具戒者」等方法夸耀,即是说不说指涉自身的戒相。\textbf{善人们说这是圣法},即佛等于蕴等善巧者说其不夸耀为「这是圣法」。\textbf{若在世间已无任何增盛},其连结为:若漏尽者在世间已无任何的贪等七种增盛\footnote{七种增盛:见\textbf{会堂经}第 521 颂注。},如是,善人们说其不夸耀为「这是圣法」。\end{enumerate}

\subsection\*{\textbf{791} {\footnotesize 〔PTS 784〕}}

\textbf{若其法遍计、造作,存有预设的不洁,\\}
\textbf{看到自身中的利益,依止缘于干扰的寂静。}

Pakappitā saṅkhatā yassa dhammā, purakkhatā santi avīvadātā;\\
yad attani passati ānisaṃsaṃ, taṃ nissito kuppa-paṭicca-santiṃ. %\hfill\textcolor{gray}{\footnotesize 5}

\begin{enumerate}\item 如是显示了漏尽者的行道,现在,为向国王显示持见之外道的行道,说了此颂。这里,\textbf{遍计},即臆测。\textbf{造作},即以缘行作。\textbf{若其},即任何持见者。\textbf{法},即见。\textbf{预设},即预先造作。\textbf{看到自身中的利益,依止缘于干扰的寂静},即对于预设这些成见、存有不洁者,这样的人因为在自身中以此见看到现法的恭敬等及来世的趣之殊胜等的利益,所以便依止这利益及这由干扰、缘生、世俗寂静而被称为缘于干扰的寂静的见,他由依止此故,以不实的功德或过失自赞毁他。\end{enumerate}

\subsection\*{\textbf{792} {\footnotesize 〔PTS 785〕}}

\textbf{见的住著实不易越过,于诸法抉择已即被摄取,\\}
\textbf{所以,人于这些住著,扬弃及执取法。}

Diṭṭhīnivesā na hi svātivattā, dhammesu niccheyya samuggahītaṃ;\\
tasmā naro tesu nivesanesu, nirassatī ādiyatī ca dhammaṃ. %\hfill\textcolor{gray}{\footnotesize 6}

\begin{enumerate}\item 且如是依止者,「见的住著……执取法」。这里,\textbf{见的住著},即被称为执著于「此是真实」的见的住著。\textbf{于诸法抉择已即被摄取},即是说于六十二见之法,抉择彼彼被摄取、被执著之法已,由转起故,见的住著实不易越过。
\item \textbf{所以,人于这些住著,扬弃及执取法},因为不易越过,所以,人便于这些见的住著,扬弃、执取种种羊戒、牛戒、狗戒、五热、荒崖、精勤蹲踞、倚于荆棘等类\footnote{据菩提比丘注 1800,这里的「羊戒、牛戒、狗戒」见\textbf{中部}·狗禁经,「五热」即正午坐于烈日下,并在周围燃起四堆火,「荒崖」即或站或坐于悬崖边,「精勤蹲踞」即保持蹲踞的姿势。}与大师、法语、僧众等类的各个法,即是说如林中猿猴舍弃、抓取各个枝条一般。当如是扬弃及执取时,由心未确立故,会以不实的功德或过失为自己或他人制造名誉、不名誉。\end{enumerate}

\subsection\*{\textbf{793} {\footnotesize 〔PTS 786〕}}

\textbf{除遣者于世间任何的有与无有不存遍计的见,\\}
\textbf{除遣者舍弃了伪善与慢,他因何能达?他无牵涉。}

Dhonassa hi natthi kuhiñci loke, pakappitā diṭṭhi bhavābhavesu;\\
māyañ ca mānañ ca pahāya dhono, sa kena gaccheyya anūpayo so. %\hfill\textcolor{gray}{\footnotesize 7}

\begin{enumerate}\item 但对具足除遣一切见等过失之慧的除遣者,「除遣者……他无牵涉」。这说的是什么?由具足除遣法,\textbf{除遣者}、已除遣一切恶的阿罗汉\textbf{于世间}任何种种\textbf{有与无有不存遍计的见}。他因无此见——而外道以之覆藏自己所作的恶业,以伪善或慢去往这非趣——\textbf{除遣者舍弃了伪善与慢,他因何能达}贪等过失?因何能在现法或后世入于地狱等趣的差别之数?\textbf{他无牵涉},因为他以无有爱、见两种牵涉而无牵涉。\end{enumerate}

\subsection\*{\textbf{794} {\footnotesize 〔PTS 787〕}}

\textbf{牵涉者于诸法参与言论,他因何、如何能说无牵涉者?\\}
\textbf{因为他没有拿起、放下,他即于此除遣了一切见。}

Upayo hi dhammesu upeti vādaṃ, anūpayaṃ kena kathaṃ vadeyya;\\
attā nirattā na hi tassa atthi, adhosi so diṭṭhi-m-idh’eva sabban ti. %\hfill\textcolor{gray}{\footnotesize 8}

\begin{enumerate}\item 而以此二种牵涉者,这「牵涉者……一切见」。这里,\textbf{牵涉者},即依止于爱、见者。\textbf{于诸法参与言论},即于或「贪染」或「恶意」等彼彼诸法参与言论。\textbf{他因何、如何能说无牵涉者},对以舍弃了爱、见的无牵涉的漏尽者,他因何贪染或嗔恨、如何能说是「贪染」或「恶意」?意即如是的无过者,他怎会如外道一般成为覆藏所作者呢?
\item \textbf{因为他没有拿起、放下},因为他没有我见或断见,或者,也没有称为拿起、放下的执持、放舍。若问「由何原因没有」?\textbf{他即于此除遣了一切见},因为他即于此自体中,以智风除遣、舍弃、扫除了一切见,即以阿罗汉为顶点完成了开示。国王听后,心满意足,礼敬世尊后便离去。\end{enumerate}

\begin{center}\vspace{1em}恶意八颂经第三\\Duṭṭhaṭṭhakasuttaṃ tatiyaṃ.\end{center}