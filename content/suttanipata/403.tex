\section{恶意八颂经}

\begin{center}Duṭṭhaṭṭhaka Sutta\end{center}\vspace{1em}

\begin{enumerate}\item 如牟尼经所说,外道们不堪于世尊及比丘僧团所得的利养、恭敬,便派遣了女游行者孙陀利。据说她是白衣的女游行者,有倾国之貌。她沐浴更衣,饰以花鬘、芳香、涂油,在舍卫国的居民听了世尊的法,从祇园离开时,便离开舍卫国,朝祇园走去。人们问「你去哪里」,她说「我去取悦沙门乔达摩和他的弟子们」,在祇园的入口徘徊,当祇园的入口关闭时进入城中,晨朝再次去往祇园,在香房附近如采花一般而行。前来侍奉佛陀的人们问「你来做什么」,她便随便说了些什么。如是,半个月后,外道们便让人夺了她性命,弃于沟渠,在晨朝制造骚乱「我们没看见孙陀利」,报告给国王,受其允许进入祇园,假装调查,把她从抛尸处挖出,置于床上,抬入城中,进行谴责,当知一切如在巴利(自说 Udāna)中所说。世尊这天早上以佛眼观察世间,得知「外道们今天会制造不名誉」,想「莫让大众信了他们,对我这样的人心生动摇,趋向苦处」,关了香房的门,唯坐在香房内,不入城乞食。然而,比丘们见门已关,便如往常一样进城了。人们见了比丘们,以种种方式责骂。于是,阿难尊者向世尊告知了本末,说「尊者!外道们制造了巨大的不名誉,不能在此居住,瞻部洲广大,我们去往别处吧」。「阿难!若在那里也出现不名誉,我们能去哪里呢」,「世尊!另一城」。于是,世尊说「别!阿难!这声音将只存在七天,七天以后,那些制造不名誉者会忆念起以前」,为向阿难长老开示法而说了此颂。\end{enumerate}

\subsection\*{\textbf{787} {\footnotesize 〔PTS 780〕}}

\textbf{有些恶意的在说,然后真实意的也在说,\\}
\textbf{但牟尼不参与生起的争论,所以牟尼无任何荒秽。}

Vadanti ve duṭṭhamanā pi eke, atho pi ve saccamanā vadanti;\\
vādañ ca jātaṃ muni no upeti, tasmā munī natthi khilo kuhiñci. %\hfill\textcolor{gray}{\footnotesize 1}

\begin{enumerate}\item 这里,\textbf{说},即责骂世尊与比丘僧团。\textbf{有些恶意的,然后真实意的},即有些是恶心,而有些是认为如此的,外道们是恶心,而听闻他们的话并相信了的是真实意的。\textbf{生起的争论},即此生起的责骂的争论。\textbf{牟尼不参与},即佛牟尼以无作及以无动摇而不参与。\textbf{所以牟尼无任何荒秽},即以此原因,当知这牟尼无处以贪等荒秽而荒秽。\end{enumerate}

\subsection\*{\textbf{788} {\footnotesize 〔PTS 781〕}}

\textbf{被欲引领、住于喜好者,如何能超越自己的见?\\}
\textbf{从事自身的完整,他只会说他所能了知的。}

Sakañ hi diṭṭhiṃ katham accayeyya, chandānunīto ruciyā niviṭṭho;\\
sayaṃ samattāni pakubbamāno, yathā hi jāneyya tathā vadeyya. %\hfill\textcolor{gray}{\footnotesize 2}

\begin{enumerate}\item 说完此颂,世尊问阿难长老「阿难!被如是叱责、蔑视的比丘们说了什么」,「世尊!什么也没有说」,说「阿难!不应于一切处以『我具戒』而保持默然,因为在世间,『若不说话,人们无由了知混杂于愚人中的智者』,阿难!比丘们应如是叱责那些人」,为开示法而说了偈颂「不实语者进入地狱」。长老受持后,对比丘们说「你们应以此颂叱责人们」。比丘们便如是做了。有智慧的人们便默然。国王也四处派遣王臣,捉住并叱责了外道们给予贿赂、派去谋害她的暴徒们,了知经过后,责骂了外道们。人们看见外道们后,朝他们扔土块、泼尘土,「他们给世尊制造了不名誉」。阿难长老看到后,报告给世尊,世尊便对长老说了此颂。
\item 这外道的见,即「让人杀了孙陀利后,再声明沙门释迦子的不光彩,以此方法,我们将享用利益、恭敬」,他如何能超越?于是,这不名誉又回到未能超越这见的外道。或者,常等论的论者也\textbf{被}见之\textbf{欲引领},及\textbf{住于}见之\textbf{喜好},如何能超越自己的见?且\textbf{从事自身的完整},即唯由自己圆满着这些见,\textbf{他只会说他所能了知的}。\end{enumerate}

\subsection\*{\textbf{789} {\footnotesize 〔PTS 782〕}}

\textbf{若人未被问及,却对别人说自己的戒禁,\\}
\textbf{若唯说自己自身,善人们说这是非圣法。}

Yo attano sīlavatāni jantu, anānupuṭṭho va paresa pāva;\\
anariyadhammaṃ kusalā tam āhu, yo ātumānaṃ sayam eva pāva. %\hfill\textcolor{gray}{\footnotesize 3}

\begin{enumerate}\item 于是,七天后,国王命人丢弃了尸骸,哺时去往寺庙,礼敬了世尊后说「尊者!发生了这样的不名誉,难道你不该告知我吗」。世尊说「大王!对于圣者来说,向他人宣告『我具戒、具德』是不合适的」,于此事由而说了余下的几颂。这里,\textbf{戒禁},即波罗提木叉等的戒及林野住者等的头陀支的禁。\end{enumerate}

\subsection\*{\textbf{790} {\footnotesize 〔PTS 783〕}}

\textbf{然而寂静的比丘,内在寂静,不夸耀持戒「我是如此」,\\}
\textbf{若在世上已无任何增盛,善人们说这是圣法。}

Santo ca bhikkhu abhinibbutatto, “iti’han” ti sīlesu akatthamāno;\\
tam ariyadhammaṃ kusalā vadanti, yass’ussadā natthi kuhiñci loke. %\hfill\textcolor{gray}{\footnotesize 4}

\begin{enumerate}\item \textbf{寂静},即以止息了贪等烦恼而寂静,\textbf{内在寂静}也如此。\textbf{若在世上已无任何增盛},若漏尽者在世上已无任何贪等七种增盛。\end{enumerate}

\begin{itemize}\item 案,\textbf{七种增盛},见会堂经第 521 颂注。\end{itemize}

\subsection\*{\textbf{791} {\footnotesize 〔PTS 784〕}}

\textbf{若其遍计、造作、预设不洁之法,\\}
\textbf{看到自身中的功德,依于缘自动摇的寂静。}

Pakappitā saṅkhatā yassa dhammā, purakkhatā santi avīvadātā;\\
yad-attani passati ānisaṃsaṃ, taṃ nissito kuppa-paṭicca-santiṃ. %\hfill\textcolor{gray}{\footnotesize 5}

\begin{enumerate}\item 如是已显明了漏尽者的行道,现在,为向国王显明持见之外道的行道而说此颂。\textbf{遍计},即臆测。\textbf{造作},即以缘行作。\textbf{预设},即预先造作。\textbf{法},即见。\textbf{看到自身中的功德,依于缘自动摇的寂静},即对于预设这些不洁之见者,这样的人因为以此见而看到自身中现法的恭敬等及后世的趣之殊胜等的功德,所以依于这功德及这由动摇、缘生、世俗寂静而被称为缘自动摇的寂静的见,他由依于此而以不实的功德或过失自赞毁他。\end{enumerate}

\subsection\*{\textbf{792} {\footnotesize 〔PTS 785〕}}

\textbf{见的住著实不易越过,于诸法抉择已即被摄取,\\}
\textbf{所以,人于这些住著,扬弃及执取法。}

Diṭṭhīnivesā na hi svātivattā, dhammesu niccheyya samuggahītaṃ;\\
tasmā naro tesu nivesanesu, nirassatī ādiyatī ca dhammaṃ. %\hfill\textcolor{gray}{\footnotesize 6}

\begin{enumerate}\item \textbf{见的住著},即执取「此是真实」。\textbf{所以,人于这些住著,扬弃及执取法},因为不易越过,所以,人于这些见的住著,扬弃、执取、舍弃、抓取种种如羊戒、牛戒、狗戒、五热、山崖、蹲踞、卧于荆棘等类与大师、法论、僧众等类的法,如林中猿猴之于各个枝条一般,当如是扬弃及执取时,由心未确定故,会以不实的功德或过失为自己或他人制造名誉、不名誉。\end{enumerate}

\begin{itemize}\item 菩提比丘:\textbf{羊戒、牛戒、狗戒}见于中部,\textbf{五热}即正午坐在烈日下,并在周围燃起四堆火,\textbf{山崖}即或站或坐于悬崖边,\textbf{蹲踞}即保持蹲踞的姿势。\end{itemize}

\subsection\*{\textbf{793} {\footnotesize 〔PTS 786〕}}

\textbf{除遣者于世上任何的有与无有没有遍计的见,\\}
\textbf{舍弃了伪善与慢,除遣者由何能达?他无所执著。}

Dhonassa hi natthi kuhiñci loke, pakappitā diṭṭhi bhavābhavesu;\\
māyañ ca mānañ ca pahāya dhono, sa kena gaccheyya anūpayo so. %\hfill\textcolor{gray}{\footnotesize 7}

\begin{enumerate}\item 由具足净法、除遣了一切恶的阿罗汉,\textbf{除遣者于世上任何}种种\textbf{有与无有没有遍计的见}。他由无此见——外道以之覆藏自身所作的恶业,因伪善或慢去往非道——\textbf{舍弃了伪善与慢,除遣者由何能达}贪等过失?由何能在现法或后世落入地狱等趣?\textbf{他无所执著},因为他无爱、见两种执著而无所执著。\end{enumerate}

\subsection\*{\textbf{794} {\footnotesize 〔PTS 787〕}}

\textbf{执著者于诸法参与争论,他以何、如何能说无执著者?\\}
\textbf{因为他没有接受、丢弃,他即于此除遣了一切见。}

Upayo hi dhammesu upeti vādaṃ, anūpayaṃ kena kathaṃ vadeyya;\\
attā nirattā na hi tassa atthi, adhosi so diṭṭhi-m idh’eva sabban ti. %\hfill\textcolor{gray}{\footnotesize 8}

\begin{enumerate}\item \textbf{执著者},即依于爱、见者。\textbf{于诸法参与争论},即「此是贪染」或「此是恶意」等,如是于种种诸法参与争论。\textbf{他以何、如何能说无执著者},对于由舍弃了爱、见的无执著的漏尽者,以何贪染或嗔恨、如何能说「此是贪染」或「此是恶意」?即如是的无过者,他怎会如外道一般覆藏所作的意思。\textbf{因为他没有接受、丢弃},因为他没有我见或断见,也没有称为接受、丢弃的执持、放弃。若问「由何原因而没有」?\textbf{他即于此除遣了一切见},因为他即于此自性中,以智风除遣了、舍弃了、扫除了一切见。(世尊)以阿罗汉为顶点而完成了开示。国王听后,心满意足,礼敬世尊后即离去。\end{enumerate}

\begin{center}\vspace{1em}恶意八颂经第三\\Duṭṭhaṭṭhakasuttaṃ tatiyaṃ.\end{center}