\section{慈达学童问}

\subsection\*{\textbf{1056} {\footnotesize 〔PTS 1049〕}}

\textbf{「我问你,世尊!请对我说说这个!」尊者慈达说,「我认为你通达诸明、修己,\\}
\textbf{「世间这些种种形相的苦,它们都从哪里产生?」}

\subsection\*{\textbf{1057} {\footnotesize 〔PTS 1050〕}}

\textbf{「你问我苦的根源,慈达!」世尊说,「我将对你说,如同了知者,\\}
\textbf{「世间这些种种形相的苦,由依持为因而产生。\footnote{此后半颂与下颂,全同\textbf{二重随观经}第 734 颂。}}

\begin{enumerate}\item 这里,\textbf{我将说,如同了知者},如同了知者所宣说,我即如是宣说。\textbf{苦由依持为因而产生},即由渴爱等依持为因,生等各种苦产生。\end{enumerate}

\subsection\*{\textbf{1058} {\footnotesize 〔PTS 1051〕}}

\textbf{「若愚钝的无知者造作依持,则再再地经历苦,\\}
\textbf{「所以,知晓者、随观苦的生与源者不应造作依持。」}

\begin{enumerate}\item 如是,在由依持为因而产生的苦中,「若愚钝的无知者……」。这里,\textbf{知晓者},即以无常等知晓诸行者。\textbf{随观苦的生与源者},即随观流转之苦的生因为「依持」者。\end{enumerate}

\subsection\*{\textbf{1059} {\footnotesize 〔PTS 1052〕}}

\textbf{「我们所问的,你已向我们宣说,我们另有所问,请你快说!\\}
\textbf{「智者们如何度过暴流,及生、老、忧悲?\\}
\textbf{「牟尼!请对我善加解释!因为这法已如是为你所知。」}

\begin{enumerate}\item \textbf{因为这法已如是为你所知},即这法既由施设所知,有情便能了知。\end{enumerate}

\subsection\*{\textbf{1060} {\footnotesize 〔PTS 1053〕}}

\textbf{「我将对你宣说法,慈达!」世尊说,「所见之法,而非传闻,\\}
\textbf{「了知此已,具念而行,便能度过世间的爱著。」}

\begin{enumerate}\item \textbf{我将对你宣说法},即我将对你开示涅槃之法及趣向涅槃的行道之法。\textbf{所见之法},即所见的苦等之法,或说即此自体\footnote{所见之法 \textit{diṭṭhe dhamme},通常作「现法」,就是「今生」的意思,对应义注给出的第二个解释。这里因为和「传闻」相对,所以根据义注的第一个解释作「所见之法」。Norman 英译作 in the world of phenomena,菩提比丘则作 seen in this very life,似包含了义注的两种意思,可见其注 2084。}。\textbf{而非传闻},即自身现量。\textbf{了知此已},即以「一切行无常」等方法思惟此法而了知已。\end{enumerate}

\subsection\*{\textbf{1061} {\footnotesize 〔PTS 1054〕}}

\textbf{「而我欢喜这无上之法,大仙!\\}
\textbf{「了知此已,具念而行,便能度过世间的爱著。」}

\begin{enumerate}\item \textbf{而我欢喜这},即我愿求你的话语,这阐明所说品类的法。\end{enumerate}

\subsection\*{\textbf{1062} {\footnotesize 〔PTS 1055〕}}

\textbf{「凡是你所知的,慈达!」世尊说,「上方、下方、四旁及中间,\\}
\textbf{「除去了其中的欢喜与住著,识便不住于有。}

\begin{enumerate}\item 在「上方、下方、四旁及中间」中,以「\textbf{上方}」说未来时,以「\textbf{下方}」说过去时,以「\textbf{四旁及中间}」说现在时。\textbf{除去了其中的欢喜与住著,识},即应除去这些上方等处的渴爱、见的住著及行作识,且除去后,\textbf{便不住于有},当如是时,便能不住于两种有\footnote{两种有:\textbf{小义释}说,即「业有及结生之再有」。}。
\item 如是,先以「除去了」一词作「应除去」,在此意义分置中(与欢喜、住著、识)连结,而在「除去了」的意义分置中与「便不住于有」连结,即是说除去了这些欢喜、住著、识,便能不住于两种有。\footnote{义注在这里追随\textbf{义释},把\textbf{识}作为\textbf{除去}的宾语,而非\textbf{不住于有}的主语。菩提比丘在其注 2089、2090 中对此有详细的辩驳,认为\textbf{识}作为主语有其义理上的合理性,并举相应部第 12:64、22:53、22:87 三经为证,今译从之。}\end{enumerate}

\subsection\*{\textbf{1063} {\footnotesize 〔PTS 1056〕}}

\textbf{「如是而住、具念、不放逸的比丘,舍弃了执为我者,\\}
\textbf{「知者便能于此舍弃生、老、忧悲之苦。」}

\begin{enumerate}\item 这除去了这些、不住于有的「如是而住……」。这里,\textbf{于此},即于此教法,或即于此自体。\end{enumerate}

\subsection\*{\textbf{1064} {\footnotesize 〔PTS 1057〕}}

\textbf{「我欢喜大仙的这番话语,乔达摩!无依持是善说,\\}
\textbf{「因为世尊确实已舍弃了苦,因为这法已如是为你所知。}

\begin{enumerate}\item 在「乔达摩!无依持是善说」中,\textbf{无依持}即涅槃,就此而称呼世尊,说「乔达摩!无依持是善说」。\end{enumerate}

\subsection\*{\textbf{1065} {\footnotesize 〔PTS 1058〕}}

\textbf{「而且,你不停教诫之人,牟尼!他们也能舍弃苦,\\}
\textbf{「遇到了你,我将礼敬,龙象!愿世尊也能不停教诫我!」}

\begin{enumerate}\item 不仅你舍弃了苦,「而且,你……」。这里,\textbf{不停},即认真,或经常。\textbf{遇到},即接近。\textbf{龙象},即为称呼世尊而说。\end{enumerate}

\subsection\*{\textbf{1066} {\footnotesize 〔PTS 1059〕}}

\textbf{「你所证知的通达诸明、无所牵绊、不取著爱欲与有的婆罗门,\\}
\textbf{「他确实已度过了这暴流,且已度彼岸、无荒秽、无疑惑。}

\begin{enumerate}\item 现在,虽然世尊如是被此婆罗门所知「世尊确实已舍弃了苦」,为以舍弃苦的补特伽罗教诫他,且不提起自己,说了此颂。其义为:\textbf{你所证知}、了知的这由排除了恶为\textbf{婆罗门},由通晓吠陀为\textbf{通达诸明},以无有牵绊为\textbf{无所牵绊},且由不取著爱欲与诸有为\textbf{不取著爱欲与有}的婆罗门,\textbf{他确实已度过了这暴流,且已度彼岸、无荒秽、无疑惑}。\end{enumerate}

\subsection\*{\textbf{1067} {\footnotesize 〔PTS 1060〕}}

\textbf{「而且于此,这知者、通达诸明之人舍遣了对有与无有的染著,\\}
\textbf{「他离爱、无患、无待,我说他已度脱了生老。」}

\begin{enumerate}\item 还有,「而且于此,这知者……」。这里,\textbf{于此},即于此教法,或于自体。其余一切处皆自明。
\item 如是,世尊仍以阿罗汉为顶点开示了此经。当开示终了,如前所述,而有法的现观。\end{enumerate}

