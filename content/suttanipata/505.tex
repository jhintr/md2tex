\section{慈达学童问}

\begin{center}Mettagū Māṇava Pucchā\end{center}\vspace{1em}

\subsection\*{\textbf{1056} {\footnotesize 〔PTS 1049〕}}

\textbf{「我问您,世尊!请对我说说这个!」尊者慈达说,「我认为您通达诸明、修己,\\}
\textbf{「世间这些种种形相的苦,它们都从哪里产生?」}

“Pucchāmi taṃ Bhagavā brūhi me taṃ, \textit{(icc āyasmā Mettagū)} maññāmi taṃ vedaguṃ bhāvitattaṃ;\\
kuto nu dukkhā samudāgatā ime, ye keci lokasmim anekarūpā”. %\hfill\textcolor{gray}{\footnotesize 1}

\subsection\*{\textbf{1057} {\footnotesize 〔PTS 1050〕}}

\textbf{「你问了我苦的根源,慈达!」世尊说,「我将对你宣说,如同了知者那样,\\}
\textbf{「世间这些种种形相的苦,由所依为因而产生。}

“Dukkhassa ve maṃ pabhavaṃ apucchasi, \textit{(Mettagū ti Bhagavā)} taṃ te pavakkhāmi yathā pajānaṃ;\\
upadhinidānā pabhavanti dukkhā, ye keci lokasmim anekarūpā. %\hfill\textcolor{gray}{\footnotesize 2}

\begin{itemize}\item 案,此颂的后半颂及第 1058 颂,全同二重随观经第 734 颂。\end{itemize}

\subsection\*{\textbf{1058} {\footnotesize 〔PTS 1051〕}}

\textbf{「若愚钝的无知者造作所依,则再再地经历苦,\\}
\textbf{「所以,知晓者、随观苦的生与源者不应造作所依。」}

Yo ve avidvā upadhiṃ karoti, punappunaṃ dukkham upeti mando;\\
tasmā pajānaṃ upadhiṃ na kayirā, dukkhassa jātippabhavānupassī”. %\hfill\textcolor{gray}{\footnotesize 3}

\subsection\*{\textbf{1059} {\footnotesize 〔PTS 1052〕}}

\textbf{「我们所问的,您都已向我们解说,我们另有所问,请您说说!\\}
\textbf{「智者们如何度过暴流,以及生、老、忧悲?\\}
\textbf{「牟尼!请对我善加解释!因为这法已如是为你所知。」}

“Yaṃ taṃ apucchimha akittayī no, aññaṃ taṃ pucchāma tad-iṅgha brūhi;\\
kathaṃ nu dhīrā vitaranti oghaṃ, jātiṃ jaraṃ sokapariddavañ ca;\\
taṃ me Muni sādhu viyākarohi, tathā hi te vidito esa dhammo”. %\hfill\textcolor{gray}{\footnotesize 4}

\subsection\*{\textbf{1060} {\footnotesize 〔PTS 1053〕}}

\textbf{「我将对你宣说法,慈达!」世尊说,「所见之法,而非基于传闻,\\}
\textbf{「了知此已,具念而行,便能越过世间的爱著。」}

“Kittayissāmi te dhammaṃ, \textit{(Mettagū ti Bhagavā)} diṭṭhe dhamme anītihaṃ;\\
yaṃ viditvā sato caraṃ, tare loke visattikaṃ”. %\hfill\textcolor{gray}{\footnotesize 5}

\begin{enumerate}\item \textbf{我将对你宣说法},我将对你开示涅槃之法及趣向涅槃的行道之法。\textbf{所见之法},即所见的苦等之法,或说即此自性。\textbf{非基于传闻},即亲眼所见。\textbf{了知此已},以「一切诸行无常」等方式思惟此法而了知已。\end{enumerate}

\begin{itemize}\item 案,\textbf{所见之法} \textit{diṭṭhe dhamme},通常作「现法」,就是「今生」的意思,对应义注给出的第二个解释。这里因为和「传闻」相对,所以根据义注的第一个解释作「所见之法」。Norman 英译作 in the world of phenomena,菩提比丘则作 seen in this very life,似包含了义注的两种意思,可见其注 2084。\end{itemize}

\subsection\*{\textbf{1061} {\footnotesize 〔PTS 1054〕}}

\textbf{「大仙!我欢喜这无上之法,\\}
\textbf{「了知此已,具念而行,便能越过世间的爱著。」}

“Tañ cāhaṃ abhinandāmi, Mahesi dhammam uttamaṃ;\\
yaṃ viditvā sato caraṃ, tare loke visattikaṃ”. %\hfill\textcolor{gray}{\footnotesize 6}

\subsection\*{\textbf{1062} {\footnotesize 〔PTS 1055〕}}

\textbf{「凡是你所知的,慈达!」世尊说,「上方、下方、四方及中间,\\}
\textbf{「除去了其中的欢喜与住著,识便不住于有。}

“Yaṃ kiñci sampajānāsi, \textit{(Mettagū ti Bhagavā)} uddhaṃ adho tiriyañ cāpi majjhe;\\
etesu nandiñ ca nivesanañ ca, panujja viññāṇaṃ bhave na tiṭṭhe. %\hfill\textcolor{gray}{\footnotesize 7}

\begin{enumerate}\item \textbf{上方}指未来时,\textbf{下方}指过去时,\textbf{四方及中间}指现在时。\textbf{除去了其中的欢喜与住著,识},即应除去上方等处的渴爱、见的住著及行作识,且除去已,\textbf{便不住于有},在这种情况下,即能不住于两种有。\end{enumerate}

\begin{itemize}\item 案,义注在这里追随义释,把\textbf{除去了}解释作\textbf{应除去},即把过去分词作命令语气解,又把\textbf{识}作为\textbf{除去}的宾语,而非\textbf{不住于有}的主语。菩提比丘在其注 2089、2090 中对后者有详细的辩驳,认为\textbf{识}作为主语有其义理上的合理性,并举相应部 SN12:64、SN22:53、SN22:87 三经为证,今译从之。\end{itemize}

\begin{itemize}\item 案,\textbf{两种有},见于小义释,即「业有及结生之再有 \textit{kammabhavo ca paṭisandhiko ca punabbhavo}」。\end{itemize}

\subsection\*{\textbf{1063} {\footnotesize 〔PTS 1056〕}}

\textbf{「如是而住、具念、不放逸的比丘,舍弃了执为我者,\\}
\textbf{「智者便能于此舍弃生、老、忧悲之苦。」}

Evaṃvihārī sato appamatto, bhikkhu caraṃ hitvā mamāyitāni;\\
jātiṃ jaraṃ sokapariddavañ ca, idh’eva vidvā pajaheyya dukkhaṃ”. %\hfill\textcolor{gray}{\footnotesize 8}

\begin{enumerate}\item \textbf{于此},即于此教法内,或即于此自性中。\end{enumerate}

\subsection\*{\textbf{1064} {\footnotesize 〔PTS 1057〕}}

\textbf{「我欢喜大仙的这番话语,乔达摩!无所依是善说,\\}
\textbf{「因为世尊确实已舍弃了苦,因为这法已如是为你所知。}

“Etābhinandāmi vaco mahesino, sukittitaṃ Gotam’anūpadhīkaṃ;\\
addhā hi Bhagavā pahāsi dukkhaṃ, tathā hi te vidito esa dhammo. %\hfill\textcolor{gray}{\footnotesize 9}

\begin{enumerate}\item \textbf{无所依},即涅槃。\end{enumerate}

\begin{itemize}\item 案,\textbf{无所依}、\textbf{善说}似可作为\textbf{话语}的修饰语,义注无文,这里从二英译。\end{itemize}

\subsection\*{\textbf{1065} {\footnotesize 〔PTS 1058〕}}

\textbf{「而且,牟尼!您常常教诫之人,他们也能舍弃苦,\\}
\textbf{「遇到了您,我将礼敬,龙象!愿世尊也能常常教诫我!」}

Te cāpi nūna-ppajaheyyu dukkhaṃ, ye tvaṃ Muni aṭṭhitaṃ ovadeyya;\\
taṃ taṃ namassāmi samecca Nāga, app-eva maṃ Bhagavā aṭṭhitaṃ ovadeyya”. %\hfill\textcolor{gray}{\footnotesize 10}

\subsection\*{\textbf{1066} {\footnotesize 〔PTS 1059〕}}

\textbf{「你所能知的通达诸明、无所牵绊、不取著爱欲与有的婆罗门,\\}
\textbf{「他确实已度过了这暴流,且已度彼岸、无荒秽、无疑惑。}

“Yaṃ brāhmaṇaṃ vedagum ābhijaññā, akiñcanaṃ kāmabhave asattaṃ;\\
addhā hi so ogham imaṃ atāri, tiṇṇo ca pāraṃ akhilo akaṅkho. %\hfill\textcolor{gray}{\footnotesize 11}

\subsection\*{\textbf{1067} {\footnotesize 〔PTS 1060〕}}

\textbf{「而且于此,这智者、通达诸明之人舍离了对有与无有的染著,\\}
\textbf{「他离爱、无患、无待,我说他已度脱了生与老。」}

Vidvā ca yo vedagū naro idha, bhavābhave saṅgam imaṃ visajja;\\
so vītataṇho anīgho nirāso, atāri so jātijaran ti brūmī” ti. %\hfill\textcolor{gray}{\footnotesize 12}

\begin{enumerate}\item 如是,世尊同样以阿罗汉为顶点开示了此经。当开示终了,与(先前)所说的一样,而有法的现观。\end{enumerate}

\begin{center}\vspace{1em}慈达学童问第四\\Mettagūmāṇavapucchā catutthī.\end{center}