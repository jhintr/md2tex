\chapter{结语}

至此\footnote{「结语」以下均是「第一义光」的内容。},这如

\begin{quoting}
礼拜了最上的应予礼拜的三宝,\\
这在小部中,由舍弃了微细行、

世间之依怙、寻求世间之出离者开示的\\
经集,我将造其释义。
\end{quoting}

所说\footnote{「礼拜了」等二颂节录自「开篇辞」。},于此,蛇品等五品所摄、蛇经等凡七十经的经集的释义业已完成。由此故说

\begin{quoting}
以我造此经集的释义、\\
欲求善法住立所得的善之

威力,愿此众人于圣者宣示的法中,\\
能速速得证繁荣、增长、广大。
\end{quoting}

\begin{center}〔依圣典之量而有四十四诵〕\end{center}

饰以最上清净的信、觉、精进,集以戒行、正直、柔和等功德,堪能深入把握自教及他教,具足慧的成就,于三藏圣典及其义注等大师的教法中有无碍的智与力,大文法家,嗓音天成、易发甜美宏大之声、音色怡人者,发言得体的论师,最胜的论师,大诗人,于饰以六神通与无碍解等功德的上人法中已善建立觉、上座传统之灯的诸长老中作为大寺住者传统的庄严者,具广大清净之觉,由诸师授名曰\textbf{觉音}的长老所造的这名为「\textbf{第一义光}」的经集的义注,

\begin{quoting}愿它存于世间!为寻求世间之度脱的\\
族姓子们显明慧清净的方法,

直至净心者、如如者、世间最胜者、\\
大仙的「佛陀」之名尚在世间转起。\footnote{「饰以最上清净的」等一大段及此二颂也见于\textbf{清净道论}的结语中,唯长行末句小异。}\end{quoting}