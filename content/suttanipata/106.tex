\section{衰败经}

\textbf{如是我闻。一时世尊住舍卫国祇树给孤独园。于是,有一容貌殊胜的天人在深夜中照亮了整座祇园,往世尊处走去,走到后,礼敬了世尊,站在一边。然后,这位站在一边的天人以偈颂对世尊说:}

\begin{enumerate}\item 缘起为何?据说,众天人听闻了吉祥经后,便想:「世尊在吉祥经中说了有情的繁荣与幸福,但说的全都是兴盛,而无衰败,噫!现在,我们还要问问有情据以减损、亡失的衰败!」于是,在说吉祥经之日的翌日,一万轮围中欲听闻衰败经的众天人聚集在这一轮围中,化作十、二十、三十、四十、五十、六十、七十乃至八十仅占据一毫尖端的微细之身,世尊则以辉光超越了一切天、魔、梵而闪耀,坐于设好的最上佛座,他们围绕而立。随后,某一天子受命于诸天之因陀帝释,便问了世尊衰败的问题。于是,世尊藉由问题而说了此经。
\item 这里,「如是我闻」等为尊者阿难所说,以「衰败之人」等方法隔一的诸颂为天子所说,以「兴盛易知」等方法隔一的诸颂及末颂为世尊所说,这一切汇集后,即被称为「衰败经」。这里,于「如是我闻」等当说的一切,我们将在吉祥经注中解说。\end{enumerate}

\subsection\*{\textbf{91}}

\textbf{「我们问问衰败之人!乔达摩!\\}
\textbf{「前来问您,何为衰败者之因?」}

\begin{enumerate}\item 而在「衰败之人」等中,\textbf{衰败},即减损、亡失。\textbf{人},即任何有情、造物。\textbf{我们问问!乔达摩},即在表明自己是与其余诸天一起后,这天子创造机会,以族姓称呼世尊。\textbf{前来问您},即因为我们将来问您,故从各个轮围而来之义,以此显示尊敬。
\item \textbf{何为衰败者之因},即请对如是前来的我们说说,对于衰败之人,何因、何门、何源、何由,好让我们了知衰败之人的意思。这里,以此「衰败之人」来问所说的衰败之人的衰败之由,因为当了知衰败之由时,便能以此通用之由了知任何衰败之人。\end{enumerate}

\subsection\*{\textbf{92}}

\textbf{「兴盛易知,衰败易知,\\}
\textbf{「爱法者兴盛,厌法者衰败。」}

\begin{enumerate}\item 于是,世尊为了向他善加廓清,先显示对立面,再以基于人\footnote{基于人 \textit{puggalādhiṭṭhāna}:与「基于法 \textit{dhammā°}」相对,见菩提比丘的导论第 72 页。}的开示显明衰败之因,说了此颂。其义为:此\textbf{兴盛}、增长、不减损之人\textbf{易知},能以轻易、不烦、不难而知。而以「他衰败、减损、亡失」的\textbf{衰败},你们为之问我衰败之人的原因,这也\textbf{易知}。
\item 如何?因为\textbf{爱法者兴盛},这爱欲、渴望、希求、听闻、践行十善业道之法者,由见闻其行道可知而易知。而另一\textbf{厌法者衰败},他嫌厌此法,不爱欲、不渴望、不希求、不听闻、不践行,由见闻其邪行可知而易知。如是于此,当知世尊为显示对立面,从语义上先显示爱法为兴盛者之因,再显示厌法为衰败者之因。\end{enumerate}

\subsection\*{\textbf{93}}

\textbf{「我们了知了这点,这是第一种衰败,\\}
\textbf{「世尊!请说说何为衰败者的第二因?」}

\begin{enumerate}\item 于是,这天人欢喜于世尊之所说而说此颂。其义为:如世尊所说,\textbf{我们了知}、把握、受持了\textbf{这点},\textbf{这是第一种衰败},这厌法之相是第一种衰败,即是说于我们前来了知的众衰败之因中,这首先是一个衰败之因。这里的拆解为:以之衰败者即衰败。且以何衰败?以衰败者之因、之由。此中仅以文字而有多样,但在语义上,「衰败」或「衰败者之因」并无二致。
\item 如是,以「我们了知了一个衰败者之因」而欢喜已,为欲了知更多而说「\textbf{世尊!请说说何为衰败者的第二因}」。且在此后的\textbf{第三、第四}等中,其义皆以此法可知。\end{enumerate}

\subsection\*{\textbf{94}}

\textbf{「他喜爱不善人,不去喜爱善人,\\}
\textbf{「赞许不善人的法,这是衰败者之因。」}

\begin{enumerate}\item 且在解答这边,因为彼彼有情具足彼彼衰败之因,非一(有情)便具足所有,亦非所有唯具足一,所以为向彼彼显示彼彼衰败之因,以「他喜爱不善人」等方法,当知仍以基于人的开示来解答种种衰败之因。
\item 于此,其简略地释义为:\textbf{不善人}谓六师,抑或其他具足不寂静之身语意业者,\textbf{他喜爱}这些不善人,如善星\footnote{善星 \textit{Sunakkhatta}:其出家前事,见\textbf{中部}·善星经,其不满于佛陀事,见\textbf{长部}·波梨经,其还俗皈依俱罗刹帝利后,世尊为舍利弗说大狮子吼经。}之于裸行者俱罗刹帝利等一般。\textbf{善人}谓佛、辟支佛及声闻众,抑或其他具足寂静之身语意业者,\textbf{不去喜爱}这些善人,即不令自己喜爱、可意、好乐、适意之义。当知于此是为了调伏而作言语的区分。
\item 或者,\textbf{不去善人},即不亲近善人之义,好比说「亲近国王」,则明声者认为此义是「去喜爱国王」。\textbf{喜爱},即喜爱者、满足者、欢喜者之义。\footnote{此段的解释是将「喜爱 \textit{piyaṃ}」认作现在分词的体格,在句中用作主语,译文割裂勉强,读者知之。}
\item \textbf{不善人的法}谓六十二见,或十不善业道。他\textbf{赞许}、渴望、希求、亲近这不善人的法。
\item 如是,以此颂说了喜爱不善人、不喜爱善人、赞许不善法等三种衰败者之因。因为具足此的人衰败、减损,于此世他世不得增长,所以说是「衰败者之因」。而对此的详论我们将在(吉祥经)「不近愚人,亲近智者」一颂的解释中说明。\end{enumerate}

\subsection\*{\textbf{95}}

\textbf{「我们了知了这点,这是第二种衰败,\\}
\textbf{「世尊!请说说何为衰败者的第三因?」}

\subsection\*{\textbf{96}}

\textbf{「若人惯于睡眠、惯于集会而不奋起,\\}
\textbf{「懒惰,现忿怒相,这是衰败者之因。」}

\begin{enumerate}\item \textbf{惯于睡眠}谓无论走着、坐着、站着、躺着都在睡眠。\textbf{惯于集会}谓喜于从事社交、闲谈。\textbf{不奋起},即无有精进之火,不惯于奋起,受到他人指责时,作为在家人或出家人才时而\footnote{时而 \textit{kadāci karahaci}:原本无,据 PTS 本增加。}开始在家或出家的工作。
\item \textbf{懒惰},即生性懒惰,极度受制于昏沉,唯于所立之处而立,唯于所坐之处而坐,不以自身的勇猛采取另一威仪。在过去,当林野起火时也不逃避之懒惰可作这里的例证,这是其中的极端部分,而较之程度稍次的懒惰当知便为「懒惰」。如旗帜之于车辆、烟之于火,以忿怒为其标识,即\textbf{现忿怒相},嗔行者、易于急躁、心似烂疮的那类人便是。
\item 以此颂说了惯于睡眠、惯于集会、不奋起、懒惰、现忿怒相等五种衰败者之因。因为具足此的在家人不得在家的增长、出家人不得出家的增长,势必唯有减损、唯有衰败,所以说是「衰败者之因」。\end{enumerate}

\subsection\*{\textbf{97}}

\textbf{「我们了知了这点,这是第三种衰败,\\}
\textbf{「世尊!请说说何为衰败者的第四因?」}

\subsection\*{\textbf{98}}

\textbf{「若对年老、青春已逝的父母,\\}
\textbf{「堪能却不赡养,这是衰败者之因。」}

\begin{enumerate}\item \textbf{父母}当知即是令生者。以身体的松弛为\textbf{年老}。\textbf{青春已逝},即以度过青春而至耄耋,无法自己完成工作。\textbf{堪能},即有能力、轻松地活命。\textbf{不赡养},即不养育。
\item 以此颂只说了不赡养、不养育、不支持父母一种衰败之因。因为凡具足此者,不得如\begin{quoting}以此敬事于父母者,智者们\\在此世赞赏他,而他死后在天界欢喜。(增支部第 4:63 经)\end{quoting}所说的赡养父母的功德,势必招致「他连父母都不赡养,怎会赡养其他人」之非难、回避及恶趣,唯有衰败,所以说是「衰败者之因」。\end{enumerate}

\subsection\*{\textbf{99}}

\textbf{「我们了知了这点,这是第四种衰败,\\}
\textbf{「世尊!请说说何为衰败者的第五因?」}

\subsection\*{\textbf{100}}

\textbf{「若对婆罗门、沙门或其他的乞食者\\}
\textbf{「以妄语欺骗,这是衰败者之因。」}

\begin{enumerate}\item 由排除了恶而为\textbf{婆罗门},由已止息而为\textbf{沙门},或者也以生于婆罗门家族为婆罗门,而得达出家为沙门,此外,\textbf{其他的乞食者},即任何乞讨者。\textbf{以妄语欺骗},即以「尊者!请说所需」邀请已,或经请求而许诺已,后来却不布施而使其希望落空。
\item 以此颂只说了以妄语欺骗婆罗门等一种衰败之因。因为具足此者,在此世招致非难,在来世得至恶趣,即便在善趣也愿求不满。因为这如\begin{quoting}恶戒、破戒者的恶名远播。(增支部第 5:213 经)\end{quoting}所说。同样,\begin{quoting}诸比丘!具足四法者,如被裹挟一般,被投入地狱。以何等四?即妄语……等。(增支部第 4:82 经)\end{quoting}同样,\begin{quoting}于此,舍利弗!有人去往沙门、婆罗门处,以「尊者!请说所需」邀请,却不以所邀请者布施,若他从此死后,还至此世,无论从事何种贸易,皆至断坏。于此,舍利弗!……却不以所愿求者布施,若他从此死后,还至此世,无论从事何种贸易,皆不如愿。(增支部第 4:79 经)\end{quoting}如是招致此非难者唯有衰败,所以说是「衰败者之因」。\end{enumerate}

\subsection\*{\textbf{101}}

\textbf{「我们了知了这点,这是第五种衰败,\\}
\textbf{「世尊!请说说何为衰败者的第六因?」}

\subsection\*{\textbf{102}}

\textbf{「那人饶有财富,有货币、有食物,\\}
\textbf{「独自享用美味,这是衰败者之因。」}

\begin{enumerate}\item \textbf{饶有财富},即饶有金、银、摩尼之宝。\textbf{有货币},即有钱币。\textbf{有食物},即具足各种咖喱与食物。\textbf{独自享用美味},即甚至都不把美味的食物给自己的孩子们,而在隐蔽处享用。
\item 以此颂对食物的贪求只说了悭吝于食物一种衰败之因。因为具足此者,招致非难、回避、恶趣等,唯有衰败,所以说是「衰败者之因」。一切皆当以所述的与经相符的方法相连,但为恐过繁,现在,我们不再显明连结之法而唯说语义。\end{enumerate}

\subsection\*{\textbf{103}}

\textbf{「我们了知了这点,这是第六种衰败,\\}
\textbf{「世尊!请说说何为衰败者的第七因?」}

\subsection\*{\textbf{104}}

\textbf{「若人以出身为傲,以财产为傲,以种姓为傲,\\}
\textbf{「鄙视自己的亲戚,这是衰败者之因。」}

\begin{enumerate}\item \textbf{以出身为傲}谓人以「我具足出身」而产生慢,以之硬傲,如鼓风的布袋般膨胀,不向任何人倾弯。于\textbf{财产}、\textbf{种姓}处同理。\textbf{鄙视自己的亲戚},即甚至以出身鄙视自己的亲戚,如众释氏之于毗琉璃,且以财产鄙视「他贫穷、困顿」,乃至不以礼相待,他的亲戚们只希望他衰败。以此颂说了于事有四、于相唯一的衰败之因。\end{enumerate}

\subsection\*{\textbf{105}}

\textbf{「我们了知了这点,这是第七种衰败,\\}
\textbf{「世尊!请说说何为衰败者的第八因?」}

\subsection\*{\textbf{106}}

\textbf{「若人沉湎女人,嗜酒,嗜赌,\\}
\textbf{「倾尽所有,这是衰败者之因。」}

\begin{enumerate}\item \textbf{沉湎女人},即迷恋女人,无论有些什么全都给与,善待一个又一个女人。同样,挥霍了自己的所有财产而事饮酒为\textbf{嗜酒},典当了衣裤还要赌博为\textbf{嗜赌}。以此三处,无论有些什么所得,全都败光,当知为\textbf{倾尽所有}。如此类者唯有衰败,因此以此颂说了三种衰败之因。\end{enumerate}

\subsection\*{\textbf{107}}

\textbf{「我们了知了这点,这是第八种衰败,\\}
\textbf{「世尊!请说说何为衰败者的第九因?」}

\subsection\*{\textbf{108}}

\textbf{「不满于自身的妻妾,混迹于娼妓,\\}
\textbf{「混迹于他人的妻妾,这是衰败者之因。」}

\begin{enumerate}\item \textbf{自身的妻妾},即自己的妻妾。若人\textbf{不满于}自己的妻妾,\textbf{混迹于娼妓}以及\textbf{他人的妻妾}\footnote{混迹 \textit{padussati, dussati}:PTS 本分别作 padissati, dissati,即「现身、露面」。},他便因给予娼妓财物、亲近他人的妻妾,受王刑惩处,唯有衰败,因此以此颂说了二种衰败之因。\end{enumerate}

\subsection\*{\textbf{109}}

\textbf{「我们了知了这点,这是第九种衰败,\\}
\textbf{「世尊!请说说何为衰败者的第十因?」}

\subsection\*{\textbf{110}}

\textbf{「青春已过的男人,带回乳房如小果者,\\}
\textbf{「出于对她的嫉妒而无法入睡,这是衰败者之因。」}

\begin{enumerate}\item \textbf{青春已过},即青春已去而至耄耋。\textbf{带回},即占有。\textbf{乳房如小果者}\footnote{小果 \textit{timbaru}:词典也作 tinduka,PED 给出的拉丁学名为 Diospyros embryopteris,当是柿属。},即乳房如小果一般的年轻女孩。\textbf{出于对她的嫉妒而无法入睡},即以「对于青年,和年纪大的人一起欢愉、共住是不适意的,愿她不要因此希求年轻人」而嫉妒,为看管她而无法入睡。他因为以贪染爱欲、嫉妒而燃烧,无法从事外在的工作,唯有衰败,因此以此颂只说了这出于嫉妒的无眠一种衰败之因。\end{enumerate}

\subsection\*{\textbf{111}}

\textbf{「我们了知了这点,这是第十种衰败,\\}
\textbf{「世尊!请说说何为衰败者的第十一因?」}

\subsection\*{\textbf{112}}

\textbf{「对上瘾、挥霍的女人,或对这样的男人,\\}
\textbf{「他委以权柄,这是衰败者之因。」}

\begin{enumerate}\item \textbf{上瘾},即贪求鱼、肉、麻醉品\footnote{麻醉品 \textit{majja}:或译作酒,原本无,据 PTS 本增加。}等的生性贪婪者。\textbf{挥霍},即为了彼等的利益挥金如土的惯于败家者。\textbf{他委以权柄},即交予印记、符契等,令其处理家中的事务或商贸等的俗事。他因为以此过失而至财尽,唯有衰败,因此以此颂只说了委此类人以权柄一种衰败之因。\end{enumerate}

\subsection\*{\textbf{113}}

\textbf{「我们了知了这点,这是第十一种衰败,\\}
\textbf{「世尊!请说说何为衰败者的第十二因?」}

\subsection\*{\textbf{114}}

\textbf{「薄财而多爱,生于刹帝利家族,\\}
\textbf{「且希求王位,这是衰败者之因。}

\begin{enumerate}\item \textbf{薄财}谓无有积蓄的财物或收入。\textbf{多爱},即具有对财物极大的渴爱,不满于所得。\textbf{希求王位},由此大渴爱,他以非法或逆次希求不属于自己继承或他人所有的王位,当他如是希求时,因为给予战士等寡薄之财,无法得到王位,唯有衰败,因此以此颂只说了希求王位一种衰败之因。\end{enumerate}

\subsection\*{\textbf{115}}

\textbf{「智者省察了世间的这些衰败,\\}
\textbf{「圣者具足知见,追随吉祥的世间。」}

\begin{enumerate}\item 从此以往,如果这天人以「世尊!请说说何为衰败者的第十三……第十万」而问,世尊也会对他谈论。然而,因为这天人想「为什么问这些,其中无一导致繁荣」,不乐于听闻这些衰败之因,连问了这些也觉后悔,便即默然,所以世尊了知了他的意乐,便结束开示,说了此颂。
\item 这里,\textbf{智者},即具足思量者。\textbf{省察},即以慧眼考察。\textbf{圣者},不是以道、不是以果,此处只是以「不以被称为衰败的非法来行止」为圣者。以知见与慧见到衰败后而能回避,由具足此为\textbf{具足知见}。\textbf{追随吉祥的世间},即是说这样的人追随、紧跟、靠近吉祥、安稳、最上、无祸的天界。
\item 在开示的终了,听闻了众多衰败之因,无数的天人如理精勤于已生起的随适的悚惧,得证须陀洹、斯陀含、阿那含果。如说:\begin{quoting}在大集会经,以及吉祥经、\\平等心、教诫罗睺罗、法轮、衰败等中\footnote{大集会经等六经:据菩提比丘注 627,分别为\textbf{长部}第 20 经、\textbf{吉祥经}、\textbf{增支部}第 2:37 经、\textbf{中部}第 147 经、\textbf{相应部}第 56:11 经以及本经。},\\无量无数的天人集会于此,\\且其中得法的现观者不计其数。\end{quoting}\end{enumerate}

