\section{彼岸道赞颂}

\begin{center}Pārāyanatthutigāthā\end{center}\vspace{1em}

\textbf{当住于摩竭陀石支提时,世尊说了这些,对十六个随侍的婆罗门请求的问题一一作了解答。如果对每个问题知晓其义、知晓其法,能修习法与随法,便能去到老死的彼岸,以这些法能趣向彼岸,所以这法门即名「彼岸道」。}

Idam avoca Bhagavā Magadhesu viharanto Pāsāṇake cetiye, paricārakasoḷasānaṃ brāhmaṇānaṃ ajjhiṭṭho puṭṭho puṭṭho pañhaṃ byākāsi. Ekamekassa ce pi pañhassa attham aññāya dhammam aññāya dhammānudhammaṃ paṭipajjeyya, gaccheyy’eva jarāmaraṇassa pāraṃ, pāraṅgamanīyā ime dhammā ti, tasmā imassa dhammapariyāyassa Pārāyanan t’eva adhivacanaṃ.

\begin{enumerate}\item 此后是结集者为赞叹开示而说。于此,十六个(婆罗门的)会众从前后左右各六由旬(围绕)而坐,方圆有十二由旬。\textbf{知晓其义},即知晓圣典之义。\textbf{知晓其法},即知晓圣典。\end{enumerate}

\subsection\*{\textbf{1131} {\footnotesize 〔PTS 1124〕}}

\textbf{阿耆多、低舍弥勒、富楼那,以及慈达,\\}
\textbf{净洗、优波湿婆、难陀,以及黄金,}

Ajito Tissametteyyo, Puṇṇako atha Mettagū;\\
dhotako Upasīvo ca, Nando ca atha Hemako. %\hfill\textcolor{gray}{\footnotesize 1}

\subsection\*{\textbf{1132} {\footnotesize 〔PTS 1125〕}}

\textbf{可教、劫波二人,与智者胶耳,\\}
\textbf{善器与生起,及婆罗门布萨罗,\\}
\textbf{有智的空王,与大仙褐者,}

Todeyya-Kappā dubhayo, Jatukaṇṇī ca paṇḍito;\\
bhadrāvudho Udayo ca, Posālo cāpi brāhmaṇo;\\
mogharājā ca medhāvī, Piṅgiyo ca mahā isi. %\hfill\textcolor{gray}{\footnotesize 2}

\subsection\*{\textbf{1133} {\footnotesize 〔PTS 1126〕}}

\textbf{他们去到佛陀、具足行的仙人处,\\}
\textbf{去到最胜的佛陀处,问着微妙的问题。}

Ete Buddhaṃ upāgacchuṃ, sampannacaraṇaṃ isiṃ;\\
pucchantā nipuṇe pañhe, Buddhaseṭṭhaṃ upāgamuṃ. %\hfill\textcolor{gray}{\footnotesize 3}

\begin{enumerate}\item \textbf{具足行},即具足作为涅槃的足处的波罗提木叉戒等。\end{enumerate}

\subsection\*{\textbf{1134} {\footnotesize 〔PTS 1127〕}}

\textbf{佛陀如实解答了他们提出的问题,\\}
\textbf{以对问题的解释,牟尼令众婆罗门满足。}

Tesaṃ Buddho pabyākāsi, pañhe puṭṭho yathātathaṃ;\\
pañhānaṃ veyyākaraṇena, tosesi brāhmaṇe muni. %\hfill\textcolor{gray}{\footnotesize 4}

\subsection\*{\textbf{1135} {\footnotesize 〔PTS 1128〕}}

\textbf{他们满足于具眼者、佛陀、日种,\\}
\textbf{便在胜慧者的跟前行梵行。}

Te tositā cakkhumatā, Buddhen’Ādiccabandhunā;\\
brahmacariyam acariṃsu, varapaññassa santike. %\hfill\textcolor{gray}{\footnotesize 5}

\begin{enumerate}\item \textbf{行梵行},即行道梵行。\end{enumerate}

\subsection\*{\textbf{1136} {\footnotesize 〔PTS 1129〕}}

\textbf{对每个问题,能如佛陀所开示的\\}
\textbf{那样去修习,他便能从此岸去到彼岸。}

Ekamekassa pañhassa, yathā Buddhena desitaṃ;\\
tathā yo paṭipajjeyya, gacche pāraṃ apārato. %\hfill\textcolor{gray}{\footnotesize 6}

\subsection\*{\textbf{1137} {\footnotesize 〔PTS 1130〕}}

\textbf{修行着最上之道,他便能从此岸去到彼岸,\\}
\textbf{这道趣向彼岸,所以名为「彼岸道」。}

Apārā pāraṃ gaccheyya, bhāvento maggam uttamaṃ;\\
maggo so pāraṃ gamanāya, tasmā Pārāyanaṃ iti. %\hfill\textcolor{gray}{\footnotesize 7}

\begin{enumerate}\item \textbf{彼岸道},即作为彼岸之涅槃的道。\end{enumerate}