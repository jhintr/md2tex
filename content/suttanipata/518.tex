\section{彼岸道赞颂}

\textbf{当住于摩竭陀的石支提时,世尊说了这些,对随侍的十六个婆罗门请求的问题一一作了解答。如果对每个问题知晓其义、知晓其法,能修行法与随法,便能去到老死的彼岸,以这些法能趣向彼岸,所以这法门即名「彼岸道」。}

\begin{enumerate}\item 此后是结集者为称赞开示而说。这里,\textbf{说了这些},即说了这彼岸道。\textbf{随侍的十六个},即与随侍波婆利的褐者一起的十六个,或随侍佛世尊的十六个,且他们都是\textbf{婆罗门}。而在此处,十六会众从前后左右各六由旬而坐,周匝十二由旬。\textbf{知晓其义},即知晓圣典之义。\textbf{知晓其法},即知晓圣典。\textbf{彼岸道},即如是提起此法门的名称后,为宣扬这些婆罗门的名字,说了「阿耆多、低舍弥勒……问着微妙的问题」。\end{enumerate}

\subsection\*{\textbf{1131} {\footnotesize 〔PTS 1124〕}}

\textbf{阿耆多、低舍弥勒、富楼那,还有慈达,\\}
\textbf{净洗、优波湿婆与难陀,还有黄金,}

\subsection\*{\textbf{1132} {\footnotesize 〔PTS 1125〕}}

\textbf{可教、劫波二人,与智者胶耳,\\}
\textbf{善器与生起,和布萨罗婆罗门,\\}
\textbf{有智的空王,与大仙褐者,}

\subsection\*{\textbf{1133} {\footnotesize 〔PTS 1126〕}}

\textbf{他们去到佛陀、具足行的仙人处,\\}
\textbf{去到最胜的佛陀处,问着微妙的问题。}

\begin{enumerate}\item 这里,\textbf{具足行},即具足作为涅槃的足处的波罗提木叉戒等。\textbf{仙人},即大仙。其余自明。\end{enumerate}

\subsection\*{\textbf{1134} {\footnotesize 〔PTS 1127〕}}

\textbf{佛陀如实地解答了他们提出的问题,\\}
\textbf{以对问题的解释,牟尼令众婆罗门满足。}

\subsection\*{\textbf{1135} {\footnotesize 〔PTS 1128〕}}

\textbf{他们满足于具眼者、佛陀、日种,\\}
\textbf{便在胜慧者的跟前行梵行。}

\begin{enumerate}\item 此后,\textbf{行梵行},即行道梵行。\end{enumerate}

\subsection\*{\textbf{1136} {\footnotesize 〔PTS 1129〕}}

\textbf{对每个问题,能如佛陀所开示的\\}
\textbf{那样去行道,他便能从此岸去到彼岸。}

\subsection\*{\textbf{1137} {\footnotesize 〔PTS 1130〕}}

\textbf{修习着最上之道,他便能从此岸去到彼岸,\\}
\textbf{这道趣向彼岸,所以名为「彼岸道」。}

\begin{enumerate}\item \textbf{彼岸道},即是说作为彼岸之涅槃的道。\end{enumerate}