\section{尼拘律劫波经}

\begin{center}Nigrodhakappa Sutta\end{center}\vspace{1em}

\textbf{如是我闻\footnote{此经旧译见杂阿含经第 1221 经、别译杂阿含经第 255 经。「婆耆舍」随旧译,「尼拘律劫波」,旧译作「尼拘律想」。}。一时世尊在旷野中,住旷野顶支提。尔时,尊者婆耆舍的亲教师,名为尼拘律劫波的长老在旷野顶支提般涅槃不久。于是,尊者婆耆舍幽居宴坐,便起如是寻思:「我的亲教师是已般涅槃,还是未般涅槃?」}

Evaṃ me sutaṃ— ekaṃ samayaṃ Bhagavā Āḷaviyaṃ viharati Aggāḷave cetiye. Tena kho pana samayena āyasmato Vaṅgīsassa upajjhāyo Nigrodhakappo nāma thero Aggāḷave cetiye aciraparinibbuto hoti. Atha kho āyasmato Vaṅgīsassa rahogatassa paṭisallīnassa evaṃ cetaso parivitakko udapādi: “parinibbuto nu kho me upajjhāyo udāhu no parinibbuto” ti?

\begin{enumerate}\item \textbf{尼拘律劫波经},也称\textbf{婆耆舍经} \textit{Vaṅgīsasutta}。缘起为何?这在其因缘中已述。这里,\textbf{如是我闻}等之义已述,因此,我们将撇开此及其它类此者,只解释未说之理。
\item \textbf{旷野顶支提},即旷野中的最上支提。因为在世尊未出世时,有旷野顶、乔达摩等众多支提,作为夜叉、龙等的居处,世尊出世后,人们便赶走它们,用作寺庙,但仍以此名称呼。由此,即是说住在被称为旷野顶支提的寺庙。\textbf{尊者婆耆舍}之中,尊者即敬称,婆耆舍为长老之名。他从出生以来(之事)当知如是。
\item 据说,他是男游行者之子,从女游行者的胎内出生,知晓某种明咒,以此威力,敲击髑髅后,能知晓有情的趣向。于是,人们会从自己亡故亲戚的墓地中取出髑髅,问他他们的趣向。他说:「已转生到某某地狱、某某人间。」人们为此惊叹,施与他许多财产。如是,他便在整个阎浮提知名。
\item 他已圆满十万劫的波罗蜜,具足志向,有五千男子随从,于村镇、国土、都城游行时,到了舍卫国。尔时,世尊住舍卫国,舍卫国的居民在饭前作了布施,在饭后穿好下衣和上衣,持了花香等去祇园听法。他见后,便问他们:「大众到哪里去?」然后,他们便告诉他:「佛陀出现于世,他为众人的利益开示法,我们去那里。」他与随从也与他们一起前往,问候完世尊,便坐在一边。
\item 然后,世尊对他说:「婆耆舍!你是不是知晓那种明咒——敲击髑髅后,能知晓有情的趣向?」「如是,乔达摩君!我知晓。」世尊便教人取来转生于地狱者的髑髅给他看,他用指甲敲击后,便说:「乔达摩君!这是转生于地狱者的髑髅。」如是,给他看了转生于所有趣的髑髅,他都如实了知并告知。然后,世尊给他看漏尽者的髑髅,他反复敲击却不知。随后,世尊说:「婆耆舍!这不是你的境域,这唯是我的境域,漏尽者的髑髅。」说了此颂:\begin{quoting}鹿的趣向是森林,鸟的趣向是天空,\\法的趣向是无有,阿罗汉的趣向是涅槃。\end{quoting}
\item 婆耆舍听完偈颂,便说:「乔达摩君!请传授我这样的明咒。」世尊说:「这明咒不是未出家者可得成就的。」他便说:「乔达摩君!度我出家,或对我做任何你愿意的事,再传授我这明咒。」那时,尼拘律劫波长老在世尊附近,世尊便命他:「那么,尼拘律劫波!你度他出家。」尼拘律劫波便度他出了家,并告知了「皮五法」的业处。婆耆舍渐次成了得证无碍解的阿罗汉。被世尊指认为上首:\begin{quoting}诸比丘!我的声闻比丘中,具辩才者的上首,即婆耆舍。(增支部第 1:212 经)\end{quoting}
\item 如是兴起的尊者婆耆舍的\textbf{亲教师}——以教人反省过失、非过失等而得名\footnote{亲教师 \textit{upajjhāya}:旧译也作邬波驮耶、和上。义注的解释是语源上的,据菩提比丘注 1209,该词来自 upa-adhi-√i,与「背诵、学习」之义有关。}——即名为\textbf{尼拘律劫波}的长老。劫波为此长老之名,由在尼拘律树下\footnote{尼拘律树:即榕树。}证得阿罗汉,而被世尊称为「尼拘律劫波」,以后比丘们便也如是称呼他。于教法内已得坚固之状为\textbf{长老}。\textbf{在旷野顶支提般涅槃不久},即在此支提般涅槃不久。\textbf{幽居宴坐},即由远离众人,以身幽居,以心宴坐,避开彼彼境域已而隐遁。\textbf{便起如是寻思},即以此行相生起了寻。为什么而起?由未在面前故、见到习行故。因为他在其般涅槃时未在面前,且先前见到其手的不安等过去的习行,而这样的(习行)既存在于未漏尽者,也因过去的习惯存在于漏尽者。
\item 因为就像宾头卢·婆罗豆婆遮饭后为昼住而去到优填(王)的庭园,是因其过去的习惯——先前为王时曾于彼自娱,㤭梵波提长老去到三十三天居处空无的天宫,是因其过去的习惯——先前为天子时曾于彼自娱,毕陵伽婆蹉比丘常以贱民之语交谈,是因其过去的习惯——连续五百世为婆罗门时便如是说。所以,由未在面前故、见到习行故,他便起如是寻思:「\textbf{我的亲教师是已般涅槃,还是未般涅槃?}」此后之义自明。\end{enumerate}

\textbf{于是,尊者婆耆舍晡时从宴坐起,往世尊处走去,走到后,礼敬了世尊,坐在一边。坐在一边的尊者婆耆舍对世尊说:「于此,尊者!我幽居宴坐,便起如是寻思『我的亲教师是已般涅槃,还是未般涅槃』。」然后,尊者婆耆舍从坐起,把衣偏覆一肩,向世尊合掌,以偈颂对世尊说:}

Atha kho āyasmā Vaṅgīso sāyanhasamayaṃ paṭisallānā vuṭṭhito yena Bhagavā ten’upasaṅkami, upasaṅkamitvā Bhagavantaṃ abhivādetvā ekamantaṃ nisīdi. Ekamantaṃ nisinno kho āyasmā Vaṅgīso Bhagavantaṃ etad avoca: “idha mayhaṃ, bhante, rahogatassa paṭisallīnassa evaṃ cetaso parivitakko udapādi: ‘parinibbuto nu kho me upajjhāyo, udāhu no parinibbuto’ ti”. Atha kho āyasmā Vaṅgīso uṭṭhāyāsanā ekaṃsaṃ cīvaraṃ katvā yena Bhagavā ten’añjaliṃ paṇāmetvā Bhagavantaṃ gāthāya ajjhabhāsi:

\begin{enumerate}\item 而此中的\textbf{把衣偏覆一肩},是以重新整理而如是说。且「一肩」,即披覆左肩而住的同义语。由此当知其义为:把衣偏覆,让其披覆左肩而住。余皆自明。\end{enumerate}

\subsection\*{\textbf{346} {\footnotesize 〔PTS 343〕}}

\textbf{「我问大师、最上慧,于此现法已断疑惑者:\\}
\textbf{「比丘在旷野顶死去,有名闻,有声望,内在寂静。}

“Pucchāmi Satthāram anomapaññaṃ, diṭṭhe va dhamme yo vicikicchānaṃ chettā;\\
Aggāḷave kālam akāsi bhikkhu, ñāto yasassī abhinibbutatto. %\hfill\textcolor{gray}{\footnotesize 1}

\begin{enumerate}\item \textbf{最上慧},有限、低劣被称为下,非下慧为最上慧,即大慧之义。\textbf{于此现法},即此现量,或即于此自体的之义。\textbf{疑惑},即如此的寻思。\textbf{有名闻},即知名。\textbf{有声望},即具足利养及随从。\textbf{内在寂静},即心有守护,或心不炽燃。\end{enumerate}

\subsection\*{\textbf{347} {\footnotesize 〔PTS 344〕}}

\textbf{「尼拘律劫波是他的名字,由你取给婆罗门,世尊!\\}
\textbf{「他礼敬你,希求解脱,勇猛精进,示现坚固法者!}

Nigrodhakappo iti tassa nāmaṃ, tayā kataṃ Bhagavā brāhmaṇassa;\\
so taṃ namassaṃ acari mutyapekkho, āraddhaviriyo daḷhadhammadassī. %\hfill\textcolor{gray}{\footnotesize 2}

\begin{enumerate}\item \textbf{由你取},由坐在尼拘律树下,由你所说而取为「尼拘律劫波」,这样说好让自己辨别。但世尊不单由坐便这样称呼他,而是由于此证得阿罗汉故。\textbf{婆罗门},即以出身来说。据说,他从富裕婆罗门的家族出家。\textbf{礼敬},即礼敬而住。\textbf{希求解脱},即希求被称为涅槃的解脱,即是说愿求涅槃。\textbf{示现坚固法者},即称呼世尊。因为坚固法以不坏义为涅槃,且世尊开示之,所以便说他为「示现坚固法者」。\end{enumerate}

\subsection\*{\textbf{348} {\footnotesize 〔PTS 345〕}}

\textbf{「释迦!我们全都希望知晓此声闻,一切眼者!\\}
\textbf{「我们的耳朵已为倾听竖立,你是我们的大师,你是无上士。}

Taṃ sāvakaṃ Sakya mayam pi sabbe, aññātum icchāma samantacakkhu;\\
samavaṭṭhitā no savanāya sotā, tuvaṃ no satthā tvam anuttaro’si. %\hfill\textcolor{gray}{\footnotesize 3}

\begin{enumerate}\item \textbf{释迦},也是以族名称呼世尊。\textbf{我们全都},即包括了无余的会众,为显明自己而说。\textbf{一切眼者},仍是以一切知智称呼世尊。\textbf{竖立},即完全竖起,经注意而立起。\textbf{为倾听},即为听闻此问题的记说。\textbf{耳朵},即耳根。\textbf{你是我们的大师,你是无上士},此唯是赞美之语。\end{enumerate}

\subsection\*{\textbf{349} {\footnotesize 〔PTS 346〕}}

\textbf{「请断除我们的疑惑!对我说他!了知般涅槃已,宏慧者!\\}
\textbf{「在我们中说!一切眼者!如千眼帝释在诸天中。}

Chind’eva no vicikicchaṃ brūhi m’etaṃ, parinibbutaṃ vedaya bhūripañña;\\
majjhe va no bhāsa samantacakkhu, Sakko va devāna sahassanetto. %\hfill\textcolor{gray}{\footnotesize 4}

\begin{enumerate}\item \textbf{请断除我们的疑惑},即非因真正的疑惑,他实无疑,只是以类似于疑惑的寻思而如是说。\textbf{对我说他},即我以「释迦!我们全都希望知晓此声闻」向你请求者,且说此婆罗门时,\textbf{了知般涅槃已,宏慧者!在我们中说},知晓般涅槃已,大慧的世尊!在我们全体中说,好让我们全体都知晓。\textbf{如千眼帝释在诸天中},此唯是赞美之语。其旨趣为:好比千眼帝释在诸天中间说为彼等恭敬领受的话语,如是,在我们中间,请说为我们领受的话语。\end{enumerate}

\subsection\*{\textbf{350} {\footnotesize 〔PTS 347〕}}

\textbf{「于此世中,任何系缚、痴路、无知的品类、疑惑处,\\}
\textbf{「在遇到如来后,便不存在,因为他是人中最胜之眼。}

Ye keci ganthā idha mohamaggā, aññāṇapakkhā vicikicchaṭhānā;\\
Tathāgataṃ patvā na te bhavanti, cakkhuñ hi etaṃ paramaṃ narānaṃ. %\hfill\textcolor{gray}{\footnotesize 5}

\begin{enumerate}\item 仍为赞美世尊,他说此颂为令生起言说之欲。其义为:\textbf{任何}贪欲等的\textbf{系缚},当未舍弃彼等时,由未舍弃愚痴与疑惑,而被称为\textbf{痴路、无知的品类、疑惑处}。这一切\textbf{在遇到如来后},以如来的开示之力而破碎,\textbf{便不存在}、消亡。什么原因?\textbf{因为他是人中最胜之眼},即是说因为如来由生起破碎一切系缚的慧眼,为人中最胜之眼。\end{enumerate}

\subsection\*{\textbf{351} {\footnotesize 〔PTS 348〕}}

\textbf{「因为若无人能驱散烦恼,如同风驱散层云,\\}
\textbf{「一切世间将黯淡、覆蔽,即便具光辉者也无法闪耀。}

No ce hi jātu puriso kilese, vāto yathā abbhaghanaṃ vihāne;\\
tamo v’assa nivuto sabbaloko, na jotimanto pi narā tapeyyuṃ. %\hfill\textcolor{gray}{\footnotesize 6}

\begin{enumerate}\item 仍为赞美,他说此颂为令生起言说之欲。这里,\textbf{能} \textit{jātu},即确然之词。\textbf{人},即指世尊而说。\textbf{具光辉者},即具有智慧之光的舍利弗等。这即是说:如果世尊未能如东方等类的风驱散层云般,以开示的冲力驱散烦恼,则世间好比为层云覆蔽而黯淡、一片黑暗,亦将为无知覆蔽而黯淡。连现在这些看来具光辉的舍利弗等,彼等众人也无法闪耀。\end{enumerate}

\subsection\*{\textbf{352} {\footnotesize 〔PTS 349〕}}

\textbf{「且智者们是制造光明者,英雄!我认为你就是如此,\\}
\textbf{「我们来到具观者、知者前,请在会众中为我们展示劫波!}

Dhīrā ca pajjotakarā bhavanti, taṃ taṃ ahaṃ vīra tath’eva maññe;\\
vipassinaṃ jānam upāgamumhā, parisāsu no āvikarohi Kappaṃ. %\hfill\textcolor{gray}{\footnotesize 7}

\begin{enumerate}\item 他仍以先前之法说此颂。其义为:\textbf{且智者们},即有智的人们,\textbf{是制造光明者},他们令智慧之光生起。所以,\textbf{英雄}、具足精勤与精进的世尊!\textbf{我认为你就是如此},我认为你就是智者和制造光明者。\textbf{我们来到具观者}——于一切法如实得见的世尊——\textbf{知者前},所以,\textbf{请在会众中为我们展示劫波},请宣说、阐明尼拘律劫波!\end{enumerate}

\subsection\*{\textbf{353} {\footnotesize 〔PTS 350〕}}

\textbf{「请快发美妙的话言!美妙者!如同天鹅伸展后,舒缓地和鸣!\\}
\textbf{「声音圆润,善加调音,我们全都正身倾听着你!}

Khippaṃ giraṃ eraya vaggu vagguṃ, haṃso va paggayha saṇikaṃ nikūja;\\
bindussarena suvikappitena, sabbe va te ujjugatā suṇoma. %\hfill\textcolor{gray}{\footnotesize 8}

\begin{enumerate}\item 他仍以先前之法说此颂。其义为:\textbf{请快发美妙的话言},请快速、不作拖延,说悦意的话语,世尊!好比金色的\textbf{天鹅}返回行处,看到天然的湖泊密林,便\textbf{伸展}、抬起了脖颈,以丹喙舒缓、不急地和鸣、发出美妙的话言,如是,请你也\textbf{舒缓地和鸣},以此大人相之一而\textbf{声音圆润},以善加准备、预备而\textbf{善加调音}。\textbf{我们全都正身}、不散乱心意,\textbf{倾听着你}的和鸣。\end{enumerate}

\subsection\*{\textbf{354} {\footnotesize 〔PTS 351〕}}

\textbf{「催促了已无余舍断了生死者、除遣者,我将请他说法,\\}
\textbf{「因为凡夫中没有随欲而行者,而诸如来则随思量而行。}

Pahīnajātimaraṇaṃ asesaṃ, niggayha dhonaṃ vadessāmi dhammaṃ;\\
na kāmakāro hi puthujjanānaṃ, saṅkheyyakāro ca Tathāgatānaṃ. %\hfill\textcolor{gray}{\footnotesize 9}

\begin{enumerate}\item 他仍以先前之法说此颂。这里,以无残余为无余,即是说不似须陀洹等残余任何,\textbf{已无余舍断了生死}。\textbf{催促},即善加请求、劝说。\textbf{除遣者},即除遣一切恶者。\textbf{因为凡夫中没有随欲而行者},即他们不能去了知或言说所愿求者。\textbf{而诸如来则随思量而行},而诸如来则随审视而行,以慧为先导而行,意即他们能了知或言说所愿求者。\end{enumerate}

\subsection\*{\textbf{355} {\footnotesize 〔PTS 352〕}}

\textbf{「你——正慧者的这圆满记说已被证实,\\}
\textbf{「施以这最后的合掌,了知者请莫愚弄!最上慧!}

Sampannaveyyākaraṇaṃ tavedaṃ\footnote{tavedaṃ:PTS 本作 tava-y-idaṃ。}, samujjupaññassa samuggahītaṃ;\\
ayam añjalī pacchimo suppaṇāmito, mā mohayī jānam anomapañña. %\hfill\textcolor{gray}{\footnotesize 10}

\begin{enumerate}\item 现在,为阐明此随思量而行,说了此颂。其义为:正如\textbf{你}——世尊——\textbf{正慧者}所说、所转\textbf{的这圆满记说},于处处\textbf{已被证实},如在「首相相续会涌起至七重多罗树高而般涅槃,善觉帝释会在第七天进入大地」等处\footnote{首相相续 \textit{Santati-mahāmatta}、善觉帝释 \textit{Suppabuddha Sakka}:两者之事均见\textbf{法句}义注。},已无颠倒得见。随后,便更作合掌而说:\textbf{施以这最后的合掌},施以这更进一步的合掌。\textbf{了知者}——了知劫波的趣向者——\textbf{请莫愚弄},莫以不谈论而愚弄。\textbf{最上慧},即称呼世尊。\end{enumerate}

\subsection\*{\textbf{356} {\footnotesize 〔PTS 353〕}}

\textbf{「知晓了各种圣法,了知者莫愚弄!最上雄!\\}
\textbf{「好比夏天患暑期待水,我期待言语,请倾吐声音吧!}

Parovaraṃ ariyadhammaṃ viditvā, mā mohayī jānam anomavīra;\\
vāriṃ yathā ghammani ghammatatto, vācābhikaṅkhāmi sutaṃ pavassa. %\hfill\textcolor{gray}{\footnotesize 11}

\begin{enumerate}\item 仍为请求莫愚弄,他以另一方法说了此颂。这里,\textbf{各种},即以世间、出世间而说的善妙、不善妙或远、近。\textbf{圣法},即四圣谛。\textbf{知晓},即通达。\textbf{了知者},即了知一切应知之法者。\textbf{我期待言语},\textbf{好比夏天患暑}之人疲累、焦渴而期待\textbf{水},如是,我期待你的言语。\textbf{请倾吐声音吧},即请倾吐、流淌、释放、转起被称为声音的声处吧!文本也作 sutassa vassa,意即倾洒所说品类的声处雨露。\end{enumerate}

\subsection\*{\textbf{357} {\footnotesize 〔PTS 354〕}}

\textbf{「劫波衍那所行的有义梵行,是否不是徒劳?\\}
\textbf{「他是涅槃了还是有余依?我们听听他如何解脱。」}

Yadatthikaṃ brahmacariyaṃ acarī, Kappāyano kacci’ssa taṃ amoghaṃ;\\
nibbāyi so ādu saupādiseso, yathā vimutto ahu taṃ suṇoma”. %\hfill\textcolor{gray}{\footnotesize 12}

\begin{enumerate}\item 现在,为阐明期待怎样的话语,说了此颂。这里,\textbf{劫波衍那},即以尊敬来说劫波。\textbf{如何解脱},即是问:是否如无学一样以无余依涅槃界,还是如有学一样以有余依?其余于此自明。\end{enumerate}

\subsection\*{\textbf{358} {\footnotesize 〔PTS 355〕}}

\textbf{「他切断了于此名色的渴爱、」世尊说,「黑者之流的长时随眠,\\}
\textbf{「无余度脱了生死。」五者最胜的世尊如是说。}

“Acchecchi taṇhaṃ idha nāmarūpe, \textit{(iti Bhagavā)} Kaṇhassa sotaṃ dīgharattānusayitaṃ;\\
atāri jātiṃ maraṇaṃ asesaṃ”, icc abravī Bhagavā pañcaseṭṭho. %\hfill\textcolor{gray}{\footnotesize 13}

\begin{enumerate}\item 如是,世尊为十二颂所请求,为作解答,说了此颂。这里,先说前句之义:欲爱等类的于此名色的渴爱,以长时未舍弃之义而为随眠,且被称为名为黑者的魔罗之流,劫波衍那断除了这作为\textbf{黑者之流的长时随眠、于此名色的渴爱}。而此中的「\textbf{世尊说}」则是结集者的话。\textbf{无余度脱了生死},他切断了这渴爱,无余度脱了生死,以无余依涅槃界而般涅槃。\textbf{五者最胜的世尊如是说},世尊为婆耆舍所问,便作此说。为最初五位弟子之五比丘中的最胜,或者,以信等五根、以戒等(五)法蕴、以极殊胜的(五)眼而为最胜。这仍是结集者的话。\end{enumerate}

\subsection\*{\textbf{359} {\footnotesize 〔PTS 356〕}}

\textbf{「听到你这言语,我得净喜,最善的仙人!\\}
\textbf{「看来我的提问并非徒劳,婆罗门没有欺骗我。}

“Esa sutvā pasīdāmi, vaco te isisattama;\\
amoghaṃ kira me puṭṭhaṃ, na maṃ vañcesi brāhmaṇo. %\hfill\textcolor{gray}{\footnotesize 14}

\begin{enumerate}\item 如是说已,婆耆舍对世尊之所说意有欢喜,说了如下几颂。这里,在初颂中,\textbf{最善的仙人},世尊为仙人,且以最上之义为最善,故为最善的仙人,又以名为毗婆尸、尸弃、毗舍婆、拘留孙、拘那含、迦叶等的六仙人与自己而为七,故为第七位仙人\footnote{最善的仙人、第七位仙人:原文均为 isisattama,义注给出两种解释。}——为称呼他而说。\textbf{没有欺骗我},因为已般涅槃,所以他没有以其般涅槃相欺骗生起希望的我,即非说谎者之义。其余于此自明。\end{enumerate}

\subsection\*{\textbf{360} {\footnotesize 〔PTS 357〕}}

\textbf{「如是说而如是行,他是佛陀的声闻,\\}
\textbf{「切断了死亡——欺瞒者——撒布的坚牢的网。}

Yathāvādī tathākārī, ahu Buddhassa sāvako;\\
acchidā maccuno jālaṃ, tataṃ māyāvino daḷhaṃ. %\hfill\textcolor{gray}{\footnotesize 15}

\begin{enumerate}\item 在第二颂中,因为他「希求解脱」而住,所以就此而说「\textbf{如是说而如是行,他是佛陀的声闻}」。\textbf{死亡撒布的网},即于三界流转中散布的魔罗的渴爱之网。\textbf{欺瞒者},即诸多伪善者。有些也读作 tathā māyāvino\footnote{这是将原文的 tataṃ māyāvino 读作 tathā māyāvino,即「撒布」作「如是」读。},彼等之意即:如是以多种伪善、多次靠近世尊的欺瞒者。\end{enumerate}

\subsection\*{\textbf{361} {\footnotesize 〔PTS 358〕}}

\textbf{「世尊!劫波见到了取著的源头,\\}
\textbf{「劫波衍那确实超越了极难度的死境。」}

Addasā Bhagavā ādiṃ, upādānassa Kappiyo;\\
accagā vata Kappāyano, maccudheyyaṃ suduttaran” ti. %\hfill\textcolor{gray}{\footnotesize 16}

\begin{enumerate}\item 在第三颂中,\textbf{源头},即原因。\textbf{取著},即流转。因为流转以可被取著之义,在此被说为「取著」。所说的意趣为:世尊!这样说是合适的——\textbf{劫波见到了}这取著的源头,无明、渴爱等类的原因。\textbf{确实超越},即确实越过。\textbf{死境},以死亡被安置于此而为死境,为三界流转的同义语。他充满喜悦,说到:他确实超越了这\textbf{极难度}的死境。其余于此自明。\end{enumerate}

\begin{center}\vspace{1em}尼拘律劫波经第十二\\Nigrodhakappasuttaṃ dvādasamaṃ.\end{center}