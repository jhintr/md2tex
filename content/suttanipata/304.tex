\section{孙陀利迦婆罗豆婆遮经}

\begin{center}Sundarikabhāradvāja Sutta\end{center}\vspace{1em}

\textbf{如是我闻。一时世尊住㤭萨罗孙陀利迦河的岸边。尔时,孙陀利迦婆罗豆婆遮婆罗门在孙陀利迦河的岸边献火供、事火祭。然后,孙陀利迦婆罗豆婆遮婆罗门献了火供、事了火祭,便从坐起,观察周围四方:「谁当享用这祭品的残留?」孙陀利迦婆罗豆婆遮婆罗门看到世尊在不远处的某棵树下蒙头而坐,看到后,左手拿了祭品的残留,右手拿了水壶,往世尊处走去。}

Evaṃ me sutaṃ— ekaṃ samayaṃ Bhagavā Kosalesu viharati Sundarikāya nadiyā tīre. Tena kho pana samayena Sundarikabhāradvājo brāhmaṇo Sundarikāya nadiyā tīre aggiṃ juhati, aggihuttaṃ paricarati. Atha kho Sundarikabhāradvājo brāhmaṇo aggiṃ juhitvā aggihuttaṃ paricaritvā uṭṭhāyāsanā samantā catuddisā anuvilokesi: “ko nu kho imaṃ habyasesaṃ bhuñjeyyā” ti? Addasā kho Sundarikabhāradvājo brāhmaṇo Bhagavantaṃ avidūre aññatarasmiṃ rukkhamūle sasīsaṃ pārutaṃ nisinnaṃ, disvāna vāmena hatthena habyasesaṃ gahetvā dakkhiṇena hatthena kamaṇḍaluṃ gahetvā yena Bhagavā ten’upasaṅkami.

\begin{enumerate}\item \textbf{祭饼经} \textit{Pūraḷāsasutta}\footnote{这是义注中的经题。}。缘起为何?世尊在饭后义务的终了,以佛眼观察世间,便见到孙陀利迦婆罗豆婆遮婆罗门具足阿罗汉的近依,且了知到「当我到达那里,将发生谈话,随后,在谈话终了,听闻了法的开示,这婆罗门将出家而圆满阿罗汉」,便去到那里,引起谈话,说了此经。
\item 这里,\textbf{如是我闻}等是结集者的话,\textbf{先生是何出身}等是这婆罗门的,\textbf{我不是婆罗门}等是世尊的,这一切汇集后,被称为「祭饼经」。这里,当知与已述相同的仍如所述之法,我们只解释未述者,且不涉及语义自明的词句。
\item \textbf{㤭萨罗},即名为㤭萨罗的国中众王子,他们所居的一方国土便俗称作「㤭萨罗」——即于此㤭萨罗国。而有些人解释说,因为先前「大喝王子」见到种种舞蹈都未露丝毫微笑,国王听闻后,便命令「若有人能让我的孩子笑出来,我就用一切璎珞装饰他」。随后,众人都丢开了犁,聚集起来。但七年多来,这些人表演了种种戏耍等,都未能让他笑出来。随后,帝释便派遣天舞者,他表演天舞后,便让王子笑了出来。于是,这些人便朝各自的住处离开,朋友亲人等在路上看见他们,就欢迎说:「您还好吧,先生!您还好吧,先生!」所以由「好 \textit{kusala}」字的发音,这地方便被称为「㤭萨罗 \textit{Kosala}」。\textbf{孙陀利迦河的岸边},即名为孙陀利迦之河的岸边。
\item \textbf{尔时},即世尊欲调伏此婆罗门而前往,在此岸边蒙头,在树下以被称为坐的威仪住而住之时。\textbf{孙陀利迦婆罗豆婆遮婆罗门},此婆罗门在此河的岸边居住,并献火供,而其族为「婆罗豆婆遮」,故而如是得称。\textbf{献火供},即将祭品投入火中燃烧。\textbf{事火祭},即以洒扫、涂膏、供奉等承事火处。
\item \textbf{谁当享用这祭品的残留},据说,这婆罗门献了火供,看到残留的粥,便想:「在火中投入的粥先由大梵享用,却仍有残留,若我布施给从梵天的口中出生的婆罗门,则我的孩子和我父亲也会高兴,且趣向梵界之路会极清净,噫!我去寻找婆罗门!」随后,为遇见婆罗门,\textbf{便从坐起,观察四方}:「谁当享用这祭品的残留?」
\item \textbf{在某棵树下},即在此密林中最胜的树下。\textbf{蒙头},即连头一起披覆身体。但世尊为什么这样做?难道有着被称为那罗衍那之力\footnote{那罗衍那之力 \textit{Nārāyana-bala}:菩提比丘注 1372 云,据\textbf{分别论义注},那罗衍那之力等于一千俱胝大象和一万俱胝人的力气。},尚不能抵御落雪和寒风吗?这自有原因。因为诸佛不全是为了料理身体而这么做,而是世尊想到「当婆罗门来时,我将揭开头,婆罗门见了我,将发起谈话,然后随着谈话,我将开示法」,为了发起谈话而这么做。
\item \textbf{看到后,左手……走去},据说,这婆罗门见到世尊,便作婆罗门想「他蒙了头,整夜精勤,施了这供水后,我再施祭品的残留」,便走去。\end{enumerate}

\textbf{于是,随着孙陀利迦婆罗豆婆遮婆罗门的脚步声,世尊便揭开了头。然后,孙陀利迦婆罗豆婆遮婆罗门想「这先生是光头,这先生是秃头」,便想从此回去。然后,孙陀利迦婆罗豆婆遮婆罗门便想:「于此,有些婆罗门也是光头,我何不前去问问出身?」然后,孙陀利迦婆罗豆婆遮婆罗门往世尊处走去,走到后,对世尊说:「先生是何出身?」}

Atha kho Bhagavā Sundarikabhāradvājassa brāhmaṇassa padasaddena sīsaṃ vivari. Atha kho Sundarikabhāradvājo brāhmaṇo: “muṇḍo ayaṃ bhavaṃ, muṇḍako ayaṃ bhavan” ti tato va puna nivattitukāmo ahosi. Atha kho Sundarikabhāradvājassa brāhmaṇassa etad ahosi: “muṇḍā pi hi idh’ekacce brāhmaṇā bhavanti, yan nūnāhaṃ upasaṅkamitvā jātiṃ puccheyyan” ti. Atha kho Sundarikabhāradvājo brāhmaṇo yena Bhagavā ten’upasaṅkami, upasaṅkamitvā Bhagavantaṃ etad avoca: “kiṃjacco bhavan” ti?

\begin{enumerate}\item \textbf{这先生是光头,这先生是秃头},甫一揭开头,他便看见发尖而说「光头」。随后,经仔细观察,连分毫也不见,便轻蔑地说「秃头」。因为婆罗门的见便是如此。\textbf{从此},即从所站立而见之处。\textbf{也是光头},即以某些原因也剃了光头。\end{enumerate}

\textbf{于是,世尊以偈颂对孙陀利迦婆罗豆婆遮婆罗门说:}

Atha kho Bhagavā Sundarikabhāradvājaṃ brāhmaṇaṃ gāthāhi ajjhabhāsi:

\subsection\*{\textbf{458} {\footnotesize 〔PTS 455〕}}

\textbf{「我既非婆罗门,亦非王子,不是吠舍或是任何其他,\\}
\textbf{「遍知了凡夫们的种姓,无所牵绊,我以考量在世间游行。}

“Na brāhmaṇo no’mhi na rājaputto, na vessāyano uda koci no’mhi;\\
gottaṃ pariññāya puthujjanānaṃ, akiñcano manta carāmi loke. %\hfill\textcolor{gray}{\footnotesize 1}

\begin{enumerate}\item \textbf{我既非婆罗门},此中的「非 \textit{na}」字为遮止,「既 \textit{no}」字为强调,如同\begin{quoting}都不能 \textit{na no}(与如来)等同。(经集第 226 颂)\end{quoting}等处,以此显示我绝非婆罗门。\textbf{亦非王子},即不是刹帝利。\textbf{或是任何其他},即我不是任何其他首陀罗或旃陀罗。如是便完整地拒斥了出身论的攻击。为什么?因为如同众流汇入大海,众族姓子既已出家,便舍弃了先前的姓名、种姓。且此处有「般诃罗陀经\footnote{般诃罗陀经 \textit{Pahārādasutta}:即\textbf{增支部}第 8:19 经。}」为证。
\item 如是拒斥了出身论后,为了如实揭露自身,便说「\textbf{遍知了凡夫们的种姓,无所牵绊,我以考量在世间游行}」。设问:如何遍知种姓?因为世尊以三遍知而遍知五蕴,而当此等遍知时,即遍知了种姓。以无任何贪等,他无所牵绊。考量、知晓已,以与智随转的身业等而行,因此说「遍知了……在世间游行」。考量即是慧,且他以此而行,因此说「我以考量在世间游行」,因韵律而作短音\footnote{这是说颂中的「考量 \textit{manta}」原本应作 mantā。关于 mantā,义注给出了两种解释,一作动词,一作名词,在此不详述。}。\end{enumerate}

\subsection\*{\textbf{459} {\footnotesize 〔PTS 456〕}}

\textbf{「穿著僧伽梨,我无家而行,剃去头发,内在寂静,\\}
\textbf{「于此不著于世人,婆罗门!你问我种姓的问题不合适。」}

Saṅghāṭivāsī agaho carāmi, nivuttakeso abhinibbutatto;\\
alippamāno idha māṇavehi, akallaṃ maṃ brāhmaṇa pucchasi gottapañhaṃ”. %\hfill\textcolor{gray}{\footnotesize 2}

\begin{enumerate}\item 如是揭露了自身,现在,为向婆罗门提出诘难「都已见到如是明显的特相,你还不知晓什么该问、什么不该问」,说了此颂。且此中,以连缀截断之义,「僧伽梨」意指三衣,以穿著彼等为\textbf{穿著僧伽梨}。\textbf{无家},意即无渴爱。虽然世尊所住之家有祇园的大香房、树圆亭、㤭赏弥寮、旃檀亭等多种,但与此无涉。\textbf{剃去头发},即脱去头发,即是说翦除须发。\textbf{内在寂静},即心极止息了热恼,或心有守护。
\item \textbf{于此不著于世人},由舍弃对资助的爱执,不著于众人、不交际、完全地独处。\textbf{婆罗门……不合适},即我如是穿著僧伽梨……于此不著于世人,婆罗门!我于过去既已出家,你问这原本的姓名、种姓之问便不恰当。\end{enumerate}

\subsection\*{\textbf{460} {\footnotesize 〔PTS 457\textit{a-b}〕}}

\textbf{「先生!婆罗门与婆罗门一起,总是问:『您是婆罗门吗?』」}

“Pucchanti ve bho brāhmaṇā brāhmaṇebhi, saha ‘brāhmaṇo no bhavan’ ti”. %\hfill\textcolor{gray}{\footnotesize 3}

\begin{enumerate}\item 如是说已,婆罗门为摆脱诘难,说了此句。这即是说:先生!我问的并非不合适。因为在我们婆罗门的集会\footnote{集会 \textit{samaya}:或可译作「教义」。}中,婆罗门与婆罗门碰到一起,总这样问出身和种姓:「您是婆罗门吗?您是婆罗豆婆遮吗?」\end{enumerate}

\subsection\*{\textbf{461} {\footnotesize 〔PTS 457\textit{c-f}〕}}

\textbf{「因为若你说是婆罗门,而说我非婆罗门,\\}
\textbf{「我就来问问这三句、二十四音节的颂诗。」}

“Brāhmaṇo hi ce tvaṃ brūsi, mañ ca brūsi abrāhmaṇaṃ;\\
taṃ taṃ sāvittiṃ pucchāmi, tipadaṃ catuvīsatakkharaṃ”. %\hfill\textcolor{gray}{\footnotesize 4}

\begin{enumerate}\item 如是说已,世尊为令婆罗门的心柔和,阐明自己熟稔颂诗,说了此颂。其义为:如果你说「我是婆罗门」,\textbf{而说我非婆罗门},那么,\textbf{我就来问问}您,\textbf{这三句、二十四音节的颂诗},请对我说!
\item 且此中,世尊是就作为第一义吠陀的三藏的开卷,就作为第一义婆罗门的一切诸佛所阐明的具足义、具足文的「我皈依佛、我皈依法、我皈依僧」的圣颂诗\footnote{颂诗 \textit{sāvitti/sāvitrī}:本是梨俱吠陀中颂诗的名称,因致敬太阳 \textit{savitṛ} 而得名,也称 gāyatrī,不过这里,义注说指的是「我皈依佛、我皈依法、我皈依僧」这三句,在巴利中正好是二十四音节。}而问的。因为要是婆罗门说其它的,世尊便会对他说「婆罗门!这在圣律中不被称为颂诗」,在显示其非坚实后,唯令他住立于此(教法)。\end{enumerate}

\subsection\*{\textbf{462} {\footnotesize 〔PTS 458\textit{a-c}〕}}

\textbf{「以何依据,仙人、世人、刹帝利、婆罗门向诸天\\}
\textbf{「举行各种献牲,在此世间?」}

“Kiṃnissitā isayo manujā, khattiyā brāhmaṇā devatānaṃ;\\
yañña-m-akappayiṃsu puthū idha loke”. %\hfill\textcolor{gray}{\footnotesize 5}

\begin{enumerate}\item 然而,婆罗门在听到「我就来问问三句、二十四音节的颂诗」,这成就了自身的教法、有着颂诗的相与文、以梵音说出的话后,得出结论「确实,这沙门在婆罗门教中已得究竟,而我以无智轻蔑道『他非婆罗门』,他实是好样的、通晓颂诗的婆罗门,噫!我来问他献牲的方法和供养的方法」,为问此义,说了这三句不等的偈颂\footnote{三句不等的偈颂:指此颂的三句音节数不等。}。
\item 其义为:\textbf{以何依据}、以何意趣、愿求什么,\textbf{仙人、刹帝利、婆罗门}与其他\textbf{世人}为了\textbf{诸天}的义利举行献牲?Yañña-m-akappayiṃsu 中的 m 字作词的连接。\textbf{举行},即安排、实行。\textbf{各种},即众多,以饮食布施等类而有许多品类,或者,各种仙人、世人、刹帝利、婆罗门以何依据举行献牲?他们的业如何成功?他以此意趣而问。\end{enumerate}

\subsection\*{\textbf{463} {\footnotesize 〔PTS 458\textit{d-e}〕}}

\textbf{「若到达边际者、通达诸明者在献牲时,能从中得到祭品,我说,他便成功。」}

“Ya-d-antagū vedagū yaññakāle, yassāhutiṃ labhe tass’ijjhe ti brūmi”. %\hfill\textcolor{gray}{\footnotesize 6}

\begin{enumerate}\item 于是,世尊为向其解释此义,说了这剩余的两句。这里,\textbf{Ya-d-antagū},即 yo antagū,a 字代替 o 字,而 d 字作词的连接,如同 asādhāraṇa-m-aññesan 等处的 m 字一般。
\item 而其义为:\textbf{若}以三遍知到达流转之苦的边际的\textbf{到达边际者},与以四道智之明穿透烦恼而通达的\textbf{通达诸明者},他\textbf{在}仙人、世人、刹帝利、婆罗门中任一的\textbf{献牲时},\textbf{能从}任何被供奉的食物——乃至林中的叶、根、果等——\textbf{中得到祭品},\textbf{我说,他}的这献牲之业\textbf{便成功}、兴盛、得大果报。\end{enumerate}

\subsection\*{\textbf{464} {\footnotesize 〔PTS 459〕}}

\textbf{「确实,这献祭成功,」婆罗门说,「当我们见了这样的通达诸明者,\\}
\textbf{「因为没有得见像你这样的人,其他人便享用了祭饼。」}

“Addhā hi tassa hutam ijjhe, \textit{(iti brāhmaṇo)} yaṃ tādisaṃ vedagum addasāma;\\
tumhādisānañ hi adassanena, añño jano bhuñjati pūraḷāsaṃ”. %\hfill\textcolor{gray}{\footnotesize 7}

\begin{enumerate}\item 于是,婆罗门听闻了世尊这与第一义相应而甚深、具足极甜美的发音与淡定之声的开示,为尊敬其身成就之清净与一切功德的成就,生起喜乐,说了此颂。这里,「\textbf{婆罗门说}」是结集者的话,其余则是婆罗门的。
\item 其义为:\textbf{确实},我的\textbf{这献祭成功},今天,这所施会成功、兴盛、得大果报,\textbf{当我们见了这样的通达诸明者},因为我们见了像您这样的通达诸明者。因为你便是这通达诸明者,而非他人。而此前\textbf{因为没有得见像你这样}通达诸明、到达边际的人,\textbf{其他人便享用了}在我们这样的献牲中准备的\textbf{祭饼}、供品和饼。\end{enumerate}

\subsection\*{\textbf{465} {\footnotesize 〔PTS 460〕}}

\textbf{「所以,婆罗门!你在此希求义利,上前来问!\\}
\textbf{「兴许于此,能发现寂静、无烟、无患、无待的善慧者。」}

“Tasmāt iha tvaṃ brāhmaṇa atthena, atthiko upasaṅkamma puccha;\\
santaṃ vidhūmaṃ anīghaṃ nirāsaṃ, app ev’idha abhivinde sumedhaṃ”. %\hfill\textcolor{gray}{\footnotesize 8}

\begin{enumerate}\item 随后,世尊在确认了婆罗门已对自己净喜、准备接受话语之后,为让他完全明了,而欲以种种方式阐明应供者,说了此颂。其义为:因为你对我净喜,\textbf{所以},为显示自身,而说:\textbf{婆罗门!你在此,上前来问}!现在,在此之前的「希求义利」一词应与后句相连:\textbf{希求义利,兴许于此},即很可能就立于此处,或于此教法,你\textbf{能发现}、能得到、证得随适于此义利希求之相的以止息烦恼之火而\textbf{寂静}、以消逝忿怒之烟而\textbf{无烟}、以无有苦而\textbf{无患}、以无各种希求而\textbf{无待}的最上慧、漏尽应供的\textbf{善慧者}。
\item 或者,当知此中也可如是连结:因为你对我净喜,所以,你,婆罗门!当在此希求义利时,请上前来问寂静、无烟、无患、无待——为显示自身而说。当如是发问时,兴许于此,能发现漏尽应供的善慧者。\end{enumerate}

\subsection\*{\textbf{466} {\footnotesize 〔PTS 461〕}}

\textbf{「我乐于献牲,乔达摩君!欲行献牲,却不知晓,\\}
\textbf{「请您教授我!何处献祭能成功?请对我说!」}

“Yaññe rato’haṃ bho Gotama, yaññaṃ yiṭṭhukāmo nāhaṃ pajānāmi;\\
anusāsatu maṃ bhavaṃ, yattha hutaṃ ijjhate brūhi me taṃ”. %\hfill\textcolor{gray}{\footnotesize 9}

\begin{enumerate}\item 于是,婆罗门如所教授而行,对世尊说了此颂。这里,\textbf{献牲}、祭祀、布施的之义相同。所以,我乐于布施,正因此乐于布施而欲行布施,但我却不知晓,\textbf{请您教授}如是无知的\textbf{我}!且教授时,\textbf{请}以明显的方法\textbf{对我说}:\textbf{何处献祭能成功}?当知如是连结此中之义。文本(何处献祭)也作「如何献祭 \textit{yathāhutaṃ}」。\end{enumerate}

\subsection\*{\textbf{467} {\footnotesize 〔PTS 462〕}}

\textbf{「既然如此,你,婆罗门!请注意听!我将对你说法:}

“Tena hi tvaṃ, brāhmaṇa, odahassu sotaṃ, dhammaṃ te desessāmi—

\begin{enumerate}\item 于是,世尊欲对他说,便说了「既然如此……我将对你说法」。且为教授已注意倾听的他,先说了此颂。\end{enumerate}

\textbf{「莫问出身,当问行为,从薪实能生火,\\}
\textbf{「卑贱之家者也可成为坚定、高贵、以惭禁止的牟尼。}

Mā jātiṃ pucchī caraṇañ ca puccha, kaṭṭhā have jāyati jātavedo;\\
nīcākulīno pi munī dhitīmā, ājāniyo hoti hirīnisedho. %\hfill\textcolor{gray}{\footnotesize 10}

\begin{enumerate}\item 这里,\textbf{莫问出身},即如果你期望祭祀成功、布施得大果报,不应问出身。因为出身不是检视应供者的原因。\textbf{当问行为},而当问戒等德的行为。因为这才是检视应供者的原因。
\item 现在,为向他阐明此义,便举例说「\textbf{从薪实能生火}」。此处的意趣为:于此,从薪生火,不是只从沙罗树等薪木所生者起火的作用,从狗槽等薪木所生者不起,而是由自身具足火焰等德而起。如是,不是只从婆罗门家族等所生者便成应供者,从旃陀罗家族等所生者不成,而是不论\textbf{卑贱之家者}或上等之家者,都\textbf{可成为坚定、以惭禁止、高贵}的漏尽\textbf{牟尼},以成就此坚定与惭为首的德而成具有出身的最上应供者。因为他以坚定保持诸德,以惭禁止过失。如说:\begin{quoting}因为惭,诸善人不作恶。\end{quoting}因此我对你说了此颂。这是略说,而详说当依「中部·马刈经」而知。\end{enumerate}

\subsection\*{\textbf{468} {\footnotesize 〔PTS 463〕}}

\textbf{「以真实而调御,具足调御,通达诸明,梵行已立,\\}
\textbf{「希求福德的婆罗门若欲祭祀,应适时给予他祭品。}

Saccena danto damasā upeto, vedantagū vūsitabrahmacariyo;\\
kālena tamhi habyaṃ pavecche, yo brāhmaṇo puññapekkho yajetha. %\hfill\textcolor{gray}{\footnotesize 11}

\begin{enumerate}\item 如是,世尊在教授了四种姓的清净后,现在,为显示在何处献祭成功及如何献祭成功之义,说了这初颂。这里,\textbf{以真实},即以第一义真实,因为证得此而为\textbf{调御},故说「以真实而调御」。\textbf{具足调御},即具足根的调御。\textbf{通达诸明},即或以诸明到达烦恼的边际,或到达诸明边际的第四道智。\textbf{梵行已立},即由不需再住而已住于道梵行\footnote{道梵行,见\textbf{吉祥经}第 270 颂注。}。\textbf{应适时给予他祭品},即注意到自己有所施之时和他人现前之时后,当在此时给予\footnote{给予 \textit{pavecchati}:一般认为 payacchati > payecchati > pavecchati,不同于义注给出的「引介 \textit{paveseti}」,详见 Norman 及 PED。}、引介、供与这样的应供者以所施。\end{enumerate}

\subsection\*{\textbf{469} {\footnotesize 〔PTS 464〕}}

\textbf{「舍弃了爱欲,无家而行,善加自制,如梭子般正直,\\}
\textbf{「希求福德的婆罗门若欲祭祀,应适时给予他们祭品。}

Ye kāme hitvā agahā caranti, susaññatattā tasaraṃ va ujjuṃ;\\
kālena tesu habyaṃ pavecche, yo brāhmaṇo puññapekkho yajetha. %\hfill\textcolor{gray}{\footnotesize 12}

\begin{enumerate}\item \textbf{爱欲},即物欲和烦恼欲\footnote{两种爱欲,见\textbf{犀牛角经}第 50 颂注。}。\end{enumerate}

\subsection\*{\textbf{470} {\footnotesize 〔PTS 465〕}}

\textbf{「离于贪染,善等持诸根,如月亮解脱于罗睺的束缚,\\}
\textbf{「希求福德的婆罗门若欲祭祀,应适时给予他们祭品。}

Ye vītarāgā susamāhitindriyā, cando va Rāhuggahaṇā pamuttā;\\
kālena tesu habyaṃ pavecche, yo brāhmaṇo puññapekkho yajetha. %\hfill\textcolor{gray}{\footnotesize 13}

\begin{enumerate}\item \textbf{善等持诸根},即善加等持诸根,即是说诸根不散乱。\textbf{如月亮解脱于罗睺的束缚},即好比月亮从罗睺的束缚中解脱,如是从烦恼的束缚中解脱,极闪耀和光辉。\end{enumerate}

\subsection\*{\textbf{471} {\footnotesize 〔PTS 466〕}}

\textbf{「无所羁绊地行于世间,始终具念,舍弃了执为我者,\\}
\textbf{「希求福德的婆罗门若欲祭祀,应适时给予他们祭品。}

Asajjamānā vicaranti loke, sadā satā hitvā mamāyitāni;\\
kālena tesu habyaṃ pavecche, yo brāhmaṇo puññapekkho yajetha. %\hfill\textcolor{gray}{\footnotesize 14}

\begin{enumerate}\item \textbf{执为我者},即以爱、见而执为我者。\end{enumerate}

\subsection\*{\textbf{472} {\footnotesize 〔PTS 467〕}}

\textbf{「舍弃了爱欲,征服而行,他知道生死的边际,\\}
\textbf{「已止息,如池水般清凉,如来应得祭饼。}

Yo kāme hitvā abhibhuyyacārī, yo vedi jātīmaraṇassa antaṃ;\\
parinibbuto udakarahado va sīto, Tathāgato arahati pūraḷāsaṃ. %\hfill\textcolor{gray}{\footnotesize 15}

\begin{enumerate}\item 从此开始,(世尊)就自身而说。这里,\textbf{舍弃了爱欲},即舍弃了烦恼欲。\textbf{征服而行},即由舍弃了彼等而于物欲征服而行。\textbf{他知道}名为涅槃的\textbf{生死的边际},即以自身的慧力而知。\textbf{如池水般清凉},即如阿耨达池、耳秃池、造车池、六牙池、杜鹃池、曼陀吉尼池、狮崖池等雪山中的七大池,由不为火、日的炎热所触而恒久清凉,由\textbf{已止息}烦恼的热恼故,如其中任一池水般清凉。\end{enumerate}

\subsection\*{\textbf{473} {\footnotesize 〔PTS 468〕}}

\textbf{「与相同者相同,与不同者差远,如来是无尽慧者,\\}
\textbf{「不染于此世或他世,如来应得祭饼。}

Samo samehi visamehi dūre, Tathāgato hoti anantapañño;\\
anūpalitto idha vā huraṃ vā, Tathāgato arahati pūraḷāsaṃ. %\hfill\textcolor{gray}{\footnotesize 16}

\begin{enumerate}\item \textbf{相同},即相等。\textbf{相同者},即毗婆尸等佛。由通达相同故,他们被称为「相同者」。他们在以通达应证的功德上,或应舍弃的过失上没有差别,但他们有时量、寿量、家族、身量、出离、精勤、菩提树与光上的差别。
\item 因为他们最少以四阿僧祇又十万劫来圆满波罗蜜,最多以十六阿僧祇又十万劫,这是他们的\textbf{时量差别}。最少投生在寿量百年之时,最多在寿量十万年之时,这是他们的\textbf{寿量差别}。投生在刹帝利家族或婆罗门家族,这是\textbf{家族差别}。高者有八十八肘量,矮者为十五、十八肘量\footnote{肘 \textit{hattha}:为长度单位,一肘为廿四指节 \textit{aṅgulapabba} 或四分之一寻 \textit{vyāma}。},这是\textbf{身量差别}。以象、马、车、轿等出离,或以空中,因为毗婆尸、拘留孙以马车出离,尸弃、拘那含以象背,毗舍婆以轿,迦叶以空中,释迦牟尼以马背,这是\textbf{出离差别}。从事精勤或七天、半月、一月、二月、三月、四月、五月、六月、一年、二三四五六年,这是\textbf{精勤差别}。或无花果树为菩提树,或榕树等中的某种,这是\textbf{菩提树差别}。他们与一寻、八十寻或无量光相应。这里,一寻光或八十寻光对一切相同,而无量光可至远处或近处,一牛呼、二牛呼、一由旬、数由旬乃至轮围的边界,吉祥佛的身光可至一万轮围。即便如此,对一切佛,唯依于意的思量,希望多远便能到达多远,这是\textbf{光差别}。除了这八种差别,他们在其余以通达应证的功德上,或应舍弃的过失上没有特殊之处,所以说是「相同者」。如是,与这些相同者相同。
\item \textbf{与不同者差远},非相同者为不同者,即辟支佛等其他一切有情,以不等同性而与这些不同者差远。因为众辟支佛即便以跏趺抵跏趺而坐,遍满整个阎浮提,其功德不及一正等正觉者的十六分之一,遑论声闻等?因此说「与不同者差远」,应以「\textbf{如来是}」两词与「差远」相连。\textbf{无尽慧者},即无量慧者。因为与世人的智慧相较,第八者\footnote{第八者 \textit{aṭṭhamaka}:即四双八辈中的须陀洹向。}的智慧更上,与其智慧相较,须陀洹的更上,如是乃至与阿罗汉的智慧相较,辟支佛的智慧更上,而与辟支佛的智慧相较,如来的智慧不应说为「更上」,而应说为「无尽」,因此说「无尽慧者」。
\item \textbf{不染},即不为爱、见所涂抹而染。\textbf{此世或他世},即在此世间或在他世间。而此中的连结为:如来与相同者相同,与不同者差远,为什么?因为他是无尽慧者,不染于此世或他世,因此,如来应得祭饼。\end{enumerate}

\subsection\*{\textbf{474} {\footnotesize 〔PTS 469〕}}

\textbf{「伪善、慢不住于他,他离贪,无我所,无待,\\}
\textbf{「去除忿怒,内在寂静,这婆罗门舍弃了忧尘,\\}
\textbf{「如来应得祭饼。}

Yamhi na māyā vasati na māno, yo vītalobho amamo nirāso;\\
paṇunnakodho abhinibbutatto, yo brāhmaṇo sokamalaṃ ahāsi;\\
Tathāgato arahati pūraḷāsaṃ. %\hfill\textcolor{gray}{\footnotesize 17}

\begin{enumerate}\item 此颂及其它类此者,当知是为舍弃对于与伪善等过失相应的婆罗门的应供之想而说。这里,\textbf{无我所},即于有情、诸行等已舍弃「这是我的」的执为我之相。\end{enumerate}

\subsection\*{\textbf{475} {\footnotesize 〔PTS 470〕}}

\textbf{「舍弃了意的住处,他没有任何执取,\\}
\textbf{「无取于此世或他世,如来应得祭饼。}

Nivesanaṃ yo manaso ahāsi, pariggahā yassa na santi keci;\\
anupādiyāno idha vā huraṃ vā, Tathāgato arahati pūraḷāsaṃ. %\hfill\textcolor{gray}{\footnotesize 18}

\begin{enumerate}\item \textbf{住处},即爱、见的住处。因为意以此而住于三有,因此称为「意的住处」。或者,因为唯住于此处,不能舍此而行,因此称为「住处」。\textbf{执取},即爱、见,或以两者所执取之法。\textbf{任何},即哪怕少许之量。\textbf{无取},即由无有此等住处、执取而无取于任何法。\end{enumerate}

\subsection\*{\textbf{476} {\footnotesize 〔PTS 471〕}}

\textbf{「等持,他度过暴流,以最高的见了知了法,\\}
\textbf{「漏尽,持最后身,如来应得祭饼。}

Samāhito yo udatāri oghaṃ, dhammaṃ c’aññāsi paramāya diṭṭhiyā;\\
khīṇāsavo antimadehadhārī, Tathāgato arahati pūraḷāsaṃ. %\hfill\textcolor{gray}{\footnotesize 19}

\begin{enumerate}\item \textbf{等持},即以道定。\textbf{了知了法},即了知了一切应知之法。\textbf{以最高的见},即以一切知智。\end{enumerate}

\subsection\*{\textbf{477} {\footnotesize 〔PTS 472〕}}

\textbf{「他的有漏与粗砺之语,已熏散、消尽而无存,\\}
\textbf{「他通达诸明,于一切处解脱,如来应得祭饼。}

Bhavāsavā yassa vacī kharā ca, vidhūpitā atthagatā na santi;\\
sa vedagū sabbadhi vippamutto, Tathāgato arahati pūraḷāsaṃ. %\hfill\textcolor{gray}{\footnotesize 20}

\begin{enumerate}\item \textbf{有漏},即伴随常见的对有、贪、禅那、欣求的贪染。\textbf{粗砺},即酷虐、粗恶。\textbf{熏散},即烧尽。\textbf{无存},即由熏散与消尽故,而两者应与两者相连\footnote{即「熏散、消尽」应分别与「有漏、粗砺之语」相连。}。\textbf{于一切处},即于一切蕴、处等。\end{enumerate}

\subsection\*{\textbf{478} {\footnotesize 〔PTS 473〕}}

\textbf{「超越执著,他已没有执著,在有慢的有情中,为无慢的有情,\\}
\textbf{「遍知了有田与物之苦,如来应得祭饼。}

Saṅgātigo yassa na santi saṅgā, yo mānasattesu amānasatto;\\
dukkhaṃ pariññāya sakhettavatthuṃ, Tathāgato arahati pūraḷāsaṃ. %\hfill\textcolor{gray}{\footnotesize 21}

\begin{enumerate}\item \textbf{有慢的有情},即以慢而固著者。\textbf{遍知了苦},以三遍知遍知了流转之苦。\textbf{有田与物},即有因与缘,即是说与业、烦恼俱。\end{enumerate}

\subsection\*{\textbf{479} {\footnotesize 〔PTS 474〕}}

\textbf{「不依希望,得见远离,超越他人所知的见,\\}
\textbf{「他没有任何所缘,如来应得祭饼。}

Āsaṃ anissāya vivekadassī, paravediyaṃ diṭṭhim upātivatto;\\
ārammaṇā yassa na santi keci, Tathāgato arahati pūraḷāsaṃ. %\hfill\textcolor{gray}{\footnotesize 22}

\begin{enumerate}\item \textbf{不依希望},即不跟随渴爱。\textbf{得见远离},即得见涅槃。\textbf{他人所知的},即由他人令知者。\textbf{超越见},即越过六十二种邪见。\textbf{所缘},即缘,即是说再有的原因。\end{enumerate}

\subsection\*{\textbf{480} {\footnotesize 〔PTS 475〕}}

\textbf{「他所体认的上下诸法,已熏散、消尽而无存,\\}
\textbf{「寂静,于取的灭尽解脱,如来应得祭饼。}

Paroparā yassa samecca dhammā, vidhūpitā atthagatā na santi;\\
santo upādānakhaye vimutto, Tathāgato arahati pūraḷāsaṃ. %\hfill\textcolor{gray}{\footnotesize 23}

\begin{enumerate}\item \textbf{上下},即尊卑、善妙不善妙,或者以外为上,以内为下。\textbf{体认},即以智通达。\textbf{诸法},即蕴、处等法。\textbf{于取的灭尽解脱},即于涅槃,由涅槃为所缘而解脱,以涅槃为所缘而得解脱之义。\end{enumerate}

\subsection\*{\textbf{481} {\footnotesize 〔PTS 476〕}}

\textbf{「得见结缚与生的尽头,他无余除去了贪路,\\}
\textbf{「清净、无过、无垢、无瑕,如来应得祭饼。}

Saṃyojanaṃ-jāti-khayantadassī, yo pānudi rāgapathaṃ asesaṃ;\\
suddho nidoso vimalo akāco, Tathāgato arahati pūraḷāsaṃ. %\hfill\textcolor{gray}{\footnotesize 24}

\begin{enumerate}\item \textbf{得见结缚与生的尽头},即得见结缚的尽头与得见生的尽头。且此中,以结缚的尽头来说有余依涅槃界,以生的尽头来说无余依。因为「尽头」是究竟灭尽的正断断的同义语。且此中的鼻音,如 vivekajaṃ pītisukhaṃ 等处一般,未予省略。\textbf{贪路},即贪的所缘,或即是贪。由为恶趣之路故,贪被称为「贪路」,如说「业道 \textit{kammapatha}」。
\item 清净、无过、无垢、无瑕等,由遍净的身正行等为\textbf{清净},由无有被称为「这人类有贪的过失、嗔的过失、痴的过失」者为\textbf{无过},以离八种人的垢秽为\textbf{无垢}\footnote{八种人的垢秽,见\textbf{法句}·垢秽品第 241~243 颂。},以无随烦恼为\textbf{无瑕}。因为被随烦恼所染污者被称为有瑕。或者,由清净而无过,由无过而无垢,由无有外尘而无垢故无瑕。因为有垢即是有瑕。或者,由无垢故不造作罪恶,因此无瑕。因为造作罪恶,由引起伤害故,被称为瑕。\end{enumerate}

\subsection\*{\textbf{482} {\footnotesize 〔PTS 477〕}}

\textbf{「他不随观自身为我,等持、正直、坚定,\\}
\textbf{「他确实无动摇、无荒秽、无疑惑,如来应得祭饼。}

Yo attano attānaṃ nānupassati, samāhito ujjugato ṭhitatto;\\
sa ve anejo akhilo akaṅkho, Tathāgato arahati pūraḷāsaṃ. %\hfill\textcolor{gray}{\footnotesize 25}

\begin{enumerate}\item \textbf{不随观自身为我},即以智相应的心对自身的诸蕴作毗婆舍那,不见另有名为我者,唯见蕴为量。因无有从真实、坚牢生起的见「我唯以自身觉知我」,故不随观自身为我,而是以慧见诸蕴。以道定\textbf{等持},无有身邪曲等为\textbf{正直},不为世间法所动摇为\textbf{坚定},以无有被称为渴爱的动摇、五种心的荒秽\footnote{五种心的荒秽,见\textbf{有财者经}第 19 颂的注。}、八事的疑惑\footnote{八事的疑惑,见\textbf{宝经}第 233 颂的注。},故\textbf{无动摇、无荒秽、无疑惑}。\end{enumerate}

\subsection\*{\textbf{483} {\footnotesize 〔PTS 478〕}}

\textbf{「他没有任何内在愚痴,以智见一切法,\\}
\textbf{「持最后身,且已证得无上吉祥的等觉,\\}
\textbf{「至此而成夜叉的清净,如来应得祭饼。」}

Mohantarā yassa na santi keci, sabbesu dhammesu ca ñāṇadassī;\\
sarīrañ ca antimaṃ dhāreti, patto ca sambodhim anuttaraṃ sivaṃ;\\
ettāvatā yakkhassa suddhi, Tathāgato arahati pūraḷāsaṃ”. %\hfill\textcolor{gray}{\footnotesize 26}

\begin{enumerate}\item \textbf{内在愚痴},即愚痴之因、愚痴之缘,即一切烦恼的同义语。\textbf{以智见一切法},即证得一切知智。因为此智通一切法,而世尊已得见此,以「我已得证」证得而住,因此说「以智见一切法」。\textbf{等觉},即阿罗汉性。\textbf{无上},即不与辟支佛、声闻共。\textbf{吉祥},即安稳、无祸害或祥瑞。\textbf{夜叉},即人。\textbf{清净},即洁净。因为于此,以无内在愚痴而无一切过失,因此断了轮回之因而持最后之身,以智见得生一切功德,因此得证无上等觉,从此更无别的应断与应证,故说「至此而成夜叉的清净」。\end{enumerate}

\subsection\*{\textbf{484} {\footnotesize 〔PTS 479〕}}

\textbf{「让我的献祭成为真实的献祭!当我得到这样的通达诸明者,\\}
\textbf{「因为梵天作证,请世尊接受我!请世尊享用我的祭饼!」}

“Hutañ ca mayhaṃ hutam atthu saccaṃ, yaṃ tādisaṃ vedagunaṃ alatthaṃ;\\
Brahmā hi sakkhi paṭigaṇhātu me Bhagavā, bhuñjatu me Bhagavā pūraḷāsaṃ”. %\hfill\textcolor{gray}{\footnotesize 27}

\begin{enumerate}\item 如是说已,婆罗门对世尊更加净喜,作净喜状,说了此颂。其义为:当我此前对梵天献了火供,我不知道我的献祭是真实还是虚妄,而现在,\textbf{让我的}这\textbf{献祭成为真实的献祭}!他请求着说「让它成为真实的献祭」。\textbf{当我得到这样的通达诸明者},因为站立于此,我得到您这样的通达诸明者。\textbf{因为梵天作证},因为你就是梵天现前,所以,\textbf{请世尊接受我}!且接受后,\textbf{请世尊享用我的祭饼}!他手授以祭品的残留说道。\end{enumerate}

\subsection\*{\textbf{485} {\footnotesize 〔PTS 480〕}}

\textbf{「我不应受用吟颂之物,对诸正观者,婆罗门!此即非法,\\}
\textbf{「诸佛拒绝吟颂之物,法既存在,婆罗门!此即行事之道。}

“Gāthābhigītaṃ me abhojaneyyaṃ, sampassataṃ brāhmaṇa n’esa dhammo;\\
gāthābhigītaṃ panudanti buddhā, dhamme satī brāhmaṇa vuttir esā. %\hfill\textcolor{gray}{\footnotesize 28}

\begin{enumerate}\item 于是,世尊以耕田婆罗豆婆遮经(第 81~82 颂)中所说的方法,说了两颂。\end{enumerate}

\subsection\*{\textbf{486} {\footnotesize 〔PTS 481〕}}

\textbf{「对整全者、大仙、漏尽者、恶作止息者,应以其它\\}
\textbf{「饮食给侍,因为他是希求福德者的良田。」}

Aññena ca kevalinaṃ mahesiṃ, khīṇāsavaṃ kukkuccavūpasantaṃ;\\
annena pānena upaṭṭhahassu, khettañ hi taṃ puññapekkhassa hoti”. %\hfill\textcolor{gray}{\footnotesize 29}

\subsection\*{\textbf{487} {\footnotesize 〔PTS 482〕}}

\textbf{「善哉!世尊!我应如是了知当享用如我等者的供品者,\\}
\textbf{「在献牲时寻求他,遵从你的教法。」}

“Sādhāhaṃ Bhagavā tathā vijaññaṃ, yo dakkhiṇaṃ bhuñjeyya mādisassa;\\
yaṃ yaññakāle pariyesamāno, pappuyya tava sāsanaṃ”. %\hfill\textcolor{gray}{\footnotesize 30}

\begin{enumerate}\item 随后,婆罗门想「他自己不想要,那所说的『对整全者、大仙、漏尽者、恶作止息者,应以饮食给侍』的其他人是谁呢」,如是未解颂义而欲知此,说了此颂。
\item 这里,\textbf{善哉},即请求之义的不变词。\textbf{如是},即以你说的方式。\textbf{他},即此应供者。\textbf{在献牲时寻求},文本省略了「我应给侍 \textit{upaṭṭhaheyyaṃ}」。\textbf{你的教法},即你的教诫。
\item 这即是说:善哉!世尊!我应按照你的教诫如是了知,请告知我这整全者,他当享用如我等者的供品,且我应在献牲时寻求、给侍他,请示予我这样的应供者,如果你不享用的话。\end{enumerate}

\subsection\*{\textbf{488} {\footnotesize 〔PTS 483〕}}

\textbf{「他已离于愤激,他的心不污浊,\\}
\textbf{「且已解脱爱欲,他已除去昏沉。}

“Sārambhā yassa vigatā, cittaṃ yassa anāvilaṃ;\\
vippamutto ca kāmehi, thinaṃ yassa panūditaṃ. %\hfill\textcolor{gray}{\footnotesize 31}

\begin{enumerate}\item 于是,世尊为以明了的方法显示这样的应供者,说了以下三颂。\end{enumerate}

\subsection\*{\textbf{489} {\footnotesize 〔PTS 484〕}}

\textbf{「界限的去除者,熟知生死者,\\}
\textbf{「具足寂默的牟尼,像这样前来献牲者,\footnote{此颂的四句都是下颂「敬礼、供养」的宾语。}}

Sīmantānaṃ vinetāraṃ, jātimaraṇakovidaṃ;\\
muniṃ moneyyasampannaṃ, tādisaṃ yaññam āgataṃ. %\hfill\textcolor{gray}{\footnotesize 32}

\begin{enumerate}\item 这里,\textbf{界限的去除者},界即边界、善人的行为,以其限、其终为其他部分,故烦恼被称为界限,即去除彼等之义。也有人说,界限是佛陀所能调伏的有学及凡夫,即彼等的调伏者\footnote{vinetā 兼有「去除者」与「调伏者」的意思,义注两释之。}。\textbf{熟知生死者},即于此善巧于「如是生、如是死」。\textbf{具足寂默},即具足慧,或具足身寂默等。\end{enumerate}

\subsection\*{\textbf{490} {\footnotesize 〔PTS 485〕}}

\textbf{「调伏了高傲,你应合掌敬礼,\\}
\textbf{「应供养饮食,供品如是成功。」}

Bhakuṭiṃ vinayitvāna, pañjalikā namassatha;\\
pūjetha annapānena, evaṃ ijjhanti dakkhiṇā”. %\hfill\textcolor{gray}{\footnotesize 33}

\begin{enumerate}\item \textbf{调伏了高傲},即调伏了某些恶觉者见到乞求者所现起的高傲,面露净喜之义。\end{enumerate}

\subsection\*{\textbf{491} {\footnotesize 〔PTS 486〕}}

\textbf{「佛陀您应得祭饼、无上的福田,\\}
\textbf{「一切世间的受献,对您的布施有大果报。」}

“Buddho bhavaṃ arahati pūraḷāsaṃ, puññakhettam anuttaraṃ;\\
āyāgo sabbalokassa, bhoto dinnaṃ mahapphalan” ti. %\hfill\textcolor{gray}{\footnotesize 34}

\begin{enumerate}\item 于是,婆罗门为赞叹世尊,说了此颂。这里,\textbf{受献},即应献,或者,以「从彼彼而来,应在此献祭」为受献,即是说作为所施的基础。
\item 此中其余及此前诸颂中所未解释者,由意义自明故,虽未解释亦能知晓,故不作解释。而此后则如耕田婆罗豆婆遮经中所述。\end{enumerate}

\textbf{于是,孙陀利迦婆罗豆婆遮婆罗门对世尊说:「希有!乔达摩君!希有!乔达摩君!好比,乔达摩君!能扶正被倾倒的,能揭示被遮蔽的,能给迷者指路,能在黑暗中持油灯,以使『具眼者能见色』,如是乔达摩君以种种方法阐明法。我皈依乔达摩君、法与比丘僧,愿我能在乔达摩君跟前出家,愿我能受具足!」}

Atha kho Sundarikabhāradvājo brāhmaṇo Bhagavantaṃ etad avoca: “abhikkantaṃ, bho Gotama, abhikkantaṃ, bho Gotama, seyyathāpi, bho Gotama, nikkujjitaṃ vā ukkujjeyya, paṭicchannaṃ vā vivareyya, mūḷhassa vā maggaṃ ācikkheyya, andhakāre vā telapajjotaṃ dhāreyya ‘cakkhumanto rūpāni dakkhantī’ ti, evam evaṃ bhotā Gotamena anekapariyāyena dhammo pakāsito. Esāhaṃ bhavantaṃ Gotamaṃ saraṇaṃ gacchāmi dhammañ ca bhikkhusaṅghañ ca, labheyyāhaṃ bhoto Gotamassa santike pabbajjaṃ, labheyyaṃ upasampadan” ti.

\textbf{于是,孙陀利迦婆罗豆婆遮婆罗门……成了众阿罗汉中的某个。}

Alattha kho Sundarikabhāradvājo brāhmaṇo…pe… arahataṃ ahosī ti.

\begin{center}\vspace{1em}孙陀利迦婆罗豆婆遮经第四\\Sundarikabhāradvājasuttaṃ catutthaṃ.\end{center}