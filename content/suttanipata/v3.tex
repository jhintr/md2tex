\chapter{大品第三}

\section{出家经}

我将宣扬出家:具眼者如何出家,\hfill\textcolor{gray}{\footnotesize \textbf{408}} \\
他如何经审视而选择了出家。


「这居家险迫,是尘垢之处,\hfill\textcolor{gray}{\footnotesize \textbf{409}} \\
「而出家闲旷」,见到如此,他便出了家。


出家后,他以身避免了恶业,\hfill\textcolor{gray}{\footnotesize \textbf{410}} \\
舍弃了语恶行,清净了活命。


佛陀到了王舍城,摩竭陀的山栏城,\hfill\textcolor{gray}{\footnotesize \textbf{411}} \\
去行乞食,显现出高贵相。


站在高楼上的频婆娑罗看到了他,\hfill\textcolor{gray}{\footnotesize \textbf{412}} \\
见到相的具足,便说了此义:


「诸君!请注意他!英俊、硕大、明净,\hfill\textcolor{gray}{\footnotesize \textbf{413}} \\
「具足行,眼见一寻之地,


「目光下视,具念,他不像来自卑贱的家族,\hfill\textcolor{gray}{\footnotesize \textbf{414}} \\
「让王使们速去:比丘将去何方?」


那些受遣的王使们便从后紧随:\hfill\textcolor{gray}{\footnotesize \textbf{415}} \\
「比丘将去何方?将住何处?」

次第行乞,守护根门,善加防护,\hfill\textcolor{gray}{\footnotesize \textbf{416}} \\
正知、忆念,他很快便装满了钵。


行乞之后,牟尼出了城,\hfill\textcolor{gray}{\footnotesize \textbf{417}} \\
去往般择婆山:他将住在此。


见到已进入住处,三个使者便就近坐下,\hfill\textcolor{gray}{\footnotesize \textbf{418}} \\
而其中一个返回,告知国王道:

「大王!这比丘在般择婆山的东坡,\hfill\textcolor{gray}{\footnotesize \textbf{419}} \\
「在山洞中,如老虎、公牛、狮子而坐。」


听了使者的话,刹帝利便以祥瑞的车乘\hfill\textcolor{gray}{\footnotesize \textbf{420}} \\
匆忙出发,前往般择婆山。


驶尽车道后,刹帝利从车上下来,\hfill\textcolor{gray}{\footnotesize \textbf{421}} \\
徒步前往,靠近了他便坐下。


落了坐,国王随即问候,\hfill\textcolor{gray}{\footnotesize \textbf{422}} \\
寒暄之后,他说了此义:

「你青春、年少,正是初出的青年,\hfill\textcolor{gray}{\footnotesize \textbf{423}} \\
「具足肤色、高大,像刹帝利出身。


「当闪耀在军队的前列,在象群之前,\hfill\textcolor{gray}{\footnotesize \textbf{424}} \\
「我赐财富,你当享用!当被问及,请告知出身!」


「前方的国土,国王!在雪山山坡,\hfill\textcolor{gray}{\footnotesize \textbf{425}} \\
「具足财产和勇气,在㤭萨罗世居。


「族名为太阳,生名为释迦,\hfill\textcolor{gray}{\footnotesize \textbf{426}} \\
「我从这家族出家,不愿求爱欲。


「见到爱欲的过患,视出离为安稳,\hfill\textcolor{gray}{\footnotesize \textbf{427}} \\
「我将向至上前行,于此我意喜乐。」


\section{至上经}

我自励于至上,朝着尼连禅河,\hfill\textcolor{gray}{\footnotesize \textbf{428}} \\
极努力而禅修,为证离轭安稳。


不解脱者前来,说着哀怜的话:\hfill\textcolor{gray}{\footnotesize \textbf{429}} \\
「你面黄肌瘦,死亡在你跟前。


「一千分去死,一分是你的活命,\hfill\textcolor{gray}{\footnotesize \textbf{430}} \\
「活着!先生!活命更好,活着你将造作福德。


「且你行梵行,并供奉火供,\hfill\textcolor{gray}{\footnotesize \textbf{431}} \\
「能积累广大的福德,为何追求至上?


「通往至上之路难行、难为、难以征服。」\hfill\textcolor{gray}{\footnotesize \textbf{432}} \\
魔罗说着这些偈颂,便站在佛陀跟前。


对这如是论说的魔罗,世尊说到:\hfill\textcolor{gray}{\footnotesize \textbf{433}} \\
「放逸的眷属!恶者!为这义利来到此处。


「对我而言,福德没有丝毫义利,\hfill\textcolor{gray}{\footnotesize \textbf{434}} \\
「魔罗应对那些福德对其有义利的人去说。


「有信,同样有精进,我还有慧,\hfill\textcolor{gray}{\footnotesize \textbf{435}} \\
「对如是自励的我,你为何追问活着?


「这风都能使河水之流干涸,\hfill\textcolor{gray}{\footnotesize \textbf{436}} \\
「为何不能枯竭自励之我的血液?


「血液枯竭时,胆汁和痰也枯竭,\hfill\textcolor{gray}{\footnotesize \textbf{437}} \\
「肌肉消尽时,心更加净喜,\\
「我的念、慧、定更加住立。


「如是而住的我,达到最剧烈的受,\hfill\textcolor{gray}{\footnotesize \textbf{438}} \\
「心不希求爱欲,请看有情的清净!


「爱欲是你的第一军,第二叫不乐,\hfill\textcolor{gray}{\footnotesize \textbf{439}} \\
「你的第三是饥渴,第四叫作渴爱,


「你的第五是昏沉睡眠,第六叫恐怖,\hfill\textcolor{gray}{\footnotesize \textbf{440}} \\
「你的第七是疑,覆藏、顽固是你的第八,


「利养、名闻、恭敬,与邪得的声誉,\hfill\textcolor{gray}{\footnotesize \textbf{441}} \\
「自赞,并且毁他,


「不解脱者!这些是你的军队,黑暗的攻击者,\hfill\textcolor{gray}{\footnotesize \textbf{442}} \\
「怯懦者无法战胜它,而战胜已则得快乐。


「若戴着文阇草,我的活命会成为羞耻!\hfill\textcolor{gray}{\footnotesize \textbf{443}} \\
「若被战胜而活,我死在战场更好。


「沉沦于此,许多沙门婆罗门未被看见,\hfill\textcolor{gray}{\footnotesize \textbf{444}} \\
「不知晓这善行者们所行之道。


「看到周围的军队,跟随驾乘的魔罗,\hfill\textcolor{gray}{\footnotesize \textbf{445}} \\
「我前去应战,他无法将我驱离此处!


「俱有天的世间不能征服你的军队,\hfill\textcolor{gray}{\footnotesize \textbf{446}} \\
「而我将以慧粉碎它们,如以石头粉碎生钵。


「控制了思惟,念也善住立,\hfill\textcolor{gray}{\footnotesize \textbf{447}} \\
「愿我从国游行至国,调伏众多弟子!


「他们不放逸、自励,我的教法的行者\hfill\textcolor{gray}{\footnotesize \textbf{448}} \\
「将不如你愿而行,所到之处即不忧伤。」


「七年间,我步步跟随着世尊,\hfill\textcolor{gray}{\footnotesize \textbf{449}} \\
「没等到具念的等正觉的陷落。


「脂肪色泽的石头,乌鸦围绕而飞:\hfill\textcolor{gray}{\footnotesize \textbf{450}} \\
「也许在此我们能发现嫩肉,也许会有美味。


「于彼未得美味,乌鸦便从此离开,\hfill\textcolor{gray}{\footnotesize \textbf{451}} \\
「如同乌鸦琢了石头,我们嫌厌了乔达摩而去。」


他被忧伤击溃,琵琶从腋间滑落,\hfill\textcolor{gray}{\footnotesize \textbf{452}} \\
那夜叉意志消沉,即于此处隐没。


\section{善说经}

如是我闻。一时世尊住舍卫国祇树给孤独园。于此,世尊告诸比丘:「诸比丘!」「大德!」这些比丘答世尊。


世尊说:「诸比丘!具足四支之语为善说,非恶说,无过,不为智者所呵责。哪四者?此处,诸比丘!比丘唯说善语而非恶语,唯说法而非非法,唯说可爱而非不可爱,唯说真实而非虚妄。诸比丘!具足这四支之语为善说,非恶说,无过,不为智者所呵责。」世尊说了这些。善逝说罢,大师进一步说:


「善人们说善语为最上,说法而非非法,这是第二,\hfill\textcolor{gray}{\footnotesize \textbf{453}} \\
「说可爱而非不可爱,这是第三,说真实而非虚妄,这是第四。」


于是,尊者婆耆舍从坐起,把衣偏覆一肩,向世尊合掌,对世尊说:「我明白了,世尊!我明白了,善逝!」「请说明白!婆耆舍!」世尊说。于是,尊者婆耆舍面对世尊,以合适的偈颂赞叹:


「应当只说这样的言语,不会因之折磨自己,\hfill\textcolor{gray}{\footnotesize \textbf{454}} \\
「也不会伤害到他人,这言语确是善说。


「应当只说可爱的言语,这言语受人欢迎,\hfill\textcolor{gray}{\footnotesize \textbf{455}} \\
「若所说的不给他人带去坏恶,便是可爱。


「真实确是甘露的言语,此乃永恒之法,\hfill\textcolor{gray}{\footnotesize \textbf{456}} \\
「善人们说,义利与法住立于真实。


「佛陀所说的安稳言语,为了证得涅槃,\hfill\textcolor{gray}{\footnotesize \textbf{457}} \\
「为尽苦的边际,它确是言语中的最上。」


\section{孙陀利迦婆罗豆婆遮经}

如是我闻。一时世尊住㤭萨罗孙陀利迦河的岸边。尔时,孙陀利迦婆罗豆婆遮婆罗门在孙陀利迦河的岸边献火供、事火祭。然后,孙陀利迦婆罗豆婆遮婆罗门献了火供、事了火祭,便从坐起,观察周围四方:「谁当享用这祭品的残留?」孙陀利迦婆罗豆婆遮婆罗门看到世尊在不远处的某棵树下蒙头而坐,看到后,左手拿了祭品的残留,右手拿了水壶,往世尊处走去。


于是,随着孙陀利迦婆罗豆婆遮婆罗门的脚步声,世尊便揭开了头。然后,孙陀利迦婆罗豆婆遮婆罗门想「这先生是光头,这先生是秃头」,便想从此回去。然后,孙陀利迦婆罗豆婆遮婆罗门便想:「于此,有些婆罗门也是光头,我何不前去问问出身?」然后,孙陀利迦婆罗豆婆遮婆罗门往世尊处走去,走到后,对世尊说:「先生是何出身?」


于是,世尊以偈颂对孙陀利迦婆罗豆婆遮婆罗门说:

「我既非婆罗门,亦非王子,不是吠舍或是任何其他,\hfill\textcolor{gray}{\footnotesize \textbf{458}} \\
「遍知了凡夫们的种姓,无所牵绊,我以考量在世间游行。


「穿著僧伽梨,我无家而行,剃去头发,内在寂静,\hfill\textcolor{gray}{\footnotesize \textbf{459}} \\
「于此不著于世人,婆罗门!你问我种姓的问题不合适。」


「先生!婆罗门与婆罗门一起,总是问:『您是婆罗门吗?』」 \hfill\textcolor{gray}{\footnotesize \textbf{460}}


「因为若你说是婆罗门,而说我非婆罗门,\hfill\textcolor{gray}{\footnotesize \textbf{461}} \\
「我就来问问这三句、二十四音节的颂诗。」


「以何依据,仙人、世人、刹帝利、婆罗门向诸天\hfill\textcolor{gray}{\footnotesize \textbf{462}} \\
「举行各种献牲,在此世间?」


「若到达边际者、通达诸明者在献牲时,能从中得到祭品,我说,他便成功。」 \hfill\textcolor{gray}{\footnotesize \textbf{463}} \\
「确实,这献祭成功,」婆罗门说,「当我们见了这样的通达诸明者,\hfill\textcolor{gray}{\footnotesize \textbf{464}} \\
「因为没有得见像你这样的人,其他人便享用了祭饼。」


「所以,婆罗门!你在此希求义利,上前来问!\hfill\textcolor{gray}{\footnotesize \textbf{465}} \\
「兴许于此,能发现寂静、无烟、无患、无待的善慧者。」


「我乐于献牲,乔达摩君!欲行献牲,却不知晓,\hfill\textcolor{gray}{\footnotesize \textbf{466}} \\
「请您教授我!何处献祭能成功?请对我说!」


「既然如此,你,婆罗门!请注意听!我将对你说法: \hfill\textcolor{gray}{\footnotesize \textbf{467}}


「莫问出身,当问行为,从薪实能生火,\\
「卑贱之家者也可成为坚定、高贵、以惭禁止的牟尼。


「以真实而调御,具足调御,通达诸明,梵行已立,\hfill\textcolor{gray}{\footnotesize \textbf{468}} \\
「希求福德的婆罗门若欲祭祀,应适时给予他祭品。


「舍弃了爱欲,无家而行,善加自制,如梭子般正直,\hfill\textcolor{gray}{\footnotesize \textbf{469}} \\
「希求福德的婆罗门若欲祭祀,应适时给予他们祭品。


「离于贪染,善等持诸根,如月亮解脱于罗睺的束缚,\hfill\textcolor{gray}{\footnotesize \textbf{470}} \\
「希求福德的婆罗门若欲祭祀,应适时给予他们祭品。


「无所羁绊地行于世间,始终具念,舍弃了执为我者,\hfill\textcolor{gray}{\footnotesize \textbf{471}} \\
「希求福德的婆罗门若欲祭祀,应适时给予他们祭品。


「舍弃了爱欲,征服而行,他知道生死的边际,\hfill\textcolor{gray}{\footnotesize \textbf{472}} \\
「已止息,如池水般清凉,如来应得祭饼。


「与相同者相同,与不同者差远,如来是无尽慧者,\hfill\textcolor{gray}{\footnotesize \textbf{473}} \\
「不染于此世或他世,如来应得祭饼。


「伪善、慢不住于他,他离贪,无我所,无待,\hfill\textcolor{gray}{\footnotesize \textbf{474}} \\
「去除忿怒,内在寂静,这婆罗门舍弃了忧尘,\\
「如来应得祭饼。


「舍弃了意的住处,他没有任何执取,\hfill\textcolor{gray}{\footnotesize \textbf{475}} \\
「无取于此世或他世,如来应得祭饼。


「等持,他度过暴流,以最高的见了知了法,\hfill\textcolor{gray}{\footnotesize \textbf{476}} \\
「漏尽,持最后身,如来应得祭饼。


「他的有漏与粗砺之语,已熏散、消尽而无存,\hfill\textcolor{gray}{\footnotesize \textbf{477}} \\
「他通达诸明,于一切处解脱,如来应得祭饼。


「超越执著,他已没有执著,在有慢的有情中,为无慢的有情,\hfill\textcolor{gray}{\footnotesize \textbf{478}} \\
「遍知了有田与物之苦,如来应得祭饼。


「不依希望,得见远离,超越他人所知的见,\hfill\textcolor{gray}{\footnotesize \textbf{479}} \\
「他没有任何所缘,如来应得祭饼。


「他所体认的上下诸法,已熏散、消尽而无存,\hfill\textcolor{gray}{\footnotesize \textbf{480}} \\
「寂静,于取的灭尽解脱,如来应得祭饼。


「得见结缚与生的尽头,他无余除去了贪路,\hfill\textcolor{gray}{\footnotesize \textbf{481}} \\
「清净、无过、无垢、无瑕,如来应得祭饼。


「他不随观自身为我,等持、正直、坚定,\hfill\textcolor{gray}{\footnotesize \textbf{482}} \\
「他确实无动摇、无荒秽、无疑惑,如来应得祭饼。


「他没有任何内在愚痴,以智见一切法,\hfill\textcolor{gray}{\footnotesize \textbf{483}} \\
「持最后身,且已证得无上吉祥的等觉,\\
「至此而成夜叉的清净,如来应得祭饼。」


「让我的献祭成为真实的献祭!当我得到这样的通达诸明者,\hfill\textcolor{gray}{\footnotesize \textbf{484}} \\
「因为梵天作证,请世尊接受我!请世尊享用我的祭饼!」


「我不应受用吟颂之物,对诸正观者,婆罗门!此即非法,\hfill\textcolor{gray}{\footnotesize \textbf{485}} \\
「诸佛拒绝吟颂之物,法既存在,婆罗门!此即行事之道。


「对整全者、大仙、漏尽者、恶作止息者,应以其它\hfill\textcolor{gray}{\footnotesize \textbf{486}} \\
「饮食给侍,因为他是希求福德者的良田。」

「善哉!世尊!我应如是了知当享用如我等者的供品者,\hfill\textcolor{gray}{\footnotesize \textbf{487}} \\
「在献牲时寻求他,遵从你的教法。」


「他已离于愤激,他的心不污浊,\hfill\textcolor{gray}{\footnotesize \textbf{488}} \\
「且已解脱爱欲,他已除去昏沉。


「界限的去除者,熟知生死者,\hfill\textcolor{gray}{\footnotesize \textbf{489}} \\
「具足寂默的牟尼,像这样前来献牲者,


「调伏了高傲,你应合掌敬礼,\hfill\textcolor{gray}{\footnotesize \textbf{490}} \\
「应供养饮食,供品如是成功。」


「佛陀您应得祭饼、无上的福田,\hfill\textcolor{gray}{\footnotesize \textbf{491}} \\
「一切世间的受献,对您的布施有大果报。」


于是,孙陀利迦婆罗豆婆遮婆罗门对世尊说:「希有!乔达摩君!希有!乔达摩君!好比,乔达摩君!能扶正被倾倒的,能揭示被遮蔽的,能给迷者指路,能在黑暗中持油灯,以使『具眼者能见色』,如是乔达摩君以种种方法阐明法。我皈依乔达摩君、法与比丘僧,愿我能在乔达摩君跟前出家,愿我能受具足!」

于是,孙陀利迦婆罗豆婆遮婆罗门……便成了众阿罗汉中的某个。

\section{摩伽经}

如是我闻。一时世尊住王舍城耆阇崛山。于是,摩伽学童往世尊处走去,走到后,问候了世尊,彼此寒暄已,坐在一边。坐在一边的摩伽学童对世尊说:


「乔达摩君!我是施者、施主、慷慨的应请者,我如法地寻求财富,如法地寻求财富后,把如法所获、如法所得的财富,施与一人,施与二、三、四、五、六、七、八、九、十人,施与二十、三十、四十、五十人,施与一百人,施与更多。乔达摩君!我如是布施、如是供奉,能带来许多福德吗?」


「确实,学童!你如是布施、如是供奉,能带来许多福德。学童!若施者、施主、慷慨的应请者,如法地寻求财富,如法地寻求财富后,把如法所获、如法所得的财富,施与一人……施与一百人,施与更多,他能带来许多福德。」


于是,摩伽学童以偈颂对世尊说:


「我问慷慨的乔达摩,」摩伽学童说,「身著袈裟,无家而行,\hfill\textcolor{gray}{\footnotesize \textbf{492}} \\
「若应请者、施主、在家人,希望福德、希求福德而作供奉,\\
「于此布施饮食给他人,供奉者的供品如何得到净化?」


「若应请者、施主、在家人,摩伽!」世尊说,「希望福德、希求福德而作供奉,\hfill\textcolor{gray}{\footnotesize \textbf{493}} \\
「于此布施饮食给他人,这样的人当藉由应供者成功。」


「若应请者、施主、在家人,」摩伽学童说,「希望福德、希求福德而作供奉,\hfill\textcolor{gray}{\footnotesize \textbf{494}} \\
「于此布施饮食给他人,世尊!请告知我应供者!」


「若无取著而行于世间,无所牵绊、整全、克己,\hfill\textcolor{gray}{\footnotesize \textbf{495}} \\
「希求福德的婆罗门若欲供奉,应适时给予他们供品。


「若一切结缚与束缚已断,调御、解脱、无患、无待,\hfill\textcolor{gray}{\footnotesize \textbf{496}} \\
「希求福德的婆罗门若欲供奉,应适时给予他们供品。


「若一切结缚已解脱,调御、解脱、无患、无待,\hfill\textcolor{gray}{\footnotesize \textbf{497}} \\
「希求福德的婆罗门若欲供奉,应适时给予他们供品。


「舍弃了贪、嗔、痴,漏尽、梵行已立,\hfill\textcolor{gray}{\footnotesize \textbf{498}} \\
「希求福德的婆罗门若欲供奉,应适时给予他们供品。


「伪善与慢不住于彼,漏尽、梵行已立,\hfill\textcolor{gray}{\footnotesize \textbf{499}} \\
「希求福德的婆罗门若欲供奉,应适时给予他们供品。

「若离贪、无我所、无待,漏尽、梵行已立,\hfill\textcolor{gray}{\footnotesize \textbf{500}} \\
「希求福德的婆罗门若欲供奉,应适时给予他们供品。

「他们确实不陷于渴爱,已度暴流,无我所而行,\hfill\textcolor{gray}{\footnotesize \textbf{501}} \\
「希求福德的婆罗门若欲供奉,应适时给予他们供品。

「他们对世上任何,对此世或他世的有与无有,没有渴爱,\hfill\textcolor{gray}{\footnotesize \textbf{502}} \\
「希求福德的婆罗门若欲供奉,应适时给予他们供品。


「舍弃了爱欲,无家而行,善加自制,如梭子般正直,\hfill\textcolor{gray}{\footnotesize \textbf{503}} \\
「希求福德的婆罗门若欲供奉,应适时给予他们供品。

「离于贪染,善等持诸根,如月亮解脱于罗睺的束缚,\hfill\textcolor{gray}{\footnotesize \textbf{504}} \\
「希求福德的婆罗门若欲供奉,应适时给予他们供品。

「平静,离于贪染而无忿恨,于此舍弃已,他们没有趣向,\hfill\textcolor{gray}{\footnotesize \textbf{505}} \\
「希求福德的婆罗门若欲供奉,应适时给予他们供品。


「无余舍弃了生死,超越一切疑惑,\hfill\textcolor{gray}{\footnotesize \textbf{506}} \\
「希求福德的婆罗门若欲供奉,应适时给予他们供品。

「若以自作洲,行于世间,无所牵绊,于一切处解脱,\hfill\textcolor{gray}{\footnotesize \textbf{507}} \\
「希求福德的婆罗门若欲供奉,应适时给予他们供品。


「若于此,如其所是地了知,『这是最后,而无再有』,\hfill\textcolor{gray}{\footnotesize \textbf{508}} \\
「希求福德的婆罗门若欲供奉,应适时给予他们供品。


「若通达诸明,乐于禅那,具念,得证等觉,众所皈依,\hfill\textcolor{gray}{\footnotesize \textbf{509}} \\
「希求福德的婆罗门若欲供奉,应适时给予他供品。」


「确实,我的问题并非徒劳,」摩伽学童说,「世尊告知了我应供者,\hfill\textcolor{gray}{\footnotesize \textbf{510}} \\
「于此,你如其所是地了知,因为这法已如是为你所知。


「若应请者、施主、在家人,希望福德、希求福德而作供奉,\hfill\textcolor{gray}{\footnotesize \textbf{511}} \\
「于此布施饮食给他人,世尊!请告知我供奉的成就。」


「你若作供奉,当供奉时,摩伽!」世尊说,「于一切处,应使心净喜,\hfill\textcolor{gray}{\footnotesize \textbf{512}} \\
「供奉者的供奉是所缘,住立于此后,舍弃过失。


「他离于贪染,去除嗔恨,培育着无量的慈心,\hfill\textcolor{gray}{\footnotesize \textbf{513}} \\
「昼夜持续而不放逸,向一切方向无量地遍满。」


「谁被净化、被解脱、被束缚?以何缘由去往梵界?\hfill\textcolor{gray}{\footnotesize \textbf{514}} \\
「牟尼!既然问到,请对无知的我说!因为就我今天所见,世尊就是梵天作证,\\
「因为你对我们,真的等同于梵天,如何投生梵界?放光者!」


「若以三种供奉的成就而作供奉,摩伽!」世尊说,「这样的人当藉由应供者成功,\hfill\textcolor{gray}{\footnotesize \textbf{515}} \\
「如是正当地供奉已,应请者投生到梵界,我说。」


如是说已,摩伽学童对世尊说:「希有!乔达摩君!……从今起,尽寿命,请乔达摩君受持我皈依为优婆塞!」

\section{会堂经}

如是我闻。一时世尊住王舍城竹林喂松鼠处。尔时,先前曾是血亲的天人向游行者会堂提出问题:「会堂!若沙门、婆罗门能向你解释这些问题,你应在他跟前行梵行。」


于是,游行者会堂在那天人跟前受持了这些问题,前往那些沙门婆罗门处,有僧团、有徒众、为徒众老师、知名、有名闻、为创教者、众所敬仰,例如富兰那·迦叶、末迦梨·瞿舍利子、阿耆多·翅舍钦婆罗、迦罗拘陀·迦栴延、先阇那·毗罗胝子、尼揵陀·若提子,走到后,向他们提出这些问题。他们不能解答游行者会堂提出的问题,不能解答而显出怨恨、嗔恨、不满,甚至还反问游行者会堂。


于是,游行者会堂想:「那些尊敬的沙门婆罗门,有僧团、有徒众、为徒众老师、知名、有名闻、为创教者、众所敬仰,例如富兰那·迦叶……尼揵陀·若提子,他们不能解答我提出的问题,不能解答而显出怨恨、嗔恨、不满,甚至还于此反问我。我何不转回低处,去享受爱欲?」


于是,游行者会堂想:「这沙门乔达摩也有僧团、有徒众、为徒众老师、知名、有名闻、为创教者、众所敬仰,我何不前往沙门乔达摩处,去问这些问题?」

于是,游行者会堂想:「那些尊敬的沙门婆罗门衰老、年迈、高龄、迟暮、岁月已逝、上座、久住、出家已久,有僧团、有徒众、为徒众老师、知名、有名闻、为创教者、众所敬仰,例如富兰那·迦叶……尼揵陀·若提子,他们尚且不能解答我提出的问题,不能解答而显出怨恨、嗔恨、不满,甚至还于此反问我,沙门乔达摩将如何解答我提出的这些问题?毕竟沙门乔达摩年纪尚轻,且新近出家。」

于是,游行者会堂想:「然而,『年轻的』沙门不应被轻视、不应被轻蔑。即便年轻,沙门乔达摩也可能有大神变、大威力,我何不前往沙门乔达摩处问这些问题呢?」


于是,游行者会堂往王舍城出发游行。渐次游行到王舍城竹林喂松鼠处,往世尊处走去,走到后,问候了世尊,彼此寒暄已,坐在一边。坐在一边游行者会堂以偈颂对世尊说:

「我带着困惑与疑问而来,」会堂说,「期待着问些问题,\hfill\textcolor{gray}{\footnotesize \textbf{516}} \\
「请了结它们!我提出的问题,请逐步、随法地向我解释!」


「你远道而来,会堂!」世尊说,「期待着问些问题,\hfill\textcolor{gray}{\footnotesize \textbf{517}} \\
「我会了结它们,你提出的问题,我会逐步、随法地向你解释。


「请问我问题!会堂!任何你心中希望的,\hfill\textcolor{gray}{\footnotesize \textbf{518}} \\
「对各个问题,我都会为你了结!」


于是,游行者会堂想:「这实在不思议!这实在未曾有!我在其他沙门婆罗门处,甚至都得不到许可,而沙门乔达摩却给了我这许可。」便心满意足、愉悦、踊跃、生起喜悦,问世尊问题:


「证得什么,人们称之为比丘?」会堂说,「因何而温顺?且如何人们称之为调御?\hfill\textcolor{gray}{\footnotesize \textbf{519}} \\
「如何被称为觉悟?我的所问,世尊!请解释!」


「以自己制造的道路,会堂!」世尊说,「得至般涅槃,已度疑惑,\hfill\textcolor{gray}{\footnotesize \textbf{520}} \\
「舍弃了离有与有,已立、灭尽再有,他即比丘。


「于一切处舍,具念,他在一切世间不伤害任何,\hfill\textcolor{gray}{\footnotesize \textbf{521}} \\
「已度的沙门不污浊,若他没有增盛,他为温顺。


「若其诸根已修,于内在及外在一切世间,\hfill\textcolor{gray}{\footnotesize \textbf{522}} \\
「突破了此世他世,等待时间,已修者为调御。


「省思了全部思惟、轮回、亡殁与投生两者,\hfill\textcolor{gray}{\footnotesize \textbf{523}} \\
「离去尘垢、无秽、清净,证得生的灭尽者,他们称其为觉悟。」


于是,游行者会堂欢喜、随喜于世尊之所说,便心满意足、愉悦、踊跃、生起喜悦,进一步问世尊问题:

「证得什么,人们称之为婆罗门?」会堂说,「因何而为沙门?且如何为沐浴者?\hfill\textcolor{gray}{\footnotesize \textbf{524}} \\
「如何被称为龙象?我的所问,世尊!请解释!」

「排除了一切恶,会堂!」世尊说,「无垢,善等持,坚定,\hfill\textcolor{gray}{\footnotesize \textbf{525}} \\
「已超轮回,整全,他无所依,被称作如如,他即是梵。


「平静,舍弃了福与恶,离尘,了知了此世他世,\hfill\textcolor{gray}{\footnotesize \textbf{526}} \\
「已越过生死,如如者被如实称作沙门。


「清洗了于内在及外在一切世间的一切恶,\hfill\textcolor{gray}{\footnotesize \textbf{527}} \\
「于天人思惟之处,他不思惟,他们称其为沐浴者。


「在世间不造作任何罪过,舍离了一切结缚和束缚,\hfill\textcolor{gray}{\footnotesize \textbf{528}} \\
「于一切处不羁绊,解脱,如如者被如实称作龙象。」


于是,游行者会堂……进一步问世尊问题:

「诸佛说谁是田地的胜者?」会堂说,「因何而为善?且如何为智者?\hfill\textcolor{gray}{\footnotesize \textbf{529}} \\
「如何被称名牟尼?我的所问,世尊!请解释!」

「省思了全部田地,会堂!」世尊说,「天、人以及梵的田地,\hfill\textcolor{gray}{\footnotesize \textbf{530}} \\
「解脱于一切田地的根本束缚,如如者被如实称作田地的胜者。


「省思了全部贮藏,天、人以及梵的贮藏,\hfill\textcolor{gray}{\footnotesize \textbf{531}} \\
「解脱于一切贮藏的根本束缚,如如者被如实称作善。


「省思了内在与外在两种淡黄,净慧者\hfill\textcolor{gray}{\footnotesize \textbf{532}} \\
「已越过黑白,如如者被如实称作智者。


「了知了不善人与善人的法,于内在及外在一切世间,\hfill\textcolor{gray}{\footnotesize \textbf{533}} \\
「为人天所应供养,已超染著与罗网,他是牟尼。」


于是,游行者会堂……进一步问世尊问题:

「证得什么,人们称之为通达诸明?」会堂说,「因何而为彻知?且如何为具精进?\hfill\textcolor{gray}{\footnotesize \textbf{534}} \\
「什么名为高贵?我的所问,世尊!请解释!」

「省思了全部吠陀,会堂!」世尊说,「凡属于沙门婆罗门的,\hfill\textcolor{gray}{\footnotesize \textbf{535}} \\
「于一切受离于贪染,已超一切吠陀,他即通达诸明。


「省思了戏论与名色,内在及外在疾病的根本,\hfill\textcolor{gray}{\footnotesize \textbf{536}} \\
「解脱于一切疾病的根本束缚,如如者被如实称作彻知。


「于此戒离一切恶,已超地狱之苦,住于精进,\hfill\textcolor{gray}{\footnotesize \textbf{537}} \\
「他具精进、具精勤,如如者被如实称作智者。


「若其切断束缚,内在及外在染著的根本,\hfill\textcolor{gray}{\footnotesize \textbf{538}} \\
「解脱于一切染著的根本束缚,如如者被如实称作高贵。


于是,游行者会堂……进一步问世尊问题:

「证得什么,人们称之为闻解圣典?」会堂说,「因何而为圣者?且如何为具行者?\hfill\textcolor{gray}{\footnotesize \textbf{539}} \\
「什么名为游行者?我的所问,世尊!请解释!」

「听闻、证知了世间的一切法,会堂!」世尊说,「任何有过与无过,\hfill\textcolor{gray}{\footnotesize \textbf{540}} \\
「征服、无疑、解脱,于一切处无患,他们称其为闻解圣典。


「斩断了诸漏与执著,这解者不再进入胎室,\hfill\textcolor{gray}{\footnotesize \textbf{541}} \\
「除去了三种想与泥沼,他不思惟,他们称其为圣者。


「若于此在诸行中已达成就,善巧,始终知法,\hfill\textcolor{gray}{\footnotesize \textbf{542}} \\
「于一切处不羁绊,心已解脱,无有嗔恚,他是具行者。


「作为苦之异熟的业,无论上方、下方或四旁中间,\hfill\textcolor{gray}{\footnotesize \textbf{543}} \\
「遍知的行者驱除已,随后对伪善、慢、贪、忿怒,\\
「终结了名色,他们称其为已达成就的游行者。」


于是,游行者会堂欢喜、随喜于世尊之所说,心满意足、愉悦、踊跃、生起喜悦,便从坐起,把上衣偏覆一肩,向世尊合掌,面对世尊,以合适的偈颂赞叹:

「这六十三种依于沙门的论点,宏慧者!\hfill\textcolor{gray}{\footnotesize \textbf{544}} \\
「依于想的标记与想的异教,已经调伏,得度冥暗的暴流。


「你到达边际、到达苦的彼岸,你是阿罗汉、正等正觉,我想你是漏尽者,\hfill\textcolor{gray}{\footnotesize \textbf{545}} \\
「放光者、具觉者、广慧者,尽苦边者!你令我得度。


「当你了知了我的疑惑,你令我得度了疑,礼敬您!\hfill\textcolor{gray}{\footnotesize \textbf{546}} \\
「牟尼!于寂默之路已达成就者!无荒秽者!日种!你为温顺。


「我先前存在的疑惑,你已为我解释,具眼者!\hfill\textcolor{gray}{\footnotesize \textbf{547}} \\
「你确是牟尼、等正觉者,你已没有诸盖。

「而且你的一切苦恼都已破碎、清除,\hfill\textcolor{gray}{\footnotesize \textbf{548}} \\
「清凉,已达调御,具足坚毅,为真实而努力。


「龙象!你这龙象、大雄之所说,\hfill\textcolor{gray}{\footnotesize \textbf{549}} \\
「一切诸天都随喜,包括那罗陀与波婆多。


「礼敬您,高贵的人!礼敬您,最上的人!\hfill\textcolor{gray}{\footnotesize \textbf{550}} \\
「在俱有天的世间中,没有与你对等的人。


「你是佛陀,你是大师,你是征服魔罗者、牟尼,\hfill\textcolor{gray}{\footnotesize \textbf{551}} \\
「你已切断了随眠,已度,你令这人类得度。


「你超越了依持,你破碎了诸漏,\hfill\textcolor{gray}{\footnotesize \textbf{552}} \\
「你是狮子,无取著,舍弃了畏与怕。


「好比美妙的白莲,不著于水,\hfill\textcolor{gray}{\footnotesize \textbf{553}} \\
「如是你不著于福与恶这两者,\\
「英雄!请伸展双足!会堂礼拜大师。」


于是,游行者会堂以头顶礼世尊的双足,对世尊说:「希有!尊者!……我皈依世尊、法与比丘僧,尊者!愿我能在世尊跟前出家,愿我能受具足!」

「会堂!若先前为外道者,希求在此法律中出家、希求具足,他要别住四月,四个月后,由心已坚定的比丘们使其出家,使其受具足为比丘,然而于此,我也知道人的差别。」


「尊者!如果先前为外道者希求在此法律中出家、希求具足要别住四月,四个月后,由心已坚定的比丘们使其出家,使其受具足为比丘,我愿别住四年,四年后,请心已坚定的比丘们使我出家,使我受具足为比丘!」


游行者会堂在世尊跟前已得出家,已得具足……尊者会堂便成了众阿罗汉中的某个。

\section{施罗经}

如是我闻。一时世尊在鸯伽北水游行,与大比丘僧团千二百五十比丘俱,到了名为市场的鸯伽北水的镇子。


萦发者翅宁听说:「真的,先生!沙门乔达摩、释迦子、从释迦族出家者,在鸯伽北水游行,与大比丘僧团千二百五十比丘俱,已到达市场,关于这乔达摩君,有如是的善称:『彼世尊亦即是阿罗汉、正等正觉者、明行足、善逝、世间解、无上士、调御丈夫、天人师、佛、世尊』,他由自己证知、证得了这俱有天、魔、梵、沙门婆罗门、天人的人世间后而宣说,他开示初中后善、有义有文的法,阐明完全圆满、遍净的梵行,善哉!得见这样的阿罗汉。」


于是,萦发者翅宁往世尊处走去,走到后,问候了世尊,彼此寒暄已,坐在一边。世尊便对坐在一边的萦发者翅宁以如法的言说显示、唤起、鼓舞、欢喜。


于是,萦发者翅宁被世尊如法的言说所显示、唤起、鼓舞、欢喜,对世尊说:「请乔达摩君明日接受我的食物,与比丘僧团一起。」如是说已,世尊对萦发者翅宁说:「翅宁!比丘僧团很大,有千二百五十比丘,且你信乐于婆罗门。」

第二次,萦发者翅宁对世尊说:「乔达摩君!尽管比丘僧团很大,有千二百五十比丘,且我信乐于婆罗门,请乔达摩君明日接受我的食物,与比丘僧团一起。」第二次,世尊对萦发者翅宁说:「翅宁!比丘僧团很大,有千二百五十比丘,且你信乐于婆罗门。」

第三次,萦发者翅宁对世尊说:「乔达摩君!尽管比丘僧团很大,有千二百五十比丘,且我信乐于婆罗门,请乔达摩君明日接受我的食物,与比丘僧团一起。」世尊便以默然而接受。


于是,萦发者翅宁已知世尊接受,从坐起,往自己的草庵走去,走到后,向朋友、僚属、亲戚、血亲宣告:「请听我说!尊敬的朋友、僚属、亲戚、血亲!沙门乔达摩受我邀请,明日与比丘僧团一起来受食,你们能为我做点体力活吗?」「如是,先生!」萦发者翅宁的朋友、僚属、亲戚、血亲答复后,有些人挖灶,有些人劈柴,有些人洗盘子,有些人装水罐,有些人备坐具,而萦发者翅宁自己则搭棚幕。


尔时,婆罗门施罗在市场定居,他精通三吠陀,及其词汇、仪轨、音韵、词源并其传承为第五,通句读、晓文法,熟稔顺世论与大人相,并教授三百学童学习颂诗。


尔时,萦发者翅宁信乐于婆罗门施罗。于是,婆罗门施罗为三百学童所随从,徒步随行、游荡,往萦发者翅宁的草庵走去。婆罗门施罗看到在萦发者翅宁的草庵,有些人挖灶,有些人劈柴……有些人备坐具,而萦发者翅宁自己则搭棚幕,看到后,对萦发者翅宁说:「翅宁君是要娶亲,还是要嫁女?是举行大供养,还是明日邀请了摩竭陀王具军·频婆娑罗及其军队?」


「施罗君!我不是要娶亲或嫁女,也不是明日邀请了摩竭陀王具军·频婆娑罗及其军队,而是举行大供养。有沙门乔达摩、释迦子、从释迦族出家者,在鸯伽北水游行,与大比丘僧团千二百五十比丘俱,已到达市场,关于这乔达摩君……佛、世尊,他受我邀请,明日与比丘僧团一起来受食。」

「翅宁君!你是说『佛陀』?」\\
「施罗君!我是说『佛陀』。」\\
「翅宁君!你是说『佛陀』?」\\
「施罗君!我是说『佛陀』。」


于是,婆罗门施罗道:「在世间,甚至这声音也难得,此即是『佛陀』。而在我们的颂诗中流传有三十二大人相,若具足的大人唯有两种趣向,而非其它。如果他居家,则成转轮王,如法的法王,四方者,征服者,国土安泰,七宝具足,他有这七宝:即轮宝、象宝、马宝、摩尼宝、女宝、家主宝及首相宝为第七。他的子嗣过千,英勇飒爽,摧伏敌军。他不以棍杖、不以刀剑,而是以法征服这以大海为边界的土地已而安居。然而,如果他从家出家,则成阿罗汉、正等正觉者、世间的去蔽者。翅宁君!现今这乔达摩君、阿罗汉、正等正觉者住在哪里?」


如是说已,萦发者翅宁伸出右臂,对婆罗门施罗说:「施罗君!在那片青林际。」


于是,婆罗门施罗与三百学童往世尊处走去。婆罗门施罗便告学童们:「诸君!请轻声而往!一步一步落足,因为彼诸世尊难以接近,如狮子般独行,且当我与沙门乔达摩交谈时,诸君!不要打断我的话,诸君请等我把话说完!」


于是,婆罗门施罗便往世尊处走去,走到后,问候了世尊,彼此寒暄已,坐在一边。坐在一边的婆罗门施罗便在世尊身上寻找三十二大人相。婆罗门施罗在世尊身上看到了大部分三十二相,除了二处,于二处大人相疑惑、怀疑、未信解、未净信:即于阴马藏及广长舌。


于是,世尊便想:「这婆罗门施罗看到我大部分三十二大人相,除了二处,于二处大人相疑惑、怀疑、未信解、未净信:即于阴马藏及广长舌。」于是,世尊便作了如此的神变预作,好让婆罗门施罗看见世尊的阴马藏。然后,世尊伸出舌头,顺触、逆触两个耳孔,顺触、逆触两个鼻孔,乃至以舌覆盖整个前额。


于是,婆罗门施罗便想:「沙门乔达摩具足圆满的三十二大人相,非不圆满,但我不知道他是不是佛陀。我听闻年迈高龄的婆罗门、老师和老师的老师说过『若他们是阿罗汉、正等正觉者,当他们自身的功德受到赞美时,会显示自身』,我何不面对沙门乔达摩,以合适的偈颂赞叹?」于是,婆罗门施罗便面对世尊,以合适的偈颂赞叹道:

「身体圆满,辉光美妙,出生良好,乐于得见,\hfill\textcolor{gray}{\footnotesize \textbf{554}} \\
「你肤色金黄,世尊!你齿牙洁白,具有精进。


「出生良好之人所拥有的特相,\hfill\textcolor{gray}{\footnotesize \textbf{555}} \\
「一切大人相,都在你的身上。


「眼睛明净,脸庞圆满,高大、端正、具有光辉,\hfill\textcolor{gray}{\footnotesize \textbf{556}} \\
「在沙门僧团之中,如太阳般闪耀。


「善于得见的比丘,皮肤好似黄金,\hfill\textcolor{gray}{\footnotesize \textbf{557}} \\
「如是具最上肤色者,你为何现沙门相?

「你应当去作王、转轮者、车乘之主、\hfill\textcolor{gray}{\footnotesize \textbf{558}} \\
「四方者、征服者,作阎浮林的主宰。


「让刹帝利和有采地的国王,都成为你的附庸!\hfill\textcolor{gray}{\footnotesize \textbf{559}} \\
「王中之王,人中的因陀,请行王权!乔达摩!」


「我是王,施罗!」世尊说,「无上的法王,\hfill\textcolor{gray}{\footnotesize \textbf{560}} \\
「我以法转轮,这轮不能倒转。」


「你自称是等正觉、」婆罗门施罗说,「无上的法王,\hfill\textcolor{gray}{\footnotesize \textbf{561}} \\
「你作此说『我以法转轮』,乔达摩!

「谁是您的大将、追随大师的弟子?\hfill\textcolor{gray}{\footnotesize \textbf{562}} \\
「谁为你续转这转起的法轮?」


「我所转之轮,施罗!」世尊说,「无上的法轮,\hfill\textcolor{gray}{\footnotesize \textbf{563}} \\
「舍利弗,承嗣如来者,能够续转。


「应证知的已证知,应修习的已修习,\hfill\textcolor{gray}{\footnotesize \textbf{564}} \\
「应舍弃的我已舍弃,所以我是佛陀,婆罗门!


「你应调伏对我的疑惑,你应信解,婆罗门!\hfill\textcolor{gray}{\footnotesize \textbf{565}} \\
「见到诸等正觉总是难得。


「他们在世间的出现确实总是难得,\hfill\textcolor{gray}{\footnotesize \textbf{566}} \\
「婆罗门!我是等正觉,无上的疗箭者。


「已成为梵,超过同侪,摧伏魔军,\hfill\textcolor{gray}{\footnotesize \textbf{567}} \\
「臣服了一切敌人,我实欣喜,无畏惧处。」


「诸君!你们倾听!如具眼者所说,\hfill\textcolor{gray}{\footnotesize \textbf{568}} \\
「疗箭者、大雄如狮子在林中咆哮。


「已成为梵、超过同侪、摧伏魔军者,\hfill\textcolor{gray}{\footnotesize \textbf{569}} \\
「谁见了能不净喜,即便是黑色出生?


「若愿意的请跟随我,若不愿意的请走!\hfill\textcolor{gray}{\footnotesize \textbf{570}} \\
「我将在此出家,在胜慧者的跟前。」

「如是,若您喜好正等正觉的教法,\hfill\textcolor{gray}{\footnotesize \textbf{571}} \\
「我们也将出家,在胜慧者的跟前。」


「这三百婆罗门合掌祈请:\hfill\textcolor{gray}{\footnotesize \textbf{572}} \\
「我们欲行梵行,世尊!在您跟前。」


「梵行已经善说,施罗!」世尊说,「是自见、无时的,\hfill\textcolor{gray}{\footnotesize \textbf{573}} \\
「对不放逸的学人,在此出家并非徒劳。」


于是,婆罗门施罗与其随从在世尊跟前已得出家,已得具足。


于是,萦发者翅宁是夜过后,在自己的草庵中令人准备了精制的硬食、软食,教人告知世尊时到:「是时,乔达摩君!食物已备好。」于是,世尊晨朝著了下衣,持了衣钵,往萦发者翅宁的草庵走去,走到后,与比丘僧团一起,坐在备好的坐处。

于是,萦发者翅宁亲手以精制的硬食、软食满足、款待了佛陀为首的比丘僧团。于是,当世尊已足食、手放开钵,萦发者翅宁便另取了低坐,坐在一边。世尊便对坐在一边的萦发者翅宁以偈颂随喜道:


「献牲以火供为上首,歌咏以颂诗为上首,\hfill\textcolor{gray}{\footnotesize \textbf{574}} \\
「国王是人中上首,大海是众河上首。


「月亮是群星的上首,太阳是照耀的上首,\hfill\textcolor{gray}{\footnotesize \textbf{575}} \\
「对希求福德者,僧伽实是供奉者的上首。」


然后,世尊以这些偈颂对萦发者翅宁随喜已,便从坐起而离开。

于是,尊者施罗与其随从独一、远离、不放逸、热忱、自励而住,此后不久……尊者施罗与其随从便成了众阿罗汉中的某个。于是,尊者施罗与其随从往世尊处走去,走到后,把衣偏覆一肩,向世尊合掌,以偈颂对世尊说:

「自我们皈依,这是第八日,具眼者!\hfill\textcolor{gray}{\footnotesize \textbf{576}} \\
「经七夜,世尊!我们在你的教法中已得调御。


「你是佛陀,你是大师,你是征服魔罗者、牟尼,\hfill\textcolor{gray}{\footnotesize \textbf{577}} \\
「你已切断了随眠,已度,你令这人类得度。


「你超越了依持,你破碎了诸漏,\hfill\textcolor{gray}{\footnotesize \textbf{578}} \\
「你是狮子,无取著,舍弃了畏与怕。

「这三百比丘合掌而立,\hfill\textcolor{gray}{\footnotesize \textbf{579}} \\
「英雄!请伸展双足!让龙象们礼拜大师!」

\section{箭经}

没有标记,无法确知,此世有死者的生命\hfill\textcolor{gray}{\footnotesize \textbf{580}} \\
艰难、有限,且它与苦相伴。


不存在这方法,能使生者不死,\hfill\textcolor{gray}{\footnotesize \textbf{581}} \\
已老者也有一死,因为群生即如是之法。


如同成熟的果实,晨朝有掉落的危险,\hfill\textcolor{gray}{\footnotesize \textbf{582}} \\
如是,已生的有死者总有死亡的危险。


又好比陶匠所造的土器,\hfill\textcolor{gray}{\footnotesize \textbf{583}} \\
一切囿于破碎,如是即有死者的生命。


年幼与年长,愚人与智者,\hfill\textcolor{gray}{\footnotesize \textbf{584}} \\
一切受制于死亡,一切归趣于死亡。


对于那些被死亡征服、从此处至他世者,\hfill\textcolor{gray}{\footnotesize \textbf{585}} \\
父亲不能救护孩子,亲戚不能救护亲戚。


亲戚们只能看着,悲痛万分,看!\hfill\textcolor{gray}{\footnotesize \textbf{586}} \\
一个又一个有死者,如待宰的牛般被牵走。


如是,世间被老与死逼迫,\hfill\textcolor{gray}{\footnotesize \textbf{587}} \\
知晓了世间的进程,所以智者不再忧伤。


若你不知所来或所去的道路,\hfill\textcolor{gray}{\footnotesize \textbf{588}} \\
不见两者的边际,则徒然地悲伤。


假如悲伤能产生任何义利,\hfill\textcolor{gray}{\footnotesize \textbf{589}} \\
当痴人伤害自己,明眼人也会如此。


因为不是以涕泣、忧伤到达心的寂静,\hfill\textcolor{gray}{\footnotesize \textbf{590}} \\
他的苦会生起得更多,身体还会败坏。


消瘦、憔悴,自己伤害着自己,\hfill\textcolor{gray}{\footnotesize \textbf{591}} \\
亡者并不因此存续,徒然悲伤。


当人不舍弃忧伤,便经历更多苦,\hfill\textcolor{gray}{\footnotesize \textbf{592}} \\
当叹泣逝者时,他即受制于忧伤。


再看看其他的行者、随业而往之人!\hfill\textcolor{gray}{\footnotesize \textbf{593}} \\
来到死亡的势下,群生正于此颤栗。


因为不管他们如何想,随后总成别样,\hfill\textcolor{gray}{\footnotesize \textbf{594}} \\
背离就是这般,看看世间的进程!


而学童即便活了百岁,甚或更久,\hfill\textcolor{gray}{\footnotesize \textbf{595}} \\
仍会背离亲族,舍弃此世的生命。

所以听闻到阿罗汉,应调伏悲伤,\hfill\textcolor{gray}{\footnotesize \textbf{596}} \\
看到逝去的亡者:他不可能因我。


好比应该用水熄灭炽燃的棚屋,\hfill\textcolor{gray}{\footnotesize \textbf{597}} \\
如是坚定、有慧、有智的善巧之人\\
应迅速驱散生起的忧伤,如风之于木棉,


还有悲伤、渴望与自身的忧虑,\hfill\textcolor{gray}{\footnotesize \textbf{598}} \\
寻求着自己的快乐,他应拔出自己的箭。


拔出了箭,无所依,能到达心的寂静,\hfill\textcolor{gray}{\footnotesize \textbf{599}} \\
超越一切忧伤,而成无忧、寂灭。


\section{婆悉吒经}

如是我闻。一时世尊住伊车能伽罗的伊车能伽罗林。尔时,有众多非常有名的富裕婆罗门在伊车能伽罗定居,即婆罗门旃基、婆罗门多梨车、婆罗门莲卧、婆罗门生闻、婆罗门可教,及其他非常有名的富裕婆罗门。


于是,当学童婆悉吒、婆罗豆婆遮徒步随行、游荡时,便发起这闲谈:「先生!如何才是婆罗门?」


学童婆罗豆婆遮如是说:「先生!只要父母双方出身良好,乃至祖上七代腹胎纯正,不因出身论而受排斥、责难,如此,他就是婆罗门。」学童婆悉吒如是说:「先生!只要具戒、具足仪法,如此,他就是婆罗门。」学童婆罗豆婆遮不能说服学童婆悉吒,学童婆悉吒也不能说服学童婆罗豆婆遮。


于是,学童婆悉吒对学童婆罗豆婆遮说:「婆罗豆婆遮君!这沙门乔达摩、释迦子、从释迦族出家者,住在伊车能伽罗的伊车能伽罗林,关于这乔达摩君,有如是的善称:『彼……佛、世尊』。让我们去!婆罗豆婆遮君!我们去往沙门乔达摩处,去到后,我们问问沙门乔达摩此义。如沙门乔达摩对我们所解释的,我们便这样受持之。」「如是,先生!」学童婆罗豆婆遮答学童婆悉吒。

于是,学童婆悉吒、婆罗豆婆遮往世尊处走去,走到后,问候了世尊,彼此寒暄已,坐在一边。坐在一边的学童婆悉吒以偈颂对世尊说:

「我俩都被认为、且自认是三明者,\hfill\textcolor{gray}{\footnotesize \textbf{600}} \\
「我是莲卧的学童、他是多梨车的。


「凡众三明者所说的,我们于此整全,\hfill\textcolor{gray}{\footnotesize \textbf{601}} \\
「我们通句读、晓文法,在读诵上和老师一样。


「我们之间于出身论存在争论,乔达摩!\hfill\textcolor{gray}{\footnotesize \textbf{602}} \\
「婆罗豆婆遮说由出身就是婆罗门,\\
「而我说由业,如是当知!具眼者!


「我们双方都不能说服彼此,\hfill\textcolor{gray}{\footnotesize \textbf{603}} \\
「便前来问先生,以等正觉著名者。

「好比月亮已无亏缺,众人前往合掌,\hfill\textcolor{gray}{\footnotesize \textbf{604}} \\
「如是在世间,我们顶礼礼敬乔达摩。


「我们问乔达摩,在世间出现的眼目,\hfill\textcolor{gray}{\footnotesize \textbf{605}} \\
「由出身就是婆罗门,抑或由业而成,\\
「请告诉无知的我们,好让我们懂得婆罗门!」


「我将对你们解释,婆悉吒!」世尊说,「按照次第、如其所是地,\hfill\textcolor{gray}{\footnotesize \textbf{606}} \\
「生类的种类分别,因为种类各不相同。


「你们也知道草木,虽然它们并不自认,\hfill\textcolor{gray}{\footnotesize \textbf{607}} \\
「它们的特征与生俱来,因为种类各不相同。


「随后,昆虫、蚱蜢,乃至于蝼蚁,\hfill\textcolor{gray}{\footnotesize \textbf{608}} \\
「它们的特征与生俱来,因为种类各不相同。


「你们也知道四足者,小的和大的,\hfill\textcolor{gray}{\footnotesize \textbf{609}} \\
「它们的特征与生俱来,因为种类各不相同。


「你们也知道以腹为足、长背的蛇,\hfill\textcolor{gray}{\footnotesize \textbf{610}} \\
「它们的特征与生俱来,因为种类各不相同。


「随后,你们也知道水生的鱼,以水为行处,\hfill\textcolor{gray}{\footnotesize \textbf{611}} \\
「它们的特征与生俱来,因为种类各不相同。


「随后,你们也知道鸟,凭羽翼而行于空中,\hfill\textcolor{gray}{\footnotesize \textbf{612}} \\
「它们的特征与生俱来,因为种类各不相同。


「如在这些种类中,存在种种与生俱来的特征,\hfill\textcolor{gray}{\footnotesize \textbf{613}} \\
「如是在人类中,不存在种种与生俱来的特征。


「不是以发、以头、以耳、以眼,\hfill\textcolor{gray}{\footnotesize \textbf{614}} \\
「不是以口、以鼻、以唇或以眉,


「不是以颈、以肩、以腹、以背,\hfill\textcolor{gray}{\footnotesize \textbf{615}} \\
「不是以臀、以胸、以阴、以媾,

「不是以手、以足,或以指、以甲,\hfill\textcolor{gray}{\footnotesize \textbf{616}} \\
「不是以胫、以股,或以色、以声,\\
「并没有如其它种类中与生俱来的特征。


「这于人类,在个别的身体上不存在,\hfill\textcolor{gray}{\footnotesize \textbf{617}} \\
「而人类中的差别,以名称而被言说。


「在人类中,凡是依护牛而活,\hfill\textcolor{gray}{\footnotesize \textbf{618}} \\
「婆悉吒!如是当知他是耕者,不是婆罗门。


「在人类中,凡是以种种技艺而活,\hfill\textcolor{gray}{\footnotesize \textbf{619}} \\
「婆悉吒!如是当知他是匠人,不是婆罗门。


「在人类中,凡是依买卖而活,\hfill\textcolor{gray}{\footnotesize \textbf{620}} \\
「婆悉吒!如是当知他是商人,不是婆罗门。

「在人类中,凡是以侍奉他人而活,\hfill\textcolor{gray}{\footnotesize \textbf{621}} \\
「婆悉吒!如是当知他是仆人,不是婆罗门。

「在人类中,凡是依不与取而活,\hfill\textcolor{gray}{\footnotesize \textbf{622}} \\
「婆悉吒!如是当知他是贼人,不是婆罗门。

「在人类中,凡是依射艺而活,\hfill\textcolor{gray}{\footnotesize \textbf{623}} \\
「婆悉吒!如是当知是战士,不是婆罗门。


「在人类中,凡是以祭祀而活,\hfill\textcolor{gray}{\footnotesize \textbf{624}} \\
「婆悉吒!如是当知他是祭司,不是婆罗门。

「在人类中,凡是食禄村庄与王国,\hfill\textcolor{gray}{\footnotesize \textbf{625}} \\
「婆悉吒!如是当知他是国王,不是婆罗门。

「而我不说从胎所生、源于母亲者即是婆罗门,\hfill\textcolor{gray}{\footnotesize \textbf{626}} \\
「若他有所牵绊,只名为敬语者,\\
「无牵绊、无执取,我说他是婆罗门。


「切断了一切结缚,他不再恐惧,\hfill\textcolor{gray}{\footnotesize \textbf{627}} \\
「超越执著、离轭,我说他是婆罗门。


「切断了带、纽、缰与辔,\hfill\textcolor{gray}{\footnotesize \textbf{628}} \\
「已拔出闩,已觉悟,我说他是婆罗门。


「无嗔者忍受骂詈、殴打、捆缚,\hfill\textcolor{gray}{\footnotesize \textbf{629}} \\
「具忍力者,具强军者,我说他是婆罗门。


「无忿怒,具仪法,具戒,无增盛,\hfill\textcolor{gray}{\footnotesize \textbf{630}} \\
「调御,最后身,我说他是婆罗门。


「如莲叶上的水珠,如锥尖上的芥子,\hfill\textcolor{gray}{\footnotesize \textbf{631}} \\
「若不著于爱欲者,我说他是婆罗门。


「若唯于此了知自身之苦的灭尽,\hfill\textcolor{gray}{\footnotesize \textbf{632}} \\
「放下重担、离轭,我说他是婆罗门。


「深慧,有智,熟知道与非道,\hfill\textcolor{gray}{\footnotesize \textbf{633}} \\
「证得最上义利,我说他是婆罗门。


「不与在家人及出家人两者交际,\hfill\textcolor{gray}{\footnotesize \textbf{634}} \\
「无家而行,少欲,我说他是婆罗门。


「对弱的与强的生物放下了棍杖,\hfill\textcolor{gray}{\footnotesize \textbf{635}} \\
「若不杀、不教人杀,我说他是婆罗门。


「敌对者中无敌对者,持棍杖者中止息者,\hfill\textcolor{gray}{\footnotesize \textbf{636}} \\
「有执取者中无执取者,我说他是婆罗门。


「若其贪、嗔、慢、覆藏都已脱落,\hfill\textcolor{gray}{\footnotesize \textbf{637}} \\
「如芥子从锥尖一般,我说他是婆罗门。


「能说不粗俗、有内容、真实之语,\hfill\textcolor{gray}{\footnotesize \textbf{638}} \\
「不以之冒犯任何人,我说他是婆罗门。


「若于此,或长或短、或细或粗、净与不净,\hfill\textcolor{gray}{\footnotesize \textbf{639}} \\
「不取世间所不与者,我说他是婆罗门。


「于此世及他世,若其不存希望,\hfill\textcolor{gray}{\footnotesize \textbf{640}} \\
「离欲、离轭,我说他是婆罗门。


「若其不存执著,已知而无疑,\hfill\textcolor{gray}{\footnotesize \textbf{641}} \\
「证得不死之地,我说他是婆罗门。


「若于此超越福与恶两者的染著,\hfill\textcolor{gray}{\footnotesize \textbf{642}} \\
「无忧、离尘、清净,我说他是婆罗门。


「如月一般离垢,清净、明净而不污浊,\hfill\textcolor{gray}{\footnotesize \textbf{643}} \\
「灭尽了喜与有,我说他是婆罗门。


「若超越了这险路、难路、轮回与愚痴,\hfill\textcolor{gray}{\footnotesize \textbf{644}} \\
「已度,已到彼岸,禅修者不动、无疑,\\
「以无所取著而止息,我说他是婆罗门。


「若于此舍弃了爱欲,无家游行,\hfill\textcolor{gray}{\footnotesize \textbf{645}} \\
「灭尽了爱欲与有,我说他是婆罗门。


「若于此舍弃了渴爱,无家游行,\hfill\textcolor{gray}{\footnotesize \textbf{646}} \\
「灭尽了渴爱与有,我说他是婆罗门。


「舍弃了人类之轭,超越了天界之轭,\hfill\textcolor{gray}{\footnotesize \textbf{647}} \\
「离于一切轭,我说他是婆罗门。


「舍弃了乐与不乐,得成清凉,无有依持,\hfill\textcolor{gray}{\footnotesize \textbf{648}} \\
「征服一切世间的英雄,我说他是婆罗门。


「若完全知晓了有情的亡殁与投生,\hfill\textcolor{gray}{\footnotesize \textbf{649}} \\
「无著、善逝、觉悟,我说他是婆罗门。


「诸天、乾闼婆及人都不知其趣向,\hfill\textcolor{gray}{\footnotesize \textbf{650}} \\
「漏尽的阿罗汉,我说他是婆罗门。


「若其前、后、中间都没有牵绊,\hfill\textcolor{gray}{\footnotesize \textbf{651}} \\
「无牵绊、无执取,我说他是婆罗门。


「公牛,高贵者,英雄,大仙,战胜者,\hfill\textcolor{gray}{\footnotesize \textbf{652}} \\
「不动,沐浴者,已觉悟,我说他是婆罗门。


「若知晓先前的住处,得见天界与苦处,\hfill\textcolor{gray}{\footnotesize \textbf{653}} \\
「然后证得生的灭尽,我说他是婆罗门。


「这世间遍计的姓名、种姓只是名称,\hfill\textcolor{gray}{\footnotesize \textbf{654}} \\
「从共许产生,随处被遍计。


「无知者的成见已长时随眠,\hfill\textcolor{gray}{\footnotesize \textbf{655}} \\
「唯无知者说『由出身就是婆罗门』。


「不由出身而是婆罗门,不由出身而非婆罗门,\hfill\textcolor{gray}{\footnotesize \textbf{656}} \\
「由业而是婆罗门,由业而非婆罗门。


「由业而是耕者,由业而是匠人,\hfill\textcolor{gray}{\footnotesize \textbf{657}} \\
「由业而是商人,由业而是仆人。


「由业而是贼人,由业而是战士,\hfill\textcolor{gray}{\footnotesize \textbf{658}} \\
「由业而是祭司,由业而是国王。

「见缘起者、熟知业与异熟者、\hfill\textcolor{gray}{\footnotesize \textbf{659}} \\
「智者,如是如实地得见此业。


「世间由业转起,人类由业转起,\hfill\textcolor{gray}{\footnotesize \textbf{660}} \\
「有情系缚于业,如行进之车的车辖。


「以苦行、以梵行、以自制、以调御,\hfill\textcolor{gray}{\footnotesize \textbf{661}} \\
「以此而成婆罗门,此是最上婆罗门。


「具足三明,寂静,灭尽再有,\hfill\textcolor{gray}{\footnotesize \textbf{662}} \\
「婆悉吒!如是当知对有识者,是梵、帝释。」


如是说已,学童婆悉吒、婆罗豆婆遮对世尊说:「希有!乔达摩君!……从今起,尽寿命,请乔达摩君受持我们皈依为优婆塞!」

\section{瞿迦梨经}

如是我闻。一时世尊住舍卫国祇树给孤独园。于是,瞿迦梨比丘往世尊处走去,走到后,礼敬了世尊,坐在一边。坐在一边的瞿迦梨比丘对世尊说:「尊者!舍利弗、目犍连是恶欲者,受制于恶欲。」


如是说已,世尊对瞿迦梨比丘说:「莫要如是,瞿迦梨!莫要如是,瞿迦梨!应使心净喜于舍利弗、目犍连,瞿迦梨!舍利弗、目犍连是端严者。」第二次……第三次,瞿迦梨比丘对世尊说:「尊者!虽然对我来说,世尊可信、可靠,然而,舍利弗、目犍连实是恶欲者,受制于恶欲。」第三次,世尊对瞿迦梨比丘说:「莫要如是,瞿迦梨!莫要如是,瞿迦梨!应使心净喜于舍利弗、目犍连,瞿迦梨!舍利弗、目犍连是端严者。」


于是,瞿迦梨比丘从坐起,礼敬了世尊,右绕而去。而离去后不久,瞿迦梨比丘全身就遍满了芥子大的疹子,长成芥子大后,变成了绿豆大,长成绿豆大后,变成了鹰嘴豆大,长成鹰嘴豆大后,变成了枣核大,长成枣核大后,变成了枣子大,长成枣子大后,变成了余甘子大,长成余甘子大后,变成了未熟的木橘大,长成未熟的木橘大后,变成了木橘大,长成木橘大后即破裂,流出脓血。于是,瞿迦梨比丘即由此病死去。且死去的瞿迦梨比丘于舍利弗、目犍连心怀嫌恨,投生到红莲花地狱。


于是,容貌殊胜的娑婆主梵天在深夜中照亮了整座祇园,往世尊处走去,走到后,礼敬了世尊,站在一边。然后,这位站在一边的娑婆主梵天对世尊说:「尊者!瞿迦梨比丘已死,而且,尊者!死去的瞿迦梨比丘于舍利弗、目犍连心怀嫌恨,已投生到红莲花地狱。」娑婆主梵天说了这些,说完这些,礼敬了世尊,右绕后,即于彼处隐没。


于是,世尊在是夜过后,告诸比丘:「诸比丘!是夜,娑婆主梵天在深夜中……诸比丘!娑婆主梵天说了这些,说完这些,右绕我后,即于彼处隐没。」如是说已,某比丘对世尊说:「尊者!红莲花地狱中的寿量有多长?」「长哉!比丘!红莲花地狱中的寿量,不能简单地用『若干年、若干百年、若干千年、若干百千年』来计算。」「尊者!那可以打个比方吗?」

「可以,比丘!」世尊说,「比丘!好比装在㤭萨罗大车上的二十石芝麻,随后,有人每过一百年取走一粒芝麻,比丘!以此方式,这装在㤭萨罗大车上的二十石芝麻会更快些趋于耗尽,而非一阿浮陀地狱。比丘!如是二十阿浮陀地狱为一尼罗浮陀地狱,二十尼罗浮陀地狱为一阿波波地狱,二十阿波波地狱为一阿吒吒地狱,二十阿吒吒地狱为一阿休休地狱,二十阿休休地狱为一白睡莲地狱,二十白睡莲地狱为一香睡莲地狱,二十香睡莲地狱为一青莲花地狱,二十青莲花地狱为一白莲花地狱,二十白莲花地狱为一红莲花地狱。比丘!而瞿迦梨比丘于舍利弗、目犍连心怀嫌恨,已投生到红莲花地狱。」世尊说了这些。善逝说罢,大师进一步说:


「对于已生之人,有斧生其口中,\hfill\textcolor{gray}{\footnotesize \textbf{663}} \\
「愚人说恶语时,以之砍伤自己。


「若赞美应受责备者,或责备应受赞美者,\hfill\textcolor{gray}{\footnotesize \textbf{664}} \\
「他便以口积累厄运,以此厄运,他不得快乐。


「若在投骰子时输财,这厄运是小事,\hfill\textcolor{gray}{\footnotesize \textbf{665}} \\
「哪怕是全部,哪怕连自己一起,\\
「若对善逝们生起恶意,这才是更大的厄运。


「十万又三十六尼罗浮陀,及五阿浮陀,\hfill\textcolor{gray}{\footnotesize \textbf{666}} \\
「谴责圣者之人以恶的语、意针对已,进入地狱。


「不实语者进入地狱,或做了却说『我没做』的也是,\hfill\textcolor{gray}{\footnotesize \textbf{667}} \\
「这两者死后都一样,在别处成为下劣之业的人。


「若冒犯无恶之人、清净无秽之人,\hfill\textcolor{gray}{\footnotesize \textbf{668}} \\
「恶便落回到这愚人,如逆风扬起细尘。


「若从事于各种贪,以言语指责他人,\hfill\textcolor{gray}{\footnotesize \textbf{669}} \\
「无信、贪婪、不慷慨、悭吝,从事于诽谤。


「险口者!不实者!非圣者!杀胎者!恶者!作恶作者!\hfill\textcolor{gray}{\footnotesize \textbf{670}} \\
「人边者!厄运者!贱生者!不要在此多说!你是堕地狱者。


「你反坌尘垢以致不利,你呵责善人,犯罪者!\hfill\textcolor{gray}{\footnotesize \textbf{671}} \\
「行了许多恶行,你会长时去向堕处。


「因为没有人的业会消失,它一来,主人就得到它,\hfill\textcolor{gray}{\footnotesize \textbf{672}} \\
「在他世,愚钝的犯罪者在自身中见到苦。


「他去到铁矛、带有刃口的铁枪击打之处,\hfill\textcolor{gray}{\footnotesize \textbf{673}} \\
「然后,食物是与之相当的炽热铁丸。


「当说时,不甜蜜地说,不热心,不作为救护而来,\hfill\textcolor{gray}{\footnotesize \textbf{674}} \\
「他们躺在炭火的卧具上,进入烈火燃烧之处。


「且被网罩住,在那里用铁锤击打,\hfill\textcolor{gray}{\footnotesize \textbf{675}} \\
「他们去到弥漫如大雾般的盲目的暗黑。


「然后,进入烈火燃烧的铜制的釜中,\hfill\textcolor{gray}{\footnotesize \textbf{676}} \\
「在其中长时地煎熬,在火堆中翻滚。


「然后,在脓血混杂中,犯罪者在那里被煎熬,\hfill\textcolor{gray}{\footnotesize \textbf{677}} \\
「举凡所到之处,在那里,触碰都是折磨。


「在虫聚的水中,犯罪者在那里被煎熬,\hfill\textcolor{gray}{\footnotesize \textbf{678}} \\
「连可去的堤岸都没有,因为整个锅全一样。


「他们进入锋利的剑叶林,肢体被完全割截,\hfill\textcolor{gray}{\footnotesize \textbf{679}} \\
「用钩子钩住舌头,反复切割后再击打。


「然后,他们去到带着刃口、带着刀锋的危险的灰河,\hfill\textcolor{gray}{\footnotesize \textbf{680}} \\
「愚钝的作恶者作了恶,在那里坠落。


「黑色与斑点的狗和渡鸦群撕咬着那里的悲泣者,\hfill\textcolor{gray}{\footnotesize \textbf{681}} \\
「还有贪婪的豺,而秃鹫和鸦则啄击着。


「于此,这犯罪之人所得的生活实在是艰难,\hfill\textcolor{gray}{\footnotesize \textbf{682}} \\
「所以于此,在余下的生命中人应尽义务,切莫放逸!


「陷入红莲花地狱者,已由知者们计算其芝麻量,\hfill\textcolor{gray}{\footnotesize \textbf{683}} \\
「计有五那由他俱胝,以及另外的一千二百俱胝。

「于此所说的地狱之苦,当在那里长久居住,\hfill\textcolor{gray}{\footnotesize \textbf{684}} \\
「所以,在清净、端严的善德者中,应常常守护语意。」

\section{那罗迦经}

阿私陀仙人在昼住处,见到庆喜、欢欣的三十三天众,\hfill\textcolor{gray}{\footnotesize \textbf{685}} \\
帝释因陀与衣服洁净的诸天持着白布,正热烈赞美着。


见到欢喜、踊跃的诸天后,表了敬意,他便在那里说道:\hfill\textcolor{gray}{\footnotesize \textbf{686}} \\
「众天人为何异常喜形于色?缘何持着白布而高兴?


「即便当时与阿修罗的战争,修罗战胜,阿修罗败北,\hfill\textcolor{gray}{\footnotesize \textbf{687}} \\
「那时也没有如这般身毛竖立,众神得见什么希有而喜悦?


「他们吹哨、歌唱、演奏,他们击掌、舞蹈,\hfill\textcolor{gray}{\footnotesize \textbf{688}} \\
「我问你们,弥卢顶的居民,请快除去我的疑虑,诸君!」


「这菩萨,无等的贵宝,为了人世间的利乐,出生\hfill\textcolor{gray}{\footnotesize \textbf{689}} \\
「在蓝毗尼地方的释氏村落,我们因此满足,异常喜形于色。


「他是一切有情之最上,至高之人,人中公牛,一切造物之最上,\hfill\textcolor{gray}{\footnotesize \textbf{690}} \\
「将在名为仙人的林中使轮转起,如同怒吼的狮子、有力的兽王。」


听闻此声,他急忙降下,然后进入净饭的居处,\hfill\textcolor{gray}{\footnotesize \textbf{691}} \\
在那里坐下后,对众释氏说:「童子在何处?我也欲见。」


随后,众释氏将孩子示与名为阿私陀者,童子如炽热的金子\hfill\textcolor{gray}{\footnotesize \textbf{692}} \\
在坩埚里善加锤炼一般,光辉闪耀,肤色胜妙。


见到童子如燃烧的火焰,如清净的众星之牛行于空中,\hfill\textcolor{gray}{\footnotesize \textbf{693}} \\
如照耀的太阳在秋日里破出云层,他庆喜,得广大喜。


众神在空中擎了多枝且千轮的伞盖,\hfill\textcolor{gray}{\footnotesize \textbf{694}} \\
金柄的拂尘上下翻飞,却不见握拂尘伞盖者。


名为黑色光辉的萦发仙人见到如黄毯上的金饰一般,\hfill\textcolor{gray}{\footnotesize \textbf{695}} \\
及被擎在头顶的白伞盖,便心生踊跃、悦意而接住。


而接住释氏的头牛后,通晓相与颂诗者审视着,\hfill\textcolor{gray}{\footnotesize \textbf{696}} \\
心生净喜,便高声唱言:「这是无上的二足尊。」


然后,随念着自己的趣向,他脸色凝重,落下泪水,\hfill\textcolor{gray}{\footnotesize \textbf{697}} \\
众释氏见后,对悲泣的仙人说:「童子是否有什么障难?」


仙人见后,对凝重的众释氏说:「我并非念及童子的不利,\hfill\textcolor{gray}{\footnotesize \textbf{698}} \\
「而且他也不会有障难,他非下劣之辈,请你们放心!


「这童子将得证至高的等觉,得见最上的清净,他将使法轮\hfill\textcolor{gray}{\footnotesize \textbf{699}} \\
「转起,他怜悯众人的利益,他的梵行将广为传播。


「而我的寿命在此所剩无多,在此期间我将死亡,\hfill\textcolor{gray}{\footnotesize \textbf{700}} \\
「我将听不到无比坚韧者的法,因此我压抑、沉沦、痛苦。」


让众释氏生了广大之喜,梵行者便从后宫出来,\hfill\textcolor{gray}{\footnotesize \textbf{701}} \\
为怜悯自己的外甥,便激励以无比坚韧者的法。


「当你从别人听到『佛陀』这声音,『证等觉者开显法之道』,\hfill\textcolor{gray}{\footnotesize \textbf{702}} \\
「你应去到那里,遍问教义,在彼世尊处行梵行。」


经他这样心怀利益、于未来见最上清净者的训诫,\hfill\textcolor{gray}{\footnotesize \textbf{703}} \\
那罗迦积集了大量福德,期待胜者而别住,守护根门。


听闻到胜者转动最胜之轮的声音,前往并见到了仙人中的牛王,既已净喜,\hfill\textcolor{gray}{\footnotesize \textbf{704}} \\
便问了高贵的牟尼最胜的寂默,当名为阿私陀者的教法来临时。


\begin{center}序颂终\end{center}

「已知这阿私陀的话语属实,\hfill\textcolor{gray}{\footnotesize \textbf{705}} \\
「我问问你,乔达摩!已度一切法者。


「对已步入无家、寻求行乞食者,\hfill\textcolor{gray}{\footnotesize \textbf{706}} \\
「牟尼!既然问到,请告诉我最上之法的寂默。」


「我将向你说明寂默,」世尊说,「难为、难以征服,\hfill\textcolor{gray}{\footnotesize \textbf{707}} \\
「噫!我将对你宣说,请约束自己!请你坚强!


「在村中受到谩骂、礼拜,应保持同分,\hfill\textcolor{gray}{\footnotesize \textbf{708}} \\
「应守护意的嗔恚,平静、不高举而行。


「在丛林中有种种出现,犹如火焰,\hfill\textcolor{gray}{\footnotesize \textbf{709}} \\
「女人们诱惑牟尼,莫要让她们诱惑你!


「戒离淫欲法,舍弃了各种爱欲,\hfill\textcolor{gray}{\footnotesize \textbf{710}} \\
「不对立、不迷恋,对弱或强的生类,


「以『他们如同我,我如同他们』,\hfill\textcolor{gray}{\footnotesize \textbf{711}} \\
「以自己作比方,不应杀,不应教人杀。


「舍弃了凡夫所执著的希望与贪,\hfill\textcolor{gray}{\footnotesize \textbf{712}} \\
「具眼者能修行,能超越这地狱。


「他应乏腹、节食、少欲、无贪,\hfill\textcolor{gray}{\footnotesize \textbf{713}} \\
「始终不饥于欲,则无欲、止息。


「他行乞后,应去到林边,\hfill\textcolor{gray}{\footnotesize \textbf{714}} \\
「在树下安顿,牟尼入坐。


「他从事禅那,坚定,应乐于林边,\hfill\textcolor{gray}{\footnotesize \textbf{715}} \\
「应在树下禅修,令自己满足。


「随后,当夜晚逝去,他应去到村边,\hfill\textcolor{gray}{\footnotesize \textbf{716}} \\
「不应喜于招请,及从村里来的供养。


「牟尼入村后,不应鲁莽地行于俗家,\hfill\textcolor{gray}{\footnotesize \textbf{717}} \\
「闭口不谈求食,不应说诱导的话语。


「『我得了任何便善哉,我未得也好』,\hfill\textcolor{gray}{\footnotesize \textbf{718}} \\
「他如如于这两者,即返回到树下。


「他以手持钵而行,不哑却被认为哑了,\hfill\textcolor{gray}{\footnotesize \textbf{719}} \\
「他不应轻蔑少施,不应鄙视施者。


「因为种种行道已由沙门阐明,\hfill\textcolor{gray}{\footnotesize \textbf{720}} \\
「他们不会两次去到彼岸,这也不会被觉知一次。


「若比丘已无爱著、已截断了水流、\hfill\textcolor{gray}{\footnotesize \textbf{721}} \\
「舍断了应作与不应作,则无热恼。


「我将向你说明寂默,应如剃刀的锋刃,\hfill\textcolor{gray}{\footnotesize \textbf{722}} \\
「用舌头抵住上颚,应自制于口腹。


「心不应沉滞,且亦不应多虑,\hfill\textcolor{gray}{\footnotesize \textbf{723}} \\
「离生腥、无所依,以梵行为归宿。


「应修学独坐,及沙门修行,\hfill\textcolor{gray}{\footnotesize \textbf{724}} \\
「独一被称为寂默,\\
「若你能乐于独一,则能照亮十方。


「听闻了智者、禅修者、舍弃爱欲者的宣言,\hfill\textcolor{gray}{\footnotesize \textbf{725}} \\
「随后,致力于我者应更加培育惭与信。


「你们以河流来了知,较之渠与沟,\hfill\textcolor{gray}{\footnotesize \textbf{726}} \\
「小渠喧嚣着前行,大水默然前行。


「欠缺者喧嚣,唯满盈者寂静,\hfill\textcolor{gray}{\footnotesize \textbf{727}} \\
「愚人譬如半桶,智者如满池。


「当沙门说许多具有并伴随义利(的话),\hfill\textcolor{gray}{\footnotesize \textbf{728}} \\
「他知晓着而开示法,他知晓着而说许多。


「若知晓着而自制,知晓着而不说许多,\hfill\textcolor{gray}{\footnotesize \textbf{729}} \\
「这牟尼应得寂默,这牟尼证得了寂默。」


\section{二重随观经}

如是我闻。一时世尊住舍卫国东园鹿母讲堂。尔时,世尊在十五布萨这天的满月夜晚,为比丘僧团围绕,坐于露地。


于是,世尊观察了默然又默然的比丘僧团后,告诸比丘:「诸比丘!若有人问『诸比丘!是何缘由听闻这些善的、圣的、出离的、趣向等觉的法』,应如是对他们说『只是为了对二重法的如实之智』。


「那什么是你们说的二重?『此是苦,此是苦之集』,这是一随观,『此是苦之灭,此是趣向苦灭之行道』,这是第二随观。诸比丘!对如是正确地随观二重的比丘,住于不放逸、热忱、自励,可期待二种果中的一果:在现法中已知,或当有余依时为阿那含。」世尊说了这些。善逝说罢,大师进一步说:


「若他们不知晓苦,以及苦的生起,\hfill\textcolor{gray}{\footnotesize \textbf{730}} \\
「与苦被完全无余地止息之处,\\
「且不知晓趣向苦的寂止之道,


「则他们缺乏心的解脱,以及慧的解脱,\hfill\textcolor{gray}{\footnotesize \textbf{731}} \\
「他们不能尽于边际,他们唯经历生老。


「若他们既知晓苦,以及苦的生起,\hfill\textcolor{gray}{\footnotesize \textbf{732}} \\
「与苦被完全无余地止息之处,\\
「并且知晓趣向苦的寂止之道,

「则他们具足心的解脱,以及慧的解脱,\hfill\textcolor{gray}{\footnotesize \textbf{733}} \\
「他们堪能尽于边际,他们不经历生老。」


「诸比丘!若有人问『还能以别的方法正确地随观二重吗』,应对他们说『有』。那是如何?『任何苦的生起,一切都由依持为缘』,这是一随观,『即由依持的无余离贪、灭,则无苦的生起』,这是第二随观。如是正确地……」大师进一步说:


「世间这些种种形相的苦,由依持为因而产生,\hfill\textcolor{gray}{\footnotesize \textbf{734}} \\
「若愚钝的无知者造作依持,则再再地经历苦,\\
「所以,知晓者、随观苦的生与源者不应造作依持。」


「诸比丘!若有人问『还能以别的方法正确地随观二重吗』,应对他们说『有』。那是如何?『任何苦的生起,一切都由无明为缘』,这是一随观,『即由无明的无余离贪、灭,则无苦的生起』,这是第二随观。如是正确地……」大师进一步说:


「若再再地进入生死的轮回,\hfill\textcolor{gray}{\footnotesize \textbf{735}} \\
「到此处与他处,这趣向唯由于无明。


「因为无明是大痴,这纠缠因之长久,\hfill\textcolor{gray}{\footnotesize \textbf{736}} \\
「但若有情具明,他们便不去往再有。」


「『还能以别的……那是如何?『任何苦的生起,一切都由行为缘』,这是一随观,『即由诸行的无余离贪、灭,则无苦的生起』,这是第二随观。如是正确地……」大师进一步说:


「任何苦的生起,一切都由行为缘,\hfill\textcolor{gray}{\footnotesize \textbf{737}} \\
「以诸行的灭,则无苦的生起。

「了知了这过患,『苦由行为缘』,\hfill\textcolor{gray}{\footnotesize \textbf{738}} \\
「由一切行的止息,由诸想的破灭,\\
「如是则苦灭尽。如实地了知此已,


「正见、通达诸明的智者们正知已,\hfill\textcolor{gray}{\footnotesize \textbf{739}} \\
「征服了魔罗的结缚,不去往再有。」


「『还能以别的……那是如何?『任何苦的生起,一切都由识为缘』,这是一随观,『即由识的无余离贪、灭,则无苦的生起』,这是第二随观。如是正确地……」大师进一步说:


「任何苦的生起,一切都由识为缘,\hfill\textcolor{gray}{\footnotesize \textbf{740}} \\
「以识的灭,则无苦的生起。

「了知了这过患,『苦由识为缘』,\hfill\textcolor{gray}{\footnotesize \textbf{741}} \\
「由识的寂止,比丘不饥而般涅槃。」


「『还能以别的……那是如何?『任何苦的生起,一切都由触为缘』,这是一随观,『即由触的无余离贪、灭,则无苦的生起』,这是第二随观。如是正确地……」大师进一步说:


「对那些被触击溃、随流于有流、\hfill\textcolor{gray}{\footnotesize \textbf{742}} \\
「于错路行道者,离结缚的灭尽甚远。


「若遍知了触,了知后而乐于寂止者,\hfill\textcolor{gray}{\footnotesize \textbf{743}} \\
「则由触的止息,不饥而般涅槃。」


「『还能以别的……那是如何?『任何苦的生起,一切都由受为缘』,这是一随观,『即由诸受的无余离贪、灭,则无苦的生起』,这是第二随观。如是正确地……」大师进一步说:


「若乐,若苦,与不苦不乐,\hfill\textcolor{gray}{\footnotesize \textbf{744}} \\
「内在及外在,凡所感受者,

「了知了『这是苦、虚妄败坏之法』,\hfill\textcolor{gray}{\footnotesize \textbf{745}} \\
「随触即见灭,他如是于此了知,\\
「比丘由诸受的灭尽,不饥而般涅槃。」


「『还能以别的……那是如何?『任何苦的生起,一切都由爱为缘』,这是一随观,『即由爱的无余离贪、灭,则无苦的生起』,这是第二随观。如是正确地……」大师进一步说:


「以爱为侣的人,轮回于漫长的旅途,\hfill\textcolor{gray}{\footnotesize \textbf{746}} \\
「到此处与他处,不得越过轮回。

「了知了这过患,『渴爱是苦的生起』,\hfill\textcolor{gray}{\footnotesize \textbf{747}} \\
「离爱、无执取,比丘应具念游行。」


「『还能以别的……那是如何?『任何苦的生起,一切都由取为缘』,这是一随观,『即由取的无余离贪、灭,则无苦的生起』,这是第二随观。如是正确地……」大师进一步说:


「由取为缘而有有,生物遭受痛苦,\hfill\textcolor{gray}{\footnotesize \textbf{748}} \\
「对于生者即有死,这是苦的生起。


「所以,由取的灭尽,智者们正知已,\hfill\textcolor{gray}{\footnotesize \textbf{749}} \\
「证知了生的灭尽,不去往再有。」


「『还能以别的……那是如何?『任何苦的生起,一切都由努力为缘』,这是一随观,『即由诸努力的无余离贪、灭,则无苦的生起』,这是第二随观。如是正确地……」大师进一步说:


「任何苦的生起,一切都由努力为缘,\hfill\textcolor{gray}{\footnotesize \textbf{750}} \\
「以诸努力的灭,则无苦的生起。

「了知了这过患,『苦由努力为缘』,\hfill\textcolor{gray}{\footnotesize \textbf{751}} \\
「舍遣了一切努力,对无努力中的解脱者、


「对切断了有爱、心已寂静的比丘,\hfill\textcolor{gray}{\footnotesize \textbf{752}} \\
「灭尽了生的轮回,他已没有再有。」

「『还能以别的……那是如何?『任何苦的生起,一切都由食为缘』,这是一随观,『即由诸食的无余离贪、灭,则无苦的生起』,这是第二随观。如是正确地……」大师进一步说:


「任何苦的生起,一切都由食为缘,\hfill\textcolor{gray}{\footnotesize \textbf{753}} \\
「以诸食的灭,则无苦的生起。

「了知了这过患,『苦由食为缘』,\hfill\textcolor{gray}{\footnotesize \textbf{754}} \\
「遍知了一切食,不依止一切食。

「正知了无病,由诸漏的遍尽,\hfill\textcolor{gray}{\footnotesize \textbf{755}} \\
「省思而受用,通达诸明的住法者不入诸数。」


「『还能以别的……那是如何?『任何苦的生起,一切都由动摇为缘』,这是一随观,『即由诸动摇的无余离贪、灭,则无苦的生起』,这是第二随观。如是正确地……」大师进一步说:


「任何苦的生起,一切都由动摇为缘,\hfill\textcolor{gray}{\footnotesize \textbf{756}} \\
「以诸动摇的灭,则无苦的生起。

「了知了这过患,『苦由动摇为缘』,\hfill\textcolor{gray}{\footnotesize \textbf{757}} \\
「所以,舍遣了动摇,破坏了诸行,\\
「不动、无取,比丘应具念游行。」


「『还能以别的……那是如何?『依止者有震动』,这是一随观,『无依止则不震动』,这是第二随观。如是正确地……」大师进一步说:


「无依止则不震动,依止则取著,\hfill\textcolor{gray}{\footnotesize \textbf{758}} \\
「到此处与他处,不得越过轮回。

「了知了这过患,『依止中有大恐怖』,\hfill\textcolor{gray}{\footnotesize \textbf{759}} \\
「无依止、无取,比丘应具念游行。」

「『还能以别的……那是如何?『诸比丘!无色较之色更加寂静』,这是一随观,『灭较之无色更加寂静』,这是第二随观。如是正确地……」大师进一步说:


「经历于色的有情,与无色处者,\hfill\textcolor{gray}{\footnotesize \textbf{760}} \\
「未能了知灭,往来于再有。

「若遍知了色,不安立于无色,\hfill\textcolor{gray}{\footnotesize \textbf{761}} \\
「在灭中解脱,他们是抛弃死亡者。」


「『还能以别的……那是如何?『诸比丘!对于俱有天、魔、梵、沙门婆罗门、天人的人世间被认为「此是真实」者,对于圣者,以如实的正慧善见「此是虚妄」』,这是一随观,『诸比丘!对于俱有天……的人世间被认为「此是虚妄」者,对于圣者,以如实的正慧善见「此是真实」』,这是第二随观。如是正确地……」大师进一步说:


「看俱有天的世间,在无我中思量我,\hfill\textcolor{gray}{\footnotesize \textbf{762}} \\
「在名色中执著,认为『此是真实』。


「无论他们如何认为,之后它总成别样,\hfill\textcolor{gray}{\footnotesize \textbf{763}} \\
「因为这对他是虚妄,短暂即虚妄之法。


「涅槃是非虚妄法,圣者们真实地知晓之,\hfill\textcolor{gray}{\footnotesize \textbf{764}} \\
「他们实由现观真实,不饥而般涅槃。」


「诸比丘!若有人问『还能以别的方法正确地随观二重吗』,应对他们说『有』。那是如何?『诸比丘!对于俱有天、魔、梵、沙门婆罗门、天人的人世间被认为「此是乐」者,对于圣者,以如实的正慧善见「此是苦」』,这是一随观,『诸比丘!对于俱有天……的人世间被认为「此是苦」者,对于圣者,以如实的正慧善见「此是乐」』,这是第二随观。诸比丘!对如是正确地随观二重的比丘,住于不放逸、热忱、自励,可期待二种果中的一果:在现法中已知,或当有余依时为阿那含。」世尊说了这些。善逝说罢,大师进一步说:


「全部的色、声、味、香、触与法,\hfill\textcolor{gray}{\footnotesize \textbf{765}} \\
「可意、可爱且适意,但凡被称为存在,


「对于俱有天的世间,这些被共许为乐,\hfill\textcolor{gray}{\footnotesize \textbf{766}} \\
「而在这些逝去之处,他们共许这是苦。


「有身的破灭被圣者们视为乐,\hfill\textcolor{gray}{\footnotesize \textbf{767}} \\
「对具见者们,这与一切世间相违。


「凡他人说是乐的,圣者们说是苦,\hfill\textcolor{gray}{\footnotesize \textbf{768}} \\
「凡他人说是苦的,圣者们知是乐。


「看!法即难以了知,无知者便于此迷惑,\hfill\textcolor{gray}{\footnotesize \textbf{769}} \\
「对被覆蔽者是冥暗,对不具见者是黑暗。


「但对善人们却是敞开,如同光对具见者,\hfill\textcolor{gray}{\footnotesize \textbf{770}} \\
「未熟习法的愚人们,在跟前也不能了知。


「被有贪击溃者,随流于有流者,\hfill\textcolor{gray}{\footnotesize \textbf{771}} \\
「陷落于魔境者,不易等觉这法。


「除了圣者,还有谁应能等觉这境界?\hfill\textcolor{gray}{\footnotesize \textbf{772}} \\
「正知了这境界,无漏者们便般涅槃。」


世尊说了这些。诸比丘心满意足,欢喜于世尊之所说。而当说此解答时,六十比丘的心以无取而从诸漏解脱。
