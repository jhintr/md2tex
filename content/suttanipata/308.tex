\section{箭经}

\subsection\*{\textbf{580} {\footnotesize 〔PTS 574〕}}

\textbf{没有标记,无法确知,此世有死者的生命\\}
\textbf{艰难、有限,且它与苦相伴。}

\begin{enumerate}\item 缘起为何?据说,世尊的护持,一个优婆塞,他的孩子死了。他被丧子之忧压垮,七日未曾进食。世尊为怜悯彼,便去到他家,为除其忧,说了此经。
\item 这里,\textbf{没有标记},即没有当作之行相的标记。好比在「当我低眼或是扬眉,以此标记,你就带走这物品」等中,存在当作之行相的标记,生命中却不如是。因为不可得「只要我做了这个或这个,你就会活,不会死」。\textbf{无法确知},正因此,不能确然得知「他可以活如许如许时间」,或以趣向中寿命的极限而知。因为好比四大王天等的寿命为固定,\textbf{有死者的}却不如是,如是亦不能确然得知。
\item \textbf{艰难},即由生活由仰赖众缘而困难,非易存活。如仰赖于入息、仰赖于出息、仰赖于大种、仰赖于段食、仰赖于热、仰赖于识。因为当无入息则不得活命,当无出息亦然。且于四界,如被「木口」等蛇牙之毒所啮,身体因地界的扰动而僵硬,如朽木一般,如说\footnote{关于「艰难」的解释中所引的「法集论注」的四颂,亦见于\textbf{清净道论}·说定品第 102 段,这里的译文小异。}:\begin{quoting}犹如给木口所啮,身成僵硬,\\地界扰动之身,亦如为木口所啮的那样。(法集论注第 584 段)\end{quoting}因水界的扰动而腐烂,由漏出脓、肉、血而余下骨与皮,如说:\begin{quoting}犹如给臭口所啮,身成腐烂,\\水界扰动之身,亦如为臭口所啮的那样。\end{quoting}因火界的扰动而如被掷入火坑一样,周身燃烧,如说:\begin{quoting}犹如给火口所啮,身成燃烧,\\火界扰动之身,亦如为火口所啮的那样。\end{quoting}因风界的扰动而关节肌腱被切断,如被大石所捣,骨头粉碎一般,如说:\begin{quoting}犹如给刀口所啮,身被切断,\\风界扰动之身,亦如为刀口所啮的那样。\end{quoting}故为界的扰动所扰乱的身体不得存活,而当此诸界彼此完成支撑等的作用,平等运作时,生命才得转起。如是,生命仰赖于大种。
\item 而在饥馑等时,因食物断绝,有情的生命显然灭尽,如是生命仰赖于段食。同样,当食饮等已消化,业生火灭尽时,有情显然至于命尽,如是生命仰赖于热。而在诸识灭时,从灭时开始,有情在世间显然不得活命,如是生命仰赖于识。如是,当知生活由仰赖众缘而艰难。
\item \textbf{有限},即短少,相较于诸天的生命,如同草尖上的露珠般,或者,以不超过一心刹那为有限。因为即便是寿命极长的有情,以过去心曾活,而非活着、将活,以未来(心)将活,而非活着、曾活,以现在(心)活着,而非曾活、将活。如说:\begin{quoting}生命与自体,以及全体苦乐,\\与一心相应,刹那飞快转起。\\那些住寿八万四千劫的风神,\\他们也不能与二心相伴而活。(大义释第 10 段)\end{quoting}
\item \textbf{且它与苦相伴},且这生命虽如是「没有标记、无法确知、艰难、有限」,尚与冷热、虻蚊等触、饥渴、行苦、坏苦、苦苦相伴。这说的是什么?因为有死者的生命既然如此,所以你只要尚未灭尽,就应增长这法行,莫再忧伤孩子!\end{enumerate}

\subsection\*{\textbf{581} {\footnotesize 〔PTS 575〕}}

\textbf{不存在这方法,能使生者不死,\\}
\textbf{已老者也有一死,因为群生即如是之法。}

\begin{enumerate}\item 于是,你也许会想「即便提供了一切资助来保护我的孩子,他还是死了,因此我忧伤」,即便如是,也莫忧伤!\textbf{不存在这方法,能使生者不死},即是说不能以任何方法保护并让已生的有情不死。随后,因为他想:「尊者!已老的人,死是适当的,我的孩子太年轻就死了。」因此说「\textbf{已老者也有一死,因为群生即如是之法}」,即是说已老者也罢,未老者也罢,都有一死,此中没有定则。\end{enumerate}

\subsection\*{\textbf{582} {\footnotesize 〔PTS 576〕}}

\textbf{如同成熟的果实,晨朝有掉落的危险,\\}
\textbf{如是,已生的有死者总有死亡的危险。}

\begin{enumerate}\item 现在,为以例证证明此义,说了此颂。其义为:好比\textbf{成熟的果实},因为从日出开始,当树被阳光照耀时,地味、水味在树中从叶到枝、从枝到干、从干到根,如是渐次从根进入地中,而从日落开始,从地到根、从根到干,如是渐次再上升到枝、叶、芽等,且如是上升时,若果实至于成熟,则不进入茎根。于是,当茎根被阳光照耀时,有热生起,因此,这些果实每早常时掉落,它们有\textbf{晨朝掉落的危险}\footnote{晨朝 \textit{pāto}:Norman 和菩提比丘都认为应从 PTS 本,作「总是 \textit{niccaṃ}」,说详其注,这里为结合义注,仍作「晨朝」。},即从掉落而生的危险之义。\textbf{如是,已生的有死者总有死亡的危险},因为有情如成熟的果实一般。\end{enumerate}

\subsection\*{\textbf{583} {\footnotesize 〔PTS 577〕}}

\textbf{又好比陶匠所造的土器,\\}
\textbf{一切囿于破碎,如是即有死者的生命。}

\begin{enumerate}\item 而且还有:\textbf{又好比陶匠……的生命}。\end{enumerate}

\subsection\*{\textbf{584} {\footnotesize 〔PTS 578〕}}

\textbf{年幼与年长,愚人与智者,\\}
\textbf{一切受制于死亡,一切归趣于死亡。}

\begin{enumerate}\item 所以,应如是把握:\textbf{年幼与……归趣于死亡}。\end{enumerate}

\subsection\*{\textbf{585} {\footnotesize 〔PTS 579〕}}

\textbf{对于那些被死亡征服、从此处至他世者\footnote{原文作「从他世 \textit{paralokato}」,Norman 说 paralokato 在语义上不通,故从锡兰本作「从此处至他世 \textit{paralok’ito}」,兹从其说。},\\}
\textbf{父亲不能救护孩子,亲戚不能救护亲戚。}

\begin{enumerate}\item 且如是把握已,亦应如是把握:\textbf{对于那些被死亡……亲戚不能救护亲戚}。\end{enumerate}

\subsection\*{\textbf{586} {\footnotesize 〔PTS 580〕}}

\textbf{亲戚们只能看着,悲痛万分,看!\\}
\textbf{一个又一个有死者,如待宰的牛般被牵走。}

\begin{enumerate}\item 且因为父亲不能救护孩子,亲戚不能救护亲戚,所以,亲戚们只能……被牵走。这里,其连结为:\textbf{亲戚们只能看着},以「母亲!亲爱的」等方法,以种种方式\textbf{万分悲痛},\textbf{一个又一个有死者}好比\textbf{待宰的牛般},\textbf{被}如是\textbf{牵走},如是,请\textbf{看}!优婆塞!世间就是这样毫无庇护。\end{enumerate}

\subsection\*{\textbf{587} {\footnotesize 〔PTS 581〕}}

\textbf{如是,世间被老与死逼迫,\\}
\textbf{知晓了世间的进程,所以智者不再忧伤。}

\begin{enumerate}\item 这里,因为佛、辟支佛等具足智慧者知晓「\textbf{如是,世间被老与死逼迫},无人能作救护」,\textbf{知晓了世间的进程,所以智者不再忧伤},即是说了知了世间的自性,不再忧伤。\end{enumerate}

\subsection\*{\textbf{588} {\footnotesize 〔PTS 582〕}}

\textbf{若你不知所来或所去的道路,\\}
\textbf{不见两者的边际,则徒然地悲伤。}

\begin{enumerate}\item 这说的是什么?\textbf{若你不知}来到母胎的\textbf{所来的道路},\textbf{或}从此殁后去往别处的\textbf{所去的}道路,\textbf{不见}这\textbf{两者的边际,则徒然地悲伤}。然而,智者得见两者,知晓了世间的进程,不再忧伤。\end{enumerate}

\subsection\*{\textbf{589} {\footnotesize 〔PTS 583〕}}

\textbf{假如悲伤能产生任何义利,\\}
\textbf{当痴人伤害自己,明眼人也会如此。}

\begin{enumerate}\item 现在,为证明在「徒然地悲伤」中所说的悲伤的徒然性,说了此颂。这里,\textbf{产生},即持有,在自身中生成之义。\textbf{当痴人伤害自己},即当痴人逼恼自己。\textbf{明眼人也会如此},若这样能产生任何义利,明眼人也会悲伤。\end{enumerate}

\subsection\*{\textbf{590} {\footnotesize 〔PTS 584〕}}

\textbf{因为不是以涕泣、忧伤到达心的寂静,\\}
\textbf{他的苦会生起得更多,身体还会败坏。}

\begin{enumerate}\item 此颂的连结为:然而,无人\textbf{以涕泣},或以\textbf{忧伤到达心的寂静},并且在涕泣、忧伤时,\textbf{他的苦会生起得更多,身体还会}因憔悴等\textbf{败坏}。\end{enumerate}

\subsection\*{\textbf{591} {\footnotesize 〔PTS 585〕}}

\textbf{消瘦、憔悴,自己伤害着自己,\\}
\textbf{亡者并不因此存续,徒然悲伤。}

\begin{enumerate}\item \textbf{亡者并不因此存续},即逝者并不因此悲伤得以存续、存活,这对他们没有助益,所以,\textbf{悲伤}为\textbf{徒然}。\end{enumerate}

\subsection\*{\textbf{592} {\footnotesize 〔PTS 586〕}}

\textbf{当人不舍弃忧伤,便经历更多苦,\\}
\textbf{当叹泣逝者时,他即受制于忧伤。}

\begin{enumerate}\item 且不仅徒然,甚至还带来非义。为什么?因为\textbf{当人不舍弃忧伤……即受制于忧伤}。\end{enumerate}

\subsection\*{\textbf{593} {\footnotesize 〔PTS 587〕}}

\textbf{再看看其他的行者、随业而往之人!\\}
\textbf{来到死亡的势下,群生正于此颤栗。}

\begin{enumerate}\item 如是显示了忧伤之毫无义利且带来非义,现在,为调伏忧伤,便作教诫,说了此颂。这里,\textbf{行者},即是说准备去往其他世间者。\textbf{群生正于此颤栗},即有情因畏惧死亡而于此颤栗。\end{enumerate}

\subsection\*{\textbf{594} {\footnotesize 〔PTS 588〕}}

\textbf{因为不管他们如何想,随后总成别样,\\}
\textbf{背离就是这般,看看世间的进程!}

\begin{enumerate}\item \textbf{不管他们}以何方式\textbf{想}「他将长寿、他将无病」,\textbf{随后}仍\textbf{总成别样},他即便所想如是,还是会死、会病。这\textbf{背离就是这般},与所想的相违而存在。优婆塞!\textbf{看看世间的}自性!如是当知此中的意趣与连结。\end{enumerate}

\subsection\*{\textbf{595} {\footnotesize 〔PTS 589〕}}

\textbf{而学童即便活了百岁,甚或更久,\\}
\textbf{仍会背离亲族,舍弃此世的生命。}

\subsection\*{\textbf{596} {\footnotesize 〔PTS 590〕}}

\textbf{所以听闻到阿罗汉,应调伏悲伤,\\}
\textbf{看到逝去的亡者:他不可能因我。}

\begin{enumerate}\item \textbf{听闻到阿罗汉},即听闻到阿罗汉开示的这样的法。\textbf{他不可能因我},即是说当了知这亡者\textbf{不可能}「现在\textbf{因我}再活过来」,便\textbf{应调伏悲伤}。\end{enumerate}

\subsection\*{\textbf{597} {\footnotesize 〔PTS 591〕}}

\textbf{好比应该用水熄灭炽燃的棚屋,\\}
\textbf{如是坚定、有慧、有智的善巧之人\\}
\textbf{应迅速驱散生起的忧伤,如风之于木棉,}

\begin{enumerate}\item 且更有「好比……木棉」。这里,当知\textbf{坚定}即具足坚毅,\textbf{有慧}即自性之慧,\textbf{有智}即博学之慧,\textbf{善巧}即生性反思,或者当与思所成、闻所成、修所成慧相结合。\end{enumerate}

\subsection\*{\textbf{598} {\footnotesize 〔PTS 592〕}}

\textbf{还有悲伤、渴望与自身的忧虑,\\}
\textbf{寻求着自己的快乐,他应拔出自己的箭。}

\begin{enumerate}\item 不仅仅是忧伤,「还有悲伤……自己的箭」。这里,\textbf{渴望},即渴爱。\textbf{忧虑},即心所之苦。\textbf{箭},即此三种品类,以难以取出之义及刺穿内部之义故为箭,或者即先前所说的七种贪等箭\footnote{见\textbf{施罗经}第 566 颂的注。}。\end{enumerate}

\subsection\*{\textbf{599} {\footnotesize 〔PTS 593〕}}

\textbf{拔出了箭,无所依,能到达心的寂静,\\}
\textbf{超越一切忧伤,而成无忧、寂灭。}

\begin{enumerate}\item 而当此箭被拔出时,「拔出了箭……寂灭」,即以阿罗汉为顶点完成了开示。这里,\textbf{无所依},即不依于爱、见。其余于此由先前已述故,其义自明,所以不再解释。\end{enumerate}

