\section{开篇辞}

\vspace{\fill}

\begin{quoting}礼拜了最上的应予礼拜的三宝,\\这在小部中,由舍弃了微细行、\end{quoting}

\begin{quoting}世间之依怙、寻求世间之出离者开示的\\经集,我将造其释义。\end{quoting}

\begin{quoting}且因这经集即含在小部中,\\所以,我将造其释义。\footnote{PTS 本无此颂。}\end{quoting}

\begin{quoting}数百偈颂汇聚,标记为应颂与记说,\\试问它为什么得称为「经集」?\end{quoting}

\begin{quoting}由善说、由生成、由善庇护义利、\\由指示及由摧毁,因而被称为经。\end{quoting}

\begin{quoting}这样的经从各处采择后\\被会集,所以它便得称如是。\end{quoting}

\begin{quoting}且所有的经都以如如者为量,\\是故,它们的言词便是此集。\end{quoting}

\begin{quoting}由没有其它特别称名的理由,\\如是,名称便定为「经集」。\end{quoting}

它既如是定名,从品上说,有蛇品、小品、大品、八颂品、彼岸道品等五品,其中蛇品为首。从经上说,蛇品中十二经,小品中十四,大品中十二,八颂品中十六,彼岸道品中十六,共七十经,其中蛇经为首。依圣典之量,则有八诵。\footnote{以上的\textbf{开篇词}为经集的义注「第一义光」的内容。}

\vspace{7em}