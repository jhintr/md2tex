\section{正游行经}

\begin{center}Sammāparibbājanīya Sutta\end{center}\vspace{1em}

\begin{enumerate}\item \textbf{正游行经},由在大集会之日讲论,也被称为\textbf{大集会经} \textit{Mahāsamayasutta}。缘起为何?缘起为问题的主导。因为被相佛提问,世尊便说了此经,它连同问题一起,被称为「正游行经」。这于此是略说,而古人则从释迦族和拘利族的缘起开始详说。
\item 于此,这即解说之道的注释:据说,劫初的国王摩诃三摩多有子名罗阇,罗阇之子为福罗阇,福罗阇之子为善好,善好之子为福善好,福善好之子为曼达多,曼达多之子为福曼达多,福曼达多之子为布萨,布萨之子为福,福之子为近福,近福之子为摩伽天,摩伽天的传承为八万四千刹帝利。他们之后,便为三个甘蔗王世系。其中第三个甘蔗王有五位王后:哈它、质多、延兜、阇利尼、毗舍佉,一一各有五百女子随从。最长的有四个儿子:坩埚、揉矢、调象与破城,五个女儿:可喜、善喜、欢喜、胜利与胜军。如是,她得了九个孩子,便去世了。
\item 然后,国王又娶了另一年轻美丽的王室之女,立为正妃。她也产下一子,名为延兜。在第五天,她妆扮了延兜王子,示与国王。国王满意,便赐惠与妃子。她与众亲戚合谋,为孩子请求王位。国王说「走开!贱人!你想给我的孩子们灾难」而未与。她反复在私下讨好国王,仍以「大王!妄语不妥」等说辞请求。然后,国王便告知众子:「亲爱的!我见到你们的幼弟延兜王子后,匆忙之下便赐惠与他的母亲,而她想要为孩子废立王位,我死之后,你们可以回来执政!」便遣散了(他们)与八位大臣。
\item 他们带上姊妹,便随四支军队离城。众多人民想到「王子们会在父亲身后返还执政,我们去护持他们」,便即追随。第一天,军队长一由旬,第二天二由旬,第三天三由旬。王子们便想:「我们部伍庞大,若进攻某位邻王,我们定能获得国土,他对我们也无能如何,但为何要因得到统治去压迫他人?阎浮提甚大,我们到林野去建城!」便朝雪山进发。
\item 于此,当他们寻找建城之处时,雪山中有名为迦毗罗的极端苦行者在莲池边的大柚木林中定居,他们便到了他的住处。他见到他们,经询问而听完了一切经过,便对他们生起怜悯。据说,他知晓名为地网的明咒,以之能见到上至八十肘高的天空、下至地下的功德与过失。于是,在一地,猪鹿会恐吓并攻击狮虎等,蛙鼠竟令群蛇畏惧,他见到这些,想「这地方乃地中最上」,便在此地建了草庵。随后,他对王子们说:「如果你们以我的名字建城,我就把此处给你们。」他们便如是许诺。苦行者说:「站在此处,旃陀罗之子都会较转轮王有力,你们在草庵上建了王宫,再去建城!」便给予此处,自己则在不远的山脚下建了草庵而住。随后,王子们在那里建了城,由在迦毗罗所住之处建造,便取名\textbf{迦毗罗卫}\footnote{迦毗罗卫 \textit{Kapilavatthu}:「卫」字为音译,即基地之义。},于彼营居。
\item 于是,大臣们想「这些孩子都已成年,要是他们的父亲在跟前,他会安排嫁娶,现在只好是我们的责任了」,便与王子们一起商量。王子们说:「我们不见与我们相当的刹帝利女子,也不见与那些姊妹相当的刹帝利王子,让我们不要混杂血统!」因畏惧血统的混杂,便把长姊置于母亲的地位,而与其余的婚配。他们的父亲听了经过,便发出慨叹:「真是能仁!这些孩子!真是最上能仁!这些孩子!」这先是\textbf{释迦族}的缘起\footnote{据菩提比丘注 1238,这里揭示了「释迦 \textit{Sakya}」的三重语源:一为「能仁 \textit{sakya}」,即有能力,一为「自己 \textit{saka}」,即与姊妹婚配,一为「柚木 \textit{sāka}」,即建城之地。}。世尊也曾说:\begin{quoting}于是,阿摩昼!甘蔗王便告近侍大臣:「先生!王子们现今止于何处?」「陛下!雪山山腹的莲池边有大柚木林,王子们现今于彼止息。他们因畏惧血统的混杂,与自己的姊妹婚配。」然后,阿摩昼!甘蔗王便发出慨叹:「真是能仁!这些孩子!真是最上能仁!这些孩子!」从此,阿摩昼!释迦族便确立,而他则是释迦族的先人。(长部·阿摩昼经第 267 段)\end{quoting}
\item 随后,他们的长姊得了麻风病,肢体如同黑檀花。王子们想「哪怕只是和她一起坐立饮食等,这病也会传染」,便送她上车,如去游园一般,进入林野后,教人挖了池子,把她和饮食之具丢在一起,在上方盖板覆土,便即离开。
\item 尔时,名为罗摩的国王得了麻风病,被侍女和舞女们嫌弃,因此悚惧,便把王位给了长子,进入林野,在那里以根叶果子为食,不久竟病愈而肤色金黄,四处巡行时,看见一棵大空心树,内有十六肘之量,便清扫了空处,安了门窗,扎好梯子,于彼营居。他在炭盆里起火,夜晚听着悲鸣与和鸣而卧。他观察到「某处狮子作吼、某处是老虎」,拂晓便去到那里,取了残肉煮了吃。
\item 于是,一天,他在黎明时分燃了火而坐。那时,老虎嗅到了那王族女子的气味,刨开地面,在铺盖板处破开洞口。她从洞口看到老虎,便恐惧地发出惨叫。他听到声音,观察到「这是女人的声音」,便清早到了那里,说:「谁在里面?」「女人,主人!」「出来!」「我不出来。」「什么原因?」「我是刹帝利女孩。」即便被埋在坑里,她仍保持尊严。他问了一切,便揭露身世说:「我也是刹帝利,来吧!现在活着就像被丢在奶里的酥。」她说:「我是麻风病人,不能出来。」他说:「我已病愈,能够治疗。」便递下梯子,拉她出来,带回自己的住处,给了自己用过的药,不久就令病愈而肤色金黄。他便与她婚配。她头一次同房就怀了孕,生下两个孩子,再次又生下两个,如是生了十六次。这样,他们就成了三十二个兄弟。当他们渐渐长大,父亲便教授了所有技艺。
\item 于是,一天,一个罗摩国王的城民来山中寻宝,到了那里,见到国王便即认出:「陛下!我认识你。」当被问到「你从哪里来」时,说:「从城里,陛下!」随后,国王便问了他所有经过。当他们如是交谈时,那些孩子回来,他便问:「他们是谁?陛下!」「我的孩子,我说!」「陛下!你现在在这林中为这三十二个孩子围绕,有何作为?来治理国家吧!」「止止!我说!此处乐。」他想「我现在有了谈资」,便去到城里,告知国王之子。国王之子想「我要带父亲回来」,便与四支军队去到那里,以种种方式请求父亲。他仍不愿意:「止止!亲爱的孩子!此处乐。」随后,王子想「国王现在不想回来,噫!那我就在此为他建城」,便砍了拘罗树造屋建城,由移除拘罗树所造及由在虎径上所造,在取了「拘罗城」及「虎迹」两个名字后,他便走了。
\item 随后,当孩子们成年,母亲命令到:「亲爱的!住在迦毗罗卫的释迦族是你们的舅氏,去娶他们的女儿!」他们在刹帝利女孩们在河中嬉戏的那天前往,堵住河津后,宣告了姓名,抢了各自愿求女子便走。释迦王族听说,想「算了!我说!也是我们的亲戚」,便默然。这是\textbf{拘利族}的缘起。
\item 如是,随着这些释迦族和拘利族互相嫁娶,世系便流传下来,直至狮颚王,当需详知。据说,狮颚王有五个孩子:净饭、无量饭、洗饭、释饭与白饭。当其中的净饭执政时,大人圆满了波罗蜜,如于「本生因缘 \textit{Jātakanidāna}」中所说之法,从兜率陀城殁后,在其夫人——国王眼药之女——摩诃摩耶夫人的胎内获取结生,渐次行了大出离,觉悟了正等正觉,转起无上法轮,顺次到了迦毗罗卫,令净饭大王等住于圣果后,出发在人间游行,且于后时又与一千五百比丘返回,住在迦毗罗卫的尼拘律园。
\item 且当世尊住于此时,释迦族与拘利族就水起了纷争。如何?据说,迦毗罗城与拘利城两者之间,有名为赤牛的河流经过。它时而水小,时而水大。当水小时,释迦族与拘利族都修了水渠,为各自的田地灌溉引水。一天,双方的人们在修渠时便互相争吵:「咄!你们的王族和姊妹婚配,像鸡、狗、豺狼等畜生一样。」「你们的王族住在空心树里,像毕舍遮一样。」如是以出身论咒骂已,便告知各自的国王。他们大怒,便准备战斗,在赤牛河边集结。军队峙立,如大海一般。
\item 于是,世尊想「亲戚们发生纷争,噫!我去遮止他们」,便从空中前往,站在两军之间——有些则说「经转向后,从舍卫国前往」——且如是站后,便说了\textbf{执杖经}\footnote{即\textbf{经集}·第四品第 15 经。}。他们听后,全都生起悚惧,丢了武器,礼敬世尊而立,并设好高价的坐处。世尊降落,在设好的坐处落坐,以\begin{quoting}手里执斧的人……(本生第 13:14 颂)\end{quoting}等讲了\textbf{颤栗本生},以\begin{quoting}我礼拜您,象!……(本生第 5:39 颂)\end{quoting}等讲了\textbf{鹌鹑本生},以\begin{quoting}互相问候,众鸟除去网而飞翔,\\而当它们争论时,便入我彀中。(本生第 1:33 颂)\end{quoting}等讲了\textbf{荷车本生},又为显明他们长时维系的亲戚关系,说了这大世系。
\item 他们想「原来我们过去是亲戚」,便极净喜。随后,释迦族与拘利族各以二百五十童子,计五百童子,给予世尊,为令随从。世尊见到彼等的前因,便说:「来!诸比丘!」他们全都具足以神变现出的八资具,从空中涌起而来,礼拜了世尊而立。世尊便带上他们,去了大林。
\item 他们的夫人远来送信,他们受到她们种种方式的诱惑,便生不满。世尊了知到彼等不满之相,便显现雪山,以\textbf{杜鹃本生}之谈,为欲除去彼等的不喜而说:「诸比丘!你们先前见过雪山吗?」「否也,世尊!」「来!诸比丘!你们看!」以自己的神变带他们到空中,示以种种品类的山——这是金山、这是银山、这是摩尼山——之后,便站在杜鹃湖畔的雄黄之原。随后决意:「让雪山中所有四足、多足等类的畜生、生类前来!在它们最后是杜鹃鸟!」且当它们前来时,以种名术语解说,示与彼等:「诸比丘!这些是天鹅,这些是白鹭,这些是轮鸟、迦陵频伽、象鼻鸟和鹤。」
\item 他们心怀惊异地看着,当看到最后到来的杜鹃鸟为一千只雌鸟围绕、坐在两只雌鸟用喙衔着的枝条中间后,生起惊异希有之心,便对世尊说:「尊者!世尊是否前生也曾在此为杜鹃王?」「唯!诸比丘!此杜鹃世系实由我创。过去,我们四人曾住在此:那烂陀·提婆罗作仙人,阿难作鹫王,富楼那牟佉作斑点布谷,我作杜鹃鸟。」便讲了全部\textbf{大杜鹃本生}。听完之后,这些比丘因前妻生起的不喜便即平息\footnote{据菩提比丘注 1245,此本生的主题是女人的贪欲与欺骗。}。
\item 随后,世尊便对他们讲了谛论,在讲论的终了,其中的末进成了须陀洹,其中的上游成了阿那含,竟无一凡夫或阿罗汉。随后,世尊便带上他们,再次降落在大林。且当返回时,这些比丘唯以自己的神变而返。然后,世尊为了他们得更上之道,又开示了法。这五百人开始作观,便住于阿罗汉。最初证得者第一个前来,想「我要告知世尊」,且来后说了「我大喜,世尊!我无不满」后,便礼拜世尊,坐在一边。如是,他们全都顺次前来,围绕世尊而坐——在逝瑟吒月\footnote{逝瑟吒月 \textit{Jeṭṭhamāsa}:即现在的五~六月。}布萨之日的晡时。
\item 随后,世尊为五百漏尽围绕,坐于最上佛座,整个一万轮围中除了无想有情及无色梵天的其余天人,都以「吉祥经注」中所说的方式化作微细之身,围绕世尊,想「我等将闻多彩、辩才的法之开示」。于此,四漏尽梵天从等至出起,未见到梵众,想「去哪里了」,经转向而了知此事,从后赶来,未得空处,便站在轮围之顶,各说了一颂。如说:\begin{quoting}于是,四净居众天人想:「彼世尊住于释氏迦毗罗卫大林中,与五百大比丘僧团一起,皆阿罗汉。且多数天人从十世界聚集,为见世尊及比丘僧团。我们何不前往世尊处,去后在世尊前各说一颂?」(相应部第 1:37 经)\end{quoting}一切当如「有偈品」中所说的方法可知。
\item 如是去到后,于此,一梵天在东方轮围之顶得了空处,便立于彼,说了此颂:\begin{quoting}在林中的大集会,天众聚集,\\我们前往这法的集会,来看不可战胜的僧伽。\end{quoting}站在西方轮围山顶者便听到了其唱颂之声。第二个在西方轮围之顶得了空处,便立于彼,听后说了此颂:\begin{quoting}众比丘于彼等持,正直己心,\\如御者持缰,智者守护诸根。\end{quoting}第三个在南方轮围之顶得了空处,便立于彼,听后说了此颂:\begin{quoting}扯断了柱子,扯断了门闩,不动者拔出了因陀柱,\\他们清净、离尘而行,年轻的龙象为具眼者善调。\end{quoting}第四个在北方轮围之顶得了空处,便立于彼,听后说了此颂:\begin{quoting}凡皈依佛者,皆不趣苦处之地,\\捐除了人身,他们将充实天众。\footnote{上述四颂,均出自\textbf{相应部}第 1:37 经。}\end{quoting}站在南方轮围山顶者便听到了其唱颂之声。如是,当此四梵天赞美会众而立时,大梵天们便蔽覆了一轮围而立。
\item 于是,世尊观察天众已,便告诸比丘:「诸比丘!对那些过去世曾存在的诸阿罗汉、正等正觉,曾聚集过的天人无过于此,如我现今,诸比丘!对将来世将存在的诸阿罗汉、正等正觉,将聚集的天人仍无过于此,如我现今。」随后,便依能与未能将天众分为两类——这些堪能、这些未能。于此,了知到「未能之众即便经百佛开示法,亦不能觉悟,而堪能之众则堪觉悟」,又将堪能之补特伽罗依性行分为六类——这些是贪行者,这些是嗔、痴、寻、信、觉行者\footnote{六种性行,见\textbf{清净道论}·说取业处品第 74 段及以下。}。
\item 如是以性行摄取已,便省思法论,再作意于此会众:「何等法的开示适宜此会众?能以自身的意乐了知,还是以他人的意乐、事件的发生、问题的主导?」随后,了知到「能以问题的主导了知」,便又转向全体会众:「有没有能提出问题的?」了知到「无有」后,想「如果我自问自答,对此会众不适宜,我何不变现出相佛」,便入基础禅那,出起后,以意生神变预作,变现出了相佛。随决意之心——让他肢体俱全、具足(大人)相、持衣钵、具足瞻视等——他便显现。他从东方世界而来,坐在与世尊齐等的坐处。如是来已,于世尊在此集会中依性行所说的六经——即前分离经、争辩争论经、小阵、大阵、迅速\footnote{上述五经,即\textbf{经集}·第四品第 10~14 经。}与此正游行——之中,为转起适宜贪行天人所讲的此经,便提出问题,说了下颂。\end{enumerate}

\subsection\*{\textbf{362} {\footnotesize 〔PTS 359〕}}

\textbf{「我问牟尼、广慧者,已度、已到彼岸、已止息的坚定者,\\}
\textbf{「从家出离,除去爱欲,那比丘应如何在世间正当地游行?」}

“Pucchāmi muniṃ pahūtapaññaṃ, tiṇṇaṃ pāraṅgataṃ parinibbutaṃ ṭhitattaṃ;\\
nikkhamma gharā panujja kāme, kathaṃ bhikkhu sammā so loke paribbajeyya”. %\hfill\textcolor{gray}{\footnotesize 1}

\begin{enumerate}\item 这里,\textbf{广慧},即大慧。\textbf{已度},即已度四暴流。\textbf{已到彼岸},即已证得涅槃。\textbf{已止息},即以有余依涅槃而止息。\textbf{坚定},即心不为世间法所动摇。\textbf{从家出离,除去爱欲},即除去物欲,从家居出离。\textbf{那比丘应如何在世间正当地游行},即是说不为世间所染而住、超越世间。此中其余皆已述及。\end{enumerate}

\subsection\*{\textbf{363} {\footnotesize 〔PTS 360〕}}

\textbf{「若其摒除了吉祥、」世尊说,「征兆、梦和相,\\}
\textbf{「他便舍弃吉祥的过失,他能在世间正当地游行。}

“Yassa maṅgalā samūhatā, \textit{(iti Bhagavā)} uppātā supinā ca lakkhaṇā ca;\\
so maṅgaladosavippahīno, sammā so loke paribbajeyya. %\hfill\textcolor{gray}{\footnotesize 2}

\begin{enumerate}\item 于是,因为无人能未证漏尽而在世间正当地游行,所以,当此一切补特伽罗之集合为贪行等摄取已,世尊为了彼彼存在过失的天人众能舍弃宿习的过失,从此颂开始,以阿罗汉为顶点,为阐明漏尽之行道,说了十五颂。
\item 这里,先就初颂中的\textbf{吉祥},即「吉祥经」中所说的见吉祥等的同义语。\textbf{摒除},即善加除去,以慧剑根除。\textbf{征兆},即对征兆转起如是的执著——流星坠落、天火燃烧等有如是果报。\textbf{梦},即对梦转起如是的执著——在上午作梦就会有此,在中午等有此,以左胁卧而见到就会有此,以右胁等有此,在梦中见到月亮就会有此,见到太阳等有此。\textbf{相},即读取棍杖之相、衣布之相等的纹路,对相转起如是的执著——以此而有彼。当知所有这些皆如「梵网经」中所说。
\item \textbf{他便舍弃吉祥的过失},即除三十八种大吉祥\footnote{三十八种大吉祥:即\textbf{吉祥经}第 262~271 颂所说的吉祥。}外,其余名为吉祥的过失。若其摒除了这些吉祥,他便「舍弃吉祥的过失」。或者,由舍弃吉祥及征兆等的过失为「舍弃吉祥的过失」,由证得圣道故,不依吉祥等认可清净。所以,\textbf{他能在世间正当地游行},这漏尽者不为世间所染,能在世间正当地游行。\end{enumerate}

\subsection\*{\textbf{364} {\footnotesize 〔PTS 361〕}}

\textbf{「比丘应当调伏对于人界与天界爱欲的贪染,\\}
\textbf{「越过了有,体认了法,他能在世间正当地游行。}

Rāgaṃ vinayetha mānusesu, dibbesu kāmesu cāpi bhikkhu;\\
atikkamma bhavaṃ samecca dhammaṃ, sammā so loke paribbajeyya. %\hfill\textcolor{gray}{\footnotesize 3}

\begin{enumerate}\item 在第二颂中,\textbf{比丘应当调伏对于人界与天界爱欲的贪染},即对于人界与天界种种爱欲,应以阿那含道引领至无生法性,调伏贪染。\textbf{越过了有,体认了法},即在如是调伏了贪染后,更进一步,以阿罗汉道由一切品类完成遍知现观等,体认了四谛等类的法,以此行道,越过了三种有。\textbf{他能……},即此比丘也能在世间正当地游行。\end{enumerate}

\subsection\*{\textbf{365} {\footnotesize 〔PTS 362〕}}

\textbf{「把诽谤抛在了背后,比丘应舍弃忿怒和贪婪,\\}
\textbf{「舍弃了顺从和违逆,他能在世间正当地游行。}

Vipiṭṭhikatvāna pesuṇāni, kodhaṃ kadarīyaṃ jaheyya bhikkhu;\\
anurodha-virodha-vippahīno, sammā so loke paribbajeyya. %\hfill\textcolor{gray}{\footnotesize 4}

\begin{enumerate}\item 在第三颂中,\textbf{舍弃了顺从和违逆},即于一切依处舍弃了贪嗔。其余唯如已述。且在所有颂中,当与「此比丘也能在世间正当地游行」连结。从此以后,也不再说连结,唯解释未曾说者。\end{enumerate}

\subsection\*{\textbf{366} {\footnotesize 〔PTS 363〕}}

\textbf{「舍弃了喜与不喜,无所取著,不依于任何处,\\}
\textbf{「从诸可结缚处解脱,他能在世间正当地游行。}

Hitvāna piyañ ca appiyañ ca, anupādāya anissito kuhiñci;\\
saṃyojaniyehi vippamutto, sammā so loke paribbajeyya. %\hfill\textcolor{gray}{\footnotesize 5}

\begin{enumerate}\item 在第四颂中,当知依有情、行有两种\textbf{喜与不喜},于此,以舍弃欲贪、嗔恚而\textbf{舍弃}。\textbf{无所取著},即不以四取\footnote{四取:即欲取、见取、戒禁取、我语取。}执持任一法。\textbf{不依于任何处},即不以百八种爱依\footnote{百八种爱依:为内外十八界,一一有过、现、未来三种,而成一百〇八种。}与六十二种见依\footnote{六十二种见依,见\textbf{长部}·梵网经。}依于任何色等法或有。\textbf{从诸可结缚处解脱},一切三界法,由为十种结缚\footnote{十种结缚:有经、论两种,见\textbf{摄阿毗达摩义}·摄集分别品。}的境域故,为可结缚处,即由道的修习及遍知故,从此一切品类解脱之义。
\item 且此中,当知以第一句说舍弃贪嗔,以第二句说无有取著与依,以第三句说从其余的不善及不善依处解脱。或者,以第一句说舍弃贪嗔,以第二句说其方法,以第三句说由舍弃彼等故,从诸可结缚处解脱。\end{enumerate}

\subsection\*{\textbf{367} {\footnotesize 〔PTS 364〕}}

\textbf{「他不在依持中寻找坚实,于诸取著调伏欲贪,\\}
\textbf{「他无依,不被他人引领,他能在世间正当地游行。}

Na so upadhīsu sāram eti, ādānesu vineyya chandarāgaṃ;\\
so anissito anaññaneyyo, sammā so loke paribbajeyya. %\hfill\textcolor{gray}{\footnotesize 6}

\begin{enumerate}\item 在第五颂中,\textbf{依持},即蕴依持等\footnote{蕴依持等,见\textbf{有财者经}第 33 颂注。}。\textbf{取著},即以可取之义而得称如是。\textbf{不被他人引领},由善见无常等故,不被任何说「此更胜」者引领。其余词义自明。
\item 这即是说:于诸取著,以第四道彻底调伏欲贪已,他即调伏了欲贪,不在这些依持中寻找坚实,以无坚实性观见一切依持。随后,于彼等不以两种依而依靠,或不被其他任何说「此更胜」者引领的漏尽比丘,他能在世间正当地游行。\end{enumerate}

\subsection\*{\textbf{368} {\footnotesize 〔PTS 365〕}}

\textbf{「不以语、意与业而违逆,正当地了知了法,\\}
\textbf{「愿求着涅槃境界,他能在世间正当地游行。}

Vacasā manasā ca kammunā ca, aviruddho sammā viditvā dhammaṃ;\\
nibbānapadābhipatthayāno, sammā so loke paribbajeyya. %\hfill\textcolor{gray}{\footnotesize 7}

\begin{enumerate}\item 在第六颂中,\textbf{不违逆},即由舍弃此三恶行故,不以诸善行违逆。\textbf{了知了法},即以道了知四谛之法。\textbf{愿求着涅槃境界},即愿求着无余依之蕴般涅槃境界。其余之义自明。\end{enumerate}

\subsection\*{\textbf{369} {\footnotesize 〔PTS 366〕}}

\textbf{「若『他礼拜我』,能不高举,遇到骂詈,比丘也能不受影响,\\}
\textbf{「得到他人的食物能不陶醉,他能在世间正当地游行。}

Yo “vandati man” ti n’unnameyya, akkuṭṭho pi na sandhiyetha bhikkhu;\\
laddhā parabhojanaṃ na majje, sammā so loke paribbajeyya. %\hfill\textcolor{gray}{\footnotesize 8}

\begin{enumerate}\item 在第七颂中,\textbf{遇到骂詈},即为十骂詈事所咒骂。\textbf{不受影响},即不怨恨、不激怒。\textbf{得到他人的食物能不陶醉},即得到他人布施的信施,不以「我出名、有名闻、有利养」而陶醉。其余之义自明。\end{enumerate}

\subsection\*{\textbf{370} {\footnotesize 〔PTS 367〕}}

\textbf{「比丘舍弃了贪与有,且戒离了割截捆绑,\\}
\textbf{「他度诸犹疑,离于箭刺,他能在世间正当地游行。}

Lobhañ ca bhavañ ca vippahāya, virato chedanabandhanā ca bhikkhu;\\
so tiṇṇakathaṅkatho visallo, sammā so loke paribbajeyya. %\hfill\textcolor{gray}{\footnotesize 9}

\begin{enumerate}\item 在第八颂中,\textbf{贪},即非理之贪。\textbf{有},即欲有等的有。如是,以二词说了对有与财的渴爱,或以前者说一切渴爱,以后者说业有。\textbf{且戒离了割截捆绑},如是,由舍弃了这些业与烦恼,便戒离了对其他有情的割截捆绑。其余唯如已述\footnote{颂中的「度诸犹疑,离于箭刺」,见\textbf{蛇经}第 17 颂注。}。\end{enumerate}

\subsection\*{\textbf{371} {\footnotesize 〔PTS 368〕}}

\textbf{「了知了自身的适宜,比丘不应伤害世间任何人,\\}
\textbf{「如实地了知了法,他能在世间正当地游行。}

Sāruppaṃ attano viditvā, no ca bhikkhu hiṃseyya kañci loke;\\
yathātathiyaṃ viditvā dhammaṃ, sammā so loke paribbajeyya. %\hfill\textcolor{gray}{\footnotesize 10}

\begin{enumerate}\item 在第九颂中,\textbf{了知了自身的适宜},即舍弃了模仿自身比丘相的邪求等,以住立于此正行道,了知了正求等的活命清净及其它。因为无事仅以所知而成。\textbf{法},即以如实知见了知了蕴、处等类,或以道了知了四谛之法。其余之义自明。\end{enumerate}

\subsection\*{\textbf{372} {\footnotesize 〔PTS 369〕}}

\textbf{「若其已无任何随眠,并且诸不善根已被铲除,\\}
\textbf{「他无意乐、无希求,他能在世间正当地游行。}

Yassānusayā na santi keci, mūlā ca akusalā samūhatāse;\\
so nirāso anāsisāno, sammā so loke paribbajeyya. %\hfill\textcolor{gray}{\footnotesize 11}

\begin{enumerate}\item 在第十颂中,\textbf{他无意乐、无希求},若其以圣道消除故,无有随眠,并且诸不善根已被铲除,他便无意乐、无渴爱,随后,以无有意乐,便不希求任何色等法,因此说「无意乐、无希求」。其余唯如已述\footnote{前半颂全同\textbf{蛇经}第 14 颂,见其注。}。\end{enumerate}

\subsection\*{\textbf{373} {\footnotesize 〔PTS 370〕}}

\textbf{「漏已尽,慢已舍,越过了一切贪路,\\}
\textbf{「已调御,已止息,坚定,他能在世间正当地游行。}

Āsavakhīṇo pahīnamāno, sabbaṃ rāgapathaṃ upātivatto;\\
danto parinibbuto ṭhitatto, sammā so loke paribbajeyya. %\hfill\textcolor{gray}{\footnotesize 12}

\begin{enumerate}\item 在第十一颂中,\textbf{漏已尽},即灭尽了四漏。\textbf{慢已舍},即舍断了九种慢。\textbf{贪路},即作为贪之境域的种种三界之法。\textbf{越过},即以遍知与舍断而超越。\textbf{已调御},即舍弃了一切门的躁动,以圣调御至于已调御之地。\textbf{已止息},即以熄灭烦恼之火而成清凉。其余唯如已述。\end{enumerate}

\subsection\*{\textbf{374} {\footnotesize 〔PTS 371〕}}

\textbf{「具信、多闻的见决定者,在人群中不随众的智者,\\}
\textbf{「调伏了贪、嗔、恚,他能在世间正当地游行。}

Saddho sutavā niyāmadassī, vaggagatesu na vaggasāri dhīro;\\
lobhaṃ dosaṃ vineyya paṭighaṃ, sammā so loke paribbajeyya. %\hfill\textcolor{gray}{\footnotesize 13}

\begin{enumerate}\item 在第十二颂中,\textbf{具信},即于佛等的功德,不由他缘,具足一切行相具足的不坏净,不以对他人的信而前去行道。如说:\begin{quoting}于此,尊者!我不以对世尊的信而前去。(增支部第 5:34 经)\end{quoting}\textbf{多闻},即由完成听闻的义务而具足第一义的闻。\textbf{见决定者},即于轮回荒漠歧途的世间,见到趣向不死之城的作为正性决定之道,即是说见道者。\textbf{在人群中不随众},人群者,即彼此对立的六十二见者,在如是成见的有情人群中不随众,不随于「这将断绝、这将如是存在」等见。\textbf{恚},即敌意,即是说心的困扰,即嗔的一个属性。其余唯如已述。\end{enumerate}

\subsection\*{\textbf{375} {\footnotesize 〔PTS 372〕}}

\textbf{「清净的胜者,去蔽者,受控于法,到达彼岸,不动者,\\}
\textbf{「善巧于行灭智,他能在世间正当地游行。}

Saṃsuddhajino vivaṭṭacchado, dhammesu vasī pāragū anejo;\\
saṅkhāranirodhañāṇakusalo, sammā so loke paribbajeyya. %\hfill\textcolor{gray}{\footnotesize 14}

\begin{enumerate}\item 在第十三颂中,\textbf{清净的胜者},即以清净的阿罗汉道战胜烦恼者。\textbf{去蔽者},即去除贪嗔痴的遮蔽者。\textbf{受控于法},即受控于四谛之法。因为他不可能既已知晓此法,却去做任何别的,因此漏尽者被称为「受控于法」。\textbf{到达彼岸},彼岸即涅槃,得至于彼,即是说以有余依而证得。\textbf{不动者},即已离去渴爱之动摇者。\textbf{善巧于行灭智},行灭即涅槃,于此之智为圣道慧,善巧于此,即是说经四次修习而娴熟。\end{enumerate}

\subsection\*{\textbf{376} {\footnotesize 〔PTS 373〕}}

\textbf{「于过去及未来克服了思惟,极度净慧者,\\}
\textbf{「从一切处解脱,他能在世间正当地游行。}

Atītesu anāgatesu cāpi, kappātīto aticca suddhipañño;\\
sabbāyatanehi vippamutto, sammā so loke paribbajeyya. %\hfill\textcolor{gray}{\footnotesize 15}

\begin{enumerate}\item 在第十四颂中,\textbf{过去},即已经转起的逝去的五蕴。\textbf{未来},即尚未转起的五蕴。\textbf{克服了思惟},即克服了「我、我所」的思惟,或一切爱、见的思惟。\textbf{极度净慧者}\footnote{义注给出了 aticca 的两种解释:作为不变词,意为「极度」,作为动词独立式,意为「越过」,颂中的译文从前者。},即极度净慧者,或净慧者越过。越过了什么?三时。因为阿罗汉越过了一切被称为「无明、行」的过去时,被称为「生、老死」的未来时,以及以识为初、以有为后的现在时,度脱了疑惑,到达了最上清净之慧而住,因此说「净慧者越过」。\textbf{一切处},即十二处。因为阿罗汉如是克服了思惟,由克服思惟且由极度净慧,不趣于未来的任何处,因此说「从一切处解脱」。\end{enumerate}

\subsection\*{\textbf{377} {\footnotesize 〔PTS 374〕}}

\textbf{「知晓了句,体认了法,明了地见到诸漏的舍断,\\}
\textbf{「灭尽了一切依持,他能在世间正当地游行。」}

Aññāya padaṃ samecca dhammaṃ, vivaṭaṃ disvāna pahānam āsavānaṃ;\\
sabbupadhīnaṃ parikkhayāno, sammā so loke paribbajeyya”. %\hfill\textcolor{gray}{\footnotesize 16}

\begin{enumerate}\item 在第十五颂中,\textbf{知晓了句},即于所说的「四句谛中胜\footnote{即\textbf{法句}·道品第 273 颂的第二句。}」中的一一句,以「前分谛差别慧」而知晓。\textbf{体认了法},此后更以四圣道体认了四谛之法。\textbf{明了地见到诸漏的舍断},然后,以省察智明了、显明、无蔽地见到得称漏尽的涅槃。\textbf{灭尽了一切依持},由灭尽了一切蕴、爱欲、烦恼、行作等类的依持,无所执著的比丘,\textbf{他能在世间正当地游行}、住,无所执著地行于世间。即完成了开示。\end{enumerate}

\subsection\*{\textbf{378} {\footnotesize 〔PTS 375〕}}

\textbf{「确实,世尊!这实如此,凡如是住的调御比丘,\\}
\textbf{「已超越一切结缚与轭,他能在世间正当地游行。」}

“Addhā hi Bhagavā tath’eva etaṃ, yo so evaṃvihārī danto bhikkhu;\\
sabbasaṃyojana-yoga-vītivatto, sammā so loke paribbajeyyā” ti. %\hfill\textcolor{gray}{\footnotesize 17}

\begin{enumerate}\item 随后,这相(佛)为赞叹法的开示,便说了此颂。这里,\textbf{凡如是住},即为显明如是以各颂所示的比丘——凡摒除了吉祥等、舍弃一切吉祥的过失而住者,凡调伏了对于人天爱欲的贪染、越过了有、现观法而住者,等等——而说。其余自明。
\item 而其连结为:\textbf{确实,世尊!这实如此},即你说了「若其摒除了吉祥」等后,在各颂的终了便说「他能在世间正当地游行」者。什么原因?\textbf{凡如是住的比丘},他以最上调御而\textbf{调御},\textbf{已超越一切}十种\textbf{结缚与}四\textbf{轭},所以,\textbf{他能在世间正当地游行},我于此无有疑惑。如是,在说了赞叹开示之颂后,便以阿罗汉为顶点完成了开示。
\item 在经的终了,百千俱胝的天人证了最上果,证须陀洹、斯陀含、阿那含果者不计其数。\end{enumerate}

\begin{center}\vspace{1em}正游行经第十三\\Sammāparibbājanīyasuttaṃ terasamaṃ.\end{center}