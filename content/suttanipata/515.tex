\section{布萨罗学童问}

\subsection\*{\textbf{1119} {\footnotesize 〔PTS 1112〕}}

\textbf{「能宣示过去者,」尊者布萨罗说,「不动者,切断疑虑者,\\}
\textbf{「已度一切法者,我带着问题前来。}

\begin{enumerate}\item 这里,\textbf{能宣示过去者},即世尊能宣示自己与他人「一生」等类的过去。\end{enumerate}

\subsection\*{\textbf{1120} {\footnotesize 〔PTS 1113〕}}

\textbf{「无有色想者,舍弃一切身者,\\}
\textbf{「于内外见『不存在任何』者,\\}
\textbf{「释迦!我问其智,这样的人应如何被引领?」}

\begin{enumerate}\item \textbf{无有色想者},即超越色想者。\textbf{舍弃一切身者},即以彼分、镇伏舍弃一切色身者,意即舍弃了色有之结生者。\textbf{见「不存在任何」者},即以于识的无有作观,见「不存在任何」者,即是说得无所有处者。\textbf{释迦!我问其智},释迦,即为称呼世尊而说,我问此人的智应当是怎样的。\textbf{应如何被引领},即他如何能成就更高的智。\end{enumerate}

\subsection\*{\textbf{1121} {\footnotesize 〔PTS 1114〕}}

\textbf{「如来证知着,布萨罗!」世尊说,「一切识住,\\}
\textbf{「知晓他住立、解脱或是志在于彼。}

\begin{enumerate}\item 于是,世尊向其先阐明自己对这样的人的无碍之智,再解释此智,说了二颂。这里,\textbf{一切识住},即以行作的四种\footnote{以行作的四种,见\textbf{相应部}第 22:53 经,即四取蕴。}、以结生的七种\footnote{以结生的七种,见\textbf{增支部}第 7:44 经,即异身异想的人、一部分天和一部分堕处,异身同想的初次转生的梵众天,同身异想的光音天,同身同想的遍净天,空无边处,识无边处,以及无所有处。}等一切识住。\textbf{知晓他住立},即知晓此人以业之行作而住立,「他将来成为如是趣者」。\textbf{解脱},即解脱于无所有处等。\textbf{志在于彼},即参与于彼。\end{enumerate}

\subsection\*{\textbf{1122} {\footnotesize 〔PTS 1115〕}}

\textbf{「了知了无所有的生起,即『欢喜是结缚』,\\}
\textbf{「如是证知此已,随后他对此修观,\\}
\textbf{「这便是那已立婆罗门的如实之智。」}

\begin{enumerate}\item \textbf{了知了无所有的生起},即了知了生起无所有处的业之行作:「什么是这障碍?」\textbf{欢喜是结缚},即了知了在此被称为无色贪的欢喜是结缚。\textbf{随后他对此修观},即随后,从无所有处定出起后,他以无常等对此定修观。\textbf{这便是那如实之智},即这便是这如是修观之人渐次生起的无颠倒的阿罗汉智。其余一切处皆自明。
\item 如是,世尊仍以阿罗汉为顶点开示了此经。当开示终了,与先前一样,而有法的现观。\end{enumerate}

