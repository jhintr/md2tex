\section{布萨罗学童问}

\begin{center}Posāla Māṇava Pucchā\end{center}\vspace{1em}

\subsection\*{\textbf{1119} {\footnotesize 〔PTS 1112〕}}

\textbf{「能宣示过去者,」尊者布萨罗说,「不动者,切断疑虑者,\\}
\textbf{「已度一切法者,我带着问题前来。}

“Yo atītaṃ ādisati, \textit{(icc āyasmā Posālo)} anejo chinnasaṃsayo;\\
pāraguṃ sabbadhammānaṃ, atthi pañhena āgamaṃ. %\hfill\textcolor{gray}{\footnotesize 1}

\begin{itemize}\item 案,此颂的前三句都是\textbf{前来}的宾语,同生起学童问的第一颂。\end{itemize}

\subsection\*{\textbf{1120} {\footnotesize 〔PTS 1113〕}}

\textbf{「无有色想者,舍弃一切身者,\\}
\textbf{「于内外见『不存在任何』者,\\}
\textbf{「我问其智,释迦!如此等者如何被引领?」}

Vibhūta-rūpasaññissa, sabbakāyappahāyino;\\
ajjhattañ ca bahiddhā ca, natthi kiñcī ti passato;\\
ñāṇaṃ Sakkānupucchāmi, kathaṃ neyyo tathāvidho”. %\hfill\textcolor{gray}{\footnotesize 2}

\begin{enumerate}\item \textbf{无有色想},即超越色想。\textbf{舍弃一切身},即以彼分、镇伏舍弃一切色身,舍弃了色有之结生的意思。\textbf{见『不存在任何』者},即以观识的不存在而见「不存在任何」者,指得无所有处(定)者。\textbf{如何被引领},即他如何能成就更高的智。\end{enumerate}

\subsection\*{\textbf{1121} {\footnotesize 〔PTS 1114〕}}

\textbf{「如来证知着,布萨罗!」世尊说,「一切识住,\\}
\textbf{「知晓他住立、解脱或是志在于彼。}

“Viññāṇaṭṭhitiyo sabbā, \textit{(Posālā ti Bhagavā)} abhijānaṃ Tathāgato;\\
tiṭṭhantam enaṃ jānāti, vimuttaṃ tapparāyaṇaṃ. %\hfill\textcolor{gray}{\footnotesize 3}

\begin{enumerate}\item \textbf{一切识住},即以行作的四种、以结生的七种等一切识住。\textbf{知晓他住立},即知晓此人以业之行作而住立,「他未来将入如是趣」。\textbf{解脱},即胜解于无所有处等。\textbf{志在于彼},即参与。\end{enumerate}

\begin{itemize}\item 案,\textbf{识住}的解释详见义释,其中\textbf{以行作的四种}见杂阿含经第 40 经、相应部·蕴相应第 22:53 经,\textbf{以结生的七种}见增支部七集第 44 经。\end{itemize}
\begin{itemize}\item 案,义注对\textbf{解脱}的解释也费解,详见菩提比丘注 2143、2150。\end{itemize}

\subsection\*{\textbf{1122} {\footnotesize 〔PTS 1115〕}}

\textbf{「了知了无所有的生起,即『欢喜是结缚』,\\}
\textbf{「如是证知此已,随后于此修观,\\}
\textbf{「这便是他,已立婆罗门的如实之智。」}

Ākiñcaññasambhavaṃ ñatvā, nandī saṃyojanaṃ iti;\\
evam etaṃ abhiññāya, tato tattha vipassati;\\
etaṃ ñāṇaṃ tathaṃ tassa, brāhmaṇassa vusīmato” ti. %\hfill\textcolor{gray}{\footnotesize 4}

\begin{enumerate}\item \textbf{了知了无所有的生起},即了知了无所有处生起的业之行作。「什么是这障碍?」\textbf{欢喜是结缚},即了知了于此被称为无色贪的欢喜是结缚。\textbf{随后于此修观},随后,从无所有处定出起后,以无常等对此定修观。\textbf{这便是他的如实之智},这便是这如是修观之人渐次生起的无颠倒的阿罗汉智。
\item 如是,世尊同样以阿罗汉为顶点开示了此经。当开示终了,与先前一样,而有法的现观。\end{enumerate}

\begin{itemize}\item 案,\textbf{已立},即「梵行已立」之已立。\end{itemize}

\begin{center}\vspace{1em}布萨罗学童问第十四\\Posālamāṇavapucchā cuddasamā.\end{center}