\section{低舍弥勒学童问}

\begin{center}Tissametteyya Māṇava Pucchā\end{center}\vspace{1em}

\begin{enumerate}\item 缘起为何?所有经的缘起都以问题为主导。因为这些婆罗门得到世尊以「都有机会,请问」所作的邀请,便问了各自的疑惑。世尊则解答了他们的各个问题。如是当知这些经唯以问题为主导。
\item 而当阿耆多的问题结束时,空王便开始以「如何观察世间,死王便不得见他」提问。世尊了知到「他的诸根尚未至成熟」,便拒绝道:「你且住!空王!让别人提问!」随后,低舍弥勒为问自己的疑惑,说了此颂。\end{enumerate}

\subsection\*{\textbf{1047} {\footnotesize 〔PTS 1040〕}}

\textbf{「谁在此对世间知足?」尊者低舍弥勒说,「谁没有动摇?\\}
\textbf{「谁证知了两端,以智慧不染于中间?\\}
\textbf{「你说谁为『大人』?谁在此超越了缝合?」}

“Ko’dha santusito loke, \textit{(icc āyasmā Tissametteyyo)} kassa no santi iñjitā;\\
ko ubh’anta-m-abhiññāya, majjhe mantā na lippati;\\
kaṃ brūsi mahāpuriso ti, ko idha sibbinim accagā”. %\hfill\textcolor{gray}{\footnotesize 1}

\begin{enumerate}\item 这里,\textbf{动摇},即为爱、见所震动。\end{enumerate}

\subsection\*{\textbf{1048} {\footnotesize 〔PTS 1041〕}}

\textbf{「于爱欲中具梵行,弥勒!」世尊说,「离爱,始终具念,\\}
\textbf{「经省思而寂灭的比丘,他没有动摇。}

“Kāmesu brahmacariyavā, \textit{(Metteyyā ti Bhagavā)} vītataṇho sadā sato;\\
saṅkhāya nibbuto bhikkhu, tassa no santi iñjitā. %\hfill\textcolor{gray}{\footnotesize 2}

\begin{enumerate}\item 世尊为对其解答此义,说了以下二颂。这里,\textbf{于爱欲中具梵行},即因由爱欲而具梵行,即是说见到爱欲中的过患而具足道梵行。
\item 至此显示了「知足」,而以「离爱」等说明无动摇。这里,\textbf{经省思而寂灭},即以无常等省察诸法,因贪等的止息而寂灭。其余由处处已述故,皆自明。\end{enumerate}

\subsection\*{\textbf{1049} {\footnotesize 〔PTS 1042〕}}

\textbf{「他证知了两端,以智慧不染于中间,\\}
\textbf{「我说他为『大人』,他在此超越了缝合。」}

So ubh’anta-m-abhiññāya, majjhe mantā na lippati;\\
taṃ brūmi mahāpuriso ti, so idha sibbinim accagā” ti. %\hfill\textcolor{gray}{\footnotesize 3}

\begin{enumerate}\item 如是,世尊仍以阿罗汉为顶点开示了此经。当开示终了,这婆罗门与一千弟子即住于阿罗汉,而其他数千人生起了法眼。余皆同前。\end{enumerate}

\begin{center}\vspace{1em}低舍弥勒学童问第二\\Tissametteyyamāṇavapucchā dutiyā.\end{center}