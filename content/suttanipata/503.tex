\section{低舍弥勒学童问}

\begin{center}Tissametteyya Māṇava Pucchā\end{center}\vspace{1em}

\begin{enumerate}\item 缘起为何?对于所有经,缘起都取决于提问。因为他们得到世尊的许可「都有机会,请问」,便问了各自的疑惑,而世尊则解答了他们的各个问题。如是当知这些经唯是取决于提问。
\item 当阿耆多的问题结束时,空王便开始提问。世尊了知到「他的诸根尚未得至成熟」,便拒绝道「你且住!空王!让别人提问」。随后,低舍弥勒便问了自己的疑惑而说了此颂。\end{enumerate}

\subsection\*{\textbf{1047} {\footnotesize 〔PTS 1040〕}}

\textbf{「谁于此满足于世间?」尊者低舍弥勒说,「谁没有动摇?\\}
\textbf{「谁在证知两端后,以智慧不染于中间?\\}
\textbf{「你称说谁为『大人』?谁于此超越了缝合?」}

“Ko’dha santusito loke, \textit{(icc āyasmā Tissametteyyo)} kassa no santi iñjitā;\\
ko ubh’anta-m abhiññāya, majjhe mantā na lippati;\\
kaṃ brūsi mahāpuriso ti, ko idha sibbinim accagā”. %\hfill\textcolor{gray}{\footnotesize 1}

\begin{enumerate}\item \textbf{动摇},即为爱、见所震动。\end{enumerate}

\subsection\*{\textbf{1048} {\footnotesize 〔PTS 1041〕}}

\textbf{「于爱欲中具梵行,弥勒!」世尊说,「离爱,始终具念,\\}
\textbf{「经省思而寂灭的比丘,他没有动摇。}

“Kāmesu brahmacariyavā, \textit{(Metteyyā ti Bhagavā)} vītataṇho sadā sato;\\
saṅkhāya nibbuto bhikkhu, tassa no santi iñjitā. %\hfill\textcolor{gray}{\footnotesize 2}

\begin{enumerate}\item \textbf{于爱欲中具梵行},即因爱欲而具梵行,见到爱欲中的过患而具足道梵行的意思。至此说明「满足」,而以\textbf{离爱}等说明无动摇。这里,\textbf{经省思而寂灭},即以无常等省察了诸法,以止息了贪等而寂灭。\end{enumerate}

\subsection\*{\textbf{1049} {\footnotesize 〔PTS 1042〕}}

\textbf{「他在证知两端后,以智慧不染于中间,\\}
\textbf{「我称说他为『大人』,他于此超越了缝合。」}

So ubh’anta-m abhiññāya, majjhe mantā na lippati;\\
taṃ brūmi mahāpuriso ti, so idha sibbinim accagā” ti. %\hfill\textcolor{gray}{\footnotesize 3}

\begin{enumerate}\item 如是,世尊同样以阿罗汉为顶点开示了此经。当开示终了,这婆罗门与一千弟子即住于阿罗汉性,而其余数千人生起了法眼。余如前说。\end{enumerate}

\begin{center}\vspace{1em}低舍弥勒学童问第二\\Tissametteyyamāṇavapucchā dutiyā.\end{center}