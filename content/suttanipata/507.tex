\section{优波湿婆学童问}

\begin{center}Upasīva Māṇava Pucchā\end{center}\vspace{1em}

\subsection\*{\textbf{1076} {\footnotesize 〔PTS 1069〕}}

\textbf{「我独自、无依,释迦!」尊者优波湿婆说,「不能度过洪大的暴流,\\}
\textbf{「请说说所缘!一切眼者!依于此,我好度过这暴流。」}

“Eko ahaṃ Sakka mahantam oghaṃ, \textit{(icc āyasmā Upasīvo)} anissito no visahāmi tārituṃ;\\
ārammaṇaṃ brūhi Samantacakkhu, yaṃ nissito ogham imaṃ tareyyaṃ”. %\hfill\textcolor{gray}{\footnotesize 1}

\begin{enumerate}\item \textbf{无依},即无依于人或法。\textbf{所缘},即依止。\textbf{依于此},即依于此人或法。\end{enumerate}

\subsection\*{\textbf{1077} {\footnotesize 〔PTS 1070〕}}

\textbf{「觉察着无所有,具念,优波湿婆!」世尊说,「依于『这不存在』,你能度过暴流,\\}
\textbf{「舍弃了爱欲,离于疑惑,昼夜寻求渴爱之灭尽!」}

“Ākiñcaññaṃ pekkhamāno satimā, \textit{(Upasīvā ti Bhagavā)} natthī ti nissāya tarassu oghaṃ;\\
kāme pahāya virato kathāhi, taṇhakkhayaṃ nattam ahābhipassa”. %\hfill\textcolor{gray}{\footnotesize 2}

\begin{enumerate}\item 现在,因为这婆罗门是得无所有处(定)者,且他不知道具有这(定)也是依止,由此,世尊向他显明这是依止以及更高的出离之道,而说此颂。于此,\textbf{觉察},即念入于无所有处定后再出定,以无常等观察。\textbf{依于『这不存在』},即把「什么都不存在」所转起的定作为所缘。\textbf{你能度过暴流},即此后,以转起的毗婆舍那,你便能随其所适地度过四种暴流。\textbf{昼夜寻求渴爱之灭尽},即于昼夜间,将涅槃显明而观,以此对他说现法乐住。\end{enumerate}

\subsection\*{\textbf{1078} {\footnotesize 〔PTS 1071〕}}

\textbf{「若于一切爱欲离贪,」尊者优波湿婆说,「依于无所有,舍弃了其它,\\}
\textbf{「解脱于最高的想之解脱,他能否住立于此,不再离开?」}

“Sabbesu kāmesu yo vītarāgo, \textit{(icc āyasmā Upasīvo)} ākiñcaññaṃ nissito hitvā m aññaṃ;\\
saññāvimokkhe parame vimutto, tiṭṭhe nu so tattha anānuyāyī”. %\hfill\textcolor{gray}{\footnotesize 3}

\begin{enumerate}\item 现在,听到「舍弃了爱欲」后,看到自己以镇伏之力舍弃爱欲,而说此颂。\textbf{其它},即其它六种较低的等至。\textbf{最高的想之解脱},即在七种想之解脱中最高的无所有处。\textbf{他能否住立于此,不再离开},即是问,此人能否不离于此无所有处梵界而住立。\end{enumerate}

\subsection\*{\textbf{1079} {\footnotesize 〔PTS 1072〕}}

\textbf{「若于一切爱欲离贪,优波湿婆!」世尊说,「依于无所有,舍弃了其它,\\}
\textbf{「解脱于最高的想之解脱,他能住立于此,不再离开。」}

“Sabbesu kāmesu yo vītarāgo, \textit{(Upasīvā ti Bhagavā)} ākiñcaññaṃ nissito hitvā m aññaṃ;\\
saññāvimokkhe parame vimutto, tiṭṭheyya so tattha anānuyāyī”. %\hfill\textcolor{gray}{\footnotesize 4}

\begin{enumerate}\item 于是,世尊认可此状态有六万劫之量,对他说了第三颂。\end{enumerate}

\subsection\*{\textbf{1080} {\footnotesize 〔PTS 1073〕}}

\textbf{「如果他能住立于此许多年,不再离开,一切眼者!\\}
\textbf{「他能否于彼处得清凉而解脱?如此等者的识是否灭去?」}

“Tiṭṭhe ce so tattha anānuyāyī, pūgam pi vassānaṃ Samantacakkhu;\\
tatth’eva so sītisiyā vimutto, cavetha viññāṇaṃ tathāvidhassa”. %\hfill\textcolor{gray}{\footnotesize 5}

\begin{enumerate}\item \textbf{他能否于彼处得清凉而解脱},此人即于彼无所有处解脱种种苦、得证清凉,即得证涅槃、成恒常已而能住立的意思。\textbf{如此等者的识是否灭去},即以「这样的人的识是否无取而般涅槃」问断灭,或以「是否为了获得结生而存在」问其结生。\end{enumerate}

\subsection\*{\textbf{1081} {\footnotesize 〔PTS 1074〕}}

\textbf{「好比火焰为疾风所扰乱,优波湿婆!」世尊说,「消逝而不可得名,\\}
\textbf{「如是,牟尼解脱于名身,消逝而不可得名。」}

“Accī yathā vātavegena khittā, \textit{(Upasīvā ti Bhagavā)} atthaṃ paleti na upeti saṅkhaṃ;\\
evaṃ munī nāmakāyā vimutto, atthaṃ paleti na upeti saṅkhaṃ”. %\hfill\textcolor{gray}{\footnotesize 6}

\begin{enumerate}\item 于是,世尊不取于断常,向他显明于此(无所有处)生起的圣弟子的无取涅槃,而说此颂。\textbf{不可得名},即不可被称为「去到名为某某的方向」。\textbf{牟尼解脱于名身},即于此生起的有学牟尼以生性解脱于先前的色身,于此转起了第四道,由遍知了法身而又解脱于名身,成俱分解脱、漏尽已,而得称为无取般涅槃的消逝,不可称名为「刹帝利或婆罗门」等。\end{enumerate}

\subsection\*{\textbf{1082} {\footnotesize 〔PTS 1075〕}}

\textbf{「这消逝,是说他不存在,还是说永远无病?\\}
\textbf{「牟尼!请对我善加解释!因为这法已如是为你所知。」}

“Atthaṅgato so uda vā so natthi, udāhu ve sassatiyā arogo;\\
taṃ me Munī sādhu viyākarohi, tathā hi te vidito esa dhammo”. %\hfill\textcolor{gray}{\footnotesize 7}

\begin{enumerate}\item 现在,听到「消逝」后,未能如理省察其义,而说此颂。\textbf{无病},即不变异法。\end{enumerate}

\subsection\*{\textbf{1083} {\footnotesize 〔PTS 1076〕}}

\textbf{「消逝者无法度量,优波湿婆!」世尊说,「对他无法有所言说,\\}
\textbf{「当一切法都已摒除,一切言路也被摒除。」}

“Atthaṅgatassa na pamāṇam atthi, \textit{(Upasīvā ti Bhagavā)} yena naṃ vajjuṃ taṃ tassa natthi;\\
sabbesu dhammesu samohatesu, samūhatā vādapathā pi sabbe” ti. %\hfill\textcolor{gray}{\footnotesize 8}

\begin{enumerate}\item 于是,世尊向他显明如是不可言说性,而说此颂。\textbf{消逝者},即无取般涅槃者。\textbf{无法度量},即无法以色等度量。\textbf{有所言说},即以贪等谈论他。\textbf{一切法},即一切蕴等法。
\item 如是,世尊同样以阿罗汉为顶点开示了此经。当开示终了,与(先前)所说的一样,而有法的现观。\end{enumerate}

\begin{itemize}\item 案,原文 \textbf{samohatesu},这里从 PTS 本作 \textit{samūhatesu}。\end{itemize}

\begin{center}\vspace{1em}优波湿婆学童问第六\\Upasīvamāṇavapucchā chaṭṭhī.\end{center}