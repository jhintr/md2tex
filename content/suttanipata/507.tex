\section{优波湿婆学童问}

\subsection\*{\textbf{1076} {\footnotesize 〔PTS 1069〕}}

\textbf{「我独自、无依,释迦!」尊者优波湿婆说,「不能度过洪大的暴流,\\}
\textbf{「请说说所缘!一切眼者!依于此,我好度过这暴流。」}

\begin{enumerate}\item 这里,\textbf{无依},即无依于人或法。\textbf{所缘},即依止。\textbf{依于此},即依于此人或法。\end{enumerate}

\subsection\*{\textbf{1077} {\footnotesize 〔PTS 1070〕}}

\textbf{「觉察着无所有,具念,优波湿婆!」世尊说,「依于『这不存在』,你能度过暴流,\\}
\textbf{「舍弃了爱欲,戒离疑惑,昼夜寻求渴爱之灭尽!」}

\begin{enumerate}\item 现在,因为这婆罗门是得无所有处(定)者,且他不知道它的存在也是依止,因此,世尊为对他显示这是依止以及更高的出离之道,说了此颂。这里,\textbf{觉察},即具念而入于无所有处定后再出起,以无常等观察。\textbf{依于「这不存在」},即把以「什么都不存在」转起的定作为所缘。\textbf{你能度过暴流},即此后,以转起的毗婆舍那,你便能随适地度过四种暴流。\textbf{昼夜寻求渴爱之灭尽},即于日夜间,令涅槃明了而观察,以此对其论述现法乐住。\end{enumerate}

\subsection\*{\textbf{1078} {\footnotesize 〔PTS 1071〕}}

\textbf{「若于一切爱欲离贪,」尊者优波湿婆说,「依于无所有,舍弃了其它,\\}
\textbf{「解脱于最高的想之解脱,他能否住立于此,不再随行?」}

\begin{enumerate}\item 现在,听到「舍弃了爱欲」后,他看到自己以镇伏所舍弃的爱欲,说了此颂。这里,\textbf{舍弃了其它},即舍弃了自此以下的六种等至。\textbf{最高的想之解脱},即在七种想之解脱中最高的无所有处。\textbf{他能否住立于此,不再随行},即是问,此人能否不偏离于此无所有处的梵界而住立。\end{enumerate}

\subsection\*{\textbf{1079} {\footnotesize 〔PTS 1072〕}}

\textbf{「若于一切爱欲离贪,优波湿婆!」世尊说,「依于无所有,舍弃了其它,\\}
\textbf{「解脱于最高的想之解脱,他能住立于此,不再随行。」}

\begin{enumerate}\item 于是,世尊认可此处有六万劫之量,对他说了第三颂。\end{enumerate}

\subsection\*{\textbf{1080} {\footnotesize 〔PTS 1073〕}}

\textbf{「如果他能住立于此若许年,不再随行,一切眼者!\\}
\textbf{「他即于此处得清凉而解脱,对这样的人,识是否消灭?」}

\begin{enumerate}\item 如是,听到「住立于此」后,现在,为问其恒常断、断灭之相,说了此颂。这里,\textbf{若许年},即无数年,即众多、积聚之义。文本也作 pūgam pi vassānī,体格在此以格的转换用作属格,或当说 pūgam 之义为「许多 \textit{bahūni}」,又或读作 pūgāni——仍以最初的文本最佳。\textbf{他即于此处得清凉而解脱},即此人即于此无所有处解脱于种种苦、得证清凉,意即得证涅槃、成为恒常而住。\textbf{对这样的人,识是否消灭},即以「对这样的人,识是否无取而般涅槃」问断灭,亦或以「是否为了获取结生而存在」问其结生。\end{enumerate}

\subsection\*{\textbf{1081} {\footnotesize 〔PTS 1074〕}}

\textbf{「好比火焰为疾风所扰乱,优波湿婆!」世尊说,「消逝而不可得名,\\}
\textbf{「如是,牟尼解脱于名身,消逝而不可得名。」}

\begin{enumerate}\item 于是,世尊不取于断常,为对其显示于此生起的圣弟子的无取涅槃,说了此颂。这里,\textbf{不可得名},即不可归于俗称的「去到名为某某的方向」。\textbf{牟尼解脱于名身},即于此生起的有学牟尼,先前已以天性解脱于色身,于此转起第四道后,由遍知了法身故,又解脱于名身,成为俱分解脱的漏尽者,被称为无取般涅槃而消逝,不可称名为「刹帝利或婆罗门」等等。\end{enumerate}

\subsection\*{\textbf{1082} {\footnotesize 〔PTS 1075〕}}

\textbf{「这消逝,是说他不存在,还是说永远无病?\\}
\textbf{「牟尼!请对我善加解释!因为这法已如是为你所知。」}

\begin{enumerate}\item 现在,听到「消逝」后,他未能如理了解其义,说了此颂。其义为:\textbf{这消逝是不存在},还是它以恒常之相为\textbf{永远无病}、非变异法,如是,\textbf{牟尼!请对我善加解释}!什么原因?\textbf{因为这法已如是为你所知}。\end{enumerate}

\subsection\*{\textbf{1083} {\footnotesize 〔PTS 1076〕}}

\textbf{「消逝者无法度量,优波湿婆!」世尊说,「对他无法有所言说,\\}
\textbf{「当一切法都已铲除,一切言路也被铲除。」}

\begin{enumerate}\item 于是,世尊为对其显示如此的不可言说性,说了此颂。这里,\textbf{消逝者},即无取般涅槃者。\textbf{无法度量},即无法以色等度量。\textbf{有所言说},即以贪等言说他。\textbf{一切法},即一切蕴等法。其余一切处皆自明。
\item 如是,世尊仍以阿罗汉为顶点开示了此经。当开示终了,如前所述,而有法的现观。\end{enumerate}

