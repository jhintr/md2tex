\section{空王学童问}

\begin{center}Mogharāja Māṇava Pucchā\end{center}\vspace{1em}

\subsection\*{\textbf{1123} {\footnotesize 〔PTS 1116〕}}

\textbf{「我两次问了释迦,」尊者空王说,「具眼者没有对我解答,\\}
\textbf{「我听说,『到第三次,天仙便会解答』。}

“Dvāhaṃ Sakkaṃ apucchissaṃ, \textit{(icc āyasmā Mogharājā)} na me byākāsi cakkhumā;\\
yāvatatiyañ ca devīsi, byākarotī ti me sutaṃ. %\hfill\textcolor{gray}{\footnotesize 1}

\begin{enumerate}\item 因为他先前在阿耆多经及低舍弥勒经的最后,两次问了世尊。然而,世尊等待着他的根成熟而未回答,所以说「我两次问了释迦」。\textbf{天仙},即成为清净的天的仙人、世尊、等正觉者。据说他在乔陀婆利岸边如是听说。\end{enumerate}

\subsection\*{\textbf{1124} {\footnotesize 〔PTS 1117〕}}

\textbf{「此世、他世、俱有天的梵世间,\\}
\textbf{「不知道您,享有名望的乔达摩的见。}

Ayaṃ loko paro loko, brahmaloko sadevako;\\
diṭṭhiṃ te nābhijānāti, Gotamassa yasassino. %\hfill\textcolor{gray}{\footnotesize 2}

\begin{itemize}\item 案,\textbf{不知道} \textit{nābhijānāti} 的主语应是上半颂中作主格的「此世、他世、俱有天的梵世间」。PTS 本作「我不知道 \textit{nābhijānāmi}」,且将上半颂纳入引号,则意思变成了「我不知道您对这些世间的见」,二英译本正是如此,义注于此无文。若结合下二颂,似以 PTS 本为佳,兹存缅文本之旧,聊备一说。\end{itemize}

\subsection\*{\textbf{1125} {\footnotesize 〔PTS 1118〕}}

\textbf{「如是,我带着问题前往具希有之见者,\\}
\textbf{「如何观察世间,则死王不得见他?」}

Evaṃ abhikkantadassāviṃ, atthi pañhena āgamaṃ;\\
kathaṃ lokaṃ avekkhantaṃ, maccurājā na passati”. %\hfill\textcolor{gray}{\footnotesize 3}

\begin{enumerate}\item \textbf{具希有之见},即具最上之见,能见俱有天的世间的意乐、胜解、趣向、志向等。\end{enumerate}

\subsection\*{\textbf{1126} {\footnotesize 〔PTS 1119〕}}

\textbf{「你应从空观察世间,空王!始终具念,\\}
\textbf{「除去了我随见,如是便越过死亡,\\}
\textbf{「如是观察世间,则死王不得见他。」}

“Suññato lokaṃ avekkhassu, Mogharāja sadā sato;\\
attānudiṭṭhiṃ ūhacca, evaṃ maccutaro siyā;\\
evaṃ lokaṃ avekkhantaṃ, maccurājā na passatī” ti. %\hfill\textcolor{gray}{\footnotesize 4}

\begin{enumerate}\item \textbf{你应从空观察世间},即以考察转起之无主或随观诸行空无等两种方式,从空看世间。\textbf{我随见},即有身见。
\item 如是,世尊同样以阿罗汉为顶点开示了此经。当开示终了,与(先前)所说的一样,而有法的现观。\end{enumerate}

\begin{center}\vspace{1em}空王学童问第十五\\Mogharājamāṇavapucchā pannarasamā.\end{center}