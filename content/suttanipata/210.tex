\section{起身经}

\begin{center}Uṭṭhāna Sutta\end{center}\vspace{1em}

\begin{enumerate}\item 缘起为何?一时,当世尊住舍卫国,晚上住在祇林大寺,晨朝则为比丘僧团围绕,入舍卫国行乞,沿东门出城后,便到鹿母讲堂昼住。据说,\end{enumerate}

\subsection\*{\textbf{334} {\footnotesize 〔PTS 331〕}}

\textbf{起身!坐正!睡眠对你们有何义利?\\}
\textbf{对于苦患、被箭射伤者,为何睡眠?}

Uṭṭhahatha nisīdatha, ko attho supitena vo;\\
āturānañ hi kā niddā, sallaviddhāna ruppataṃ. %\hfill\textcolor{gray}{\footnotesize 1}

\subsection\*{\textbf{335} {\footnotesize 〔PTS 332〕}}

\textbf{起身!坐正!为了寂静,努力而学!\\}
\textbf{莫让死王得知你们放逸,好愚弄入其彀中者!}

Uṭṭhahatha nisīdatha, daḷhaṃ sikkhatha santiyā;\\
mā vo pamatte viññāya, maccurājā amohayittha vasānuge. %\hfill\textcolor{gray}{\footnotesize 2}

\begin{enumerate}\item \textbf{寂静}有三种,究竟寂静、彼分寂静、世俗寂静,分别是涅槃、毗婆舍那、成见 \textit{diṭṭhigata} 的同义语,而这里是究竟寂静、涅槃的意思。\end{enumerate}

\subsection\*{\textbf{336} {\footnotesize 〔PTS 333〕}}

\textbf{那诸天与人系缚、希求而住立的\\}
\textbf{爱著,越过它!莫让你们的时机流逝!\\}
\textbf{因为错过时机者转入地狱,随而忧伤。}

Yāya devā manussā ca, sitā tiṭṭhanti atthikā;\\
tarath’etaṃ visattikaṃ, khaṇo vo mā upaccagā;\\
khaṇātītā hi socanti, nirayamhi samappitā. %\hfill\textcolor{gray}{\footnotesize 3}

\begin{enumerate}\item \textbf{时机},即行沙门法的时机。以无味之义而得称\textbf{地狱} \textit{nirassādaṭṭhena nirayasaññite}。\end{enumerate}

\begin{itemize}\item 案,此颂的「越过」与上颂的「入其彀中」相对。\end{itemize}

\subsection\*{\textbf{337} {\footnotesize 〔PTS 334〕}}

\textbf{放逸是尘垢,跟随放逸的放逸是尘垢,\\}
\textbf{以不放逸,以明,他能拔出自己的箭。}

Pamādo rajo pamādo, pamādānupatito rajo;\\
appamādena vijjāya, abbahe sallam attano ti. %\hfill\textcolor{gray}{\footnotesize 4}

\begin{enumerate}\item \textbf{明},即漏尽智。\end{enumerate}

\begin{center}\vspace{1em}起身经第十\\Uṭṭhānasuttaṃ dasamaṃ.\end{center}