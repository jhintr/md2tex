\section{起身经}

\begin{center}Uṭṭhāna Sutta\end{center}\vspace{1em}

\begin{enumerate}\item 缘起为何?一时,当世尊住舍卫国,晚上住在祇林寺,晨朝则为比丘僧团随从,入舍卫国行乞,沿东门出城后,便到鹿母讲堂昼住。据说,这是世尊的宿习,晚上住在祇林寺,便去鹿母讲堂昼住,若晚上住在鹿母讲堂,便去祇林昼住。为什么?为摄受二家族,且为显示大遍舍的功德。且鹿母讲堂的下层有五百尖顶内室,于中有五百比丘居住。当世尊住在讲堂下层时,众比丘出于对世尊的尊重,便不上到讲堂上层。
\item 而这天,世尊进入讲堂上层的尖顶内室,因此五百比丘便进入讲堂下层的五百内室。且他们全都是新学、新来,于此法律掉举、不逊,放任诸根。他们进入后,睡了午觉,黄昏起来后,便聚集在大露台,作种种利益之论,发出高声、大声:「今天在食堂你得了什么?你去了哪里?——朋友!我去了㤭萨罗国王的家,我去了给孤独的,那里有这般这般食物的样式。」
\item 世尊听闻此声,想「他们与我住时,尚且如是放逸,哎!不相应行者」,便想大目犍连长老前来。尊者大目犍连即刻了知世尊之心,便以神变前来,在足下礼拜。随后,世尊便告知他:「目犍连!你的这些同梵行者放逸,善哉!去警示他们!」「唯!尊者!」尊者大目犍连答了世尊,立刻入了水遍,以足趾晃动立于方寸之地上的大讲堂与其所建的区域,如大风晃动船只一般。然后,那些比丘便恐惧、发出惨叫,丢了各自的衣服,从四门逃窜。
\item 世尊为对他们显现自身,便如沿另一门进入香房般,他们见了世尊,便礼拜而立。世尊问:「诸比丘!你们为何恐惧?」他们便说:「尊者!这鹿母讲堂晃动。」「诸比丘!你们知道由谁吗?」「我们不知,尊者!」然后,世尊说到「诸比丘!为令像你们这样失念、不正知、住于放逸者生起悚惧,由目犍连晃动」,为向这些比丘开示法,便说了此经。\end{enumerate}

\subsection\*{\textbf{334} {\footnotesize 〔PTS 331〕}}

\textbf{起身!坐正!睡眠对你们有何义利?\\}
\textbf{对于苦患、被箭射伤者,为何睡眠?}

Uṭṭhahatha nisīdatha, ko attho supitena vo;\\
āturānañ hi kā niddā, sallaviddhāna ruppataṃ. %\hfill\textcolor{gray}{\footnotesize 1}

\begin{enumerate}\item 这里,\textbf{起身},即从坐而起、奋起、精进,莫要懈怠!\textbf{坐正},即结跏趺已,为从事业处而坐!\textbf{睡眠对你们有何义利},即睡眠对你们——为无取般涅槃而出家者——有何义利?因为不能以睡眠成就任何义利。\textbf{对于苦患、被箭射伤者,为何睡眠},即对于在一小部分身体内为生起的眼疾等病所苦患者,对于为铁箭、骨箭、牙箭、角箭、木箭等中的任一箭仅射伤一、二指节之量的人,都无有睡眠,而于此,对于为破坏了你们整个心身相续而生起的种种品类的烦恼疾病所苦患者,为何睡眠?且由为贪箭等五箭射穿内心,为被箭射伤。\end{enumerate}

\subsection\*{\textbf{335} {\footnotesize 〔PTS 332〕}}

\textbf{起身!坐正!为了寂静,努力修学!\\}
\textbf{莫让死王得知你们放逸,好愚弄入其彀中者!}

Uṭṭhahatha nisīdatha, daḷhaṃ sikkhatha santiyā;\\
mā vo pamatte viññāya, maccurājā amohayittha vasānuge. %\hfill\textcolor{gray}{\footnotesize 2}

\begin{enumerate}\item 如是说已,世尊为再次以更甚之量激励、警示这些比丘,说了此颂。于此,其旨趣与连结的释义为:诸比丘!你们如是为烦恼之箭射中,是时候醒来了。什么原因?诸比丘!此梵行为醍醐味,大师现前,而此前你们长时睡眠——在山上睡、在河边睡、在平处睡、在不平处睡,乃至在树梢睡——由未见圣谛故,所以,为了结此睡眠,起身!坐正!为了寂静,努力修学!
\item 这里,前句之义已如所述。而第二句中,\textbf{寂静}有三种,即究竟寂静、彼分寂静、世俗寂静,分别是涅槃、毗婆舍那、成见的同义语,而这里是指究竟寂静,即涅槃之意。所以即是说:为了涅槃,努力修学!应不弛懈、勇猛而修学!什么原因?\textbf{莫让死王得知你们放逸,好愚弄入其彀中者},莫让魔罗——以死王为异名者——如是了知你们「这些人放逸」,好愚弄入其彀中者,即是说你们莫要进入他的控制,好让他愚弄入其彀中者。\end{enumerate}

\subsection\*{\textbf{336} {\footnotesize 〔PTS 333〕}}

\textbf{诸天与人以之束缚、欲求而持存的\\}
\textbf{爱著,越过它!莫要让你们的时机流逝!\\}
\textbf{因为错过时机者被付诸地狱,随即忧伤。}

Yāya devā manussā ca, sitā tiṭṭhanti atthikā;\\
tarath’etaṃ visattikaṃ, khaṇo vo mā upaccagā;\\
khaṇātītā hi socanti, nirayamhi samappitā. %\hfill\textcolor{gray}{\footnotesize 3}

\begin{enumerate}\item 既然进入其彀中,「诸天与人……忧伤」。\textbf{诸天与人以之欲求}——即色声香味触之欲求,\textbf{束缚}、依止、紧随于色等\textbf{而持存}者,请\textbf{越过}、超越\textbf{它}——即于种种品类的境域由蔓延、绵延、广布而为\textbf{爱著}的对有与财之渴爱。\textbf{莫要让你们的时机流逝},这是你们行沙门法的时机,莫令空度!\textbf{因为}对于令这样的时机空度者,以及空度此时机者,他们\textbf{错过时机},\textbf{被付诸地狱,随即忧伤},住立于以无乐味之义而被称为地狱的四种苦处,以「我们竟未行善」等法而忧伤。\end{enumerate}

\subsection\*{\textbf{337} {\footnotesize 〔PTS 334〕}}

\textbf{放逸是尘垢,跟随放逸的放逸是尘垢,\\}
\textbf{以不放逸,以明,他能拔出自己的箭。}

Pamādo rajo pamādo, pamādānupatito rajo;\\
appamādena vijjāya, abbahe sallam attano ti. %\hfill\textcolor{gray}{\footnotesize 4}

\begin{enumerate}\item 如是,世尊激励、警示了这些比丘,现在,在呵责他们住于此放逸后,为敦促他们全体不放逸,说了此颂。
\item 这里,\textbf{放逸},略说即离念,它以心的垢秽之义为\textbf{尘垢}。跟随此放逸为\textbf{跟随放逸},由跟随放逸故,后后生起者仍是放逸,彼亦是\textbf{尘垢}。因为放逸无时是非尘垢。以此说明什么?你们莫过自信:「现在还年轻,我们将来会知晓。」因为在年轻时的放逸为尘垢,中年、老年时则由跟随放逸故,便为大尘垢、尘堆。好比家里一、二日的尘垢只是尘垢,而经年增长便为尘堆。
\item 如是情形下,早年精通佛语、余年行沙门法,或早年精通、中年听闻、晚年行沙门法的比丘,便非住放逸者,由随顺不放逸而行道故。而若于一切年时住于放逸,从事午睡和利益之论者,好比你们,其早年的放逸为尘垢,余年跟随放逸的大放逸便为大尘垢。
\item 如是呵责了彼等之住于放逸,为敦促不放逸而说「\textbf{以不放逸,以明,他能拔出自己的箭}」,其义为:因为如是,此放逸于一切时为尘垢,所以,以被称为不离念的不放逸,及以被称为诸漏灭尽之智的明,族姓子之智者便能拔除自己依于内心的贪等五种箭,即以阿罗汉为顶点完成了开示。在开示终了,这五百比丘生起悚惧,便作意于此法的开示,经省察、作观而住于阿罗汉。\end{enumerate}

\begin{center}\vspace{1em}起身经第十\\Uṭṭhānasuttaṃ dasamaṃ.\end{center}