\section{褐者学童问}

\subsection\*{\textbf{1127} {\footnotesize 〔PTS 1120〕}}

\textbf{「我已年迈、无力、失去光泽,」尊者褐者说,「眼不明净,听不安顺,\\}
\textbf{「莫让我在愚痴之间消逝!\\}
\textbf{「于此,请说我能了知的舍弃生老之法!」}

\begin{enumerate}\item 这里,\textbf{我已年迈、无力、失去光泽},据说,这婆罗门为老所迫,生已百二十岁,且无力,想「举足于此」,却迈到别处,且失去先前皮肤的光泽,因此说「我已年迈、无力、失去光泽」。\textbf{莫让我在愚痴之间消逝},即莫让我未证得你的法,便在无知中消逝。\textbf{于此,舍弃生老},即于此,您的足下,或于石支提上,请对我说我能了知的舍弃生老的涅槃法。\end{enumerate}

\subsection\*{\textbf{1128} {\footnotesize 〔PTS 1121〕}}

\textbf{「看到在色中遘难,褐者!」世尊说,「放逸的人们被色恼害,\\}
\textbf{「所以,褐者!你应不放逸,为了不再有,应舍弃色!」}

\begin{enumerate}\item 现在,因为褐者以关切身体,说了「我已年迈」一颂,因此,世尊为了他舍弃对身体的爱执,说了此颂。这里,\textbf{在色中},即以色为因、以色为缘。\textbf{遘难},即由业的原因等受伤害。\textbf{被色恼害},即仍以色为因,人们被眼疾等恼害、迫害。\end{enumerate}

\subsection\*{\textbf{1129} {\footnotesize 〔PTS 1122〕}}

\textbf{「四方、四维、上方、下方这十方,\\}
\textbf{「世间没有任何是你所未见、未闻、未觉、未知的,\\}
\textbf{「于此,请说我能了知的舍弃生老之法!」}

\begin{enumerate}\item 如是,虽然听了世尊所说的直至阿罗汉的行道,褐者由老迈无力之故,仍未证得殊胜,再以此颂称赞世尊,请求开示。\end{enumerate}

\subsection\*{\textbf{1130} {\footnotesize 〔PTS 1123〕}}

\textbf{「觉察着陷入渴爱的人们,褐者!」世尊说,「种种热恼,被老击溃,\\}
\textbf{「所以,褐者!你应不放逸,为了不再有,应舍弃渴爱!」}

\begin{enumerate}\item 于是,世尊为再对其显明直至阿罗汉的行道,说了此颂。其余一切处皆自明。
\item 世尊仍以阿罗汉为顶点开示了此经。当开示终了,褐者即住于阿那含果。据说,他无间地想到:「这样丰富而有辩才的开示,我的舅舅波婆利却没听到!」因此爱执的散乱而不能证得阿罗汉。然而,他的一千个萦发弟子都证得了阿罗汉,全都持着神变所成的衣钵,成了「来!比丘」。\end{enumerate}

\begin{center}褐者学童问第十六\end{center}