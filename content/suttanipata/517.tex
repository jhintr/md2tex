\section{褐者学童问}

\begin{center}Piṅgiya Māṇava Pucchā\end{center}\vspace{1em}

\subsection\*{\textbf{1127} {\footnotesize 〔PTS 1120〕}}

\textbf{「我已年迈、无力、失去颜色,」尊者褐者说,「眼不明净,听不安顺,\\}
\textbf{「莫让我在愚痴中逝去!\\}
\textbf{「于此,请说我能了知的舍弃生与老之法!」}

“Jiṇṇo’ham asmi abalo vītavaṇṇo, \textit{(icc āyasmā Piṅgiyo)} nettā na suddhā savanaṃ na phāsu;\\
māhaṃ nassaṃ momuho antarā va;\\
ācikkha dhammaṃ yam ahaṃ vijaññaṃ, jātijarāya idha vippahānaṃ”. %\hfill\textcolor{gray}{\footnotesize 1}

\begin{enumerate}\item \textbf{我已年迈、无力、失去颜色},据说这婆罗门为老所迫,生已百二十岁,且无力,想「去这里」,却迈步到别处,且失去先前皮肤的颜色。\textbf{莫让我在愚痴中逝去},莫让我未证得你的法,而在无知中逝去。\textbf{于此,舍弃生与老},即于您的足下,或于石支提,请对我说我能了知的舍弃生与老的涅槃法。\end{enumerate}

\subsection\*{\textbf{1128} {\footnotesize 〔PTS 1121〕}}

\textbf{「看到了在色中遭难,褐者!」世尊说,「放逸的人们被色恼害,\\}
\textbf{「所以,褐者!你应不放逸,为了不再有,应舍弃色!」}

“Disvāna rūpesu vihaññamāne, \textit{(Piṅgiyā ti Bhagavā)} ruppanti rūpesu janā pamattā;\\
tasmā tuvaṃ Piṅgiya appamatto, jahassu rūpaṃ apunabbhavāya”. %\hfill\textcolor{gray}{\footnotesize 2}

\begin{enumerate}\item 现在,因为褐者关切身体而说「我已年迈」,故世尊为了他舍弃对身体的爱执而说此颂。这里,\textbf{在色中},即以色为因、以色为缘。\textbf{遭难},即由业的原因等受伤害。\textbf{被色恼害},即以色为因,人们被眼疾等恼害、迫害。\end{enumerate}

\subsection\*{\textbf{1129} {\footnotesize 〔PTS 1122〕}}

\textbf{「四方、四维、上方、下方,这十方,\\}
\textbf{「世间没有任何是你所未见、未闻、未觉、未知的,\\}
\textbf{「于此,请说我能了知的舍弃生与老之法!」}

“Disā catasso vidisā catasso, uddhaṃ adho dasa disā imāyo;\\
na tuyhaṃ adiṭṭhaṃ asutaṃ amutaṃ, atho aviññātaṃ kiñcanam atthi loke;\\
ācikkha dhammaṃ yam ahaṃ vijaññaṃ, jātijarāya idha vippahānaṃ”. %\hfill\textcolor{gray}{\footnotesize 3}

\begin{enumerate}\item 如是,虽然听了世尊所说的直至阿罗汉的行道,褐者由于年迈无力,未证得殊胜,而再次以此颂赞叹世尊,请求开示。\end{enumerate}

\subsection\*{\textbf{1130} {\footnotesize 〔PTS 1123〕}}

\textbf{「觉察着陷入渴爱的人们,褐者!」世尊说,「热恼,被衰老击溃,\\}
\textbf{「所以,褐者!你应不放逸,为了不再有,应舍弃渴爱!」}

“Taṇhādhipanne manuje pekkhamāno, \textit{(Piṅgiyā ti Bhagavā)} santāpajāte jarasā parete;\\
tasmā tuvaṃ Piṅgiya appamatto, jahassu taṇhaṃ apunabbhavāyā” ti. %\hfill\textcolor{gray}{\footnotesize 4}

\begin{enumerate}\item 于是,世尊再次显明了直至阿罗汉的行道而说此颂。
\item 世尊同样以阿罗汉为顶点开示了此经。当开示终了,褐者即住于阿那含果。据说他无间地想到:「这样丰富而有辩才的开示,我的舅舅波婆利却没听到!」由此爱执的散乱而不能证得阿罗汉。然而,他的一千个萦发弟子都证得了阿罗汉,全都持着神变所成的衣钵,成了「来!比丘」。\end{enumerate}

\begin{center}\vspace{1em}褐者学童问第十六\\Piṅgiyamāṇavapucchā soḷasamā.\end{center}