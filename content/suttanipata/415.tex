\section{执杖经}

\begin{enumerate}\item 缘起为何?如正游行经的缘起中所说,释迦族和拘利族因为水而起纷争,世尊知晓后,想「亲族们起了争辩,噫!我去遮止他们」,便站在两军之间,说了此经。\end{enumerate}

\subsection\*{\textbf{942} {\footnotesize 〔PTS 935〕}}

\textbf{从执杖产生怖畏,请看人的纠纷!\\}
\textbf{我将宣说如我所经历的悚惧。}

\begin{enumerate}\item 这里,初颂之义为:凡是世间生起的现法或来世的怖畏,这一切\textbf{怖畏从执杖产生},从自己的恶行之因产生,在如是的情况下,\textbf{请看人的纠纷},请看这释迦族等的人彼此的纠纷、伤害、恼害!如是,在指责了这敌对、邪行道的人后,以显示自己的正行道,为令其生起其悚惧,而说「\textbf{我将宣说如我所经历的悚惧}」,意即先前尚是菩萨。\end{enumerate}

\subsection\*{\textbf{943} {\footnotesize 〔PTS 936〕}}

\textbf{看到颤栗的人类,好比少水中的鱼,\\}
\textbf{看到彼此的敌对,怖畏便进入了我。}

\begin{enumerate}\item 现在,为显示因以悚惧的品类,说了以下几颂。这里,\textbf{颤栗},即因渴爱等震动。\textbf{看到彼此的敌对},即看到种种有情与彼此敌对。\end{enumerate}

\subsection\*{\textbf{944} {\footnotesize 〔PTS 937〕}}

\textbf{世间无处坚实,一切方向动荡,\\}
\textbf{希求着自己的居处,我不见未被居住者。}

\begin{enumerate}\item \textbf{世间无处坚实},即从地狱开始,环绕周围的世间都不坚实,无有常等的坚实。\textbf{一切方向动荡},一切方向都为无常等震动。\textbf{希求着自己的居处},即希求着自己的庇护处。\textbf{我不见未被居住者},即我却不见一处未被老等占据者。\end{enumerate}

\subsection\*{\textbf{945} {\footnotesize 〔PTS 938〕}}

\textbf{但最后却看到敌对,我生起了不喜,\\}
\textbf{然后,我见到其中的箭,难以察觉、依附于心。}

\begin{enumerate}\item \textbf{但最后却看到敌对,我生起了不喜},即却在青春等的最后、终点、毁灭处,看到与老等敌对、心受打击的有情,我生起了不喜。\textbf{其中的箭},即这些有情中的贪等箭。\end{enumerate}

\subsection\*{\textbf{946} {\footnotesize 〔PTS 939〕}}

\textbf{被这箭射中者,四处彷徨,\\}
\textbf{拔出了这箭,他不再奔突、不再沉沦。}

\begin{enumerate}\item 若问:箭有何威力?「被这箭射中者……」。这里,\textbf{四处彷徨},即在一切恶行的方向、东方等的方向上奔突。\textbf{他不再奔突、不再沉沦},即他在这些方向不再奔突,且于四暴流不再沉沦。\end{enumerate}

\subsection\*{\textbf{947} {\footnotesize 〔PTS 940〕}}

\textbf{于此,众学被诵出,\\}
\textbf{凡在世间被结缚者,不应从事于其中,\\}
\textbf{突破了一切爱欲,应修学自己的涅槃。}

\begin{enumerate}\item 如是,当有情被大威力的箭射中,「于此,众学被诵出……」。其义为:\textbf{凡在世间},种种五欲因被贪求获得,被称为\textbf{结缚},或由长时习行故被称为结缚,\textbf{于此}因由,象学等多种\textbf{众学}被讨论或学习。且看!这世间多么放逸!因此,智者、族姓子\textbf{不应}投入于这些结缚或这些众学中,相反,他以见无常等,\textbf{突破了一切爱欲},唯\textbf{应修学自己的涅槃}。\end{enumerate}

\subsection\*{\textbf{948} {\footnotesize 〔PTS 941〕}}

\textbf{他应真实,不鲁莽,不伪善,去除诽谤,\\}
\textbf{不忿怒,牟尼应超越贪之恶与悭贪。}

\begin{enumerate}\item 现在,为显示涅槃所应修学者,说了以下几颂。这里,\textbf{真实},即具足言语真实、智真实与道真实。\textbf{去除诽谤},即舍弃诽谤。\textbf{悭贪},即悭吝。\end{enumerate}

\subsection\*{\textbf{949} {\footnotesize 〔PTS 942〕}}

\textbf{他应忍耐睡眠、倦怠、昏沉,不应放逸而住,\\}
\textbf{存意涅槃的人,不应住于傲慢。}

\begin{enumerate}\item \textbf{他应忍耐睡眠、倦怠、昏沉},即他应克服瞌睡、身的懒散、心的懒散这三法。\textbf{存意涅槃},即心倾向涅槃。\end{enumerate}

\subsection\*{\textbf{950} {\footnotesize 〔PTS 943〕}}

\textbf{不应堕入妄语,不应爱执于色,\\}
\textbf{且应遍知慢,应戒离暴力而行。}

\begin{enumerate}\item \textbf{暴力},即贪染者的贪行等类的暴力之行。\end{enumerate}

\subsection\*{\textbf{951} {\footnotesize 〔PTS 944〕}}

\textbf{不应喜于故旧,不应偏爱新者,\\}
\textbf{于正消逝者不应忧伤,不应束缚于钩牵。}

\begin{enumerate}\item \textbf{不应喜于故旧},即不应喜于过去的色等。\textbf{新者},即现在。\textbf{不应束缚于钩牵},即不应依止于渴爱。因为渴爱由色等钩牵故,而被称为钩牵。\end{enumerate}

\subsection\*{\textbf{952} {\footnotesize 〔PTS 945〕}}

\textbf{我说贪求为大暴流,我说渴望为奔流,\\}
\textbf{所缘为震动,而爱欲的泥沼难以超越。}

\begin{enumerate}\item 设问:因为什么,不应束缚于钩牵?「我说贪求……」。其义为:因为我\textbf{说}这被称为钩牵的渴爱由贪求色等为\textbf{贪求},更以下抛之义为\textbf{暴流},以奔赴之义为\textbf{奔流},以呢喃「这是我的、这是我的」之因为\textbf{渴望},以难释放之义为\textbf{所缘},以起震动为\textbf{震动}\footnote{震动:原文为 pakappanaṃ,PTS 本同,这里从义注的 pakampanaṃ 译出。},且其对于世间,以障碍之义及难可超越之义,\textbf{爱欲的泥沼难以超越}。
\item 或者,当说「不应束缚于钩牵」时,设问:什么是这钩牵?「我说贪求……」,则此颂的连结当知如是。这里,词的连结为:我说贪求为钩牵,同样,我说凡是大暴流者,我说奔流,我说渴望,我说震动,我说这在俱有天的世间难以超越的爱欲的泥沼(为钩牵)。\end{enumerate}

\subsection\*{\textbf{953} {\footnotesize 〔PTS 946〕}}

\textbf{牟尼不偏离真实,婆罗门立于高地,\\}
\textbf{他舍遣了一切,他实被称为寂静者。}

\begin{enumerate}\item 如是,不依止这贪求等方法的钩牵,「牟尼不偏离真实……」。其义为:\textbf{不偏离}先前所说的三种\textbf{真实}\footnote{三种真实:即第 948 颂注所说的「言语真实、智真实与道真实」。},以证得寂默得称「\textbf{牟尼}」,\textbf{婆罗门立于}涅槃的\textbf{高地},如\textbf{他}这样的\textbf{舍遣了一切}入处,\textbf{实被称为寂静者}。\end{enumerate}

\subsection\*{\textbf{954} {\footnotesize 〔PTS 947〕}}

\textbf{他实为知者,他通达诸明,了知了法而无依止,\\}
\textbf{他在世间举止正当,于此无所羡慕。}

\begin{enumerate}\item 更有「他实为知者……」。这里,\textbf{了知了法},即以无常等方法了知了有为法。\textbf{他在世间举止正当},即由舍弃了引起不当举止的烦恼,他在世间举止正当。\end{enumerate}

\subsection\*{\textbf{955} {\footnotesize 〔PTS 948〕}}

\textbf{若于此超越了爱欲,世间难以超越的染著,\\}
\textbf{他便不再忧伤,不再忧虑,截断了水流,没有束缚。}

\begin{enumerate}\item 而如是不羡慕者,「若于此超越了爱欲……」。这里,\textbf{染著},即超越了七种染著。\textbf{不再忧虑},即不再贪求。\end{enumerate}

\subsection\*{\textbf{956} {\footnotesize 〔PTS 949〕}}

\textbf{让先前的凋萎,你切莫有任何后来,\\}
\textbf{如果你不执取中间,你将寂静而行。}

\begin{enumerate}\item 所以,你们中若希望成为这样的人,我对他说「让先前的……」。这里,\textbf{先前},即就过去诸行生起的法,种种烦恼与过去的业。\textbf{你切莫有任何后来},即就将来诸行生起的法,莫有任何贪等。\textbf{如果你不执取中间},即如果你也不执取现在的色等法。\end{enumerate}

\subsection\*{\textbf{957} {\footnotesize 〔PTS 950〕}}

\textbf{于一切名色,没有执为我者,\\}
\textbf{且不因不存在而忧伤,他在世间便不衰损。}

\begin{enumerate}\item 如是,以「你将寂静而行」显示了阿罗汉的证得,现在,以称赞阿罗汉说了此后的几颂。这里,此颂的\textbf{执为我者},即引起我执者,或以「这是我的」所执取之物。\textbf{不衰损},即不趣于衰亡。\end{enumerate}

\subsection\*{\textbf{958} {\footnotesize 〔PTS 951〕}}

\textbf{他没有任何「这是我的」,抑或「他人的」,\\}
\textbf{他找不到执为我者,不忧伤「这不是我的」。}

\begin{enumerate}\item 还有「他没有任何……」。这里,\textbf{任何},即任何色等种种法。\end{enumerate}

\subsection\*{\textbf{959} {\footnotesize 〔PTS 952〕}}

\textbf{他不苛刻,不贪求,不动摇,于一切处平等,\\}
\textbf{当被问及,我说这即是不动摇者的利益。}

\begin{enumerate}\item 还有「他不苛刻……」。这里,\textbf{不苛刻}\footnote{不苛刻 \textit{aniṭṭhurī}:PTS 本经文同,但 PTS 本义注则作「不骄傲 \textit{anuddharī}」。},即不妒忌,有些也读作 aniddhurī。\textbf{于一切处平等},意即舍。这说的是什么?凡不以「这不是我的」而忧伤者,当我被问及此\textbf{不动摇}之人时,\textbf{我说}于此人有不苛刻、不贪求、不动摇、于一切处平等等四种\textbf{利益}。\end{enumerate}

\subsection\*{\textbf{960} {\footnotesize 〔PTS 953〕}}

\textbf{不动摇的了知者,已无任何的行作,\\}
\textbf{他戒离各种努力,于一切处见安稳。}

\begin{enumerate}\item 还有「不动摇的……」。这里,\textbf{行作},即福行等的任何行。因为它被积累,或积累,所以被称为「行作」。\textbf{各种努力},即各种福行等的努力。\textbf{于一切处见安稳},即于一切处唯见无畏。\end{enumerate}

\subsection\*{\textbf{961} {\footnotesize 〔PTS 954〕}}

\textbf{牟尼不说同等、下等、上等,\\}
\textbf{他寂静,离于悭吝,不执取,不扬弃。}

\begin{enumerate}\item 如是见者,「牟尼不说……」。这里,\textbf{不说},即不以「我是相同」等慢说自己同等、下等、上等。\textbf{不执取,不扬弃},即不取色等中的任何法,也不扬弃。其余一切处皆自明。
\item 如是,以阿罗汉为顶点完成了开示。当开示终了,五百释迦童子和拘利童子以「来!比丘」出家,世尊便带上他们,入于大林。\end{enumerate}

