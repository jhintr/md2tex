\section{执杖经}

\begin{center}Attadaṇḍa Sutta\end{center}\vspace{1em}

\begin{enumerate}\item 如正游行经的缘起中所说,释迦族和拘利族因为水而起纷争,世尊知晓后,想「亲族们起了争执,那么,让我去遮止他们」,站在两军之间,便说了此经。\end{enumerate}

\subsection\*{\textbf{942} {\footnotesize 〔PTS 935〕}}

\textbf{从执杖产生怖畏,请看人的纠纷!\\}
\textbf{我将宣说如我所经历的悚惧。}

“Attadaṇḍā bhayaṃ jātaṃ, janaṃ passatha medhagaṃ;\\
saṃvegaṃ kittayissāmi, yathā saṃvijitaṃ mayā. %\hfill\textcolor{gray}{\footnotesize 1}

\begin{enumerate}\item 凡在世间生起的现法或来世的怖畏,这一切\textbf{怖畏从执杖产生},从自己的恶行产生。在如是的情况下,\textbf{请看人的纠纷},请看这释迦族等的人彼此的纠纷、伤害、恼害。如是,在谴责了这敌对、邪行道的人后,以显示自己的正行道,为令生起其悚惧,而说\textbf{我将宣说如我所经历的悚惧},即先前尚是菩萨(所经历)的意思。\end{enumerate}

\subsection\*{\textbf{943} {\footnotesize 〔PTS 936〕}}

\textbf{看到颤栗的人类,好比少水中的鱼,\\}
\textbf{看到彼此的敌对,怖畏便进入了我。}

Phandamānaṃ pajaṃ disvā, macche appodake yathā;\\
aññamaññehi byāruddhe, disvā maṃ bhayam āvisi. %\hfill\textcolor{gray}{\footnotesize 2}

\subsection\*{\textbf{944} {\footnotesize 〔PTS 937〕}}

\textbf{世间无处坚实,一切方向动荡,\\}
\textbf{希求着自己的居处,我却见不到未被占据者。}

Samantam asāro loko, disā sabbā sameritā;\\
icchaṃ bhavanam attano, nāddasāsiṃ anositaṃ. %\hfill\textcolor{gray}{\footnotesize 3}

\begin{enumerate}\item \textbf{世间无处坚实},从地狱开始,所有世间都不坚实,无有常等的坚实。\textbf{一切方向动荡},一切方向都为无常等震动。\textbf{希求着自己的居处},希求着自己的庇护处。\textbf{我却见不到未被占据者},我却见不到有一处未被老等占据者。\end{enumerate}

\subsection\*{\textbf{945} {\footnotesize 〔PTS 938〕}}

\textbf{但最后却看到敌对,我生起了不喜,\\}
\textbf{然后,于此我见到了箭,难以察觉、依附于心。}

Osāne tv-eva byāruddhe, disvā me aratī ahu;\\
ath’ettha sallam addakkhiṃ, duddasaṃ hadayanissitaṃ. %\hfill\textcolor{gray}{\footnotesize 4}

\begin{enumerate}\item \textbf{但最后却看到敌对,我生起了不喜},但在青春等的最后、终点、毁灭处,却看到与老等敌对、心受打击的有情,我生起了不喜。\textbf{于此},于此等有情处。\textbf{箭},即贪等箭。\end{enumerate}

\subsection\*{\textbf{946} {\footnotesize 〔PTS 939〕}}

\textbf{被这箭射中者,四处彷徨,\\}
\textbf{拔出了这箭,他不再奔突、不再沉沦。}

Yena sallena otiṇṇo, disā sabbā vidhāvati;\\
tam eva sallam abbuyha, na dhāvati na sīdati. %\hfill\textcolor{gray}{\footnotesize 5}

\begin{enumerate}\item \textbf{四处彷徨},在一切恶行的方向、东方等的方向上奔突。\textbf{不再沉沦},于四暴流不沉沦。\end{enumerate}

\subsection\*{\textbf{947} {\footnotesize 〔PTS 940〕}}

\textbf{于此,赞颂众学,\\}
\textbf{凡结缚于世间者,不应从事其中,\\}
\textbf{突破了一切爱欲,应修学自己的涅槃。}

Tattha sikkhānugīyanti;\\
yāni loke gadhitāni, na tesu pasuto siyā;\\
nibbijjha sabbaso kāme, sikkhe nibbānam attano. %\hfill\textcolor{gray}{\footnotesize 6}

\begin{enumerate}\item \textbf{于世间},五欲以为人所贪求获得,故被称为\textbf{结缚},或由长时习行故被称为结缚,\textbf{于此}因由,象学等多种\textbf{众学}被讨论或学习。\end{enumerate}

\begin{itemize}\item 案,首句费解,Norman 说义注的两种解释,即「讨论或学习」,说明句子的意思不明朗,但无论如何,首句的意思与全颂并不相关,因而建议原本是对念诵者的指示,很早就混进了原文,因为无论是义释还是义注都进行了解释。\end{itemize}

\subsection\*{\textbf{948} {\footnotesize 〔PTS 941〕}}

\textbf{他应真实,不鲁莽,不伪善,无有诽谤,\\}
\textbf{无忿怒,牟尼应超越贪之恶与悭贪。}

Sacco siyā appagabbho, amāyo rittapesuṇo;\\
akkodhano lobhapāpaṃ, vevicchaṃ vitare muni. %\hfill\textcolor{gray}{\footnotesize 7}

\begin{enumerate}\item \textbf{真实},即具足言语真实、智真实、道真实。\end{enumerate}

\subsection\*{\textbf{949} {\footnotesize 〔PTS 942〕}}

\textbf{他应忍耐睡眠、倦怠、昏沉,不应放逸而活,\\}
\textbf{存意涅槃的人,不应住于傲慢。}

Niddaṃ tandiṃ sahe thīnaṃ, pamādena na saṃvase;\\
atimāne na tiṭṭheyya, nibbānamanaso naro. %\hfill\textcolor{gray}{\footnotesize 8}

\begin{enumerate}\item \textbf{他应忍耐睡眠、倦怠、昏沉},他应克服动摇、身懈怠、心懈怠等三法。\textbf{存意涅槃},即心倾向涅槃。\end{enumerate}

\subsection\*{\textbf{950} {\footnotesize 〔PTS 943〕}}

\textbf{不应堕入妄语,不应爱执于色,\\}
\textbf{且应遍知慢,应离于暴力而行。}

Mosavajje na nīyetha, rūpe snehaṃ na kubbaye;\\
mānañ ca parijāneyya, sāhasā virato care. %\hfill\textcolor{gray}{\footnotesize 9}

\subsection\*{\textbf{951} {\footnotesize 〔PTS 944〕}}

\textbf{不应喜于故旧,不应期望新者,\\}
\textbf{于正消逝者不应忧伤,不应系缚于钩牵。}

Purāṇaṃ nābhinandeyya, nave khantiṃ na kubbaye;\\
hiyyamāne na soceyya, ākāsaṃ na sito siyā. %\hfill\textcolor{gray}{\footnotesize 10}

\begin{enumerate}\item \textbf{不应系缚于钩牵},不应依止于渴爱,因为渴爱由色等钩牵,故被称为钩牵。\end{enumerate}

\subsection\*{\textbf{952} {\footnotesize 〔PTS 945〕}}

\textbf{我说贪求为「大暴流」,我说渴望为奔流,\\}
\textbf{所缘为震动,而爱欲的泥沼难以超越。}

Gedhaṃ brūmi ‘mahogho’ ti, ājavaṃ brūmi jappanaṃ;\\
ārammaṇaṃ pakappanaṃ, kāmapaṅko duraccayo. %\hfill\textcolor{gray}{\footnotesize 11}

\begin{itemize}\item 案,PTS 本也作 \textbf{pakappanaṃ},这里从义注的 \textit{pakampanaṃ} 译出。\end{itemize}

\subsection\*{\textbf{953} {\footnotesize 〔PTS 946〕}}

\textbf{牟尼不离于真实,婆罗门立于高地,\\}
\textbf{已舍弃了一切,他实被称为寂静者。}

Saccā avokkamma muni, thale tiṭṭhati brāhmaṇo;\\
sabbaṃ so paṭinissajja, sa ve santo ti vuccati. %\hfill\textcolor{gray}{\footnotesize 12}

\subsection\*{\textbf{954} {\footnotesize 〔PTS 947〕}}

\textbf{他实为智者,他通达诸明,了知了法而无所依,\\}
\textbf{他在世间举止正当,于此无所羡慕。}

Sa ve vidvā sa vedagū, ñatvā dhammaṃ anissito;\\
sammā so loke iriyāno, na pihetīdha kassaci. %\hfill\textcolor{gray}{\footnotesize 13}

\begin{enumerate}\item \textbf{了知了法},即以无常等了知了有为法。\end{enumerate}

\subsection\*{\textbf{955} {\footnotesize 〔PTS 948〕}}

\textbf{若于此超越了爱欲,世间难以超越的染著,\\}
\textbf{他便不再忧伤,不再忧恼,截断了水流,无所束缚。}

Yo’dha kāme accatari, saṅgaṃ loke duraccayaṃ;\\
na so socati nājjheti, chinnasoto abandhano. %\hfill\textcolor{gray}{\footnotesize 14}

\subsection\*{\textbf{956} {\footnotesize 〔PTS 949〕}}

\textbf{让先前的凋萎,你切莫有任何后来,\\}
\textbf{如果你不执取中间,你将寂静而行。}

Yaṃ pubbe taṃ visosehi, pacchā te māhu kiñcanaṃ;\\
majjhe ce no gahessasi, upasanto carissasi. %\hfill\textcolor{gray}{\footnotesize 15}

\subsection\*{\textbf{957} {\footnotesize 〔PTS 950〕}}

\textbf{于一切名色,不执为我所,\\}
\textbf{也不因不存在者而忧伤,他在世间便无衰损。}

Sabbaso nāmarūpasmiṃ, yassa natthi mamāyitaṃ;\\
asatā ca na socati, sa ve loke na jīyati. %\hfill\textcolor{gray}{\footnotesize 16}

\subsection\*{\textbf{958} {\footnotesize 〔PTS 951〕}}

\textbf{他没有任何「这是我的」,抑或「他人的」,\\}
\textbf{他找不到执为我者,不忧伤「这不是我的」。}

Yassa natthi ‘idaṃ me’ ti, ‘paresaṃ’ vā pi kiñcanaṃ;\\
mamattaṃ so asaṃvindaṃ, ‘natthi me’ ti na socati. %\hfill\textcolor{gray}{\footnotesize 17}

\subsection\*{\textbf{959} {\footnotesize 〔PTS 952〕}}

\textbf{他不粗砺,不贪求,不动摇,于一切处平等,\\}
\textbf{当被问及,我说这即是不动摇者的功德。}

Aniṭṭhurī ananugiddho, anejo sabbadhī samo;\\
tam ānisaṃsaṃ pabrūmi, pucchito avikampinaṃ. %\hfill\textcolor{gray}{\footnotesize 18}

\begin{itemize}\item 案,\textbf{不粗砺} \textit{Aniṭṭhurī},英译作 harsh/bitter,义注给出另读 \textit{Aniddhurī}。\end{itemize}

\subsection\*{\textbf{960} {\footnotesize 〔PTS 953〕}}

\textbf{不动摇的了知者,已无任何的行作,\\}
\textbf{他离于种种努力,于一切处见安稳。}

Anejassa vijānato, natthi kāci nisaṅkhati;\\
virato so viyārabbhā, khemaṃ passati sabbadhi. %\hfill\textcolor{gray}{\footnotesize 19}

\begin{enumerate}\item \textbf{行作},即福行等的任何行。\textbf{种种努力},即种种福行等的努力。\end{enumerate}

\begin{itemize}\item 案,\textbf{行作} \textit{nisaṅkhati},义注同,英译及词典均作 \textit{nisaṅkhiti}。\end{itemize}

\subsection\*{\textbf{961} {\footnotesize 〔PTS 954〕}}

\textbf{牟尼不说(自己)同等、下等、上等,\\}
\textbf{他寂静,离于悭吝,不执取,不扬弃。}

Na samesu na omesu, na ussesu vadate muni;\\
santo so vītamaccharo, nādeti na nirassatī” ti. %\hfill\textcolor{gray}{\footnotesize 20}

\begin{enumerate}\item 如是,以阿罗汉为顶点而完成了开示。当开示终了,五百释迦童子和拘利童子以「来,比丘」的方式出家,世尊摄受彼等已,即入于大林。\end{enumerate}

\begin{center}\vspace{1em}执杖经第十五\\Attadaṇḍasuttaṃ pannarasamaṃ.\end{center}