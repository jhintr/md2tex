\section{迅速经}

\begin{center}Tuvaṭaka Sutta\end{center}\vspace{1em}

\begin{enumerate}\item 亦在此大集会中,有些天人生起「那么,什么是导向证得阿罗汉的行道」之心,为显明其义,以如前所述的方法,使相佛问自己问题后而说。\end{enumerate}

\subsection\*{\textbf{922} {\footnotesize 〔PTS 915〕}}

\textbf{「我问你,日种!远离以及寂静的境地,大仙!\\}
\textbf{「如何见后,比丘即涅槃,于此世间都无所取?」}

“Pucchāmi taṃ Ādiccabandhu, vivekaṃ santipadañ ca Mahesi;\\
kathaṃ disvā nibbāti bhikkhu, anupādiyāno lokasmiṃ kiñci”. %\hfill\textcolor{gray}{\footnotesize 1}

\subsection\*{\textbf{923} {\footnotesize 〔PTS 916〕}}

\textbf{「名为戏论的根本,」世尊说,「『我是』等的一切,他应以智慧止息,\\}
\textbf{「凡是内在的渴爱,调伏彼等已,应始终具念而修学。}

“Mūlaṃ papañcasaṅkhāya, \textit{(iti Bhagavā)} mantā ‘asmī’ ti sabbam uparundhe;\\
yā kāci taṇhā ajjhattaṃ, tāsaṃ vinayā sadā sato sikkhe. %\hfill\textcolor{gray}{\footnotesize 2}

\begin{enumerate}\item 现在,因为如是见者能止息烦恼,转起此见已即涅槃,所以世尊为阐明此义,以种种行相激励此天人众以舍弃烦恼,而说了以下的五颂。由被称为戏论,戏论即是\textbf{名为戏论},其\textbf{根本}即无明等烦恼,\textbf{他应以智慧止息}这名为戏论的根本以及以「\textbf{我是}」转起的慢\textbf{等的一切}。\end{enumerate}

\subsection\*{\textbf{924} {\footnotesize 〔PTS 917〕}}

\textbf{「任何他所证知的法,内在的或外在的,\\}
\textbf{「不应以此固执,因为这不是善人们所说的寂灭。}

Yaṃ kiñci dhammam abhijaññā, ajjhattaṃ atha vā pi bahiddhā;\\
na tena thāmaṃ kubbetha, na hi sā nibbuti sataṃ vuttā. %\hfill\textcolor{gray}{\footnotesize 3}

\begin{enumerate}\item 如是,在第一颂中以阿罗汉为顶点显明了与三学有关的开示后,又为开示舍弃慢而说此颂。\textbf{任何他所证知的法,内在的},即任何他所了知的出生高贵等自身的功德。\textbf{或外在的},或所了知的外在的阿阇黎、亲教师们的功德。\end{enumerate}

\subsection\*{\textbf{925} {\footnotesize 〔PTS 918〕}}

\textbf{「他不应以此认为更好、更劣,或是相同,\\}
\textbf{「为种种形相所触时,他不应将自己定位。}

Seyyo na tena maññeyya, nīceyyo atha vā pi sarikkho;\\
phuṭṭho anekarūpehi, nātumānaṃ vikappayaṃ tiṭṭhe. %\hfill\textcolor{gray}{\footnotesize 4}

\subsection\*{\textbf{926} {\footnotesize 〔PTS 919〕}}

\textbf{「他唯应寂止其内在,比丘不应从别处寻求寂静,\\}
\textbf{「对于内在寂静者,没有接受,又何来丢弃?}

Ajjhattam ev’upasame, na aññato bhikkhu santim eseyya;\\
ajjhattaṃ upasantassa, natthi attā kuto nirattā vā. %\hfill\textcolor{gray}{\footnotesize 5}

\begin{enumerate}\item 如是显明了舍弃慢后,现在为开示止息一切烦恼而说此颂。\textbf{他唯应寂止其内在},即他唯应寂止其自身的贪等一切烦恼。\textbf{比丘不应从别处寻求寂静},即除了念处等,不应以别的方法寻求寂静。\end{enumerate}

\subsection\*{\textbf{927} {\footnotesize 〔PTS 920〕}}

\textbf{「好比在大海中间,不起波浪而住立,\\}
\textbf{「如是他应住立不动,比丘不应生起任何增盛。」}

Majjhe yathā samuddassa, ūmi no jāyatī ṭhito hoti;\\
evaṃ ṭhito anej’assa, ussadaṃ bhikkhu na kareyya kuhiñci”. %\hfill\textcolor{gray}{\footnotesize 6}

\begin{enumerate}\item 现在,为显明内在寂静的漏尽者的如性而说此颂。\textbf{好比在大海中间},在四千由旬之量的被称为上下分的中间,或住立于山间的大海的中间,\textbf{不起波浪而住立}、不动摇,\textbf{如是},\textbf{不动}的漏尽者于得(失)等\textbf{应住立}不动摇,这样的\textbf{比丘不应生起任何}贪等的\textbf{增盛}。\end{enumerate}

\subsection\*{\textbf{928} {\footnotesize 〔PTS 921〕}}

\textbf{「开眼者已解说了调伏危难的亲证之法,\\}
\textbf{「请宣说行道,大德!波罗提木叉以及三摩地!」}

“Akittayī Vivaṭacakkhu, sakkhi dhammaṃ parissayavinayaṃ;\\
paṭipadaṃ vadehi Bhaddante, pātimokkhaṃ atha vāpi samādhiṃ”. %\hfill\textcolor{gray}{\footnotesize 7}

\begin{enumerate}\item 现在,相佛随喜以阿罗汉为顶点的法的开示,并问到这阿罗汉性的初行道,而说此颂。\textbf{开眼者},即具足开敞、无碍的五眼。\textbf{亲证之法},即自身证知、自己现量之法。\textbf{大德},是以「愿您吉祥」称呼世尊,或说是「请宣说您吉祥、调柔的行道」。\end{enumerate}

\subsection\*{\textbf{929} {\footnotesize 〔PTS 922〕}}

\textbf{「切勿以眼贪求,于村谈应遮耳,\\}
\textbf{「不应贪求众味,也不应执取世间任何为我所。}

“Cakkhūhi n’eva lol’assa, gāmakathāya āvaraye sotaṃ;\\
rase ca nānugijjheyya, na ca mamāyetha kiñci lokasmiṃ. %\hfill\textcolor{gray}{\footnotesize 8}

\subsection\*{\textbf{930} {\footnotesize 〔PTS 923〕}}

\textbf{「当他为触所触时,比丘不应起任何悲伤,\\}
\textbf{「不应渴望有,于诸恐怖也不应震颤。}

Phassena yadā phuṭṭh’assa, paridevaṃ bhikkhu na kareyya kuhiñci;\\
bhavañ ca nābhijappeyya, bheravesu ca na sampavedheyya. %\hfill\textcolor{gray}{\footnotesize 9}

\begin{enumerate}\item \textbf{触},即疾病之触。\textbf{不应渴望有},不应为驱除此触而渴望欲有等的有。\textbf{于诸恐怖也不应震颤},于此触之缘的狮虎等诸恐怖也不应震颤,或不应震颤于其余的鼻根、意根之境,如是即圆满了根律仪。或说先显示了根律仪,而以此显示「住于林野者见、闻恐怖已,不应震颤」。\end{enumerate}

\subsection\*{\textbf{931} {\footnotesize 〔PTS 924〕}}

\textbf{「然后,食物、饮品、硬食、衣服等,\\}
\textbf{「获得后不应积贮,未获得这些时也不应恐惧。}

Annānam atho pānānaṃ, khādanīyānaṃ atho pi vatthānaṃ;\\
laddhā na sannidhiṃ kayirā, na ca parittase tāni alabhamāno. %\hfill\textcolor{gray}{\footnotesize 10}

\subsection\*{\textbf{932} {\footnotesize 〔PTS 925〕}}

\textbf{「应禅修,不应游步,应远离恶作,不应放逸,\\}
\textbf{「然后,比丘应居于安静的坐卧处。}

Jhāyī na pādalol’assa, virame kukkuccā nappamajjeyya;\\
ath’āsanesu sayanesu, appasaddesu bhikkhu vihareyya. %\hfill\textcolor{gray}{\footnotesize 11}

\subsection\*{\textbf{933} {\footnotesize 〔PTS 926〕}}

\textbf{「不应多睡眠,应保持警觉,热忱,\\}
\textbf{「应舍弃倦怠、伪善、戏笑、嬉戏、淫欲以及严饰。}

Niddaṃ na bahulīkareyya, jāgariyaṃ bhajeyya ātāpī;\\
tandiṃ māyaṃ hassaṃ khiḍḍaṃ, methunaṃ vippajahe savibhūsaṃ. %\hfill\textcolor{gray}{\footnotesize 12}

\subsection\*{\textbf{934} {\footnotesize 〔PTS 927〕}}

\textbf{「不应使用阿闼婆、梦、相,以及星占,\\}
\textbf{「我的弟子不应从事鸣声、助孕、医疗。}

Āthabbaṇaṃ supinaṃ lakkhaṇaṃ, no vidahe atho pi nakkhattaṃ;\\
virutañ ca gabbhakaraṇaṃ, tikicchaṃ māmako na seveyya. %\hfill\textcolor{gray}{\footnotesize 13}

\begin{enumerate}\item \textbf{阿闼婆},即使用阿闼婆咒语。\textbf{梦},即占梦术。\textbf{相},即摩尼之相等。\textbf{鸣声},即(解释)鹿等的鸣声。\end{enumerate}

\subsection\*{\textbf{935} {\footnotesize 〔PTS 928〕}}

\textbf{「比丘不应震颤于责备,被赞赏时不应骄傲,\\}
\textbf{「应去除贪以及悭吝、忿怒与诽谤。}

Nindāya nappavedheyya, na uṇṇameyya pasaṃsito bhikkhu;\\
lobhaṃ saha macchariyena, kodhaṃ pesuṇiyañ ca panudeyya. %\hfill\textcolor{gray}{\footnotesize 14}

\subsection\*{\textbf{936} {\footnotesize 〔PTS 929〕}}

\textbf{「比丘不应从事买卖,在任何处不应引来责难,\\}
\textbf{「且在村中不应冒犯,不应为好乐利养而与人闲谈。}

Kayavikkaye na tiṭṭheyya, upavādaṃ bhikkhu na kareyya kuhiñci;\\
gāme ca nābhisajjeyya, lābhakamyā janaṃ na lapayeyya. %\hfill\textcolor{gray}{\footnotesize 15}

\subsection\*{\textbf{937} {\footnotesize 〔PTS 930〕}}

\textbf{「比丘既不应夸耀,也不应说有企图的话,\\}
\textbf{「不应变得鲁莽,不应谈论争议之事。}

Na ca katthitā siyā bhikkhu, na ca vācaṃ payuttaṃ bhāseyya;\\
pāgabbhiyaṃ na sikkheyya, kathaṃ viggāhikaṃ na kathayeyya. %\hfill\textcolor{gray}{\footnotesize 16}

\subsection\*{\textbf{938} {\footnotesize 〔PTS 931〕}}

\textbf{「不应堕入妄语,不应知而狡诈,\\}
\textbf{「且不应以活命、慧、戒禁鄙视他人。}

Mosavajje na nīyetha, sampajāno saṭhāni na kayirā;\\
atha jīvitena paññāya, sīlabbatena nāññam atimaññe. %\hfill\textcolor{gray}{\footnotesize 17}

\subsection\*{\textbf{939} {\footnotesize 〔PTS 932〕}}

\textbf{「当被激怒时,听到沙门或凡夫们的众多言语后,\\}
\textbf{「不应以恶口回答他们,因为善人们不制造敌对。}

Sutvā rusito bahuṃ vācaṃ, samaṇānaṃ vā puthujanānaṃ;\\
pharusena ne na paṭivajjā, na hi santo paṭisenikaronti. %\hfill\textcolor{gray}{\footnotesize 18}

\subsection\*{\textbf{940} {\footnotesize 〔PTS 933〕}}

\textbf{「知晓了这法,比丘审视着,应始终具念而修学,\\}
\textbf{「了知了『寂静』之为寂灭,于乔达摩的教法中不应放逸。}

Etañ ca dhammam aññāya, vicinaṃ bhikkhu sadā sato sikkhe;\\
‘santī’ ti nibbutiṃ ñatvā, sāsane Gotamassa na pamajjeyya. %\hfill\textcolor{gray}{\footnotesize 19}

\subsection\*{\textbf{941} {\footnotesize 〔PTS 934〕}}

\textbf{「因为他是征服者,不可征服,得见亲证之法,而非基于传闻,\\}
\textbf{「所以,在彼世尊的教法内不放逸,始终礼敬而随学。」}

Abhibhū hi so anabhibhūto, sakkhi dhammam anītiham adassī;\\
tasmā hi tassa Bhagavato sāsane, appamatto sadā namassam anusikkhe” ti. %\hfill\textcolor{gray}{\footnotesize 20}

\begin{enumerate}\item 总的来说,这里以「切勿以眼贪求」等说根律仪,以拒绝积贮「食物、饮品」等说资具受用戒,以淫欲、妄语、诽谤等说别解脱律仪戒,以「阿闼婆、梦、相」等说活命遍净戒,以「应禅修」说定,以「比丘审视着」说慧,以「应始终具念而修学」再次略说三学,以「比丘应居于安静的坐卧处,不应多睡眠」等说戒定慧的摄受资益与除遣伤害。\end{enumerate}

\begin{itemize}\item 菩提比丘:显然,第一句中的\textbf{他}指的是世尊,而非比丘。\end{itemize}

\begin{center}\vspace{1em}迅速经第十四\\Tuvaṭakasuttaṃ cuddasamaṃ.\end{center}