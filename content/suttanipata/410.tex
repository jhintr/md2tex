\section{前分离经}

\begin{center}Purābheda Sutta\end{center}\vspace{1em}

\begin{enumerate}\item 缘起为何?此经及随后的争辩争论、小阵、大阵、迅速、执杖等五经的缘起,由与正游行经的缘起中述说之法相同故,即如所述。而差别之处在于,正如在那大集会中,因顺适贪行者的众天人而说法,让「相佛」问自己问题后,说了正游行经,如是,仍在那大集会中,了知了生起「在身体分离之前,应当做些什么」之心的天人的心后,为了摄受彼等,从空中引来为一千二百五十比丘围绕的相佛,让他问自己问题后,说了此经。\end{enumerate}

\subsection\*{\textbf{855} {\footnotesize 〔PTS 848〕}}

\textbf{「何等知见、何等戒,才能被称为寂静?\\}
\textbf{「乔达摩!请告诉我!当被问到这最上之人!」}

“Kathaṃdassī kathaṃsīlo, upasanto ti vuccati;\\
taṃ me Gotama pabrūhi, pucchito uttamaṃ naraṃ”. %\hfill\textcolor{gray}{\footnotesize 1}

\begin{enumerate}\item 这里,先说在提问中,这相(佛)以\textbf{何等知见}问增上慧,以\textbf{何等戒}问增上戒,以\textbf{寂静}问增上心。其余自明。\end{enumerate}

\subsection\*{\textbf{856} {\footnotesize 〔PTS 849〕}}

\textbf{「在分离之前离爱,」世尊说,「不依止前端,\\}
\textbf{「在中间不可估量,他没有预设。}

“Vītataṇho purā bhedā, \textit{(iti Bhagavā)} pubbam antam anissito;\\
vemajjhe n’upasaṅkheyyo, tassa natthi purakkhataṃ. %\hfill\textcolor{gray}{\footnotesize 2}

\begin{enumerate}\item 而在解答中,世尊却未同样地解答增上慧等,而是因增上慧等的威力,彼等烦恼的寂止被称为「寂静」,便以随顺种种天人的意乐显明彼等的寂止,说了以此颂为首的诸颂。这里,当知此颂为首的八颂与「我说他为寂静\footnote{即第 864 颂}」一颂相连,随后的几颂则与末句「他实被称为寂静者\footnote{即第 868 颂。}」相连。
\item 而以逐词解释之法,\textbf{在分离之前离爱},即在身体分离之前,舍弃渴爱。\textbf{不依止前端},即不依止过去时等的前端。\textbf{在中间不可估量},即在现在时,也不可被「染著」等方法估量。\textbf{他没有预设},由无有两种预设故,这阿罗汉于未来时也没有预设。当知此处应与「我说他为寂静」相连,于一切处仿此,此后也不再显示连接,我们将只解释非自明之词。\end{enumerate}

\subsection\*{\textbf{857} {\footnotesize 〔PTS 850〕}}

\textbf{「不忿怒,不惊怖,不吹嘘,不恶作,\\}
\textbf{「思而后言,不掉举,他实为制语的牟尼。}

Akkodhano asantāsī, avikatthī akukkuco;\\
mantabhāṇī anuddhato, sa ve vācāyato muni. %\hfill\textcolor{gray}{\footnotesize 3}

\begin{enumerate}\item \textbf{不惊怖},即不因未得彼彼而惊怖。\textbf{不吹嘘},即不惯于吹嘘戒等。\textbf{不恶作},即无有手的不安等。\textbf{思而后言},即以考量把握后发言。\textbf{他实为制语},即他于言语控制、制御,而说无有四种过失的言语。\end{enumerate}

\subsection\*{\textbf{858} {\footnotesize 〔PTS 851〕}}

\textbf{「不执于未来,不忧伤过去,\\}
\textbf{「于诸触得见远离,于诸见不被引领。}

Nirāsatti anāgate, atītaṃ nānusocati;\\
vivekadassī phassesu, diṭṭhīsu ca na nīyati. %\hfill\textcolor{gray}{\footnotesize 4}

\begin{enumerate}\item \textbf{不执},即不渴爱。\textbf{于诸触得见远离},即于现在的眼触等看见「我」等之相的远离。\textbf{于诸见不被引领},即不被六十二见中的任一见所引领。\end{enumerate}

\subsection\*{\textbf{859} {\footnotesize 〔PTS 852〕}}

\textbf{「内向,不诡诈,不渴望,不悭吝,\\}
\textbf{「不鲁莽,不嫌厌,且不从事诽谤。}

Patilīno akuhako, apihālu amaccharī;\\
appagabbho ajeguccho, pesuṇeyye ca no yuto. %\hfill\textcolor{gray}{\footnotesize 5}

\begin{enumerate}\item \textbf{内向},即由舍弃贪等,背离彼等。\textbf{不诡诈},即不以三种诡诈事欺骗。\textbf{不渴望},即是说无有愿求与渴爱。\textbf{不悭吝},即无有五种悭吝。\textbf{不鲁莽},即无有身鲁莽等\footnote{身鲁莽等,见\textbf{慈经}第 144 颂注。}。\textbf{不嫌厌},即因具足戒等而不嫌厌,纯粹而适意。\textbf{且不从事诽谤},即不从事以二种行相编造的诽谤之业。\end{enumerate}

\subsection\*{\textbf{860} {\footnotesize 〔PTS 853〕}}

\textbf{「不享受愉悦,且不存傲慢,\\}
\textbf{「柔和,富有辩才,不信,不离染。}

Sātiyesu anassāvī, atimāne ca no yuto;\\
saṇho ca paṭibhānavā, na saddho na virajjati. %\hfill\textcolor{gray}{\footnotesize 6}

\begin{enumerate}\item \textbf{不享受愉悦},即于愉悦的事物、种种爱欲,无有渴爱与亲密。\textbf{柔和},即具足柔和的身业等。\textbf{富有辩才},即具足学习、遍问、证得的辩才。\textbf{不信},即对亲身证得的法,不信任何人。\textbf{不离染},即由贪的灭尽而已离染故,现在不必离染。\end{enumerate}

\subsection\*{\textbf{861} {\footnotesize 〔PTS 854〕}}

\textbf{「不为欲求利养而修学,也不恼怒于无利养,\\}
\textbf{「不对立,也不因渴爱而贪求众味。}

Lābhakamyā na sikkhati, alābhe ca na kuppati;\\
aviruddho ca taṇhāya, rasesu nānugijjhati. %\hfill\textcolor{gray}{\footnotesize 7}

\begin{enumerate}\item \textbf{不为欲求利养而修学},即不为愿求利养而修学经等。\textbf{不对立,也不因渴爱而贪求众味},即不以违逆之相而成对立,不因渴爱贪求根味等。\end{enumerate}

\subsection\*{\textbf{862} {\footnotesize 〔PTS 855〕}}

\textbf{「舍,始终具念,在世间不认为是同等、\\}
\textbf{「殊胜或低下,他已没有增盛。}

Upekkhako sadā sato, na loke maññate samaṃ;\\
na visesī na nīceyyo, tassa no santi ussadā. %\hfill\textcolor{gray}{\footnotesize 8}

\begin{enumerate}\item \textbf{舍},即具足六支舍\footnote{六支舍,见\textbf{清净道论}·说地遍品第 157 段。}。\textbf{具念},即从事身随观等的念。\end{enumerate}

\subsection\*{\textbf{863} {\footnotesize 〔PTS 856〕}}

\textbf{「他已没有依止,了知了法而无依止,\\}
\textbf{「他没有对有或离有的渴爱。}

Yassa nissayanā natthi, ñatvā dhammaṃ anissito;\\
bhavāya vibhavāya vā, taṇhā yassa na vijjati. %\hfill\textcolor{gray}{\footnotesize 9}

\begin{enumerate}\item \textbf{依止},即爱、见的依止。\textbf{了知了法},即以无常等行相了知了法。\textbf{无依止},即如是不依止于彼等依止,以此显明「除了法之智外,没有、无有依止」。\textbf{有或离有},即常或断。\end{enumerate}

\subsection\*{\textbf{864} {\footnotesize 〔PTS 857〕}}

\textbf{「我说他为寂静,不关切爱欲,\\}
\textbf{「他已没有系缚,已度过爱著。}

Taṃ brūmi upasanto ti, kāmesu anapekkhinaṃ;\\
ganthā tassa na vijjanti, atarī so visattikaṃ. %\hfill\textcolor{gray}{\footnotesize 10}

\begin{enumerate}\item \textbf{我说他为寂静},即我说一一颂中所说的这样的人为寂静。\textbf{已度过爱著},即他已度过这以蔓延等相而被称为爱著的大渴爱。\end{enumerate}

\subsection\*{\textbf{865} {\footnotesize 〔PTS 858〕}}

\textbf{「他没有孩子、牲畜、田地、物品,\\}
\textbf{「在他那里,没有执取或扬弃。}

Na tassa puttā pasavo, khettaṃ vatthuñ ca vijjati;\\
attā vā pi nirattā vā, na tasmiṃ upalabbhati. %\hfill\textcolor{gray}{\footnotesize 11}

\begin{enumerate}\item 现在,为赞叹此寂静,说了「他没有孩子」等等。这里,\textbf{孩子}有亲生等四种,且当知这里以孩子等之名来说孩子和资产等。因为它们对他不存在,或者因无有彼等,孩子等不存在。\end{enumerate}

\subsection\*{\textbf{866} {\footnotesize 〔PTS 859〕}}

\textbf{「凡夫以及沙门、婆罗门对他所说的,\\}
\textbf{「他不以为意,所以他于论议不动摇。}

Yena naṃ vajjuṃ puthujjanā, atho samaṇabrāhmaṇā;\\
taṃ tassa apurakkhataṃ, tasmā vādesu n’ejati. %\hfill\textcolor{gray}{\footnotesize 12}

\begin{enumerate}\item \textbf{凡夫以及沙门、婆罗门对他所说的},即凡夫、一切人天与除此之外的沙门、婆罗门以贪等罪过对他所说的「染著、恶意」等。\textbf{他不以为意},即这阿罗汉不以贪等罪过为意。\textbf{所以他于论议不动摇},即源于此,他于责难之语不动摇。\end{enumerate}

\subsection\*{\textbf{867} {\footnotesize 〔PTS 860〕}}

\textbf{「离贪求,不悭吝,牟尼不说上等、\\}
\textbf{「同等或下等,无思惟者不起思惟。}

Vītagedho amaccharī, na ussesu vadate muni;\\
na samesu na omesu, kappaṃ n’eti akappiyo. %\hfill\textcolor{gray}{\footnotesize 13}

\begin{enumerate}\item \textbf{牟尼不说上等},即不将自己置于殊胜者之中,因傲慢而说「我殊胜」。另外两者亦然。\textbf{无思惟者不起思惟},即他这样的人不起两种思惟。为什么?因为是无思惟者,即是说已舍弃了思惟。\end{enumerate}

\subsection\*{\textbf{868} {\footnotesize 〔PTS 861〕}}

\textbf{「他在世间没有自我,也不因不存在者忧伤,\\}
\textbf{「不趣向诸法,他实被称为寂静者。」}

Yassa loke sakaṃ natthi, asatā ca na socati;\\
dhammesu ca na gacchati, sa ve santo ti vuccatī” ti. %\hfill\textcolor{gray}{\footnotesize 14}

\begin{enumerate}\item \textbf{自我},即执取为「我所」者。\textbf{不趣向诸法},即不以欲等趣向一切法。\textbf{他实被称为寂静者},即他这样的最上之人被称为「寂静」,即以阿罗汉为顶点完成了开示。当开示终了,百千俱胝的天人证得了阿罗汉,须陀洹等不计其数。\end{enumerate}

\begin{center}\vspace{1em}前分离经第十\\Purābhedasuttaṃ dasamaṃ.\end{center}