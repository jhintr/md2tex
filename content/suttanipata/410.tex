\section{前分离经}

\begin{center}Purābheda Sutta\end{center}\vspace{1em}

\begin{enumerate}\item 此经及随后五经的缘起与正游行经的缘起相同。差别之处在于,在大集会中,为顺适贪行者的众天人而说法,使「相佛」问自己问题后,说了正游行经,同样,在此大集会中,了知了生起「在身体分离之前,应当做些什么」之心的天人的心后,为摄受彼等,从空中带来一千二百五十比丘围绕的相佛,使他问自己问题后,说了此经。\end{enumerate}

\subsection\*{\textbf{855} {\footnotesize 〔PTS 848〕}}

\textbf{「如何见、如何持戒,才能被称为寂静?\\}
\textbf{「乔达摩!既然问到,请告诉我这最上之人!」}

“Kathaṃdassī kathaṃsīlo, upasanto ti vuccati;\\
taṃ me Gotama pabrūhi, pucchito uttamaṃ naraṃ”. %\hfill\textcolor{gray}{\footnotesize 1}

\begin{enumerate}\item 这里,在提问中,相(佛)先就增上慧而问\textbf{如何见},就增上戒而问\textbf{如何持戒},就增上心而问\textbf{寂静}。\end{enumerate}

\subsection\*{\textbf{856} {\footnotesize 〔PTS 849〕}}

\textbf{「在分离前离爱,」世尊说,「不依于先前,\\}
\textbf{「不估量中间,他没有预设。}

“Vītataṇho purā bhedā, \textit{(iti Bhagavā)} pubbam antam anissito;\\
vemajjhe n’upasaṅkheyyo, tassa natthi purakkhataṃ. %\hfill\textcolor{gray}{\footnotesize 2}

\begin{enumerate}\item 然而在解答中,世尊却未同样地解答增上慧等,而是由于以增上慧等的力量,烦恼的寂止被称为寂静,为显明其寂止,以随顺种种天人的意乐,而说以此颂为首的诸颂。这里,当知此颂为首的八颂与第 864 颂「我说他为寂静」相连,随后的几颂则与末颂「他实被称为寂静者」相连。
\item \textbf{在分离前离爱},即在身体分离前,舍弃渴爱。\textbf{不依于先前},不依于过去等的先前。\textbf{不估量中间},即在现在,不应以「染著」等估量。\textbf{他没有预设},由于无两种预设,这阿罗汉于未来也没有预设,当知此处应与「我说他为寂静」相连,于一切处仿此。此后不再显示连接,将只解释不明了的句子。\end{enumerate}

\subsection\*{\textbf{857} {\footnotesize 〔PTS 850〕}}

\textbf{「不忿怒,不惊怖,不吹嘘,不恶作,\\}
\textbf{「思而后言,不掉举,他实为控制言语的牟尼。}

Akkodhano asantāsī, avikatthī akukkuco;\\
mantabhāṇī anuddhato, sa ve vācāyato muni. %\hfill\textcolor{gray}{\footnotesize 3}

\begin{enumerate}\item \textbf{不惊怖},即不因未得彼彼而惊怖。\textbf{不吹嘘},即不惯于吹嘘戒等。\textbf{不恶作},即免于手的恶作等。\textbf{思而后言},即省察所思后发言。\textbf{他实为控制言语},即他于言语控制、制御,而说免于四种过失的言语。\end{enumerate}

\subsection\*{\textbf{858} {\footnotesize 〔PTS 851〕}}

\textbf{「不执于未来,不忧伤过去,\\}
\textbf{「于诸触得见远离,于诸见不被引领。}

Nirāsatti anāgate, atītaṃ nānusocati;\\
vivekadassī phassesu, diṭṭhīsu ca na nīyati. %\hfill\textcolor{gray}{\footnotesize 4}

\begin{enumerate}\item \textbf{不执},即不渴爱。\textbf{于诸触得见远离},即于现在的眼触等得见我等性之远离。\textbf{于诸见不被引领},即不被六十二见中的任一见所引领。\end{enumerate}

\subsection\*{\textbf{859} {\footnotesize 〔PTS 852〕}}

\textbf{「不沉滞,不诡诈,不渴望,不悭吝,\\}
\textbf{「不鲁莽,不可鄙,且不从事于诽谤。}

Patilīno akuhako, apihālu amaccharī;\\
appagabbho ajeguccho, pesuṇeyye ca no yuto. %\hfill\textcolor{gray}{\footnotesize 5}

\begin{enumerate}\item \textbf{不沉滞},即由舍弃贪等离开彼等。\textbf{不诡诈},即不以三种诡诈事而欺骗。\textbf{不渴望},即无有希求与渴爱。\textbf{不悭吝},即无五种悭吝。\textbf{不鲁莽},即无身的鲁莽等。\textbf{不可鄙},即由具足戒等而不可鄙,纯粹而适意。\end{enumerate}

\begin{itemize}\item 案,\textbf{鲁莽}见慈经第 144 颂注。\end{itemize}

\subsection\*{\textbf{860} {\footnotesize 〔PTS 853〕}}

\textbf{「不享受愉悦,不心存傲慢,\\}
\textbf{「柔和,富有辩才,不迷信,不离染。}

Sātiyesu anassāvī, atimāne ca no yuto;\\
saṇho ca paṭibhānavā, na saddho na virajjati. %\hfill\textcolor{gray}{\footnotesize 6}

\begin{enumerate}\item \textbf{不享受愉悦},即于愉悦的事物、种种爱欲无渴爱与亲密。\textbf{柔和},即具足柔和的身业等。\textbf{富有辩才},即具足学习、遍问、证得的辩才。\textbf{不迷信},即于自己证得之法,不迷信任何人。\textbf{不离染},即由贪的灭尽而已离染,故现在不必离染。\end{enumerate}

\subsection\*{\textbf{861} {\footnotesize 〔PTS 854〕}}

\textbf{「不为欲求利养而修学,也不恼怒于无利养,\\}
\textbf{「不对立,也不因渴爱而贪求众味。}

Lābhakamyā na sikkhati, alābhe ca na kuppati;\\
aviruddho ca taṇhāya, rasesu nānugijjhati. %\hfill\textcolor{gray}{\footnotesize 7}

\subsection\*{\textbf{862} {\footnotesize 〔PTS 855〕}}

\textbf{「舍,始终具念,在世间不以为是同等、\\}
\textbf{「殊胜或低下,他已没有增盛。}

Upekkhako sadā sato, na loke maññate samaṃ;\\
na visesī na nīceyyo, tassa no santi ussadā. %\hfill\textcolor{gray}{\footnotesize 8}

\begin{enumerate}\item \textbf{舍},即具足六支舍。\textbf{具念},即从事身随观等的念。\end{enumerate}

\begin{itemize}\item 案,\textbf{六支舍},参见清净道论·说地遍品第 157 段。\end{itemize}

\subsection\*{\textbf{863} {\footnotesize 〔PTS 856〕}}

\textbf{「他已没有依止,了知了法而无所依,\\}
\textbf{「他没有对有或离有的渴爱。}

Yassa nissayanā natthi, ñatvā dhammaṃ anissito;\\
bhavāya vibhavāya vā, taṇhā yassa na vijjati. %\hfill\textcolor{gray}{\footnotesize 9}

\begin{enumerate}\item \textbf{依止},即爱、见的依止。\textbf{了知了法},即以无常等行相了知了法。\textbf{有或离有},即常或断。\end{enumerate}

\subsection\*{\textbf{864} {\footnotesize 〔PTS 857〕}}

\textbf{「我说他为寂静,不关切爱欲,\\}
\textbf{「他已没有系缚,已度过爱著。}

Taṃ brūmi upasanto ti, kāmesu anapekkhinaṃ;\\
ganthā tassa na vijjanti, atarī so visattikaṃ. %\hfill\textcolor{gray}{\footnotesize 10}

\subsection\*{\textbf{865} {\footnotesize 〔PTS 858〕}}

\textbf{「他没有儿子、牲畜、田地、物品,\\}
\textbf{「在他那里,没有接受或丢弃。}

Na tassa puttā pasavo, khettaṃ vatthuñ ca vijjati;\\
attā vā pi nirattā vā, na tasmiṃ upalabbhati. %\hfill\textcolor{gray}{\footnotesize 11}

\subsection\*{\textbf{866} {\footnotesize 〔PTS 859〕}}

\textbf{「凡夫以及沙门、婆罗门对他所说的,\\}
\textbf{「非他思量,所以,他于言论不动摇。}

Yena naṃ vajjuṃ puthujjanā, atho samaṇabrāhmaṇā;\\
taṃ tassa apurakkhataṃ, tasmā vādesu n’ejati. %\hfill\textcolor{gray}{\footnotesize 12}

\subsection\*{\textbf{867} {\footnotesize 〔PTS 860〕}}

\textbf{「不贪求,不悭吝,牟尼不说上等、\\}
\textbf{「同等或下等,无思惟者不起思惟。}

Vītagedho amaccharī, na ussesu vadate muni;\\
na samesu na omesu, kappaṃ n’eti akappiyo. %\hfill\textcolor{gray}{\footnotesize 13}

\subsection\*{\textbf{868} {\footnotesize 〔PTS 861〕}}

\textbf{「他在世间没有自我,也不因不存在者而忧伤,\\}
\textbf{「不趣向诸法,他实被称为寂静者。」}

Yassa loke sakaṃ natthi, asatā ca na socati;\\
dhammesu ca na gacchati, sa ve santo ti vuccatī” ti. %\hfill\textcolor{gray}{\footnotesize 14}

\begin{enumerate}\item 当开示终了,百千俱胝的天人证得了阿罗汉,须陀洹等不计其数。\end{enumerate}

\begin{center}\vspace{1em}前分离经第十\\Purābhedasuttaṃ dasamaṃ.\end{center}