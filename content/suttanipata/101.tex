\chapter{蛇品第一}

\section{蛇经}

\begin{center}Uraga Sutta\end{center}\vspace{1em}

\subsection\*{\textbf{1}}

\textbf{若调伏生起的忿怒,如同用众药(调伏)蔓延的蛇毒,\\}
\textbf{那比丘舍弃此岸彼岸,如蛇(舍弃)先前的老皮。}

Yo uppatitaṃ vineti kodhaṃ, visaṭaṃ sappavisaṃ va osadhehi;\\
so bhikkhu jahāti orapāraṃ, urago jiṇṇam iva ttacaṃ purāṇaṃ.

\begin{enumerate}\item 如是,从品、经、圣典之量来说,\begin{quoting}若调伏生起的忿怒,如同用众药(调伏)蔓延的蛇毒,\\那比丘舍弃此岸彼岸,如蛇(舍弃)先前的老皮。\end{quoting}即其初颂。所以,从此开始,为造其释义而说:\begin{quoting}由谁、何处、何时、为何而说此颂,\\在阐明了这规定后,我将造其释义。\end{quoting}
\item 那么,此颂由谁、何处、何时且为何而说?当答:即彼世尊在二十四佛跟前得到授记,直到毗输安多罗本生\footnote{毗输安多罗 \textit{Vessantara/Viśvantara}:旧译作「一切持、自在间」,又名须达拏 \textit{Sudāna},意即「善施」。}方才圆满了波罗蜜,投生到兜率天的居处,从此殁后,得投生至释迦王族,渐次作了大出离、在菩提树下证等正觉、转法轮,为人天的利益而开示法,即此世尊、自成者、无师者、等正觉所说。且此在旷野中说。且当施设「生物学处\footnote{生物学处:即波逸提之十一。}」时,为来至彼处者开示法而说。这于此是略答。
\item 若详细地说,当知有远因、近因、直接因等。这里,\textbf{远因}即从燃灯至今的故事,\textbf{近因}即从兜率天的居处至今的故事,\textbf{直接因}即从菩提座至今的故事。这里,因为近因与直接因都统摄于远因,所以当知于此唯以远因详答。而这在本生的义注中已述,此处不再详说,唯当如彼处详说。而其差别是,彼处第一颂之事发生于舍卫国,此处则在旷野。如说:\begin{quoting}尔时,佛世尊在旷野中,住旷野顶支提。尔时,旷野的比丘们正从事建造,或伐树,或教人伐。某个旷野的比丘也在伐树,住在那树上的天人对这比丘说:「尊者!莫为了建造自己的住处而伐了我的住处!」这比丘没听,仍旧伐了,还敲到了这天人的孩子的胳膊。于是,这天人想「我何不在此取了这比丘的性命」,然而又想「取了这比丘的性命,这对我不合适,我何不把这事告知世尊」。于是这天人往世尊处走去,走到后把这事告知了世尊。「善哉!善哉!天人!善哉!天人!你没有取了这比丘的性命,天人!如果今天你取了这比丘的性命,那么,天人!你将产生许多非福。去!天人!在某处有僻静之树,到那里去!」(律藏·经分别·波逸提之十一)\end{quoting}如是说已,世尊又为了调伏这天人生起的忿怒,说了此颂:\begin{quoting}若能抑忿发,如止急行车。(法句·忿怒品第 222 颂)\end{quoting}随后,人们如是讥嫌道「沙门释迦子如何能伐树或教人伐?沙门释迦子恼害一根者」,世尊经众比丘听后告知,便施设了这学处:\begin{quoting}毁坏生物者,为波逸提。\end{quoting}为来至彼处者开示法,说了此颂:\begin{quoting}若调伏生起的忿怒,如同用众药(调伏)蔓延的蛇毒。\end{quoting}如是,即此一事摄于三处:律、法句、经集。
\item 至此,凡是大纲中所立的\begin{quoting}由谁、何处、何时、为何而说此颂,\\在阐明了这规定后,我将造其释义。\end{quoting}除了释义,业已从略及从详作了阐明。
\item 而此中的释义为:\textbf{若},即如是从刹帝利家族出家,或从婆罗门家族出家的新学、中座或上座。\textbf{生起}\footnote{关于解释「生起」的一大段,参见\textbf{清净道论}·说智见清净品第 81~91 段。},即向上投入、到达、转起之义,即是说发生。且这「生起」以正在发生、受用后逝去、创造机会、地之获得等而有多种。此中,一切有为法以具有生起等为\textbf{正在发生的生起},就此而说\begin{quoting}已生起之法、未生起之法、具有生起之法。(法集论·三法总纲)\end{quoting}被称为「体验后逝去」的体验了所缘之味而灭去的善与不善,与被称为「受用后逝去」的到达了生起等三者而灭去的其余有为法,名为\textbf{受用后逝去的生起},这可在\begin{quoting}有如是的恶见生起。(中部第 22 经)\end{quoting}及\begin{quoting}且如同已生起的念觉支的修习圆满。(中部第 10 经)\end{quoting}如是等经中见到。如\begin{quoting}凡是他先前所作的业。(中部第 129 经)\end{quoting}等方法所说的业,虽是过去的存在,但由阻止其它异熟、创造自己异熟的机会而持存,并由如是创造机会的异熟虽未生起,但在机会被如是创造时势必发生,名为\textbf{创造机会的生起}。于彼彼地中未被根除的不善,名为\textbf{地之获得的生起}。
\item 且此中,当知地与地之获得的差异。此即是,作为观的所缘的三界之五蕴,名为地,于彼等中能得生起的种种烦恼,名为地之获得。因为此地即是获得,所以称为「地之获得」。且它不是以所缘,因为以所缘,烦恼会就一切过去等类,及就漏尽者已遍知的诸蕴生起,如商人之子输罗耶与难陀学童就大迦旃延与莲花色等的诸蕴\footnote{商人之子输罗耶痴迷大迦旃延事、难陀学童强暴莲花色事,见\textbf{法句}义注。}。如果这也名为地之获得,则由其无法舍弃,无人能舍弃有之根本。
\item 然而,当知以依处名为「地之获得」。因为凡是以观未遍知的诸蕴生起之处,作为流转根本的种种烦恼,即于彼处,从(诸蕴)生起开始便于其中随眠。当知以未舍弃此之义名为「地之获得的生起」。且此处,若烦恼未于其诸蕴舍弃随眠,则彼诸蕴即为彼等烦恼的依处,而非其它诸蕴。且于过去诸蕴未舍弃随眠的烦恼的依处,唯是过去诸蕴,而非其它,于未来等也一样。同样,于欲界诸蕴未舍弃随眠的烦恼的依处,唯是欲界诸蕴,而非其它,于色、无色界也一样。
\item 然而,于任一须陀洹等圣者的诸蕴,彼彼流转根本的种种烦恼为彼彼道所舍弃,则彼彼诸蕴,对于已舍弃的彼彼流转根本的烦恼,由非依处,不得名为地。然而对于凡夫,由未舍弃一切流转根本的烦恼,所作的任何业都是善或不善,如此则缘其烦恼,流转增长。对他而言,不应说「此流转根本唯于色蕴,而非于受蕴等……或唯于识蕴,而非于色蕴等」。为什么?由在五蕴中无差别地随眠故。
\item 如何?如地味等之于树。因为好比植立于地轮的大树,依地味、水味,以之为缘,以根干、枝叉、叶芽、花果等增长已,覆蔽了天空,直至劫末,当因种子的传递、树的谱系相续住立时,不应说「此地味等唯在根中,不在干等中,或唯在果中,不在根等中」。为什么?由在一切根等中无差别地周流故,如是。
\item 然而,好比有人厌弃于这树的花果等,在这树的四周施以名为曼荼迦刺的毒药。于是,这树为毒触所触,由地味、水味的耗尽,而至无法生长,不能再起相续。如是,厌弃于蕴之转起的族姓子,如在树的四周施毒般,于此人自身的相续开始四道的修习。于是,他的这蕴相续,由一切流转根本的烦恼为此四道的毒触所耗尽,所经验的身业等的一切业仅是唯作性,而至无法转起未来的再有,不能起有间\footnote{有间 \textit{bhavantara}:这是字面的直译,DoP 的解释为「另一存在 \textit{another existence}」,可以指过去生,也可以指未来生。}的相续。由最后识的灭去,全都如无薪之火一般,无取而般涅槃。如是当知为地与地之获得的差异。
\item 复次,依现行、把握所缘、未镇伏、未根除等,有四种生起。这里,正在发生的生起即\textbf{现行的生起}。在所缘到达眼等境域的前分尚未生起的种种烦恼,由所缘已被把握,在后分势必发生,被称为\textbf{把握所缘的生起}。在妙善村行乞的大低舍长老\footnote{大低舍长老事,见\textbf{增支部}义注。}由见到异性之色而生起的种种烦恼,于此可作一例。其加行当如\begin{quoting}已生起的欲寻。(中部第 2 经)\end{quoting}等中所示。未以止观中的某个镇伏的种种烦恼,虽未入于心相续,但由无防止发生之因,名为\textbf{未镇伏的生起}。当如\begin{quoting}诸比丘!这入出息念之定已修习、已多作,则为寂静、高贵、纯粹、乐住,令已生起的恶不善法由是退去。(相应部第 54:9 经)\end{quoting}等处所示。为止观所镇伏的种种烦恼,由未被圣道根除,未失发生之法性,被称为\textbf{未根除的生起}。得八等至而行于空中的长老,听到鲜花盛开的林中正在采花的女子以甜美的声音所唱的歌声而生起的种种烦恼,于此可作一例。其加行当如\begin{quoting}多作八支圣道者,在恶不善法生起之时,便令其退去。(相应部第 45:157 经)\end{quoting}等中所示。当知此把握所缘、未镇伏、未根除的生起等三种即为「地之获得」所摄。
\item 如是,在如上所说品类的生起中,这忿怒的生起当依地之获得、把握所缘、未镇伏、未根除的生起来知。为什么?由对如是种类可调伏故。因为唯有如是种类的生起能以某种调伏来调伏,而这被称为正在发生、受用后逝去、创造机会、现行的生起,努力于此为无果且无能为。在受用后逝去中努力为无果,由尚在努力中,它已灭去。在创造机会中也一样。且在正在发生、现行的生起中为无能为,由杂染与净化无法一起发生。
\item 而此中的\textbf{调伏},\begin{quoting}有两种调伏,一一中有五,\\以其中八种,得称为调伏。\end{quoting}两种调伏即律仪调伏、舍断调伏。且于此两种调伏中,一一分为五。\textbf{律仪调伏},即戒律仪、念律仪、智律仪、忍律仪、精进律仪等五种\footnote{五种律仪,参见\textbf{清净道论}·说戒品第 18 段。}。\textbf{舍断调伏},即彼分舍断、镇伏舍断、正断舍断、安息舍断、出离舍断等五种。
\item 这里,当知\begin{quoting}具足、完全具足此波罗提木叉律仪。(分别论)\end{quoting}等处为\textbf{戒律仪},\begin{quoting}守护眼根,于眼根成就律仪。(中部第 27 经)\end{quoting}等处为\textbf{念律仪},\begin{quoting}「世间的这些众流,阿耆多!」世尊说,「念是它们的障碍,\\「我说众流的防护,它们应以慧来遮止。」(经集第 1042 颂)\end{quoting}等处为\textbf{智律仪},\begin{quoting}忍受寒暑。(中部第 2 经)\end{quoting}等处为\textbf{忍律仪},\begin{quoting}对已生的欲寻不容忍、舍断、除遣。(中部第 2 经)\end{quoting}等处为\textbf{精进律仪}。且所有这些律仪,各自由律仪应予律仪、调伏应予调伏的身语恶行而被称为律仪、调伏。如是当知律仪调伏先分为五种。
\item 同样,以名色限定等毗婆舍那支中的彼彼智,只要对自身未退失而转起,对彼彼非义相续的舍断,此即是:以名色差别对有身见,以缘摄受对无因、误因见,以即彼后分之疑惑度脱对疑惑的状态,以聚的思惟对「我、我所」之执取,以道、非道的差别对非道处的道想,以见生对断见,以见灭对常见,以见怖畏对有怖畏处的无怖畏想,以见过患对味想,以厌离随观对喜想,以欲解脱智对不欲解脱,以舍智对不舍,以随顺对法住与涅槃处违逆的状态,以种姓对执取诸行之相的舍断,名为\textbf{彼分舍断}\footnote{彼分舍断,参见\textbf{清净道论}·说智见清净品第 112 段。}。
\item 而以近行、安止等类的定,只要对自身未退失而转起,对所摧服的诸盖及各自的寻等敌对法的被称为未发生的舍断,名为\textbf{镇伏舍断}。而由四圣道的修习,于具有彼彼道的自身的相续,各自以\begin{quoting}舍弃诸见。(法集论第 277 段)\end{quoting}等方法,对所说的集(谛)一侧的烦恼聚,以究竟不再转起而被称为正断的舍断,名为\textbf{正断舍断}。在果的刹那,对烦恼的安息舍断,名为\textbf{安息舍断}。由出离一切有为,舍断一切有为的涅槃,名为\textbf{出离舍断}。且这一切舍断,因为以舍弃之义为舍断,以调伏之义为调伏,所以被称为舍断调伏,或者由彼彼的舍断而有彼彼调伏的生成,被称为舍断调伏。如是当知舍断调伏也分为五种。
\item 如是,由一一分为五种,则此调伏有十。其中除安息舍断及出离舍断,以其余八种调伏的彼彼方法得称为调伏。如何?当以戒律仪调伏身语恶行时,即调伏了与之相应的忿怒,当以念、慧律仪调伏贪忧等时,即调伏了与忧相应的忿怒,当以忍律仪忍耐寒等时,即调伏了彼彼嫌恨事生成的忿怒,当以精进律仪调伏嗔寻时,即调伏了与之相应的忿怒。当彼分、镇伏、正断舍断所凭之法,以在自身发生而舍断彼彼法时,即调伏了应以彼分、镇伏、正断当断的忿怒。且此中,毋宁说调伏非以舍断调伏而生成,而是当以舍断所凭之法来调伏时,由此方法而称为「以舍断调伏来调伏」。而在安息舍断时,由无应调伏者,以及对出离舍断,由无应生起者,故说以两者不调伏任何。如是,其中除安息舍断及出离舍断,以其余八种调伏的彼彼方法得称为调伏。
\item 或者,如\begin{quoting}诸比丘!这五种嫌恨的对治,于比丘生起嫌恨之处,皆应予以对治。哪五种?诸比丘!若于此人生起嫌恨,应于此人修习慈、悲、舍,应于此人无念、无作意,应如是于此人对治嫌恨,或者应于此人胜解业之自身性「此尊者将是业的自身……继承者」。(增支部第 5:161 经)\end{quoting}所说的五种嫌恨的对治,及以\begin{quoting}朋友!这五种嫌恨的对治,于比丘生起嫌恨之处,皆应予以对治。哪五种?朋友!有人于此身行不遍净、语行遍净,朋友!应于此类人对治嫌恨……(增支部第 5:162 经)\end{quoting}等方法所说的五种嫌恨的对治,当以其中任一嫌恨的对治调伏时,得称为\textbf{调伏}。
\item 并且,因为\begin{quoting}诸比丘!若下作的众盗贼以双面锯割截肢体,于此意起嗔怒,他便因此而非我的教法的行者。(中部第 21 经)\end{quoting}如是,大师的教诫为:\begin{quoting}若对忿怒发怒,他便因此成为恶者,\\不对忿怒发怒,他赢得难胜的战斗。\\他行双方的义利,自己的和对方的,\\了知到对方被激怒,他具念而寂止。(相应部第 7:2 经)\end{quoting}\begin{quoting}诸比丘!忿怒的女子或男子招致这七种悦敌、树敌之法。哪七种?于此,诸比丘!仇敌如是希望仇敌:「哎!愿他丑陋!」这是什么原因?诸比丘!仇敌不欢喜仇敌美貌。诸比丘!这忿怒的男人被忿怒征服、被忿怒制服,无论他善沐浴、善涂油、梳理须发、穿著白衣,他被忿怒征服,便是很丑。诸比丘!这是忿怒的女子或男子招致的第一种悦敌、树敌之法。\\复次,诸比丘!仇敌如是希望仇敌:「哎!愿他失眠……愿他无多利益……愿他无财富……愿他无名誉……愿他无朋友……愿他身坏死后,投生苦处、恶趣、堕处、地狱!」这是什么原因?诸比丘!仇敌不欢喜仇敌至善趣。诸比丘!这忿怒的男人被忿怒征服、被忿怒制服,以身行恶行,以语……以意行恶行,他以身语意行恶行已,他被忿怒征服,身坏死后,投生苦处、恶趣、堕处、地狱。\\忿怒者不知义利,忿怒者不得见法……(增支部第 7:64 经)\end{quoting}\begin{quoting}忿怒的有情因忿怒去向恶趣,\\修观者正知了这忿怒,即予舍弃。(如是语第 4 经)\end{quoting}\begin{quoting}舍弃于忿怒,除灭于我慢,解脱一切缚。(法句·忿怒品第 221 颂)\end{quoting}\begin{quoting}忿怒是非义之母,忿怒是心的动荡。(增支部第 7:64 经)\end{quoting}\begin{quoting}广慧者!请忍耐小过!智者不由忿怒之力得成。(本生第 15:19 颂)\end{quoting}以如是等方法省察忿怒的过患,忿怒得以调伏,所以,当如是省察而调伏忿怒时,得称为\textbf{调伏}。
\item \textbf{忿怒},即以经中「『他对我行非义』而生嫌恨」(增支部第 9:29 经)等方法所说的九种,以及与之相对的「他不对我行义利」等九种,即此十八种与「树桩、荆棘」等无缘由者共十九种嫌恨事中某一嫌恨事而生的嫌恨。\textbf{蔓延},即扩散。\textbf{蛇毒},即蛇之毒。\textbf{如同} \textit{iva},即譬喻之词,略去 i 后便只说 va。\textbf{药},即解毒剂。这是说,好比解难的医生,以根、干、皮、叶、花等中的某些配制种种药剂后,能以所制的众药迅速调伏被蛇所咬、弥漫全身而持存、蔓延的蛇毒,如是,他以所说的调伏的方法中的任一种方法调伏、不忍、舍弃、除去、终止以所说之义生起、遍满了心相续而持存的忿怒。
\item \textbf{那比丘舍弃此岸彼岸},即当那比丘如是调伏忿怒时,因为忿怒被第三道完全舍断,所以,当知即舍弃了被称为此岸彼岸的五下分结\footnote{五下分结之「下」与此岸之「此」的巴利文同为 ora。}。因为「彼岸」以无差别而为岸名,所以合此岸与这作为轮回之海的彼岸而说「此岸彼岸」。
\item 或者,以「若调伏生起的忿怒,如同用众药(调伏)蔓延的蛇毒」,他已经以第三道完全调伏了忿怒,住于阿那含果的比丘舍弃此岸彼岸。这里,此岸即自己的自体,彼岸即他人的自体,或者此岸即六内处,彼岸即六外处,同样,此岸即人世间,彼岸即天世间,此岸即欲界,彼岸即色、无色界,此岸即欲、色有,彼岸即无色有,此岸即自体,彼岸即自体的乐与资助。如是,在这此岸彼岸,当以第四道舍断欲贪时,便说「舍弃此岸彼岸」。且此中,虽然阿那含由舍弃对爱欲的贪染,于此自体等已无欲贪,但如同为了对他的第三道等予以表彰,便把一切摄于这此岸彼岸等类,于此以舍断欲贪而说「舍弃此岸彼岸」。
\item 现在,为阐明其义而说譬喻「如蛇(舍弃)先前的老皮」。这里,以腹而行者为\textbf{蛇}。它有两种,形相随所欲者、形相非随所欲者。形相随所欲者又有两种,水生者、陆生者。水生者唯于水中得所欲之形相,非于陆上,如僧佉波罗本生中的僧佉波罗龙王一般。陆生者唯于陆上,非于水中。它以老弱的状态得称为\textbf{老},以历时长久得称为\textbf{先前}。
\item 在蜕皮时,以四种而蜕:保持为同类、生起厌恶、以依靠、以强力。同类,即长身的蛇类。因为蛇于五处不越同类:于投生、于死、于出入冬眠、于与同类交媾、于除去老皮。因为蛇在蜕皮时,它唯保持为同类而蜕。保持为同类,再生起厌恶而蜕。生起厌恶,即当一半解脱、一半悬而不解时,焦急着去蜕。如是生起厌恶,依靠棍杖间、树根间或岩石间而蜕。以依靠去蜕时,生起强力,提起发奋,以精进弯曲尾巴,吐着气,昂起蛇头而蜕。如是蜕已,朝着所欲的方向前行。如是,这欲舍弃此岸彼岸的比丘以四种而舍:保持为同类、生起厌恶、以依靠、以强力。同类,即\begin{quoting}由圣生而生。(中部第 86 经)\end{quoting}所说的比丘的戒,因此才说「住戒有慧人」。如是保持为同类的比丘,对这自己的自体等类的此岸彼岸,如先前的老皮般,以于处处见所生之苦的过患而生起厌恶,依靠善知识,生起称为极度精进的强力,以\begin{quoting}在白昼以经行、打坐,令心离于障碍法而得净化。(增支部第 3:16 经)\end{quoting}所说的方法,分昼夜为六分,奋起、精进,如蛇弯曲尾巴般结跏趺坐,如蛇吐着气般,他不懈怠地努力、精进,如蛇昂头般,他生起广慧,如蛇蜕皮般,舍弃此岸彼岸。舍弃已,如蛇舍离了皮朝着所欲的方向般,他舍离重担,朝着无余依涅槃界的方向前行。因此世尊说:\begin{quoting}若调伏生起的忿怒,如同用众药(调伏)蔓延的蛇毒,\\那比丘舍弃此岸彼岸,如蛇(舍弃)先前的老皮。\end{quoting}如是,世尊以阿罗汉为顶点开示了这第一颂。\end{enumerate}

\subsection\*{\textbf{2}}

\textbf{若无余地断除了贪染,如拔擢了临于水面的莲花,\\}
\textbf{那比丘舍弃此岸彼岸,如蛇(舍弃)先前的老皮。}

Yo rāgam udacchidā asesaṃ, bhisapupphaṃ va saroruhaṃ vigayha;\\
so bhikkhu jahāti orapāraṃ, urago jiṇṇam iva ttacaṃ purāṇaṃ.

\begin{enumerate}\item 现在,轮到为第二颂释义。于此,这大纲也是\begin{quoting}由谁、何处、何时、为何而说此颂,\\在阐明了这规定后,我将造其释义。\end{quoting}且此后一切偈颂(均同)。但恐过于繁复,从此便略去大纲,唯将以显示缘起的方法为显明彼彼之义而造释义,即如此第二颂。
\item 其缘起为:一时,世尊住舍卫国祇树给孤独园。尔时,尊者舍利弗的侍者,某个金匠之子在长老跟前出家。长老思量「对年轻人,不净是适宜的」,为除贪染便授以不净业处。他的心在其中连熟习之量也未得。他便告知长老:「这对我没有助益。」长老思量「这对年轻人是适宜的」,便再次对他解说。如是,四个月过后,他未得些许之量的殊胜。随后,长老把他带到世尊跟前。世尊便说「舍利弗!他的适宜非你的境域所能了知,他应由佛调伏」,以神变化出极清净的莲花,置于其手中:「噫!比丘!在住处的阴面,把茎插入砂地,安置好!然后面朝它,结跏趺坐,以『赤、赤』而转向。」据说,他五百生都是金匠,因此世尊便知晓「赤相对他是适宜的」,授以赤的业处。他这样做了后,于此立刻顺次证得了四禅,开始以顺逆等的方法游戏于禅那。
\item 于是,世尊便决意「令这莲花枯萎」。他从禅那出起,见到它枯萎黯淡后,即得了无常想「极清净之色因衰老而磨灭」。随后,便收摄于其内在。随后,以「无常者即苦,苦者即无我」而见三有如炽燃。如是见时,在离他不远处有一莲池。众小儿潜入其中,采折了莲花后堆积起来。在他看来,这水中的莲花如苇丛中的火聚一般,凋落的叶子如落入深渊一般,被丢在地上的莲花的枯萎花尖如被火烧一般。于是,他据此省虑一切法,三有愈加如火宅一般无所归趣而现起。随后,世尊仍坐于香房,在这比丘上方放出身光,它即散布在他面前。随后,他转向于「这是什么」,如见到世尊前来、站在附近般,便从坐起而合掌。于是,世尊已知其适宜,为开示法,说了这光明之颂。
\item 这里,以享乐为\textbf{贪染},即贪染种种五欲的同义语。\textbf{断除了},即断除、破坏、消除,虽然在诗体中作过去时,晓韵律者仍视作现在时。\textbf{无余},即连同随眠。\textbf{拔擢},即潜入、进入之义\footnote{这里的译文从两种英译本,但 Norman 说,义注的解释与此颂的 BHS 本及 PDhp 本相同,两处作 vigāhya。}。其余均同前颂。
\item 这是说的什么?正好比这些小儿潜入池后,折断临于水面的莲花,如是,这比丘潜入这三界世间的共住后,随顺\begin{quoting}无火如贪欲。(法句·乐品第 202 颂)\end{quoting}\begin{quoting}我因对爱欲的贪染而燃烧,我的心在燃烧。(相应部第 8:4 经)\end{quoting}\begin{quoting}彼耽于欲随欲流,投自结网如蜘蛛。(法句·爱欲品第 347 颂)\end{quoting}\begin{quoting}朋友!受染者,为贪染征服、占据其心者,甚至伤害生命。(增支部第 3:56 经)\end{quoting}如是等方法,以省察贪染的过患、以如前所说品类的戒律仪等的律仪,及以对有识、无识依处的不净想,一点一点以阿那含道有余地断除贪染,随后再以阿罗汉道无余地断除,这比丘也以先前所说的品类舍弃此岸彼岸,如蛇(舍弃)先前的老皮。如是,世尊以阿罗汉为顶点开示了此颂。且当开示终了,这比丘住于阿罗汉。\end{enumerate}

\subsection\*{\textbf{3}}

\textbf{若无余地断除了渴爱,(如)竭涸了疾驶的川流,\\}
\textbf{那比丘舍弃此岸彼岸,如蛇(舍弃)先前的老皮。}

Yo taṇham udacchidā asesaṃ, saritaṃ sīghasaraṃ visosayitvā;\\
so bhikkhu jahāti orapāraṃ, urago jiṇṇam iva ttacaṃ purāṇaṃ.

\begin{enumerate}\item 缘起为何?世尊住舍卫国。某位比丘住在伽伽罗莲池岸边,为渴爱所迫而起不善寻。世尊了知其意乐后,说了这光明之颂。
\item 这里,以渴求为\textbf{渴爱},即于境域不得厌足之义,是欲爱、有爱、无有爱的同义语。\textbf{川流},即已至、已起\footnote{此颂的第二句没有 va 字,解释陷入两难。义注将 sarita 作为过去分词,用来修饰渴爱,故作是解。而 Norman 则认可 Brough 的建议,认为原文 visosayitvā 应作 va sosayitvā。这里的译文据上下几颂的语境补充了「如」字。},是说直至有顶\footnote{有顶:即阿迦腻吒、色究竟天,为五净居地之一。菩提比丘注 297,指非想非非想处。},淹没已而持存的意思。\textbf{疾驶},即疾趋,是说不计现世、来世的过患,能在瞬间到达别的轮围或有顶的意思。如是,对这(如)疾驶川流的一切品类的渴爱,\begin{quoting}向上、洪大、难以填满的希求蔓延四趋,\\若于此贪求,即成携轮者\footnote{携轮者:据菩提比丘注 299,轮即剃刀之轮,为地狱中的刑具。}。(本生第 10:6 颂)\end{quoting}\begin{quoting}以爱为侣的人,轮回于漫长的旅途,\\到此处与他处,不得越过轮回。(经集第 746 颂)\end{quoting}\begin{quoting}大王!世间亏欠、无厌,为渴爱的奴隶。(中部第 82 经)\end{quoting}如是以省察过患、以所说品类的戒律仪等,一点一点\textbf{竭涸}已,以阿罗汉道\textbf{无余地断除},这比丘便在这刹那也舍弃一切品类的此岸彼岸。当开示终了,这比丘住于阿罗汉。\end{enumerate}

\subsection\*{\textbf{4}}

\textbf{若无余地扫除了慢,如同洪流(扫除了)危脆的苇堤,\\}
\textbf{那比丘舍弃此岸彼岸,如蛇(舍弃)先前的老皮。}

Yo mānam udabbadhī asesaṃ, naḷasetuṃ va sudubbalaṃ mahogho;\\
so bhikkhu jahāti orapāraṃ, urago jiṇṇam iva ttacaṃ purāṇaṃ.

\begin{enumerate}\item 缘起为何?世尊住舍卫国。某位比丘住在恒河岸边,见到在热季少水之流中所建的苇堤,其后被来至的洪流所漂没,便起了惊怖「诸行无常」。世尊了知其意乐后,说了这光明之颂。
\item 这里,\textbf{慢},即依于出身等的心的高举,有「我是胜、我是等、我是劣」等三种,又有「我比胜者胜、与胜者等、比胜者劣,比等者胜、与等者等、比等者劣,比劣者胜、与劣者等、比劣者劣」等九种。对这一切品类的慢,以\begin{quoting}为慢迷醉的有情们去向恶趣。(如是语第 6 经)\end{quoting}等的方法,于此以省察过患、以所说品类的戒律仪等一点一点去除,对烦恼中由无力、羸力而如\textbf{苇堤}般(的慢),以出世间法中由极强力而如\textbf{洪流}般的阿罗汉道\textbf{无余地扫除},是说当以无余舍断断除时去除的意思。这比丘便在这刹那也舍弃一切品类的此岸彼岸。当开示终了,这比丘住于阿罗汉。\end{enumerate}

\subsection\*{\textbf{5}}

\textbf{若于诸有中不得坚实,如寻觅于无花果林(不得)花,\\}
\textbf{那比丘舍弃此岸彼岸,如蛇(舍弃)先前的老皮。}

Yo nājjhagamā bhavesu sāraṃ, vicinaṃ puppham ivā udumbaresu;\\
so bhikkhu jahāti orapāraṃ, urago jiṇṇam iva ttacaṃ purāṇaṃ.

\begin{enumerate}\item 缘起为何?此颂及随后的另外十二(颂)的缘起为一。一时,世尊住舍卫国。尔时,某个婆罗门在自己女儿的婚礼前想:「我要用没被那些贱民用过的花装扮女儿,再送到夫家。」他遍寻舍卫国内外,也不见任何先前未被用过的花草。于是,他看到一群生性无赖的婆罗门小儿聚集,想「我去问他们,这群人里肯定有人知道」,便前去询问。他们为嘲弄婆罗门,便说:「就是优昙婆罗花,婆罗门!在世间还没人用过,用它装扮女儿再送去吧!」
\item 第二天,他按时起来,带上口粮,便去到阿致罗筏底河岸边的无花果林,一棵棵树地找着,却连一根花的茎也没看到。于是,过了中午,他便去到对岸。那里,某位比丘在宜人的树下坐而昼住,正作意于业处。他到了那里,未曾注意,时而坐下,时而蹲踞,时而站起,在这树的所有枝、叉、叶中找寻而疲惫。随后,这比丘便对他说:「婆罗门!你在找什么?」「优昙婆罗花,先生!」「婆罗门!世间没有优昙婆罗花,这话虚妄,别费力气!」于是,世尊了知这比丘的意乐后,便放出光明,对生起、存念于尊重的(比丘)说了这些光明之颂。
\item 这里,先就第一颂中的\textbf{诸有},即欲、色、无色、想、无想、非想非非想、一蕴、四蕴、五蕴有。\textbf{坚实},即常性或我性。\textbf{寻觅},即以慧寻求。\textbf{于无花果林(不得)花},好比这婆罗门在无花果树上寻觅花而不得,如是,修行者在一切有中以慧寻求,也不得任何坚实。他以不坚实之义,从无常、无我对诸法修观,渐次证得出世间法,舍弃此岸彼岸,如蛇(舍弃)先前的老皮。此即其文义与章句。在其余诸颂中,便不再说其章句,而将仅说文义的差别处。\end{enumerate}

\subsection\*{\textbf{6}}

\textbf{若其中间无诸忿恨,且已超越这有与无有的状态,\\}
\textbf{那比丘舍弃此岸彼岸,如蛇(舍弃)先前的老皮。}

Yass’antarato na santi kopā, itibhavābhavatañ ca vītivatto;\\
so bhikkhu jahāti orapāraṃ, urago jiṇṇam iva ttacaṃ purāṇaṃ.

\begin{enumerate}\item 此中,先就\textbf{中间}一词,如\begin{quoting}河岸边、驻足处、会堂以及道路边,\\人们聚集后,谈论着我、你与其间。(相应部第 9:8 经)\end{quoting}\begin{quoting}以些许所证的殊胜,他便中道而废。(增支部第 10:84 经)\end{quoting}\begin{quoting}忿怒是非义之母,忿怒是心的动荡,\\从中所生的怖畏,人尚未能悟。(增支部第 7:64 经)\end{quoting}而有原由、中途、心等诸多意义,而此处是指心。由此,\textbf{若其中间无诸忿恨},即由以第三道根除之故,其心无诸忿恨之义。
\item 而因为有即成功,离有即失败,同样,有即增益,离有即减损,有即常,离有即断,有即福,离有即恶,且离有与无有同义,所以,此中\textbf{且已超越这有与无有的状态},是以成功、失败、增益、减损、常、断、福、恶等来说这诸多品类的有与无有\footnote{有与无有 \textit{bhavābhava}:义注显然是按 bhava-abhava 的连声来处理,Norman 说本应作 bhavabhava,出于诗律而将元音拉长,故应译作「再再的有、种种的有」,菩提比丘赞同其说,并举中部义注中的解释为证,见其注 1418。事实上,义注对第 783、808 等颂的解释也作「再再的有」,这里的译文为贴合义注,仍作「有与无有」,且在全书保持一致。},亦由四道,分别以彼彼方法超越这有与无有的状态,如是当知其义。\end{enumerate}

\subsection\*{\textbf{7}}

\textbf{若其诸寻业已熏散,内在已善加廓清而无余,\\}
\textbf{那比丘舍弃此岸彼岸,如蛇(舍弃)先前的老皮。}

Yassa vitakkā vidhūpitā, ajjhattaṃ suvikappitā asesā;\\
so bhikkhu jahāti orapāraṃ, urago jiṇṇam iva ttacaṃ purāṇaṃ.

\begin{enumerate}\item \textbf{若其诸寻},即此中这比丘的欲、嗔、害寻等三,亲族、国土、不死寻等三,哀悯他人相应、利养恭敬名闻、不遭轻贱相应寻等三,这九寻以在「普贤\footnote{\textbf{普贤} \textit{Samantabhaddaka} 为书名。菩提比丘注云,似为\textbf{分别论}第十七·杂事分别 \textit{Khuddakavatthu°} 的别称,亦见于第 17 颂义注,待考。}」中所说的方法,于彼彼处省察过患已,以对治的确定,及以能舍断彼彼的下三道而被\textbf{熏散}、充分熏散、燃尽、烧尽之义。
\item 且如是熏散后,\textbf{内在已善加廓清而无余},在作为自己之内在的自身的蕴相续中,及作为内在之内在的心中,如是以阿罗汉道切断而无余,以使再不生起。因为切断即被称为清除,如说「剃除须发」(相应部第 3:11 经)。如是当知此中之义。\end{enumerate}

\subsection\*{\textbf{8}}

\textbf{若既不超前,也不折返,超越了这一切戏论,\\}
\textbf{那比丘舍弃此岸彼岸,如蛇(舍弃)先前的老皮。}

Yo nāccasārī na paccasārī, sabbaṃ accagamā imaṃ papañcaṃ;\\
so bhikkhu jahāti orapāraṃ, urago jiṇṇam iva ttacaṃ purāṇaṃ.

\begin{enumerate}\item 这是说的什么?因为以过度发起精进而落入掉举者\textbf{超前},以过度松弛而落入懈怠者\textbf{折返},同样,为了有爱而折磨自己者超前,为了欲爱而沉湎欲乐者折返,以常见超前,以断见折返,伤悼于过去者超前,希冀于未来者折返,以对过去的见超前,以对未来的见折返,所以,即是说避开这二端而行中间的行道者既不超前、也不折返的意思。
\item \textbf{超越了这一切戏论},即以阿罗汉道为终了的中间的行道超越、超过、完全超过这一切从受、想、寻产生的称为爱、慢、见的三种戏论之义。\end{enumerate}

\subsection\*{\textbf{9}}

\textbf{若既不超前,也不折返,了知了世间「这一切虚妄」,\\}
\textbf{那比丘舍弃此岸彼岸,如蛇(舍弃)先前的老皮。}

Yo nāccasārī na paccasārī, “sabbaṃ vitatham idan” ti ñatva loke;\\
so bhikkhu jahāti orapāraṃ, urago jiṇṇam iva ttacaṃ purāṇaṃ.

\begin{enumerate}\item 而在随后的颂中,差别处唯有\textbf{了知了世间「这一切虚妄」}。其义为:\textbf{一切}即是说无余、全体、无缺,即便如此,这里的意思仅是毗婆舍那所经验的世间蕴、处、界等品类的有为。\textbf{虚妄},即离如的状态,以种种烦恼之势,被愚人们执取为「常、乐、净、我」者,从如的状态来说即是虚妄。\textbf{这},即这一切,他为以现量的状态来显示而说。\textbf{了知},即以道慧了知,但这是由不痴迷,非由境域\footnote{但这是由不痴迷,非由境域:据菩提比丘注 311,这样说是因为道的实际境域是涅槃,而证得涅槃则去除痴迷,使人了知有为法之虚妄。}。\textbf{世间},即器世间。其连结为:对一切蕴等类的法生者,了知了「这是虚妄」。\end{enumerate}

\subsection\*{\textbf{10}}

\textbf{若既不超前,也不折返,以「这一切虚妄」离贪,\\}
\textbf{那比丘舍弃此岸彼岸,如蛇(舍弃)先前的老皮。}

Yo nāccasārī na paccasārī, “sabbaṃ vitatham idan” ti vītalobho;\\
so bhikkhu jahāti orapāraṃ, urago jiṇṇam iva ttacaṃ purāṇaṃ.

\begin{enumerate}\item 现在,在此后的另外四颂中,差别处即\textbf{离贪、离染、离嗔、离痴}。其中的\textbf{贪},即囊括一切的第一不善根的同义语,或不正之贪的(同义语),如说:\begin{quoting}且有时,贪法会于如母者生起,贪法会于如妹者生起,贪法会于如女者生起。(相应部第 35:127 经)\end{quoting}\textbf{染}即种种五欲之染的同义语。\textbf{嗔}即之前所说的忿怒的同义语。\textbf{痴}即于四圣谛不知晓的同义语。
\item 这里,因为这比丘嫌厌于贪,便开始作观,「愿我某时调伏贪已,能离贪而住」,所以,为对他显示见一切诸行虚妄的状态为舍弃贪的方法,以及舍弃此岸彼岸为舍弃贪的功德,说了此颂。此后的另外几颂也一样。
\item 然而,有些人说:「此中的一一颂都是以如前所述的方法,对嫌厌于这些法而开始作观的彼彼比丘而说的。」可随所好而解。\end{enumerate}

\subsection\*{\textbf{11}}

\textbf{若既不超前,也不折返,以「这一切虚妄」离染,\\}
\textbf{那比丘舍弃此岸彼岸,如蛇(舍弃)先前的老皮。}

Yo nāccasārī na paccasārī, “sabbaṃ vitatham idan” ti vītarāgo;\\
so bhikkhu jahāti orapāraṃ, urago jiṇṇam iva ttacaṃ purāṇaṃ.

\subsection\*{\textbf{12}}

\textbf{若既不超前,也不折返,以「这一切虚妄」离嗔,\\}
\textbf{那比丘舍弃此岸彼岸,如蛇(舍弃)先前的老皮。}

Yo nāccasārī na paccasārī, “sabbaṃ vitatham idan” ti vītadoso;\\
so bhikkhu jahāti orapāraṃ, urago jiṇṇam iva ttacaṃ purāṇaṃ.

\subsection\*{\textbf{13}}

\textbf{若既不超前,也不折返,以「这一切虚妄」离痴,\\}
\textbf{那比丘舍弃此岸彼岸,如蛇(舍弃)先前的老皮。}

Yo nāccasārī na paccasārī, “sabbaṃ vitatham idan” ti vītamoho;\\
so bhikkhu jahāti orapāraṃ, urago jiṇṇam iva ttacaṃ purāṇaṃ.

\subsection\*{\textbf{14}}

\textbf{若其已无任何随眠,并且诸不善根已被铲除,\\}
\textbf{那比丘舍弃此岸彼岸,如蛇(舍弃)先前的老皮。}

Yassānusayā na santi keci, mūlā ca akusalā samūhatāse;\\
so bhikkhu jahāti orapāraṃ, urago jiṇṇam iva ttacaṃ purāṇaṃ.

\begin{enumerate}\item 此后的另外四颂也一样。而其中的释义如下:以未舍弃之义而眠伏于相续者为\textbf{随眠},即欲贪、嗔恚、慢、见、疑、有贪、无明的同义语。以诸相应法与自己的行相一致之义为\textbf{根},以不安稳之义为\textbf{不善},亦以作为诸法的住立为根,以有过及苦异熟之义为不善。两者也都是贪、嗔、痴的同义语,因为它们都以「诸比丘!贪是不善及不善根」等方法来说明。
\item 如是,这些随眠由被彼彼道舍断,对他已无任何存在,并且这些不善根也同样\textbf{已被铲除} \textit{samūhatāse},即 samūhatā 之义。因为语法学家想对复数体格附加 se 音节,而诸义注师却解释说 se 是不变词,可随所好而解。
\item 然而此中既然说「若这样的比丘是漏尽者,而漏尽者既不执取,也不舍弃,舍弃已而住」,同样也是在接近现在的情况下,以现在时态说「舍弃此岸彼岸」。或者,当知是说以无余依涅槃界而般涅槃者舍弃称为自身内外处的此岸彼岸。
\item 这里,当知有依烦恼的次第与依道的次第两种随眠的灭去。依烦恼的次第,欲贪随眠、嗔恚随眠以第三道灭去,慢随眠以第四道,见随眠、疑随眠以第一道,有贪随眠、无明随眠也以第四道。而依道的次第,见随眠、疑随眠以第一道灭去,欲贪随眠、嗔恚随眠以第二道薄,以第三道彻底灭去,慢随眠、有贪随眠、无明随眠以第四道灭去。
\item 这里,因为不是所有随眠都是不善根,即欲贪、有贪随眠唯归于贪不善根,嗔恚随眠、无明随眠即归于嗔、痴不善根,而见、慢、疑随眠不是任何不善根,或因为依无随眠及依不善根已被铲除而令烦恼舍断,所以世尊作如是说。\end{enumerate}

\subsection\*{\textbf{15}}

\textbf{若其已无任何恼患所生的、回到此岸的众缘,\\}
\textbf{那比丘舍弃此岸彼岸,如蛇(舍弃)先前的老皮。}

Yassa darathajā na santi keci, oraṃ āgamanāya paccayāse;\\
so bhikkhu jahāti orapāraṃ, urago jiṇṇam iva ttacaṃ purāṇaṃ.

\begin{enumerate}\item 此中,最初生起的烦恼,以热恼之义名为恼患,而后后生起者,由以这些恼患所生,名为\textbf{恼患所生}。\textbf{此岸},即是说有身,如说:\begin{quoting}此岸,比丘!即有身的同义语。(相应部第 35:238 经)\end{quoting}\textbf{回到},即发生。
\item 这说的是什么?对这执取取蕴者,由以圣道舍弃众缘之故,已无任何与恼患所生同义的烦恼,以先前所说的方法,这比丘舍弃此岸彼岸。\end{enumerate}

\subsection\*{\textbf{16}}

\textbf{若其已无任何欲念所生的、系缚于有的众因,\\}
\textbf{那比丘舍弃此岸彼岸,如蛇(舍弃)先前的老皮。}

Yassa vanathajā na santi keci, vinibandhāya bhavāya hetukappā;\\
so bhikkhu jahāti orapāraṃ, urago jiṇṇam iva ttacaṃ purāṇaṃ.

\begin{enumerate}\item 当知此中的欲念所生与恼患所生相似,其在语义上的差别为:以恳请或接近为欲,即请求、亲近、结交之义,为渴爱的同义语。因为以对境域的希求、受用被称为欲。以缠之力敷展、扩散此欲为欲念,为渴爱随眠的同义语。欲念之所生为\textbf{欲念所生}。
\item 然而,有些人说:「一切烦恼以执取之义被称为欲念,而后后的生起为欲念所生。」前者是此蛇经中的意思,而后者则是法句颂中的。
\item \textbf{系缚于有},即为有所系缚,或者,即系缚于心的境域及未来的发生之义。\end{enumerate}

\subsection\*{\textbf{17}}

\textbf{若已舍弃五盖,无患,度诸犹疑,离于箭刺,\\}
\textbf{那比丘舍弃此岸彼岸,如蛇(舍弃)先前的老皮。}

Yo nīvaraṇe pahāya pañca, anigho tiṇṇakathaṅkatho visallo;\\
so bhikkhu jahāti orapāraṃ, urago jiṇṇam iva ttacaṃ purāṇan ti.

\begin{enumerate}\item 这里,障碍心或有益的行道为\textbf{盖},即覆蔽之义。\textbf{五},即彼数量的限定。由无恼患为\textbf{无患}。由已度疑惑而\textbf{度诸犹疑}。由离于箭刺而\textbf{离于箭刺}。
\item 这是说的什么?若比丘对欲欲等的五盖,以在「普贤」中所说的方法,从总与别见到诸盖中的过患已,以彼彼道舍弃之,且由舍弃彼等之故,以无有称为烦恼之苦的患难而\textbf{无患}。由以\begin{quoting}我在过去曾存在否(中部第 2 经)\end{quoting}等方法度过转起的疑惑而\textbf{度诸犹疑}。由离于以「这里,什么是五箭?贪箭、嗔箭、痴箭、慢箭、见箭」所说的五箭而\textbf{离于箭刺}。这比丘即以先前所说的方法舍弃此岸彼岸。
\item 且此处,当知也有依烦恼的次第与依道的次第两种舍弃诸盖。依烦恼的次第,欲欲盖、嗔恚盖以第三道舍弃,昏沉睡眠盖、掉举盖以第四道,以\begin{quoting}我实未行善。(中部第 129 经)\end{quoting}等方法转起的称为追悔的恶作盖、疑盖以第一道。而依道的次第,恶作盖、疑盖以第一道舍弃,欲欲盖、嗔恚盖以第二道薄,以第三道无余舍弃,昏沉睡眠盖、掉举盖以第四道舍弃。
\item 如是,世尊以阿罗汉为顶点完成了开示。当开示终了,这比丘住于阿罗汉。有些人说:「以彼彼方式,在对这些比丘所开示的偈颂的终了,即以彼彼方式,彼彼比丘住于阿罗汉。」\end{enumerate}

\begin{center}\vspace{1em}蛇经第一\\Uragasuttaṃ paṭhamaṃ.\end{center}

%\begin{flushright}癸卯七月初五二稿\end{flushright}