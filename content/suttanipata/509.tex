\section{黄金学童问}

\subsection\*{\textbf{1091} {\footnotesize 〔PTS 1084〕}}

\textbf{「在乔达摩的教法之前,」尊者黄金说,「他们先前对我所说的,\\}
\textbf{「『这曾是如此、这将是如此』,这一切都是传闻,\\}
\textbf{「这一切都是寻的增长,我对此不喜乐。\footnote{PTS 本将本颂末句归入下颂。又,此颂的前五句与\textbf{诵彼岸道颂}第 1142 颂大致相同。}}

\begin{enumerate}\item 这里,\textbf{他们先前对我所说的},即波婆利等先前对我解说的自己的见解。\textbf{这一切都是寻的增长},即这一切都是欲寻等的增长。\end{enumerate}

\subsection\*{\textbf{1092} {\footnotesize 〔PTS 1085〕}}

\textbf{「请你对我宣说根除渴爱之法!牟尼!\\}
\textbf{「了知此已,具念而行,便能度过世间的爱著。」}

\subsection\*{\textbf{1093} {\footnotesize 〔PTS 1086〕}}

\textbf{「于此,对所见、所闻、所觉、所知的可喜之色,黄金!\\}
\textbf{「驱除欲贪,即是不殁的涅槃境地。}

\begin{enumerate}\item 于是,世尊为对其宣说此法,说了以下二颂。\end{enumerate}

\subsection\*{\textbf{1094} {\footnotesize 〔PTS 1087〕}}

\textbf{「知晓此已,具念的现法寂灭者\\}
\textbf{「便始终寂静,得度世间的爱著。」}

\begin{enumerate}\item 这里,\textbf{知晓此已,具念},即以「一切诸行无常」等方法修观而渐次知晓了这不殁的涅槃境地,以身随观念等而具念。\textbf{现法寂灭者},即由已知法、由已见法,且以贪等的止息而寂灭。其余一切处皆自明。
\item 如是,世尊仍以阿罗汉为顶点开示了此经。当开示终了,与先前一样,而有法的现观。\end{enumerate}

\begin{center}黄金学童问第八\end{center}