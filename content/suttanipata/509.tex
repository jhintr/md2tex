\section{黄金学童问}

\begin{center}Hemaka Māṇava Pucchā\end{center}\vspace{1em}

\subsection\*{\textbf{1091} {\footnotesize 〔PTS 1084〕}}

\textbf{「在乔达摩的教法之前,」尊者黄金说,「他们先前对我所说的,\\}
\textbf{「『这曾是如此、这将是如此』,这一切都是传闻,\\}
\textbf{「这一切都是寻的增长,于此我不喜乐。}

“Ye me pubbe viyākaṃsu, \textit{(icc āyasmā Hemako)} huraṃ Gotamasāsanā;\\
icc āsi iti bhavissati, sabbaṃ taṃ itihītihaṃ;\\
sabbaṃ taṃ takkavaḍḍhanaṃ, nāhaṃ tattha abhiramiṃ. %\hfill\textcolor{gray}{\footnotesize 1}

\begin{enumerate}\item \textbf{他们先前对我所说的},即波婆利等先前对我解说的自己的见解。\textbf{这一切都是寻的增长},这一切都是欲寻等的增长。\end{enumerate}

\begin{itemize}\item 案,PTS 本将本颂末句归入下颂,则此颂与诵彼岸道颂第 1142 颂大致相同。\end{itemize}

\subsection\*{\textbf{1092} {\footnotesize 〔PTS 1085〕}}

\textbf{「而你对我宣说根除渴爱之法,牟尼!\\}
\textbf{「了知此已,具念而行,便能越过世间的爱著。」}

Tvañ ca me dhammam akkhāhi, taṇhā-nigghātanaṃ Muni;\\
yaṃ viditvā sato caraṃ, tare loke visattikaṃ”. %\hfill\textcolor{gray}{\footnotesize 2}

\subsection\*{\textbf{1093} {\footnotesize 〔PTS 1086〕}}

\textbf{「于此,对喜爱的所见、所闻、所觉、所知,黄金!\\}
\textbf{「驱散欲贪,即是不殁的涅槃境地。}

“Idha diṭṭha-suta-muta-viññātesu, piyarūpesu Hemaka;\\
chandarāgavinodanaṃ, nibbānapadam accutaṃ. %\hfill\textcolor{gray}{\footnotesize 3}

\subsection\*{\textbf{1094} {\footnotesize 〔PTS 1087〕}}

\textbf{「知晓此已,具念的现法寂灭者\\}
\textbf{「便始终寂静,越过世间的爱著。」}

Etad-aññāya ye satā, diṭṭhadhammābhinibbutā;\\
upasantā ca te sadā, tiṇṇā loke visattikan” ti. %\hfill\textcolor{gray}{\footnotesize 4}

\begin{enumerate}\item \textbf{知晓此已,具念},即以「一切行无常」等方法修观而渐次知晓了此不死的涅槃境地,以身随观念等而具念。\textbf{现法寂灭者},即由已知法、由已见法,且以贪等的止息而寂灭。
\item 如是,世尊同样以阿罗汉为顶点开示了此经。当开示终了,与先前一样,而有法的现观。\end{enumerate}

\begin{center}\vspace{1em}黄金学童问第八\\Hemakamāṇavapucchā aṭṭhamā.\end{center}