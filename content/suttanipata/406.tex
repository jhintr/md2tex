\section{老经}

\begin{center}Jarā Sutta\end{center}\vspace{1em}

\begin{enumerate}\item 缘起为何?一时,世尊在舍卫国度过雨安居后,对于诸佛,有身体康健、施设未生起的学处、调御可调伏者、为相关的事由谈论本生等人间游行的理由,虑及这些,便出发作人间游行。当渐次游行时,便在晚上到达沙计多,进入漆林。沙计多的居民听后,想「现在不是去见世尊的时候」,当夜转亮时,便持了花鬘、芳香等去到世尊跟前,作了供养、礼拜、问候等,围绕而立,直到世尊入村的时间。
\item 于是,世尊为比丘僧团随从,便入村乞食。某位沙计多的富裕婆罗门正从城中离开,在城门看到了他。看到后便生起了爱子之情,悲泣道「孩子!我好久没见你」,迎面而来。世尊便告知诸比丘:「诸比丘!这婆罗门想做什么,就让他做,不要遮止。」婆罗门如舐犊的母牛般而来,从前后左右四周抱住世尊的身体,说着「孩子!好久没见你,好久没有你」。而他如果不这么做,他会心碎而死。他对世尊说:「世尊!我能布施食物给与您一起来的比丘们,请您摄受我!」世尊便以默然而同意。
\item 婆罗门取了世尊的钵走在前面,派人告诉婆罗门尼:「我的孩子来了,设好座位。」她照做后,看着来人而立,当看到刚到半路的世尊后,生起了爱子之情,「孩子!我好久没见你」,捉着双脚嚎啕大哭,带回家后,恭敬地供以食物。食毕,婆罗门便取了钵。世尊知晓他们的适宜,便开示了法,当开示终了,两人都成了须陀洹。
\item 于是,他们请求世尊:「尊者!只要世尊依此城而住,当唯来我家取食。」世尊拒绝道:「诸佛不会如是唯去向固定的一家。」他们便说:「那么,尊者!与比丘僧团一起行乞后,你们仍来此进食,开示法后再返回寺庙。」世尊为摄受他们,便照做了。人们便称婆罗门、婆罗门尼为「佛父、佛母」,这家也得名「佛家」。
\item 阿难长老问世尊:「我认识世尊的父母,为什么他们却说『我是佛母、我是佛父』呢?」世尊说:「阿难!婆罗门尼与婆罗门曾连续五百生为我的父母,五百生为父母的兄嫂,五百生为(父母的)弟妹,他们是因过去的亲情而说的。」便说此颂:\begin{quoting}因过去的共住,或因现在的利益,\\如是对他而生亲情,好比水中的青莲。(本生第 2:174 颂)\end{quoting}
\item 随后,世尊在沙计多随意住了之后,再次游行,仍去了舍卫国。这婆罗门与婆罗门尼前往诸比丘处,听闻了适当的法的开示,圆满了余下的道,以无余依涅槃界而般涅槃。众婆罗门在城内集会,「我们来恭敬我们的亲族」。须陀洹、斯陀含、阿那含的优婆塞、优婆夷们也集会,「我们来恭敬我们的同法」。他们全都在灵台上堆起毯子,供养花鬘、芳香等,出城而去。
\item 世尊也在当天拂晓时分,以佛眼观察世间,得知他们入般涅槃后,了知到「当我到了那里,众人听闻法的开示后,将得法的现观」,便持了衣钵,离了舍卫国,进入火葬处。人们看到后,想「世尊来了,欲行父母的葬礼」,便礼拜后站立。城民们供养了灵台,运至火葬处,便问世尊:「应如何供养居家的圣弟子?」世尊说:「当如供养无学一般供养他们。」以此意趣,为显明他们的无学牟尼之相,说了此颂:\begin{quoting}彼无害牟尼,常调伏其身,\\到达不死境,无有悲忧处。(法句·第 225 颂)\end{quoting}且在观察了会众后,为开示随适那时的法,说了此经。\end{enumerate}

\subsection\*{\textbf{811} {\footnotesize 〔PTS 804〕}}

\textbf{此生实在短暂,不及百年就要死去,\\}
\textbf{即便能活更久,仍会由衰老而死去。}

Appaṃ vata jīvitaṃ idaṃ, oraṃ vassasatā pi miyyati;\\
yo ce pi aticca jīvati, atha kho so jarasā pi miyyati. %\hfill\textcolor{gray}{\footnotesize 1}

\begin{enumerate}\item 这里,\textbf{此生实在短暂},即此人类的生命,因存续之有限、自作用之有限,实在短暂、有限,亦如箭经(第 580 颂)中所说。\textbf{更久},即超过百年。\end{enumerate}

\subsection\*{\textbf{812} {\footnotesize 〔PTS 805〕}}

\textbf{人们忧伤于执为我者,因为资产无法永存,\\}
\textbf{看到了这分离确实存在,他不应居于俗家。}

Socanti janā mamāyite, na hi santi niccā pariggahā;\\
vinābhāvasantam ev’idaṃ, iti disvā nāgāram āvase. %\hfill\textcolor{gray}{\footnotesize 2}

\begin{enumerate}\item \textbf{于执为我者},即因执为我的依处。\textbf{这分离确实存在},即这实是存在的分离、现存的分离,即是说不能以不分离而存。\end{enumerate}

\subsection\*{\textbf{813} {\footnotesize 〔PTS 806〕}}

\textbf{凡人以为「这是我的」,都随死亡被抛弃,\\}
\textbf{智者知晓此已,我的同仁不应倾向于我执。}

Maraṇena pi taṃ pahīyati, yaṃ puriso “mam’idan” ti maññati;\\
etam pi viditvā paṇḍito, na mamattāya nametha māmako. %\hfill\textcolor{gray}{\footnotesize 3}

\begin{enumerate}\item \textbf{我的同仁},即被归为「我的优婆塞」或「比丘」者,或执取佛等依处者。\end{enumerate}

\subsection\*{\textbf{814} {\footnotesize 〔PTS 807〕}}

\textbf{好比醒来的人见不到梦中的所遇,\\}
\textbf{如是,他也见不到已死亡的爱人。}

Supinena yathā pi saṅgataṃ, paṭibuddho puriso na passati;\\
evam pi piyāyitaṃ janaṃ, petaṃ kālaṅkataṃ na passati. %\hfill\textcolor{gray}{\footnotesize 4}

\begin{enumerate}\item \textbf{所遇},即所遇,或先前所见。\end{enumerate}

\subsection\*{\textbf{815} {\footnotesize 〔PTS 808〕}}

\textbf{人们被见到、被听闻,他们的名字被称及,\\}
\textbf{而对亡者,唯有名字留存,可供谈论。}

Diṭṭhā pi sutā pi te janā, yesaṃ nāmam idaṃ pavuccati;\\
nāmaṃ yevāvasissati, akkheyyaṃ petassa jantuno. %\hfill\textcolor{gray}{\footnotesize 5}

\begin{enumerate}\item \textbf{唯有名字留存,可供谈论},即一切色等种种之法都被抛弃,只留名字,如「佛护、法护」等可思量、可谈论。\end{enumerate}

\subsection\*{\textbf{816} {\footnotesize 〔PTS 809〕}}

\textbf{贪求于我所者不舍弃忧、悲、悭吝,\\}
\textbf{所以,得见安稳的牟尼舍弃了资产而行。}

Sokapparidevamaccharaṃ, na jahanti giddhā mamāyite;\\
tasmā munayo pariggahaṃ, hitvā acariṃsu khemadassino. %\hfill\textcolor{gray}{\footnotesize 6}

\begin{enumerate}\item \textbf{牟尼},即漏尽牟尼。\textbf{得见安稳},即得见涅槃。\end{enumerate}

\subsection\*{\textbf{817} {\footnotesize 〔PTS 810〕}}

\textbf{对内向而行、亲近远离坐处的比丘,\\}
\textbf{不在居处显示自身,人们说这对他是合适的。}

Patilīnacarassa bhikkhuno, bhajamānassa vivittam āsanaṃ;\\
sāmaggiyam āhu tassa taṃ, yo attānaṃ bhavane na dassaye. %\hfill\textcolor{gray}{\footnotesize 7}

\begin{enumerate}\item 第七颂是为了显示在如是被死亡逼迫的世间中随适的行道而说。这里,\textbf{内向而行},即令心从彼彼内向而行。\textbf{比丘},即善妙凡夫或有学。\textbf{不在居处显示自身,人们说这对他是合适的},即如是行道者不在地狱等类的居处显示自身,人们说这对他是合适的,意即如是他便能从此死亡中解脱。\end{enumerate}

\subsection\*{\textbf{818} {\footnotesize 〔PTS 811〕}}

\textbf{牟尼不依于一切处,不喜,也无不喜,\\}
\textbf{悲泣与悭吝之于他,好比水不著于叶。}

Sabbattha munī anissito, na piyaṃ kubbati no pi appiyaṃ;\\
tasmiṃ paridevamaccharaṃ, paṇṇe vāri yathā na limpati. %\hfill\textcolor{gray}{\footnotesize 8}

\begin{enumerate}\item 现在,为了赞扬如「不在居处显示自身」所阐明的漏尽者,说了此后的三颂。这里,\textbf{一切处}即十二入处。\end{enumerate}

\subsection\*{\textbf{819} {\footnotesize 〔PTS 812〕}}

\textbf{又好比水滴之于莲叶,好比水不著于莲花,\\}
\textbf{如是,牟尼不著于所见、所闻或所觉。}

Udabindu yathā pi pokkhare, padume vāri yathā na limpati;\\
evaṃ muni nopalimpati, yad idaṃ diṭṭhasutaṃ mutesu vā. %\hfill\textcolor{gray}{\footnotesize 9}

\begin{enumerate}\item 而在「\textbf{所见、所闻或所觉}」之中,当知应如是连结:\textbf{如是,牟尼不著于}所见、所闻,或此中所觉的诸法。\end{enumerate}

\subsection\*{\textbf{820} {\footnotesize 〔PTS 813〕}}

\textbf{因为除遣者不因所见、所闻或所觉而思量,\\}
\textbf{不希望以其它而清净,因为他既不染著,也不离染。}

Dhono na hi tena maññati, yad idaṃ diṭṭhasutaṃ mutesu vā;\\
nāññena visuddhim icchati, na hi so rajjati no virajjatī ti. %\hfill\textcolor{gray}{\footnotesize 10}

\begin{enumerate}\item \textbf{因为除遣者不因所见、所闻或所觉而思量},此处仍应如是连结:不因所见、所闻之依处而思量,或思量所觉的诸法。\textbf{因为他既不染著,也不离染},既不似愚痴凡夫一般染著,也不似善妙凡夫与有学一般离染,而是由贪的灭尽而被称为「离贪」。其余一切处皆自明。当开示终了,八万四千生类得了法的现观。\end{enumerate}

\begin{center}\vspace{1em}老经第六\\Jarāsuttaṃ chaṭṭhaṃ.\end{center}