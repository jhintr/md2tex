\section{老经}

\begin{center}Jarā Sutta\end{center}\vspace{1em}

\begin{enumerate}\item 一时,世尊在舍卫国度过雨安居后,考虑到为了诸佛的身体康健、施设未生起的学处、调御可调伏者、为相关的事由而论说本生等人间游行的理由,便出发作人间游行。当渐次游行时,在晚上到达沙计多 \textit{Sāketa},进入漆黑树林 \textit{Añjanavana}。沙计多的居民听后,想「现在不是去见世尊的时候」,当夜转亮时,持了花鬘、芳香等去到世尊跟前,作了供养、礼拜、寒暄等,围绕而立,直到世尊入村的时间,于是世尊为比丘僧团随从,便入村乞食。某个沙计多的富裕婆罗门正从城中离开,在城门看到了他。看到后便生起了爱子之情,「孩子!我好久没见你」,悲泣着迎面而来。世尊便告知诸比丘:「诸比丘!这婆罗门想做什么,就让他做,不要遮止。」
\item 婆罗门如舐犊的母牛般而来,从前后左右四周抱着世尊的身体,说着「孩子!好久没见你,好久没有你」。而他如果不这么做,他会心碎而死。他对世尊说:「世尊!我能布施食物给与你一起来的比丘们,请您摄受我!」世尊便以默然而同意。婆罗门持了世尊的钵走在前面,告诉婆罗门尼:「我的孩子来了,准备好座位。」她照做后站着,看着来人,当看到半路上的世尊后,生起了爱子之情,「孩子!我好久没见你」,捉着双脚嚎啕大哭,带回家后,恭敬地供以食物。食毕,婆罗门取走了钵。世尊知晓了适合于他们的,便开示了法,当开示终了,两人都成了须陀洹。于是,他们请求世尊:「尊者!只要世尊依此城而住,当唯来我家取食。」世尊拒绝道:「诸佛不会如是唯去向固定的一家。」他们便说:「既然如此,尊者!与比丘僧团一起行乞后,你们便来此进食,开示法后再返回住处。」世尊为摄受他们便照做了。人们便称婆罗门、婆罗门尼为「佛父、佛母」。这家也得名「佛家」。
\item 阿难长老问世尊:「我知道世尊的父母,为什么他们说『我是佛母、我是佛父』呢?」世尊说:「阿难!婆罗门尼与婆罗门曾连续五百生为我的父母,五百生为父母的兄嫂,五百生为(父母的)弟妹,他们是就过去的亲情而说的。」便说此颂(见本生):「由过去的牵连,或由现在的利益,如是对他而生亲情,好比水中的睡莲。」
\item 随后,世尊在沙计多随意住了之后,再次游行去了舍卫国。这婆罗门与婆罗门尼去到诸比丘处,听闻了适当的开示,圆满了余下的道,以无余涅槃界而般涅槃。众婆罗门在城内集会,「我们来敬仰我们的亲戚」。须陀洹、斯陀含、阿那含的优婆塞、优婆夷们也集会,「我们来敬仰我们的同法」。他们全都在灵台上覆以毛毯,供养着花鬘、芳香等,出城而去。
\item 世尊也在当天拂晓时分,以佛眼观察世间,得知他们入般涅槃后,了知到「于此,当我到达时,众人听闻法的开示后,将得法的现观」,便持了衣钵,离了舍卫国,进入火葬处。人们看到后,想「世尊来了,想参加父母的葬礼」,便礼拜后站立。城民们供养了灵台,运至火葬处,便问世尊:「应如何供养居家的圣弟子?」世尊说:「当如供养无学一般供养他们。」以此意趣,为显明他们的无学牟尼的状态而说此颂(法句·忿怒品第 225 颂):「那些无害的牟尼们,常常以身防护,他们去往不死之处,去到那里后,不再忧伤。」在观察了会众后,为开示随适当时的法,而说此经。\end{enumerate}

\subsection\*{\textbf{811} {\footnotesize 〔PTS 804〕}}

\textbf{此生实在短暂,不及百年就要死去,\\}
\textbf{即便能活更久,仍会由衰老而死去。}

Appaṃ vata jīvitaṃ idaṃ, oraṃ vassasatā pi miyyati;\\
yo ce pi aticca jīvati, atha kho so jarasā pi miyyati. %\hfill\textcolor{gray}{\footnotesize 1}

\subsection\*{\textbf{812} {\footnotesize 〔PTS 805〕}}

\textbf{人们忧伤于我所,因为财产无法永存,\\}
\textbf{看到了这分离确实存在,他不应居于俗家。}

Socanti janā mamāyite, na hi santi niccā pariggahā;\\
vinābhāvasantam ev’idaṃ, iti disvā nāgāram āvase. %\hfill\textcolor{gray}{\footnotesize 2}

\subsection\*{\textbf{813} {\footnotesize 〔PTS 806〕}}

\textbf{凡是人以为「这是我的」的,都随死亡被舍弃,\\}
\textbf{智者知晓此已,我的同仁不应倾向于我性。}

Maraṇena pi taṃ pahīyati, yaṃ puriso “mam’idan” ti maññati;\\
etam pi viditvā paṇḍito, na mamattāya nametha māmako. %\hfill\textcolor{gray}{\footnotesize 3}

\begin{enumerate}\item \textbf{我的同仁},即被称为「我的优婆塞」或「比丘」者,或被认为是佛(法僧)等的依处。\end{enumerate}

\subsection\*{\textbf{814} {\footnotesize 〔PTS 807〕}}

\textbf{好比醒来的人无法见到梦中的所遇,\\}
\textbf{如是他也无法见到业已死亡的所爱之人。}

Supinena yathā pi saṅgataṃ, paṭibuddho puriso na passati;\\
evam pi piyāyitaṃ janaṃ, petaṃ kālaṅkataṃ na passati. %\hfill\textcolor{gray}{\footnotesize 4}

\subsection\*{\textbf{815} {\footnotesize 〔PTS 808〕}}

\textbf{人们被见到、被听闻,他们的名字被称及,\\}
\textbf{而对亡者,唯有名字留存,可供谈论。}

Diṭṭhā pi sutā pi te janā, yesaṃ nāmam idaṃ pavuccati;\\
nāmaṃ yevāvasissati, akkheyyaṃ petassa jantuno. %\hfill\textcolor{gray}{\footnotesize 5}

\subsection\*{\textbf{816} {\footnotesize 〔PTS 809〕}}

\textbf{贪求于我所的人们不舍弃忧、悲、悭吝,\\}
\textbf{所以,得见安稳的牟尼舍弃了财产而行。}

Sokapparidevamaccharaṃ, na jahanti giddhā mamāyite;\\
tasmā munayo pariggahaṃ, hitvā acariṃsu khemadassino. %\hfill\textcolor{gray}{\footnotesize 6}

\subsection\*{\textbf{817} {\footnotesize 〔PTS 810〕}}

\textbf{对于不沉滞而行、亲近远离坐处的比丘,\\}
\textbf{不在住处显示自身,人们说这对他是合适的。}

Patilīnacarassa bhikkhuno, bhajamānassa vivittam āsanaṃ;\\
sāmaggiyam āhu tassa taṃ, yo attānaṃ bhavane na dassaye. %\hfill\textcolor{gray}{\footnotesize 7}

\begin{enumerate}\item 在如是被死亡逼迫的世间,为显示顺适的行道而说第七颂。\textbf{比丘},即善妙的凡夫或有学。\textbf{不在住处显示自身,人们说这对他是合适的},即如是行道者不在地狱等类的住处显示自身,人们说这对他是合适的,因为如是他即能从此死亡中解脱的意思。\end{enumerate}

\subsection\*{\textbf{818} {\footnotesize 〔PTS 811〕}}

\textbf{牟尼不依于一切处,不喜,也无不喜,\\}
\textbf{这悲与悭吝不著于他,好比水珠之于莲叶。}

Sabbattha munī anissito, na piyaṃ kubbati no pi appiyaṃ;\\
tasmiṃ paridevamaccharaṃ, paṇṇe vāri yathā na limpati. %\hfill\textcolor{gray}{\footnotesize 8}

\begin{enumerate}\item 既已如上颂说明了漏尽者,现在为赞美他而说此后的三颂。这里,\textbf{一切处}即十二处。\end{enumerate}

\subsection\*{\textbf{819} {\footnotesize 〔PTS 812〕}}

\textbf{好比水滴不著于莲叶,好比水珠之于莲花,\\}
\textbf{如是,牟尼不著于所见、所闻或所觉。}

Udabindu yathā pi pokkhare, padume vāri yathā na limpati;\\
evaṃ muni nopalimpati, yad-idaṃ diṭṭhasutaṃ mutesu vā. %\hfill\textcolor{gray}{\footnotesize 9}

\subsection\*{\textbf{820} {\footnotesize 〔PTS 813〕}}

\textbf{因为除遣者不由所见、所闻或所觉而思量,\\}
\textbf{不希望由其它而清净,因为他既不染著,也不离染。}

Dhono na hi tena maññati, yad-idaṃ diṭṭhasutaṃ mutesu vā;\\
nāññena visuddhim icchati, na hi so rajjati no virajjatī ti. %\hfill\textcolor{gray}{\footnotesize 10}

\begin{enumerate}\item \textbf{因为他既不染著,也不离染},即不似愚痴凡夫一般染著,也不似善妙凡夫与有学一般离染,而是由贪的灭尽而被称为「离贪」。当开示终了,有八万四千生命得了法的现观。\end{enumerate}

\begin{center}\vspace{1em}老经第六\\Jarāsuttaṃ chaṭṭhaṃ.\end{center}