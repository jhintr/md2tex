\section{大阵经}

\begin{center}Mahābyūha Sutta\end{center}\vspace{1em}

\begin{enumerate}\item 缘起为何?仍在此大集会中,有些天人生起「那些住于见者,在有智者处是受责备还是受赞赏」之心,为揭示此义,以先前的方法,让相佛问自己问题后而说。\end{enumerate}

\subsection\*{\textbf{902} {\footnotesize 〔PTS 895〕}}

\textbf{「凡是那些住于见者,争论道『唯此真实』,\\}
\textbf{「他们全都引来责备,还是于此也受到赞赏?」}

“Ye kec’ime diṭṭhi paribbasānā, ‘idam eva saccan’ ti vivādayanti;\\
sabbe va te nindam anvānayanti, atho pasaṃsam pi labhanti tattha”. %\hfill\textcolor{gray}{\footnotesize 1}

\begin{enumerate}\item 这里,\textbf{引来},即再再带来。\end{enumerate}

\subsection\*{\textbf{903} {\footnotesize 〔PTS 896〕}}

\textbf{「而这实微少,不足以平息,我说有两种争论的果,\\}
\textbf{「见到此后,观照着不争之地的安稳,他不应争论。}

“Appañ hi etaṃ na alaṃ samāya, duve vivādassa phalāni brūmi;\\
etam pi disvā na vivādayetha, khemābhipassaṃ avivādabhūmiṃ. %\hfill\textcolor{gray}{\footnotesize 2}

\begin{enumerate}\item 现在,因为这些说着「唯此真实」的成见的论者们虽在某时某处也受到赞赏,但这被称为赞赏的论说之果,实在微少,不足以平息贪等,何况于第二种责备之果中的论说?所以,为显示此义,先说了这答颂。这里,\textbf{两种争论的果},即责备和赞赏,或与其同分之胜负等。\textbf{见到此后},即以「责备实不可意,赞赏也不足以平息」见到此争论之果中的过患后。\textbf{观照着不争之地的安稳},即看着不争之地的涅槃为「安稳」。\end{enumerate}

\subsection\*{\textbf{904} {\footnotesize 〔PTS 897〕}}

\textbf{「凡是那些共许、凡俗的,知者不入于所有这些,\\}
\textbf{「无牵涉者不堪忍耐所见、所闻,如何能生牵涉?}

Yā kāc’imā sammutiyo puthujjā, sabbā va etā na upeti vidvā;\\
anūpayo so upayaṃ kim eyya, diṭṭhe sute khantim akubbamāno. %\hfill\textcolor{gray}{\footnotesize 3}

\begin{enumerate}\item 因为如是不争论者,「凡是那些……」。这里,\textbf{共许},即见。\textbf{凡俗的},即凡夫所生的。\textbf{如何能生牵涉},即他如何能与色等中,以接近之义而为牵涉的一法有所牵涉?或者,以何原因牵涉?\textbf{不堪忍耐所见、所闻},即于所见所闻的清净不生亲爱。\end{enumerate}

\subsection\*{\textbf{905} {\footnotesize 〔PTS 898〕}}

\textbf{「以戒为最高者以自制说清净,受持了禁戒而护持,\\}
\textbf{「『我们唯于此修学,然后便能清净』,陷入于有,自称善巧。}

Sīluttamā saññamenāhu suddhiṃ, vataṃ samādāya upaṭṭhitāse;\\
‘idh’eva sikkhema ath’assa suddhiṃ’, bhavūpanītā kusalā vadānā. %\hfill\textcolor{gray}{\footnotesize 4}

\begin{enumerate}\item 而对于此外的「以戒为最高者……」。其义为:认为仅有戒为最高的\textbf{以戒为最高}的一些尊者,只\textbf{以自制说清净},且\textbf{受持了}象禁等的\textbf{禁戒而护持},说「\textbf{唯于此}见,\textbf{便能}有导师的\textbf{清净}」,\textbf{陷入于有},即他们在染著于有时说,并且他们还\textbf{自称善巧},即如是论:我们善巧。\end{enumerate}

\subsection\*{\textbf{906} {\footnotesize 〔PTS 899〕}}

\textbf{「如果丧失戒禁,他因违犯了业而颤栗,\\}
\textbf{「他渴望、愿求清净,如离家出行者失去了商队。}

Sace cuto sīlavatato hoti, pavedhatī kamma virādhayitvā;\\
pajappatī patthayatī ca suddhiṃ, satthā va hīno pavasaṃ gharamhā. %\hfill\textcolor{gray}{\footnotesize 5}

\begin{enumerate}\item 且如是,在这些以戒为最高者中,任何如此行道者「如果丧失……」。其义为:\textbf{如果}此后因其它而不关心,或不堪能而\textbf{丧失戒禁},\textbf{他因违犯了}戒禁等业或福行等\textbf{业而颤栗}。且他不仅颤栗,而且\textbf{渴望}、哀叹、\textbf{愿求}这戒禁的\textbf{清净}。如同什么?\textbf{如离家出行者失去了商队},好比离家出行者从商队离散,愿求那家,或是商队。\end{enumerate}

\subsection\*{\textbf{907} {\footnotesize 〔PTS 900〕}}

\textbf{「舍弃了一切戒禁,以及有过、无过的业,\\}
\textbf{「不愿求清净、不清净,应戒离而行,无取于寂静。}

Sīlabbataṃ vāpi pahāya sabbaṃ, kammañ ca sāvajjanavajjam etaṃ;\\
‘suddhiṃ asuddhin’ ti apatthayāno, virato care santim anuggahāya. %\hfill\textcolor{gray}{\footnotesize 6}

\begin{enumerate}\item 然而,圣弟子「\textbf{舍弃了一切戒禁}」等以戒为最高者的颤栗之因。这里,\textbf{有过、无过},即一切不善与世间的善。\textbf{不愿求清净、不清净},即不愿求种种五欲等类的清净、不善等类的不清净。\textbf{应戒离而行},即应戒离清净、不清净而行。\textbf{无取于寂静},即无取于见。\end{enumerate}

\subsection\*{\textbf{908} {\footnotesize 〔PTS 901〕}}

\textbf{「依于苦行或厌离,抑或所见、所闻、所觉,\\}
\textbf{「上流者们赞叹清净,未离于对有或无有的渴爱。}

Tamūpanissāya\footnote{这里按 PTS 本的 Tapūpanissāya 译出,以与义注一致。} jigucchitaṃ vā, atha vā pi diṭṭhaṃ va sutaṃ mutaṃ vā;\\
uddhaṃsarā suddhim anutthunanti, avītataṇhāse bhavābhavesu. %\hfill\textcolor{gray}{\footnotesize 7}

\begin{enumerate}\item 如是显示了此外以戒为最高者、以自制说清净者等的困扰,以及舍弃了戒禁的阿罗汉的行道,现在,为显示以另外的方式说清净的外道,说了此颂。其义为:还有其他沙门婆罗门,他们\textbf{依于厌离}、旨在不死的\textbf{苦行}或\textbf{所见}之清净等中的某个,或以无作见而成为\textbf{上流者}\footnote{上流者 \textit{uddhaṃsara}:这里按字面译出。\textbf{大义释}第 136 段中说:「有些沙门、婆罗门是上流论者。哪些沙门、婆罗门是上流论者?凡沙门、婆罗门是究竟清净者、以轮回为清净者、无作见者、常论者,这些沙门、婆罗门为上流论者。」似是针对与中道相对的两边中的苦行边来说的,与耽于欲乐的「下流者」相对。},\textbf{未离于对有或无有的渴爱,赞叹}、宣说、谈论\textbf{清净}。\end{enumerate}

\subsection\*{\textbf{909} {\footnotesize 〔PTS 902〕}}

\textbf{「愿求者有渴望,或颤栗于遍计,\\}
\textbf{「于此无亡殁与转生者,他为何颤栗,又渴望何方?」}

Patthayamānassa hi jappitāni, pavedhitaṃ vā pi pakappitesu;\\
cutūpapāto idha yassa natthi, sa kena vedheyya kuhiṃ va jappe”. %\hfill\textcolor{gray}{\footnotesize 8}

\begin{enumerate}\item 意即:如是未离渴爱、赞叹清净者中,若认为自己已达清净,则他由未离渴爱故,\textbf{愿求}着有与无有中的彼彼依处,仍\textbf{有渴望}反复存在。因为习行渴爱便增长渴爱。且不仅有渴望,\textbf{或颤栗于遍计},即是说还在其以爱、见遍计的依处中颤栗。然而,由对有与无有已离渴爱故,\textbf{于此无}未来的\textbf{亡殁与转生者,他为何颤栗,又渴望何方}?这即此颂的连结。其余在「义释」中已述。\end{enumerate}

\subsection\*{\textbf{910} {\footnotesize 〔PTS 903〕}}

\textbf{「有些说是『最上』的法,但其他人却说是『卑下』,\\}
\textbf{「此中何者是真实之论?因为他们全都自称是善巧者。」}

“Yam āhu dhammaṃ ‘paraman’ ti eke, tam eva ‘hīnan’ ti panāhu aññe;\\
sacco nu vādo katamo imesaṃ, sabbe va h’īme kusalā vadānā”. %\hfill\textcolor{gray}{\footnotesize 9}

\begin{enumerate}\item 此为问颂。\footnote{此颂的下半颂全同\textbf{小阵经}第 886 颂的下半颂。}\end{enumerate}

\subsection\*{\textbf{911} {\footnotesize 〔PTS 904〕}}

\textbf{「因为他们说自己的法圆满,便说他人的法卑下,\\}
\textbf{「他们如是争执而争论,说各自共许的真实。}

“Sakañ hi dhammaṃ paripuṇṇam āhu, aññassa dhammaṃ pana hīnam āhu;\\
evam pi viggayha vivādayanti, sakaṃ sakaṃ sammutim āhu saccaṃ. %\hfill\textcolor{gray}{\footnotesize 10}

\begin{enumerate}\item 现在,因为此中连一个真实之论也没有,因为他们仅以见而说,所以,为显明此义,先说此答颂。这里,\textbf{共许},即见。\end{enumerate}

\subsection\*{\textbf{912} {\footnotesize 〔PTS 905〕}}

\textbf{「如果因被对方蔑视便卑下,则诸法中便没有殊胜的,\\}
\textbf{「因为他们都说其他的法低下,努力宣扬着自己。}

Parassa ce vambhayitena hīno, na koci dhammesu visesi assa;\\
puthū hi aññassa vadanti dhammaṃ, nihīnato samhi daḷhaṃ vadānā. %\hfill\textcolor{gray}{\footnotesize 11}

\begin{enumerate}\item 如是,在这些说自己的法圆满,而说其他的法卑下的人中,任何人「如果因被对方蔑视便卑下……」。其义为:如果因\textbf{对方的}责备\textbf{便卑下},\textbf{则诸法中便没有殊胜的}、最上的。什么原因?\textbf{因为他们都说其他的法低下},而他们全都\textbf{努力宣扬着自己},即努力论说自己的法。\end{enumerate}

\subsection\*{\textbf{913} {\footnotesize 〔PTS 906〕}}

\textbf{「好比他们赞赏自己的路,同样也供养他们的正法,\\}
\textbf{「那么一切论说都将是如实的,因为清净对他们唯是各别的。}

Saddhammapūjā pi nesaṃ tath’eva, yathā pasaṃsanti sakāyanāni;\\
sabbe va vādā tathiyā bhaveyyuṃ, suddhī hi nesaṃ paccattam eva. %\hfill\textcolor{gray}{\footnotesize 12}

\begin{enumerate}\item 还有什么?「好比他们……」。其义为:且这些外道\textbf{好比他们赞赏自己的路,同样也供养他们的正法},因为他们极度恭敬导师等。这里,如果这些是标准,当如是时,\textbf{那么一切论说都将是如实的}。什么原因?\textbf{因为清净对他们唯是各别的},非于别处成就,亦非由第一义故。因为对那些由他缘引领而觉者,仅把握自己的见。\end{enumerate}

\subsection\*{\textbf{914} {\footnotesize 〔PTS 907〕}}

\textbf{「婆罗门不受他人引领,于诸法抉择已不被摄取,\\}
\textbf{「所以,超越了争论,因为他不视其他法为更胜。}

Na brāhmaṇassa paraneyyam atthi, dhammesu niccheyya samuggahītaṃ;\\
tasmā vivādāni upātivatto, na hi seṭṭhato passati dhammam aññaṃ. %\hfill\textcolor{gray}{\footnotesize 13}

\begin{enumerate}\item 但若由排除了恶而为婆罗门,则此「婆罗门不受他人引领……」。其义为:因为\textbf{婆罗门}以「一切行无常\footnote{即\textbf{法句}第 277 颂。}」等方法,由善见故,没有被他人引领的智,\textbf{于}见之\textbf{诸法抉择已},也\textbf{不被摄取}。以此为由,他越过了见的争辩,且\textbf{他不视其他法为更胜},除了念处等。\end{enumerate}

\subsection\*{\textbf{915} {\footnotesize 〔PTS 908〕}}

\textbf{「『我知、我见,此即如是』,有些人以见认可清净,\\}
\textbf{「如果他看到了,这对他有什么?他们忽略后,以其它说清净。}

‘Jānāmi passāmi tath’eva etaṃ’, diṭṭhiyā eke paccenti suddhiṃ;\\
addakkhi ce kiñ hi tumassa tena, atisitvā aññena vadanti suddhiṃ. %\hfill\textcolor{gray}{\footnotesize 14}

\begin{enumerate}\item 此颂的连结与语义为:如是,首先,因为第一义的婆罗门不视其他的法为最胜,但其他外道以他心智等知、见时,作如是说「\textbf{我知、我见,此即如是}」,并\textbf{以见认可清净}。为什么?因为即便其中有一人以此他心智等\textbf{看到了}如实之义,\textbf{这对他有什么}?这知见对他有什么用?能成就苦之遍知,或断集等的某个吗?因为这些外道于一切处越过圣道后,唯\textbf{以其它说清净},或者,这些外道越过了佛等,唯以其它说清净。\end{enumerate}

\subsection\*{\textbf{916} {\footnotesize 〔PTS 909〕}}

\textbf{「当人看时,便能见名色,既见已,便能知晓唯有这些,\\}
\textbf{「让他随意看多或少,因为善巧者们说不能以此而清净。}

Passaṃ naro dakkhati nāmarūpaṃ, disvāna vā ñassati tāni-m-eva;\\
kāmaṃ bahuṃ passatu appakaṃ vā, na hi tena suddhiṃ kusalā vadanti. %\hfill\textcolor{gray}{\footnotesize 15}

\begin{enumerate}\item 此颂的连结与语义为:还有什么?若他以他心智等看了,\textbf{当人看时},他\textbf{便能见名色},而非过此,\textbf{既见已,便能知晓唯有这些}名色为常或乐,而非其它。当他如是看时,\textbf{让他随意看多或少}的名色为常乐,\textbf{因为善巧者们说},以如此的知见,\textbf{不能以此而清净}。\end{enumerate}

\subsection\*{\textbf{917} {\footnotesize 〔PTS 910〕}}

\textbf{「住著论者实在不易调伏,预设着遍计的见,\\}
\textbf{「对所依止的,自许清净者于此说是净,他于此作如是见。}

Nivissavādī na hi subbināyo, pakappitaṃ diṭṭhi purakkharāno;\\
yaṃ nissito tattha subhaṃ vadāno, suddhiṃvado tattha tath’addasā so. %\hfill\textcolor{gray}{\footnotesize 16}

\begin{enumerate}\item 此颂的连结与语义为:而以此知见,在甚至不存在清净时,若如是作\textbf{住著论}「我知、我见,此即如是」,或者,缘此知见,以见认定清净,如是作住著论「唯此真实」,他便\textbf{实在不易调伏},如是\textbf{预设着遍计}、行作\textbf{的见}。这\textbf{自许清净者},\textbf{对所依止的}导师等,\textbf{于此说是净},认为自己「我是遍净论者,或遍净见者」,\textbf{他于此作如是见},他于此以自己的见,并未见到颠倒。意即:正如此见所转,他便如是见之,不希望以其它方式去见。\end{enumerate}

\subsection\*{\textbf{918} {\footnotesize 〔PTS 911〕}}

\textbf{「婆罗门经省察,不起思惟,不流于见,也不缚于智,\\}
\textbf{「且他了知凡俗的共许后,舍,让他人执取。}

Na brāhmaṇo kappam upeti saṅkhā, na diṭṭhisārī na pi ñāṇabandhu;\\
ñatvā ca so sammutiyo puthujjā, upekkhatī uggahaṇanti-m-aññe. %\hfill\textcolor{gray}{\footnotesize 17}

\begin{enumerate}\item 在如是预设着遍计的见的外道中,「婆罗门经省察,不起思惟……」。这里,\textbf{省察},即了知之义。\textbf{也不缚于智},不被等至之智等缚于爱、见。这里「也不缚于智」的拆解为:也不存在被此智所缚。\textbf{共许},即见的共许。\textbf{凡俗的},即凡夫所生的。\textbf{让他人执取},即是说让他人执取这共许。\end{enumerate}

\subsection\*{\textbf{919} {\footnotesize 〔PTS 912〕}}

\textbf{「舍离了系缚,牟尼于此世间的种种争论,不随流于众,\\}
\textbf{「于诸不寂静中,他寂静、舍,无取,让他人执取。}

Vissajja ganthāni munīdha loke, vivādajātesu na vaggasārī;\\
santo asantesu upekkhako so, anuggaho uggahaṇanti-m-aññe. %\hfill\textcolor{gray}{\footnotesize 18}

\begin{enumerate}\item 还有什么?「舍离了系缚……」。这里,\textbf{无取},即以没有执取、非执取者为无取,或以不执取为无取。\end{enumerate}

\subsection\*{\textbf{920} {\footnotesize 〔PTS 913〕}}

\textbf{「舍弃了过去诸漏,不造新者,不随欲而往,也非住著论者,\\}
\textbf{「这智者解脱于见,不染于世间,不谴责自己。}

Pubbāsave hitvā nave akubbaṃ, na chandagū no pi nivissavādī;\\
sa vippamutto diṭṭhigatehi dhīro, na lippati loke anattagarahī. %\hfill\textcolor{gray}{\footnotesize 19}

\begin{enumerate}\item 还有什么?他这样的「舍弃了过去诸漏……」。这里,\textbf{过去诸漏},即就过去的色等生起的烦恼法。\textbf{新者},即就现在的色等生起的法。\textbf{不随欲而往},即不因为欲而前往。\textbf{不谴责自己},即不以作或未作谴责自己。\end{enumerate}

\subsection\*{\textbf{921} {\footnotesize 〔PTS 914〕}}

\textbf{「他平定了一切法,或任何所见、所闻、所觉,\\}
\textbf{「这牟尼放下重担,已解脱,不思惟,不抑止,不愿求。」}

Sa sabbadhammesu visenibhūto, yaṃ kiñci diṭṭhaṃ va sutaṃ mutaṃ vā;\\
sa pannabhāro muni vippamutto, na kappiyo nūparato na patthiyo” ti. %\hfill\textcolor{gray}{\footnotesize 20}

\begin{enumerate}\item 且如是不谴责自己者,「他平定了一切法……」。这里,\textbf{一切法},即六十二见法,\textbf{或任何所见}等品类。\textbf{不思惟},即不起两种思惟之义。\textbf{不抑止},即非如善妙凡夫与有学一般,不再是具抑止者。\textbf{不愿求},即离渴爱。因为渴爱以愿求为愿求,若无愿求即不愿求。其余则以处处明了故不述。如是,以阿罗汉为顶点完成了开示。当开示终了,仍与「前分离经」所说的一样,而有现观。\end{enumerate}

\begin{center}\vspace{1em}大阵经第十三\\Mahābyūhasuttaṃ terasamaṃ.\end{center}