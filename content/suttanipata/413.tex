\section{大阵经}

\begin{center}Mahābyūha Sutta\end{center}\vspace{1em}

\begin{enumerate}\item 亦在此大集会中,有些天人生起「住于见者在有智者处是仅受到责备,还是也受赞赏」之心,为显明其义,以如前所述的方法,使相佛问自己问题后而说。\end{enumerate}

\begin{itemize}\item 案,此经第 902、910 二颂是发问,其余是回答。\end{itemize}

\subsection\*{\textbf{902} {\footnotesize 〔PTS 895〕}}

\textbf{「凡是那些住于见者,争论『唯此是真实』,\\}
\textbf{「他们全都引来责备,还是于此也受到赞赏?」}

“Ye kec’ime diṭṭhi paribbasānā, ‘idam eva saccan’ ti vivādayanti;\\
sabbe va te nindam anvānayanti, atho pasaṃsam pi labhanti tattha”. %\hfill\textcolor{gray}{\footnotesize 1}

\subsection\*{\textbf{903} {\footnotesize 〔PTS 896〕}}

\textbf{「而这实微少,不足以平息,我说有两种争论的果,\\}
\textbf{「见到此后,观见着不争之地的安稳,他不应争论。}

“Appañ hi etaṃ na alaṃ samāya, duve vivādassa phalāni brūmi;\\
etam pi disvā na vivādayetha, khemābhipassaṃ avivādabhūmiṃ. %\hfill\textcolor{gray}{\footnotesize 2}

\begin{enumerate}\item 现在,因为这些说着「唯此是真实」的具见的论者们在某时某处也受到赞赏,但这被称为赞赏的论说之果,实在微少,不足以平息贪等,更何况于第二种责备之果中的论说?所以,为显示此义,而说此答颂。这里,\textbf{两种争论的果},即责备和赞赏,或其同分之胜负等。\textbf{见到此后},即「责备实不可喜,赞赏也不足以平息」,见到此争论之果中的过患后。\textbf{观见着不争之地的安稳},即看着不争之地的涅槃为「安稳」。\end{enumerate}

\subsection\*{\textbf{904} {\footnotesize 〔PTS 897〕}}

\textbf{「凡是那些凡夫所认可的,智者不入于所有这些,\\}
\textbf{「无执著者,不堪忍耐所见所闻,如何能起执著?}

Yā kāc’imā sammutiyo puthujjā, sabbā va etā na upeti vidvā;\\
anūpayo so upayaṃ kim eyya, diṭṭhe sute khantim akubbamāno. %\hfill\textcolor{gray}{\footnotesize 3}

\begin{enumerate}\item \textbf{所认可的},即见。\textbf{不堪忍耐所见所闻},即于所见所闻的清净不生亲爱。\end{enumerate}

\subsection\*{\textbf{905} {\footnotesize 〔PTS 898〕}}

\textbf{「以戒为最高者说以自制而清净,受持了禁戒而从事,\\}
\textbf{「『我们唯于此修学,然后能得清净』,自称是善巧,陷入于有。}

Sīluttamā saññamenāhu suddhiṃ, vataṃ samādāya upaṭṭhitāse;\\
‘idh’eva sikkhema ath’assa suddhiṃ’, bhavūpanītā kusalā vadānā. %\hfill\textcolor{gray}{\footnotesize 4}

\subsection\*{\textbf{906} {\footnotesize 〔PTS 899〕}}

\textbf{「如果丧失戒禁,他因违犯了业而颤栗,\\}
\textbf{「他渴望、欲求清净,好比离家出行者失去了商队。}

Sace cuto sīlavatato hoti, pavedhatī kamma virādhayitvā;\\
pajappatī patthayatī ca suddhiṃ, satthā va hīno pavasaṃ gharamhā. %\hfill\textcolor{gray}{\footnotesize 5}

\subsection\*{\textbf{907} {\footnotesize 〔PTS 900〕}}

\textbf{「舍弃了一切戒禁,以及有罪、无罪的业,\\}
\textbf{「不欲求『清净、不清净』,应远离而行,无取于寂静。}

Sīlabbataṃ vāpi pahāya sabbaṃ, kammañ ca sāvajjanavajjam etaṃ;\\
‘suddhiṃ asuddhin’ ti apatthayāno, virato care santim anuggahāya. %\hfill\textcolor{gray}{\footnotesize 6}

\begin{enumerate}\item \textbf{无取于寂静},即无取于见。\end{enumerate}

\subsection\*{\textbf{908} {\footnotesize 〔PTS 901〕}}

\textbf{「依于苦行或厌离,抑或所见、所闻、所觉,\\}
\textbf{「上流者们宣扬清净,未离于对有或无有的渴爱。}

Tamūpanissāya jigucchitaṃ vā, atha vā pi diṭṭhaṃ va sutaṃ mutaṃ vā;\\
uddhaṃsarā suddhim anutthunanti, avītataṇhāse bhavābhavesu. %\hfill\textcolor{gray}{\footnotesize 7}

\begin{itemize}\item 案,\textbf{Tamūpanissāya},这里按 PTS 本的 \textit{Tapūpanissāya} 译出。\textbf{上流者},系按字面 \textit{Uddhaṃsarā} 译出,义释 \textit{Niddesa} 中说「有些沙门、婆罗门是上流论者。哪些沙门、婆罗门是上流论者?凡沙门、婆罗门以边际为清净者、以轮回即清净者、持不作之见者、常论者,这些沙门、婆罗门为上流论者」,似是针对与中道相对的两边中的苦行边来说的,与耽于欲乐的「下流者」相对。\end{itemize}

\subsection\*{\textbf{909} {\footnotesize 〔PTS 902〕}}

\textbf{「欲求者们有诸渴望,或颤栗于诸遍计,\\}
\textbf{「于此已无亡殁与转生者,他为何颤栗,又会渴望何方?」}

Patthayamānassa hi jappitāni, pavedhitaṃ vā pi pakappitesu;\\
cutūpapāto idha yassa natthi, sa kena vedheyya kuhiṃ va jappe”. %\hfill\textcolor{gray}{\footnotesize 8}

\subsection\*{\textbf{910} {\footnotesize 〔PTS 903〕}}

\textbf{「有些说是『最上』的法,其他人却说是『卑下』,\\}
\textbf{「此中何者是真实之论?因为他们全都宣称是善巧者。」}

“Yam āhu dhammaṃ ‘paraman’ ti eke, tam eva ‘hīnan’ ti panāhu aññe;\\
sacco nu vādo katamo imesaṃ, sabbe va h’īme kusalā vadānā”. %\hfill\textcolor{gray}{\footnotesize 9}

\begin{itemize}\item 案,此颂上半颂与小阵经第 890 颂的上半颂类似,后半颂全同第 886 颂的后半颂。\end{itemize}

\subsection\*{\textbf{911} {\footnotesize 〔PTS 904〕}}

\textbf{「因为他们说自己的法圆满,便说他人的法卑下,\\}
\textbf{「他们如是争执而争论,说各自认可的真实。}

“Sakañ hi dhammaṃ paripuṇṇam āhu, aññassa dhammaṃ pana hīnam āhu;\\
evam pi viggayha vivādayanti, sakaṃ sakaṃ sammutim āhu saccaṃ. %\hfill\textcolor{gray}{\footnotesize 10}

\subsection\*{\textbf{912} {\footnotesize 〔PTS 905〕}}

\textbf{「如果以对方的蔑视而卑下,则诸法之中便没有殊胜的,\\}
\textbf{「因为他们都说别人的法低下,而努力宣扬着自己。}

Parassa ce vambhayitena hīno, na koci dhammesu visesi assa;\\
puthū hi aññassa vadanti dhammaṃ, nihīnato samhi daḷhaṃ vadānā. %\hfill\textcolor{gray}{\footnotesize 11}

\subsection\*{\textbf{913} {\footnotesize 〔PTS 906〕}}

\textbf{「好比他们赞赏自己的道,同样也供养他们的正法,\\}
\textbf{「那么一切论说都将是如实的,因为对于他们,清净唯是各别的。}

Saddhammapūjā pi nesaṃ tath’eva, yathā pasaṃsanti sakāyanāni;\\
sabbe va vādā tathiyā bhaveyyuṃ, suddhī hi nesaṃ paccattam eva. %\hfill\textcolor{gray}{\footnotesize 12}

\subsection\*{\textbf{914} {\footnotesize 〔PTS 907〕}}

\textbf{「对于婆罗门,不受他人的引领,于诸法抉择已,也不被摄取,\\}
\textbf{「所以已超越争论,因为他不视其他的法为更胜。}

Na brāhmaṇassa paraneyyam atthi, dhammesu niccheyya samuggahītaṃ;\\
tasmā vivādāni upātivatto, na hi seṭṭhato passati dhammam aññaṃ. %\hfill\textcolor{gray}{\footnotesize 13}

\subsection\*{\textbf{915} {\footnotesize 〔PTS 908〕}}

\textbf{「『我知、我见,此即如是』,有些人以见认可清净,\\}
\textbf{「如果他已看到,这对他又有什么?略过已,他们以其它而说清净。}

‘Jānāmi passāmi tath’eva etaṃ’, diṭṭhiyā eke paccenti suddhiṃ;\\
addakkhi ce kiñ hi tumassa tena, atisitvā aññena vadanti suddhiṃ. %\hfill\textcolor{gray}{\footnotesize 14}

\begin{enumerate}\item 如是,因为第一义的婆罗门不视其他的法为更胜,但其他外道以他心智等知、见时,说着「\textbf{我知、我见,此即如是}」,\textbf{以见认可清净}。为什么?因为即便他以此他心智等\textbf{已看到}其中如实之义,\textbf{这对他又有什么},对他,这所见有什么用,能成就苦之遍知,或断集等的任何一个吗?因为于一切处越过了圣道,这些外道唯\textbf{以其它而说清净},或者这些外道越过了佛等,唯以其它而说清净。\end{enumerate}

\subsection\*{\textbf{916} {\footnotesize 〔PTS 909〕}}

\textbf{「当人看时,便能见名色,既见已,便能知晓唯有这些,\\}
\textbf{「让他随意看多或少,因为善人们说并不能由此而清净。}

Passaṃ naro dakkhati nāmarūpaṃ, disvāna vā ñassati tāni-m eva;\\
kāmaṃ bahuṃ passatu appakaṃ vā, na hi tena suddhiṃ kusalā vadanti. %\hfill\textcolor{gray}{\footnotesize 15}

\begin{enumerate}\item 若他以他心智等看,他\textbf{便能见名色},不能过此,\textbf{既见已,便能知晓唯有这些}名色为常或乐,而非其它。当他如是看时,\textbf{让他随意看多或少}的名色为常乐,\textbf{因为善人们说},以如此的所见,\textbf{并不能由此而清净}。\end{enumerate}

\subsection\*{\textbf{917} {\footnotesize 〔PTS 910〕}}

\textbf{「住著论者预设着遍计的见,实在不易调伏,\\}
\textbf{「凡所依者,即于此说是净,自许清净者于此如是而见。}

Nivissavādī na hi subbināyo, pakappitaṃ diṭṭhi purakkharāno;\\
yaṃ nissito tattha subhaṃ vadāno, suddhiṃvado tattha tath’addasā so. %\hfill\textcolor{gray}{\footnotesize 16}

\subsection\*{\textbf{918} {\footnotesize 〔PTS 911〕}}

\textbf{「婆罗门经省察,不起思惟,不流于见,也不缚于智,\\}
\textbf{「且了知了凡夫所认可的,他不关心,(任凭)他人执取。}

Na brāhmaṇo kappam upeti saṅkhā, na diṭṭhisārī na pi ñāṇabandhu;\\
ñatvā ca so sammutiyo puthujjā, upekkhatī uggahaṇanti m aññe. %\hfill\textcolor{gray}{\footnotesize 17}

\begin{enumerate}\item \textbf{也不缚于智},不被等至之智等缚于爱、见。\end{enumerate}

\begin{itemize}\item 案,\textbf{不关心},即舍。\end{itemize}

\subsection\*{\textbf{919} {\footnotesize 〔PTS 912〕}}

\textbf{「舍离了系缚,牟尼于此世间,当争论生起时,不随流于众,\\}
\textbf{「于不寂静中,他寂静、舍,无取,(任凭)他人执取。}

Vissajja ganthāni munīdha loke, vivādajātesu na vaggasārī;\\
santo asantesu upekkhako so, anuggaho uggahaṇanti m aññe. %\hfill\textcolor{gray}{\footnotesize 18}

\subsection\*{\textbf{920} {\footnotesize 〔PTS 913〕}}

\textbf{「舍弃了过去诸漏,不造新者,不随欲而往,也非住著论者,\\}
\textbf{「这智者解脱于见,不染于世间,不谴责自己。}

Pubbāsave hitvā nave akubbaṃ, na chandagū no pi nivissavādī;\\
sa vippamutto diṭṭhigatehi dhīro, na lippati loke anattagarahī. %\hfill\textcolor{gray}{\footnotesize 19}

\begin{enumerate}\item \textbf{过去诸漏},即关于过去的色等生起之烦恼法。\textbf{新者},即关于现在的色等生起之法。\textbf{不谴责自己},即不以作或未作谴责自己。\end{enumerate}

\subsection\*{\textbf{921} {\footnotesize 〔PTS 914〕}}

\textbf{「他破除了一切法,或任何所见、所闻、所觉,\\}
\textbf{「这牟尼放下重担,已解脱,不思惟,不抑止,不欲求。」}

Sa sabbadhammesu visenibhūto, yaṃ kiñci diṭṭhaṃ va sutaṃ mutaṃ vā;\\
sa pannabhāro muni vippamutto, na kappiyo nūparato na patthiyo” ti. %\hfill\textcolor{gray}{\footnotesize 20}

\begin{enumerate}\item \textbf{不抑止},非如善妙凡夫与有学一般,不再具有抑止。\end{enumerate}

\begin{center}\vspace{1em}大阵经第十三\\Mahābyūhasuttaṃ terasamaṃ.\end{center}