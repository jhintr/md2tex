\section{法行经}

\begin{center}Dhammacariya Sutta\end{center}\vspace{1em}

\begin{enumerate}\item \textbf{迦毗罗经}\footnote{迦毗罗经 \textit{Kapilasutta}:这是义注中的经题。}。缘起为何?仍以在雪山经中所述之法,当迦叶世尊般涅槃后,二族姓子兄弟离家,在(迦叶佛的)弟子们跟前出了家。年长的名为输陀那,年幼的名为迦毗罗。他俩的母亲名为娑陀尼,幼妹名为多般那,两人也在比丘尼中出了家。随后,他俩也以雪山经中所述之法问到「教法内有几种责任」,且听闻后,年长的想「我要圆满居住的责任」,便在阿阇黎与和尚跟前住满五年,满五年后,直至阿罗汉,一直以听闻业处、住阿兰若而精进,便证了阿罗汉。迦毗罗想「我尚年幼,到长大时我再圆满居住的责任」,便著手书卷的责任,成了三藏。依其学习便有了随从,依于随从便有了利养。
\item 他为博学之慢陶醉,自以为是智者,且于所不知亦以为已知,他人说为许可、不许可、有过、无过者,他说为不许可、许可、无过、有过。当为正派的比丘们以「朋友迦毗罗!莫作此说」等方法教诫时,他仍以「你们知道什么?和空空的拳头一样」等言语咒骂、蔑视而行。比丘们便将此事告知了他的哥哥。他也前去对他说:「朋友迦毗罗!教法的寿命,在于如你等者正当的行道,朋友迦毗罗!莫将许可、不许可、有过、无过者说为不许可、许可、无过、有过!」他仍不接受他的言语。随后,输陀那长老说过两三次,便以\begin{quoting}怜悯者会说上一语或二语,\\过此则不应说,如奴隶在主人跟前。(本生第 19:34 颂)\end{quoting}回避而离开:「你啊,朋友!会被自身的业所知!」从此以后,正派的比丘们便抛弃了他。
\item 他成为恶正行者后,为恶正行者围绕而住,一天,他想「我要重行布萨」,上到狮子座,拿了彩绘的扇子而坐,说了三次:「此中,朋友!有比丘会波罗提木叉吗?」然后,甚至没有一个比丘说「我会」,且他或他们都不会波罗提木叉。随后,他想「听不听波罗提木叉,律都不存在」,便从坐起。如是,他便令迦叶世尊的教法消亡、败坏。于是,输陀那长老在同一天般涅槃。这迦毗罗也在如是令教法消亡后死去,转生于无间大地狱。他的那母亲和妹妹也追随他的见,骂詈、指责正派的比丘们,死后转生于地狱。
\item 且就在那时,五百个以洗劫村庄等盗窃活命的人,被国人追捕而逃亡,进入林野后,在那里没看到任何密林或庇护,在不远处见到某位住在石下的林野比丘,礼拜后说:「尊者!请庇护我们!」长老便说:「对于你们,没有与戒相等的庇护,你们都应受持五戒!」他们以「善哉」领受后,便受持了戒。长老说:「你们既具有戒,现在,即便对败坏自己的性命者,也不应污染心意!」他们便以「善哉」领受。于是,那些国人到来,四处寻觅,见到盗贼后,便把他们都杀了。
\item 他们死后转生在欲界天,其中,盗贼的长者成了天子的长者,其余为其随从。他们经顺逆轮回,在天界度过了一佛的间隔,当我们的世尊时,从天界殁后,天子的长者——在舍卫城门外有一渔村——便于彼处五百家渔夫的长者的夫人胎内获取了结生,余者便在其余渔夫夫人的(胎内)。如是,他们便在同一天获取结生并出生。于是,渔夫的长者寻思「这村里有没有其他孩子在今天出生」,见到那些孩子后想「他们会是我儿子的朋友」,便给予全体补贴。他们全都成了朋友,一起玩耍泥巴,渐次长大。耶输延是他们的首领。
\item 那时,地狱里的迦毗罗也以剩余的异熟在不久前转生为金身但口臭的鱼。于是,一天,所有渔夫的孩子拿了网,说「我们去捕鱼」,到了河边撒网。这鱼便进了他们的网里。见后,整个渔村便作高声、大声:「我们的孩子第一次捕鱼就捕了金色的鱼,这些孩子长大了,现在,国王将给我们广大的财富!」于是,这五百个孩子把鱼扔到船里,起航到了国王跟前。国王见后便说:「这是什么?我说!」「鱼,陛下!」国王见到鱼的金身,想「世尊会知道它金色的原因」,便让人捉了鱼,去到世尊跟前。当鱼开口时,祇园便极度恶臭。
\item 国王便问世尊:「尊者!为什么鱼生而金身?又为什么它口吐恶臭?」「大王!他曾是迦叶世尊的教法下名为迦毗罗的比丘,多闻且精通阿含,对不接受自己话语的比丘骂詈、指责,且令彼世尊的教法败坏,以令彼世尊教法败坏的业,转生于无间大地狱,并以剩余的异熟现在生而为鱼,以长时教说佛语、赞美佛陀的等流,便获这般颜色,以曾骂詈、指责比丘,它便口吐恶臭。」「我要让它开口吗?大王!」「唯!世尊!」
\item 于是,世尊便与鱼谈话:「你是迦毗罗吗?」「唯!世尊!我是迦毗罗。」「你从何而来?」「从无间大地狱,世尊!」「输陀那去往何处?」「已般涅槃,世尊!」「娑陀尼去往何处?」「已转生至大地狱,世尊!」「多般那去往何处?」「已转生至大地狱,世尊!」「你现在将去何处?」「大地狱,世尊!」随即为追悔所胜,以头触船而死,转生至大地狱。大众悚惧,身毛为竖。于是,世尊为向到此的在家、出家会众开示随适于此时的法,便说了此经。\end{enumerate}

\subsection\*{\textbf{277} {\footnotesize 〔PTS 274〕}}

\textbf{法行、梵行,他们说这是最上的财富。\\}
\textbf{若他即便已出家,从家至于非家,}

Dhammacariyaṃ brahmacariyaṃ, etad āhu vasuttamaṃ;\\
pabbajito pi ce hoti, agārā anagāriyaṃ. %\hfill\textcolor{gray}{\footnotesize 1}

\begin{enumerate}\item 这里,\textbf{法行},即身善行等的法行。\textbf{梵行},即道梵行\footnote{道梵行:见\textbf{吉祥经}第 270 颂注。}。\textbf{他们说这是最上的财富},即这二种世、出世间的善行,由能导向天界、解脱之乐,圣者们说是最上的财富。最上的财富,即最上之宝,意为追随、依附于自我,不与国王等共。
\item 至此已显示「对在家人或出家人,唯正当的行道为庇护」,现在,以显示无有行道的出家人的不实性来指责迦毗罗及其他如此者,说了「若他即便已出家」等。此处,其释义为:\textbf{若}任何人除去了在家相,\textbf{他即便}仅以执持秃头、袈裟等\textbf{已出家},以先前所说之义\textbf{从家至于非家},\end{enumerate}

\subsection\*{\textbf{278} {\footnotesize 〔PTS 275〕}}

\textbf{若他生性饶舌,乐于恼害,粗野,\\}
\textbf{则其活命甚恶,自身的尘垢增长。}

So ce mukharajātiko, vihesābhirato mago;\\
jīvitaṃ tassa pāpiyo, rajaṃ vaḍḍheti attano. %\hfill\textcolor{gray}{\footnotesize 2}

\begin{enumerate}\item \textbf{若他生性饶舌}、恶口,由乐于以种种行相恼害而\textbf{乐于恼害},以无有惭愧、犹如野兽而\textbf{粗野},\textbf{则其活命甚恶},则如此之人的活命极恶、极低劣。为什么?因为以此邪行道,\textbf{自身的}贪等多种\textbf{尘垢增长}。\end{enumerate}

\subsection\*{\textbf{279} {\footnotesize 〔PTS 276〕}}

\textbf{乐于争辩的比丘,为愚痴之法所遮蔽,\\}
\textbf{即便被告知,也不了知佛陀开示的法。}

Kalahābhirato bhikkhu, mohadhammena āvuto;\\
akkhātam pi na jānāti, dhammaṃ Buddhena desitaṃ. %\hfill\textcolor{gray}{\footnotesize 3}

\begin{enumerate}\item 且不仅仅以此原因其活命甚恶,而且这样的\textbf{比丘}由生性饶舌而\textbf{乐于争辩},\textbf{为}惑于分晓善说之义的\textbf{愚痴之法所遮蔽},\textbf{即便被}诸正派的比丘们以「朋友迦毗罗!莫如是说,请以此法门把握之」等方法\textbf{告知},\textbf{也不了知佛陀开示的法}。佛陀开示的法,即便被以种种行相告知自己,也不了知。如是,其活命亦甚恶。\end{enumerate}

\subsection\*{\textbf{280} {\footnotesize 〔PTS 277〕}}

\textbf{恼乱修己者,由无明前导,\\}
\textbf{不知杂染与趣向地狱之道。}

Vihesaṃ bhāvitattānaṃ, avijjāya purakkhato;\\
saṅkilesaṃ na jānāti, maggaṃ nirayagāminaṃ. %\hfill\textcolor{gray}{\footnotesize 4}

\begin{enumerate}\item 同样,这样的人由乐于恼乱而\textbf{恼乱修己者},即以「你们,年长的出家人!不知律,不知经、阿毗达摩」等方式恼乱修己的漏尽比丘,如输陀那等——此属格用作业格\footnote{这说的是 bhāvitattānaṃ 一词的属格形式。}。或者,仍以如前所说的方法,当知文本为「对修己者进行恼乱 \textit{vihesaṃ bhāvitattānaṃ karonto}」的省略,如是则径作属格即可\footnote{义注这样解释时,是补全了动词「进行 \textit{karonto}」,同时将「恼乱 \textit{vihesaṃ}」作为直接宾语而非现在分词的体格,将「修己者 \textit{bhāvitattānaṃ}」作为间接宾语。}。
\item \textbf{由无明前导},当恼乱修己者时,由遮蔽得见过患的无明前导、指派、发起,\textbf{不知}以恼乱其余修己的出家人之相所转起者,于现法因障碍心为\textbf{杂染},且于未来因导向地狱为\textbf{趣向地狱之道}。\end{enumerate}

\subsection\*{\textbf{281} {\footnotesize 〔PTS 278〕}}

\textbf{落入堕处,从胎到胎,从暗到暗,\\}
\textbf{这样的比丘,死后必然经历痛苦。}

Vinipātaṃ samāpanno, gabbhā gabbhaṃ tamā tamaṃ;\\
sa ve tādisako bhikkhu, pecca dukkhaṃ nigacchati. %\hfill\textcolor{gray}{\footnotesize 5}

\begin{enumerate}\item 且不知者,以此道\textbf{落入}分为四种苦处的\textbf{堕处}。且于此堕处,\textbf{从胎到胎,从暗到暗},于一一部类百遍、千遍地从母胎到母胎,并从日月也无法摧破的阿修罗众之黑暗落入黑暗。\textbf{这样的比丘,死后必然}从此去往他界,如这迦毗罗鱼一般,\textbf{经历}种种品类的\textbf{痛苦}。\end{enumerate}

\subsection\*{\textbf{282} {\footnotesize 〔PTS 279〕}}

\textbf{好比是经年满盈的粪坑,\\}
\textbf{这样的人也是如此,因有垢而难清洗。}

Gūthakūpo yathā assa, sampuṇṇo gaṇavassiko;\\
yo ca evarūpo assa, dubbisodho hi sāṅgaṇo. %\hfill\textcolor{gray}{\footnotesize 6}

\begin{enumerate}\item 什么原因?\textbf{好比是经年满盈的粪坑},好比厕所经年、多年的粪坑,许多年被粪填满至顶而满盈,它即便被百瓶水、千瓶水清洗,由去不掉恶臭、恶色而难清洗。如是,\textbf{这样的人也是如此},长久以杂染营生,如粪坑为粪(充满)一般,由为恶充满而为有垢之人,他\textbf{因有垢而难清洗},即便久后经历这垢的异熟,也不得清净,所以以年计数,即便历时无量,这样的比丘死后必然经历痛苦。
\item 或者,此颂的连接为:即如所说「这样的比丘,死后必然经历痛苦」,于此,若你们问:他死后能不经历痛苦吗?不能。为什么?因为「好比是……难清洗」。\end{enumerate}

\subsection\*{\textbf{283} {\footnotesize 〔PTS 280〕}}

\textbf{你们得知这样的人,诸比丘!依赖俗家,\\}
\textbf{恶欲,恶思惟,恶正行与行处,}

Yaṃ evarūpaṃ jānātha, bhikkhavo gehanissitaṃ;\\
pāpicchaṃ pāpasaṅkappaṃ, pāpa-ācāragocaraṃ. %\hfill\textcolor{gray}{\footnotesize 7}

\begin{enumerate}\item 既然事先\textbf{你们得知这样的人,诸比丘!依赖俗家},即你们能够了知这样依赖种种五欲的人,由具足转起希求不实功德之行相的恶欲为\textbf{恶欲},由具足欲寻等为\textbf{恶思惟},由具足身违犯等及布施竹等类的恶正行为\textbf{恶正行}\footnote{恶正行:见\textbf{清净道论}·说戒品第 44 段。},由娼妓等恶行处为\textbf{恶行处}\footnote{恶行处:见\textbf{清净道论}·说戒品第 45 段。}。\end{enumerate}

\subsection\*{\textbf{284} {\footnotesize 〔PTS 281〕}}

\textbf{你们应当全体和合后,远离他,\\}
\textbf{应当清扫垃圾,应当清除沉滓。}

Sabbe samaggā hutvāna, abhinibbajjiyātha naṃ;\\
kāraṇḍavaṃ niddhamatha, kasambuṃ apakassatha. %\hfill\textcolor{gray}{\footnotesize 8}

\begin{enumerate}\item \textbf{你们应当全体和合后,远离他},这里,远离,即回避、莫结交,且莫以仅仅远离他便作少待,而\textbf{应当清扫垃圾,应当清除沉滓},这人作为尘土,如尘土般,应不顾而清扫,作为渣滓,他进入刹帝利等之中,如破裂、滴淌的麻风旃陀罗般,应当清除,捉住手或头后驱逐。好比尊者大目犍连,以臂捉住那恶法之人,令出门廊之外,便落下闩——即示以应如是清除。什么原因?伽蓝者,乃是为具戒者所造,非为恶戒者。\end{enumerate}

\subsection\*{\textbf{285} {\footnotesize 〔PTS 282〕}}

\textbf{然后清除自认为沙门的非沙门之糠,\\}
\textbf{清扫了恶欲、恶正行与行处者,}

Tato palāpe vāhetha, assamaṇe samaṇamānine;\\
niddhamitvāna pāpicche, pāpa-ācāragocare. %\hfill\textcolor{gray}{\footnotesize 9}

\begin{enumerate}\item \textbf{然后清除自认为沙门的非沙门之糠},因为好比糠,没有内在的粒实,由外在的稃而看似稻谷,如是恶比丘没有内在的戒等,由外在的袈裟等资具而看似比丘,所以被称为糠。清除、筛选这些糠,驱赶第一义上的非沙门、仅以衣装而自认为沙门者。\end{enumerate}

\subsection\*{\textbf{286} {\footnotesize 〔PTS 283〕}}

\textbf{让清净、遵从者与清净者共住!\\}
\textbf{然后,和合而贤明,你们得尽苦的边际!}

Suddhā suddhehi saṃvāsaṃ, kappayavho patissatā;\\
tato samaggā nipakā, dukkhass’antaṃ karissathā ti. %\hfill\textcolor{gray}{\footnotesize 10}

\begin{enumerate}\item 如是\textbf{清扫了……共住}!这里,遵从,即彼此具尊重、具遵从。\textbf{然后,和合而贤明,你们得尽苦的边际},即从此,如是,你们清净者与清净者共住,以共同的见、戒而和合,以次第成熟的慧而贤明,你们得尽这一切轮回之苦等苦的边际!即以阿罗汉为顶点完成了开示。
\item 在开示的终了,这五百渔夫子生起悚惧,希求尽苦边际,在世尊跟前出家已,不久便尽苦边,与世尊一起受用同一个不动住等至法。且彼等如是与世尊作此同一受用,当以「自说」所说的耶输延经\footnote{耶输延经 \textit{Yasojasutta}:即\textbf{自说}第 3:3 经。}而知。\end{enumerate}

\begin{center}\vspace{1em}法行经第六\\Dhammacariyasuttaṃ chaṭṭhaṃ.\end{center}