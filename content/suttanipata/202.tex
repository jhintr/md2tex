\section{生腥经}

\begin{center}Āmagandha Sutta\end{center}\vspace{1em}

\begin{enumerate}\item 缘起为何?世尊未出世时,名为生腥的婆罗门与五百学童一起出家为苦行者,进入雪山后,教人在山间建了草庵,以林木的根果为食,在那里定居,不吃任何鱼肉。于是,这些苦行者不食盐、酸等,便得了黄疸病。随后,他们想「为了服用盐、酸等,我们去人境吧」,到达边鄙的村庄。那里的人们便令他们净喜,请令进食,当他们食事已毕,便提供桌椅、水皿、涂足(油)等说:「尊者!请住在此,莫要不安!」示以住处后,便离开了。第二天,仍为他们布施,然后,便每天逐家布施。众苦行者在那里住了四月,因服用盐、酸等而身体强壮,便告知人们:「朋友!我们将行。」人们便施以油、米等。他们取了后,仍回到自己的草庵。他们每年便这样前往这村庄。人们也知道他们何时前来,备好了米等等待布施,并在到来时仍如此礼遇他们。
\item 于是,世尊出世,转起无上法轮,渐次到了舍卫国,当住在那里时,见到这些苦行者具足近依后,便从那里出发,为比丘僧团所随从而游行,渐次到达这村庄。人们见到世尊,便作了大布施。世尊为他们开示了法。他们因此法的开示,有些证得须陀洹,有些证得斯陀含,有些证得阿那含,有些出家证得阿罗汉。世尊便再次回到舍卫国。
\item 于是,那些苦行者回到这村庄,人们见到苦行者后,便不像先前那样热切。众苦行者便问:「朋友!为什么这些人不像先前那样?难道这村庄为王刑所扰,或是饥馑,或是某些比我们更具戒等德的出家人到了这村庄?」他们便说:「尊者!这村庄并非为王刑或饥馑所扰,而是佛陀出世,彼世尊为众人的利益开示法而来至此处。」
\item 生腥苦行者听后说:「长者们!你们说『佛陀』?」「尊者!我们说『佛陀』。」说了三遍后,他心满意足,发出满足之语「在世间,连听闻这『佛陀』之声也难得」,便问到:「这佛陀吃不吃生腥?」「尊者!什么是生腥?」「长者们!生腥者,即是鱼肉。」「尊者!世尊吃鱼肉。」苦行者听后便后悔:「那他不会是佛陀了!」但又想:「诸佛之显现极难,去后见了佛陀,问问就知道!」随后,他沿着佛陀所行的道路询问众人,像是渴望牛犊的母牛般,急匆匆在各处只住一夜,到了舍卫国便与自己的随从进了祇林。
\item 尔时,世尊正为开示法而坐于坐处。众苦行者前往世尊处,未予行礼便默然坐在一边。世尊以「众仙人!你们还好吗?」等方法与他们相问候。他们也说了「乔达摩君!还好」等等。随后,生腥便问世尊:「乔达摩君!你吃不吃生腥?」「婆罗门!你说的这生腥是什么?」「鱼肉,乔达摩君!」世尊说:「婆罗门!鱼肉并非生腥,生腥者,是一切烦恼、恶、不善法,婆罗门!不是只有你现在问起生腥,过去名为低舍的婆罗门曾问过迦叶世尊,他也如是问,而世尊为其如是答。」为令婆罗门知晓,便引出低舍婆罗门与迦叶世尊所说的偈颂,以这些偈颂说了此经。以上即此经当前的缘起。
\item 而在过去,据说,迦叶菩萨圆满了八阿僧祇劫又十万劫的波罗蜜,便在波罗奈梵赐婆罗门名为俱财的婆罗门尼的胎内获取结生。上首弟子也在那天从天界下堕,转生于副祭司妻子的胎中。如是,他俩同一天获取结生,同一天出胎,一名迦叶,一名低舍。这两个好友一起玩耍泥巴,渐渐长大。低舍的父亲便命儿子:「亲爱的!迦叶离欲出家后将成佛陀,你也应在他跟前出家,出离于有!」他答道「善哉」,去到菩萨跟前说:「兄弟!我俩将来都去出家吧!」菩萨答以「善哉」。
\item 随后,当成年时,低舍便对菩萨说:「来!兄弟!我们去出家!」菩萨却不离开。低舍想「其智未至成熟」,便自己离开,出家为仙人,在山麓的林野建了草庵而住。后来,菩萨尚在俗家,即领悟了入出息念,增长了四种禅那与众多神通,便带着宫殿去到菩提座附近,决意「让宫殿在合适之处落下」,而它唯落在原地——据说,未出家者无法接近菩提座。他便出家,到达菩提座坐下,在七日内精勤修行,以七日证得正等正觉。那时,在仙人堕处有二万出家人定居。于是,迦叶世尊便对他们演说,转起法轮。在经的终了,全都成了阿罗汉。彼世尊为二万比丘所随从,就在仙人堕处住下。且迦尸国王松鸦以四资具护持他。
\item 于是,某天,一个在波罗奈居住的人为在山里寻找旃檀心材等,到了低舍苦行者的草庵,顶礼后站在一边。苦行者见到后,便问:「你从哪里来?」「从波罗奈,尊者!」「那里有什么消息吗?」「尊者!名为迦叶的正等正觉者出现在那里。」苦行者听闻到难得之语,便生喜悦,问到:「那么他吃不吃生腥?」「尊者!什么是生腥?」「鱼肉,朋友!」「尊者!世尊吃鱼肉。」苦行者听后便生后悔,但又想:「我去后再问他,要是他说『我吃生腥』,我就以『尊者!这对你这样的出身、家族、族姓都不相称』遮止,在他跟前出家,出离于有!」他轻装前往,在各处只住一夜,在晡时到达并进了仙人堕处。
\item 尔时,世尊正为开示法而坐于坐处。苦行者前往世尊处,未予行礼便默然站在一边。世尊见后,便以先前所说的方法问候。他也说了「迦叶君!还好」等等,坐在一边,便问世尊:「迦叶君!你吃不吃生腥?」「婆罗门!我不吃生腥。」「善哉,善哉!迦叶君!你不吃它者的尸骸,甚是善妙,这与迦叶君的出身、家族、族姓相称。」随后,世尊想「我就烦恼而说『我不吃生腥』,婆罗门却理解为鱼肉,我明天何不不入村乞食,而受用从松鸦国王之家带来的饮食,如是将会就生腥发起谈论,随后,我将会以法的开示令婆罗门知晓」,第二天,便按时洗漱完毕,进入香房。众比丘见香房之门关闭,便知晓「世尊今天不欲与众比丘一起入村」,右绕香房已,前去乞食。
\item 世尊便离开香房,坐在设好的坐处。苦行者也煮了菜叶吃完,坐在世尊跟前。迦尸国王松鸦见到行乞的众比丘,问到:「尊者!世尊在哪里?」听说「大王!在寺庙」后,便给世尊送去具有种种调味、各种肉制的饮食。众大臣带到寺庙,告知了世尊,施与供养与水,当施食时,首先施以具有种种肉制的粥。苦行者见后,便站而思索「他会不会吃」。世尊在其注视下喝了粥,把鱼肉放入口中。苦行者见而忿怒。喝完粥,他们又施以种种调味的饮食,他取来吃下,见后便极忿怒:「吃着鱼肉,却说『我不吃』。」于是,世尊食事已毕,洗濯手足而坐,便前往说到:「迦叶君!你说谎,这非智者所为。因为妄语为诸佛谴责,那些住在山麓以林木的根果等存活的仙人,他们倒不说谎。」为赞扬仙人之德,而说「稗子、穗子、豆子」等等。\end{enumerate}

\subsection\*{\textbf{242} {\footnotesize 〔PTS 239〕}}

\textbf{「稗子、穗子、豆子,以及绿叶、根茎、蔓果,\\}
\textbf{「善人们吃着如法的所得,不会因爱欲而说谎。}

“Sāmāka-ciṅgūlaka-cīnakāni ca, pattapphalaṃ mūlaphalaṃ gavipphalaṃ;\\
dhammena laddhaṃ satam asnamānā, na kāmakāmā alikaṃ bhaṇanti. %\hfill\textcolor{gray}{\footnotesize 1}

\begin{enumerate}\item 这里,\textbf{稗子}是脱壳后的稻穗,或经挑拣后残剩的草谷。同样,\textbf{穗子}是夹竹桃花状的穗。\textbf{豆子}是在森林、山麓未经培育的豆。\textbf{绿叶},即任何新鲜的叶子。\textbf{根茎},即任何球根。\textbf{蔓果},即任何藤蔓之果。或者,当知以根摄球根,以茎摄藤蔓之果,以蔓果摄水生的十字形菱角等果。
\item \textbf{如法的所得},即舍弃差役走使等的邪命,在林中通过拾掇的所得。\textbf{善人们},即善的圣者们。\textbf{吃},即受用。\textbf{不会因爱欲而说谎},即无我所、无所有、受用这些稗子等的众仙人,不像你那样,希求着美味等的爱欲,受用着生腥,却说着「婆罗门!我不吃生腥」而说谎,他们不会因爱欲而说谎,以赞叹众仙人而表明非难。\end{enumerate}

\subsection\*{\textbf{243} {\footnotesize 〔PTS 240〕}}

\textbf{「吃着善预备、善烹饪,由他人施与、馈赠的美味,\\}
\textbf{「享用着稻粱,迦叶!你在受用生腥。}

Ya-d-asnamāno sukataṃ suniṭṭhitaṃ, parehi dinnaṃ payataṃ paṇītaṃ;\\
sālīnam annaṃ paribhuñjamāno, so bhuñjasī Kassapa āmagandhaṃ. %\hfill\textcolor{gray}{\footnotesize 2}

\begin{enumerate}\item 如是,以赞叹众仙人非难了世尊后,现在则以自己显示了非难之事的意趣,为直接非难世尊说了此颂。这里,\textbf{d} 字作词的连接。而其义为:对任何的兔肉或是鹌鹑肉,以清洗、切割等前务\textbf{善预备},以烹煮、腌制等后务\textbf{善烹饪},非由父母,而是\textbf{由}认为「他是应供」的爱法的\textbf{他人施与},以恭敬之行\textbf{馈赠}的美味佳肴,\textbf{吃着}、食用着具有最上之味、营养丰富、堪能维持体力的\textbf{美味},且不仅仅是各种肉,还有稻粱,\textbf{享用着}拣去砂砾的稻粒饭,\textbf{迦叶!你在受用生腥},你受用着各种肉,享用着这稻粱,迦叶!你在受用生腥,即以族姓称呼世尊。\end{enumerate}

\subsection\*{\textbf{244} {\footnotesize 〔PTS 241〕}}

\textbf{「你却说『生腥不适宜我』,梵天的眷属!\\}
\textbf{「享用着稻粱,与善料理的禽肉,\\}
\textbf{「我问你,迦叶!怎样对你才算生腥?」}

‘Na āmagandho mama kappatī’ ti, icc eva tvaṃ bhāsasi Brahmabandhu;\\
sālīnam annaṃ paribhuñjamāno, sakuntamaṃsehi susaṅkhatehi;\\
pucchāmi taṃ Kassapa etam atthaṃ, kathaṃpakāro tava āmagandho”. %\hfill\textcolor{gray}{\footnotesize 3}

\begin{enumerate}\item 如是从食物非难世尊已,现在则归属于妄语而非难,便说「你却说……禽肉」。其义为:先前我问起时,\textbf{你却说「生腥不适宜我」},你如是确定地说,\textbf{梵天的眷属}!即指责道:无有婆罗门的功德,仅以出身为婆罗门。\textbf{稻粱},即稻粒饭。\textbf{享用},即受用。\textbf{善料理的禽肉},此时,他正指着为世尊拿来的禽肉而说。
\item 正如是说时,他又下起足底、上至发尖地察视世尊的身体,见到三十二胜相、八十随形好的成就与周匝一寻的光明,想道:「这样为大人相等所严饰身体之人,不应该说谎,因为他唯以有间真实语的等流,才能在眉间生出洁白、柔软、如同木棉的白毫,且一一毛孔唯有一毛,他现在又如何能说谎呢?看来他的生腥另有所指,就彼而言『婆罗门!我不吃生腥』,我何不问彼为何?」生起敬意,仍以族姓相称,说了余颂:「\textbf{我问你,迦叶!怎样对你才算生腥?}」\end{enumerate}

\subsection\*{\textbf{245} {\footnotesize 〔PTS 242〕}}

\textbf{「杀生,鞭打、割截、捆缚,盗窃、妄语,虚诳、欺诈,\\}
\textbf{「研究无义之事,亲近他人妻妾,这是生腥,而非食肉。}

“Pāṇātipāto vadha-cheda-bandhanaṃ, theyyaṃ musāvādo nikati-vañcanāni ca;\\
ajjhenakuttaṃ paradārasevanā, esāmagandho na hi maṃsabhojanaṃ. %\hfill\textcolor{gray}{\footnotesize 4}

\begin{enumerate}\item 于是,世尊为他解答生腥而说了「杀生」等等。这里,\textbf{杀生},即杀戮生类。\textbf{鞭打、割截、捆缚}中,以棍杖等敲击有情为鞭打,切断手足等为割截,以绳等捆缚为捆缚。\textbf{虚诳},即以「我会施、我会做」等方法先给人以希望,再令落空。\textbf{欺诈},即让人把非黄金当作黄金,等等。\textbf{研究无义之事},即学习各种无义的典籍。\textbf{亲近他人妻妾},即于他人所拥有的而行交际。
\item \textbf{这是生腥,而非食肉},即这杀生等的不善法之行是生腥、腥臭、尸臭。什么原因?由令人不愉悦,由掺杂烦恼、不净,由为善人所嫌厌,以及由带来最上的恶臭之相。因为有情烦恼增盛,因而变得极为恶臭,然而无烦恼者,即便其死尸也不臭,所以说「这是生腥」。至于食肉,若不见、不闻、不疑则无过,所以食肉不是生腥。\end{enumerate}

\subsection\*{\textbf{246} {\footnotesize 〔PTS 243〕}}

\textbf{「若人在此于爱欲不自制,贪图众味,掺杂不净,\\}
\textbf{「持空无见,不正,固执,这是生腥,而非食肉。}

Ye idha kāmesu asaññatā janā, rasesu giddhā asucibhāvamassitā;\\
natthikadiṭṭhī visamā durannayā, esāmagandho na hi maṃsabhojanaṃ. %\hfill\textcolor{gray}{\footnotesize 5}

\begin{enumerate}\item 如是由基于法的开示,以一种方法解答了生腥,现在,因为种种有情具有种种生腥,而非一具所有,亦非所有具一,所以为向彼等阐明种种生腥,以「若人于此于爱欲不自制」等方法,先由基于人的开示,为解答生腥,说了二颂。
\item 这里,\textbf{若人在此于爱欲不自制},即任何在此世间的凡夫,于被称为受用爱欲的爱欲中,连母亲、姨母等的禁忌也罔顾,以破除防护而不自制。\textbf{贪图众味},即于舌所识知的众味,贪图、束缚、沉迷、染著,不见其过患,无出离之慧而享用众味。\textbf{掺杂不净},即因此贪图众味,为了获得众味,掺杂被称为种种品类邪命的不净。
\item \textbf{持空无见},即具足「无所施」为首的十事邪见。\textbf{不正},即具足不正的身业等。\textbf{固执},即因难教化而执取己见、顽固、难以舍弃。\textbf{这是生腥},即在此颂中,以基于人所示的「于爱欲不自制、贪图众味、欠缺活命、持空无见、身恶行等不正、固执」等另六种,以先前所说之义,当知这是生腥。\textbf{而非食肉},而食肉,仍以所说之义,并非生腥。\end{enumerate}

\subsection\*{\textbf{247} {\footnotesize 〔PTS 244〕}}

\textbf{「若粗鄙,强暴,背后噬人,背叛朋友,毫无悲悯而傲慢,\\}
\textbf{「生性吝啬,且不施与任何人,这是生腥,而非食肉。}

Ye lūkhasā dāruṇā piṭṭhimaṃsikā, mittadduno nikkaruṇātimānino;\\
adānasīlā na ca denti kassaci, esāmagandho na hi maṃsabhojanaṃ. %\hfill\textcolor{gray}{\footnotesize 6}

\begin{enumerate}\item 在第二颂中,\textbf{粗鄙},即无味,从事于自我折磨之义。\textbf{强暴},即酷虐、难以劝谏。\textbf{背后噬人},即当面甜言蜜语,而在别人面前诋毁。因为他们不能当面直视,而似啃咬背后之肉般,因此被称为「背后噬人」。\textbf{背叛朋友},即伤害朋友,即是说在妻妾、财产、生命等过从亲厚的朋友处行邪行。\textbf{毫无悲悯},即无有悲悯,不欲有情得义利。\textbf{傲慢},即具足如\begin{quoting}于此,有人以出身……或以其它依处藐视他人,像这样的慢、心的欲求显赫。(分别论第 880 段)\end{quoting}等所说的傲慢。
\item \textbf{生性吝啬},即天性不施、决意不施、不喜分享之义。\textbf{且不施与任何人},即因生性吝啬,即便受到乞请也不施与任何人任何物,与从未布施之家中的人相似,以烧渴的饿鬼为归宿。有人读作「生性好取 \textit{ādānasīlā}」,即仅习惯索取,却不给任何人任何物。\textbf{这是生腥,而非食肉},即在此颂中,以基于人所示的「粗鄙、强暴、背后噬人、背叛朋友、毫无悲悯、傲慢、生性吝啬、不施」等另八种,以先前所说之义,当知这是生腥,而非食肉。\end{enumerate}

\subsection\*{\textbf{248} {\footnotesize 〔PTS 245〕}}

\textbf{「忿怒,㤭慢,顽固,敌对,伪善,妒忌,言谈吹嘘,\\}
\textbf{「慢过慢,与不善者亲密,这是生腥,而非食肉。}

Kodho mado thambho paccupaṭṭhāpanā, māyā usūyā bhassasamussayo ca;\\
mānātimāno ca asabbhi santhavo, esāmagandho na hi maṃsabhojanaṃ. %\hfill\textcolor{gray}{\footnotesize 7}

\begin{enumerate}\item 如是,由基于人的开示说了二颂后,又了知到此苦行者所随顺的意趣,仍由基于法的开示,说了一颂。这里,\textbf{忿怒},当知如蛇经(第 1 颂)中所说。\textbf{㤭慢},即分别论中以\begin{quoting}出身㤭慢、族姓㤭慢、无病㤭慢……(分别论第 832 段)\end{quoting}等方法所说种类的心的迷醉相。\textbf{顽固},即坚硬相。\textbf{敌对},即置于对立,对如法如理的所说加以反驳。
\item \textbf{伪善}\footnote{伪善 \textit{māyā}:玄奘译为「诳」。},即分别论中以\begin{quoting}于此,有人以身行恶行……(分别论第 894 段)\end{quoting}等方法所分别的对已作之恶的覆藏。\textbf{妒忌},即嫉妒他人的利养恭敬等。\textbf{言谈吹嘘},即夸大的言谈,即是说自赞。
\item \textbf{慢过慢}\footnote{慢过慢 \textit{mānātimāna}:这里沿用玄奘的译语。},即分别论中所分别的\begin{quoting}于此,有人以出身……或以其它依处,先时认为自己与他人等同,后时认为自己优胜,认为他人低劣,像这样的慢、心的欲求显赫。(分别论第 880 段)\end{quoting}\textbf{与不善者亲密},即与不善人亲密。\textbf{这是生腥,而非食肉},即忿怒等九种不善聚,以先前所说之义,当知这是生腥,而非食肉。\end{enumerate}

\subsection\*{\textbf{249} {\footnotesize 〔PTS 246〕}}

\textbf{「若生性为恶,毁弃债务且中伤,扭曲诉讼,于此假装,\\}
\textbf{「若卑鄙之人于此造作罪行,这是生腥,而非食肉。}

Ye pāpasīlā iṇaghāta-sūcakā, vohārakūṭā idha pāṭirūpikā;\\
narādhamā ye’dha karonti kibbisaṃ, esāmagandho na hi maṃsabhojanaṃ. %\hfill\textcolor{gray}{\footnotesize 8}

\begin{enumerate}\item 如是,由基于法的开示显示了九种生腥,再以先前所说的方法,由基于人的开示解答生腥,对低舍说了几颂。这里\textbf{生性为恶},即因恶行而在世间以「生性为恶」知名。\textbf{毁弃债务且中伤},即以贱民经(第 120 颂)中所说的方法,借了债却不还为「毁弃债务」,并以两舌为「中伤」。
\item \textbf{扭曲诉讼,于此假装},即在裁断处收取贿赂,击败所有者,由具足扭曲的诉讼为「扭曲诉讼」,由假装裁断为「假装」。或者,「于此」即于教法内,「假装」即恶戒。因为他们因具足彼等的威仪而假装具戒,所以为假装。
\item \textbf{若卑鄙之人于此造作罪行},即若于此世间,卑鄙之人于父母、佛、辟支佛等处造作被认为是邪行道的罪行。\textbf{这是生腥,而非食肉},即在此颂中,以基于人所示的「生性为恶、毁弃债务、中伤、扭曲诉讼、假装、造作罪行」等另六种,以先前所说之义,当知这是生腥,而非食肉。\end{enumerate}

\subsection\*{\textbf{250} {\footnotesize 〔PTS 247〕}}

\textbf{「若人在此于生命不自制,夺取他人的并实施伤害,\\}
\textbf{「恶戒、残忍,恶口,不敬,这是生腥,而非食肉。}

Ye idha pāṇesu asaññatā janā, paresam ādāya vihesam uyyutā;\\
dussīlaluddā pharusā anādarā, esāmagandho na hi maṃsabhojanaṃ. %\hfill\textcolor{gray}{\footnotesize 9}

\begin{enumerate}\item \textbf{若人在此于生命不自制},即那些人在此世间,于生命随所欲为,杀害百千,因毫无哀悯而不自制。\textbf{夺取他人的并实施伤害},即夺取他人的 \textit{paresam ādāya} 所有,或是财产,或是生命,随后对以「莫如是做」的乞请者或遮止者,用拳头、土块、棍杖等实施伤害。或者,受取其他 \textit{pare samādāya} 有情,即以「今天十个、今天二十个」等受取已,以杀戮、捆缚等实施伤害\footnote{义注在此对 paresamādāya 作了两种语法分析,即 paresam-ādāya 与 pare-samādāya,颂中的译文据第一种作出。}。
\item \textbf{恶戒、残忍},由恶行而离于戒,且以手沾鲜血的凶残行为为残忍,这里指的是捕鱼、缚鹿、罗鸟等。\textbf{不敬},即如「我们现在不做、不远离这些」般无有敬意。\textbf{这是生腥,而非食肉},即在此颂中,以基于人所示的,以「杀生,鞭打、割截、捆缚」等方法在先前已说及未说的「于生命不自制、伤害他人、恶戒、残忍、恶口、不敬」等另六种,当知这是生腥,而非食肉。
\item 而先前已说的,是为了听众乐闻、强调、巩固等原因而再次地说。因此在后面(第 254 颂)说「如是,世尊已再三宣说此义,通晓颂诗者业已明了」。\end{enumerate}

\subsection\*{\textbf{251} {\footnotesize 〔PTS 248〕}}

\textbf{「于此贪求、对立、杀生,总是热衷,死后前往暗冥,\\}
\textbf{「众有情落入地狱,头朝下方,这是生腥,而非食肉。}

Etesu giddhā viruddhātipātino, nicc’uyyutā pecca tamaṃ vajanti ye;\\
patanti sattā nirayaṃ avaṃsirā, esāmagandho na hi maṃsabhojanaṃ. %\hfill\textcolor{gray}{\footnotesize 10}

\begin{enumerate}\item \textbf{于此贪求、对立、杀生},即于此生命,因贪而贪求,因嗔而对立,因痴而不见过患,再三以违犯而杀生,或者于此以「杀生,鞭打、割截、捆缚」等方法所说的恶业,分别以被称为贪求、对立、杀生的贪、嗔、痴而为贪求、对立、杀生者。\textbf{总是热衷},即总是热衷于造作不善,从未以反省而戒离。
\item \textbf{死后},即从此世间去往后世。\textbf{前往暗冥,众有情落入地狱,头朝下方},即他们前往被称为世间间隔之黑暗或低劣之家等类的暗冥,且众有情落入阿鼻等类的地狱,头朝下方。\textbf{这是生腥},即对这些有情,前往暗冥、落入地狱之因的贪求、对立、杀生等,作为所有生腥之根本,以所说之义为三种生腥。\textbf{而非食肉},食肉却非生腥。\end{enumerate}

\subsection\*{\textbf{252} {\footnotesize 〔PTS 249〕}}

\textbf{「不是鱼肉、断食,不是裸行,不是秃头、萦发污身、\\}
\textbf{「粗皮,不是火供侍奉,或世间众多旨在不死的苦行,\\}
\textbf{「以及颂诗、祭品、献牲、时习,能够净化未度疑惑的有死者。}

Na macchamaṃsānam anāsakattaṃ, na naggiyaṃ na muṇḍiyaṃ jaṭājallaṃ;\\
kharājināni nāggihuttass’upasevanā, ye vā pi loke amarā bahū tapā;\\
mantāhutī yaññam utūpasevanā, sodhenti maccaṃ avitiṇṇakaṅkhaṃ. %\hfill\textcolor{gray}{\footnotesize 11}

\begin{enumerate}\item 如是,世尊从第一义解答了生腥,且阐明了其恶趣之道之相,现在,苦行者于此鱼肉之食作生腥想、作恶趣之道想,欲以不食彼而清净故不食彼,为显明此及类此者的不堪净化之相,说了这六句之颂。这里,所有句子都应与末句相连,如「不是鱼肉能够净化未度疑惑的有死者、不是祭品、献牲、时习能够净化未度疑惑的有死者」。
\item 且此中,古人解释说,\textbf{不是鱼肉},即不吃鱼肉不能净化,\textbf{断食}也同样。然而若如是则更善:断食鱼肉不能净化有死者。那么,在此情况下,断食是否被略去了呢?并非如此,由旨在不死的苦行所摄故。因为此中以「或世间众多旨在不死的苦行」归纳了所有未说的自我折磨。
\item \textbf{裸行},即无衣。\textbf{萦发污身},即萦发与泥污。\textbf{粗皮},即粗糙的羊皮。\textbf{火供侍奉},即敬事火。\textbf{旨在不死},即以希求不死的状态而转起的身烦恼。\textbf{众多},即由竭力蹲踞等为多种。\textbf{苦行},即身体的折磨。\textbf{颂诗},即吠陀。\textbf{祭品},即火供之业。\textbf{献牲、时习},即马牲等献牲与时习。时习,即在热季站在烈日下,在雨季住在树下,在寒季进入水中。
\item \textbf{不能够净化未度疑惑的有死者},即不能以清净烦恼或清净有来净化未度疑惑的有死者。因为当有疑惑之尘垢,则无清净,而你正有疑惑。且此中「未度疑惑」,当知不是指听闻世尊所说的「不是鱼肉……」后,苦行者生起的疑惑:「是否以不吃鱼肉等便是清净之道?」而是就他听到「他吃鱼肉」而对诸佛生起的疑惑来说的。\end{enumerate}

\subsection\*{\textbf{253} {\footnotesize 〔PTS 250〕}}

\textbf{「守护众流,知根而行,住立于法,乐于正直、柔和,\\}
\textbf{「超越执著,舍弃一切苦,智者不染于所见所闻。」}

Sotesu gutto viditindriyo care, dhamme ṭhito ajjavamaddave rato;\\
saṅgātigo sabbadukkhappahīno, na lippati diṭṭhasutesu dhīro”. %\hfill\textcolor{gray}{\footnotesize 12}

\begin{enumerate}\item 如是,在显示断食鱼肉等不堪净化之相后,现在,为显示堪能净化之法,说了此颂。这里,\textbf{众流}即六根,\textbf{守护}即具足根律仪守护,至此显明根律仪防护戒。\textbf{知根而行},即是说以知遍知而知六根已,使其明白而行、而住,至此显明戒清净者的名色区分。\textbf{住立于法},即以圣道住立于应证的四谛之法,以此显明须陀洹地。\textbf{乐于正直、柔和},以此显明斯陀含地。因为斯陀含由造成身的邪曲等及心的坚硬相的贪、嗔的薄而乐于正直、柔和。
\item \textbf{超越执著},即超越贪、嗔的执著,以此显明阿那含地。\textbf{舍弃一切苦},即以舍弃一切流转之苦的因而舍弃一切苦,以此显明阿罗汉地。\textbf{智者不染于所见所闻},即这如是渐次证得阿罗汉、具足坚毅的智者,于所见所闻之法,不为任何烦恼所染。且不仅于所见所闻,于所觉所知也不为所染,确然证得最上的清净。如是,以阿罗汉为顶点完成了开示。\end{enumerate}

\subsection\*{\textbf{254} {\footnotesize 〔PTS 251〕}}

\textbf{如是,世尊已再三宣说此义,通晓颂诗者业已明了,\\}
\textbf{离生腥、无所依、难引导的牟尼已用种种偈颂阐明。}

Icc etam atthaṃ Bhagavā punappunaṃ, akkhāsi naṃ vedayi mantapāragū;\\
citrāhi gāthāhi munī pakāsayi, nirāmagandho asito durannayo. %\hfill\textcolor{gray}{\footnotesize 13}

\begin{enumerate}\item 此后的两颂为结集者所说。其义为:\textbf{如是},迦叶\textbf{世尊}以多首偈颂,由基于法及基于人的开示,为了让苦行者知晓,\textbf{已再三宣说}、谈论、详绎\textbf{此义}。\textbf{通晓颂诗者业已明了},即通晓颂诗、通晓吠陀的低舍婆罗门业已明了、知晓。
\item 什么原因?因为从义、从句、从开示的方法,\textbf{牟尼已用种种偈颂阐明}。何等样人?\textbf{离生腥、无所依、难引导},由无生腥的烦恼之相为离生腥,由无爱、见的所依之相为无所依,由不能以「这更好、这更胜」的任何外道之见来引导为难引导。\end{enumerate}

\subsection\*{\textbf{255} {\footnotesize 〔PTS 252〕}}

\textbf{听闻了佛陀善说的离生腥、除一切苦的语句,\\}
\textbf{他心怀谦卑,便顶礼如来,在此处宣告出家。}

Sutvāna Buddhassa subhāsitaṃ padaṃ, nirāmagandhaṃ sabbadukkhappanūdanaṃ;\\
nīcamano vandi Tathāgatassa, tatth’eva pabbajjam arocayitthā ti. %\hfill\textcolor{gray}{\footnotesize 14}

\begin{enumerate}\item 如是,\textbf{听闻了}这阐明者、\textbf{佛陀善说}、善论的法的开示,听闻了\textbf{离生腥}——即离烦恼轭、\textbf{除一切苦}——即除一切流转之苦的\textbf{语句,他心怀谦卑,便顶礼如来},低舍婆罗门在如来脚下五体投地地顶礼。\textbf{在此处宣告出家},即是说就在此处,低舍婆罗门向坐于坐处的迦叶世尊宣告、乞请出家。
\item 世尊便对他说:「来!比丘!」他便在那刹那具八资具,从空中前往,如百岁长老般顶礼了世尊,不久便通达了声闻波罗蜜之智,成为名为低舍的上首弟子,而第二位则名为婆罗豆婆遮。如是,彼世尊的一双弟子便名为低舍、婆罗豆婆遮。
\item 而我们的世尊则带出低舍婆罗门开头所说的三颂、迦叶世尊中间的九颂、彼时结集者最后的两颂等全部十四颂,使之完整,以生腥经向阿阇黎为首的五百苦行者解答了生腥。婆罗门听后,也同样心怀谦卑,在世尊脚下顶礼,与随从一起乞请出家。世尊便说:「来!诸比丘!」他们也同样得至「来!比丘」的状态,从空中前往,顶礼了世尊,不久全都住于最上的阿罗汉果。\end{enumerate}

\begin{center}\vspace{1em}生腥经第二\\Āmagandhasuttaṃ dutiyaṃ.\end{center}