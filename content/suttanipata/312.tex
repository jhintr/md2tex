\section{二重随观经}

\begin{center}Dvayatānupassanā Sutta\end{center}\vspace{1em}

\textbf{如是我闻。一时世尊住舍卫国东园鹿母讲堂。尔时,世尊在十五布萨这天的满月夜晚,为比丘僧团围绕,坐于露地。}

Evaṃ me sutaṃ— ekaṃ samayaṃ Bhagavā Sāvatthiyaṃ viharati Pubbārāme Migāramātu pāsāde. Tena kho pana samayena Bhagavā tadahuposathe pannarase puṇṇāya puṇṇamāya rattiyā bhikkhusaṅghaparivuto abbhokāse nisinno hoti.

\begin{enumerate}\item 缘起为何?此经的缘起为自身的意乐。因为世尊以自身的意乐开示了此经。这于此是略说,而其详说将在释义中揭晓。
\item 这里,\textbf{如是我闻}等仍如已述。\textbf{东园},即在舍卫城东方之园。\textbf{鹿母讲堂}之中,优婆夷毗舍佉由于被自己的公公、商人鹿子置于母亲的地位而被称为「鹿母」,这鹿母花费了价值九俱胝的大蔓草珠宝,教人建了上下各五百内室的重阁,这尖顶的千室被称为「鹿母讲堂」。即于此鹿母讲堂。
\item \textbf{尔时,世尊},即世尊依舍卫国、住东园鹿母讲堂之时。\textbf{布萨这天},即是说布萨日。\textbf{十五},此即以布萨所摄而言,排除落入其它布萨之语。\textbf{满月夜晚},即以十五日为日数、以无有云翳等的杂染为夜晚功德之成就、以圆满为圆满、以遍满之月为满月的夜晚。\textbf{坐于露地},即在鹿母珍宝重阁僧寮的露地,上无遮蔽之处,坐于设好的最上佛座。\end{enumerate}

\textbf{于是,世尊观察了默然又默然的比丘僧团后,告诸比丘:「诸比丘!若有人问『诸比丘!是何缘由听闻这些善的、圣的、出离的、趣向等觉的法』,应如是对他们说『只是为了对二重法的如实之智』。}

Atha kho Bhagavā tuṇhībhūtaṃ tuṇhībhūtaṃ bhikkhusaṅghaṃ anuviloketvā bhikkhū āmantesi: “‘Ye te, bhikkhave, kusalā dhammā ariyā niyyānikā sambodhagāmino, tesaṃ vo, bhikkhave, kusalānaṃ dhammānaṃ ariyānaṃ niyyānikānaṃ sambodhagāmīnaṃ kā upanisā savanāyā’ ti, iti ce, bhikkhave, pucchitāro assu, te evam assu vacanīyā: ‘yāvad eva dvayatānaṃ dhammānaṃ yathābhūtaṃ ñāṇāyā’ ti.

\begin{enumerate}\item \textbf{默然又默然},即极度默然,或凡所观察之处皆是默然,语默然且身默然。\textbf{观察了比丘僧团后},即对围绕他而坐的数千比丘之量的默然又默然的比丘僧团,为界定开示合适的法而从处处观察「此中有若许须陀洹、若许斯陀含、若许阿那含、若许开始修观的善凡夫,对这比丘僧团开示何等法合适」。
\item \textbf{善法},即举凡以无病之义、无过之义、果报可意之义及生起善巧之义而为善的三十七菩提分法,或显明彼的圣典之法。\textbf{圣的、出离的、趣向等觉的},即以应趋近之义为圣,以从世间出离之义为出离,以趣向被称为等觉的阿罗汉性为趣向等觉。\textbf{是何缘由听闻},即是何缘由、何原因、何目的听闻,即是说你们听闻这些法有何义利。
\item 在「只是为了对二重法的如实之智」中,\textbf{只是},即限定、强调之语。彼等的二部分即\textbf{二重},文本也作 dvayānaṃ。\textbf{如实之智},即无颠倒智。这是说的什么?即为了这以世、出世间等类所确定的二种被称为观的的如实之智,非为过此。因为这些是以听闻得成,而过此则是以修习而证的殊胜。\end{enumerate}

\textbf{「那什么是你们说的二重?『此是苦,此是苦之集』,这是一随观,『此是苦之灭,此是趣向苦灭之行道』,这是第二随观。诸比丘!对如是正确地随观二重的比丘,住于不放逸、热忱、自励,可期待二种果中的一果:在现法中已知,或当有余依时为阿那含。」世尊说了这些。善逝说罢,大师进一步说:}

Kiñ ca dvayataṃ vadetha? ‘Idaṃ dukkhaṃ, ayaṃ dukkhasamudayo’ ti ayam ekānupassanā, ‘ayaṃ dukkhanirodho, ayaṃ dukkhanirodhagāminī paṭipadā’ ti, ayaṃ dutiyānupassanā. Evaṃ sammā dvayatānupassino kho, bhikkhave, bhikkhuno appamattassa ātāpino pahitattassa viharato dvinnaṃ phalānaṃ aññataraṃ phalaṃ pāṭikaṅkhaṃ: diṭṭhe va dhamme aññā, sati vā upādisese anāgāmitā” ti. Idam avoca Bhagavā. Idaṃ vatvāna Sugato athāparaṃ etad avoca Satthā:

\begin{enumerate}\item 而在「\textbf{那什么是你们说的二重}」中,其意趣为:诸比丘!如果有人问你们:「尊者!那什么是你们说的二重?」而词义为:那什么是你们说的二重之相?
\item 随后,世尊为显示二重,便说了「此是苦」等等。这里,二重四谛法的「\textbf{此是苦,此是苦之集}」,即以见世间的一部分,或以见有因之苦,\textbf{这是一随观},另外以见出世间的第二部分,或以见有方便之灭,\textbf{这是第二随观}。且此中,初(随观)由第三、第四清净而成,第二(随观)由第五清净\footnote{第三、第四清净为见清净、度疑清净,第五清净为道非道智见清净。}。
\item \textbf{对如是正确地随观二重的(比丘)},即对以此所说的方法正确地随观二重法者,以不失念为\textbf{不放逸},以身、心的精进、热忱为\textbf{热忱},由不顾念身、命为\textbf{自励}。\textbf{可期待},即可希望。\textbf{在现法中已知},即于此自体成阿罗汉。\textbf{或当有余依时为阿那含},为再有所当取的残余的蕴被称为「余依」,即显示「或当此时,可期待阿那含性」。这里,虽然如是随观二重者也可得低的果,但为令其于高的果生起勇猛而如是说。
\item 「说了这些」等等是结集者们的话。这里,\textbf{这些},即「诸比丘!若有人问……」等已说之所示。\textbf{进一步说},即如「若他们不知晓苦」等将说的偈颂之所示。且这些偈颂由显明四谛,唯疏通已说之义,如此,则是为了好乐偈颂者,后至者,因不堪能而未把握先前所说、希望「现在若能说(偈颂)则善妙」者,以及心散乱者的义利而说。或者,为疏通特殊之义,即在显示未修观者及修观者后,显示其流转与还灭,所以是为了显示特殊之义而说。此方法也适用于此后的偈颂之语中。\end{enumerate}

\subsection\*{\textbf{730} {\footnotesize 〔PTS 724〕}}

\textbf{「若他们不知晓苦,以及苦的生起,\\}
\textbf{「与苦被完全无余地止息之处,\\}
\textbf{「且不知晓趣向苦的寂止之道,}

“Ye dukkhaṃ nappajānanti, atho dukkhassa sambhavaṃ;\\
yattha ca sabbaso dukkhaṃ, asesaṃ uparujjhati;\\
tañ ca maggaṃ na jānanti, dukkhūpasamagāminaṃ. %\hfill\textcolor{gray}{\footnotesize 1}

\begin{enumerate}\item 这里,\textbf{与苦……之处}是显示涅槃。因为在涅槃中苦完全止息、一切品类止息、连同其因止息、无余地止息。\textbf{道},即八支道。\end{enumerate}

\subsection\*{\textbf{731} {\footnotesize 〔PTS 725〕}}

\textbf{「则他们缺乏心的解脱,以及慧的解脱,\\}
\textbf{「他们不能尽于边际,他们唯经历生老。}

Cetovimuttihīnā te, atho paññāvimuttiyā;\\
abhabbā te antakiriyāya, te ve jātijarūpagā. %\hfill\textcolor{gray}{\footnotesize 2}

\begin{enumerate}\item 在「\textbf{他们缺乏心的解脱,以及慧的解脱}」中,当知阿罗汉果定由离贪为心解脱,阿罗汉果慧由离无明为慧解脱。或者,爱行者以安止禅那之力镇伏烦恼已,得证阿罗汉果而离贪为心解脱,见行者仅转起近行禅那后修观,得证阿罗汉果而离无明为慧解脱\footnote{爱行者、见行者,见\textbf{清净道论}·说取业处品第 78 段。}。或者,阿那含果就对爱欲的贪染而言,以离贪为心解脱,阿罗汉果从一切品类离无明为慧解脱。\textbf{尽于边际},尽于流转之苦的边际。\textbf{经历生老},当知即未解脱生老。此中其余,从头开始,唯皆自明。\end{enumerate}

\subsection\*{\textbf{732} {\footnotesize 〔PTS 726〕}}

\textbf{「若他们既知晓苦,以及苦的生起,\\}
\textbf{「与苦被完全无余地止息之处,\\}
\textbf{「并且知晓趣向苦的寂止之道,}

Ye ca dukkhaṃ pajānanti, atho dukkhassa sambhavaṃ;\\
yattha ca sabbaso dukkhaṃ, asesaṃ uparujjhati;\\
tañ ca maggaṃ pajānanti, dukkhūpasamagāminaṃ. %\hfill\textcolor{gray}{\footnotesize 3}

\subsection\*{\textbf{733} {\footnotesize 〔PTS 727〕}}

\textbf{「则他们具足心的解脱,以及慧的解脱,\\}
\textbf{「他们堪能尽于边际,他们不经历生老。」}

Cetovimuttisampannā, atho paññāvimuttiyā;\\
bhabbā te antakiriyāya, na te jātijarūpagā” ti. %\hfill\textcolor{gray}{\footnotesize 4}

\begin{enumerate}\item 在颂的终了,六十位比丘领受了这开示而修观,即于坐上圆满了阿罗汉。且一切章节处都如此。\end{enumerate}

\textbf{「诸比丘!若有人问『还能以别的方法正确地随观二重吗』,应对他们说『有』。那是如何?『任何苦的生起,一切都由依持为缘』,这是一随观,『即由依持的无余离贪、灭,则无苦的生起』,这是第二随观。如是正确地……」大师进一步说:}

“‘Siyā aññena pi pariyāyena sammā dvayatānupassanā’ ti, iti ce, bhikkhave, pucchitāro assu, ‘siyā’ ti’ssu vacanīyā. Kathañ ca siyā? ‘yaṃ kiñci dukkhaṃ sambhoti sabbaṃ upadhipaccayā’ ti, ayam ekānupassanā, ‘upadhīnaṃ tv eva asesavirāganirodhā natthi dukkhassa sambhavo’ ti, ayaṃ dutiyānupassanā. Evaṃ sammā…pe… athāparaṃ etad avoca Satthā:

\begin{enumerate}\item 此后,世尊以「\textbf{还能以别的方法}」等方法,从多种行相说了二重随观。这里,在第二节中,\textbf{由依持为缘},即由有漏的业缘。因为有漏的业即此处「依持」之意。\textbf{无余离贪、灭},即以无余离贪而灭,或被称为无余离贪的灭。\end{enumerate}

\subsection\*{\textbf{734} {\footnotesize 〔PTS 728〕}}

\textbf{「世间这些种种形相的苦,由依持为因而产生,\\}
\textbf{「若愚钝的无知者造作依持,则再再地经历苦,\\}
\textbf{「所以,知晓者、随观苦的生与源者不应造作依持。」}

“Upadhinidānā pabhavanti dukkhā, ye keci lokasmim anekarūpā;\\
yo ve avidvā upadhiṃ karoti, punappunaṃ dukkham upeti mando;\\
tasmā pajānaṃ upadhiṃ na kayirā, dukkhassa jātippabhavānupassī” ti. %\hfill\textcolor{gray}{\footnotesize 5}

\begin{enumerate}\item \textbf{由依持为因},即由业缘。\textbf{随观苦的生与源},即随观「依持为流转之苦的生起与原因」。其余于此自明。如是,此节也在显明四谛之后,以阿罗汉为顶点而说。且一切章节都如此。\end{enumerate}

\textbf{「诸比丘!若有人问『还能以别的方法正确地随观二重吗』,应对他们说『有』。那是如何?『任何苦的生起,一切都由无明为缘』,这是一随观,『即由无明的无余离贪、灭,则无苦的生起』,这是第二随观。如是正确地……」大师进一步说:}

“‘Siyā aññena pi pariyāyena sammā dvayatānupassanā’ ti, iti ce, bhikkhave, pucchitāro assu, ‘siyā’ ti’ssu vacanīyā. Kathañ ca siyā? ‘yaṃ kiñci dukkhaṃ sambhoti sabbaṃ avijjāpaccayā’ ti, ayam ekānupassanā, ‘avijjāya tv eva asesavirāganirodhā natthi dukkhassa sambhavo’ ti, ayaṃ dutiyānupassanā. Evaṃ sammā…pe… athāparaṃ etad avoca Satthā:

\begin{enumerate}\item 这里,在第三节中,\textbf{由无明为缘},即由作为趣向有的业之资粮的无明为缘。而\textbf{苦}在一切处皆为流转之苦。\end{enumerate}

\subsection\*{\textbf{735} {\footnotesize 〔PTS 729〕}}

\textbf{「若再再地进入生死的轮回,\\}
\textbf{「到此处与他处,这趣向唯由于无明。}

“Jātimaraṇasaṃsāraṃ, ye vajanti punappunaṃ;\\
itthabhāvaññathābhāvaṃ, avijjāy’eva sā gati. %\hfill\textcolor{gray}{\footnotesize 6}

\begin{enumerate}\item \textbf{生死的轮回},蕴的转起为生,蕴的分离为死,蕴的连续为轮回。\textbf{进入},即去往、经历。\textbf{此处与他处},即此人类及其余别的部类。\textbf{趣向},即缘的状态。\end{enumerate}

\subsection\*{\textbf{736} {\footnotesize 〔PTS 730〕}}

\textbf{「因为无明是大痴,这纠缠因之长久,\\}
\textbf{「但若有情具明,他们便不去往再有。」}

Avijjā h’āyaṃ mahāmoho, yen’idaṃ saṃsitaṃ ciraṃ;\\
vijjāgatā ca ye sattā, na te gacchanti punabbhavan” ti. %\hfill\textcolor{gray}{\footnotesize 7}

\begin{enumerate}\item \textbf{但若有情具明},即但若以阿罗汉道之明熄灭了烦恼,而至漏尽的有情。其余之义自明。\end{enumerate}

\textbf{「『还能以别的……那是如何?『任何苦的生起,一切都由行为缘』,这是一随观,『即由诸行的无余离贪、灭,则无苦的生起』,这是第二随观。如是正确地……」大师进一步说:}

“‘Siyā aññena pi…pe… Kathañ ca siyā? ‘yaṃ kiñci dukkhaṃ sambhoti sabbaṃ saṅkhārapaccayā’ ti, ayam ekānupassanā, ‘saṅkhārānaṃ tv eva asesavirāganirodhā natthi dukkhassa sambhavo’ ti, ayaṃ dutiyānupassanā. Evaṃ sammā…pe… athāparaṃ etad avoca Satthā:

\begin{enumerate}\item 在第四节中,\textbf{由行为缘},即由福、非福、不动的行作为缘。\end{enumerate}

\subsection\*{\textbf{737} {\footnotesize 〔PTS 731〕}}

\textbf{「任何苦的生起,一切都由行为缘,\\}
\textbf{「以诸行的灭,则无苦的生起。}

“Yaṃ kiñci dukkhaṃ sambhoti, sabbaṃ saṅkhārapaccayā;\\
saṅkhārānaṃ nirodhena, natthi dukkhassa sambhavo. %\hfill\textcolor{gray}{\footnotesize 8}

\subsection\*{\textbf{738} {\footnotesize 〔PTS 732〕}}

\textbf{「了知了这过患,『苦由行为缘』,\\}
\textbf{「由一切行的止息,由诸想的破灭,\\}
\textbf{「如是则苦灭尽。如实地了知此已,}

Etam ādīnavaṃ ñatvā, ‘dukkhaṃ saṅkhārapaccayā’;\\
sabbasaṅkhārasamathā, saññānaṃ uparodhanā;\\
evaṃ dukkhakkhayo hoti, etaṃ ñatvā yathātathaṃ. %\hfill\textcolor{gray}{\footnotesize 9}

\begin{enumerate}\item \textbf{了知了这过患},即了知了这「\textbf{苦由行为缘}」之过患。\textbf{一切行的止息},即一切所说行相的诸行以道智而止息,即是说破灭了果报的可能。\textbf{诸想},即欲想等,仍以道\textbf{破灭}。\textbf{如实地了知此已},即无颠倒地了知此苦之灭尽已。\end{enumerate}

\subsection\*{\textbf{739} {\footnotesize 〔PTS 733〕}}

\textbf{「正见、通达诸明的智者们正知已,\\}
\textbf{「征服了魔罗的结缚,不去往再有。」}

Sammaddasā vedaguno, sammadaññāya paṇḍitā;\\
abhibhuyya Mārasaṃyogaṃ, na gacchanti punabbhavan” ti. %\hfill\textcolor{gray}{\footnotesize 10}

\begin{enumerate}\item \textbf{正知已},即从无常等了知有为、从常等了知无为。\textbf{魔罗的结缚},即三界的流转。其余之义自明。\end{enumerate}

\textbf{「『还能以别的……那是如何?『任何苦的生起,一切都由识为缘』,这是一随观,『即由识的无余离贪、灭,则无苦的生起』,这是第二随观。如是正确地……」大师进一步说:}

“‘Siyā aññena pi…pe… Kathañ ca siyā? ‘yaṃ kiñci dukkhaṃ sambhoti sabbaṃ viññāṇapaccayā’ ti, ayam ekānupassanā, ‘viññāṇassa tv eva asesavirāganirodhā natthi dukkhassa sambhavo’ ti, ayaṃ dutiyānupassanā. Evaṃ sammā…pe… athāparaṃ etad avoca Satthā:

\begin{enumerate}\item 在第五节中,\textbf{由识为缘},即由与业俱生的行作识为缘。\end{enumerate}

\subsection\*{\textbf{740} {\footnotesize 〔PTS 734〕}}

\textbf{「任何苦的生起,一切都由识为缘,\\}
\textbf{「以识的灭,则无苦的生起。}

“Yaṃ kiñci dukkhaṃ sambhoti, sabbaṃ viññāṇapaccayā;\\
viññāṇassa nirodhena, natthi dukkhassa sambhavo. %\hfill\textcolor{gray}{\footnotesize 11}

\subsection\*{\textbf{741} {\footnotesize 〔PTS 735〕}}

\textbf{「了知了这过患,『苦由识为缘』,\\}
\textbf{「由识的寂止,比丘不饥而般涅槃。」}

Etam ādīnavaṃ ñatvā, ‘dukkhaṃ viññāṇapaccayā’;\\
viññāṇūpasamā bhikkhu, nicchāto parinibbuto” ti. %\hfill\textcolor{gray}{\footnotesize 12}

\begin{enumerate}\item \textbf{不饥},即离渴爱。\textbf{般涅槃},即以烦恼的止息而般涅槃。其余自明。\end{enumerate}

\textbf{「『还能以别的……那是如何?『任何苦的生起,一切都由触为缘』,这是一随观,『即由触的无余离贪、灭,则无苦的生起』,这是第二随观。如是正确地……」大师进一步说:}

“‘Siyā aññena pi…pe… Kathañ ca siyā? ‘yaṃ kiñci dukkhaṃ sambhoti sabbaṃ phassapaccayā’ ti, ayam ekānupassanā, ‘phassassa tv eva asesavirāganirodhā natthi dukkhassa sambhavo’ ti, ayaṃ dutiyānupassanā. Evaṃ sammā…pe… athāparaṃ etad avoca Satthā:

\begin{enumerate}\item 在第六节中,\textbf{由触为缘},即由与行作识相应的触为缘之义。如是,这里未说按次第当说的名色、六入处而说触,因为它们由与色混杂而不与业相应,而这流转之苦则从业或与业相应之法而生。\end{enumerate}

\subsection\*{\textbf{742} {\footnotesize 〔PTS 736〕}}

\textbf{「对那些被触击溃、随流于有流、\\}
\textbf{「于错路行道者,离结缚的灭尽甚远。}

“Tesaṃ phassaparetānaṃ, bhavasotānusārinaṃ;\\
kummaggapaṭipannānaṃ, ārā saṃyojanakkhayo. %\hfill\textcolor{gray}{\footnotesize 13}

\begin{enumerate}\item \textbf{随流于有流},即随流于渴爱。\end{enumerate}

\subsection\*{\textbf{743} {\footnotesize 〔PTS 737〕}}

\textbf{「若遍知了触,了知后而乐于寂止者,\\}
\textbf{「则由触的止息,不饥而般涅槃。」}

Ye ca phassaṃ pariññāya, aññāy’upasame ratā;\\
te ve phassābhisamayā, nicchātā parinibbutā” ti. %\hfill\textcolor{gray}{\footnotesize 14}

\begin{enumerate}\item \textbf{遍知},即以三遍知而遍知。\textbf{了知},即以阿罗汉道慧而知。\textbf{乐于寂止},即以果定而乐于涅槃。\textbf{触的止息},即触的灭。其余自明。\end{enumerate}

\textbf{「『还能以别的……那是如何?『任何苦的生起,一切都由受为缘』,这是一随观,『即由诸受的无余离贪、灭,则无苦的生起』,这是第二随观。如是正确地……」大师进一步说:}

“‘Siyā aññena pi…pe… Kathañ ca siyā? ‘yaṃ kiñci dukkhaṃ sambhoti sabbaṃ vedanāpaccayā’ ti, ayam ekānupassanā, ‘vedanānaṃ tv eva asesavirāganirodhā natthi dukkhassa sambhavo’ ti, ayaṃ dutiyānupassanā. Evaṃ sammā…pe… athāparaṃ etad avoca Satthā:

\begin{enumerate}\item 在第七节中,\textbf{由受为缘},即由与业相应的受为缘。\end{enumerate}

\subsection\*{\textbf{744} {\footnotesize 〔PTS 738〕}}

\textbf{「若乐,若苦,与不苦不乐,\\}
\textbf{「内在及外在,凡所感受者,}

“Sukhaṃ vā yadi vā dukkhaṃ, adukkhamasukhaṃ saha;\\
ajjhattañ ca bahiddhā ca, yaṃ kiñci atthi veditaṃ. %\hfill\textcolor{gray}{\footnotesize 15}

\subsection\*{\textbf{745} {\footnotesize 〔PTS 739〕}}

\textbf{「了知了『这是苦、虚妄败坏之法』,\\}
\textbf{「随触即见灭,他如是于此了知,\\}
\textbf{「比丘由诸受的灭尽,不饥而般涅槃。」}

‘Etaṃ dukkhan’ ti ñatvāna, ‘mosadhammaṃ palokinaṃ’;\\
phussa phussa vayaṃ passaṃ, evaṃ tattha vijānati;\\
vedanānaṃ khayā bhikkhu, nicchāto parinibbuto” ti. %\hfill\textcolor{gray}{\footnotesize 16}

\begin{enumerate}\item \textbf{了知了这是苦},即了知了这一切所感受的是「苦的原因」,或者以变异、保持、无知之苦而了知了苦\footnote{以变异、保持、无知之苦:\textbf{中部}第 44 经云:朋友毗舍佉!乐受的保持为乐,变异为苦,苦受的保持为苦,变异为乐,不苦不乐受的觉知为乐,无知为苦。}。\textbf{虚妄之法},即消亡之法。\textbf{败坏},即因老死而崩坏之法。\textbf{随触},即以生灭智而触。\textbf{见灭},即于边际唯见破坏。\textbf{他如是于此了知},即他如是了知这受,或者,他于此了知苦的状态。\textbf{诸受的灭尽},即此后因道智,与业相应的受灭尽。其余自明。\end{enumerate}

\textbf{「『还能以别的……那是如何?『任何苦的生起,一切都由爱为缘』,这是一随观,『即由爱的无余离贪、灭,则无苦的生起』,这是第二随观。如是正确地……」大师进一步说:}

“‘Siyā aññena pi…pe… Kathañ ca siyā? ‘yaṃ kiñci dukkhaṃ sambhoti sabbaṃ taṇhāpaccayā’ ti, ayam ekānupassanā, ‘taṇhāya tv eva asesavirāganirodhā natthi dukkhassa sambhavo’ ti, ayaṃ dutiyānupassanā. Evaṃ sammā…pe… athāparaṃ etad avoca Satthā:

\begin{enumerate}\item 在第八节中,\textbf{由爱为缘},即由作为业之资粮的爱为缘。\end{enumerate}

\subsection\*{\textbf{746} {\footnotesize 〔PTS 740〕}}

\textbf{「以爱为侣的人,轮回于漫长的旅途,\\}
\textbf{「到此处与他处,不得越过轮回。}

“Taṇhādutiyo puriso, dīgham addhāna saṃsaraṃ;\\
itthabhāvaññathābhāvaṃ, saṃsāraṃ nātivattati. %\hfill\textcolor{gray}{\footnotesize 17}

\subsection\*{\textbf{747} {\footnotesize 〔PTS 741〕}}

\textbf{「了知了这过患,『渴爱是苦的生起』,\\}
\textbf{「离爱、无执取,比丘应具念游行。」}

Etam ādīnavaṃ ñatvā, ‘taṇhaṃ dukkhassa sambhavaṃ’;\\
vītataṇho anādāno, sato bhikkhu paribbaje” ti. %\hfill\textcolor{gray}{\footnotesize 18}

\begin{enumerate}\item \textbf{了知了这过患,渴爱是苦的生起},即了知了这苦的生起是渴爱的过患。其余自明。\end{enumerate}

\textbf{「『还能以别的……那是如何?『任何苦的生起,一切都由取为缘』,这是一随观,『即由取的无余离贪、灭,则无苦的生起』,这是第二随观。如是正确地……」大师进一步说:}

“‘Siyā aññena pi…pe… Kathañ ca siyā? ‘yaṃ kiñci dukkhaṃ sambhoti sabbaṃ upādānapaccayā’ ti, ayam ekānupassanā, ‘upādānānaṃ tv eva asesavirāganirodhā natthi dukkhassa sambhavo’ ti, ayaṃ dutiyānupassanā. Evaṃ sammā…pe… athāparaṃ etad avoca Satthā:

\begin{enumerate}\item 在第九节中,\textbf{由取为缘},即以作为业之资粮的取为缘。\end{enumerate}

\subsection\*{\textbf{748} {\footnotesize 〔PTS 742〕}}

\textbf{「由取为缘而有有,生物遭受痛苦,\\}
\textbf{「对于生者即有死,这是苦的生起。}

“Upādānapaccayā bhavo, bhūto dukkhaṃ nigacchati;\\
jātassa maraṇaṃ hoti, eso dukkhassa sambhavo. %\hfill\textcolor{gray}{\footnotesize 19}

\begin{enumerate}\item \textbf{有},即异熟有、蕴的显现。\textbf{生物遭受痛苦},即已生的生物遭受流转之苦。\textbf{对于生者即有死},即为显示「愚人认为『生物遭受快乐』之处,亦有痛苦」而说「对于生者即有死」。\end{enumerate}

\subsection\*{\textbf{749} {\footnotesize 〔PTS 743〕}}

\textbf{「所以,由取的灭尽,智者们正知已,\\}
\textbf{「证知了生的灭尽,不去往再有。」}

Tasmā upādānakkhayā, sammadaññāya paṇḍitā;\\
jātikkhayaṃ abhiññāya, na gacchanti punabbhavan” ti. %\hfill\textcolor{gray}{\footnotesize 20}

\begin{enumerate}\item 第二颂的连结为:智者们以无常等正知已,由取的灭尽,证知了涅槃,不去往再有。\end{enumerate}

\textbf{「『还能以别的……那是如何?『任何苦的生起,一切都由努力为缘』,这是一随观,『即由诸努力的无余离贪、灭,则无苦的生起』,这是第二随观。如是正确地……」大师进一步说:}

“‘Siyā aññena pi…pe… Kathañ ca siyā? ‘yaṃ kiñci dukkhaṃ sambhoti sabbaṃ ārambhapaccayā’ ti, ayam ekānupassanā, ‘ārambhānaṃ tv eva asesavirāganirodhā natthi dukkhassa sambhavo’ ti, ayaṃ dutiyānupassanā. Evaṃ sammā…pe… athāparaṃ etad avoca Satthā:

\begin{enumerate}\item 在第十节中,\textbf{由努力为缘},即由与业相应的精进为缘。\end{enumerate}

\subsection\*{\textbf{750} {\footnotesize 〔PTS 744〕}}

\textbf{「任何苦的生起,一切都由努力为缘,\\}
\textbf{「以诸努力的灭,则无苦的生起。}

“Yaṃ kiñci dukkhaṃ sambhoti, sabbaṃ ārambhapaccayā;\\
ārambhānaṃ nirodhena, natthi dukkhassa sambhavo. %\hfill\textcolor{gray}{\footnotesize 21}

\subsection\*{\textbf{751} {\footnotesize 〔PTS 745〕}}

\textbf{「了知了这过患,『苦由努力为缘』,\\}
\textbf{「舍遣了一切努力,对无努力中的解脱者、}

Etam ādīnavaṃ ñatvā, ‘dukkhaṃ ārambhapaccayā’;\\
sabbārambhaṃ paṭinissajja, anārambhe vimuttino. %\hfill\textcolor{gray}{\footnotesize 22}

\begin{enumerate}\item \textbf{无努力中的解脱者},即无努力之涅槃中的解脱者。其余自明。\end{enumerate}

\subsection\*{\textbf{752} {\footnotesize 〔PTS 746〕}}

\textbf{「对切断了有爱、心已寂静的比丘,\\}
\textbf{「灭尽了生的轮回,他已没有再有。」}

Ucchinnabhavataṇhassa, santacittassa bhikkhuno;\\
vikkhīṇo jātisaṃsāro, natthi tassa punabbhavo” ti. %\hfill\textcolor{gray}{\footnotesize 23}

\textbf{「『还能以别的……那是如何?『任何苦的生起,一切都由食为缘』,这是一随观,『即由诸食的无余离贪、灭,则无苦的生起』,这是第二随观。如是正确地……」大师进一步说:}

“‘Siyā aññena pi…pe… Kathañ ca siyā? ‘yaṃ kiñci dukkhaṃ sambhoti sabbaṃ āhārapaccayā’ ti, ayam ekānupassanā, ‘āhārānaṃ tv eva asesavirāganirodhā natthi dukkhassa sambhavo’ ti, ayaṃ dutiyānupassanā. Evaṃ sammā…pe… athāparaṃ etad avoca Satthā:

\begin{enumerate}\item 在第十一节中,\textbf{由食为缘},即由与业相应的食为缘。另一方法是,有四类有情:依于色、依于受、依于想、依于行。这里,十一种欲界中的有情由受用段食而\textbf{依于色},除无想外的色界中的有情由受用触食而\textbf{依于受},下三种无色界中的有情由受用想所转起的意思食而\textbf{依于想},有顶中的有情由受用行所转起的识食而\textbf{依于行}。如是,当知「任何苦的生起,一切都由食为缘」。\end{enumerate}

\subsection\*{\textbf{753} {\footnotesize 〔PTS 747〕}}

\textbf{「任何苦的生起,一切都由食为缘,\\}
\textbf{「以诸食的灭,则无苦的生起。}

“Yaṃ kiñci dukkhaṃ sambhoti, sabbaṃ āhārapaccayā;\\
āhārānaṃ nirodhena, natthi dukkhassa sambhavo. %\hfill\textcolor{gray}{\footnotesize 24}

\subsection\*{\textbf{754} {\footnotesize 〔PTS 748〕}}

\textbf{「了知了这过患,『苦由食为缘』,\\}
\textbf{「遍知了一切食,不依止一切食。}

Etam ādīnavaṃ ñatvā, ‘dukkhaṃ āhārapaccayā’;\\
sabbāhāraṃ pariññāya, sabbāhāram anissito. %\hfill\textcolor{gray}{\footnotesize 25}

\subsection\*{\textbf{755} {\footnotesize 〔PTS 749〕}}

\textbf{「正知了无病,由诸漏的遍尽,\\}
\textbf{「省思而受用,通达诸明的住法者不入诸数。」}

Ārogyaṃ sammadaññāya, āsavānaṃ parikkhayā;\\
saṅkhāya sevī dhammaṭṭho, saṅkhyaṃ nopeti vedagū” ti. %\hfill\textcolor{gray}{\footnotesize 26}

\begin{enumerate}\item \textbf{无病},即涅槃。\textbf{省思而受用},省察四资具而受用,或省思世间为「五蕴、十二处、十八界」,以「无常、苦、无我」之智受用。\textbf{住法者},即住于四谛法者。\textbf{不入诸数},即不入「人、天」等之数。其余自明。\end{enumerate}

\textbf{「『还能以别的……那是如何?『任何苦的生起,一切都由动摇为缘』,这是一随观,『即由诸动摇的无余离贪、灭,则无苦的生起』,这是第二随观。如是正确地……」大师进一步说:}

“‘Siyā aññena pi…pe… Kathañ ca siyā? ‘yaṃ kiñci dukkhaṃ sambhoti sabbaṃ iñjitapaccayā’ ti, ayam ekānupassanā, ‘iñjitānaṃ tv eva asesavirāganirodhā natthi dukkhassa sambhavo’ ti, ayaṃ dutiyānupassanā. Evaṃ sammā…pe… athāparaṃ etad avoca Satthā:

\begin{enumerate}\item 在第十二节中,\textbf{由动摇为缘},即于爱、慢、见、业、烦恼的动摇中,由任一作为业之资粮的动摇为缘。\end{enumerate}

\subsection\*{\textbf{756} {\footnotesize 〔PTS 750〕}}

\textbf{「任何苦的生起,一切都由动摇为缘,\\}
\textbf{「以诸动摇的灭,则无苦的生起。}

“Yaṃ kiñci dukkhaṃ sambhoti, sabbaṃ iñjitapaccayā;\\
iñjitānaṃ nirodhena, natthi dukkhassa sambhavo. %\hfill\textcolor{gray}{\footnotesize 27}

\subsection\*{\textbf{757} {\footnotesize 〔PTS 751〕}}

\textbf{「了知了这过患,『苦由动摇为缘』,\\}
\textbf{「所以,舍遣了动摇,破坏了诸行,\\}
\textbf{「不动、无取,比丘应具念游行。」}

Etam ādīnavaṃ ñatvā, ‘dukkhaṃ iñjitapaccayā’;\\
tasmā hi ejaṃ vossajja, saṅkhāre uparundhiya;\\
anejo anupādāno, sato bhikkhu paribbaje” ti. %\hfill\textcolor{gray}{\footnotesize 28}

\begin{enumerate}\item \textbf{舍遣了动摇},即舍弃了渴爱。\textbf{破坏了诸行},即寂灭了业及与业相应的诸行。其余自明。\end{enumerate}

\textbf{「『还能以别的……那是如何?『依止者有震动』,这是一随观,『无依止则不震动』,这是第二随观。如是正确地……」大师进一步说:}

“‘Siyā aññena pi…pe… Kathañ ca siyā? ‘nissitassa calitaṃ hotī’ ti, ayam ekānupassanā, ‘anissito na calatī’ ti, ayaṃ dutiyānupassanā. Evaṃ sammā…pe… athāparaṃ etad avoca Satthā:

\begin{enumerate}\item 在第十三节中,\textbf{依止者有震动},即以爱,或以爱、见、慢依止于诸蕴者,如(相应部)狮子经中的天人一般,有怖畏的震动。其余自明。\end{enumerate}

\subsection\*{\textbf{758} {\footnotesize 〔PTS 752〕}}

\textbf{「无依止则不震动,依止则取著,\\}
\textbf{「到此处与他处,不得越过轮回。}

“Anissito na calati, nissito ca upādiyaṃ;\\
itthabhāvaññathābhāvaṃ, saṃsāraṃ nātivattati. %\hfill\textcolor{gray}{\footnotesize 29}

\subsection\*{\textbf{759} {\footnotesize 〔PTS 753〕}}

\textbf{「了知了这过患,『依止中有大恐怖』,\\}
\textbf{「无依止、无取,比丘应具念游行。」}

Etam ādīnavaṃ ñatvā, ‘nissayesu mahabbhayaṃ’;\\
anissito anupādāno, sato bhikkhu paribbaje” ti. %\hfill\textcolor{gray}{\footnotesize 30}

\textbf{「『还能以别的……那是如何?『诸比丘!无色较之色更加寂静』,这是一随观,『灭较之无色更加寂静』,这是第二随观。如是正确地……」大师进一步说:}

“‘Siyā aññena pi…pe… Kathañ ca siyā? ‘rūpehi, bhikkhave, arūpā santatarā’ ti, ayam ekānupassanā, ‘arūpehi nirodho santataro’ ti, ayaṃ dutiyānupassanā. Evaṃ sammā…pe… athāparaṃ etad avoca Satthā:

\begin{enumerate}\item 在第十四节中,\textbf{较之色},即较之色有,或较之色之等至。\textbf{无色},即无色有,或无色之等至。\textbf{灭},即涅槃。\end{enumerate}

\subsection\*{\textbf{760} {\footnotesize 〔PTS 754〕}}

\textbf{「经历于色的有情,与无色处者,\\}
\textbf{「未能了知灭,往来于再有。}

“Ye ca rūpūpagā sattā, ye ca arūpaṭṭhāyino;\\
nirodhaṃ appajānantā, āgantāro punabbhavaṃ. %\hfill\textcolor{gray}{\footnotesize 31}

\subsection\*{\textbf{761} {\footnotesize 〔PTS 755〕}}

\textbf{「若遍知了色,不安立于无色,\\}
\textbf{「在灭中解脱,他们是抛弃死亡者。」}

Ye ca rūpe pariññāya, arūpesu asaṇṭhitā;\\
nirodhe ye vimuccanti, te janā maccuhāyino” ti. %\hfill\textcolor{gray}{\footnotesize 32}

\begin{enumerate}\item \textbf{抛弃死亡者},即是说舍弃了作为死的死亡、作为烦恼的死亡、作为天子的死亡等这三种死亡而行者。其余自明。\end{enumerate}

\textbf{「『还能以别的……那是如何?『诸比丘!对于俱有天、魔、梵、沙门婆罗门、天人的人世间被认为「此是真实」者,对于圣者,以如实的正慧善见「此是虚妄」』,这是一随观,『诸比丘!对于俱有天……的人世间被认为「此是虚妄」者,对于圣者,以如实的正慧善见「此是真实」』,这是第二随观。如是正确地……」大师进一步说:}

“‘Siyā aññena pi…pe… Kathañ ca siyā? ‘yaṃ, bhikkhave, sadevakassa lokassa samārakassa sabrahmakassa sassamaṇabrāhmaṇiyā pajāya sadevamanussāya “idaṃ saccan” ti upanijjhāyitaṃ tadam ariyānaṃ “etaṃ musā” ti yathābhūtaṃ sammappaññāya sudiṭṭhaṃ’, ayam ekānupassanā, ‘yaṃ, bhikkhave, sadevakassa…pe… sadevamanussāya “idaṃ musā” ti upanijjhāyitaṃ, tadam ariyānaṃ “etaṃ saccan” ti yathābhūtaṃ sammappaññāya sudiṭṭhaṃ’, ayaṃ dutiyānupassanā. Evaṃ sammā…pe… athāparaṃ etad avoca Satthā:

\begin{enumerate}\item 在第十五节中,\textbf{对于……者}是就名色而言。因为它以常、净、乐、我等被世间认为、见为、视为「此是真实」。\textbf{Tadam},即 idaṃ,是省略了鼻音和字母 i 而说。\textbf{此是虚妄},即便它被常等所执取,仍是虚妄,即并非如此。下一个「\textbf{对于……者}」是就涅槃而言。因为它由无有色、受等,被世间认为「此是虚妄,一无所有」。\textbf{对于圣者,此是真实},即对于圣者,它以不离于被称为无烦恼的净相、被称为转起与苦相对的乐相、被称为极度寂静的常相,从第一义如实的正慧善见为「真实」。\end{enumerate}

\subsection\*{\textbf{762} {\footnotesize 〔PTS 756〕}}

\textbf{「看俱有天的世间,在无我中思量我,\\}
\textbf{「在名色中执著,认为『此是真实』。}

“Anattani attamāniṃ, passa lokaṃ sadevakaṃ;\\
niviṭṭhaṃ nāmarūpasmiṃ, ‘idaṃ saccan’ ti maññati. %\hfill\textcolor{gray}{\footnotesize 33}

\begin{enumerate}\item \textbf{在无我中思量我},即在无我的名色中思量我。\textbf{认为「此是真实」},即他以常等认为此名色是「真实」。\end{enumerate}

\subsection\*{\textbf{763} {\footnotesize 〔PTS 757〕}}

\textbf{「无论他们如何认为,之后它总成别样,\\}
\textbf{「因为这对他是虚妄,短暂即虚妄之法。}

Yena yena hi maññanti, tato taṃ hoti aññathā;\\
tañ hi tassa musā hoti, mosadhammañ hi ittaraṃ. %\hfill\textcolor{gray}{\footnotesize 34}

\begin{enumerate}\item \textbf{无论他们如何认为},即无论他们如何于色、受以「我的色、我的受」等方法来认为。\textbf{之后它总成别样},即之后这名色总成为与所思的行相不同的别样。什么原因?\textbf{因为这对他是虚妄},意即因为如其所思的行相是虚妄,所以成为别样。但为什么是虚妄?\textbf{短暂即虚妄之法},因为凡短暂的有限现起,彼即虚妄之法、毁灭之法,而名色即如是。\end{enumerate}

\subsection\*{\textbf{764} {\footnotesize 〔PTS 758〕}}

\textbf{「涅槃是非虚妄法,圣者们真实地知晓之,\\}
\textbf{「他们实由现观真实,不饥而般涅槃。」}

Amosadhammaṃ nibbānaṃ, tad ariyā saccato vidū;\\
te ve saccābhisamayā, nicchātā parinibbutā” ti. %\hfill\textcolor{gray}{\footnotesize 35}

\begin{enumerate}\item \textbf{现观真实},即理解真实。其余自明。\end{enumerate}

\textbf{「诸比丘!若有人问『还能以别的方法正确地随观二重吗』,应对他们说『有』。那是如何?『诸比丘!对于俱有天、魔、梵、沙门婆罗门、天人的人世间被认为「此是乐」者,对于圣者,以如实的正慧善见「此是苦」』,这是一随观,『诸比丘!对于俱有天……的人世间被认为「此是苦」者,对于圣者,以如实的正慧善见「此是乐」』,这是第二随观。诸比丘!对如是正确地随观二重的比丘,住于不放逸、热忱、自励,可期待二种果中的一果:在现法中已知,或当有余依时为阿那含。」世尊说了这些。善逝说罢,大师进一步说:}

“‘Siyā aññena pi pariyāyena sammā dvayatānupassanā’ ti, iti ce, bhikkhave, pucchitāro assu, ‘siyā’ ti’ssu vacanīyā. Kathañ ca siyā? ‘yaṃ, bhikkhave, sadevakassa lokassa samārakassa sabrahmakassa sassamaṇabrāhmaṇiyā pajāya sadevamanussāya “idaṃ sukhan” ti upanijjhāyitaṃ, tadam ariyānaṃ “etaṃ dukkhan” ti yathābhūtaṃ sammappaññāya sudiṭṭhaṃ’, ayam ekānupassanā, ‘yaṃ, bhikkhave, sadevakassa…pe… sadevamanussāya “idaṃ dukkhan” ti upanijjhāyitaṃ tadam ariyānaṃ “etaṃ sukhan” ti yathābhūtaṃ sammappaññāya sudiṭṭhaṃ’, ayaṃ dutiyānupassanā. Evaṃ sammā dvayatānupassino kho, bhikkhave, bhikkhuno appamattassa ātāpino pahitattassa viharato dvinnaṃ phalānaṃ aññataraṃ phalaṃ pāṭikaṅkhaṃ: diṭṭhe va dhamme aññā, sati vā upādisese anāgāmitā” ti. Idam avoca Bhagavā. Idaṃ vatvāna Sugato athāparaṃ etad avoca Satthā:

\begin{enumerate}\item 在第十六节中,\textbf{对于……者}是就六种可意所缘而言。因为它被世间认为「此是乐」,如蛾之于火、鱼之于钩、猿之于胶一般。\textbf{对于圣者,此是苦},即这对于圣者,如\begin{quoting}爱欲实在多彩、甜蜜而悦意,以各色形相搅乱着心。(经集·犀牛角经第 50 颂)\end{quoting}等方法\textbf{以如实的正慧善见}「此是苦」。下一个「\textbf{对于……者}」仍是就涅槃而言。因为它由无有种种爱欲被世间认为是苦。\textbf{对于圣者},即这对于圣者,由第一义之乐,\textbf{以如实的正慧善见「此是乐」}。\end{enumerate}

\subsection\*{\textbf{765} {\footnotesize 〔PTS 759〕}}

\textbf{「全部的色、声、味、香、触与法,\\}
\textbf{「可意、可爱且适意,但凡被称为存在,}

“Rūpā saddā rasā gandhā, phassā dhammā ca kevalā;\\
iṭṭhā kantā manāpā ca, yāvat’atthī ti vuccati. %\hfill\textcolor{gray}{\footnotesize 36}

\begin{enumerate}\item \textbf{全部},即无余。\textbf{可意},即被希望、被愿求。\textbf{可爱},即喜爱。\textbf{适意},即令意增长。\textbf{但凡被称为存在},即但凡这六所缘被称为存在。当知是格的转换。\end{enumerate}

\subsection\*{\textbf{766} {\footnotesize 〔PTS 760〕}}

\textbf{「对于俱有天的世间,这些被共许为乐,\\}
\textbf{「而在这些逝去之处,他们共许这是苦。}

Sadevakassa lokassa, ete vo sukhasammatā;\\
yattha c’ete nirujjhanti, taṃ nesaṃ dukkhasammataṃ. %\hfill\textcolor{gray}{\footnotesize 37}

\begin{enumerate}\item 这里的 \textbf{vo} 只是不变词。\end{enumerate}

\subsection\*{\textbf{767} {\footnotesize 〔PTS 761〕}}

\textbf{「有身的破灭被圣者们视为乐,\\}
\textbf{「对具见者们,这与一切世间相违。}

‘Sukhan’ ti diṭṭham ariyehi, sakkāyass’uparodhanaṃ;\\
paccanīkam idaṃ hoti, sabbalokena passataṃ. %\hfill\textcolor{gray}{\footnotesize 38}

\begin{enumerate}\item \textbf{有身的破灭被圣者们视为乐},即五蕴的灭被圣者们视为乐,即是说涅槃。\textbf{这相违},即此知见为逆反。\textbf{具见者们},即是说智者们。\end{enumerate}

\subsection\*{\textbf{768} {\footnotesize 〔PTS 762\textit{a-d}〕}}

\textbf{「凡他人说是乐的,圣者们说是苦,\\}
\textbf{「凡他人说是苦的,圣者们知是乐。}

Yaṃ pare sukhato āhu, tad ariyā āhu dukkhato;\\
yaṃ pare dukkhato āhu, tad ariyā sukhato vidū. %\hfill\textcolor{gray}{\footnotesize 39}

\begin{enumerate}\item \textbf{凡他人}中的「凡」是就物欲而言。下一个「\textbf{凡他人}」中的是就涅槃而言。\end{enumerate}

\subsection\*{\textbf{769} {\footnotesize 〔PTS 762\textit{ef}-763\textit{ab}〕}}

\textbf{「看!法即难以了知,无知者便于此迷惑,\\}
\textbf{「对被覆蔽者是冥暗,对不具见者是黑暗。\footnote{PTS 本将此颂的前两句归于其 762 颂,后两句归于其 763 颂。}}

Passa dhammaṃ durājānaṃ, sampamūḷh’etth’aviddasu;\\
nivutānaṃ tamo hoti, andhakāro apassataṃ. %\hfill\textcolor{gray}{\footnotesize 40}

\begin{enumerate}\item \textbf{看},即称呼听众。\textbf{法},即涅槃法。\textbf{无知者便于此迷惑},即无知的愚人便于此迷惑。什么原因迷惑?\textbf{对被覆蔽者是冥暗,对不具见者是黑暗},对被无明覆蔽、淹没的愚人,便是制造黑暗的冥暗,因之不能得见涅槃法。\end{enumerate}

\subsection\*{\textbf{770} {\footnotesize 〔PTS 763\textit{c-f}〕}}

\textbf{「但对善人们却是敞开,如同光对具见者,\\}
\textbf{「未熟习法的愚人们,在跟前也不能了知。}

Satañ ca vivaṭaṃ hoti, āloko passatām iva;\\
santike na vijānanti, magā dhammass’akovidā. %\hfill\textcolor{gray}{\footnotesize 41}

\begin{enumerate}\item \textbf{但对善人们却是敞开,如同光对具见者},即对善人们,涅槃如同光对具见者一般敞开。\textbf{未熟习法的愚人们,在跟前也不能了知},即这由在自身身体中确定了皮等五法便无间可证,或由自身诸蕴的灭而在跟前的涅槃,它即便如是存在于跟前,愚人,或不熟习道非道法之真实法的人,却不能了知。\end{enumerate}

\subsection\*{\textbf{771} {\footnotesize 〔PTS 764〕}}

\textbf{「被有贪击溃者,随流于有流者,\\}
\textbf{「陷落于魔境者,不易等觉这法。}

Bhavarāgaparetehi, bhavasotānusāribhi;\\
māradheyyānupannehi, nāyaṃ dhammo susambudho. %\hfill\textcolor{gray}{\footnotesize 42}

\begin{enumerate}\item 在一切处,「被有贪……不易等觉这法」。这里,\textbf{陷落于魔境者},即陷落于三界流转者。\end{enumerate}

\subsection\*{\textbf{772} {\footnotesize 〔PTS 765〕}}

\textbf{「除了圣者,还有谁应能等觉这境界?\\}
\textbf{「正知了这境界,无漏者们便般涅槃。」}

Ko nu aññatra-m-ariyehi, padaṃ sambuddhum arahati;\\
yaṃ padaṃ sammadaññāya, parinibbanti anāsavā” ti. %\hfill\textcolor{gray}{\footnotesize 43}

\begin{enumerate}\item 末颂的连结为:对如是不易等觉者,除了圣者,还有谁?其义为:除了圣者,还有其他谁应能知晓涅槃的境界?这境界以第四圣道正知已,便无间而成无漏,以止息烦恼而般涅槃,或者,正知而成无漏已,在最后以无余依涅槃界而般涅槃,即以阿罗汉为顶点完成了开示。\end{enumerate}

\textbf{世尊说了这些。诸比丘心满意足,欢喜于世尊之所说。而当说此解答时,六十比丘的心以无取而从诸漏解脱。}

Idam avoca Bhagavā. Attamanā te bhikkhū Bhagavato bhāsitaṃ abhinandun ti. Imasmiñ ca pana veyyākaraṇasmiṃ bhaññamāne saṭṭhimattānaṃ bhikkhūnaṃ anupādāya āsavehi cittāni vimucciṃsū ti.

\begin{enumerate}\item \textbf{心满意足},即满意。\textbf{此解答},即此第十六解答。其余自明。如是,在所有十六解答中各有六十,计九百六十比丘的\textbf{心以无取而从诸漏解脱}。此中依可调伏,十六次各有四,从种种品类开示了六十四谛。\end{enumerate}

\textbf{其总颂曰:}

\begin{quoting}谛、依持、无明、诸行以及识为第五,\\
触、受、爱、取、努力、食,\\
动摇、震动、色、真实,以苦为十六。\end{quoting}

Tass’uddānaṃ —

\begin{quoting}Saccaṃ upadhi avijjā ca, saṅkhāre viññāṇapañcamaṃ;\\
Phassavedaniyā taṇhā, upādānārambha-āhārā;\\
Iñjitaṃ calitaṃ rūpaṃ, saccaṃ dukkhena soḷasā ti.\end{quoting}

\begin{center}\vspace{1em}二重随观经第十二\\Dvayatānupassanāsuttaṃ dvādasamaṃ.\end{center}

\textbf{其总颂曰:}

\begin{quoting}出家与至上,善说与孙陀利,\\
摩伽与会堂,施罗与箭,\\
婆悉吒以及瞿迦梨、那罗迦、二重随观,\\
这十二经,被称为「大品」。\end{quoting}

Tass’uddānaṃ —

\begin{quoting}Pabbajjā ca Padhānañ ca, Subhāsitañ ca Sundari;\\
Māghasuttaṃ Sabhiyo ca, Selo Sallañ ca vuccati;\\
Vāseṭṭho cāpi Kokāli, Nālako Dvayatānupassanā;\\
dvādas’etāni suttāni, Mahāvaggo ti vuccatī ti.\end{quoting}

\begin{center}\vspace{1em}大品第三\\Mahāvaggo tatiyo.\end{center}

%\begin{flushright}乙巳八月初八二稿\end{flushright}