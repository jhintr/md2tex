\chapter{小品第二}

\section{宝经}

\begin{center}Ratana Sutta\end{center}\vspace{1em}

\begin{enumerate}\item 缘起为何?据说,过去在毗舍离发生了饥馑等的灾祸。众离车为平息彼等,便来到王舍城请求,将世尊带至毗舍离。如是到来的世尊为平息这些灾祸,便说了此经。这于此是略说。
\item 而古人们则从「毗舍离之事」开始解释其缘起。当知如是:据说,波罗奈国王的正妃怀了胎。她知晓后,便告知了国王。国王便给予照料。她受到妥善的照料,当临盆之时,便进入产房。具福的女人会在黎明时分分娩,而她即是其中之一,因此在黎明时分,便分娩出与红色花瓣的午时花一样的肉片。随后,她想到「其他王妃们都产下金色身形的孩子,正妃却产下肉片,在国王面前会对我不美」,因畏惧这不美,便将那肉片弃于一罐,再用它物包裹,封以王室印记,教人弃诸恒河水流之中。刚被人们丢弃,众天人便接手守护,并用天然的朱砂在金布条上写下「波罗奈王正妃之子嗣」后绑扎好。随后,这罐便不为波浪等所扰,随恒河之水而流。
\item 尔时,某个苦行者依止牧牛人家住在恒河岸边。他早上进入恒河,看见这罐漂来,以为是垃圾,便捡起来。随后,看见这字条和封好的王室印记,打开后,便看见这肉片。他见后便想「也许是胎儿,因为并无恶臭腐败之相」,便携至草庵内,置于洁净之处。然后,经过半月,它成了两片肉片,苦行者见后,便置于更善之处。随后又经半月,于一一肉片起了趋于手、足、头等的五个疹块。于是又经半月,一肉片成了金色身形的男孩,一则成了女孩。苦行者便对他们起了爱子之情,且从其拇指泌出乳汁,从此便有乳水作餐。他用完餐后,将乳水灌入两个孩子的口中,凡是进入他俩腹中之物,都如至于摩尼瓶般可见。他俩便如是而为无表皮者,而另有人说「他俩的表皮彼此粘连,如缝合而成」,如是,由无表皮 \textit{nicchavi},或由表皮的粘连 \textit{līnacchavi},他们便得称为\textbf{离车} \textit{Licchavī}。
\item 苦行者为养育孩子,在午后入村乞食,日暮方返。众牧牛人知晓其所为后,便说:「尊者!养育孩子是出家人的障碍,把孩子给我们,我们来养,你们行自己的业!」苦行者便答以「善哉」。第二天,众牧牛人便平整道路,铺以鲜花,竖起旗幡,奏乐前往草庵。苦行者说「两个孩子有大福德,应不放逸地抚养!且抚养后令彼此婚嫁,以五种奶制品取悦国王,取得封地后建城,就在那里为童子灌顶」,便给了两个孩子。他们答以「善哉」,即带回两个孩子养育。
\item 两个孩子随后长大,在嬉戏时与其他牧牛人的孩子发生争辩,动手动脚打架,他们就哭了。当被父母问及「你们为什么哭」,就说:「这两个苦行者养育的孤儿狠狠地打我们。」随后,他们的父母便说:「这两个孩子让其他孩子受恼、受苦,他俩不应被摄受,他俩应予驱离 \textit{vajjetabbā}。」从此,据说这地方,连同方圆一百由旬,被称为\textbf{跋耆} \textit{Vajjī}。
\item 于是,众牧牛人取悦国王,获得了此地,便在那里建城,给到了十六岁的童子灌顶后,便封作国王,并为他与这女孩办了婚礼,立下规约:「不得从外带回女孩,女孩不得从此外嫁。」他们初次同房,便生下一儿一女两个孩子,如是十六次各生下两个孩子。此后,这些孩子陆续长大,不能获得足够的园囿居住之处与资具,便三次以一牛呼的间隔围以城垣。因其再再予以拓宽 \textit{visālīkata},便得名\textbf{毗舍离} \textit{Vesālī}。此即\textbf{毗舍离之事}。
\item 而在世尊出世时,这毗舍离已臻富有广大,因为那里仅王室就有七千七百又七位王,侯王、将军、司库等等也如是,如说:\begin{quoting}尔时,毗舍离富有昌盛,人口繁庶,食物易得,且有七千七百又七座楼阁、七千七百又七座尖顶、七千七百又七个园林、七千七百又七个池塘。(大品第 326 段)\end{quoting}后来某时,它竟饥馑、干旱、歉收。先是贫穷的人们死亡,他们便弃于城外。因死人的腐臭,非人便进到城里。随后,更多的人们死亡,且由此厌逆,有情间便爆发了瘟疫。如是,为饥馑、非人、瘟疫三种怖畏所扰的毗舍离城民便前往国王处说:「大王!此城发生三种怖畏,此前七代王族都未曾如此,我说,如今之事是因你们的非法所为。」国王便将所有人召集到议事厅说:「请调查我们的非法之相!」他们便调查了所有谱系,竟无所见。
\item 随后,他们未见到国王的过失,便想:「我们要如何平息这怖畏呢?」于此,有些人便指认六师:「随着他们进入,这就将平息。」有些人则说:「据说佛陀出世,彼世尊为了一切有情的利益开示法,有大神变、大威力,随着他进入,一切怖畏将会平息。」他们因此心满意足,便说:「那如今彼世尊住在何处?若我们遣使,他会来吗?」于是其他人便说:「诸佛皆具怜悯,为何会不来?彼世尊如今住在王舍城,但频婆娑罗王护持他,有时未必会让来。」「如此,我们就劝谏国王,予以带回。」二离车王便用大军带了许多礼物,遣使到国王跟前:「你们劝谏频婆娑罗,带回世尊!」
\item 他们去后,给了国王礼物,告知经过后便说:「大王!请派遣世尊到我们城里!」国王不领受,说:「你们自己告知!」他们去到世尊处,顶礼后说:「尊者!在我们城内发生三种怖畏,若世尊能来,我们就会平安!」世尊经转向「当在毗舍离说了宝经,这守护将遍满百千俱胝的轮围,在经的终了,八万四千生类将得法的现观」,便允诺了。于是,频婆娑罗王得知世尊允诺,便在城内布告「世尊允诺毗舍离之行」,去到世尊处说:「尊者!是否领受毗舍离之行?」「唯,大王!」「如此,尊者!请稍待!我将平治道路!」
\item 于是,频婆娑罗王平整了王舍城与恒河间五由旬的地面,每隔一由旬建了寺庙,便告知世尊出行时到。世尊为五百比丘随从,便即出发。国王教人在五由旬的道路上铺以没膝的五色鲜花,竖起旗幡\footnote{原文尚有「盛满的水罐 \textit{puṇṇaghaṭa}」,但考虑水罐用「竖起」不妥,故从 PTS 本。}蕉叶等,教人为世尊撑起两顶白伞,且为每位比丘各撑一顶,与自己的随从一起作花、香等的供养,在每个寺庙让世尊住下后作大布施,以五天送到恒河岸边。于此,以一切庄严庄严了船只,遣使致命毗舍离人:「世尊前来,平治道路后,全体出迎!」他们答「我们将作双倍的供养」,平整了毗舍离与恒河间三由旬的地面,为世尊送去四顶白伞,且为每位比丘各送两顶,作着供养,前往恒河岸边等候。
\item 频婆娑罗连接好两条船,建了亭子,庄严以花环,在此设好以一切宝物所造的佛座。世尊便坐于此。五百比丘也上了船,如仪坐下。国王跟随世尊,涉水直至及颈,说了「尊者!直到世尊回来,我就住在此恒河岸边」后方返。上方直至阿迦腻吒居处的天人作了供养,下方住在恒河的钦婆罗、阿输多罗等龙\footnote{钦婆罗 \textit{Kambala}、阿输多罗 \textit{Assatara} 等龙,见\textbf{长部}·大集会经。}也作了供养。以如是的大供养,世尊度过一由旬的恒河之旅,进入毗舍离人的界内。
\item 随后,二离车王便作频婆娑罗所作的双倍供养,在及颈深的水中迎接世尊。就在这一刹那、这一瞬间,环绕电光、广布黑暗且尖耸的大云隆隆作响,在四方集起。于是,当世尊在恒河岸边刚踏下第一步,便天雨莲花。欲受润泽者皆被润泽,不欲受润泽者则不被润泽。在一切处,涌来及膝、及股、及腰、及颈深的水,所有腐臭都被水冲入恒河,一方地面得以遍净。
\item 二离车王让世尊每隔一由旬住下,作大布施,以三天作双倍的供养送到毗舍离。当世尊到达毗舍离,诸天之因陀帝释作为天众的先导便即前来,因大威德诸天的会集,非人便大多逃逸。世尊站在城门,告阿难长老:「阿难!受持此宝经,取得供奉的器具后,与众离车童子一起在三道城垣间巡行,以作护卫!」便说了宝经。如是,古人们从毗舍离之事开始,对「此经由谁、何时、何处及为何而说」等这些问题的解答予以详释。
\item 如是,世尊在到达毗舍离的当天、于毗舍离城门、为了防御这些灾祸而说的这宝经,尊者阿难受持后,为作护卫而唱诵,便以世尊的钵取了水,洒布全城而行。且当长老刚说到第三颂,先前未逃逸、依于尘堆墙壁等处的非人即从四门逃逸,四门便堵塞了。随后,有些无法通过城门,则冲破城垣逃逸。非人甫一离去,人们肢体上的疾病即平息,他们便出来以一切香、花等供养长老。大众在城中的议事厅涂以众香,搭建华盖,以一切庄严庄严已,便在那里设好佛座,引入世尊。
\item 世尊进入议事厅,便坐于所设之座。比丘僧团及众国王、人民也在适当之处落坐。诸天之因陀帝释和其他天人也与天众一起,在两处天界近坐。阿难长老则巡行全毗舍离,作罢护卫,与毗舍离城民一起回来后,也坐在一边。于此,世尊便为全体仍说了这宝经。\end{enumerate}

\subsection\*{\textbf{224} {\footnotesize 〔PTS 222〕}}

\textbf{凡聚集在此的生物,或为地居,或在天上,\\}
\textbf{愿这一切生物欢喜!然后恭敬地聆听所说!}

Yānīdha bhūtāni samāgatāni, bhummāni vā yāni va antalikkhe;\\
sabbe va bhūtā sumanā bhavantu, atho pi sakkacca suṇantu bhāsitaṃ. %\hfill\textcolor{gray}{\footnotesize 1}

\begin{enumerate}\item 这里,初颂中的\textbf{凡},即如此等的或少威德、或大威德者。\textbf{在此},即于此方,是就在这刹那所会集之处而说。\textbf{生物},尽管生物一词在\begin{quoting}真实者 \textit{bhūtasmiṃ},为波逸提。(波逸提第 8 条)\end{quoting}等处指存在,在\begin{quoting}诸比丘!你们得见「此已生成 \textit{bhūtam}」不?(中部·大爱尽经第 401 段)\end{quoting}等处指五蕴,在\begin{quoting}比丘!四大种为因。(中部·大满月经第 86 段)\end{quoting}等处指四种地界等的色,在\begin{quoting}且吞噬时间的生物……(本生第 1:190 颂)\end{quoting}等处指漏尽者,在\begin{quoting}在世间,一切生物都将舍弃积集。(长部·大般涅槃经第 220 段)\end{quoting}等处指一切有情,在\begin{quoting}毁坏生物者。(波逸提第 11 条)\end{quoting}等处指树木等,在\begin{quoting}觉知生物是生物。(中部·根本方法经第 3 段)\end{quoting}等处执取四大王天以下的有情聚而转起,但这里通常应视作非人。\textbf{聚集},即会集。
\item \textbf{地居},即在地面转生者。\textbf{或},即可选。以「凡聚集在此的地居生物」作一可选后,又为再作可选而说\textbf{或在天上},即凡一切聚集在此的在天上转生的生物之义。且此中,自夜摩天直至阿迦腻吒转生的生物,由转生于空中显现的天宫中,当知为「天上的生物」,自此以下,在从须弥山开始直至地面的树木、蔓藤等处居住,及在地上转生的生物,彼等一切由转生于地面与依附于地面的树木、蔓藤、山岭等处,当知为「地居的生物」。
\item 如是,世尊以「或为地居,或在天上」两句分置了一切非人之生物,再以一句概括而说「愿这一切生物欢喜」。\textbf{一切},即无余。\textbf{这}即强调,意即一个也不排除。\textbf{生物},即非人。\textbf{愿欢喜},即愿幸福、生起喜悦之义。\textbf{然后} \textit{atho pi},即为了指向另一话题而把握语句的一对不变词。\textbf{恭敬地聆听所说},即注意、作意、收摄一切心神后,请聆听我带来天之成就与出世间之乐的开示!
\item 如是,此中,世尊以「凡聚集在此的生物」的不定之语概括生物后,再以「或为地居,或在天上」的两分分置,随后再以「这一切生物」作结,以「愿欢喜」之语敦促意乐的成就,以「恭敬地聆听所说」敦促加行的成就,同样,敦促如理作意的成就与外来之声的成就,同样,敦促自身正愿与依止善人的成就以及定、慧之因的成就,便完结了颂。\end{enumerate}

\subsection\*{\textbf{225} {\footnotesize 〔PTS 223〕}}

\textbf{所以,生物们!请全体倾听!散播慈爱给人的子孙!\\}
\textbf{他们日夜供奉牺牲,所以,请守护他们,不要放逸!}

Tasmā hi bhūtā nisāmetha sabbe, mettaṃ karotha mānusiyā pajāya;\\
divā ca ratto ca haranti ye baliṃ, tasmā hi ne rakkhatha appamattā. %\hfill\textcolor{gray}{\footnotesize 2}

\begin{enumerate}\item 这里,\textbf{所以},即原因之语。\textbf{生物们},即呼格。\textbf{请倾听},即请聆听。\textbf{全体},即无余。这说的是什么?因为你们舍弃了天住及彼处受用之成就,为了闻法在此聚集,而非为了观听歌舞,所以,生物们!请全体倾听!或者,因「愿欢喜、恭敬地聆听」之语,见到他们欢喜的状态及恭敬欲闻后,便说「因为你们以欢喜的状态而与自身正愿、如理作意、意乐清净相应,以恭敬欲闻而与依止善人、外来之声为足处的加行清净相应,所以,生物们!请全体倾听」。或者,就上一颂末所说的「所说」,为陈述原因,便说「因为我的所说由避开八无暇\footnote{八无暇 \textit{aṭṭhakkhaṇa}:旧译作「八难」,即当佛、法出世,投生于地狱、畜生、饿鬼、长寿天、边地,或生中国而执邪见、聋盲喑哑、未值佛世等,见\textbf{增支部}第 8:29 经。}之有暇的难得,且由以慧与悲的功德转起多种利益,故极难得,且我欲说此,便说『请聆听所说』,所以,生物们!请全体倾听」,这即是颂中此句之所说。
\item 如是阐述了这原因,敦促留意于自己的所说后,开始说应予倾听者:\textbf{散播慈爱给人的子孙}。其义为:对这些为三种灾祸所扰的人的子孙,请现起友好的状态、利益之意乐。而有人读作「在人间的子孙 \textit{mānusiyaṃ pajan}」,由不成依格之义故不恰当,且其他人所释之义也不恰当。此中的旨趣为:我不以主宰之力而说「我是佛陀」,而是为了你们及人的子孙的利益而说「散播慈爱给人的子孙」。且此中,以\begin{quoting}那些王仙战胜了有情聚落的大地,巡游祭祀:\\马牲、人牲、掷棒祭、娑摩祭、无遮祭\footnote{马牲、人牲、掷棒祭、娑摩祭、无遮祭,见\textbf{婆罗门法经}第 306 颂及其义注。},\\他们不及善习慈心者的十六分之一。\\心不嗔恶哪怕一个生类,他也因行慈成为善人,\\而以意怜悯一切生类,圣者便造广大之福。(增支部第 8:1 经)\end{quoting}如是等经以及十一种利益\footnote{十一种利益,见\textbf{增支部}第 11:15 经与\textbf{清净道论}·说梵住品第 60 段及以下。},当知散播慈爱者的慈之利益,而以\begin{quoting}为天怜悯之人,常常得见祥瑞。(长部·大般涅槃经第 153 段)\end{quoting}如是等经,当知被散播者的利益。
\item 如是,为显示两者的利益说了「散播慈爱给人的子孙」后,现在,为显示资助而说「\textbf{他们日夜供奉牺牲,所以,请守护他们,不要放逸}」。其义为:人们以彩绘、木雕等造了天人后,去往支提、树下等处,为天人在白天作供奉,且在黑分等的夜晚作供奉。或者,布施了行筹食等,为守护的天人乃至梵天的天人,以许愿、还愿在白天作供奉,且教人以布置伞、灯、花鬘而作了整夜的闻法等后,以许愿、还愿在夜晚作供奉,他们如何不应受到守护?既然他们如是日夜为你们作供奉,所以,请守护他们!所以,出于供奉之因,也应守护、保护这些人们!为他们除去不利、带来利益,不要放逸,在心中感恩,常常随念。\end{enumerate}

\subsection\*{\textbf{226} {\footnotesize 〔PTS 224〕}}

\textbf{无论此界或他界的财富,或若天界的胜妙珍宝,\\}
\textbf{都不能与如来等同,\\}
\textbf{这即是佛中的胜妙珍宝,愿以此真实而得平安!}

Yaṃ kiñci vittaṃ idha vā huraṃ vā, saggesu vā yaṃ ratanaṃ paṇītaṃ,\\
na no samaṃ atthi Tathāgatena;\\
idam pi Buddhe ratanaṃ paṇītaṃ, etena saccena suvatthi hotu. %\hfill\textcolor{gray}{\footnotesize 3}

\begin{enumerate}\item 如是显示了人对诸天的资助后,为了平息他们的灾祸而以表明佛等的功德,并为了人天得以闻法而以「无论……财富」等方法,开始发出真实之语。这里,\textbf{无论},是以不定的方式穷尽无余,即言称所及的无论何处。\textbf{财富},即财产,以产生喜悦故为财富。\textbf{此界}指人世间,\textbf{他界}指此外的其余世间。且除人间外,当涉及摄一切世间时,由其后已说「或若天界」,故当知以此摄除人间与天界外的其余龙、金翅鸟等。如是,以此二句,凡人间的言称所及与庄严、受用所及的金、银、珍珠、摩尼、琉璃、珊瑚、红宝石、琥珀等,以及凡在珍珠、摩尼之沙铺设的地面上由珍宝所造的天宫内、宽广数百由旬的居处内所现的龙、金翅鸟等的财富,均被指涉。
\item \textbf{天界}指欲界、色界的天世间。因为他们以净业而不被战胜、得以到达,故为天界,或亦以善妙的顶点故为天界。\textbf{若},即有主者或无主者。\textbf{珍宝},以导向、带来、产生、增长喜乐为珍宝,举凡受尊崇、贵重、无比、难得一见与上人所受用者,即其同义语。如说:\begin{quoting}受尊崇与贵重、无比、难得一见、\\上人所受用者,因此被称为珍宝。\end{quoting}\textbf{胜妙},即最上、最胜、无可餍足。如是,以此偈句,凡量可数百由旬、由一切珍宝所造的天宫,如善法城\footnote{善法城 \textit{Sudhamma},待考。}与最胜殿\footnote{最胜殿 \textit{Vejayanta}:为帝释在三十三天的宫殿。}等中的有主者,以及凡因无佛出世,当有情唯充塞于苦处时,附著于空殿的无主者,或是其它依附于大地、大海、雪山等的无主之宝,均被指涉。
\item 都不能与如来等同:\textbf{不}即遮止,\textbf{都}即强调,\textbf{等同}即相等,\textbf{能}即存在,\textbf{如来}即佛陀。这说的是什么?凡所表明的财富与珍宝,其中连一个与佛宝相等的珍宝也无。
\item 因为凡是\textbf{以受尊崇之义的珍宝},此即如转轮王的轮宝与摩尼宝,当出现时,大众不于别处行尊崇,无人持花、香等去往夜叉处或生物处,全民唯尊崇、供养轮宝与摩尼宝,希求某某愿,且其所求者有些得以成就,但此宝仍不能与佛宝等同。因为若以受尊崇之义为珍宝,则唯如来是珍宝。当如来出世,任何大威德的天、人不于别处行尊崇,不供养任何其他人。如是,娑婆主梵天以量如须弥山的宝环供养如来,其他天、人及频婆娑罗、㤭萨罗王、给孤独等也尽力供养。且对业已般涅槃的世尊,阿育大王在分发了九十六俱胝的财产后,命人在整个阎浮提起了八万四千寺,遑论其他尊崇?况且,还有谁在般涅槃后如世尊般,在其出生、觉悟、转法轮、般涅槃处,或就其影像、支提等转起如是的尊崇、尊重?如是,以受尊崇之义的珍宝无有与如来等同者。
\item 同样,凡是\textbf{以贵重之义的珍宝},此即如迦尸布,如说:\begin{quoting}即便老旧,诸比丘!迦尸布也光鲜、乐触且贵重。(增支部第 3:100 经)\end{quoting}它也无法与佛宝等同。因为若以贵重之义为珍宝,则唯如来是珍宝。因为如来仅受纳其尘土,于彼等也成大果报、大利益,如于阿育王\footnote{阿育王事,见\textbf{天譬喻} \textit{Divyāvadāna} 之廿九 Aśokāvadāna,有梵文本。},这于彼即为贵重。且此中,对贵重之语的无过失证明,当知如这段经文:\begin{quoting}他若受纳其衣、食、坐卧处、疗病之药品资具,这于彼等成大果报、大利益,我说这于彼即为贵重。好比,诸比丘!贵重的迦尸布,我说此人便如此譬。(增支部第 3:100 经)\end{quoting}如是,以贵重之义的珍宝无有与如来等同者。
\item 同样,凡是\textbf{以无比之义的珍宝},此即如转轮王的\textbf{轮宝}出现,轮毂由蓝宝石所造,千辐由七宝所造,轮辋由珊瑚所造,销钉由赤金所造。每十辐上有一总辐,用以捕风作声,所发之声如极娴熟演奏的具足五支的乐声。在轮毂的两侧各有一狮面,内部即车轮的空隙。无有造者或令造者,以业缘随时节等起。当国王履行了十种转轮义务,在十五满月的布萨当日洗濯了头发,持守斋戒,上至最胜的殿堂,洁净诸戒而坐,便见其如满月、又如皦日般升起,其声得闻十二由旬,其色得见一由旬,大众见了便激起哗然「我说,第二个月亮或是太阳升起了」,来至城市上方后,在后宫的东侧不过高、不过低,为大众能以香、花等供养,在合适之处似停轴而住。
\item 随之而来,有\textbf{象宝}出现,通体白色、赤足,七处圆满,具有神变,飞行空中,来自布萨族或六牙族,若来自布萨族,则最长者来,若来自六牙族,则最幼者来,业已修学,具足调御。它带了十二由旬的随从,周行整个阎浮提已,在早餐前回到自己的王城。
\item 随之而来,亦有\textbf{马宝}出现,通体白色,赤足乌首,鬃如萱草,来自乌云之王族。其余皆同象宝。
\item 随之而来,亦有\textbf{摩尼宝}出现,这摩尼琉璃清净纯正,八面玲珑,长如车毂,从方广山而来,它即便在具足四支黑暗的黯黑之中,也能从国王的旗顶放光一由旬,因其光芒,人们以为是白天,便从事工作,甚至能见蝼蚁之细。
\item 随之而来,亦有\textbf{女宝}出现,或生为正妃,或从北俱卢或末陀王族而来,无有过高等的六种欠缺,超越人的美貌而不及天的美貌,当国王寒凉之时,其肢体温暖,当暑热之时则寒凉,触如析为百分之木棉,体散旃檀之香,口吐青莲之息,且具足早起等多种功德。
\item 随之而来,亦有\textbf{家主宝}出现,生来为国王从事财务,当轮宝甫一出现,他便显现天眼,以之得见周匝一由旬内有主或无主的伏藏。他去到国王处,提议道:「请您少待!陛下!我将为您以财理财!」
\item 随之而来,亦有\textbf{首相宝}出现,生为国王长子,当轮宝甫一出现,他便具足极度之黠慧,以心遍知十二由旬会众之心已,堪能折服、振奋。他去到国王处,提议道:「请您少待!陛下!我将为您治理王国!」
\item 或者其它类似以无比之义的珍宝——其价值难以经考量、推度作「价值百千或俱胝」,其中亦无一宝与佛宝等同。因为若以无比之义为珍宝,则唯如来是珍宝。如来不能由戒或定或慧等之中的任何一个经考量、推度而断言「如许功德,以此或有等同者、或有对等者」。如是,以无比之义的珍宝无有与如来等同者。
\item 同样,凡是\textbf{以难得一见之义的珍宝},此即如难得显现的转轮王及其轮等珍宝,也无法与佛宝等同。因为若以难得一见之义为珍宝,则唯如来是珍宝,转轮王等如何成宝?彼等在一劫之内出现数次,而因为在阿僧祇劫内世间空无如来,所以如来唯有时而现,难得一见。且世尊在般涅槃时说:\begin{quoting}阿难!天人们不满道:「我们远道而来,为见如来,在世间,诸如来、阿罗汉、正等正觉者有时出现,却将在今晚的后夜入般涅槃,而这大威德的比丘站在世尊前予以阻止,令我们不得在最后时刻一见如来。」(长部·大般涅槃经第 200 段)\end{quoting}如是,以难得一见之义的珍宝无有与如来等同者。
\item 同样,凡是\textbf{以上人所受用之义的珍宝},此即如转轮王的轮宝等。因为这,对于旃陀罗、篾匠、猎人、车匠、混血者等低贱之家、下贱之人,即便财可百千俱胝,即便住在七层楼阁之上,在梦中也无法受用,而对于(父母)两边出生良好的刹帝利王,履行了十种转轮义务则能受用,唯上人所受用,也无法与佛宝等同。因为若以上人所受用之义为珍宝,则唯如来是珍宝。因为如来,对于在世间被许可为上人、未具足近依、颠倒知见的富楼那迦叶等六师以及其余类似者,在梦中也无法受用,而对于具足近依、乃至在四句偈的终了便能证得阿罗汉、抉择知见的树皮衣者跋西耶等等,以及对于其他出生于大家族者、众大弟子则能受用。因为他们在完成见无上、闻无上、敬事无上等\footnote{六应证法:即见无上、闻无上、利养无上、学无上、敬事无上、随念无上等,见\textbf{长部}·十上经第 356 段。}时,如是如是受用彼。如是,以上人所受用之义的珍宝无有与如来等同者。
\item 而无差别地,凡是\textbf{以产生喜乐之义的珍宝},此即如转轮王的轮宝。因为转轮王见之便心满意足,如是,它令国王产生喜乐。复次,转轮王左手持金瓶,右手洒向轮宝:「愿尊轮宝转动!愿尊轮宝征服!」随后,轮宝便如具足五支的乐器般发出甜美的声音,从空中去往东方,转轮王紧随其后,以轮的威力,绵延十二由旬的四支军队不过高、不过低地——从高树的下部、低树的上部——取了树中的花、果、芽等礼物,他从来者手中接过礼物,并对以「来!大王」等最高的臣服前来的敌王们以「不应杀害生类」等方法进行训诫。
\item 而在国王欲饮食或欲午休之处,轮宝便从空中降落至此,在水等一切设施适可的平整地面上似停轴而住。当国王又生起出行之心时,仍以先前的方法发声而行,听闻之后,十二由旬的会众也从空中前行。轮宝渐次潜入东海,当潜入时,分开一由旬之量的水,似壁而立。大众随意取用七宝。国王又持了金瓶洒水,说了「从此以后,就是我的领土」后返回。军队在前、轮宝在后、国王居中,轮宝撤回之处,水即遍满。仍以此方式,去往南、西、北海。
\item 如是随行了四方,轮宝升至三由旬高的空中。国王站立于此,观察以轮宝的威力所胜的缀以二千小洲的四大洲——缀以五百小洲、周长七千由旬的东毗提诃,周长八千由旬的北俱卢,周长同为七千由旬的西瞿耶尼,以及周长一万由旬的阎浮提——整个轮围好似盛开的芬陀利花丛。他如是观察时,生起匪浅的喜乐,也不能与佛宝等同。因为若以产生喜乐之义为珍宝,则唯如来是珍宝。这轮宝能有什么作为?因为较此圣洁的喜乐,以轮宝等一切所生的转轮王之喜乐不可得名、不及微分、不及微分之份,如来令遵循自己教诫的阿僧祇数的天人生起较此喜乐更上、更胜的初禅之喜乐,二三四五禅之喜乐,空无边处之喜乐,识无边处、无所有处、非想非非想处之喜乐,须陀洹道之喜乐,须陀洹果之喜乐,斯陀含、阿那含、阿罗汉道果之喜乐。如是,以产生喜乐之义的珍宝无有与如来等同者。
\item 而且,名为珍宝者有两种:有识者与无识者。这里,无识者即轮宝、摩尼宝,或其它无根所系的金银等,有识者即从象宝至首相宝,或其他类似的有根所系者。且在如是两种之中,有识的珍宝被称为上。为什么?因为无识的金银摩尼珍珠等珍宝,被用于庄严有识的象宝等。
\item 有识的珍宝又有两种:畜生之珍宝与人之珍宝。这里,人之珍宝被称为上。为什么?因为畜生之珍宝供人之珍宝骑乘。人之珍宝又有两种:女宝与男宝。这里,男宝被称为上。为什么?因为女宝为男宝提供侍奉。男宝又有两种:在家宝与出家宝。这里,出家宝被称为上。为什么?因为在家宝中的上首、转轮王也要对具有戒等功德的出家宝以五体投地作礼拜,且经给侍、承事得达天人的成就,并在最终得达涅槃的成就。
\item 如是,出家宝又以圣者、凡夫而有两种。圣者宝以学、无学而有两种。无学宝又以干观者、止乘者而有两种。止乘者宝又以得达、未达声闻波罗蜜而有两种。这里,得达声闻波罗蜜者被称为上。为什么?功德大故。较之得达声闻波罗蜜者,辟支佛宝被称为上。为什么?功德大故。因为数百与舍利弗、目犍连相等的声闻,也不及一辟支佛功德的百分之一。较之辟支佛宝,正等正觉宝被称为上。为什么?功德大故。因为众辟支佛即便填满整个阎浮提,跏趺抵着跏趺而坐,较之正等正觉的功德,也不可得名、不及微分、不及微分之份。且亦如世尊所说:\begin{quoting}诸比丘!举凡有情,或无足……如来被称为彼等之上。(相应部第 45:139 经)\end{quoting}如是,无论以何方法,无有与如来等同的珍宝。所以世尊说:\textbf{都不能与如来等同}。
\item 如是,世尊在说了佛宝较其余珍宝的不等同性后,现在,为平息彼等有情已生起的灾祸,不依止出身、族姓、族姓子性、肤色之美等,而是依止在从无间以至有顶的世间中,佛宝由戒、定蕴等功德的不等同相,发出真实之语:「这即是佛中的胜妙珍宝,愿以此真实而得平安!」
\item 其义为:无论此界、或他界、或天界的财富、珍宝,由因彼彼功德而不与其等同,\textbf{这即是佛中的胜妙珍宝}。若此为真实,则\textbf{以此真实},\textbf{愿}这些生类\textbf{而得平安},即愿净美者能存在、无病、免离灾祸。且此中,如在\begin{quoting}眼,阿难!以我或以我所而空。(相应部第 35:85 经)\end{quoting}等处,是以我相或以我所相之义,否则便以「眼是我或我所」而不成否定。如是,当知「胜妙珍宝」即宝性为胜妙、宝相为胜妙之义,否则佛陀竟不成为宝了。因为非于有宝之处而成为宝,而是在被称为受尊崇等义之处,或是以任何适当的方式存在相容的宝性,因为据此宝性被施设为宝,所以由此宝性的存在而成为宝。\footnote{「且此中,如在……而成为宝」的一段文字,是就颂中「佛中 \textit{buddhe}」一词为依格而作的解释。}或者,\textbf{这即是佛中的珍宝},即以此原因,唯佛陀是珍宝,如是当知此中之义。
\item 且世尊甫一说完此颂,王族便生平安,怖畏得以平息。此颂的敕命为百千俱胝轮围中的非人所领受。\end{enumerate}

\subsection\*{\textbf{227} {\footnotesize 〔PTS 225〕}}

\textbf{灭尽、离贪、不死、胜妙,这等持的释迦牟尼之所证,\\}
\textbf{无有任何可与这法等同,\\}
\textbf{这即是法中的胜妙珍宝,愿以此真实而得平安!}

Khayaṃ virāgaṃ amataṃ paṇītaṃ, yad ajjhagā Sakyamunī samāhito,\\
na tena dhammena sam’atthi kiñci;\\
idam pi dhamme ratanaṃ paṇītaṃ, etena saccena suvatthi hotu. %\hfill\textcolor{gray}{\footnotesize 4}

\begin{enumerate}\item 如是以佛德说了真实后,现在开始以涅槃法说此颂。这里,因为由证得涅槃,贪等便尽、遍尽,或者因为它即彼等的无生、灭、灭尽,所以得称\textbf{灭尽},且因为它从相应与所缘都与贪等离于相应,或者因为当其得证,贪等便究竟离染、离去、破碎,所以得称\textbf{离贪}。而因为它被认为无生、无老、无住的变异,所以它合「不生、不老、不死」而得称\textbf{不死},以最上之义与无餍足之义得称\textbf{胜妙}。
\item \textbf{这所证},即这所证、所寻、所获,以自身的智力所证得者。\textbf{释迦牟尼},由出生释迦家族而为释迦,由具足寂默法而为牟尼,释迦即牟尼,为释迦牟尼。\textbf{等持},即以圣道之定等持其心。\textbf{无有任何可与这法等同},即无有任何种种法可与此灭尽等名称、为释迦牟尼所证之法等同。所以在经中亦说:\begin{quoting}诸比丘!但凡有为或无为之法,离贪被称为彼等之最上。(增支部第 4:34 经)\end{quoting}
\item 如是,世尊说了涅槃法与其它诸法的不等同性,现在为平息彼等有情已生起的灾祸,依于涅槃法宝由灭尽、离贪、不死、胜妙诸功德的不相等相,发出真实之语:「\textbf{这即是法中的胜妙珍宝,愿以此真实而得平安!}」其义仍以前颂所说的方法当知。此颂的敕命亦为百千俱胝轮围中的非人所领受。\end{enumerate}

\subsection\*{\textbf{228} {\footnotesize 〔PTS 226〕}}

\textbf{这最胜的觉者所赞叹的纯洁,他们称之为无间的定,\\}
\textbf{无有与这定等同者,\\}
\textbf{这即是法中的胜妙珍宝,愿以此真实而得平安!}

Yaṃ buddhaseṭṭho parivaṇṇayī suciṃ, samādhim ānantarikañ ñam āhu,\\
samādhinā tena samo na vijjati;\\
idam pi dhamme ratanaṃ paṇītaṃ, etena saccena suvatthi hotu. %\hfill\textcolor{gray}{\footnotesize 5}

\begin{enumerate}\item 如是以涅槃法之德说了真实后,现在开始以道法之德说此颂。这里,以「觉悟真谛」等方法而为觉者,以最上与应受称赏为最胜,觉者与此最胜为\textbf{最胜的觉者},或以在被称为随觉与辟支觉的觉悟者中为最胜,为最胜的觉者。此最胜的觉者\textbf{所赞叹的},即以\begin{quoting}且八支为道中(最胜),安稳、得证涅槃。(中部·摩根提耶经第 215 段)\end{quoting}与\begin{quoting}诸比丘!我将开示俱基础、俱资助的圣正定。(中部·大四十经第 136 段)\end{quoting}等方法随处称赏、显明者。\textbf{纯洁},即由正断烦恼尘垢而究竟洁净。
\item \textbf{他们称之为无间的定},即由在其自身转起的等无间必然给予果报,他们便称为「无间定」。因为当生起道定时,无有任何遮止其果报发生的障碍。如说:\begin{quoting}且此人若为证得须陀洹果而行道,且劫火时至,则此人未证得须陀洹果,劫便不能起火,此人即被称为住劫者,而所有具道之人都是住劫者。(人施设论第 17 段)\end{quoting}
\item \textbf{无有与这定等同者},即无有任何与这最胜的觉者所赞叹的纯洁、无间定等同的色界定或无色界定。为什么?对于由修习彼等而投生各处梵界者,仍有再生于地狱等的发生,而对于由修习此阿罗汉定的圣人,已根除一切再生的发生。所以在经中亦说:\begin{quoting}诸比丘!但凡是有为法,八支圣道被称为彼等之最上。(增支部第 4:34 经)\end{quoting}
\item 如是,世尊说了无间定与其它定的不等同性,现在仍以前法,依于道法之宝的不相等相,发出真实之语:「\textbf{这即是法中的胜妙珍宝,愿以此真实而得平安!}」其义仍以先前所说的方法当知。此颂的敕命亦为百千俱胝轮围中的非人所领受。\end{enumerate}

\subsection\*{\textbf{229} {\footnotesize 〔PTS 227〕}}

\textbf{那些善人称赞的八补特伽罗,便是这四双,\\}
\textbf{他们是应予供养的善逝弟子,于彼等布施有大果报,\\}
\textbf{这即是僧中的胜妙珍宝,愿以此真实而得平安!}

Ye puggalā aṭṭha sataṃ pasatthā, cattāri etāni yugāni honti;\\
te dakkhiṇeyyā Sugatassa sāvakā, etesu dinnāni mahapphalāni;\\
idam pi saṅghe ratanaṃ paṇītaṃ, etena saccena suvatthi hotu. %\hfill\textcolor{gray}{\footnotesize 6}

\begin{enumerate}\item 如是以道法之德说了真实后,现在开始以僧伽之德说此颂。这里,\textbf{那些},即不定之泛指。\textbf{补特伽罗},即有情。\textbf{八},即彼等数量的限定。因为他们以四行道与四住果而为八。\textbf{善人称赞},即为佛、辟支佛、声闻等善人及其他天人所称赏。为什么?与俱生的戒等功德相应故。因为如素馨、紫荆花等俱生色香等,彼等俱生戒、定等功德,因此,如具足色香等的花,彼等为天、人之善者喜爱、悦意、称赞。因此而说「那些善人称赞的八补特伽罗」。
\item 或者,\textbf{那些},即不定之泛指。\textbf{补特伽罗},即有情。\textbf{百八}\footnote{百八 \textit{aṭṭhasataṃ}:义注是将「善人 \textit{sant}」的复数属格解释为「一百 \textit{sata}」的体格。},即彼等数量的限定。因为他们以一种子、家家、仅七次为三种须陀洹,以于欲、色、无色有中证果为三种斯陀含,两者都依四种行道而成二十四种,在无烦天中以中般涅槃、生般涅槃、有行般涅槃、无行般涅槃、上流至阿迦腻吒为五种,在无热、善见、善现天中也如是,而在阿迦腻吒中,除上流而为四种,如是成二十四种阿那含,以干观者、止乘者为二种阿罗汉,加四住道者共五十四,他们又都以信轭、慧轭成两倍,即成百八。余如前述。
\item \textbf{便是这四双},即那些以八或百八详示的补特伽罗,略说即以须陀洹住道、住果为一双,如是直至以阿罗汉住道、住果为一双,便是四双。
\item 在「他们应予供养」中,\textbf{他们},即先前不定所示的确定指称。他们即那些以详说为八或百八、以略说为四双的全体。值得供养为\textbf{应予供养}。供养,即信业与业果、不希求「他会给予我医疗,或为我走使传讯」等等而被布施的所施,值得此者即与戒等德相应的人,且他们便是如此,因此被称为「应予供养」。
\item \textbf{善逝弟子},由与善净而来相应,以及由至于善净之处、由善来、由善语,世尊为善逝,即此善逝。所有那些听闻言语者为弟子,其他人虽也听闻,却闻后不作当作的义务,而这些人在闻后作了当作的法、随法之行道,得至大果报,所以被称为「弟子」。
\item \textbf{于彼等布施有大果报},于彼等善逝弟子即便作少量施舍之布施,由藉领受者而至供养清净相,有大果报。所以在经中亦说:\begin{quoting}诸比丘!但凡是僧伽或僧众,如来弟子之僧伽被称为彼等之最上,此即是四双八辈,这世尊的声闻僧伽……最上异熟。(增支部第 4:34 经)\end{quoting}
\item 如是,世尊以所有住道、住果者说了僧宝之德,现在即依此德,发出真实之语:「\textbf{这即是僧中……}」其义仍以先前所说的方法当知。此颂的敕命亦为百千俱胝轮围中的非人所领受。\end{enumerate}

\subsection\*{\textbf{230} {\footnotesize 〔PTS 228〕}}

\textbf{那些在乔达摩的教法中以坚固的心意善加致力的无欲者,\\}
\textbf{他们已达成就,跃入不死,享用着无偿获得的寂灭,\\}
\textbf{这即是僧中的胜妙珍宝,愿以此真实而得平安!}

Ye suppayuttā manasā daḷhena, nikkāmino Gotamasāsanamhi;\\
te pattipattā amataṃ vigayha, laddhā mudhā nibbutiṃ bhuñjamānā;\\
idam pi saṅghe ratanaṃ paṇītaṃ, etena saccena suvatthi hotu. %\hfill\textcolor{gray}{\footnotesize 7}

\begin{enumerate}\item 如是以住道、住果者的僧伽之德说了真实后,现在开始仅以部分体验果定之乐的漏尽补特伽罗之德说此颂。这里,\textbf{那些},即不定泛指之语。\textbf{善加致力},即舍弃了各种邪求,依清净的活命,开始致力自身于毗婆舍那之义。或者,善加致力即具足遍净的身语加行,以此显明彼等的戒蕴。\textbf{坚固的心意},即与牢固的定相应的心之义,以此显明彼等的定蕴。\textbf{无欲者},即不关切身命,以慧轭之精进从一切烦恼已出离者,以此显明彼等具足精进的慧蕴。\textbf{在乔达摩的教法中},即在由族姓而为乔达摩如来的教法中。
\item \textbf{他们},即先前不定泛指的确指之语。\textbf{已达成就},此中以「应至」为成就,应至即值得到达,所达为究竟离轭安稳,即阿罗汉果的同义语,已达此成就为已达成就。\textbf{不死},即涅槃。\textbf{跃入},即随所缘跃入。\textbf{无偿},即无开销,连一硬币也不花费。\textbf{寂灭},即安息了烦恼、恼患的果定。\textbf{享用},即体验。
\item 这说的是什么?那些在此乔达摩的教法中由具足戒而善加致力,由具足定而心意坚固,由具足慧而无欲之人,他们以此正行道跃入不死,享用着被认为是无偿获得的果定的寂灭,而已达成就。
\item 如是,世尊仅以体验果定之乐的漏尽补特伽罗说了僧宝之德,现在即依此德,发出真实之语:「\textbf{这即是僧中……}」其义仍以先前所说的方法当知。此颂的敕命亦为百千俱胝轮围中的非人所领受。\end{enumerate}

\subsection\*{\textbf{231} {\footnotesize 〔PTS 229〕}}

\textbf{好比嵌于地中的因陀柱,不为四方的风所动摇,\\}
\textbf{我说像这样的便是善人,他已了知而得见圣谛,\\}
\textbf{这即是僧中的胜妙珍宝,愿以此真实而得平安!}

Yath’indakhīlo pathavissito siyā, catubbhi vātehi asampakampiyo;\\
tathūpamaṃ sappurisaṃ vadāmi, yo ariyasaccāni avecca passati;\\
idam pi saṅghe ratanaṃ paṇītaṃ, etena saccena suvatthi hotu. %\hfill\textcolor{gray}{\footnotesize 8}

\begin{enumerate}\item 如是以漏尽补特伽罗之德说了基于僧伽的真实后,现在开始仅以众人现量\footnote{众人现量 \textit{bahujana-paccakkha}:菩提比丘注 970 亦云难解,且三藏及注疏内仅此一见,但对比下颂之区别三种须陀洹而言,当是合而论之之意。与此相对的「自己现量 \textit{atta-paccakkha}」,见\textbf{迅速经}第 928 颂。}的须陀洹之德说此颂。这里,\textbf{好比},即譬喻之词。\textbf{因陀柱},即为了防护城门,在边界上掘开八或十手土地后被锤入的心木制成的柱子的同义语。\textbf{地中},即土中。\textbf{嵌于},即进入内部而倚靠。\textbf{四方的风},即四方来风。\textbf{不为所动摇},即不可令其晃动或撼动。\textbf{像这样的},即如此种类的。\textbf{善人},即至上之人。\textbf{他已了知\footnote{了知 \textit{avecca}:即旧译「不坏净 \textit{avecca-pasāda}」之不坏,义注以「潜入 \textit{ajjhogāhetvā}」解之。}而得见圣谛},即他已以慧潜入而得见四圣谛。这里,圣谛当以清净道论中所述之法而知\footnote{见\textbf{清净道论}·说根谛品第 13 段及以下。}。
\item 而此中,其略义为:好比因陀柱根基深厚,嵌于地中,不为四方的风所动摇,我说这善人也像这样,他已了知而得见圣谛。为什么?因为他也如因陀柱之于四方的风一般,不为一切外道教旨之风所动摇,藉由其知见,不可令其晃动或撼动。所以在经中亦说:\begin{quoting}好比,诸比丘!铁柱或因陀柱根基深厚,善加挖掘,不晃不摇,若狂风暴雨从东方来,不能动摇、晃动、撼动它,从西方……南方……北方来,不能……撼动它。其因为何?诸比丘!因陀柱的根基深厚故、善掘故。如是,诸比丘!任何如实正知「此是苦……行道」的沙门、婆罗门,他们不会观察其他沙门、婆罗门的脸说:「此君莫非以知而知、以见而见?」其因为何?诸比丘!由善见四圣谛故。(相应部第 56:39 经)\end{quoting}
\item 如是,世尊仅以众人现量的须陀洹说了僧宝之德,现在即依此德,发出真实之语:「\textbf{这即是僧中……}」其义仍以先前所说的方法当知。此颂的敕命亦为百千俱胝轮围中的非人所领受。\end{enumerate}

\subsection\*{\textbf{232} {\footnotesize 〔PTS 230〕}}

\textbf{那些彻晓由深慧者所善开示的圣谛者,\\}
\textbf{他们即使极度放逸,也不会取第八有,\\}
\textbf{这即是僧中的胜妙珍宝,愿以此真实而得平安!}

Ye ariyasaccāni vibhāvayanti, gambhīrapaññena sudesitāni;\\
kiñcāpi te honti bhusaṃ pamattā, na te bhavaṃ aṭṭhamam ādiyanti;\\
idam pi saṅghe ratanaṃ paṇītaṃ, etena saccena suvatthi hotu. %\hfill\textcolor{gray}{\footnotesize 9}

\begin{enumerate}\item 如是无差别地以须陀洹之德说了基于僧伽的真实后,现在开始以一种子、家家、仅七次三种须陀洹中最低的仅七次者之德说此颂。如说:\begin{quoting}于此,有些人由尽三结而成须陀洹……他唯经转生一有便尽苦边,此即一种子者。同样,经转世、轮回二三家便尽苦边,此即家家者。同样,在天、人之中经七次转世、轮回便尽苦边,此即仅七次者。(人施设论第 31~33 段)\end{quoting}
\item 这里,\textbf{圣谛},已如前述。\textbf{彻晓},即以智慧之光驱散遮蔽真谛的烦恼暗冥,使之对己明白畅晓。\textbf{深慧},即由慧之不可量,而以俱有天的世间之智也不可得至圆满之慧,即是说一切知。\textbf{善开示},即以综合、分析、全体、部分等种种方法善加开示。
\item \textbf{他们即使极度放逸},即那些彻晓圣谛的人们,即使有天之王位、转轮王位等放逸的缘由而极度放逸,也同样以须陀洹道智,由行作识之灭,除七有外,由在无始轮回中能生起的名与色的灭、没,\textbf{不会取第八有},而在第七有中便开始修观,圆满阿罗汉。
\item 如是,世尊以仅七次者说了僧宝之德,现在即依此德,发出真实之语:「\textbf{这即是僧中……}」其义仍以先前所说的方法当知。此颂的敕命亦为百千俱胝轮围中的非人所领受。\end{enumerate}

\subsection\*{\textbf{233} {\footnotesize 〔PTS 231\textit{a-d}〕}}

\textbf{伴随其知见的成就,三法就已舍断:\\}
\textbf{有身见与疑,或是任何存在的戒禁,}

Sahā v’assa dassanasampadāya, tayas su dhammā jahitā bhavanti;\\
sakkāyadiṭṭhī vicikicchitañ ca, sīlabbataṃ vā pi yad atthi kiñci. %\hfill\textcolor{gray}{\footnotesize 10}

\begin{enumerate}\item 如是以仅七有者的不取第八有之德说了基于僧伽的真实后,现在开始即以此取七有者相较于其他取未舍断之有的补特伽罗的殊胜之德说此颂。这里,\textbf{其},即「也不会取第八有」中所说的某个。\textbf{知见的成就},即须陀洹道的成就。因为须陀洹道见到涅槃后,由应作义务的成就,以最初得见涅槃而被称为「知见」,其在自身的显现为「知见的成就」。\textbf{三法就已舍断}中的「就 \textit{su}」,即补足语句的不变词,如\begin{quoting}舍利弗!这就是我极污秽的饮食。(中部·大狮子吼经第 156 段)\end{quoting}等处一般。此中之义为:因为伴随其知见的成就,三法已舍断、已舍弃。
\item 现在,为了显示被舍断之法而说后半颂。这里,在存在之身中、在现存的被称为五取蕴之身中的二十事之见为\textbf{有身见}。或者,在此身中存在之见为有身见,即在所说品类之身中现存之见之义。或者,在存在之身中的见为有身见,即在现存的所说品类之身中以「被称为色等的即我」转起之见之义。且由其舍断,便舍断一切见,因其为彼等的根本。
\item 由平息一切烦恼病患,慧被称为「治疗」,此慧之治疗自此离去,或从此慧之治疗的离去为\textbf{疑},以「疑惑于大师」等方法所说的对八事的疑虑\footnote{八事的疑虑:即疑大师、法、僧、学、过去、未来、过去未来、此缘性之缘生法。}为其同义语。由其舍断,便舍断一切疑,因其为彼等的根本。
\item 在\begin{quoting}对于此外的沙门、婆罗门,以戒而清净,以禁而清净。(法集论第 1222 段)\end{quoting}如是等处所及的牛戒、狗戒等的戒,牛禁、狗禁等的禁,被称为\textbf{戒禁}。由其舍断,便舍断一切裸行、秃头等旨在不死的苦行,因其为彼等的根本。因此在以上的结尾处说\textbf{任何存在的}。且此中当知以苦的知见成就而舍断有身见,以集的知见成就而舍断疑,以道、涅槃的知见成就而舍断戒禁。\end{enumerate}

\subsection\*{\textbf{234} {\footnotesize 〔PTS 231\textit{e-h}〕}}

\textbf{已脱离于四苦处,且不可能犯六重罪,\\}
\textbf{这即是僧中的胜妙珍宝,愿以此真实而得平安!}

Catūh’apāyehi ca vippamutto, chac cābhiṭhānāni abhabba kātuṃ;\\
idam pi saṅghe ratanaṃ paṇītaṃ, etena saccena suvatthi hotu. %\hfill\textcolor{gray}{\footnotesize 11}

\begin{enumerate}\item 如是在显示了烦恼轮的舍断后,现在为显示在彼烦恼轮存在时,异熟轮必以之而存,由舍断彼(烦恼轮)则亦舍断此(异熟轮),而说\textbf{已脱离于四苦处}。这里,四苦处即地狱、畜生、饿鬼域、阿修罗众,他即便受取七有,也脱离于此等之义。
\item 如是在显示了异熟轮的舍断后,现在为显示作为此异熟轮根本的业轮的舍断,而说\textbf{且不可能犯六重罪}。这里,重罪即粗重罪,他不可能犯此六事。而这些,当知即如\begin{quoting}诸比丘!这具足见的人,无可能、无机会夺取母亲的性命。(增支部第 1:271 经)\end{quoting}等方法,在(增支部)一集中所说的杀母、杀父、杀阿罗汉、出(佛身)血、破僧、认可其他大师等业。
\item 虽然具足见的圣弟子连蝼蚁的性命都不会夺取,但为了呵责凡夫之相而说此等。因为凡夫由未具足见,甚至会犯如是有极大罪过的重罪,而具足知见者却不可能去犯此等。且此中提及的不可能,是为显示即便在有间也不犯。因为在有间,他即便不知晓自己的圣弟子之相,却以法性或不犯此六事,或不犯以天性杀生等五种敌对以及认可其他大师等六事——有人据此读作「六及六重罪 \textit{cha chābhiṭhānāni}」。拾取死鱼等的圣弟子村童可作此中的例证。
\item 如是,世尊以即便取七有的圣弟子相较于其他取未舍断之有的补特伽罗的殊胜之德说了僧宝之德,现在即依此德,发出真实之语:「\textbf{这即是僧中……}」其义仍以先前所说的方法当知。此颂的敕命亦为百千俱胝轮围中的非人所领受。\end{enumerate}

\subsection\*{\textbf{235} {\footnotesize 〔PTS 232〕}}

\textbf{他即便以身、语、意造作恶业,\\}
\textbf{也不可能覆藏它,称之为已见境地的不可能性,\\}
\textbf{这即是僧中的胜妙珍宝,愿以此真实而得平安!}

Kiñcāpi so kamma karoti pāpakaṃ, kāyena vācā uda cetasā vā;\\
abhabba so tassa paṭicchadāya, abhabbatā diṭṭhapadassa vuttā;\\
idam pi saṅghe ratanaṃ paṇītaṃ, etena saccena suvatthi hotu. %\hfill\textcolor{gray}{\footnotesize 12}

\begin{enumerate}\item 如是以即便取七有者相较于其他取未舍断之有的补特伽罗的殊胜之德说了基于僧伽的真实后,现在开始以「具足知见者不仅不可能犯六重罪,连造了少许恶业也不可能覆藏它」之具足知见却住于放逸者无有覆藏所作之德说此颂。
\item 其义为:\textbf{他},具足知见者,\textbf{即便}以失念而至住于放逸,除世尊就非故意违犯\footnote{非故意违犯:PTS 本作「故意违犯」。}的世间罪所说的\begin{quoting}我为弟子所施设的学处,我的弟子即便因性命也不违犯。(小品第 385 段)\end{quoting}以外,\textbf{以身造作}其它如建造寮房\footnote{建造寮房:即僧残第 6 条等。}、共住\footnote{共住:即波逸提第 5,6 条。}等,或被称为施设、罪过、违犯等佛陀所叱责的\textbf{恶业},或者\textbf{以语}造作逐句诵法\footnote{逐句诵法:即波逸提第 4 条。}、说法过五六句\footnote{说法过五六句:即波逸提第 7 条。}、绮语、恶口等,或者\textbf{以意}造作任何生起贪嗔、接受金银、受用衣等时未予省思等恶业。\textbf{不可能覆藏它},他了知到「这不合适、不应作」后,不会作片刻的覆藏,而是在那刹那到大师或有智的同梵行处发露、如法忏悔,或以「我不再犯」防护应防护者。
\item 为什么?因为\textbf{称之为已见境地的不可能性},即称之为造作如是恶业后,已见涅槃境地的具足知见之人去覆藏它的不可能性之义。如何?\begin{quoting}诸比丘!好比年幼、未蒙、仰卧的童子或以手或以足踏上焦炭便迅速收回,如是,诸比丘!此即具足见之人的法性,他即便违犯如是种类的罪——其出罪已被施设,然而,他迅速在大师或有智的同梵行处坦白、揭露、澄清,坦白、揭露、澄清后,防护于未来。(中部·㤭赏弥经第 496 段)\end{quoting}
\item 如是,世尊以具足知见却住于放逸者无有覆藏所作之德说了僧宝之德,现在即依此德,发出真实之语:「\textbf{这即是僧中……}」其义仍以先前所说的方法当知。此颂的敕命亦为百千俱胝轮围中的非人所领受。\end{enumerate}

\subsection\*{\textbf{236} {\footnotesize 〔PTS 233〕}}

\textbf{好比林木丛薮中的花冠,正值热季月份的初暑,\\}
\textbf{他开示了像这样的最上法,趣向涅槃,为了最高的利益,\\}
\textbf{这即是佛中的胜妙珍宝,愿以此真实而得平安!}

Vanappagumbe yatha phussitagge, gimhāna māse paṭhamasmiṃ gimhe;\\
tathūpamaṃ dhammavaraṃ adesayi, nibbānagāmiṃ paramaṃhitāya;\\
idam pi Buddhe ratanaṃ paṇītaṃ, etena saccena suvatthi hotu. %\hfill\textcolor{gray}{\footnotesize 13}

\begin{enumerate}\item 如是以系属于僧伽的补特伽罗的种种品类之德说了基于僧伽的真实后,现在依于世尊为显明三宝之德而在此简略及别处详细开示的圣典之法,再次开始说基于佛陀之真实的此颂。
\item 这里,由相邻排布所界定的树木的集合为「林木」,由根、心材、边材、皮、枝、叶所繁滋的树丛为「丛薮」,林木中的丛薮为林木丛薮,这即是说(作依格的)\textbf{林木丛薮中},如在「有有寻有伺中,有无寻唯伺中,苦乐之生中」等处也可以这样说一般。\textbf{好比},即譬喻之词。以花为其冠,即\textbf{花冠},即在所有枝桠中所开的花之义,它也以先前所说的方法为依格。\textbf{热季月份的初暑},即四个热月中的某月。若问在哪个月?初暑,即制怛罗月\footnote{制怛罗月 \textit{Citramāsa}:即热季的第一个月,在三月中至四月中,随后便是卫塞节所在的毗舍佉月 \textit{Visākhā}。}之义——它被称为初暑或仲春。此后的词义自明。
\item 而此中,其全义为:好比在名为初暑的仲春,包含种种树的林木中枝头盛开花朵,又名为灌木丛的丛薮极度灿烂,如是,\textbf{像这样}由蕴处等、念处正勤等、或戒定蕴等种种品类的义类之花而极度灿烂,由显明趣向涅槃之道而\textbf{趣向涅槃}的\textbf{最上}圣典之\textbf{法},他既非因利养,也非因恭敬等,而仅仅出于大悲便勇猛其心,\textbf{为了}有情\textbf{最高的利益}而\textbf{开示}。在 paramaṃhitāya 中,为易于结颂而插入鼻音 \textit{ṃ},即「他为最高的利益、为涅槃而开示」之义。
\item 如是,世尊说了这与盛开的林木丛薮相似的圣典之法,现在即依此,发出基于佛陀的真实之语:「\textbf{这即是佛中……}」其义仍以先前所说的方法当知,只是应连接为:这被称为所说品类的圣典之法,即是佛中的胜妙珍宝。此颂的敕命亦为百千俱胝轮围中的非人所领受。\end{enumerate}

\subsection\*{\textbf{237} {\footnotesize 〔PTS 234〕}}

\textbf{最上、知最上、施最上、持最上的无上士开示了最上法,\\}
\textbf{这即是佛中的胜妙珍宝,愿以此真实而得平安!}

Varo varaññū varado varāharo, Anuttaro dhammavaraṃ adesayi;\\
idam pi Buddhe ratanaṃ paṇītaṃ, etena saccena suvatthi hotu. %\hfill\textcolor{gray}{\footnotesize 14}

\begin{enumerate}\item 如是世尊以圣典之法说了基于佛陀的真实后,现在开始以出世间法说此颂。这里,\textbf{最上},即为信解胜妙者所希望「哎!我们也要这样」,或与最上的功德相应为最上,即最高、最胜之义。\textbf{知最上},即知涅槃。因为涅槃以一切法之最高之义为最上,且他在菩提树下经亲自通达而了知此。\textbf{施最上},即对五众\footnote{五众 \textit{pañcavaggiya}:即㤭陈如等五比丘。}、贤众\footnote{贤众 \textit{bhaddavaggiya},见\textbf{律藏}·大品第 36 段。}、萦发\footnote{萦发 \textit{jaṭila},见\textbf{律藏}·大品第 54 段,即为说\textbf{燃烧经}者。}等与其它天人施以抉择分、熏习分之最上法之义。\textbf{持最上},即由持来最上之道而被称为「持最上」。因为彼世尊自燃灯始,为圆满整三十波罗蜜,持来了为先前的正等正觉者们追随的古昔最上之道,因此被称为「持最上」。而且,以获得一切知智为最上,以证得涅槃为知最上,以施以有情解脱之乐为施最上,以持来最高的行道为持最上,由无有较此等出世间德更多者为\textbf{无上士}。
\item 另一方法为:以圆满寂止摄持为最上,以圆满慧摄持为知最上,以圆满舍摄持为施最上,以圆满谛摄持持来最上的道谛为持最上\footnote{摄持 \textit{adhiṭṭhāna}:或可译作依处、基础。上述四种摄持,见\textbf{中部}·界分别经。}。同样,以福德积集为最上,以慧积集为知最上,以给予欲求佛性者以方法为施最上,以为欲求辟支佛性者持来方法为持最上,以于处处无等同性,或由自己而成无师者,以作为他人的老师为无上士。\textbf{开示了最上法},即为了欲求声闻性者的义利,开示与善说等德相应的最上法。
\item 如是,世尊以九种出世间法说了自身的功德,现在即依此德,发出基于佛陀的真实之语:「\textbf{这即是佛中……}」其义仍以先前所说的方法当知,只是应如是连接:他所了知、施与、持来、开示的最上的九出世间法,这即是佛中的胜妙珍宝。此颂的敕命亦为百千俱胝轮围中的非人所领受。\end{enumerate}

\subsection\*{\textbf{238} {\footnotesize 〔PTS 235〕}}

\textbf{旧已灭尽,新无生起,于未来有心已离染,\\}
\textbf{他们种子灭尽,欲不增长,智者们消尽,如那明灯,\\}
\textbf{这即是僧中的胜妙珍宝,愿以此真实而得平安!}

Khīṇaṃ purāṇaṃ nava natthi sambhavaṃ, virattacitt’āyatike bhavasmiṃ;\\
te khīṇabījā avirūḷhichandā, nibbanti dhīrā yathāyaṃ padīpo;\\
idam pi saṅghe ratanaṃ paṇītaṃ, etena saccena suvatthi hotu. %\hfill\textcolor{gray}{\footnotesize 15}

\begin{enumerate}\item 如是世尊依圣典之法与出世间法以两颂说了基于佛陀的真实后,现在依于那些听闻了圣典之法,并以遵循所闻行道而证得了九种出世间法者所证的无余依涅槃之德,再次开始说基于僧伽之真实的此颂。
\item 这里,\textbf{灭尽},即破坏。\textbf{旧},即先前的。\textbf{新},即现今正在发生的。\textbf{无生起},即显现为不存在。\textbf{未来有},即未来时的再有。\textbf{他们},即那些旧已灭尽、新无生起以及于未来有心已离染的漏尽比丘。\textbf{种子灭尽},即种子已坏。\textbf{欲不增长},即无有欲的增长。\textbf{消尽},即熄灭。\textbf{智者},即具足坚毅。
\item 这说的是什么?虽然有情过去时的旧业生已即灭,由未舍弃渴爱之黏腻,堪能带来结生故未灭尽,而对那些以阿罗汉道烧尽了渴爱之黏腻者,旧业如以火烧尽的种子一般,在未来不能给予异熟故灭尽。且他们现在以供佛等转起的业被称为「新」,它也以舍弃渴爱,如断根之树的花一般,在未来不能给予果报故,对彼等无有生起。
\item 且那些以舍弃渴爱而于未来有心已离染的漏尽比丘,以在\begin{quoting}业为田,识为种子。(增支部第 3:77 经)\end{quoting}中所说的结生识因业尽而灭尽故,种子灭尽。而先前为了被称为再有的增长而曾存的欲,也因集的舍弃而舍弃故,以不如过去一般在死时生起而欲不增长,具足坚毅的智者们以最后识的灭,如那明灯灭去般消尽,又超越了「俱色、不俱色」等施设方式。据说,就在此时,为了供养城内的天人而燃的灯中,有一盏熄灭了,为显明此而说「如那明灯」。
\item 如是,世尊说了那些听闻了以前两颂所说的圣典之法,并以遵循所闻行道而证得了九种出世间法者所证的无余依涅槃之德,现在即依此德,发出基于僧伽的真实之语而总结道:「\textbf{这即是僧中……}」其义仍以先前所说的方法当知,只是应如是连接:这以所说品类而被称为漏尽比丘的涅槃,即是僧中的胜妙珍宝。此颂的敕命亦为百千俱胝轮围中的非人所领受。
\item 在开示终了,王族便获平安,一切灾祸得以平息,八万四千生类得了法的现观。\end{enumerate}

\subsection\*{\textbf{239} {\footnotesize 〔PTS 236〕}}

\textbf{凡聚集在此的生物,或为地居,或在天上,\\}
\textbf{我们礼敬人天供养的如来的佛,愿得平安!}

Yānīdha bhūtāni samāgatāni, bhummāni vā yāni va antalikkhe;\\
Tathāgataṃ devamanussapūjitaṃ, Buddhaṃ namassāma suvatthi hotu. %\hfill\textcolor{gray}{\footnotesize 16}

\subsection\*{\textbf{240} {\footnotesize 〔PTS 237〕}}

\textbf{凡聚集在此的生物,或为地居,或在天上,\\}
\textbf{我们礼敬人天供养的如来的法,愿得平安!}

Yānīdha bhūtāni samāgatāni, bhummāni vā yāni va antalikkhe;\\
Tathāgataṃ devamanussapūjitaṃ, dhammaṃ namassāma suvatthi hotu. %\hfill\textcolor{gray}{\footnotesize 17}

\subsection\*{\textbf{241} {\footnotesize 〔PTS 238〕}}

\textbf{凡聚集在此的生物,或为地居,或在天上,\\}
\textbf{我们礼敬人天供养的如来的僧,愿得平安!}

Yānīdha bhūtāni samāgatāni, bhummāni vā yāni va antalikkhe;\\
Tathāgataṃ devamanussapūjitaṃ, saṅghaṃ namassāma suvatthi hotū ti. %\hfill\textcolor{gray}{\footnotesize 18}

\begin{enumerate}\item 于是,诸天之因陀帝释想道「世尊依三宝之德,发出真实之语,令城获安,我也应为了城的平安,依三宝之德说些什么」,便在最后说了三颂。这里,因为佛陀,如同为了世间的利益具有热情者当来而来,又如同彼等当去而去,或者如同彼等当觉知而觉知,又如同当了知而了知,且即如所是而语,故被称为\textbf{如来}。且因为他为\textbf{人天}以花、香等外来的资助及法、随法行等发生于自身者所极度\textbf{供养},所以诸天之因陀帝释将全体天众与自己混同起来说:「\textbf{我们礼敬}人天供养的如来的\textbf{佛,愿得平安!}」
\item 而因为在法中,道法,如同双运止观之力正断烦恼边者当行而行,故为如来,而涅槃法,如同已至者、以慧通达者成就破碎一切苦,为佛等所到达,所以被称为如来。因为僧伽,如同为了自己的利益而行道者当行,即以彼彼道而行,所以也被称为如来\footnote{这里,就「法、僧」来说,应当译作「如去、如行」,但为了和上文和习惯统一,仍作「如来」。}。所以在最后两颂说「\textbf{我们礼敬人天供养的如来的法,我们礼敬人天供养的如来的僧}」。其余仍如所述。
\item 如是,诸天之因陀帝释说完三颂,右绕了世尊,便与天众一起回到天城。世尊则在第二天又开示了这宝经,又有八万四千生类得了法的现观。如是,世尊一直开示到第七天,每天都有如是法的现观。
\item 世尊在毗舍离住了半月,便告知二王:「我们将行。」随后,二王以双倍的恭敬经三天送世尊到恒河岸边。转生于恒河的诸龙王便想:「人类已对如来行了恭敬,我们如何不行?」建造了金银摩尼制成的船只,设好金银摩尼制成的座位,以五色莲花覆盖水面,走到世尊处:「请惠顾我等!」世尊应允后,便登上宝船,五百比丘也上了各自的船。诸龙王便带世尊与比丘僧团到了龙宫。世尊在那里整夜为龙众开示了法。第二天,他们便以天人的硬食、软食作了大供养。世尊随喜后,便从龙宫离开。
\item 地居的天人们想:「人类与龙已对如来行了恭敬,我们如何不行?」便在林木、丛薮、树、山等处撑起层层叠叠的伞。以此方式,直至阿迦腻吒梵天居处,便发起了殊胜的大恭敬。频婆娑罗也以从离车来时所作的双倍行了恭敬,仍以先前所述的方法,以五天送世尊到王舍城。
\item 当世尊到达王舍城时,饭后,聚集在圆亭的众比丘间便生起这闲谈:「哎!佛世尊的威力!仰仗于此,恒河两岸八由旬的地面高下得平,铺以细沙,覆以香花,恒河内一由旬的水中开满种种颜色的莲花,直至阿迦腻吒的居处撑起层层叠叠的伞。」世尊了知其经过,便从香房出来,以随适于彼时的神变来到圆亭,坐在设好的最上佛座。坐下后,世尊告诸比丘:「诸比丘!你们现今聚集,作何言谈?」众比丘便告知一切。世尊便说:「诸比丘!此殊胜的供养非因我佛陀的威力而发生,也非因龙、天、梵的威力,而是因先前少许遍舍的威力而发生。」众比丘便说:「尊者!我们尚不知晓此少许的遍舍,善哉!请世尊对我等讲述,好让我等知晓!」
\item 世尊便说——往昔,诸比丘!在呾叉始罗有婆罗门名为商佉。他的儿子是名为善界的学童,当年满十六岁时,一天,他来到父亲处,顶礼后站在一边。父亲便问他:「怎么了?亲爱的善界!」他便说:「亲爱的!我想去波罗奈学习学业。」「那么,亲爱的善界!名为某某的婆罗门是我的朋友,去他的跟前学习吧!」便给了一千钱币。他拿了后,便顶礼了父母,渐次到了波罗奈,以合于侍奉的适当方式前往老师处,顶礼后,表明了自身。老师想「我朋友的儿子」,便接受了学童,善加款待。他除遣了旅途的劳顿,便将一千钱币置于老师的足下,请求学习学业。老师便提供机会,令其学习。
\item 他学得又快又多,且于所学,如投入金盆中的狮油一般,忆持而不忘失,仅几个月就完成了十二年的学业。他在诵经时,只看见初、中却无终了,于是前往老师处说:「我只看见此学的初、中,却没看见终了。」老师便说:「亲爱的!我也是这样。」「那么,老师!谁知晓此学的终了?」「亲爱的!在仙人堕处有众仙人,他们能知晓。」「我要前去问他们,老师!」「去吧!亲爱的!随所乐意。」
\item 他到了仙人堕处,前往众辟支佛处便问:「你们知晓初、中、后吗?」「是的,朋友!我们知晓。」「请你们也教教我!」「那么,朋友!请你出家!未出家者不堪受教。」「善哉!尊者!请度我出家,或随意而为,令我知晓此终了。」他们便度他出家,却不能指导业处,以「当如是著下衣、如是披上衣」等方法教授等正行。他在那里学习时,由具足近依,不久竟等觉了辟支菩提。在整个波罗奈以「善界辟支佛」而知名,到达利养名闻的顶点,具有随从。他由曾作导致短寿的业,不久就般涅槃了。众辟支佛和大众礼葬他后,便拾取舍利,在城门建了塔。
\item 于是,商佉婆罗门想「我儿子去了很久,我却得不到他的消息」,想见儿子,离开呾叉始罗,渐次到了波罗奈,看到大众聚集,想到「众人之中,肯定有一个知晓我儿子的消息」,便前去问道:「善界学童前已到此,你们知晓他的消息吗?」他们便说:「唯!婆罗门!我们知晓,他在此城内的婆罗门跟前通晓了三明,在众辟支佛跟前出家成了辟支佛,以无余依涅槃界而般涅槃了,这塔就为他而建。」
\item 他以手捶地,嚎啕大哭,去到支提的庭院,除了草,用上衣带了沙子,铺在辟支佛支提的庭院里,把水罐的水浇洒在周围的地上,以林木的鲜花作了供养,用上衣竖起幢幡,在塔的上方绑上自己的伞,便离开了。
\item 如是显示了过去后,为将此本生与现在相续,对众比丘说了法论——你们是否觉得,诸比丘!尔时的商佉婆罗门另有其人?不应作如是见,尔时的商佉婆罗门便是我。我为善界辟支佛的支提庭院除了草,以我此业的等流,他们便扫除树桩荆棘,使八由旬的道路平整洁净。我在那里铺了沙子,以我此业的等流,他们便在八由旬的道路上铺上沙子。我在那里以林木的鲜花作了供养,以我此业的等流,他们便在包含陆地与水面的九由旬的道路上以种种鲜花作了花毯。我在那支提竖起了幢幡并绑了伞,以我此业的等流,直至阿迦腻吒的居处幢幡竖立,层伞叠起。如是,诸比丘!这对我的殊胜的供养非因佛陀的威力而发生,也非因龙、天、梵的威力,而是因少许遍舍的威力而发生。在开示法的终了,便说了此颂:\begin{quoting}若弃于小乐,得见于大乐,\\智者弃小乐,当见于大乐。(法句·杂品第 290 颂)\end{quoting}\end{enumerate}

\begin{center}\vspace{1em}宝经第一\\Ratanasuttaṃ paṭhamaṃ.\end{center}