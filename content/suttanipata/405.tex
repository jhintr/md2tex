\section{最上八颂经}

\begin{center}Paramaṭṭhaka Sutta\end{center}\vspace{1em}

\begin{enumerate}\item 缘起为何?据说,当世尊住舍卫国时,种种外道集会后,显明各自的见,「此是最上、此是最上」,引发纷争后,他们便告知国王。国王便教人召集了许多生盲者,命令道:「给他们看象!」王臣们便教人召集了盲人,让象在前方卧倒,说:「你们看!」他们便抚摸了象的各个肢体。
\item 随后,国王问「我说,象是什么样子」,那摸了象鼻的便说:「大王!好比犁梁。」那些摸了象牙等的便指责这人道「先生!莫在国王前说谎」,而说「大王!好比墙上的挂钩」等等。国王全都听完后,便遣散了诸外道:「你们的教义便是如此。」
\item 某位乞食者知晓了经过,就告知世尊。世尊即于此事因缘,告诸比丘,说「诸比丘!好比众生盲者不了知象,抚摸了各个肢体而起争论,如是众外道不了知以解脱为边际的法,执取各自的见而起争论」,为开示法,说了此经。\end{enumerate}

\subsection\*{\textbf{803} {\footnotesize 〔PTS 796〕}}

\textbf{以「最上」住于诸见,人将其置于世间之上,\\}
\textbf{他说其余一切较此「卑下」,所以不能超越争论。}

“Paraman” ti diṭṭhīsu paribbasāno, yad uttari kurute jantu loke;\\
“hīnā” ti aññe tato sabbam āha, tasmā vivādāni avītivatto. %\hfill\textcolor{gray}{\footnotesize 1}

\begin{enumerate}\item 这里,\textbf{以「最上」住于诸见},即执持「此是最上」已,住于各自的见。\textbf{置于之上},即将自己的大师等置于最胜。\textbf{他说其余一切较此「卑下」},即除了自己的大师等,其余的一切相较于此,他说「此是卑下」。\textbf{所以不能超越争论},职此之由,他便不能超越见的纷争。\end{enumerate}

\subsection\*{\textbf{804} {\footnotesize 〔PTS 797〕}}

\textbf{凡他在自身所见、所闻、戒禁或所觉中见到的利益,\\}
\textbf{他即于此执持之,而视其余的一切为下劣。}

Yad attanī passati ānisaṃsaṃ, diṭṭhe sute sīlavate mute vā;\\
tad eva so tattha samuggahāya, nihīnato passati sabbam aññaṃ. %\hfill\textcolor{gray}{\footnotesize 2}

\begin{enumerate}\item 第二颂之义为:且如是未超越者,在\textbf{所见、所闻、戒禁或所觉}等这些依处中,\textbf{见到}被称为生起的见的先前所说品类的\textbf{自身}的\textbf{利益},\textbf{他即于此}以自己的见执著为「此是最胜」,\textbf{而视其余的一切},即别的大师等,\textbf{为下劣}。\end{enumerate}

\subsection\*{\textbf{805} {\footnotesize 〔PTS 798〕}}

\textbf{凡依止者视其余为卑下,善人们说这是系缚,\\}
\textbf{所以比丘不应依止所见、所闻、所觉或戒禁。}

Taṃ vāpi ganthaṃ kusalā vadanti, yaṃ nissito passati hīnam aññaṃ;\\
tasmā hi diṭṭhaṃ va sutaṃ mutaṃ vā, sīlabbataṃ bhikkhu na nissayeyya. %\hfill\textcolor{gray}{\footnotesize 3}

\begin{enumerate}\item 第三颂之义为:且对如是见者,\textbf{凡依止}自己的大师等\textbf{者,视其余}别的大师等\textbf{为卑下},而\textbf{善人们说}此知见只\textbf{是系缚},即是说束缚。正因为此,\textbf{所以比丘不应依止所见、所闻、所觉或戒禁},即是说不应执著。\end{enumerate}

\subsection\*{\textbf{806} {\footnotesize 〔PTS 799〕}}

\textbf{不应以智,或者以戒禁,在世间构建见,\\}
\textbf{不应表示自己相等,也不应认为卑下或殊胜。}

Diṭṭhim pi lokasmiṃ na kappayeyya, ñāṇena vā sīlavatena vā pi;\\
“samo” ti attānam anūpaneyya, “hīno” na maññetha “visesi” vā pi. %\hfill\textcolor{gray}{\footnotesize 4}

\begin{enumerate}\item 第四颂之义为:不仅不应依止所见、所闻等,而且\textbf{不应在世间构建}、生成更多未生的\textbf{见}。有哪些呢?\textbf{以智,或者以戒禁},即不应构建以等至之智等或以戒禁构建的见。且不仅不应构建见,而且\textbf{不应}出于慢,以生等依处\textbf{表示自己相等,也不应认为卑下或殊胜}。\end{enumerate}

\subsection\*{\textbf{807} {\footnotesize 〔PTS 800〕}}

\textbf{舍弃了执取,无所取著,他也不依止智,\\}
\textbf{他在异议中不追随群体,他也不认可任何见。}

Attaṃ pahāya anupādiyāno, ñāṇe pi so nissayaṃ no karoti;\\
sa ve viyattesu na vaggasārī, diṭṭhim pi so na pacceti kiñci. %\hfill\textcolor{gray}{\footnotesize 5}

\begin{enumerate}\item 第五颂之义为:既如是不构建见且不起慢,\textbf{舍弃了执取,无所取著},即于此舍弃了先前已执取者,更不执取其它,于所说品类的\textbf{智},\textbf{不}作两种\textbf{依止}。且当不作时,\textbf{他在异议中},在为种种见分裂的有情中,\textbf{不追随群体},以欲等而成非趣法\footnote{以欲等而成非趣法:据菩提比丘注 1830,见\textbf{增支部}第 4:17 第一非趣经:「由欲、嗔、怖畏、痴而违犯法。」},于六十二见\textbf{也不认可任何见},即是说不退回。\end{enumerate}

\subsection\*{\textbf{808} {\footnotesize 〔PTS 801〕}}

\textbf{若其于此两端、有与无有、此世或他世已无誓愿,\\}
\textbf{于诸法抉择已,他已无任何摄取的执著。}

Yassūbhayante paṇidhīdha natthi, bhavābhavāya idha vā huraṃ vā;\\
nivesanā tassa na santi keci, dhammesu niccheyya samuggahītaṃ. %\hfill\textcolor{gray}{\footnotesize 6}

\begin{enumerate}\item 现在,为赞扬上颂所说的漏尽者,说了以下三颂。这里,\textbf{两端},即先前所说的触等类\footnote{触等类:即触与触集等的两部分,见\textbf{洞窟八颂经}第 785 颂注。}。\textbf{誓愿},即渴爱。\textbf{有与无有},即再再的有。\textbf{此世或他世},即自己自体等类的此世,或他人自体等类的别处。\end{enumerate}

\subsection\*{\textbf{809} {\footnotesize 〔PTS 802〕}}

\textbf{于此,他于所见、所闻或所觉已无些许遍计的想,\\}
\textbf{这无取于见的婆罗门,在此世间,谁与同类?}

Tassīdha diṭṭhe va sute mute vā, pakappitā natthi aṇū pi saññā;\\
taṃ brāhmaṇaṃ diṭṭhim anādiyānaṃ, kenīdha lokasmiṃ vikappayeyya. %\hfill\textcolor{gray}{\footnotesize 7}

\begin{enumerate}\item \textbf{于所见},即于所见之清净。所闻等仿此。\textbf{想},即由想等起的见。\end{enumerate}

\subsection\*{\textbf{810} {\footnotesize 〔PTS 803〕}}

\textbf{他们不设想,不预设,他们也不接受诸法,\\}
\textbf{婆罗门不被戒禁引领,已到彼岸,不再返回而如如。}

Na kappayanti na purekkharonti, dhammā pi tesaṃ na paṭicchitāse;\\
na brāhmaṇo sīlavatena neyyo, pāraṅgato na pacceti tādī ti. %\hfill\textcolor{gray}{\footnotesize 8}

\begin{enumerate}\item \textbf{他们也不接受诸法},即如是,他们以「唯此是谛,余皆虚妄」不接受六十二见之法。\textbf{已到彼岸,不再返回而如如},即已到涅槃彼岸,不再去往以彼彼道舍弃的烦恼,且以五种行相而如如\footnote{以五种行相而如如:如\textbf{大义释}云:于可意、不可意如如,以舍弃而如如,以已度而如如,以解脱而如如,彼名称而如如。}。其余自明。\end{enumerate}

\begin{center}\vspace{1em}最上八颂经第五\\Paramaṭṭhakasuttaṃ pañcamaṃ.\end{center}