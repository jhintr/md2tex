\section{最上八颂经}

\begin{center}Paramaṭṭhaka Sutta\end{center}\vspace{1em}

\begin{enumerate}\item 据说,当世尊住在舍卫国时,种种外道集会后,阐明各自的见,「此是最上,此是最上」,引发争论后,告知了国王。国王命人召集了许多生盲者,命令道「给他们看象」。王臣们便命人召集了盲人,让象在前方卧倒,说「你们看吧」。他们便抚摸了象的各个肢体。随后,国王问「象是什么样子」,那摸了象鼻的便说「大王!好比犁梁」,那些摸了象牙等的便辱骂这人道「你不要在国王跟前说谎」,而说「大王!好比墙上的挂钩」等。国王全都听完后,便遣散了诸外道「你们的教法便是如此」。有乞食者知晓了经过,就告知世尊。世尊即于此事因缘,对诸比丘说「诸比丘!好比众生盲者不了知象,抚摸各个肢体而起争论,如是众外道不了知以解脱为边际的法,执持各自的见而起争论」,说完后,为开示法而说此经。\end{enumerate}

\subsection\*{\textbf{803} {\footnotesize 〔PTS 796〕}}

\textbf{住于诸见,(以为)最上,人将其置于世上最高,\\}
\textbf{他说其余的一切为「卑下」,因而不能超越争论。}

“Paraman” ti diṭṭhīsu paribbasāno, yad-uttari kurute jantu loke;\\
“Hīnā” ti aññe tato sabbam āha, tasmā vivādāni avītivatto. %\hfill\textcolor{gray}{\footnotesize 1}

\begin{enumerate}\item \textbf{住于诸见,(以为)最上},即执持「此是最上」已而住于自身的见。\textbf{置于最高},即将自己的老师等置于最胜。\textbf{他说其余的一切为「卑下」},即除了自己的老师等,他说其余的一切「此是卑下」。\textbf{因而不能超越争论},职此之由,他即不能超越见的争论。\end{enumerate}

\subsection\*{\textbf{804} {\footnotesize 〔PTS 797〕}}

\textbf{凡他在自己所见、所闻、戒禁或所觉中见到的功德,\\}
\textbf{他即于此执持之,而视其余的一切为低下。}

Yad-attanī passati ānisaṃsaṃ, diṭṭhe sute sīlavate mute vā;\\
tad-eva so tattha samuggahāya, nihīnato passati sabbam aññaṃ. %\hfill\textcolor{gray}{\footnotesize 2}

\subsection\*{\textbf{805} {\footnotesize 〔PTS 798〕}}

\textbf{有所依者视其余为卑下,善人们说这是系缚,\\}
\textbf{所以比丘不应依于所见、所闻、所觉或戒禁。}

Taṃ vāpi ganthaṃ kusalā vadanti, yaṃ nissito passati hīnam aññaṃ;\\
tasmā hi diṭṭhaṃ va sutaṃ mutaṃ vā, sīlabbataṃ bhikkhu na nissayeyya. %\hfill\textcolor{gray}{\footnotesize 3}

\subsection\*{\textbf{806} {\footnotesize 〔PTS 799〕}}

\textbf{不应以智,或者以戒禁,在世间构建见,\\}
\textbf{不应表示自己相等,也不应认为卑下或殊胜。}

Diṭṭhim pi lokasmiṃ na kappayeyya, ñāṇena vā sīlavatena vā pi;\\
“samo” ti attānam anūpaneyya, “hīno” na maññetha “visesi” vā pi. %\hfill\textcolor{gray}{\footnotesize 4}

\begin{enumerate}\item 不仅不应依于所见所闻等,而且\textbf{不应在世间构建}、制造更多未生起的\textbf{见}。有哪些呢?\textbf{以智,或者以戒禁},不应构建以等至之智或以戒禁构建的见。不仅不应构建见,而且\textbf{不应}出于慢,以出生等的事由\textbf{表示自己相等,也不应认为卑下或殊胜}。\end{enumerate}

\subsection\*{\textbf{807} {\footnotesize 〔PTS 800〕}}

\textbf{舍弃了所得,无所执取,他也不依于智,\\}
\textbf{他在异议中不追随群体,他也不落回任何见。}

Attaṃ pahāya anupādiyāno, ñāṇe pi so nissayaṃ no karoti;\\
sa ve viyattesu na vaggasārī, diṭṭhim pi so na pacceti kiñci. %\hfill\textcolor{gray}{\footnotesize 5}

\begin{enumerate}\item 既如是不构建见且不起慢,\textbf{舍弃了所得,无所执取},于此,舍弃了先前已执取者,更不执取其它,\textbf{不依于}两种已说的\textbf{智}。当不依时,\textbf{他在异议中},在为种种见所分裂的有情众中\textbf{不追随群体},不由欲等而成非趣法,\textbf{不落回任何见},即是说不回返的意思。\end{enumerate}

\begin{itemize}\item 案,菩提比丘注,\textbf{不由欲等而成非趣法},见增支部四集·行品·第一非趣经,即由于欲、嗔、痴、怖畏而违犯法 \textit{chandā dosā bhayā mohā yo dhammaṃ ativattati}。\end{itemize}

\subsection\*{\textbf{808} {\footnotesize 〔PTS 801〕}}

\textbf{于此,他已无对两端的誓愿,对有与无有,此世或他世,\\}
\textbf{于诸法抉择已,他已无任何执取的住著。}

Yassūbhayante paṇidhīdha natthi, bhavābhavāya idha vā huraṃ vā;\\
nivesanā tassa na santi keci, dhammesu niccheyya samuggahītaṃ. %\hfill\textcolor{gray}{\footnotesize 6}

\begin{enumerate}\item 现在,为称赞以上颂所说的漏尽者,而说以下三颂。这里,\textbf{两端},即先前所说的触等类。\textbf{誓愿},即渴爱。\textbf{有与无有},即再再的有。\end{enumerate}

\begin{itemize}\item 案,\textbf{两端},即触与触集等的两边,见洞窟八颂经第 785 颂注。\textbf{有与无有},见蛇经第 6 颂注。原文中的 \textbf{samuggahītaṃ} 从 PTS 本作 \textit{samuggahītā}。\end{itemize}

\subsection\*{\textbf{809} {\footnotesize 〔PTS 802〕}}

\textbf{于此,他于所见、所闻或所觉已无些许遍计的想,\\}
\textbf{这无取于见的婆罗门,于此世间,谁与同类?}

Tassīdha diṭṭhe va sute mute vā, pakappitā natthi aṇū pi saññā;\\
taṃ brāhmaṇaṃ diṭṭhim anādiyānaṃ, kenīdha lokasmiṃ vikappayeyya. %\hfill\textcolor{gray}{\footnotesize 7}

\begin{enumerate}\item \textbf{于所见},即于所见之清净,所闻等仿此。\textbf{想},即由想等起的见。\end{enumerate}

\subsection\*{\textbf{810} {\footnotesize 〔PTS 803〕}}

\textbf{他们不造作,不偏好,他们也不接受诸法,\\}
\textbf{婆罗门不被戒禁引领,已到彼岸,如如者不再回返。}

Na kappayanti na purekkharonti, dhammā pi tesaṃ na paṭicchitāse;\\
na brāhmaṇo sīlavatena neyyo, pāraṅgato na pacceti tādī ti. %\hfill\textcolor{gray}{\footnotesize 8}

\begin{enumerate}\item \textbf{他们也不接受诸法},即「唯此是谛,余皆虚妄」,如是,他们不接受六十二见之法。\textbf{已到彼岸,如如者不再回返},即已到涅槃的彼岸,如如者不再去向已由种种道舍弃的烦恼,且以五种行相而「如如」。\end{enumerate}

\begin{itemize}\item 案,\textbf{以五种行相而「如如」},见大义释:于可意、不可意如如,以舍弃故如如,以已度故如如,以解脱故如如,彼义释如如 \textit{iṭṭhāniṭṭhe tādī, cattāvī ti tādī, tiṇṇāvī ti tādī, muttāvī ti tādī, taṃniddesā tādī}。\end{itemize}

\begin{center}\vspace{1em}最上八颂经第五\\Paramaṭṭhakasuttaṃ pañcamaṃ.\end{center}