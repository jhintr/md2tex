\section{旷野经}

\begin{center}Āḷavaka Sutta\end{center}\vspace{1em}

\textbf{如是我闻\footnote{此经旧译见杂阿含经第 1326 经、别译杂阿含经第 325 经。「旷野」之译名从别译杂阿含经。}。一时世尊住在旷野中旷野夜叉的居处。}

Evaṃ me sutaṃ— ekaṃ samayaṃ Bhagavā Āḷaviyaṃ viharati Āḷavakassa yakkhassa bhavane.

\begin{enumerate}\item 缘起为何?其缘起当唯以释义的方法显示。且在释义中,「\textbf{如是我闻一时世尊}」之义业已叙述\footnote{关于「如是我闻一时世尊」的义注,见\textbf{耕田婆罗豆婆遮经}的开始部分。}。而在「\textbf{住在旷野中旷野夜叉的居处}」之中,什么是旷野?且为什么世尊住在此夜叉的居处?
\item 当答:\textbf{旷野}既是国家,也是城市,或于此兼指两者。因为住在旷野城的附近可说为「住在旷野中」,而此城附近不远的一牛呼之处即彼居处,则住在旷野国也可说为「住在旷野中」,彼居处即在旷野国中。
\item 因为旷野国王每七天便会为了驱逐盗贼、阻止敌王及进行锻炼而抛开种种舞伎的享乐前往狩猎,一天,便与军队订下规约:野兽从哪边逃脱,便是哪边的责任。然而野兽竟从他的一边逃脱,捷足的国王便提了弓,徒步追了那野兽三由旬。而羚羊只有三由旬的气力,于是当那野兽势速已尽,进到水边站着,他便将它杀了,劈作两半,意不在肉,而是为了免于「他抓不到野兽」的指责,用棍棒扛了回来,在离城不远处,看到枝叶茂密的大榕树,为解疲乏,走近这树下。
\item 然而,旷野夜叉从大王\footnote{此处的「大王」当指义注结尾处的毗沙门大王。}跟前得到恩赐而住于此树,可以吞吃在中午时分进入此树阴影所覆之处的生类。看到国王后,他便走近欲吃。于是,国王便与他订下规约:放了我,我会每天给你送人和笾豆来。夜叉说「你因王室享乐即放逸忘失,而我却不能吞吃未走近居处者或未予承诺者,那我岂不是连你也失去了」,便不放。国王说「我哪天不送,你就在那天抓了我吃」,拿自己作了承诺,以此得脱,便朝城里走来。
\item 军队在路上结营等待,看见国王后,说着「大王!为何只是畏惧不名誉便如此劳苦」,便前去迎接。国王并未告知经过,去到城内后,用完早餐,才传唤城守,告知此事。城守说:「陛下!可曾限定时日?」国王说:「我说,未曾。」「糟糕,陛下!因为非人只可得有限之物,当无限定时,将成国家之累。也罢!陛下!虽然如此,不用操心,请您享用王室之乐!我会对此做我该做的。」
\item 他按时起身,去到监狱后,召集了凡应被处死者,对他们说:「想要活命的出列!」把第一个出列的带回家,教人沐浴授食后,派遣道「把这笾豆送给夜叉」。他刚进入树下,夜叉就变现出恐怖的身躯,像吃萝卜一样吃了他。据说,人类的整个身体,从头发等开始,会因夜叉的威力变成生酥团一般。前来为夜叉送食的人们见此而恐惧,便告知各自的朋友。从此以后,人们以「国王抓了盗贼送给夜叉」而远离行盗。此后的某时,因无有新来的盗贼且原有的盗贼灭尽,监狱为之一空。
\item 于是,城守便告知国王。国王教人把自己的财物丢到街上「也许有人会因贪来取」,但甚至没人用脚碰它。他抓不到盗贼,便告知了众大臣。大臣们说:「我们挨家挨户地派遣老人吧,他本来也快死了。」国王遮止道:「人们将会以『他送走我们的父亲、我们的祖父』发起动乱,你们休想!」他们便说:「那么,陛下!我们派遣婴孩吧,因为对其尚无『吾父吾母』般的爱执。」经国王认可,他们便照做。城内,孩子的母亲们带了孩子与孕妇们纷纷逃亡,到别的国家去抚养孩子。如是便过了十二年。
\item 此后某天,在巡视了整座城也找不到一个孩子后,他们便告知国王:「陛下!城内没有孩子了,除了后宫里您的孩子旷野王子。」国王说:「正如我喜爱孩子,一切世间都是如此,却还不如喜爱自己更多,去吧!连他也送走,守护我的性命!」这时,旷野王子的母亲让人沐浴、装饰了孩子后,垫上黄麻布,正躺着在怀里哄睡。王臣们以国王的命令来至彼处,当她与其一万六千位乳母私语时,夺了他便离开:「明天他就是夜叉的食物。」
\item 那天,世尊在黎明时分起身后,在祇林大寺的香房里入大悲等至,当再以佛眼观察世间时,见到旷野王子生起阿那含果的近依及夜叉生起须陀洹果的近依,且在开示终了,八万四千生类获得法眼。所以,当夜色褪去,作了饭前义务,尚未究竟饭后义务,当黑分布萨日的太阳落下时,便独一无侣,持了衣钵,唯以步行,从舍卫国行了三十由旬,到了这夜叉的居处,因此说\textbf{在旷野夜叉的居处}。
\item 那么,世尊是住在旷野居处所在的榕树下,抑或唯在居处?当答:唯在居处。因为正好比众夜叉视之为自己的居处,世尊亦然。他到此后,便立于居处的门前。此时,旷野去到雪山与夜叉集会。随后,旷野的门卫,名叫驴的夜叉往世尊处走去,顶礼后说:「尊者!世尊为何非时前来?」「唯!驴!我来了,要是不麻烦你的话,我就在旷野的居处住一晚。」「尊者!我不麻烦,只是这夜叉酷虐、粗恶,甚至对父母都不行礼拜等,世尊不会喜欢于此居住的!」「我知道他的酷虐,驴!对我而言没有障碍,要是不麻烦你的话,我就住一晚。」
\item 第二次,驴夜叉对世尊说:「尊者!旷野就像被火烤热的锅,父母也罢,沙门婆罗门也罢,法也罢,一概不认,他扰乱到此者的心识,撕碎其心脏,捉住脚抛到别的大海或轮围去。」第二次,世尊说:「我知道,驴!要是不麻烦你的话,我就住一晚。」
\item 第三次,驴夜叉对世尊说:「尊者!旷野就像被火烤热的锅,父母也罢,沙门婆罗门也罢,法也罢,一概不认,他扰乱到此者的心识,撕碎其心脏,捉住脚抛到别的大海或轮围去。」第三次,世尊说:「我知道,驴!要是不麻烦你的话,我就住一晚。」「尊者!我不麻烦,只是不告知他就同意,这夜叉会夺了我性命,尊者!让我去告知他!」「随你乐意,驴!去告知吧!」「那么,尊者!唯你知晓!」顶礼了世尊后,便朝着雪山出发了。
\item 居处的门自动为世尊敞开来。世尊进入内室后,在旷野于被视作吉祥之日等时坐而享受荣光的天宝台上坐下,放出金光。夜叉的女人们见到后便赶来,顶礼了世尊,围绕而坐。世尊以「你们先前布施、持戒、供养而得此成就,现在也应如是行止,莫要彼此为嫉悭所胜而住」等方法为她们说了种种法。她们听了世尊悦耳的声音,鼓掌千遍,仍围绕世尊而坐。而驴去到雪山后,便告知旷野:「大人!您应知晓,世尊坐在您的宫中!」他便对驴作出手势:「安静!去到后我会做该做的。」据说,他因男人的慢而感到羞耻,所以便遮止道:「莫让会众听见!」
\item 此时,七岳与雪山在祇园顶礼了世尊后,想「我们去参加夜叉集会」,与随从以种种车乘经空中前往。对夜叉来说,空中并非随处是路,避开处于空中的宫殿,唯依道路之处方是道路。而旷野的宫殿处于地上,严加守护,围有城墙,城门、塔楼、门阙都善加整饬,上方覆以铜网,如曼珠沙般高三由旬,其上便是道路。他们来到此处便无法通过。因为诸佛所坐之处的上方,直至有顶,无人能行。他们转向于「这是为何」,看到世尊后,像在空中被抛出的土块般降落顶礼,闻法后作右绕,说「世尊!我们去参加夜叉集会」,赞美着三依处,便去参加夜叉集会。
\item 旷野看到他们后,撤步让位道:「请在此坐!」他们便告诉旷野:「旷野!你有收获,世尊住在你的居处,去吧!朋友!承事世尊!」如是,世尊唯住在居处,而非旷野居处所在的榕树下。因此说「一时世尊住在旷野中旷野夜叉的居处」。\end{enumerate}

\textbf{于是,旷野夜叉往世尊处走去,走到后,对世尊说:「出来!沙门!」「善哉!朋友!」世尊便出来。「进去!沙门!」「善哉!朋友!」世尊便进去。第二次……第三次,旷野夜叉对世尊说:「出来!沙门!」「善哉!朋友!」世尊便出来。「进去!沙门!」「善哉!朋友!」世尊便进去。}

Atha kho Āḷavako yakkho yena Bhagavā ten’upasaṅkami, upasaṅkamitvā Bhagavantaṃ etad avoca: “nikkhama, samaṇā” ti. “Sādh’āvuso” ti Bhagavā nikkhami. “Pavisa, samaṇā” ti. “Sādh’āvuso” ti Bhagavā pāvisi. Dutiyam pi kho…pe… tatiyam pi kho Āḷavako yakkho Bhagavantaṃ etad avoca: “nikkhama, samaṇā” ti. “Sādh’āvuso” ti Bhagavā nikkhami. “Pavisa, samaṇā” ti. “Sādh’āvuso” ti Bhagavā pāvisi.

\begin{enumerate}\item \textbf{于是,旷野……出来!沙门},那他为什么要这样说?当答:为欲激怒。这里,当如是从头开始了知其情形。因为对不信者难以谈论信,如对恶戒者等难以谈论戒等一般,所以,他从这些夜叉跟前一听到对世尊的赞叹后,心因内在的忿怒,如同火中投下的盐块般噼啪噼啪,便说:「那个进到我居处的世尊是谁?」他们便说:「朋友!你不知道我们的大师世尊?他在兜率天的居处观察了五大观察……」,一直讲到转法轮,当说了在结生等时的三十二种征兆后,便责备道:「朋友!你竟没看到这些奇异!」他虽曾见,却因忿怒而说「我没看到」。
\item 「朋友旷野!无论你是否想看,看或不看对你有何义利?你能对我们的大师做什么呢?相较于他,你看起来就好比晃动着驼峰的大公牛面前当天出生的牛犊,好比作出三种破坏的狂象面前的幼象,好比肩上垂饰以闪亮鬃毛的兽王面前的老迈豺狼,好比翼展一百五十由旬的金翅鸟王面前的折翅雏鸦,去!你该做什么就做吧!」
\item 如是说已,旷野一怒而起,左脚踩在雄黄之原「现在要你们看看是你们的大师还是我有大威力」,右脚便踏在六十由旬外的冈仁波齐峰顶,如被铁锤锤炼纯净的铁丸般迸出火花。他站在那里,大喝道「我是旷野」,声音响彻整个阎浮提。
\item 据说,在整个阎浮提有\textbf{四种声音}能被听到。一即夜叉大将富楼那\footnote{夜叉大将富楼那 \textit{Puṇṇaka},见\textbf{本生}义注。}在与俱卢国王胜财赌博时,在赢后拍手宣告「我赢了」,一即诸天因陀帝释在迦叶世尊的教法衰落时,将毗舍羯磨天子变成了狗,让他宣告「我要吃掉恶比丘、恶比丘尼、优婆塞、优婆夷等一切非法论者」,一即固沙本生\footnote{固沙本生 \textit{Kusajātaka},见\textbf{本生}义注。}中,当城因波婆伐蒂被七国国王围困时,大人将波婆伐蒂与自己扶上象背,冲出城去,宣告「我是狮音,固沙大王」,一即旷野站在冈仁波齐峰顶「我是旷野」。此时,就与他站在整个阎浮提各家各户的门口发出宣告一样,甚至连宽广三千由旬的雪山也因夜叉的威力而震动。
\item 他便兴起了\textbf{飓风}:「我将以此驱逐沙门」。这些东方等类的风兴起后,破碎了半由旬、一由旬、二由旬、三由旬范围的山尖,拔起了林树灌木等,朝着旷野城内进发,碾碎朽败的象厩等,令瓦片在空中旋转。世尊便决意「莫让它伤了谁」。这些风遇到十力后,竟连动摇衣角也不能够。
\item 随后,他兴起了\textbf{暴雨}:「我要用水淹死沙门」。以其威力,百层千层等类的雨云在上方集起后便下起了雨,因雨流的冲激,大地裂开,大暴流在林树等的上方落下,却于十力之衣不能沾染露珠之量。
\item 随后,他兴起了\textbf{石雨},大块大块的山尖冒着烟冒着火从空中而来,遇到了十力,便变成了天花簇。
\item 随后,他兴起了\textbf{击打之雨},单面刃与双面刃的刀剑矢簇等冒着烟冒着火从空中而来,遇到了十力,便成了天花。
\item 随后,他兴起了\textbf{炭雨},紧叔迦色的炭块从空中而来,在十力的脚下成了天花,敷散开来。
\item 随后,他兴起了\textbf{灰雨},酷热的灰烬从空中而来,在十力的脚下成了旃檀末而落下。
\item 随后,他兴起了\textbf{沙雨},细密的沙尘冒着烟冒着火从空中而来,在十力的脚下成了天花而落下。
\item 随后,他兴起了\textbf{泥雨},这泥雨冒着烟冒着火从空中而来,在十力的脚下成了天香而落下。
\item 随后,他兴起了\textbf{黑暗}:「我将恐吓沙门而驱逐之」。这如具足四支黑暗般的黑暗遇到了十力,如被日光驱散般消尽。
\item 如是,夜叉无法以风、雨、石、击打、炭、灰、沙、泥、黑暗等九种雨驱逐世尊,便亲自带领作种种击打之势、混杂以各类样貌之群生的四支大军前来。这些群生制造了各式的混乱,喊着「抓呀杀呀」,好似来到世尊上方,但却如苍蝇之于洁净的铜丸般,无法叮触世尊。即便如是,好比魔罗来到菩提座时撤退一般,他们在未撤退前制造了半夜的骚乱。如是,旷野以半夜展示的各式恐怖也无法撼动世尊,便想:「我何不释出无人能胜的布武器?」
\item 据说,\textbf{四种武器}在世间最胜:帝释的金刚武器、毗沙门的棒武器、阎摩的目武器以及旷野的布武器。因为帝释若是发怒,便在须弥山顶捶击金刚武器,能在贯穿十六万八千由旬的须弥山后从下而行。当毗沙门尚为凡夫时,释出的棒在数千夜叉的头上抡下后,又回到臂展之内而立。而为忿怒的阎摩的目武器仅仅一瞥,数千鸠槃荼如热锅上芝麻般颤栗消亡。旷野发怒,若在空中释出布武器,天便十二年不得下雨,若在地面释出,一切草木等凋萎,且十二年间不得生长,若在海上释出,一切水如热锅上的水滴般干涸,若在山上释出,即便如须弥山,也会裂成块块。
\item 他解开上衣,便取出有如是大威力的布武器。一万世界中的绝大多数天人便迅速集合:「今天世尊将要调御旷野,我们去那里听法!」想看打斗的天人也集合起来。如是,整个空中便充满了天人。
\item 于是,旷野在世尊周围的上方巡行后,释出了布武器。它在空中放出如雷鸣般可怕的声响,冒着烟冒着火落向世尊,为粉碎夜叉的慢,竟成了抹脚布跌在脚下。旷野见后,如公牛断了角,如毒蛇拔了牙,失去了光彩,失去了㤭慢,偃伏了慢的旗帜,便想:「连布武器也不能征服沙门,到底是什么原因?这原因就是,沙门与慈住相应,噫!让我激怒他,让他离于慈!」在此情形下,即说「\textbf{于是,旷野往世尊处……出来!沙门}」。
\item 这里,其意趣为:「为什么你未经我的许可便进入我的居处,像家主一般坐在妻妾中间?受用不与者并与女人交际,这对沙门难道合适吗?所以,如果你住于沙门法,那就出来!沙门!」而有些人说,在说了这些及其它粗恶之语后,他才说了这句。
\item 于是,因为世尊了知到「硬傲者不能以对其硬傲的方式来调伏,因为当他被待以硬傲的方式时,好比在恶犬的鼻子上挤上胆汁,会变得更加凶恶,如是,他便更加硬傲,然而,他能以柔和来调伏」,便以「善哉!朋友」之可喜的话语领受了他的话而出来。因此说「\textbf{善哉!朋友!}」\textbf{世尊便出来}。
\item 随后,旷野想「这沙门真是易语,仅一语便出来,如是毫无原因便易于请出的沙门,我却整晚对其进行打斗」,心即柔和,又想:「但现在还无法得知是由于易语便出来,抑或由于忿怒,噫!我来考察他。」随后,便说:「\textbf{进去!沙门!}」
\item 于是,世尊为了以「易语」令业已柔和的心得以确定,又再次说着可喜的话语「\textbf{善哉!朋友!}」\textbf{便进去}。旷野则再再考察其易语,又第二次、第三次说「\textbf{出来、进去}」。世尊便仍照做。
\item 如果不做,则出于天性,硬傲夜叉之心会更加硬傲,难成论法之器。所以,正好比母亲对于哭闹的孩子,随其所欲地或予或行,加以抚慰,同样,世尊为抚慰以烦恼哭闹的夜叉,随其所说地照做。又好比乳母给未吮奶的小儿以任何东西,爱抚令吮,同样,世尊为令夜叉吮出世间法之乳,以其所希求的话语爱抚而照做。又好比想在葫芦中装满四蜜\footnote{四蜜:即酥油、蜂蜜、糖、芝麻油的混合物。}的男子,清洗其内部,如是,世尊想在夜叉心中装满出世间之四蜜,为清洗其内部的忿怒尘垢,乃至三次出入。\end{enumerate}

\textbf{第四次,旷野夜叉对世尊说:「出来!沙门!」「朋友!我不会再出来,你该做什么就做吧!」}

Catuttham pi kho Āḷavako yakkho Bhagavantaṃ etad avoca: “nikkhama, samaṇā” ti. “Na khvāhaṃ taṃ, āvuso, nikkhamissāmi, yaṃ te karaṇīyaṃ, taṃ karohī” ti.

\begin{enumerate}\item 于是,夜叉想「这沙门易语,叫出来就出来,叫进去就进去,我何不整晚这般折磨这沙门,捉住脚抛到恒河对岸去」,生起恶心,第四番说「\textbf{出来!沙门!}」。世尊了知已,便说「我不会再……」,即了知到「当如是说时,他便寻求更上的应作,想该问的问题,则会有论法的入处」,便说「我不会再……」。
\item 这里,\textbf{不会}即遮止,\textbf{再}即强调,\textbf{我}即指称自己。因此,此中所示之义即「因为你这样想,所以,朋友!我就不会出来,你该做什么就做吧」。\end{enumerate}

\textbf{「沙门!我将问你问题,如果你不能向我解答,我就扰乱你的心识,撕碎你的心脏,捉住脚抛到恒河对岸去。」「朋友!我实不见在这俱有天、魔、梵、沙门婆罗门、天人的人世间,有人能扰乱我的心识、撕碎心脏、捉住脚抛到恒河对岸去的,但是,朋友!问你所愿吧!」于是,旷野夜叉以偈颂对世尊说:}

“Pañhaṃ taṃ, samaṇa, pucchissāmi, sace me na byākarissasi, cittaṃ vā te khipissāmi, hadayaṃ vā te phālessāmi, pādesu vā gahetvā pāragaṅgāya khipissāmī” ti. “Na khvāhaṃ taṃ, āvuso, passāmi sadevake loke samārake sabrahmake sassamaṇabrāhmaṇiyā pajāya sadevamanussāya yo me cittaṃ vā khipeyya hadayaṃ vā phāleyya pādesu vā gahetvā pāragaṅgāya khipeyya, api ca tvaṃ, āvuso, puccha yad ākaṅkhasī” ti. Atha kho Āḷavako yakkho Bhagavantaṃ gāthāya ajjhabhāsi:

\begin{enumerate}\item 随后,因为先前,具神变而来的苦行游行者们在从空中经过自己的宫殿时,想「这是黄金的宫殿,还是白银或摩尼的宫殿?噫!我们去看看」,旷野便问他们问题,以扰乱心识等恼害不能解答者。如何?因为非人会以显现恐怖的相貌或以压迫心依处等两种方式扰乱心识。而他因为了知到「具神变者不会因显现恐怖的相貌而惧怕」,以自身神变的威力化出微细的身体,进入他们的内部后,压迫心依处,随后,心相续便不能安立,以其不能安立,心发疯狂而扰乱。如是,他再剖开这些心识扰乱者的胸膛,捉住他们的脚抛到恒河对岸「这样的人就不要再来我的居处了」。所以,他忆念起这些问题,想「现在,我何不如是去恼害这沙门」,便说「\textbf{沙门!我将问你问题……}」等等。
\item 但他的这些问题从何而来?据说,他的父母曾承事迦叶世尊,并学到了八个问题及其解答,他们曾教旷野在孩童时学习过。他因年深日久,忘记了解答,随后,想「别再让这些问题也亡失了」,便教人把问题以天然的朱砂书写在金册页上,置于宫中。如是,这些佛陀之问唯是佛陀之境域。
\item 世尊听后,因为无人能对诸佛制造既舍利养的障碍或性命的障碍,或抵御一切知智及一寻的身光,所以,为显明这世间不共的佛陀威力而说「\textbf{朋友!我实不见在这俱有天……}」。
\item 这里,由「以俱有天之语摄五欲界天」等方法,已对这些词语略示其语义\footnote{关于「俱有天、魔、梵、沙门婆罗门、天人的人世间」的略注,见\textbf{耕田婆罗豆婆遮经}的结尾部分。},而未以连贯章句的步骤加以详绎。当说:因为以\textbf{俱有天}之语,即便除去高贵部分的一切诸天皆为所摄,有人对囊括于此的群天尚存疑虑「魔罗有大威力,为六欲界之主宰、自在,乐于敌对,厌恶于法,行事残忍,他难道还不能行扰乱心识等事吗」,为除他们的疑虑而说\textbf{俱有魔}。随后,有人尚存「梵天有大威力,能以一指在一千轮围产生光明,以二指……以十指在一万轮围产生光明,且体验无上禅那等至之乐,他难道还不能吗」,为除他们的疑虑而说\textbf{俱有梵}。然后,有人尚存「种种沙门婆罗门为教法之怨敌,具有颂诗等力,他们难道还不能吗」,为除他们的疑虑而说\textbf{俱有沙门婆罗门的人世间}。如是,在显明于高贵处该人之不存在后,现在,以\textbf{俱有天人}之语所指的「共许的天与其余的人」中,仍以高贵的部分显明了在其余的有情世间该人之不存在。如是,当知此中连贯章句的步骤。
\item 如是,世尊制止了其伤害之心,为令于提问生起勇猛而说「\textbf{但是,朋友!问你所愿吧}」。其义为:问吧!如果你愿意,对我来说,没有解答问题的负担,或者说「问你所愿的!我将对你解答一切」,以不共于辟支佛、上首弟子、大弟子们的一切知作出邀请,因为他们会说「问吧!朋友!我们听后将会知晓」,而诸佛则说:\begin{quoting}朋友!问你所愿吧!(相应部)\end{quoting}或说:\begin{quoting}婆娑婆\footnote{婆娑婆 \textit{Vāsava}:为帝释的一个称号。}!问我任何你心中所想的问题吧!(长部)\end{quoting}或说:\begin{quoting}波婆利的与你的,或所有人的一切疑虑,\\都有机会,请问任何心中希望(问的)!(经集第 1037 颂)\end{quoting}以如是等方式向天人们作出一切知的邀请。
\item 而这,即世尊在证得佛地后能如是邀请,尚非希有,他尚在菩萨地的部分知时,便已如是为众仙人祈请:\begin{quoting}㤭陈如!请解说问题!善类的仙人们请求您。\\㤭陈如!这是人类中的法则:责任趋于长者。(本生第 17:60 颂)\end{quoting}在为萨罗绷伽\footnote{萨罗绷伽 \textit{Sarabhaṅga}:意译为断箭,\textbf{清净道论}·说神通品第 18 段有提及。}之时,他便说:\begin{quoting}已得机会的诸君,请问任何心中所愿的问题!\\因为亲身了知了此世后世,我将为你们作答!(本生第 17:61 颂)\end{quoting}且在 Sambhava 本生中,生年七岁的他正在路边玩耍泥巴,被巡行了整个阎浮提三遍也不见诸问题之终结的净喜婆罗门提问,便作了一切知的邀请:\begin{quoting}我当然会对你宣说,如善巧者一般,\\而国王便会了知,无论他去做与否。(本生第 16:172 颂)\end{quoting}
\item 如是,当世尊对旷野作了一切知的邀请后,\textbf{于是,旷野夜叉以偈颂对世尊说}了下颂。\end{enumerate}

\subsection\*{\textbf{183} {\footnotesize 〔PTS 181〕}}

\textbf{「什么是人在此世最上的财富?什么善加习行能带来乐?\\}
\textbf{「味中究竟什么更甜?他们说如何生活才是最上的生活?」}

“Kiṃ sū’dha vittaṃ purisassa seṭṭhaṃ, kiṃ su suciṇṇaṃ sukham āvahāti;\\
kiṃ su have sādutaraṃ rasānaṃ, kathaṃjīviṃ jīvitam āhu seṭṭhaṃ”. %\hfill\textcolor{gray}{\footnotesize 1}

\begin{enumerate}\item 这里,\textbf{什么},即发问之语。\textbf{su},即补足语句的不变词。\textbf{在此世},即在此世间。\textbf{财富}即喜悦,生喜者即财富,为财产的同义语。\textbf{善加习行},即善加作为。\textbf{乐},即身心的悦意。\textbf{带来},即是说导致、给予、回报。\textbf{究竟},即强调之义的不变词。\textbf{更甜},即非常甜,文本也作「更善」。\textbf{味},即被称作味的法。\textbf{如何},即以何方式。文本也作「kathaṃjīviṃ jīvatam」,其义为「对于生活者,如何生活」。其余于此自明。
\item 如是,他以此颂问了这四个问题:「什么是人在此世最上的财富?什么善加习行能带来乐?味中究竟什么更甜?他们说如何生活才是最上的生活?」\end{enumerate}

\subsection\*{\textbf{184} {\footnotesize 〔PTS 182〕}}

\textbf{「信是人在此世最上的财富,法的善加习行能带来乐,\\}
\textbf{「味中真实更甜,他们说有慧的生活才是最上的生活。」}

“Saddh’īdha vittaṃ purisassa seṭṭhaṃ, dhammo suciṇṇo sukham āvahāti;\\
saccaṃ have sādutaraṃ rasānaṃ, paññājīviṃ jīvitam āhu seṭṭhaṃ”. %\hfill\textcolor{gray}{\footnotesize 2}

\begin{enumerate}\item 于是,世尊仍以迦叶十力解答的方式来作答,说了此颂。
\item 这里,好比金钱等财富带来享受之乐,抵挡饥渴之苦,止息穷乏,是获得珠玉等宝之因,且带来世间的满足,如是,世出世间的信也各自带来世出世间的异熟乐,以持有信而抵挡行道者的生老等苦,止息功德的穷乏,是获得念觉支等宝之因,且由\begin{quoting}正信而具戒,得誉及财者,\\彼至于何处,处处受尊敬。(法句第 303 颂)\end{quoting}之语带来世间的满足而被说为「财富」。且因为此信之财富始终伴随,不与他共,为一切成就之因,亦是世间金钱等财富之源,因为唯有信者作了布施等福德可得财富,而对无信者,财富唯生非利,所以说为\textbf{最上}。男\textbf{人}是就高贵的部分来说,所以不仅是对男人,当知对女人等,信的财富也是最上的。
\item \textbf{法},即十善业道之法,或布施、持戒、修习之法。\textbf{善加习行},即善加作为、善加行止。\textbf{带来乐},即带来如商人之子输那与护国等的人之乐,帝释等的天之乐,以及大莲花等的涅槃之乐。
\item \textbf{真实},此「真实」一词被用于多种意义。如在\begin{quoting}谛语不嗔恚。(法句第 224 颂)\end{quoting}等中,为言语真实。在\begin{quoting}且沙门婆罗门住立于真实。(本生第 21:433 颂)\end{quoting}等中,为戒离之真实。在\begin{quoting}那么,为何宣扬各种真实?论说者们都自称善巧。(经集第 892 颂)\end{quoting}等中,为见之真实。在\begin{quoting}诸比丘!这是四种婆罗门之真实。(增支部第 4:185 经)\end{quoting}等中,为婆罗门之真实。在\begin{quoting}因为真实唯一,而非有二。(经集第 891 颂)\end{quoting}等中,为第一义真实。在\begin{quoting}四圣谛中,几种为善?(分别论)\end{quoting}等中,则为圣谛。而在此是指第一义真实之涅槃,或包含戒离之真实的言语真实,以其威力,得运对水等的支配,得度生老死的彼岸。如说:\begin{quoting}以真实之语在水上奔跑,智者们以真实破除毒素,\\天以真实雷鸣雨施,住立于真实者希求寂灭。\\凡是地上存在之味,这些众味之中真实更甜,\\且沙门婆罗门住立于真实,得度生死的彼岸。(本生第 21:433 颂)\end{quoting}
\item \textbf{更甜},即更甜蜜、更胜妙。\textbf{味},即凡是以\begin{quoting}根味、茎味(法集论)\end{quoting}等方法所述的可尝之味,以及凡是以\begin{quoting}诸比丘!我听许一切果味。(律藏·大品)\end{quoting}\begin{quoting}乔达摩君是无味者。\end{quoting}\begin{quoting}婆罗门!凡是色味、声味……(律藏·波罗夷)\end{quoting}\begin{quoting}于味味\footnote{味味 \textit{rasarasa},见众学法第 29,34 条,据\textbf{疑惑度脱},指的是除豆酱外的所有酱汁。},无犯。(律藏·波逸提)\end{quoting}\begin{quoting}此法律为一味,即解脱之味。(律藏·小品)\end{quoting}\begin{quoting}世尊为义味、法味的分有者。(义释)\end{quoting}等方法所述的流露、正行、酱汁、硬食等其余特相的诸法,被称为「味」。
\item 这些味中\textbf{真实更甜},唯有真实更甜,或更善、更上、更高。因为根味等滋益身体,并带来杂染之乐,在真实之味中,戒离之真实与言语真实的味则以止观等滋益心,并带来无杂染之乐,解脱之味由遍修第一义真实之味而甜,且义味、法味依作为证得此之方法的义、法而转起。
\item 而此中\textbf{有慧的生活},即在盲人、独目、双目者中,双目之人的在家众对工作之从事、皈依、布施之分发、戒之受持、布萨等在家的行道,或出家众对被称为带来无后悔的戒,以及此上的心清净等类的出家的行道,以慧履行而生活。如是,当知其义为「此有慧生活者的生活,或此有慧生活之生活,他们说为最上」。\end{enumerate}

\subsection\*{\textbf{185} {\footnotesize 〔PTS 183〕}}

\textbf{「如何度过暴流?如何度过大海?\\}
\textbf{「如何克服苦?如何得净化?」}

“Kathaṃ su tarati oghaṃ, kathaṃ su tarati aṇṇavaṃ;\\
kathaṃ su dukkham acceti, kathaṃ su parisujjhati”. %\hfill\textcolor{gray}{\footnotesize 3}

\begin{enumerate}\item 如是,听了世尊对四个问题的解答,夜叉心满意足,为问其余四个问题,说了此颂。\end{enumerate}

\subsection\*{\textbf{186} {\footnotesize 〔PTS 184〕}}

\textbf{「以信度过暴流,以不放逸于大海,\\}
\textbf{「以精进克服苦,以慧得净化。」}

“Saddhāya tarati oghaṃ, appamādena aṇṇavaṃ;\\
vīriyena dukkham acceti, paññāya parisujjhati”. %\hfill\textcolor{gray}{\footnotesize 4}

\begin{enumerate}\item 于是,世尊仍以先前的方法而解答,说了此颂。
\item 这里,虽然若人度过四种暴流,他便也度过了轮回大海、克服了流转之苦、净化了烦恼尘垢,即便如是,但因为无信者不信暴流可度而不跃入,放逸者因心投入五欲之中,唯于此取著纠缠,不能度过轮回大海,懈怠者为诸不善法所覆而住于苦,无慧者不知清净之道而不得净化,所以世尊为显明彼等对治,而说此颂。
\item 如是,依此所说,因为信根为预流支的足处,所以,用「\textbf{以信度过暴流}」之句显示度过见之暴流、须陀洹道及须陀洹。
\item 又因为须陀洹为修习诸善法,具足被称为常恒行的不放逸,在履行了第二道后,除仅来此世间一次,度过其余未以须陀洹道得度的有之暴流为依处的轮回大海,所以,用「\textbf{以不放逸于大海}」之句显示度过有之暴流、斯陀含道及斯陀含。
\item 因为斯陀含以精进履行了第三道,克服未以斯陀含道超越的欲之暴流的依处,以及被称为欲之暴流的爱欲之苦,所以,用「\textbf{以精进克服苦}」之句显示度过欲之暴流、阿那含道及阿那含。
\item 又因为阿那含以离于爱欲泥沼的遍净之慧履行了极遍净的第四道慧,舍弃了未以阿那含道舍弃的被称为无明的最上尘垢,所以,用「\textbf{以慧得净化}」之句显示度过无明之暴流、阿罗汉道及阿罗汉。
\item 且在此以阿罗汉之顶点所说的颂的终了,夜叉便住立于须陀洹果。\end{enumerate}

\subsection\*{\textbf{187} {\footnotesize 〔PTS 185〕}}

\textbf{「如何获得智慧?如何得到财产?\\}
\textbf{「如何成就名声?如何交结朋友?\\}
\textbf{「从此世到他世,如何死后无忧?」}

“Kathaṃ su labhate paññaṃ, kathaṃ su vindate dhanaṃ;\\
kathaṃ su kittiṃ pappoti, kathaṃ mittāni ganthati;\\
asmā lokā paraṃ lokaṃ, kathaṃ pecca na socati”. %\hfill\textcolor{gray}{\footnotesize 5}

\begin{enumerate}\item 现在,他抓住此「以慧得净化」中所说的智慧一词,以自己的辩才,为问杂以世出世间的问题,说了此六句之颂。
\item 这里,\textbf{如何},即于此一切处践行义利的发问。因为了知到慧等的义利,而问其践行:如何,即以何践行、以何原因\textbf{获得智慧}?\textbf{财产}等处仿此。\end{enumerate}

\subsection\*{\textbf{188} {\footnotesize 〔PTS 186〕}}

\textbf{「置信于诸阿罗汉的得达涅槃之法,\\}
\textbf{「愿欲听闻、不放逸的明眼人获得智慧。}

“Saddahāno arahataṃ, dhammaṃ nibbānapattiyā;\\
sussūsaṃ labhate paññaṃ, appamatto vicakkhaṇo. %\hfill\textcolor{gray}{\footnotesize 6}

\begin{enumerate}\item 于是,世尊为以四种原因向其显明慧之获得,说了如下几颂。
\item 其义为:佛、辟支佛、声闻等诸阿罗汉在前分以身善行等类,及在后分以三十七道品等类的法得达涅槃,\textbf{置信于}此\textbf{诸阿罗汉的得达涅槃之法},\textbf{获得}世出世间的\textbf{智慧}。且此不仅仅是以信,而是因为\begin{quoting}起信者前往,前往者承事,承事者倾耳,倾耳者闻法。(中部·翅吒山经)\end{quoting}所以,\textbf{愿欲听闻}者以前往开始直至闻法可得(智慧)。
\item 这说的是什么?于此法置信已,按时前往阿阇黎及和尚处,以行义务承事已,当心满意于承事而欲说些什么时,则以不失欲闻而倾耳,当听闻时可得(智慧)。
\item 且如是愿欲听闻者以念不离失而\textbf{不放逸},唯知晓善说、恶说的\textbf{明眼人}可得,而非其他。因此说「不放逸的明眼人」。
\item 如是,因为以信修行导向获得智慧的行道,以愿欲听闻恭敬地听闻证得智慧的方法,以不放逸不忘失所受持者,以明眼不增不减、不颠倒地把握并详绎。或者,以愿欲听闻,倾耳者听闻获得智慧之因的法,以不放逸,闻已而受持法,以明眼考察所受持法之义,然后渐次证得第一义谛。所以,世尊被他问以「如何获得智慧」,为显明四种原因,说了此颂。\end{enumerate}

\subsection\*{\textbf{189} {\footnotesize 〔PTS 187〕}}

\textbf{「行事得体、负责的奋起者得到财产,\\}
\textbf{「以真实成就名声,施予者交结朋友。}

Patirūpakārī dhuravā, uṭṭhātā vindate dhanaṃ;\\
saccena kittiṃ pappoti, dadaṃ mittāni ganthati. %\hfill\textcolor{gray}{\footnotesize 7}

\begin{enumerate}\item 现在,为解答随后的三个问题,说了此颂。
\item 这里,不疏于地点、时间等,为了世间或出世间的财产,得体地运用获得的方法,为\textbf{行事得体}。\textbf{负责},即以心精进不放弃责任。\textbf{奋起},即以\begin{quoting}若思虑寒暑,不多过茅草。(长老偈第 232 颂)\end{quoting}等身精进的方法具足奋起,勇猛不懈。\textbf{得到财产},即如小弟子\footnote{小弟子 \textit{Cūḷantevāsī} 事,见\textbf{本生}义注。}般,以一只老鼠不久便得到二十万的世间财产,以及如大低舍长老般得到出世间财产。因为他以「我将以三种威仪而住」起誓已,当昏沉睡眠来临时,便将稻草圈浸湿,套在头上,进入齐颈深的水中以驱除昏沉睡眠,经十二年,便证得了阿罗汉。
\item \textbf{以真实},即以「真实说者、如实说者」等的言语真实,以「佛、辟支佛、圣弟子」的第一义真实,如是成就名声。
\item \textbf{施予者},即施予任何所希望、所希求者。\textbf{交结朋友},即成就、作为之义。或者,施予难以施予者交结,或者当知以布施为首摄四摄事\footnote{四摄事 \textit{saṅgahavatthu}:即布施、爱语、利行、同事,见\textbf{增支部}第 4:256 经。},即是说以此作为朋友。\end{enumerate}

\subsection\*{\textbf{190} {\footnotesize 〔PTS 188〕}}

\textbf{「对于有信的寻求居家者,存在这四种法:\\}
\textbf{「真实、如法、坚定、舍,他必死后无忧。}

Yass’ete caturo dhammā, saddhassa gharam esino;\\
saccaṃ dhammo dhiti cāgo, sa ve pecca na socati. %\hfill\textcolor{gray}{\footnotesize 8}

\begin{enumerate}\item 如是,以共通于在家出家、杂以世出世间的方法解答了四个问题后,现在,为以在家之义解答「如何死后无忧」这第五个问题,说了此颂。
\item 其义为:\textbf{对于}由具足「置信于诸阿罗汉」中所说的生起一切善法的信而为\textbf{有信的寻求居家者},或对于寻求、寻觅种种五欲,受用爱欲的在家人,当存在「以真实成就名声」中所说品类的\textbf{真实},「愿欲听闻者获得智慧」中以愿欲听闻慧之名所说的\textbf{如法},「负责的奋起者」中以负责之名、奋起之名所说的\textbf{坚定},「施予者交结朋友」中所说品类的\textbf{舍},\textbf{他必死后无忧},从此世去到彼世,他必无忧。\end{enumerate}

\subsection\*{\textbf{191} {\footnotesize 〔PTS 189〕}}

\textbf{「去!你也可以问问其他各个沙门婆罗门,\\}
\textbf{「是否于此有比真实、调御、舍、忍辱更好的?」}

Iṅgha aññe pi pucchassu, puthū samaṇabrāhmaṇe;\\
yadi saccā damā cāgā, khantyā bhiyyo’dha vijjati”. %\hfill\textcolor{gray}{\footnotesize 9}

\begin{enumerate}\item 如是,世尊在解答了第五个问题后,为敦促彼夜叉,说了此颂。
\item 这里,\textbf{去},即敦促之义的不变词。\textbf{其他},即\textbf{可以问问}其他诸法的\textbf{各个沙门婆罗门},或可以问问其他如富楼那等\footnote{富楼那等:即指六师外道等。}自称一切知的各个沙门婆罗门。\textbf{是否于此有比}我们「以真实成就名声」中所说品类的\textbf{真实}更好的成就名声之因,或比「愿欲听闻者获得智慧」中以愿欲听闻慧之部分所说的\textbf{调御}更好的获得世出世间慧之因,或比「施予者交结朋友」中所说品类的\textbf{舍}更好的交结朋友之因,或比「负责的奋起者」中缘彼彼用意以负责之名、奋起之名所说的以承受重担之义至于勤勇状态而被称为精进的\textbf{忍辱}更好的得到世出世间财产之因,或比「真实、如法、坚定、舍」如是所说的这四法\textbf{更好的}从此世到彼世无忧之因。这是此中附以简略章句的释义,而详细者,当知应逐句以抽取意义、抽取词语的注释之法来分别。\end{enumerate}

\subsection\*{\textbf{192} {\footnotesize 〔PTS 190〕}}

\textbf{「为何现在我还要去问其他各个沙门婆罗门?\\}
\textbf{「今天,我了知了来世的义利。}

“Kathaṃ nu dāni puccheyyaṃ, puthū samaṇabrāhmaṇe;\\
yo’haṃ ajja pajānāmi, yo attho samparāyiko. %\hfill\textcolor{gray}{\footnotesize 10}

\begin{enumerate}\item 如是说已,夜叉由舍弃了以之去问他人的疑虑而说「\textbf{为何现在我还要去问其他各个沙门婆罗门}」,为令尚不知晓其不问之因者能知晓而说「今天,我了知了来世的义利」。
\item 这里,\textbf{今天},即以今天为始之意。\textbf{了知},即以如所说的品类而知晓。\textbf{义利},即显明至此以「愿欲听闻者获得智慧」等方法所说的现法者。\textbf{来世},即以「存在这四种法」所说的可令死后无忧的来世者。且义利是原因的同义语,因为「义利」一词在\begin{quoting}有义有文。\end{quoting}等中作为文本之义,在\begin{quoting}长者!我有金钱之利。(中部·贤愚经)\end{quoting}等中为作用,在\begin{quoting}有具戒者之义利。(本生第 1:11 颂)\end{quoting}等中为增长,在\begin{quoting}众人因义利而结交。(本生第 15:89 颂)\end{quoting}等中为财产,在\begin{quoting}行双方的义利。(本生第 7:66 颂)\end{quoting}等中为利益,在\begin{quoting}当义利生起时,则(期望)智者。(本生第 1:92 颂)\end{quoting}等中为原因,但于此为原因,所以,对于获得智慧等的现法之因与死后无忧的来世之因,我今天以世尊所说的方法亲自了知,为何现在我还要去问其他各个沙门婆罗门?如是当略知此中之义。\end{enumerate}

\subsection\*{\textbf{193} {\footnotesize 〔PTS 191〕}}

\textbf{「佛陀确是为了我的义利,才来旷野居住,\\}
\textbf{「今天,我了知了何处布施有大果报。}

Atthāya vata me Buddho, vāsāy’Āḷavim āgamā;\\
yo’haṃ ajja pajānāmi, yattha dinnaṃ mahapphalaṃ. %\hfill\textcolor{gray}{\footnotesize 11}

\begin{enumerate}\item 如是,夜叉在说了「了知了来世的义利」后,为显明其智以世尊为根本,说了此颂。
\item 这里,\textbf{义利},即利益或增长。\textbf{何处布施有大果报},即以\begin{quoting}存在这四种法……(第 190 颂)\end{quoting}之中所说的舍等,若何处布施有大果报,我了知到佛陀为最上的应供之义。而有人说「他就僧伽而如是说的」。\end{enumerate}

\subsection\*{\textbf{194} {\footnotesize 〔PTS 192〕}}

\textbf{「我将从村到村、从城到城地游行,\\}
\textbf{「礼敬着等正觉,以及法的善法性。」}

So ahaṃ vicarissāmi, gāmā gāmaṃ purā puraṃ;\\
namassamāno Sambuddhaṃ, dhammassa ca sudhammatan” ti. %\hfill\textcolor{gray}{\footnotesize 12}

\begin{enumerate}\item 如是,在以上颂显明了自身所得的利益后,现在,为显明为了他人利益的行道,说了此颂。其义当以在雪山经中所说的方法了知。
\item 如是,此颂的终了以及夜色之褪去、鼓掌声之响起、旷野王子被带至夜叉的居处均在同一刹那发生。王臣们听到鼓掌声后,转向于「这样的鼓掌声,除了诸佛,不会为别人生起,难道是世尊来了」,见到世尊的身光后,不似先前站立于外,毫不犹豫进入其内,便看见世尊正坐在夜叉的居处,夜叉则合掌站立。见到后,他们便对夜叉说:「大夜叉!这王子已带来供奉给你,噫!你或吃或啖,或随缘而为!」他因须陀洹而羞耻,且尤其在世尊跟前被这样说,于是以双手受纳了这王子,授予世尊:「尊者!这王子被送来给我,我交给世尊,诸佛皆具哀愍,尊者!请世尊为其利乐受纳这孩子!」并说了此颂:\begin{quoting}这王子有百福之相,肢体健全,相貌圆满,\\我心踊跃欢喜,交给您,请具眼者为了世间的利益受纳!\end{quoting}
\item 世尊便受纳了王子,且在受纳时,为令夜叉及王子吉祥,说了缺句之颂,夜叉则为使王子皈依,三次补足第四句,即:\begin{quoting}愿这王子长寿!\\夜叉!也愿你快乐!\\愿你们无病,为了世间的利益而住!\\这王子皈依佛……法……僧。\end{quoting}
\item 世尊便把王子交给王臣们:「抚养长大后,就交给我。」如是,这王子由从王臣们的手到夜叉的手,从夜叉的手到世尊的手,再从世尊的手到王臣们的手,便得名「旷野之手」。
\item 耕樵等人看到王臣们带了他回来,害怕地问到:「难道夜叉因为王子太幼而不想要吗?」王臣们说「莫怕!世尊已令安稳」,便告知了一切。随后,整个旷野城齐声朝着夜叉发出喊声:「善哉!善哉!」当世尊行乞之时到来,夜叉也持了衣钵,行至中途而返。
\item 于是,世尊在城内行乞后,食事已毕,在城门的某处僻静的树下,设好最上佛座而坐。随后,国王、城民与大众一起结伴而来,走到世尊处,礼拜后围绕而坐,便问:「尊者!如何能调御如是强暴的夜叉?」世尊便从打斗开始,「如是下了九种雨,如是制造恐怖,如是问了问题,我如是为他解答」,对他们说了这旷野经。
\item 随后,国王及城民在毗沙门大王的居处附近造了夜叉的居处,举行具足花、香等恭敬的常祀。且对已至智熟的王子解答说:「你仰赖世尊而得活命,去!承事世尊及比丘僧团!」他当承事世尊及比丘僧团时,不久便住立于阿那含果,学习了一切佛语,为五百优婆塞所随从。且世尊指认他为上首:\begin{quoting}诸比丘!我的声闻优婆塞中,以四摄事摄众的上首,即此旷野之手\footnote{增壹阿含经·清净士品第六则说「恒坐禅思,呵侈阿罗婆是」,见大正藏 T2.559c。}。(增支部第 1:251 经)\end{quoting}\end{enumerate}

\begin{center}\vspace{1em}旷野经第十\\Āḷavakasuttaṃ dasamaṃ.\end{center}

%\begin{flushright}甲辰五月十一二稿\end{flushright}