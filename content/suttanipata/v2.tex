\chapter{小品第二}

\section{宝经}


凡聚集在此的生物,或为地居,或在天上,\hfill\textcolor{gray}{\footnotesize \textbf{224}} \\
愿这一切生物欢喜!然后恭敬地聆听所说!


所以,生物们!请全体倾听!散播慈爱给人的子孙!\hfill\textcolor{gray}{\footnotesize \textbf{225}} \\
他们日夜供奉牺牲,所以,请守护他们,不要放逸!


无论此界或他界的财富,或若天界的胜妙珍宝,\hfill\textcolor{gray}{\footnotesize \textbf{226}} \\
都不能与如来等同,\\
这即是佛中的胜妙珍宝,愿以此真实而得平安!


灭尽、离贪、不死、胜妙,这等持的释迦牟尼之所证,\hfill\textcolor{gray}{\footnotesize \textbf{227}} \\
无有任何可与这法等同,\\
这即是法中的胜妙珍宝,愿以此真实而得平安!


这最胜的觉者所赞叹的纯洁,他们称之为无间的定,\hfill\textcolor{gray}{\footnotesize \textbf{228}} \\
无有与这定等同者,\\
这即是法中的胜妙珍宝,愿以此真实而得平安!


那些善人称赞的八补特伽罗,便是这四双,\hfill\textcolor{gray}{\footnotesize \textbf{229}} \\
他们是应予供养的善逝弟子,于彼等布施有大果报,\\
这即是僧中的胜妙珍宝,愿以此真实而得平安!


那些在乔达摩的教法中以坚固的心意善加致力的无欲者,\hfill\textcolor{gray}{\footnotesize \textbf{230}} \\
他们已达成就,跃入不死,享用着无偿获得的寂灭,\\
这即是僧中的胜妙珍宝,愿以此真实而得平安!


好比嵌于地中的因陀柱,不为四方的风所动摇,\hfill\textcolor{gray}{\footnotesize \textbf{231}} \\
我说像这样的便是善人,他已了知而得见圣谛,\\
这即是僧中的胜妙珍宝,愿以此真实而得平安!


那些彻晓由深慧者所善开示的圣谛者,\hfill\textcolor{gray}{\footnotesize \textbf{232}} \\
他们即使极度放逸,也不会取第八有,\\
这即是僧中的胜妙珍宝,愿以此真实而得平安!


伴随其知见的成就,三法就已舍断:\hfill\textcolor{gray}{\footnotesize \textbf{233}} \\
有身见与疑,或是任何存在的戒禁,


已脱离于四苦处,且不可能犯六重罪,\hfill\textcolor{gray}{\footnotesize \textbf{234}} \\
这即是僧中的胜妙珍宝,愿以此真实而得平安!


他即便以身、语、意造作恶业,\hfill\textcolor{gray}{\footnotesize \textbf{235}} \\
也不可能覆藏它,称之为已见境地的不可能性,\\
这即是僧中的胜妙珍宝,愿以此真实而得平安!


好比林木丛薮中的花冠,正值热季月份的初暑,\hfill\textcolor{gray}{\footnotesize \textbf{236}} \\
他开示了像这样的最上法,趣向涅槃,为了最高的利益,\\
这即是佛中的胜妙珍宝,愿以此真实而得平安!


最上、知最上、施最上、持最上的无上士开示了最上法,\hfill\textcolor{gray}{\footnotesize \textbf{237}} \\
这即是佛中的胜妙珍宝,愿以此真实而得平安!


旧已灭尽,新无生起,于未来有心已离染,\hfill\textcolor{gray}{\footnotesize \textbf{238}} \\
他们种子灭尽,欲不增长,智者们消尽,如那明灯,\\
这即是僧中的胜妙珍宝,愿以此真实而得平安!


凡聚集在此的生物,或为地居,或在天上,\hfill\textcolor{gray}{\footnotesize \textbf{239}} \\
我们礼敬人天供养的如来的佛,愿得平安!

凡聚集在此的生物,或为地居,或在天上,\hfill\textcolor{gray}{\footnotesize \textbf{240}} \\
我们礼敬人天供养的如来的法,愿得平安!

凡聚集在此的生物,或为地居,或在天上,\hfill\textcolor{gray}{\footnotesize \textbf{241}} \\
我们礼敬人天供养的如来的僧,愿得平安!


\section{生腥经}


「稗子、穗子、豆子,以及绿叶、根茎、蔓果,\hfill\textcolor{gray}{\footnotesize \textbf{242}} \\
「善人们吃着如法的所得,不会因爱欲而说谎。


「吃着善预备、善烹饪,由他人施与、馈赠的美味,\hfill\textcolor{gray}{\footnotesize \textbf{243}} \\
「享用着稻粱,迦叶!你在受用生腥。


「你却说『生腥不适宜我』,梵天的眷属!\hfill\textcolor{gray}{\footnotesize \textbf{244}} \\
「享用着稻粱,与善料理的禽肉,\\
「我问你,迦叶!怎样对你才算生腥?」


「杀生,鞭打、割截、捆缚,盗窃、妄语,虚诳、欺诈,\hfill\textcolor{gray}{\footnotesize \textbf{245}} \\
「研究无义之事,亲近他人妻妾,这是生腥,而非食肉。


「若人在此于爱欲不自制,贪图众味,掺杂不净,\hfill\textcolor{gray}{\footnotesize \textbf{246}} \\
「持空无见,不正,固执,这是生腥,而非食肉。


「若粗鄙,强暴,背后噬人,背叛朋友,毫无悲悯而傲慢,\hfill\textcolor{gray}{\footnotesize \textbf{247}} \\
「生性吝啬,且不施与任何人,这是生腥,而非食肉。


「忿怒,㤭慢,顽固,敌对,伪善,妒忌,言谈吹嘘,\hfill\textcolor{gray}{\footnotesize \textbf{248}} \\
「慢过慢,与不善者亲密,这是生腥,而非食肉。


「若生性为恶,毁弃债务且中伤,扭曲诉讼,于此假装,\hfill\textcolor{gray}{\footnotesize \textbf{249}} \\
「若卑鄙之人于此造作罪行,这是生腥,而非食肉。


「若人在此于生命不自制,夺取他人的并实施伤害,\hfill\textcolor{gray}{\footnotesize \textbf{250}} \\
「恶戒、残忍,恶口,不敬,这是生腥,而非食肉。


「于此贪求、对立、杀生,总是热衷,死后前往暗冥,\hfill\textcolor{gray}{\footnotesize \textbf{251}} \\
「众有情落入地狱,头朝下方,这是生腥,而非食肉。


「不是鱼肉、断食,不是裸行,不是秃头、萦发污身、\hfill\textcolor{gray}{\footnotesize \textbf{252}} \\
「粗皮,不是火供侍奉,或世间众多旨在不死的苦行,\\
「以及颂诗、祭品、献牲、时习,能够净化未度疑惑的有死者。


「守护众流,知根而行,住立于法,乐于正直、柔和,\hfill\textcolor{gray}{\footnotesize \textbf{253}} \\
「超越执著,舍弃一切苦,智者不染于所见所闻。」


如是,世尊已再三宣说此义,通晓颂诗者业已明了,\hfill\textcolor{gray}{\footnotesize \textbf{254}} \\
离生腥、无所依、难引导的牟尼已用种种偈颂阐明。


听闻了佛陀善说的离生腥、除一切苦的语句,\hfill\textcolor{gray}{\footnotesize \textbf{255}} \\
他心怀谦卑,便顶礼如来,在此处宣告出家。


\section{惭经}


摆脱着惭,嫌厌着,说着「我是你的」,\hfill\textcolor{gray}{\footnotesize \textbf{256}} \\
而不承担堪能的工作,应知「他不是我的」。


若对朋友说了爱语而不遵行,\hfill\textcolor{gray}{\footnotesize \textbf{257}} \\
智者便知晓(他)言而无行。


他不是朋友:若总是留心,担心背叛,唯挑剔瑕疵,\hfill\textcolor{gray}{\footnotesize \textbf{258}} \\
若在此倚靠,如孩子在胸前,若不被他人分裂,他才是朋友。


承担起为人的责任,希求果报者培育\hfill\textcolor{gray}{\footnotesize \textbf{259}} \\
生起欢喜、带来赞赏、快乐之因。


已饮了远离之味与寂止之味,\hfill\textcolor{gray}{\footnotesize \textbf{260}} \\
他无有恐怖、无有恶,饮着法喜之味。


\section{吉祥经}


如是我闻。一时世尊住舍卫国祇树给孤独园。于是,有一容貌殊胜的天人在深夜中照亮了整座祇园,往世尊处走去,走到后,礼敬了世尊,站在一边。然后,这位站在一边的天人以偈颂对世尊说:


众多天人与人都曾思惟吉祥,\hfill\textcolor{gray}{\footnotesize \textbf{261}} \\
希求幸福,请您说说最上的吉祥。


不亲近愚人,而亲近智者,\hfill\textcolor{gray}{\footnotesize \textbf{262}} \\
供养应供者,这是最上的吉祥。


住于适宜处,过去曾培福,\hfill\textcolor{gray}{\footnotesize \textbf{263}} \\
自身正誓愿,这是最上的吉祥。


博学,技艺,且善学律仪,\hfill\textcolor{gray}{\footnotesize \textbf{264}} \\
若善说话语,这是最上的吉祥。


给侍父母,摄护妻儿,\hfill\textcolor{gray}{\footnotesize \textbf{265}} \\
营生无惑,这是最上的吉祥。


布施,法行,摄护亲族,\hfill\textcolor{gray}{\footnotesize \textbf{266}} \\
诸业无过,这是最上的吉祥。


回避、戒离于恶,克制饮酒,\hfill\textcolor{gray}{\footnotesize \textbf{267}} \\
于诸法不放逸,这是最上的吉祥。


尊重,谦逊,知足,知恩,\hfill\textcolor{gray}{\footnotesize \textbf{268}} \\
时常闻法,这是最上的吉祥。


忍耐,易教,得见沙门,\hfill\textcolor{gray}{\footnotesize \textbf{269}} \\
时常论法,这是最上的吉祥。


苦行,梵行,得见圣谛,\hfill\textcolor{gray}{\footnotesize \textbf{270}} \\
证悟涅槃,这是最上的吉祥。


为世间法所触,其心不动摇,\hfill\textcolor{gray}{\footnotesize \textbf{271}} \\
无忧、离尘、安稳,这是最上的吉祥。


已作此等者,随处皆不败,\hfill\textcolor{gray}{\footnotesize \textbf{272}} \\
随处平安行,这是他们最上的吉祥。


\section{针毛经}

如是我闻。一时世尊住在伽耶石榻针毛夜叉的居处。尔时,粗夜叉与针毛夜叉在世尊不远处经过。于是,粗夜叉对针毛夜叉说:「这是沙门。」「这不是沙门,这是伪沙门,很快我就知道他是沙门还是伪沙门。」


于是,针毛夜叉往世尊处走去,走到后,把身体靠近世尊。然后,世尊便移开身体。于是,针毛夜叉对世尊说:「沙门!你怕我吗?」「我并不怕你,朋友!而是你的触碰粗恶。」


「沙门!我将问你问题,如果你不能向我解答,我就扰乱你的心识,撕碎你的心脏,捉住脚抛到恒河对岸去。」「朋友!我实不见在这俱有天、魔、梵、沙门婆罗门、天人的人世间,有人能扰乱我的心识、撕碎心脏、捉住脚抛到恒河对岸去的,但是,朋友!问你所愿吧!」于是,针毛夜叉以偈颂对世尊说:


贪嗔由何为因?不乐、乐与汗毛竖立由何而生?\hfill\textcolor{gray}{\footnotesize \textbf{273}} \\
寻由何起(驱散)意,如同孩童驱散乌鸦?


贪嗔由此为因,不乐、乐与汗毛竖立由此而生,\hfill\textcolor{gray}{\footnotesize \textbf{274}} \\
寻由此起(驱散)意,如同孩童驱散乌鸦。


从黏腻而生,从自我而成,如同榕树的茎生,\hfill\textcolor{gray}{\footnotesize \textbf{275}} \\
各各纠缠于爱欲,如同藤蔓蔓延于林中。


若知晓那由何为因,他们便除遣之,夜叉!听!\hfill\textcolor{gray}{\footnotesize \textbf{276}} \\
他们度过这先前未度的难度的暴流,无有再有。


\section{法行经}


法行、梵行,他们说这是最上的财富。\hfill\textcolor{gray}{\footnotesize \textbf{277}} \\
若他即便已出家,从家至于非家,


若他生性饶舌,乐于恼害,粗野,\hfill\textcolor{gray}{\footnotesize \textbf{278}} \\
则其活命甚恶,自身的尘垢增长。


乐于争辩的比丘,为愚痴之法所遮蔽,\hfill\textcolor{gray}{\footnotesize \textbf{279}} \\
即便被告知,也不了知佛陀开示的法。


恼乱修己者,由无明前导,\hfill\textcolor{gray}{\footnotesize \textbf{280}} \\
不知杂染与趣向地狱之道。


落入堕处,从胎到胎,从暗到暗,\hfill\textcolor{gray}{\footnotesize \textbf{281}} \\
这样的比丘,死后必然经历痛苦。


好比是经年满盈的粪坑,\hfill\textcolor{gray}{\footnotesize \textbf{282}} \\
这样的人也是如此,因有垢而难清洗。


你们得知这样的人,诸比丘!依赖俗家,\hfill\textcolor{gray}{\footnotesize \textbf{283}} \\
恶欲,恶思惟,恶正行与行处,


你们应当全体和合后,远离他,\hfill\textcolor{gray}{\footnotesize \textbf{284}} \\
应当清扫垃圾,应当清除沉滓。


然后清除自认为沙门的非沙门之糠,\hfill\textcolor{gray}{\footnotesize \textbf{285}} \\
清扫了恶欲、恶正行与行处者,


让清净、遵从者与清净者共住!\hfill\textcolor{gray}{\footnotesize \textbf{286}} \\
然后,和合而贤明,你们得尽苦的边际!


\section{婆罗门法经}

如是我闻。一时世尊住舍卫国祇树给孤独园。尔时,众多㤭萨罗的富裕婆罗门衰老、年迈、高龄、迟暮、岁月已逝,往世尊处走去,走到后,问候了世尊,彼此寒暄已,坐在一边。


这些坐在一边的富裕婆罗门对世尊说:「乔达摩君!现今还有婆罗门在古昔婆罗门的婆罗门法上吗?」「众婆罗门!现今没有婆罗门在古昔婆罗门的婆罗门法上。」「善哉!若不麻烦乔达摩君的话,请乔达摩君对我们说说古昔婆罗门的婆罗门法!」「那么,众婆罗门!谛听!善加作意!我将说法。」「如是,先生!」这些富裕婆罗门答世尊。世尊说:


从前的仙人们是自制的苦行者,\hfill\textcolor{gray}{\footnotesize \textbf{287}} \\
舍弃了种种五欲,践行自己的义利。


婆罗门没有牲畜,没有货币,没有谷物,\hfill\textcolor{gray}{\footnotesize \textbf{288}} \\
他们以诵经为财产和谷物,守护梵天的伏藏。


那为他们所制的、放在门前的食物,\hfill\textcolor{gray}{\footnotesize \textbf{289}} \\
他们认为这应当布施给寻求信制者。


以多彩的衣服,以及卧具、住所,\hfill\textcolor{gray}{\footnotesize \textbf{290}} \\
众繁荣的地方和王国礼敬这些婆罗门。


婆罗门不可侵犯,不可战胜,受法守护,\hfill\textcolor{gray}{\footnotesize \textbf{291}} \\
没有人能以任何方式在家门前拒绝他们。


他们行持四十八年的童贞梵行,\hfill\textcolor{gray}{\footnotesize \textbf{292}} \\
在过去,婆罗门践行对明行的遍求。


婆罗门不去别处,他们也不买妻子,\hfill\textcolor{gray}{\footnotesize \textbf{293}} \\
唯以相爱结合,他们乐于同居。


除了这时期,直到月经结束,\hfill\textcolor{gray}{\footnotesize \textbf{294}} \\
在此期间,婆罗门不行交媾。


梵行、戒、诚实、温柔、苦行、\hfill\textcolor{gray}{\footnotesize \textbf{295}} \\
调柔、不害及忍辱,为人称颂。


其中的最上者是努力勇猛的梵,\hfill\textcolor{gray}{\footnotesize \textbf{296}} \\
他即便在梦中也不行交媾。


效仿其行仪,于此,一些生性有智者\hfill\textcolor{gray}{\footnotesize \textbf{297}} \\
也称颂梵行、戒及忍辱。


如法乞求米粒、卧具、衣服及酥油,\hfill\textcolor{gray}{\footnotesize \textbf{298}} \\
汇集已,随后举行献牲。


在备好的献牲中,他们并不杀牛,\hfill\textcolor{gray}{\footnotesize \textbf{299}} \\
好比母亲、父亲、兄弟,或其他亲戚,\\
牛是我们最好的朋友,从中产出药物,


且其给予食、给予力,还给予色、给予乐,\hfill\textcolor{gray}{\footnotesize \textbf{300}} \\
了知了这用意,他们并不杀牛。


曼妙、硕大、美貌、享有名声的\hfill\textcolor{gray}{\footnotesize \textbf{301}} \\
婆罗门凭自身的法,热心于应作、不应作,\\
只要他们在世间转起,这人类便增长快乐。


他们生起颠倒,自从渐渐见到了\hfill\textcolor{gray}{\footnotesize \textbf{302}} \\
国王的华丽,以及严饰的女人们,


套有纯种马匹、精制、彩缀的车辆,\hfill\textcolor{gray}{\footnotesize \textbf{303}} \\
以及按部分规划、丈量的住处居所。


牛群遍布,美女簇拥,\hfill\textcolor{gray}{\footnotesize \textbf{304}} \\
婆罗门便贪求显赫的人间财富。


于此,他们编集了颂诗,然后去往甘蔗王处:\hfill\textcolor{gray}{\footnotesize \textbf{305}} \\
「你将有丰厚的财产和谷物,\\
「献祭吧!你的资产众多,献祭吧!你的财产众多。」


随后,被婆罗门说服的国王、车乘之主,\hfill\textcolor{gray}{\footnotesize \textbf{306}} \\
马牲、人牲、掷棒祭、娑摩祭、无遮祭——\\
举行完这些祭祀,他便赐予婆罗门财产。


牛群、卧具与衣服,严饰的女人们,\hfill\textcolor{gray}{\footnotesize \textbf{307}} \\
套有纯种马匹、精致、彩缀的车辆,


惬意、按部分善加规划的住处\hfill\textcolor{gray}{\footnotesize \textbf{308}} \\
盈满了种种谷物,他便赐予婆罗门财产。

他们于此获得财产后,便乐于贮藏,\hfill\textcolor{gray}{\footnotesize \textbf{309}} \\
陷入希求的他们,增长了更多渴爱,\\
于此,他们编集了颂诗,再次去往甘蔗王处:


「好比水、地、货币、财产和谷物,\hfill\textcolor{gray}{\footnotesize \textbf{310}} \\
「牛对人也如是,因为这是生命的资助,\\
「献祭吧!你的资产众多,献祭吧!你的财产众多。」


随后,被婆罗门说服的国王、车乘之主\hfill\textcolor{gray}{\footnotesize \textbf{311}} \\
便在献牲中杀了数百千头牛。


它们并未用足、用角伤害任何人,\hfill\textcolor{gray}{\footnotesize \textbf{312}} \\
牛如同山羊,温顺,产成桶的奶,\\
国王捉住角,用刀杀了它们。

随后,诸天、诸父、因陀、阿修罗、罗刹\hfill\textcolor{gray}{\footnotesize \textbf{313}} \\
便呼喊道「非法」,当刀落在牛上。


过去有三种疾病:欲望、饥饿、衰老,\hfill\textcolor{gray}{\footnotesize \textbf{314}} \\
从杀戮牲畜以后,乃有九十八种到来。


这刑罚之非法便流传,成了古昔,\hfill\textcolor{gray}{\footnotesize \textbf{315}} \\
无辜者被杀害,祭司们从法跌落。


如是,这古昔微末的法受智者谴责,\hfill\textcolor{gray}{\footnotesize \textbf{316}} \\
当看到这样的事,人们便谴责祭司。


当法如是差谬时,首陀罗与吠舍分离,\hfill\textcolor{gray}{\footnotesize \textbf{317}} \\
刹帝利各各分离,妻子轻视丈夫。


刹帝利和梵天的眷属,以及其他受种姓守护者,\hfill\textcolor{gray}{\footnotesize \textbf{318}} \\
摈弃了出身论,沦于爱欲的控制。


如是说已,这些富裕婆罗门对世尊说:「希有!乔达摩君!……从今起,尽寿命,请乔达摩君受持我们皈依为优婆塞!」


\section{船经}


人对能从其了知法者,应如诸天对因陀般尊敬,\hfill\textcolor{gray}{\footnotesize \textbf{319}} \\
那受到尊敬、对其心生净喜的多闻者会阐明法。


智者留心、留意于此,践行着法之随法,\hfill\textcolor{gray}{\footnotesize \textbf{320}} \\
不放逸者结交此等,成为有智、明辨、微妙。


亲近着下劣、愚痴、不达义利、妒忌者,\hfill\textcolor{gray}{\footnotesize \textbf{321}} \\
于此便不明了法,未度疑惑,趣近死亡。


好比人落入了河中,水流洪大湍急,\hfill\textcolor{gray}{\footnotesize \textbf{322}} \\
他且随流漂没,如何能将他人渡过?


同样,不明了法,不倾听多闻者的语义,\hfill\textcolor{gray}{\footnotesize \textbf{323}} \\
自身且无知、未度疑惑,如何能使他人理解?


又如登上坚固的船,具备浆舵,\hfill\textcolor{gray}{\footnotesize \textbf{324}} \\
知晓此理、善巧、具慧,他于此能渡脱旁人。


如是,若通达诸明、修己、多闻、具不动之法,\hfill\textcolor{gray}{\footnotesize \textbf{325}} \\
他既遍知,故能使具足倾听及基础的他人理解。


所以,一定要结交既有智、又多闻的善人,\hfill\textcolor{gray}{\footnotesize \textbf{326}} \\
知晓义利并践行,了知法,他便获得快乐。


\section{何戒经}

何戒、何行事、增长何业的\hfill\textcolor{gray}{\footnotesize \textbf{327}} \\
人能完全住立,且圆满最高的义利?


应尊敬长者、不妒忌,且应知时去见老师,\hfill\textcolor{gray}{\footnotesize \textbf{328}} \\
知道谈论法义的时机,应恭敬地听闻善说。


应适时去到老师近前,撇去顽固,举止谦卑,\hfill\textcolor{gray}{\footnotesize \textbf{329}} \\
对于语义、法、自制、梵行,应随念并实践。


乐法,喜法,住立于法,知抉择法,\hfill\textcolor{gray}{\footnotesize \textbf{330}} \\
不应从事污法的言论,应受如实的善说引领。


戏笑、闲谈、悲、恼、行伪善、诡诈、贪求、慢、\hfill\textcolor{gray}{\footnotesize \textbf{331}} \\
愤激、粗俗、恶浊及沉迷,舍弃已,应离㤭慢而坚定。


所知是善说的精髓,所闻与所知的精髓是三摩地,\hfill\textcolor{gray}{\footnotesize \textbf{332}} \\
若人急躁、放逸,则其智慧与所闻不会增长。


喜于圣者宣说之法者,他们在语、意、行上都无比,\hfill\textcolor{gray}{\footnotesize \textbf{333}} \\
他们住立于寂静调柔、三摩地,得证所闻与智慧的精髓。


\section{起身经}


起身!坐正!睡眠对你们有何义利?\hfill\textcolor{gray}{\footnotesize \textbf{334}} \\
对于苦患、被箭射伤者,为何睡眠?


起身!坐正!为了寂静,努力修学!\hfill\textcolor{gray}{\footnotesize \textbf{335}} \\
莫让死王得知你们放逸,好愚弄入其彀中者!


诸天与人以之束缚、欲求而持存的\hfill\textcolor{gray}{\footnotesize \textbf{336}} \\
爱著,越过它!莫要让你们的时机流逝!\\
因为错过时机者被付诸地狱,随即忧伤。


放逸是尘垢,跟随放逸的放逸是尘垢,\hfill\textcolor{gray}{\footnotesize \textbf{337}} \\
以不放逸,以明,他能拔出自己的箭。


\section{罗睺罗经}

「是否因为经常共住,你不轻视智者?\hfill\textcolor{gray}{\footnotesize \textbf{338}} \\
「为人类持炬者,是否受到你的尊敬?」


「因为经常共住,我不轻视智者,\hfill\textcolor{gray}{\footnotesize \textbf{339}} \\
「为人类持炬者,始终受到我的尊敬。」


「舍弃了可爱、悦意的种种五欲,\hfill\textcolor{gray}{\footnotesize \textbf{340}} \\
「因信出家已,当成得尽苦边者!


「应亲近善知识,与边鄙、远离、\hfill\textcolor{gray}{\footnotesize \textbf{341}} \\
「少愦闹的住处!于饮食当知量!


「衣服、乞食,以及资具、住处,\hfill\textcolor{gray}{\footnotesize \textbf{342}} \\
「莫于这些生渴爱!莫再来世间!


「防护于波罗提木叉与五根,\hfill\textcolor{gray}{\footnotesize \textbf{343}} \\
「应作身至念,应多多厌逆!


「回避净的、伴有贪染的相!\hfill\textcolor{gray}{\footnotesize \textbf{344}} \\
「于不净修习心,一境、善等持!


「且应修习无相!舍弃慢的随眠!\hfill\textcolor{gray}{\footnotesize \textbf{345}} \\
「随后,因慢的止息,寂静得行。」


如此,世尊常常以这些偈颂教诫尊者罗睺罗。


\section{尼拘律劫波经}

如是我闻。一时世尊在旷野中,住旷野顶支提。尔时,尊者婆耆舍的亲教师,名为尼拘律劫波的长老在旷野顶支提般涅槃不久。于是,尊者婆耆舍幽居宴坐,便起如是寻思:「我的亲教师是已般涅槃,还是未般涅槃?」


于是,尊者婆耆舍晡时从宴坐起,往世尊处走去,走到后,礼敬了世尊,坐在一边。坐在一边的尊者婆耆舍对世尊说:「于此,尊者!我幽居宴坐,便起如是寻思『我的亲教师是已般涅槃,还是未般涅槃』。」然后,尊者婆耆舍从坐起,把衣偏覆一肩,向世尊合掌,以偈颂对世尊说:


「我问大师、最上慧,于此现法已断疑惑者:\hfill\textcolor{gray}{\footnotesize \textbf{346}} \\
「比丘在旷野顶死去,有名闻,有声望,内在寂静。


「尼拘律劫波是他的名字,由你取给婆罗门,世尊!\hfill\textcolor{gray}{\footnotesize \textbf{347}} \\
「他礼敬你,希求解脱,勇猛精进,示现坚固法者!


「释迦!我们全都希望知晓此声闻,一切眼者!\hfill\textcolor{gray}{\footnotesize \textbf{348}} \\
「我们的耳朵已为倾听竖立,你是我们的大师,你是无上士。


「请断除我们的疑惑!对我说他!了知般涅槃已,宏慧者!\hfill\textcolor{gray}{\footnotesize \textbf{349}} \\
「在我们中说!一切眼者!如千眼帝释在诸天中。


「于此世中,任何系缚、痴路、无知的品类、疑惑处,\hfill\textcolor{gray}{\footnotesize \textbf{350}} \\
「在遇到如来后,便不存在,因为他是人中最胜之眼。


「因为若无人能驱散烦恼,如同风驱散层云,\hfill\textcolor{gray}{\footnotesize \textbf{351}} \\
「一切世间将黯淡、覆蔽,即便具光辉者也无法闪耀。


「且智者们是制造光明者,英雄!我认为你就是如此,\hfill\textcolor{gray}{\footnotesize \textbf{352}} \\
「我们来到具观者、知者前,请在会众中为我们展示劫波!


「请快发美妙的话言!美妙者!如同天鹅伸展后,舒缓地和鸣!\hfill\textcolor{gray}{\footnotesize \textbf{353}} \\
「声音圆润,善加调音,我们全都正身倾听着你!


「催促了已无余舍断了生死者、除遣者,我将请他说法,\hfill\textcolor{gray}{\footnotesize \textbf{354}} \\
「因为凡夫中没有随欲而行者,而诸如来则随思量而行。


「你——正慧者的这圆满记说已被证实,\hfill\textcolor{gray}{\footnotesize \textbf{355}} \\
「施以这最后的合掌,了知者请莫愚弄!最上慧!


「知晓了各种圣法,了知者莫愚弄!最上雄!\hfill\textcolor{gray}{\footnotesize \textbf{356}} \\
「好比夏天患暑期待水,我期待言语,请倾吐声音吧!


「劫波衍那所行的有义梵行,是否不是徒劳?\hfill\textcolor{gray}{\footnotesize \textbf{357}} \\
「他是涅槃了还是有余依?我们听听他如何解脱。」


「他切断了于此名色的渴爱、」世尊说,「黑者之流的长时随眠,\hfill\textcolor{gray}{\footnotesize \textbf{358}} \\
「无余度脱了生死。」五者最胜的世尊如是说。


「听到你这言语,我得净喜,最善的仙人!\hfill\textcolor{gray}{\footnotesize \textbf{359}} \\
「看来我的提问并非徒劳,婆罗门没有欺骗我。


「如是说而如是行,他是佛陀的声闻,\hfill\textcolor{gray}{\footnotesize \textbf{360}} \\
「切断了死亡——欺瞒者——撒布的坚牢的网。


「世尊!劫波见到了取著的源头,\hfill\textcolor{gray}{\footnotesize \textbf{361}} \\
「劫波衍那确实超越了极难度的死境。」


\section{正游行经}


「我问牟尼、广慧者,已度、已到彼岸、已止息的坚定者,\hfill\textcolor{gray}{\footnotesize \textbf{362}} \\
「从家出离,除去爱欲,那比丘应如何在世间正当地游行?」


「若其摒除了吉祥、」世尊说,「征兆、梦和相,\hfill\textcolor{gray}{\footnotesize \textbf{363}} \\
「他便舍弃吉祥的过失,他能在世间正当地游行。


「比丘应当调伏对于人界与天界爱欲的贪染,\hfill\textcolor{gray}{\footnotesize \textbf{364}} \\
「越过了有,体认了法,他能在世间正当地游行。


「把诽谤抛在了背后,比丘应舍弃忿怒和贪婪,\hfill\textcolor{gray}{\footnotesize \textbf{365}} \\
「舍弃了顺从和违逆,他能在世间正当地游行。


「舍弃了喜与不喜,无所取著,不依于任何处,\hfill\textcolor{gray}{\footnotesize \textbf{366}} \\
「从诸可结缚处解脱,他能在世间正当地游行。


「他不在依持中寻找坚实,于诸取著调伏欲贪,\hfill\textcolor{gray}{\footnotesize \textbf{367}} \\
「他无依,不被他人引领,他能在世间正当地游行。


「不以语、意与业而违逆,正当地了知了法,\hfill\textcolor{gray}{\footnotesize \textbf{368}} \\
「愿求着涅槃境界,他能在世间正当地游行。


「若『他礼拜我』,能不高举,遇到骂詈,比丘也能不受影响,\hfill\textcolor{gray}{\footnotesize \textbf{369}} \\
「得到他人的食物能不陶醉,他能在世间正当地游行。


「比丘舍弃了贪与有,且戒离了割截捆绑,\hfill\textcolor{gray}{\footnotesize \textbf{370}} \\
「他度诸犹疑,离于箭刺,他能在世间正当地游行。


「了知了自身的适宜,比丘不应伤害世间任何人,\hfill\textcolor{gray}{\footnotesize \textbf{371}} \\
「如实地了知了法,他能在世间正当地游行。


「若其已无任何随眠,并且诸不善根已被铲除,\hfill\textcolor{gray}{\footnotesize \textbf{372}} \\
「他无意乐、无希求,他能在世间正当地游行。


「漏已尽,慢已舍,越过了一切贪路,\hfill\textcolor{gray}{\footnotesize \textbf{373}} \\
「已调御,已止息,坚定,他能在世间正当地游行。


「具信、多闻的见决定者,在人群中不随众的智者,\hfill\textcolor{gray}{\footnotesize \textbf{374}} \\
「调伏了贪、嗔、恚,他能在世间正当地游行。


「清净的胜者,去蔽者,受控于法,到达彼岸,不动者,\hfill\textcolor{gray}{\footnotesize \textbf{375}} \\
「善巧于行灭智,他能在世间正当地游行。


「于过去及未来克服了思惟,极度净慧者,\hfill\textcolor{gray}{\footnotesize \textbf{376}} \\
「从一切处解脱,他能在世间正当地游行。


「知晓了句,体认了法,明了地见到诸漏的舍断,\hfill\textcolor{gray}{\footnotesize \textbf{377}} \\
「灭尽了一切依持,他能在世间正当地游行。」


「确实,世尊!这实如此,凡如是住的调御比丘,\hfill\textcolor{gray}{\footnotesize \textbf{378}} \\
「已超越一切结缚与轭,他能在世间正当地游行。」


\section{如法经}

如是我闻。一时世尊住舍卫国祇树给孤独园。尔时,优婆塞如法与五百优婆塞一起,往世尊处走去,走到后,礼敬了世尊,坐在一边。坐在一边的优婆塞如法以偈颂对世尊说:


「我问你,宏慧的乔达摩!如何行事的弟子才是善的?\hfill\textcolor{gray}{\footnotesize \textbf{379}} \\
「无论他从家至于非家,或是居家的优婆塞。


「因为你知晓俱有天的世间的趣向和归宿,\hfill\textcolor{gray}{\footnotesize \textbf{380}} \\
「且无有等同,见微妙义者!他们说你是高贵的佛陀。


「你证得了一切智,为怜悯有情而阐明法,\hfill\textcolor{gray}{\footnotesize \textbf{381}} \\
「你是去蔽者、一切眼者,无垢者在一切世间闪耀。


「名为伊罗婆那的象王来到你的面前,听闻『胜者』后,\hfill\textcolor{gray}{\footnotesize \textbf{382}} \\
「他也向你讨教,听后说了『善哉』,喜形于色而离开。


「王者毗沙门·俱鞞罗也亲近你而遍问法,\hfill\textcolor{gray}{\footnotesize \textbf{383}} \\
「你回答了他的提问,智者!他听后也喜形于色。


「任何习于思辨的外道,无论是邪命者还是尼乾陀,\hfill\textcolor{gray}{\footnotesize \textbf{384}} \\
「全都无法以慧超越你,如站立者之于急速而行者。


「任何习于思辨的年长婆罗门,与任何存在的婆罗门,\hfill\textcolor{gray}{\footnotesize \textbf{385}} \\
「以及其他自认为是论师者,全都靠你获得义利。


「因为这法微妙且乐,它由你,世尊!所善宣说,\hfill\textcolor{gray}{\footnotesize \textbf{386}} \\
「全都想听闻它,当被问及,请对我们说它!最胜的佛陀!


「所有这些共坐的比丘,与众优婆塞都同样想听,\hfill\textcolor{gray}{\footnotesize \textbf{387}} \\
「让他们听无垢者所随觉的法!如诸天听婆娑婆的善说。」


「请听我!诸比丘!我让你们闻除遣之法,你们都应奉行之,\hfill\textcolor{gray}{\footnotesize \textbf{388}} \\
「见到义利的具慧者,应从事这随顺出家的威仪路。


「比丘不应在非时游行,而应适时在村中行乞,\hfill\textcolor{gray}{\footnotesize \textbf{389}} \\
「因为执著系缚非时行者,所以诸佛不在非时游行。


「那些让有情迷醉的色、声、味、香、触,\hfill\textcolor{gray}{\footnotesize \textbf{390}} \\
「应调伏对这些法的欲,他应按时进早餐。


「比丘按时得了团食,独自回返,应坐于幽僻处,\hfill\textcolor{gray}{\footnotesize \textbf{391}} \\
「内省,不应向外用意,端摄自体。


「若他与弟子,或其他人,或任何比丘交谈,\hfill\textcolor{gray}{\footnotesize \textbf{392}} \\
「应谈论殊胜的法,不诽谤,也不指责他人。


「因为有些人反对言论,我们不赞叹这些小慧者,\hfill\textcolor{gray}{\footnotesize \textbf{393}} \\
「执著从此系缚他们,因为他们把心放逐到远处。


「团食、寺庙、坐卧具与洗除僧伽梨尘土的水,\hfill\textcolor{gray}{\footnotesize \textbf{394}} \\
「听闻了善逝开示的法,胜慧的弟子经省思而受用。


「所以,团食、坐卧具与洗除僧伽梨尘土的水,\hfill\textcolor{gray}{\footnotesize \textbf{395}} \\
「对于这些法不染的比丘,好比莲花上的水珠。


「我对你们说在家的义务,如是行事的弟子才是善的,\hfill\textcolor{gray}{\footnotesize \textbf{396}} \\
「因为这全部的比丘法不能为有资产者触及。


「不应伤害生命,不应教人伤害,不应认可其他伤害者,\hfill\textcolor{gray}{\footnotesize \textbf{397}} \\
「对一切生物放下了棍杖,凡存在于世间的强者和弱者。


「此后,有觉知的弟子应避免任何场所的任何未给予物,\hfill\textcolor{gray}{\footnotesize \textbf{398}} \\
「不应教人取,也不应认可取者,能避免一切未给予物。


「智者应避免非梵行,如避免燃烧的火坑,\hfill\textcolor{gray}{\footnotesize \textbf{399}} \\
「而不能行梵行者,不应侵犯他人的妻妾。


「若有人去往会堂,或去往集会,不应对人妄语,\hfill\textcolor{gray}{\footnotesize \textbf{400}} \\
「不应教人说,不应认可说者,能避免一切非实。


「若在家人乐于此法,便不应饮用麻醉与饮品,\hfill\textcolor{gray}{\footnotesize \textbf{401}} \\
「了知到这『导致疯狂』,不应教人饮,不应认可饮者。


「因为愚人们因迷醉而作恶,还教其他放逸的人也作,\hfill\textcolor{gray}{\footnotesize \textbf{402}} \\
「应回避这致人疯狂、愚痴、愚人爱悦的非福处。


「不应伤害生命,不应取不予物,不应妄语,不应饮麻醉,\hfill\textcolor{gray}{\footnotesize \textbf{403}} \\
「应戒离非梵行、淫欲,不应在夜间受用非时食,


「不应持花鬘,不应涂香,应睡在床、地面、卧具上,\hfill\textcolor{gray}{\footnotesize \textbf{404}} \\
「他们称这为八支布萨,由行到苦边的佛陀阐发。


「此后,遵守布萨,在半月的十四、十五、八日,\hfill\textcolor{gray}{\footnotesize \textbf{405}} \\
「与神变半月,意怀净喜,八支具足而完整。


「此后,早晨,遵守布萨者应以饮食对比丘僧团\hfill\textcolor{gray}{\footnotesize \textbf{406}} \\
「如其所应地分享,智者心怀净喜而随喜。


「他应如法赡养父母,应从事合法的贸易,\hfill\textcolor{gray}{\footnotesize \textbf{407}} \\
「这在家人坚持、不放逸,得至名为自身光芒的天人。」
