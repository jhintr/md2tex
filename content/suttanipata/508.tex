\section{难陀学童问}

\subsection\*{\textbf{1084} {\footnotesize 〔PTS 1077〕}}

\textbf{「人们说『世间有牟尼』,」尊者难陀说,「这是如何?\\}
\textbf{「他们是说具足智者是牟尼,还是说具足活命者?」}

\begin{enumerate}\item 这里,初颂之义为:刹帝利等\textbf{人们}就活命者与尼乾陀等\textbf{说「世间有牟尼」},\textbf{这是如何}?\textbf{他们是说}由具足等至之智等的智的\textbf{具足智者是牟尼},\textbf{还是说具足}种种品类被称为粗鄙活命的\textbf{活命者}?\end{enumerate}

\subsection\*{\textbf{1085} {\footnotesize 〔PTS 1078〕}}

\textbf{「善巧者不以见、不以闻、不以智而说是此处的牟尼,难陀!\\}
\textbf{「消灭了敌军,无患、无待而行者,我说他们是牟尼。」}

\begin{enumerate}\item 于是,世尊拒斥了两者,为对其显示牟尼,说了此颂。\end{enumerate}

\subsection\*{\textbf{1086} {\footnotesize 〔PTS 1079〕}}

\textbf{「凡是这些沙门、婆罗门,」尊者难陀说,「以见闻而说清净,\\}
\textbf{「以戒禁而说清净,以多种方式而说清净,\\}
\textbf{「世尊!他们于此自制而行,是否得度生与老?先生!\\}
\textbf{「我问你,世尊!请对我说说这个!」}

\begin{enumerate}\item 现在,为了舍弃对「以所见等为清净」论者之论说的疑惑,问了此颂。这里,\textbf{多种方式},即庆典、祥瑞等。\textbf{于此自制而行},即于此以有身见守护而住。\end{enumerate}

\subsection\*{\textbf{1087} {\footnotesize 〔PTS 1080〕}}

\textbf{「凡是这些沙门、婆罗门,难陀!」世尊说,「以见闻而说清净,\\}
\textbf{「以戒禁而说清净,以多种方式而说清净,\\}
\textbf{「即便他们于此自制而行,我说无法得度生老。」}

\begin{enumerate}\item 于是,世尊为对其显明如是并非清净,说了第二颂。\end{enumerate}

\subsection\*{\textbf{1088} {\footnotesize 〔PTS 1081〕}}

\textbf{「凡是这些沙门、婆罗门,」尊者难陀说,「以见闻而说清净,\\}
\textbf{「以戒禁而说清净,以多种方式而说清净,\\}
\textbf{「牟尼!如果你说无法度过暴流,那么在天与人的世间,谁能\\}
\textbf{「得度生与老?先生!我问你,世尊!请对我说说这个!」}

\begin{enumerate}\item 如是,听到「无法得度」后,现在,他欲闻得度者,问了此颂。\end{enumerate}

\subsection\*{\textbf{1089} {\footnotesize 〔PTS 1082〕}}

\textbf{「我不说所有的沙门、婆罗门,难陀!」世尊说,「为生老覆蔽,\\}
\textbf{「那些于此舍弃了一切所见、所闻、所觉或戒禁,\\}
\textbf{「舍弃了种种一切方式,遍知了渴爱的无漏者们,\\}
\textbf{「我说这些人已度过暴流。」}

\begin{enumerate}\item 于是,世尊为对其以度过暴流为首显示得度生老,说了第三颂。这里,\textbf{覆蔽},即遮盖、束缚。\textbf{遍知了渴爱},即以三遍知遍知了渴爱。其余一切处由先前已述故,皆自明。\end{enumerate}

\subsection\*{\textbf{1090} {\footnotesize 〔PTS 1083〕}}

\textbf{「我欢喜大仙的这番话语,乔达摩!无依持是善说,\\}
\textbf{「那些于此舍弃了一切所见、所闻、所觉或戒禁,\\}
\textbf{「舍弃了种种一切方式,遍知了渴爱的无漏者们,\\}
\textbf{「我也说他们已度过暴流。」}

\begin{enumerate}\item 如是,世尊仍以阿罗汉为顶点完成了开示。当开示终了,难陀欢喜于世尊之所说,说了此颂。且于此,与先前所说的一样,而有法的现观。\end{enumerate}

\begin{center}难陀学童问第七\end{center}