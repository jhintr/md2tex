\section{难陀学童问}

\begin{center}Nanda Māṇava Pucchā\end{center}\vspace{1em}

\subsection\*{\textbf{1084} {\footnotesize 〔PTS 1077〕}}

\textbf{「人们说『世间有牟尼』,」尊者难陀说,「这是什么意思?\\}
\textbf{「他们说具足智者是牟尼,还是说具足活命者?」}

“‘Santi loke munayo’, \textit{(icc āyasmā Nando)} janā vadanti ta-y-idaṃ kathaṃ su;\\
ñāṇūpapannaṃ no muniṃ vadanti, udāhu ve jīvitenūpapannaṃ”. %\hfill\textcolor{gray}{\footnotesize 1}

\begin{enumerate}\item 在世间,刹帝利等\textbf{人们}就邪命者与尼乾陀而\textbf{说「有牟尼」},\textbf{这是什么意思}?他们是说由具足等至之智等的智的\textbf{具足智者}为牟尼,还是说具足以种种行相被称为粗鄙活命的活命的\textbf{具足活命者}?\end{enumerate}

\subsection\*{\textbf{1085} {\footnotesize 〔PTS 1078〕}}

\textbf{「善巧者不以见、不以闻、不以智而说是此间的牟尼,难陀!\\}
\textbf{「消灭了敌军,无患、无待而行者,我说他们是牟尼。」}

“Na diṭṭhiyā na sutiyā na ñāṇena, munīdha Nanda kusalā vadanti;\\
visenikatvā anīghā nirāsā, caranti ye te munayo ti brūmi”. %\hfill\textcolor{gray}{\footnotesize 2}

\begin{enumerate}\item 于是,世尊拒斥了两者,为显明牟尼而说此颂。\end{enumerate}

\subsection\*{\textbf{1086} {\footnotesize 〔PTS 1079〕}}

\textbf{「凡是这些沙门、婆罗门,」尊者难陀说,「以见闻而说清净,\\}
\textbf{「以戒禁而说清净,以多种方式而说清净,\\}
\textbf{「世尊!他们于此自制而行,是否得度生与老?先生!\\}
\textbf{「我问您,世尊!请对我说说这个!」}

“Ye kec’ime samaṇabrāhmaṇāse, \textit{(icc āyasmā Nando)} diṭṭhassutenāpi vadanti suddhiṃ;\\
sīlabbatenāpi vadanti suddhiṃ, anekarūpena vadanti suddhiṃ;\\
kacci ssu te Bhagavā tattha yatā carantā, atāru jātiñ ca jarañ ca Mārisa;\\
pucchāmi taṃ Bhagavā brūhi me taṃ”. %\hfill\textcolor{gray}{\footnotesize 3}

\begin{enumerate}\item \textbf{多种方式},即节庆、祝祭等。\textbf{于此自制而行},即于此以有身见守护而住。\end{enumerate}

\subsection\*{\textbf{1087} {\footnotesize 〔PTS 1080〕}}

\textbf{「凡是这些沙门、婆罗门,难陀!」世尊说,「以见闻而说清净,\\}
\textbf{「以戒禁而说清净,以多种方式而说清净,\\}
\textbf{「即便他们于此自制而行,我说也无法得度生与老。」}

“Ye kec’ime samaṇabrāhmaṇāse, \textit{(Nandā ti Bhagavā)} diṭṭhassutenāpi vadanti suddhiṃ;\\
sīlabbatenāpi vadanti suddhiṃ, anekarūpena vadanti suddhiṃ;\\
kiñcāpi te tattha yatā caranti, nātariṃsu jātijaran ti brūmi”. %\hfill\textcolor{gray}{\footnotesize 4}

\subsection\*{\textbf{1088} {\footnotesize 〔PTS 1081〕}}

\textbf{「凡是这些沙门、婆罗门,」尊者难陀说,「以见闻而说清净,\\}
\textbf{「以戒禁而说清净,以多种方式而说清净,\\}
\textbf{「牟尼!如果你说无法度过暴流,那么在天与人的世间,谁能\\}
\textbf{「得度生与老?先生!我问您,世尊!请对我说说这个!」}

“Ye kec’ime samaṇabrāhmaṇāse, \textit{(icc āyasmā Nando)} diṭṭhassutenāpi vadanti suddhiṃ;\\
sīlabbatenāpi vadanti suddhiṃ, anekarūpena vadanti suddhiṃ;\\
te ce Muni brūsi anoghatiṇṇe, atha ko carahi devamanussaloke;\\
atāri jātiñ ca jarañ ca Mārisa, pucchāmi taṃ Bhagavā brūhi me taṃ”. %\hfill\textcolor{gray}{\footnotesize 5}

\subsection\*{\textbf{1089} {\footnotesize 〔PTS 1082〕}}

\textbf{「我不说所有的沙门、婆罗门,难陀!」世尊说,「为生与老所覆蔽,\\}
\textbf{「那些于此舍弃了所有的所见、所闻、所觉、或是戒禁,\\}
\textbf{「舍弃了所有种种方式,遍知了渴爱而无漏者,\\}
\textbf{「我说这些人已度过暴流。」}

“Nāhaṃ sabbe samaṇabrāhmaṇāse, \textit{(Nandā ti Bhagavā)} jātijarāya nivutā ti brūmi;\\
ye s’īdha diṭṭhaṃ va sutaṃ mutaṃ vā, sīlabbataṃ vā pi pahāya sabbaṃ;\\
anekarūpam pi pahāya sabbaṃ, taṇhaṃ pariññāya anāsavāse;\\
te ve narā oghatiṇṇā ti brūmi”. %\hfill\textcolor{gray}{\footnotesize 6}

\begin{enumerate}\item \textbf{遍知了渴爱},即以三遍知了渴爱。\end{enumerate}

\subsection\*{\textbf{1090} {\footnotesize 〔PTS 1083〕}}

\textbf{「我欢喜大仙的这番话语,乔达摩!无所依是善说,\\}
\textbf{「那些于此舍弃了所有的所见、所闻、所觉、或是戒禁,\\}
\textbf{「舍弃了所有种种方式,遍知了渴爱而无漏者,\\}
\textbf{「我也说这些人已度过暴流。」}

“Etābhinandāmi vaco mahesino, sukittitaṃ Gotam’anūpadhīkaṃ;\\
ye s’īdha diṭṭhaṃ va sutaṃ mutaṃ vā, sīlabbataṃ vā pi pahāya sabbaṃ;\\
anekarūpam pi pahāya sabbaṃ, taṇhaṃ pariññāya anāsavāse;\\
aham pi te oghatiṇṇā ti brūmī” ti. %\hfill\textcolor{gray}{\footnotesize 7}

\begin{enumerate}\item 如是,世尊同样以阿罗汉为顶点开示了此经。当开示终了,难陀欢喜于世尊的所说,而说了此颂。于此,与先前所说的一样,而有法的现观。\end{enumerate}

\begin{center}\vspace{1em}难陀学童问第七\\Nandamāṇavapucchā sattamā.\end{center}