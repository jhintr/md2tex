\section{净洗学童问}

\subsection\*{\textbf{1068} {\footnotesize 〔PTS 1061〕}}

\textbf{「我问你,世尊!请对我说说这个!」尊者净洗说,「我期待你的话语,大仙!\\}
\textbf{「听了你的教训,我将为自己修学涅槃。」}

\begin{enumerate}\item 这里,\textbf{我将为自己修学涅槃},即我将为了自己的贪等的止息修学增上戒等。\end{enumerate}

\subsection\*{\textbf{1069} {\footnotesize 〔PTS 1062〕}}

\textbf{「那么,请提起热忱!净洗!」世尊说,「于此贤明、具念,\\}
\textbf{「从此听取声明,你将为自己修学涅槃。」}

\begin{enumerate}\item \textbf{从此},即从我面前。\end{enumerate}

\subsection\*{\textbf{1070} {\footnotesize 〔PTS 1063〕}}

\textbf{「在天与人的世间,我看见无所牵绊而行止的婆罗门,\\}
\textbf{「我礼敬您,一切眼者!请脱我出诸疑惑!释迦!」}

\begin{enumerate}\item 如是说已,净洗心满意足,称赏着世尊,请求解脱疑惑,说了此颂。\end{enumerate}

\subsection\*{\textbf{1071} {\footnotesize 〔PTS 1064〕}}

\textbf{「我不能让世间任何有疑惑者解脱,净洗!\\}
\textbf{「然而,证知殊胜的法,如是,你便能度过这暴流。」}

\begin{enumerate}\item 于是,世尊以度过暴流为首,为显示唯有依靠自己者才能解脱疑惑,说了此颂。\end{enumerate}

\subsection\*{\textbf{1072} {\footnotesize 〔PTS 1065〕}}

\textbf{「带着悲悯,梵天!请教训我所能了知的远离之法!\\}
\textbf{「犹如虚空不被妨碍,我将于此寂静、无所依而行。」}

\begin{enumerate}\item 如是说已,净洗更加满意,称赏着世尊,为请求教训,说了此颂。这里,\textbf{梵天},即最胜之语,他以此召唤世尊,说道:「梵天!请教训!」\textbf{远离之法},即一切诸行的远离涅槃法。\textbf{不被妨碍},即不被种种品类所碍。\textbf{于此寂静},即于此存在\footnote{义注将「寂静 \textit{santo}」视为现在分词「存在 \textit{samāno}」。}。\textbf{无所依},即无依止。\end{enumerate}

\subsection\*{\textbf{1073} {\footnotesize 〔PTS 1066〕}}

\textbf{「我将对你宣说寂静,净洗!」世尊说,「所见之法,而非传闻,\\}
\textbf{「了知此已,具念而行,便能度过世间的爱著。」}

\begin{enumerate}\item 此后的二颂如「慈达经\footnote{即第 1060~1061 颂。}」中所说,差别处只是彼处的「法」在这里作「寂静」。\end{enumerate}

\subsection\*{\textbf{1074} {\footnotesize 〔PTS 1067〕}}

\textbf{「而我欢喜这无上寂静,大仙!\\}
\textbf{「了知此已,具念而行,便能度过世间的爱著。」}

\subsection\*{\textbf{1075} {\footnotesize 〔PTS 1068〕}}

\textbf{「凡是你所知的,净洗!」世尊说,「上方、下方、四旁及中间,\\}
\textbf{「了知了这在世间是染著,切莫对有与无有起渴爱!」}

\begin{enumerate}\item 第三颂中,前半颂也如彼处\footnote{即第 1062 颂。}所说,后半颂中,\textbf{染著},即羁绊处,即是说固著。其余一切处皆自明。
\item 如是,世尊仍以阿罗汉为顶点开示了此经。当开示终了,如前所述,而有法的现观。\end{enumerate}

\begin{center}净洗学童问第五\end{center}