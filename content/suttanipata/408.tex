\section{般修罗经}

\begin{center}Pasūra Sutta\end{center}\vspace{1em}

\begin{enumerate}\item 缘起为何?据说,当世尊住舍卫国时,有个名叫般修罗的游行者,是大论师,他说「我凭论议在整个阎浮提中为上首,所以,正如阎浮树是阎浮提的标志,也应是我的标志」,竖起阎浮枝作为幡幢,在整个阎浮提未遭论难,渐次来到舍卫国,在城门聚起沙堆,立树枝于上,说完「若能与我共作论议,就请他折了这树枝」,就入了城。大众便围绕此地而立。此时,尊者舍利弗食事已毕,便离舍卫国而去。他看见这(树枝)后,就问一众村童:「孩子们!这是什么?」他们就告知了一切。他说「那么,你们取下它来,用脚折断,然后说『意欲论议者请去寺庙』」,便离去。
\item 游行者行乞后,食事已毕,回来看到被取下折断的树枝,便问:「这是谁干的?」当听到是「佛陀的弟子舍利弗」后即生欢喜,想「今天智者们将看见我的胜利和沙门的失败」,进入舍卫国,招来考量问题的判官,在路上、四衢街道、广场上巡行,布告说:「在与沙门乔达摩上首弟子的论议中,想要听闻智慧辩才的诸君请出来!」「我们将听闻智者的话语」,对教法净喜或未净喜的众人便出来。随后,般修罗为大众围绕,寻思着「当如是说时,我将如是答」等等,便到了寺庙。长老想「莫在寺庙内制造高声、大声和人群的骚乱」,便命人在祇园的门廊备好座位坐下。
\item 游行者去往长老处,便说:「出家先生!是您教人折了我的阎浮幡幢吗?」当听到「唯!游行者」后,便说:「先生!是否欲有所论?」在得到长老的答复「请!游行者」后,便说:「沙门!你问!我将作答。」随后,长老对他说:「游行者!什么比较难?提问还是回答?」「出家先生!是回答,提问有何难?因为任何人都能有所问。」「那么,游行者!你问!我将作答。」
\item 如是说已,游行者心生惊异「让人折断树枝的也许是个可敬的比丘」,便问长老:「什么是人的爱欲?」长老便说:「思惟之贪染是人的爱欲\footnote{\textbf{增支部}第 6:63 经。}。」他听后,对长老起敌对想,欲令他败,便说:「出家先生!你不说形形色色的所缘是人的爱欲吗?」「唯!游行者!我不说。」随后,游行者又让他确认了三次,对考量问题的判官们说「诸君!请听沙门论议中的过失」,说:「出家先生!你们的同梵行者是住在林野吗?」「唯!游行者!他们住(在林野)。」「他们住在那里,是否寻思欲寻等的寻?」「唯!游行者!凡夫或起寻思。」「如果这样,他们如何能成为沙门?难道他们不应该成为在家人,受用爱欲吗?」如是说已,更说颂道:\begin{quoting}世间的形形色色并非爱欲,你说爱欲是思惟之贪染,\\思惟不善寻的比丘,也将受用这些爱欲。\end{quoting}
\item 于是,长老为显示游行者论议中的过失,便说:「游行者!你不说思惟之贪染是人的爱欲,而说是形形色色的所缘吗?」「唯!出家先生!」随后,长老又让他确认了三次,对考量问题的判官们说「朋友!请听游行者论议中的过失」,说:「朋友般修罗!你有导师吗?」「唯!出家人!有。」「他看到眼所识的色所缘,或交接声等所缘吗?」「唯!出家人!他交接。」「如果这样,他如何能成为导师?难道他不应该成为在家人,受用爱欲吗?」如是说已,更说颂道:\begin{quoting}世间的形形色色即是爱欲,你不说爱欲是思惟之贪染,\\看到可意之色,听到可意之声,\\闻到可意之香,尝到可意之味,\\触到可意之触,导师也将受用这些爱欲。\end{quoting}
\item 如是说已,辩才顿失的游行者想「这出家人是大论师,我将在他跟前出家,修学论辩之术」,便入舍卫国寻求衣钵,进入祇园时,在那里看到罗优陀夷\footnote{罗优陀夷 \textit{Lāludāyī} 事,见\textbf{本生}、\textbf{法句}义注等。}金黄的肤色、具足身形、身体举止一切悦意,想「这比丘有大智慧,是大论师」,在他跟前出家后,以论议驳倒了他,唯以形相混迹教内,又和上次一样,在舍卫国内布告说「我将与沙门乔达摩共作论议」,为大众围绕,说着「我将如是驳倒沙门乔达摩」等等,来到祇园。
\item 住在祇园门廊的天人想「他不成器」,封了他的口。他前往世尊处,便像哑巴一样落坐。人们望着他的脸,想「现在他要问了、现在他要问了」,便发出高声、大声:「说吧!般修罗君!说吧!般修罗君!」于是,世尊说「般修罗要说什么」,便在此为了对到来的会众开示法,说了此经。\end{enumerate}

\subsection\*{\textbf{831} {\footnotesize 〔PTS 824〕}}

\textbf{他们说「唯于此清净」,说在其它的法中没有清净,\\}
\textbf{凡所依止,即此说是净,个个执著于各自的真实中。}

“Idh’eva suddhī” iti vādayanti, nāññesu dhammesu visuddhim āhu;\\
yaṃ nissitā tattha subhaṃ vadānā, paccekasaccesu puthū niviṭṭhā. %\hfill\textcolor{gray}{\footnotesize 1}

\begin{enumerate}\item 这里,先说第一颂的略义:这些成见者就自己的见,\textbf{说「唯于此清净」,说在其它的法中没有清净}。当如是时,便唯对\textbf{所依止}的自己的导师等,\textbf{即此}以「此论清净」等\textbf{说是净},\textbf{个个}沙门婆罗门\textbf{执著于}「世间是常」等\textbf{各自的真实中}。\end{enumerate}

\subsection\*{\textbf{832} {\footnotesize 〔PTS 825〕}}

\textbf{他们渴求论议,投入集会后,互相认定对方是愚人,\\}
\textbf{他们依止不同,展开议论,渴求赞赏,自称是善巧。}

Te vādakāmā parisaṃ vigayha, bālaṃ dahantī mithu aññamaññaṃ;\\
vadanti te aññasitā kathojjaṃ, pasaṃsakāmā kusalā vadānā. %\hfill\textcolor{gray}{\footnotesize 2}

\begin{enumerate}\item 如是执著者,「他们渴求论议……」。这里,\textbf{互相认定对方是愚人},即二人如是以「这是愚人、这是愚人」认定彼此为愚人,视如愚人。\textbf{他们依止不同,展开议论},即他们依止各自的大师等,进行争辩。\textbf{渴求赞赏,自称是善巧},即希求赞赏,双方都作如是想「我们是善巧论者、有智论者」。\end{enumerate}

\subsection\*{\textbf{833} {\footnotesize 〔PTS 826〕}}

\textbf{在集会中进行论述,希望赞赏,变得焦虑,\\}
\textbf{而受驳斥便生愧畏,他因责备而被激怒,寻找破绽。}

Yutto kathāyaṃ parisāya majjhe, pasaṃsam icchaṃ vinighāti hoti;\\
apāhatasmiṃ pana maṅku hoti, nindāya so kuppati randham esī. %\hfill\textcolor{gray}{\footnotesize 3}

\begin{enumerate}\item 当他们如是自称时,每人必然「在集会中……」。这里,\textbf{进行论述},即热衷于争论之说。\textbf{希望赞赏,变得焦虑},即希望赞赏自己,以「我将如何折服」等方式在会话之前便疑惑、焦虑。\textbf{受驳斥},即论说被判官们以「你的所说无义、你的所说无文」等方式驳斥。\textbf{他因责备而被激怒},即当论说受到如是驳斥时,他因生起的责备而被激怒。\textbf{寻找破绽},即唯寻觅他人的破绽。\end{enumerate}

\subsection\*{\textbf{834} {\footnotesize 〔PTS 827〕}}

\textbf{当判官们说他的论议缺损、被驳斥,\\}
\textbf{论败者便生悲忧,叹泣「他超过了我」。}

Yam assa vādaṃ parihīnam āhu, apāhataṃ pañhavimaṃsakāse;\\
paridevati socati hīnavādo, “upaccagā man” ti anutthunāti. %\hfill\textcolor{gray}{\footnotesize 4}

\begin{enumerate}\item 不仅被激怒,而且「当判官们……」。这里,\textbf{说……缺损、被驳斥},即说在义、文上被驳斥、缺损。\textbf{悲},即基于此,他以「我要转向其它」等悲叹。\textbf{忧},即就「胜利是他的」等忧伤。\textbf{叹泣「他超过了我」},即以「他以论议超越了我」等方式愈加悲叹。\end{enumerate}

\subsection\*{\textbf{835} {\footnotesize 〔PTS 828〕}}

\textbf{于沙门众中生起的这些争论,其间有胜有负,\\}
\textbf{见到此后,应远离议论,因为除了赞赏利养,无他义利。}

Ete vivādā samaṇesu jātā, etesu ugghāti nighāti hoti;\\
etam pi disvā virame kathojjaṃ, na h’aññadatth’atthi pasaṃsalābhā. %\hfill\textcolor{gray}{\footnotesize 5}

\begin{enumerate}\item 而「于沙门众中的这些争论」中,外道游行者被称为\textbf{沙门众}。\textbf{其间有胜有负},即在这些论议中,依胜败等,当成就心的扬抑时,而成胜者、负者。\textbf{应远离议论},即应舍弃争辩。\textbf{因为除了赞赏利养,无他义利},即因为此中除了赞赏利养,并无其他义利。\end{enumerate}

\subsection\*{\textbf{836} {\footnotesize 〔PTS 829〕}}

\textbf{但或在集会之中发表论议后,于此受到赞赏,\\}
\textbf{他因此喜笑且高兴,达成了义利,如其心意。}

Pasaṃsito vā pana tattha hoti, akkhāya vādaṃ parisāya majjhe;\\
so hassatī unnamatī ca tena, pappuyya tam atthaṃ yathāmano ahu. %\hfill\textcolor{gray}{\footnotesize 6}

\begin{enumerate}\item 第六颂之义为:且因为除了赞赏利养,无他义利,所以,为得到最高的利养,\textbf{在集会之中}显明这\textbf{论议后},\textbf{于此}以见\textbf{受到赞赏}「此是善妙」,随后,\textbf{他因此}得胜的义利满足、露齿而\textbf{喜笑},\textbf{且}因慢而\textbf{高兴}。什么原因?因为\textbf{达成了}得胜的\textbf{义利},\textbf{如其心意}而生。\end{enumerate}

\subsection\*{\textbf{837} {\footnotesize 〔PTS 830〕}}

\textbf{这高兴便是其困扰之地,但他仍慢与傲慢地在论说,\\}
\textbf{见到此后,不应争论,因为善人们说不由此而清净。}

Yā unnatī sāssa vighātabhūmi, mānātimānaṃ vadate pan’eso;\\
etam pi disvā na vivādayetha, na hi tena suddhiṃ kusalā vadanti. %\hfill\textcolor{gray}{\footnotesize 7}

\begin{enumerate}\item 且对如是高兴者,「这高兴便是……」。这里,\textbf{但他仍慢与傲慢地在论说},即他未觉知到这高兴是「困扰之地」,仍慢且傲慢地在论说。\end{enumerate}

\subsection\*{\textbf{838} {\footnotesize 〔PTS 831〕}}

\textbf{好比以王家之食供给的英雄,咆哮而来,渴望敌手,\\}
\textbf{奔赴他的所在!英雄!先前就已没有了可战斗之事。}

Sūro yathā rājakhādāya puṭṭho, abhigajjam eti paṭisūram icchaṃ;\\
yen’eva so tena palehi sūra, pubbe va natthi yad idaṃ yudhāya. %\hfill\textcolor{gray}{\footnotesize 8}

\begin{enumerate}\item 如是显示了论议中的过失,现在,为不领受此论议,说了此颂。这里,\textbf{王家之食},即是说饮食之费。\textbf{咆哮而来,渴望敌手},即显示好比他渴望敌手,咆哮而来,成见者对成见者也如是。\textbf{奔赴他的所在},即朝着你的敌手所在而去。\textbf{先前就已没有了可战斗之事},然而,这可为之战斗的种种烦恼,先前就已没有了,即显示早在菩提树下已被舍弃了。\end{enumerate}

\subsection\*{\textbf{839} {\footnotesize 〔PTS 832〕}}

\textbf{若执取见而争论,并说道「唯此真实」,\\}
\textbf{你应对他们说「于此生起的论议中,并无你的敌军」。}

Ye diṭṭhim uggayha vivādayanti, “idam eva saccan” ti ca vādayanti;\\
te tvaṃ vadassū “na hi te’dha atthi, vādamhi jāte paṭisenikattā”. %\hfill\textcolor{gray}{\footnotesize 9}

\begin{enumerate}\item 其余几颂的连结自明。这里,\textbf{敌军},即逆行者。\end{enumerate}

\subsection\*{\textbf{840} {\footnotesize 〔PTS 833〕}}

\textbf{然而,若他们消灭了敌军而行,不以见抵制见,\\}
\textbf{你能从其中得到什么?般修罗!无物在此被他们执取为最上。}

Visenikatvā pana ye caranti, diṭṭhīhi diṭṭhiṃ avirujjhamānā;\\
tesu tvaṃ kiṃ labhetho Pasūra, yes’īdha natthī param’uggahītaṃ. %\hfill\textcolor{gray}{\footnotesize 10}

\begin{enumerate}\item \textbf{消灭了敌军},即消除了烦恼之军。\textbf{你能得到什么},即你将得到什么对手?\textbf{般修罗},即称呼这游行者。\end{enumerate}

\subsection\*{\textbf{841} {\footnotesize 〔PTS 834〕}}

\textbf{现在,你又开始寻思,心中思索着成见,\\}
\textbf{你已与除遣者相遇,却仍不堪有所进益。}

Atha tvaṃ pavitakkam āgamā, manasā diṭṭhigatāni cintayanto;\\
dhonena yugaṃ samāgamā, na hi tvaṃ sakkhasi sampayātave ti. %\hfill\textcolor{gray}{\footnotesize 11}

\begin{enumerate}\item \textbf{寻思},即以「我是否会胜」等寻思。\textbf{已与除遣者相遇},即你已与除遣了烦恼的佛陀相遇。\textbf{却仍不堪有所进益},即如野干等遇到狮子一般,既与除遣者相遇,仍不能前进一步,或不能有所成就。其余一切处皆自明。\end{enumerate}

\begin{center}\vspace{1em}般修罗经第八\\Pasūrasuttaṃ aṭṭhamaṃ.\end{center}