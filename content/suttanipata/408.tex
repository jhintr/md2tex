\section{般修罗经}

\begin{center}Pasūra Sutta\end{center}\vspace{1em}

\begin{enumerate}\item 据说,当世尊住在舍卫国时,有个名叫般修罗的游行者,是大论师,他说「在整个阎浮提中,我由论议为上首,所以,正如阎浮树是阎浮提的标志,也应是我的(标志)」,竖起阎浮枝作为幡幢,在整个阎浮提未遭论难,渐次来到舍卫国,在城门聚起沙堆,立好树枝,说「若能与我共作论议,就请他折了这树枝」,便入了城。大众就围绕着此地站着。这时,尊者舍利弗食事已毕,便离舍卫国而去。他看见这(树枝)后,就问村童们「孩子们!这是什么」,他们就告知了一切。他说「既然如此,你们取下它来,用脚折断,然后说『意欲论议者请去寺庙』」,便离去。
\item 游行者行乞后,食事已毕,回来看到被取下折断的树枝,便问「这是谁干的」,当听到是「佛陀的弟子舍利弗」后即生欢喜,「今天智者们将看见我的胜利和沙门的失败」,进入舍卫国去招来考量问题的判官,在路上、街口、广场上游行,布告说「在与沙门乔达摩上首弟子的论议中,想要听闻智慧辩才的诸君请出来」。「我们将听闻智者的话语」,对教法净喜或未净喜的众人便出来。随后,般修罗为大众围绕,寻思着「若如是说,我将如是答」等,来到了寺庙。长老想「莫在寺庙内制造高声、大声和人群的骚乱」,便命人在祇园的门房备好座位坐下。
\item 游行者朝长老走去,说「出家人!是您教人折了我的阎浮幡幢吗」,当听到「游行者!是的」后,便说「先生!欲有所论」,得到长老的答复「游行者!请」后,说「沙门!你问,我来回答」。随后长老对他说:「游行者!什么比较难?提问还是回答?」「出家人!是回答,提问有何难?因为任何人都能有所问。」「既然如此,游行者!你问,我来回答。」如是说已,游行者惊异道「让人折断树枝的也许是个可敬的比丘」,便问长老:「什么是人的爱欲?」长老说:「思惟之贪染是人的爱欲。」他听后,对长老起敌对想,想要他败,说:「出家人!你不说形形色色的所缘是人的爱欲吗?」「是的,游行者!我不说。」随后,游行者又让他确认了三次,对考量问题的判官们说「诸君!请听沙门论议中的过失」,说:「出家人!你们的同梵行者们是住在林野吗?」「是的,游行者!他们住(在林野)。」「他们住在那里,是否寻思欲寻等的寻?」「是的,游行者!凡夫或起寻思。」「如果这样,他们如何能成为沙门?难道他们不应该成为在家人,受用爱欲吗?」如是说已,更说颂曰:「世间的形形色色并非爱欲,你说爱欲是思惟之贪染,思惟不善寻的比丘,也将受用这些爱欲。」
\item 于是,长老为显示游行者论议中的过失,说:「游行者!你不说思惟之贪染是人的爱欲,而说是形形色色的所缘吗?」「是的,出家人!」随后,长老又让他确认了三次,对考量问题的判官们说「朋友!请听游行者论议中的过失」,说:「朋友般修罗!你有导师吗?」「是的,出家人!有。」「他看到眼所识的色所缘,或交接声等所缘吗?」「是的,出家人!他交接。」「如果这样,他如何能成为导师?难道他不应该成为在家人,受用爱欲吗?」如是说已,更说颂曰:「世间的形形色色即是爱欲,你不说爱欲是思惟之贪染,当看到可意之色,听到可意之声,闻到可意之香,尝到可意之味,触到可意之触,导师也将受用这些爱欲。」
\item 如是说已,辩才顿失的游行者想「这出家人是大论师,在他跟前出家后,我将修学论辩之术」,便入舍卫国寻求衣钵,再入祇园,看到 Lāludāyī 金色的肤色、具足身形、身体举止一切悦意,想「这比丘有大智慧,是大论师」,在他跟前出家已,以论议驳倒了他,唯以形相混迹教内,又和上次一样,在舍卫国内布告说「我将与沙门乔达摩共作论议」,为大众围绕,说着「我将如是驳倒沙门乔达摩」等,来到祇园。住在祇园门房的天人想「他不成器」,便封了他的口。他接近世尊后,像哑巴一样坐着。人们看着他的脸,想「现在他要提问了,现在他要提问了」,发出高声、大声「说吧!般修罗君!说吧!般修罗君」。于是,世尊说「般修罗要说什么」,于此为向到来的会众开示法而说此经。\end{enumerate}

\subsection\*{\textbf{831} {\footnotesize 〔PTS 824〕}}

\textbf{他们宣说「唯于此清净」,说在其它的法中没有清净,\\}
\textbf{凡所依者,即于此说是净,个个住立于各自的真实中。}

“Idh’eva suddhī” iti vādayanti, nāññesu dhammesu visuddhim āhu;\\
yaṃ nissitā tattha subhaṃ vadānā, paccekasaccesu puthū niviṭṭhā. %\hfill\textcolor{gray}{\footnotesize 1}

\subsection\*{\textbf{832} {\footnotesize 〔PTS 825〕}}

\textbf{他们爱好论议,投入集会后,互相认定对方是愚人,\\}
\textbf{他们所依不同,展开议论,渴求赞赏,自称是善巧。}

Te vādakāmā parisaṃ vigayha, bālaṃ dahantī mithu aññamaññaṃ;\\
vadanti te aññasitā kathojjaṃ, pasaṃsakāmā kusalā vadānā. %\hfill\textcolor{gray}{\footnotesize 2}

\subsection\*{\textbf{833} {\footnotesize 〔PTS 826〕}}

\textbf{在集会之中进行论述,希望获得赞赏而变得焦虑,\\}
\textbf{受到驳斥时便生愧畏,他因责备而恼怒,寻找着破绽。}

Yutto kathāyaṃ parisāya majjhe, pasaṃsam icchaṃ vinighāti hoti;\\
apāhatasmiṃ pana maṅku hoti, nindāya so kuppati randham esī. %\hfill\textcolor{gray}{\footnotesize 3}

\subsection\*{\textbf{834} {\footnotesize 〔PTS 827〕}}

\textbf{当考量问题的判官们说他的论议欠缺、被驳斥,\\}
\textbf{论败者则生悲忧,哀泣道「他超过了我」。}

Yam assa vādaṃ parihīnam āhu, apāhataṃ pañhavimaṃsakāse;\\
paridevati socati hīnavādo, “upaccagā man” ti anutthunāti. %\hfill\textcolor{gray}{\footnotesize 4}

\subsection\*{\textbf{835} {\footnotesize 〔PTS 828〕}}

\textbf{于沙门众中生起的这些争论,其间有胜有负,\\}
\textbf{见到此后,应远离议论,因为除了赞赏与利养,无他义利。}

Ete vivādā samaṇesu jātā, etesu ugghāti nighāti hoti;\\
etam pi disvā virame kathojjaṃ, na h’aññadatth’atthi pasaṃsalābhā. %\hfill\textcolor{gray}{\footnotesize 5}

\subsection\*{\textbf{836} {\footnotesize 〔PTS 829〕}}

\textbf{又或在集会之中发表论议后,于此受到赞赏,\\}
\textbf{他因而喜笑且高兴,如其心意,达成了义利。}

Pasaṃsito vā pana tattha hoti, akkhāya vādaṃ parisāya majjhe;\\
so hassatī unnamatī ca tena, pappuyya tam atthaṃ yathā mano ahu. %\hfill\textcolor{gray}{\footnotesize 6}

\subsection\*{\textbf{837} {\footnotesize 〔PTS 830〕}}

\textbf{高兴便是他恼害的根由,但他仍因慢与傲慢在论说,\\}
\textbf{见到此后,不应争论,因为善人们说并不能由此而清净。}

Yā unnatī sāssa vighātabhūmi, mānātimānaṃ vadate pan’eso;\\
etam pi disvā na vivādayetha, na hi tena suddhiṃ kusalā vadanti. %\hfill\textcolor{gray}{\footnotesize 7}

\subsection\*{\textbf{838} {\footnotesize 〔PTS 831〕}}

\textbf{好比国王供奉的英雄,渴望着敌手,咆哮而来,\\}
\textbf{奔赴他的所在,英雄!早先就已没有了可战斗之事。}

Sūro yathā rājakhādāya puṭṭho, abhigajjam eti paṭisūram icchaṃ;\\
yen’eva so tena palehi Sūra, pubbeva natthi yad-idaṃ yudhāya. %\hfill\textcolor{gray}{\footnotesize 8}

\begin{enumerate}\item \textbf{早先就已没有了可战斗之事},即这可战斗的烦恼所生之事,早先就已没有了,即在菩提树下已被舍弃了。\end{enumerate}

\subsection\*{\textbf{839} {\footnotesize 〔PTS 832〕}}

\textbf{若执持见而争论,宣说「唯此是真实」,\\}
\textbf{你应当对他们说「于此生起的论议中,并没有你的敌军」。}

Ye diṭṭhim uggayha vivādayanti, “idam eva saccan” ti ca vādayanti;\\
te tvaṃ vadassū “na hi te’dha atthi, vādamhi jāte paṭisenikattā”. %\hfill\textcolor{gray}{\footnotesize 9}

\subsection\*{\textbf{840} {\footnotesize 〔PTS 833〕}}

\textbf{然而,若他们消灭了敌军而行,不以见抵制见,\\}
\textbf{般修罗!你能从其中得到什么?对于他们,无物被执取为最上。}

Visenikatvā pana ye caranti, diṭṭhīhi diṭṭhiṃ avirujjhamānā;\\
tesu tvaṃ kiṃ labhetho Pasūra, yes’īdha natthī param’uggahītaṃ. %\hfill\textcolor{gray}{\footnotesize 10}

\subsection\*{\textbf{841} {\footnotesize 〔PTS 834〕}}

\textbf{现在,你又开始寻思,心中思索着众见,\\}
\textbf{你已遇到了除遣者,却仍不堪有所进益。}

Atha tvaṃ pavitakkam āgamā, manasā diṭṭhigatāni cintayanto;\\
dhonena yugaṃ samāgamā, na hi tvaṃ sakkhasi sampayātave ti. %\hfill\textcolor{gray}{\footnotesize 11}

\begin{center}\vspace{1em}般修罗经第八\\Pasūrasuttaṃ aṭṭhamaṃ.\end{center}