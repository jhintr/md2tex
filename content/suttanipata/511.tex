\section{劫波学童问}

\subsection\*{\textbf{1099} {\footnotesize 〔PTS 1092〕}}

\textbf{「当大怖畏的暴流生起,」尊者劫波说,「对佇立于中流者,\\}
\textbf{「对被老死征服者,请告知洲渚!先生!\\}
\textbf{「请您对我宣说洲渚!好让这没有更多。」}

\begin{enumerate}\item 这里,\textbf{中流},即是说由前后际无法识知而作为中间的轮回。\textbf{好让这没有更多},即好让这苦不再存在。\end{enumerate}

\subsection\*{\textbf{1100} {\footnotesize 〔PTS 1093〕}}

\textbf{「当大怖畏的暴流生起,劫波!」世尊说,「对佇立于中流者,\\}
\textbf{「对被老死征服者,我告知你洲渚,劫波!\\}

\begin{enumerate}\item 于是,世尊为对其解释此义,说了三颂。\end{enumerate}

\subsection\*{\textbf{1101} {\footnotesize 〔PTS 1094〕}}

\textbf{「无牵绊,无执取,这即是没有更多的洲渚,\\}
\textbf{「我说这即是涅槃,老死的灭尽。}

\begin{enumerate}\item 这里,\textbf{无牵绊},即对治牵绊,\textbf{无执取},即对治执取,即是说牵绊、执取的止息。\textbf{没有更多},即无有更多相似的洲渚,即是说最胜。\end{enumerate}

\subsection\*{\textbf{1102} {\footnotesize 〔PTS 1095〕}}

\textbf{「知晓此已,那些具念的现法寂灭者,\\}
\textbf{「他们不受制于魔罗,他们不侍奉于魔罗。」}

\begin{enumerate}\item \textbf{他们不侍奉于魔罗},即他们不是魔罗的奴仆、佣人、弟子。其余一切处皆自明。
\item 如是,世尊仍以阿罗汉为顶点开示了此经。当开示终了,与先前一样,而有法的现观。\end{enumerate}

\begin{center}劫波学童问第十\end{center}