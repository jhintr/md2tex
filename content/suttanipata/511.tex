\section{劫波学童问}

\begin{center}Kappa Māṇava Pucchā\end{center}\vspace{1em}

\subsection\*{\textbf{1099} {\footnotesize 〔PTS 1092〕}}

\textbf{「当大可畏的暴流生起,」尊者劫波说,「对佇立于中流者,\\}
\textbf{「对被老死征服者,请告知洲渚!先生!\\}
\textbf{「请您对我宣说洲渚!好让这没有更多。」}

“Majjhe sarasmiṃ tiṭṭhataṃ, \textit{(icc āyasmā Kappo)} oghe jāte mahabbhaye;\\
jarāmaccuparetānaṃ, dīpaṃ pabrūhi mārisa;\\
tvañ ca me dīpam akkhāhi, yathā-y-idaṃ nāparaṃ siyā”. %\hfill\textcolor{gray}{\footnotesize 1}

\begin{enumerate}\item \textbf{中流},即由前后际无法识知而作为中间的轮回。\textbf{好让这没有更多},即好让这苦不再存在。\end{enumerate}

\subsection\*{\textbf{1100} {\footnotesize 〔PTS 1093〕}}

\textbf{「当大可畏的暴流生起,劫波!」世尊说,「对佇立于中流者,\\}
\textbf{「对被老死征服者,我告知你洲渚,劫波!\\}

“Majjhe sarasmiṃ tiṭṭhataṃ, \textit{(Kappā ti Bhagavā)} oghe jāte mahabbhaye;\\
jarāmaccuparetānaṃ, dīpaṃ pabrūmi Kappa te. %\hfill\textcolor{gray}{\footnotesize 2}

\subsection\*{\textbf{1101} {\footnotesize 〔PTS 1094〕}}

\textbf{「无牵绊,无执取,这即是更无有别的洲渚,\\}
\textbf{「我说这即是涅槃,老死的灭尽。}

Akiñcanaṃ anādānaṃ, etaṃ dīpaṃ anāparaṃ;\\
nibbānaṃ iti naṃ brūmi, jarāmaccu-parikkhayaṃ. %\hfill\textcolor{gray}{\footnotesize 3}

\begin{enumerate}\item \textbf{更无有别},即没有其它相似的、最胜的意思。\end{enumerate}

\subsection\*{\textbf{1102} {\footnotesize 〔PTS 1095〕}}

\textbf{「知晓此已,具念的现法寂灭者,\\}
\textbf{「他们不受制于魔罗,他们不侍奉于魔罗。」}

Etad-aññāya ye satā, diṭṭhadhammābhinibbutā;\\
na te Māra-vasānugā, na te Mārassa paddhagū” ti. %\hfill\textcolor{gray}{\footnotesize 4}

\begin{enumerate}\item \textbf{他们不侍奉于魔罗},即他们不是魔罗的奴仆、用人、弟子。
\item 如是,世尊同样以阿罗汉为顶点开示了此经。当开示终了,与先前一样,而有法的现观。\end{enumerate}

\begin{center}\vspace{1em}劫波学童问第十\\Kappamāṇavapucchā dasamā.\end{center}