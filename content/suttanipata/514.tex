\section{生起学童问}

\subsection\*{\textbf{1112} {\footnotesize 〔PTS 1105〕}}

\textbf{「离尘、安坐的禅修者,」尊者生起说,「应作已作的无漏者,\\}
\textbf{「已度一切法者,我带着问题前来,\\}
\textbf{「请告知破碎无明的了知解脱!」}

\begin{enumerate}\item 这里,\textbf{了知解脱},即是问以慧的威力所生的解脱。\end{enumerate}

\subsection\*{\textbf{1113} {\footnotesize 〔PTS 1106〕}}

\textbf{「舍弃欲贪,生起!」世尊说,「与忧虑两者,\\}
\textbf{「除去昏沉,防止恶作,}

\begin{enumerate}\item 于是,因为生起是得第四禅者,所以世尊以他所得的第四禅,为从种种品类显示了知解脱,说了二颂。这里,\textbf{舍弃欲贪},即转起初禅者的舍弃欲贪,我说这也是了知解脱。所有句子当如是连接。\end{enumerate}

\subsection\*{\textbf{1114} {\footnotesize 〔PTS 1107〕}}

\textbf{「舍念清净,以法寻为前行,\\}
\textbf{「我说即是破碎无明的了知解脱。」}

\begin{enumerate}\item \textbf{舍念清净},即以第四禅的舍与念而清净。\textbf{以法寻为前行},即说,以此住于此第四禅解脱已,于禅支修观,证得阿罗汉解脱。因为与道相应的正思惟等类的法寻是阿罗汉解脱的前行,因此说「以法寻为前行」。\textbf{破碎无明},即此了知解脱由依于被称为破碎无明的涅槃而生故,以其近于原因,我说即是「破碎无明」。\end{enumerate}

\subsection\*{\textbf{1115} {\footnotesize 〔PTS 1108〕}}

\textbf{「什么是世间的结缚?什么是它的运作?\\}
\textbf{「因舍弃什么,而被称为涅槃?」}

\begin{enumerate}\item 如是,听到已破碎无明所说的涅槃,为问「因舍弃什么而得称」,说了此颂。这里,\textbf{运作},即运行的原因。\textbf{因舍弃什么},即因因舍弃什么名称的法。\end{enumerate}

\subsection\*{\textbf{1116} {\footnotesize 〔PTS 1109〕}}

\textbf{「欢喜是世间的结缚,寻是它的运作,\\}
\textbf{「因舍弃渴爱,而被称为涅槃。」}

\begin{enumerate}\item 于是,世尊为对其解答此义,说了此颂。这里,\textbf{寻},即欲寻等的寻。\end{enumerate}

\subsection\*{\textbf{1117} {\footnotesize 〔PTS 1110〕}}

\textbf{「对具念而行者,识如何灭去?\\}
\textbf{「我们前来问世尊,愿听你的话语!」}

\begin{enumerate}\item 现在,为问此涅槃之道,说了此颂。这里,\textbf{识},即行作识。\end{enumerate}

\subsection\*{\textbf{1118} {\footnotesize 〔PTS 1111〕}}

\textbf{「对于不欢喜内与外之受、\\}
\textbf{「如是具念而行者,识即灭去。」}

\begin{enumerate}\item 于是,世尊为对其论述道,说了此颂。这里,\textbf{如是具念},即如是具念正知。其余一切处皆自明。
\item 如是,世尊仍以阿罗汉为顶点开示了此经。当开示终了,与先前一样,而有法的现观。\end{enumerate}

\begin{center}生起学童问第十三\end{center}