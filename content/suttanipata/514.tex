\section{生起学童问}

\begin{center}Udaya Māṇava Pucchā\end{center}\vspace{1em}

\subsection\*{\textbf{1112} {\footnotesize 〔PTS 1105〕}}

\textbf{「离尘、安坐的禅修者、」尊者生起说,「应作已作的无漏者,\\}
\textbf{「已度一切法者,我带着问题前来,\\}
\textbf{「请告知破碎无明的了知解脱!」}

“Jhāyiṃ virajam āsīnaṃ, \textit{(icc āyasmā Udayo)} katakiccaṃ anāsavaṃ;\\
pāraguṃ sabbadhammānaṃ, atthi pañhena āgamaṃ;\\
aññāvimokkhaṃ pabrūhi, avijjāya pabhedanaṃ”. %\hfill\textcolor{gray}{\footnotesize 1}

\begin{enumerate}\item \textbf{了知解脱},即以慧的威力而生的解脱。\end{enumerate}

\begin{itemize}\item 案,此颂的前三句都是\textbf{前来}的宾语,译文未调整原文的语序,而是随原文将宾语前置。\end{itemize}

\subsection\*{\textbf{1113} {\footnotesize 〔PTS 1106〕}}

\textbf{「舍弃欲贪,生起!」世尊说,「与忧虑两者,\\}
\textbf{「除去昏沉,防止恶作,}

“Pahānaṃ kāmacchandānaṃ, \textit{(Udayā ti Bhagavā)} domanassāna cūbhayaṃ;\\
thinassa ca panūdanaṃ, kukkuccānaṃ nivāraṇaṃ. %\hfill\textcolor{gray}{\footnotesize 2}

\begin{enumerate}\item 于是,因为生起是得第四禅者,所以世尊以他所得的第四禅,从种种行相显明了知解脱,而说了二颂。\textbf{舍弃欲贪},即生起初禅者的舍弃欲贪,我说这也是了知解脱。如是(我说即是……了知解脱)当与所有句子连接。\end{enumerate}

\subsection\*{\textbf{1114} {\footnotesize 〔PTS 1107〕}}

\textbf{「舍念清净,以法寻为前行,\\}
\textbf{「我说即是破碎无明的了知解脱。」}

Upekkhā-sati-saṃsuddhaṃ, dhammatakkapurejavaṃ;\\
aññāvimokkhaṃ pabrūmi, avijjāya pabhedanaṃ”. %\hfill\textcolor{gray}{\footnotesize 3}

\begin{enumerate}\item \textbf{舍念清净},即以第四禅的舍与念而清净。\textbf{以法寻为前行},以此而说住于此第四禅解脱已,于禅支修观而证得阿罗汉解脱,因为对于阿罗汉解脱,与道相应的正思惟等的法寻是前行,故说「以法寻为前行」。\end{enumerate}

\subsection\*{\textbf{1115} {\footnotesize 〔PTS 1108〕}}

\textbf{「什么是世间的结缚?什么是它的运作?\\}
\textbf{「由舍弃什么,而被称为涅槃?」}

“Kiṃ su saṃyojano loko, kiṃ su tassa vicāraṇaṃ;\\
kiss’assa vippahānena, nibbānaṃ iti vuccati”. %\hfill\textcolor{gray}{\footnotesize 4}

\begin{enumerate}\item \textbf{运作},即运行的原因。\end{enumerate}

\begin{itemize}\item 案,\textbf{运作} \textit{vicāraṇaṃ},词典作「调查、搜索、注意」与「安排、计划」等,Norman 作「调查 \textit{investigation}」,菩提比丘英译作 means of traveling about,是基于义注的翻译。\end{itemize}

\subsection\*{\textbf{1116} {\footnotesize 〔PTS 1109〕}}

\textbf{「世间以欢喜为结缚,寻是它的运作,\\}
\textbf{「由舍弃渴爱,而被称为涅槃。」}

“Nandisaṃyojano loko, vitakk’assa vicāraṇaṃ;\\
taṇhāya vippahānena, nibbānaṃ iti vuccati”. %\hfill\textcolor{gray}{\footnotesize 5}

\begin{enumerate}\item \textbf{寻},即欲寻等。\end{enumerate}

\subsection\*{\textbf{1117} {\footnotesize 〔PTS 1110〕}}

\textbf{「对于具念而行者,识如何灭去?\\}
\textbf{「我们前来询问世尊,愿听你的话语!」}

“Kathaṃ satassa carato, viññāṇaṃ uparujjhati;\\
bhagavantaṃ puṭṭhum āgamma, taṃ suṇoma vaco tava”. %\hfill\textcolor{gray}{\footnotesize 6}

\begin{enumerate}\item \textbf{识},即行作识。\end{enumerate}

\subsection\*{\textbf{1118} {\footnotesize 〔PTS 1111〕}}

\textbf{「对于不欢喜内与外的受的\\}
\textbf{「具念而行者,如是,识即灭去。」}

“Ajjhattañ ca bahiddhā ca, vedanaṃ nābhinandato;\\
evaṃ satassa carato, viññāṇaṃ uparujjhatī” ti. %\hfill\textcolor{gray}{\footnotesize 7}

\begin{enumerate}\item 如是,世尊同样以阿罗汉为顶点开示了此经。当开示终了,与先前一样,而有法的现观。\end{enumerate}

\begin{center}\vspace{1em}生起学童问第十三\\Udayamāṇavapucchā terasamā.\end{center}