\section{可教学童问}

\begin{center}Todeyya Māṇava Pucchā\end{center}\vspace{1em}

\subsection\*{\textbf{1095} {\footnotesize 〔PTS 1088〕}}

\textbf{「若爱欲不居于他,」尊者可教说,「他也没有渴爱,\\}
\textbf{「且已度脱疑惑,则他的解脱是怎样的?」}

“Yasmiṃ kāmā na vasanti, \textit{(icc āyasmā Todeyyo)} taṇhā yassa na vijjati;\\
kathaṅkathā ca yo tiṇṇo, vimokkho tassa kīdiso”. %\hfill\textcolor{gray}{\footnotesize 1}

\subsection\*{\textbf{1096} {\footnotesize 〔PTS 1089〕}}

\textbf{「若爱欲不居于他,可教!」世尊说,「他也没有渴爱,\\}
\textbf{「且已度脱疑惑,则他已无更多的解脱。」}

“Yasmiṃ kāmā na vasanti, \textit{(Todeyyā ti Bhagavā)} taṇhā yassa na vijjati;\\
kathaṅkathā ca yo tiṇṇo, vimokkho tassa nāparo”. %\hfill\textcolor{gray}{\footnotesize 2}

\begin{enumerate}\item \textbf{已无更多的解脱},即没有其它的解脱。\end{enumerate}

\subsection\*{\textbf{1097} {\footnotesize 〔PTS 1090〕}}

\textbf{「他是无欲还是有所希求?他具有智慧,还是作智慧想?\\}
\textbf{「释迦!请对我说明牟尼!好让我能了知他,一切眼者!」}

“Nirāsaso so uda āsasāno, paññāṇavā so uda paññakappī;\\
muniṃ ahaṃ Sakka yathā vijaññaṃ, taṃ me viyācikkha Samantacakkhu”. %\hfill\textcolor{gray}{\footnotesize 3}

\begin{enumerate}\item 如是,当说「爱尽即是解脱」时,未能了解其义,而以此颂再次发问。这里,\textbf{作智慧想},即以等至之智等的智起爱想或见想。\end{enumerate}

\begin{itemize}\item 案,\textbf{作智慧想} \textit{paññakappī},菩提比丘英译作 just a wise manner,并注云 \textit{°kappī} 作后缀时有「相似」之意,如「像犀牛角一样 \textit{khaggavisāṇakappo}」,在这里更贴切。这里的汉译仍从义注,录其文以备一说。\end{itemize}

\subsection\*{\textbf{1098} {\footnotesize 〔PTS 1091〕}}

\textbf{「他无欲且无所希求,他具有智慧,不作智慧想,\\}
\textbf{「如是,可教!应知牟尼无所牵绊、不取著爱欲与有!」}

“Nirāsaso so na ca āsasāno, paññāṇavā so na ca paññakappī;\\
evam pi Todeyya muniṃ vijāna, akiñcanaṃ kāmabhave asattan” ti. %\hfill\textcolor{gray}{\footnotesize 4}

\begin{enumerate}\item 如是,世尊同样以阿罗汉为顶点开示了此经。当开示终了,与先前一样,而有法的现观。\end{enumerate}

\begin{center}\vspace{1em}可教学童问第九\\Todeyyamāṇavapucchā navamā.\end{center}