\section{施罗经}

\begin{center}Sela Sutta\end{center}\vspace{1em}

\textbf{如是我闻\footnote{此经即\textbf{中部}第 92 经,旧译见增壹阿含经卷第四十六·放牛品第四十九·六。施罗 \textit{Sela}、翅宁 \textit{Keṇiya} 都从旧译。}。一时世尊在鸯伽北水游行,与大比丘僧团千二百五十比丘俱,到了名为市场的鸯伽北水的镇子。}

Evaṃ me sutaṃ— ekaṃ samayaṃ Bhagavā Aṅguttarāpesu cārikaṃ caramāno mahatā bhikkhusaṅghena saddhiṃ aḍḍhateḷasehi bhikkhusatehi yena Āpaṇaṃ nāma Aṅguttarāpānaṃ nigamo tad avasari.

\begin{enumerate}\item 缘起为何?这在其因缘中已述。而在释义的过程中,其与先前相同者当知仍按先前所述之法。先前未述者,我们则避开意义自明之词,再作解释。
\item \textbf{鸯伽北水},鸯伽本身是一个国家,而由离大河以北的水不远故,被称为「北水」。这水在哪条大河以北?摩诃摩醯河。这里,为说明此河,从头开始解释。据说,阎浮提量有一万由旬,其中有四千由旬的地方为水淹没,成为「海洋」,三千由旬内由人类居住,三千由旬内雪山住立,高五百由旬,严以八万四千峰,周围流淌着五百条各色的河流。那里有长宽深各五十由旬、周长一百五十由旬的阿耨达等七大池,如「祭饼经(第 472 颂)注」中所说。
\item 其中,阿耨达池为善见峰、多彩峰、黑暗峰、香醉峰、冈仁波齐峰等五山围绕。于此,善见峰由金子所成,高二百由旬,内曲如鸦喙形,蔽覆此池而立,多彩峰由一切珍宝所成,黑暗峰由黑漆所成,香醉峰由高原所成,内如豆色,长满种种药草,在黑月的布萨日,如燃烧的炭火而立,冈仁波齐峰由银子所成,一切高度、形状都如善见峰,蔽覆此池而立,所有山峰以天、龙的威力降雨,且众河流经彼等。这所有的水都注入阿耨达池。日月在或南或北而行时,从山间照耀它,而在直行时却不照耀,以此得称「阿耨达」。于此,浴场善加修缮,有悦意的石板,无有鱼龟,水如水晶般无垢,佛、辟支佛、漏尽者与众仙人沐浴其中,诸天、夜叉等嬉戏其间。
\item 且在其四边,有狮口、象口、马口、牛口等四口,从中流出四河。从狮口流出的河边,狮子更多,从象口等流出的则象、马、牛更多。从东方流出的河右绕阿耨达三匝,避开其它三条河,沿雪山东方的无人之路而去,注入大海。从西方、北方流出的河也如是右绕,沿雪山西方、北方的无人之路而去,注入大海。
\item 从南方流出的河则右绕三匝后,向南直行,在石背上行了六十由旬,拍山而出后,形成了宽三牛呼之量的激流,在空中行了六十由旬,落在名为「三道闩」的岩石上,岩石被激流的急速所冲裂,就此形成五十由旬之量名为「三道闩」的池子。冲破池堤后,进入岩石行了六十由旬。随后,冲破坚硬的地面,沿隧道行了六十由旬,在撞击了名为温迪亚\footnote{温迪亚 \textit{Viñjha/Vindhya}:也作「文底耶」,即分隔南北印度的山脉。}的横亘山脉后,如掌上的五指一般,分成五股而转起。
\item 它右绕阿耨达三匝后所行之处被称为\textbf{回转河},直行而在石背上所行六十由旬之处被称为\textbf{黑河},在空中所行六十由旬之处被称为\textbf{天空河},在三道闩岩石上五十由旬之处被称为\textbf{三道闩池},冲破堤岸进入岩石所行六十由旬之处被称为\textbf{厚密河},冲破地面在隧道所行六十由旬之处被称为\textbf{隧道河},撞击了名为温迪亚的横亘山脉而成五股转起之处称为「恒伽、阎牟那、阿致罗筏底、萨罗由、摩醯」五者。如是,这五大河肇源于雪山,其中第五名为摩醯,即为此处的「摩诃摩醯河」。这水在此河以北,由离其不远故,这国家当知为「鸯伽北水」。即于此鸯伽北水国土。
\item \textbf{游行},即行于路途。这里,世尊有速游行、非速游行两种游行。于此,即便在远处,看到堪能之人而急速前往,为速游行,此在与大迦叶相遇等处可见。因为世尊与其相遇,瞬间便行了三牛呼,为调伏旷野则行了三十由旬,为了央掘魔罗也同样,而为弗迦逻娑利\footnote{弗迦逻娑利 \textit{Pukkusāti},见\textbf{中部}第 140 经,译名从中阿含经。}行了四十五由旬,为摩诃劫宾那行了二千由旬,为了有财者行了七百由旬的路途,这名为\textbf{速游行}。而以村、镇、城顺次乞食,摄受世间之行,名为\textbf{非速游行},即此处之所指。即如是游行。
\item \textbf{大},即数量大与功德大。\textbf{比丘僧团},即沙门众。\textbf{到了……镇子},因市场众多,这镇子得名\textbf{市场}。据说,在此分有两万个市场的出入口。或者,沿道路的方向,便能到达这\textbf{鸯伽北水}国土\textbf{的镇子},即是说到了此镇。\end{enumerate}

\textbf{萦发者翅宁听说:「真的,先生!沙门乔达摩、释迦子、从释迦族出家者,在鸯伽北水游行,与大比丘僧团千二百五十比丘俱,已到达市场,关于这乔达摩君,有如是的善称:『彼世尊亦即是阿罗汉、正等正觉者、明行足、善逝、世间解、无上士、调御丈夫、天人师、佛、世尊』,他由自己证知、证得了这俱有天、魔、梵、沙门婆罗门、天人的人世间后而宣说,他开示初中后善、有义有文的法,阐明完全圆满、遍净的梵行,善哉!得见这样的阿罗汉。」}

Assosi kho Keṇiyo jaṭilo: “samaṇo khalu, bho, Gotamo Sakyaputto Sakyakulā pabbajito Aṅguttarāpesu cārikaṃ caramāno mahatā bhikkhusaṅghena saddhiṃ aḍḍhateḷasehi bhikkhusatehi Āpaṇaṃ anuppatto, taṃ kho pana bhavantaṃ Gotamaṃ evaṃ kalyāṇo kittisaddo abbhuggato: ‘iti pi so Bhagavā arahaṃ sammāsambuddho vijjācaraṇasampanno sugato lokavidū anuttaro purisadammasārathi satthā devamanussānaṃ buddho bhagavā’ ti, so imaṃ lokaṃ sadevakaṃ samārakaṃ sabrahmakaṃ sassamaṇabrāhmaṇiṃ pajaṃ sadevamanussaṃ sayaṃ abhiññā sacchikatvā pavedeti, so dhammaṃ deseti ādikalyāṇaṃ majjhekalyāṇaṃ pariyosānakalyāṇaṃ sātthaṃ sabyañjanaṃ, kevala-paripuṇṇaṃ parisuddhaṃ brahmacariyaṃ pakāseti, sādhu kho pana tathārūpānaṃ arahataṃ dassanaṃ hotī” ti.

\begin{enumerate}\item \textbf{萦发者翅宁},翅宁为名,萦发即苦行者。据说他是富裕婆罗门,为守护财富而受持出家苦行,赠予国王礼物,分得土地后,在此建立草庵居住,成为一千户人家的依止。且人们说,在他的草庵,棕榈树每天结一个金色的果子。他白天穿袈裟并结发髻,晚上随其所乐,置身五欲而自娱。\textbf{释迦子},即显明家族高贵。\textbf{从释迦族出家者},即显明以信出家,即是说并非为衰败所征服,而是舍弃尚未败落的家族,以信出家。\textbf{关于这},即作为如是名称之义的业格,即属格之义。\textbf{善},即具足善的功德,即是说最胜。\textbf{称},即名称,或称赞之声。而在「\textbf{彼世尊亦即是……}」等中,先说连结:彼世尊亦即是阿罗汉、亦即是正等正觉者……亦即是世尊,且以各自的原由而说。
\item 这里,由回避故,由破贼及辐故,由应受资具等故,由无幽僻作恶故,先依这些原由,当知彼世尊亦即是\textbf{阿罗汉}。因为他回避一切烦恼,由以道摧破带有习气的烦恼故,即由回避故为阿罗汉。且以此道破彼等烦恼贼,即由破贼故为阿罗汉。且此无明与有爱所造之毂、福等行作之辐、老死之辋,以漏集所造之轴贯穿已,组合于三有之车上的无始之时转起的轮回之轮,以此(道)于菩提座以精进之双足站立于戒地上,以信之手举起业尽智之斧而破了一切辐,即由破辐故为阿罗汉。且由最上应供故,应受衣等资具,与恭敬、尊重等,即由应受资具等故为阿罗汉。且好比世间某些自认为是智者的愚人,因惧怕不名誉而于幽僻作恶,他则从不如是而行,即由无幽僻作恶故为阿罗汉。且此中,\begin{quoting}这牟尼由回避故,及破烦恼贼故,\\已破轮回的车辐,且应受资具等,\\不于幽僻作恶,因此称为阿罗汉。\end{quoting}
\item 完全且由自己觉悟真谛故,为\textbf{正等正觉者}。由具足极殊胜清净之明与超上之行故,为\textbf{明行足}。由洁净之行故、行至善妙之处故、善行故及正语故,为\textbf{善逝}。由从一切处了解世间故,为\textbf{世间解}。因为彼世尊从自性、集、灭、灭之方法等一切处了解了蕴、处等类的行世间,\begin{quoting}一世间,即一切有情依食而住。二世间,即名与色。三世间,即三受。四世间,即四食。五世间,即五取蕴。六世间,即六内处。七世间,即七识住。八世间,即八世间法。九世间,即九有情居。十世间,即十处。十二世间,即十二处。十八世间,即十八界。(无碍解道·大品第 112 段)\end{quoting}如是从一切处了解了行世间。知晓有情的意乐,知晓随眠,知晓性行,知晓信解,知晓少尘、多尘、利根、钝根、善行相、恶行相、易教化、难教化、有能、无能的有情,即从一切处了解了有情世间。
\item 同样,一轮围长宽为一百二十万三千四百五十由旬,周长三百六十一万三百五十。这里,\begin{quoting}二十万又四那由他\footnote{那由他 \textit{nahuta}:即一万。},\\这大地的厚数如许。\\四十万又八那由他,\\住立于风中之水的厚数如许。\\上升到虚空的风,有九十万\\又六万,此即是世间的安立。\end{quoting}且在如是安立中,就由旬而言,\begin{quoting}八万四千潜入大海,\\超出也如此,须弥为山中最高。\\以较之依次减半之量而\\潜入超出,圣洁、饰以种种珍宝的\\持双、持辕、迦陵频伽、善见、\\持辋、象鼻、马耳大山。\\这些七大岩山,在须弥的周围,\\是诸大王的住所,为天与夜叉沉迷。\\雪山高五百由旬,\\长、宽三千由旬,\\严以八万四千峰。\\树干周长十五由旬,名为那伽,\\树干枝长五十由旬,四周\\宽一百由旬,高也一样的\\阎浮树,以其辉煌,阎浮提便知名。\\八万二千潜入大海,\\超出也如此,轮围岩垒\\包围了这一切,此轮围住立。\end{quoting}这里,月轮四十九由旬,日轮五十由旬,三十三天的居处一万由旬,阿修罗的居处、无间大地狱与阎浮提也一样,西瞿耶尼七千由旬,东毗提诃也一样,北俱卢八千由旬。且一一大洲各有五百小洲围绕。这一切即一轮围、一世界。在轮围之间,是世间之间的地狱。如是无尽的轮围、无尽的世界,他以无尽的佛智知晓,即从一切处了解了器世间。如是,彼世尊由从一切处了解世间故,当知为世间解。
\item 而无有任何较自身的功德更殊胜者,为\textbf{无上士}。以多样的调伏方法引导可调御者,为\textbf{调御丈夫}。以现法、来世及第一义恰当地教授、度脱为\textbf{大师}。包含\textbf{天人}是因高贵的部分,且因摄受堪能之补特伽罗而为之,但他亦以世间的义利教授龙等。由以解脱尽智觉悟一切应知之法故,为\textbf{佛陀}。而因为他\begin{quoting}具吉祥,具破,与卓越相应,且具分别,\\具亲近,离弃于诸有而行,所以为世尊。\end{quoting}以上只是略说,而这些字词当如清净道论(说六随念品第 4~64 段)详说。
\item \textbf{他……这世间},即彼世尊……这世间。现在说明当说者。\textbf{俱有天}等,仍如耕田婆罗豆婆遮经、旷野经中所述。\textbf{由自己},即不由他人引领。\textbf{证得},即作为现量。\textbf{宣说},即令觉悟、令知晓、令明了。
\item \textbf{他开示初中后善的法},即彼世尊出于对有情的悲悯,甚至舍弃无上远离之乐而开示法。且在或多或少地开示时,唯开示初善等的品类。如何?因为即便是一颂,由法的总体贤善故,以第一句为初善,第二、第三句为中善,最后句为后善。对一节的经,以因缘为初善,以结论为后善,其余为中善。对多节者,以第一节为初善,以最后为后善,其余为中善。对整个教法,以作为自己义利的戒为初善,以止观道果为中善,以涅槃为后善。或者,以戒、定为初善,以观、道为中善,以果、涅槃为后善。或者,以佛的善觉性为初善,以法的善法性为中善,以僧的善行道为后善。或者,经听闻而如是行道,以于应证者现等觉为初善,以辟支菩提为中善,以声闻菩提为后善。且听闻者由镇伏盖等,仅以听闻即能带来的善为初善,行道者由带来止观之乐,以行道带来的善为中善,且如是行道者由在行道之果完成时带来如如的状态,以行道之果带来的善为后善。且由源于怙主,以发源的清净为初善,以义利的清净为中善,以作用的清净为后善。由此当知在或多或少地开示时,唯开示初善等的品类。
\item 而在\textbf{有义有文}\footnote{有义有文:旧译作「善义善味」。}等中,因为在开示此法时,他阐明教梵行与道梵行,以种种方法显示,分别以义利的成就为有义,以文字的成就为有文,由解释、阐明、揭示、区分、澄清概念与义句的结合为有义,由音节、词句、文字、语源、释义的成就为有文,以义利甚深、通达甚深为有义,以法甚深、开示甚深为有文,由义、辩无碍解之境为有义,由法、词无碍解之境为有文,由智者可知,相仿之人所净喜为有义,由可信,为世间之人所净喜为有文,由意趣甚深为有义,由文句彰明为有文。由无可增加,以全体圆满为\textbf{完全圆满},由无可减损,以无有过失为\textbf{遍净}\footnote{完全圆满、遍净:旧译作「纯一满净」。},由摄受三学,以作为梵的最胜者所应行且由彼等所行为\textbf{梵行}。所以说「他开示……的梵行」。
\item 复次,因为在开示时有因缘、有缘起,他开示初善,由随适于可调伏者、不颠倒义利及与因喻\footnote{因 \textit{hetu}、喻 \textit{udāharaṇa}:即因明三支中的二支。}结合,他开示中善,以听闻者得信及以结论,他开示后善。且当如是开示时,他阐明梵行。由以行道显示证得为有义,由以圣典显示阿含为有文,由相应于戒等五法蕴为完全圆满,由离随烦恼、为了超脱而转起、不希求世间利益为遍净,由以最胜之义而为梵的佛、辟支佛、声闻的所行为梵行,所以也说「他开示……的梵行」。
\item \textbf{善哉},即善妙,即是说带来义利、带来乐。\end{enumerate}

\textbf{于是,萦发者翅宁往世尊处走去,走到后,问候了世尊,彼此寒暄已,坐在一边。世尊便对坐在一边的萦发者翅宁以如法的言说显示、唤起、鼓舞、欢喜。}

Atha kho Keṇiyo jaṭilo yena Bhagavā ten’upasaṅkami, upasaṅkamitvā Bhagavatā saddhiṃ sammodi, sammodanīyaṃ kathaṃ sāraṇīyaṃ vītisāretvā ekamantaṃ nisīdi. Ekamantaṃ nisinnaṃ kho Keṇiyaṃ jaṭilaṃ Bhagavā dhammiyā kathāya sandassesi samādapesi samuttejesi sampahaṃsesi.

\begin{enumerate}\item \textbf{如法的言说},即与饮品的功德相关者。因为翅宁在晡时听闻世尊来到,羞于空手去见世尊,想到「虽然戒离非时食,但饮品是合适的」,便令人以五百扁担抬了善加准备的枣汤前往。一切当知如「药犍度」中所说:\begin{quoting}于是,萦发者翅宁想:我何不带给沙门乔达摩?(律藏·大品第 300 段)\end{quoting}随后,世尊如在有学经\footnote{即\textbf{中部}第 53 经。}中对释氏说与住处的功德相关者,在牛角娑罗林\footnote{即\textbf{中部}第 31 经。}中对三族姓子说与和合之味的功德相关者,在传车经\footnote{即\textbf{中部}第 24 经。}中对生地比丘说与十事论相关者,如是,以随适于这时机的与饮品的功德相关的言说\textbf{显示}布施饮品的功德,再为敦促应作如是功德而\textbf{唤起},为令生起热情而\textbf{鼓舞},为以现世、后世果报之殊胜令欢喜而\textbf{欢喜}\footnote{显示、唤起、鼓舞、欢喜 \textit{sandassesi samādapesi samuttejesi sampahaṃsesi}:旧译作「示教利喜」,其中旧译作「教」者,PED 的解释是令接受、激励、唤起,故结合义注译作「唤起」。},因此说「以如法的言说……欢喜」。\end{enumerate}

\textbf{于是,萦发者翅宁被世尊如法的言说所显示、唤起、鼓舞、欢喜,对世尊说:「请乔达摩君明日接受我的食物,与比丘僧团一起。」如是说已,世尊对萦发者翅宁说:「翅宁!比丘僧团很大,有千二百五十比丘,且你信乐于婆罗门。」}

Atha kho Keṇiyo jaṭilo Bhagavatā dhammiyā kathāya sandassito samādapito samuttejito sampahaṃsito Bhagavantaṃ etad avoca: “adhivāsetu me bhavaṃ Gotamo svātanāya bhattaṃ saddhiṃ bhikkhusaṅghenā” ti. Evaṃ vutte, Bhagavā Keṇiyaṃ jaṭilaṃ etad avoca: “mahā kho, Keṇiya, bhikkhusaṅgho aḍḍhateḷasāni bhikkhusatāni, tvañ ca brāhmaṇesu abhippasanno” ti.

\textbf{第二次,萦发者翅宁对世尊说:「乔达摩君!尽管比丘僧团很大,有千二百五十比丘,且我信乐于婆罗门,请乔达摩君明日接受我的食物,与比丘僧团一起。」第二次,世尊对萦发者翅宁说:「翅宁!比丘僧团很大,有千二百五十比丘,且你信乐于婆罗门。」}

Dutiyam pi kho Keṇiyo jaṭilo Bhagavantaṃ etad avoca: “kiñcāpi, bho Gotama, mahā bhikkhusaṅgho aḍḍhateḷasāni bhikkhusatāni, ahañ ca brāhmaṇesu abhippasanno, adhivāsetu me bhavaṃ Gotamo svātanāya bhattaṃ saddhiṃ bhikkhusaṅghenā” ti. Dutiyam pi kho Bhagavā Keṇiyaṃ jaṭilaṃ etad avoca: “mahā kho, Keṇiya, bhikkhusaṅgho aḍḍhateḷasāni bhikkhusatāni, tvañ ca brāhmaṇesu abhippasanno” ti.

\textbf{第三次,萦发者翅宁对世尊说:「乔达摩君!尽管比丘僧团很大,有千二百五十比丘,且我信乐于婆罗门,请乔达摩君明日接受我的食物,与比丘僧团一起。」世尊便以默然而接受。}

Tatiyam pi kho Keṇiyo jaṭilo Bhagavantaṃ etad avoca: “kiñcāpi, bho Gotama, mahā bhikkhusaṅgho aḍḍhateḷasāni bhikkhusatāni, ahañ ca brāhmaṇesu abhippasanno, adhivāsetu me bhavaṃ Gotamo svātanāya bhattaṃ saddhiṃ bhikkhusaṅghenā” ti. Adhivāsesi Bhagavā tuṇhībhāvena.

\begin{enumerate}\item 他因对世尊更加净喜,便邀请了世尊,世尊拒绝了三次而接受,因此说「\textbf{于是,萦发者翅宁……世尊便以默然而接受}」。但世尊为什么要拒绝?因为以再再地请求,则他的福德将增长,且会准备更多,因此为千二百五十比丘准备的能足够千五百五十人的。若问「另三百人从哪里来」?因为世尊见到,食物尚未准备好时,婆罗门施罗及其三百学童将出家,故如是说。\end{enumerate}

\textbf{于是,萦发者翅宁已知世尊接受,从坐起,往自己的草庵走去,走到后,向朋友、僚属、亲戚、血亲宣告:「请听我说!尊敬的朋友、僚属、亲戚、血亲!沙门乔达摩受我邀请,明日与比丘僧团一起来受食,你们能为我做点体力活吗?」「如是,先生!」萦发者翅宁的朋友、僚属、亲戚、血亲回答后,有些人挖灶,有些人劈柴,有些人洗盘子,有些人装水罐,有些人备坐具,而萦发者翅宁自己则搭棚幕。}

Atha kho Keṇiyo jaṭilo Bhagavato adhivāsanaṃ viditvā uṭṭhāyāsanā yena sako assamo ten’upasaṅkami, upasaṅkamitvā mittāmacce ñātisālohite āmantesi: “suṇantu me bhavanto mittāmaccā ñātisālohitā, samaṇo me Gotamo nimantito svātanāya bhattaṃ saddhiṃ bhikkhusaṅghena, yena me kāyaveyyāvaṭikaṃ kareyyāthā” ti. “Evaṃ, bho” ti kho Keṇiyassa jaṭilassa mittāmaccā ñātisālohitā Keṇiyassa jaṭilassa paṭissutvā app ekacce uddhanāni khaṇanti, app ekacce kaṭṭhāni phālenti, app ekacce bhājanāni dhovanti, app ekacce udakamaṇikaṃ patiṭṭhāpenti, app ekacce āsanāni paññāpenti, Keṇiyo pana jaṭilo sāmaṃ yeva maṇḍalamāḷaṃ paṭiyādeti.

\begin{enumerate}\item \textbf{朋友、僚属},即朋友和工人。\textbf{亲戚、血亲},即相同血缘、一胎所系和儿女等其他亲属。\textbf{体力活},即以身体服务。\textbf{搭棚幕},即造白色华盖的亭子。\end{enumerate}

\textbf{尔时,婆罗门施罗在市场定居,他精通三吠陀,及其词汇、仪轨、音韵、词源并其传承为第五,通句读、晓文法,熟稔顺世论与大人相,并教授三百学童学习颂诗。}

Tena kho pana samayena Selo brāhmaṇo Āpaṇe paṭivasati, tiṇṇaṃ vedānaṃ pāragū sanighaṇḍu-keṭubhānaṃ sākkhara-ppabhedānaṃ itihāsa-pañcamānaṃ padako veyyākaraṇo lokāyata-mahāpurisalakkhaṇesu anavayo, tīṇi ca māṇavakasatāni mante vāceti.

\begin{enumerate}\item \textbf{三吠陀},即梨俱吠陀、夜柔吠陀、娑摩吠陀。\textbf{词汇},即阐明树等名词词汇的同义词的学科。\textbf{仪轨},即应作仪轨之分支,对诗人有助益的学科。\textbf{音韵、词源},即语音和语源。\textbf{传承为第五},即以阿闼婆吠陀为第四,与「曾经如此、曾经如此」等这样的言说相关、被称为故老传说的传承为彼等的第五\footnote{传承为第五:前四似指「词汇、仪轨、音韵、词源」等吠陀支,而非四吠陀,菩提比丘注 1530 亦作是说。吠陀支,即研究吠陀的学科,有语音 \textit{śikṣā}、韵律 \textit{chanda}、语法 \textit{vyākaraṇa}、语源 \textit{nirukta}、仪轨 \textit{kalpa} 与占星 \textit{jyotiṣa} 六者,与义注的说法有些出入。}。学习、明了词句及除此之外的语法为\textbf{通句读、晓文法}。于顺世论,即诡辩论术,以及于大人相的研究,即一万二千大人相的典籍,无欠缺而圆满,为\textbf{熟稔顺世论与大人相},即是说不生疏。若生疏者,则不能从意义、从文本受持彼等。\end{enumerate}

\textbf{尔时,萦发者翅宁信乐于婆罗门施罗。于是,婆罗门施罗为三百学童所随从,徒步随行、游荡,往萦发者翅宁的草庵走去。婆罗门施罗看到在萦发者翅宁的草庵,有些人挖灶,有些人劈柴……有些人备坐具,而萦发者翅宁自己则搭棚幕,看到后,对萦发者翅宁说:「翅宁君是要娶亲,还是要嫁女?是举行大供养,还是明日邀请了摩竭陀王具军·频婆娑罗及其军队?」}

Tena kho pana samayena Keṇiyo jaṭilo Sele brāhmaṇe abhippasanno hoti. Atha kho Selo brāhmaṇo tīhi māṇavakasatehi parivuto jaṅghāvihāraṃ anucaṅkamamāno anuvicaramāno yena Keṇiyassa jaṭilassa assamo ten’upasaṅkami. Addasā kho Selo brāhmaṇo Keṇiyassa jaṭilassa assame app ekacce uddhanāni khaṇante…pe… app ekacce āsanāni paññapente, Keṇiyaṃ pana jaṭilaṃ sāmaṃ yeva maṇḍalamāḷaṃ paṭiyādentaṃ, disvāna Keṇiyaṃ jaṭilaṃ etad avoca: “kiṃ nu kho bhoto Keṇiyassa āvāho vā bhavissati, vivāho vā bhavissati, mahāyañño vā paccupaṭṭhito, rājā vā Māgadho Seniyo Bimbisāro nimantito svātanāya saddhiṃ balakāyenā” ti?

\begin{enumerate}\item 为利益小腿而\textbf{徒步},即是说为除去久坐等所生的疲劳而伸展小腿,并非长途游行。\textbf{随行},即经行。\textbf{游荡},即四处游行。由具足大军为\textbf{具军}。频婆即金色,以与真金相同的肤色为\textbf{频婆娑罗}。\end{enumerate}

\textbf{「施罗君!我不是要娶亲或嫁女,也不是明日邀请了摩竭陀王具军·频婆娑罗及其军队,而是举行大供养。有沙门乔达摩、释迦子、从释迦族出家者,在鸯伽北水游行,与大比丘僧团千二百五十比丘俱,已到达市场,关于这乔达摩君……佛、世尊,他受我邀请,明日与比丘僧团一起来受食。」}

“Na me, bho Sela, āvāho vā bhavissati vivāho vā, nāpi rājā Māgadho Seniyo Bimbisāro nimantito svātanāya saddhiṃ balakāyena, api ca kho me mahāyañño paccupaṭṭhito. Atthi samaṇo Gotamo Sakyaputto Sakyakulā pabbajito Aṅguttarāpesu cārikaṃ caramāno mahatā bhikkhusaṅghena saddhiṃ aḍḍhateḷasehi bhikkhusatehi Āpaṇaṃ anuppatto, taṃ kho pana bhavantaṃ Gotamaṃ…pe… buddho bhagavā ti, so me nimantito svātanāya bhattaṃ saddhiṃ bhikkhusaṅghenā” ti.

\textbf{「翅宁君!你是说『佛陀』?」\\}
\textbf{「施罗君!我是说『佛陀』。」\\}
\textbf{「翅宁君!你是说『佛陀』?」\\}
\textbf{「施罗君!我是说『佛陀』。」}

“Buddho ti, bho Keṇiya, vadesi”? “Buddho ti, bho Sela, vadāmi”. “Buddho ti, bho Keṇiya, vadesi”? “Buddho ti, bho Sela, vadāmī” ti.

\begin{enumerate}\item 然后,婆罗门由先前曾作侍奉,甫闻「佛陀」之声,便如甘露灌顶,出于惊异而说:「\textbf{翅宁君!你是说『佛陀』}?」另一个则如实地告知:「\textbf{施罗君!我是说『佛陀』}。」随后,他为了确认再次发问,另一个也同样地告知。\end{enumerate}

\textbf{于是,婆罗门施罗道:「在世间,甚至这声音也难得,此即是『佛陀』。而在我们的颂诗中流传有三十二大人相,若具足的大人唯有两种趣向,而非其它。如果他居家,则成转轮王,如法的法王,四方者,征服者,国土安泰,七宝具足,他有这七宝:即轮宝、象宝、马宝、摩尼宝、女宝、家主宝及首相宝为第七。他的子嗣过千,英勇飒爽,摧伏敌军。他不以棍杖、不以刀剑,而是以法征服这以大海为边界的土地已而安居。然而,如果他从家出家,则成阿罗汉、正等正觉者、世间的去蔽者。翅宁君!现今这乔达摩君、阿罗汉、正等正觉者住在哪里?」}

Atha kho Selassa brāhmaṇassa etad ahosi: “ghoso pi kho eso dullabho lokasmiṃ, yad idaṃ buddho ti. Āgatāni kho pan’amhākaṃ mantesu dvattiṃsa-mahāpurisa-lakkhaṇāni, yehi samannāgatassa mahāpurisassa dve va gatiyo bhavanti anaññā. Sace agāraṃ ajjhāvasati rājā hoti cakkavattī dhammiko dhammarājā cāturanto vijitāvī janapadatthāvariyappatto sattaratanasamannāgato, tass’imāni satta ratanāni bhavanti, seyyathidaṃ— cakkaratanaṃ, hatthiratanaṃ, assaratanaṃ, maṇiratanaṃ, itthiratanaṃ, gahapatiratanaṃ, pariṇāyakaratanam eva sattamaṃ. Parosahassaṃ kho pan’assa puttā bhavanti sūrā vīraṅgarūpā parasenappamaddanā. So imaṃ pathaviṃ sāgarapariyantaṃ adaṇḍena asatthena dhammena abhivijiya ajjhāvasati. Sace kho pana agārasmā anagāriyaṃ pabbajati, arahaṃ hoti sammāsambuddho loke vivaṭṭacchado. Kahaṃ pana, bho Keṇiya, etarahi so bhavaṃ Gotamo viharati arahaṃ sammāsambuddho” ti?

\begin{enumerate}\item 于是,为显明即便经十万劫也难得「佛陀」之声而说「\textbf{在世间,甚至这声音也难得,此即是『佛陀』}」。这里,\textbf{此即是}为不变词。
\item 于是,婆罗门听闻佛陀之声,便欲考察「他是真的佛陀,还是徒有佛陀之名呢」,便想道,或径说道:「\textbf{而在我们的颂诗中流传……去蔽者}。」这里,\textbf{颂诗},即吠陀。而在事先,净居天人们想「据说如来将要出世」,便以婆罗门的衣装,糅合了(大人)相而教授吠陀,想「依据此,具大威德的有情就会认出如来」。因此,先前的吠陀中传有大人相,而当如来般涅槃后就逐渐失传,以故现今无存。\textbf{大人},即受持誓愿、受持智、具大悲悯等功德之人。
\item \textbf{唯有两种趣向},即唯有两种结局。当然,这趣向一词在\begin{quoting}舍利弗!兹有五种趣向。(中部·大狮子吼经第 153 段)\end{quoting}等处指有的种类,在\begin{quoting}鹿的趣向为林薮。(律藏·附随第 339 段)\end{quoting}等处指住处,在\begin{quoting}如是具有极度智慧 \textit{gatimanto}。(中部·大狮子吼经第 161 段)\end{quoting}等处指慧,在\begin{quoting}已达趣向。(律藏·小品第 204 段)\end{quoting}等处指蔓延的状态,而在此处,当知是指结局。这里,虽然任何转轮王所具足之相都非佛陀的,但由种类相同故两者并称。因此说「\textbf{若具足的}」。
\item \textbf{则成转轮王},以四不思议法及四摄事愉悦世间为王,令轮宝转动,以四成就轮而转,且以彼等令他人转动,且为了他人利益,他有威仪之轮的义务,为转轮者,且此中的王是通称,转轮者为属性。以法而行即\textbf{如法},即以理、以公正而转之义。以法获得统治成为国王为\textbf{法王}。或者,以行他人利益之法为如法,以行自身利益之法为法王。四方的主宰为\textbf{四方者},即为四大海与四类洲所饰的大地的主宰之义。征服了内在的忿怒等怨敌与外在的一切诸王为\textbf{征服者}。\textbf{国土安泰},即在国土中至于坚牢、安泰的状态,不为任何所动摇,或者,国土在其治下至于安泰,宁静,乐于自业,不动不摇,亦为国土安泰。
\item \textbf{即}为不变词,即这些是什么之义。\textbf{轮宝……首相宝为第七},这些已在宝经(第 226 颂)注中从一切品类而说。其中,这转轮王以轮宝战胜未被战胜者,以象、马宝在领土随意巡行,以首相宝守护领土,以其余领受受用之乐。且以第一(轮宝),其与勇猛相应的能力圆满,以象、马、家主宝,与统治相应的能力圆满,以首相宝,与智慧相应的能力圆满,且以女、摩尼宝为与三种能力相应的果报。他以女、摩尼宝领受财富之乐,以其余领受主宰之乐。且特别地,当知前三种以无嗔善根所生之业的威力成就,中(三种)以无贪善根所生之业的威力,后一种以无痴善根所生之业的威力。
\item \textbf{过千},即一千有余。\textbf{英勇},即生来无畏。\textbf{飒爽},即身如天子一般,有些人如是说。而此中,其自性为:最上英勇被称为英雄 \textit{vīra},英雄之支为英雄支 \textit{vīraṅga},即是说英雄的原因、精进,他们有英雄支之容貌为飒爽 \textit{vīraṅgarūpa},即是说身体如精进所造一般。\textbf{摧伏敌军},如果对面立有敌军,便能摧伏之意。\textbf{以法},即以\begin{quoting}不应杀生。(长部·大善见经第 244 段)\end{quoting}等的五戒法。
\item 而在「\textbf{则成阿罗汉、正等正觉者、世间的去蔽者}」中,在为贪、嗔、痴、慢、见、无明、恶行的七种屋盖覆蔽的烦恼黑暗的世间,揭去此屋盖已,让一切焕发光明而住,即去蔽者。这里,当知以第一词说应供养性,以第二说其因,即因为正等正觉,以第三说佛性之因为去蔽。或者,还灭与去盖为去蔽者,即是说无有流转、无有覆盖,因此,以无流转为阿罗汉,以无覆盖为正等正觉者,如是即是说前两词的两种因。且此中,以第二无所畏\footnote{四无所畏:见\textbf{中部}·大狮子吼经,俱舍论译作「正等觉无畏、漏永尽无畏、说障法无畏、说出道无畏」。}成就最初,以第一无所畏成就第二,以第三、第四无所畏成就第三。且当知最初达成法眼,第二达成佛眼,第三达成一切眼。
\item 现在,他欲去到世尊跟前,便说:「\textbf{现今这……住在哪里}?」\end{enumerate}

\textbf{如是说已,萦发者翅宁伸出右臂,对婆罗门施罗说:「施罗君!在那片青林际。」}

Evaṃ vutte, Keṇiyo jaṭilo dakkhiṇaṃ bāhuṃ paggahetvā Selaṃ brāhmaṇaṃ etad avoca: “yen’esā, bho Sela, nīlavanarājī” ti.

\begin{enumerate}\item 在「如是说已」等中,\textbf{青林际},即绿树的行列。据说树林\footnote{树林 \textit{vana}:PTS 本作「省藤林 \textit{vetasavana}」。}如云的行列一般。他为指示世尊当时所住之处而说:「施罗君!在那片青林际。」这里省略了文本「他住」,或者具格为依格之义\footnote{这是指 yena 一词而说。}。\end{enumerate}

\textbf{于是,婆罗门施罗与三百学童往世尊处走去。婆罗门施罗便告学童们:「诸君!请轻声而往!一步一步落足,因为彼诸世尊难以接近,如狮子般独行,且当我与沙门乔达摩交谈时,诸君!不要打断我的话,诸君请等我把话说完!」}

Atha kho Selo brāhmaṇo tīhi māṇavakasatehi saddhiṃ yena Bhagavā ten’upasaṅkami. Atha kho Selo brāhmaṇo te māṇavake āmantesi: “appasaddā bhonto āgacchantu, pade padaṃ nikkhipantā, durāsadā hi te Bhagavanto sīhā va ekacarā, yadā cāhaṃ, bho, samaṇena Gotamena saddhiṃ manteyyaṃ, mā me bhonto antarantarā kathaṃ opātetha, kathāpariyosānaṃ me bhavanto āgamentū” ti.

\begin{enumerate}\item \textbf{一步一步},即在足的近处落足,以此遮止速行。\textbf{难以接近}是说原因,因为彼等难以接近,所以诸君应如是前往。那是什么原因难以接近呢?\textbf{如狮子般独行}。因为好比狮子,由无需朋友的服务而独行,如是,彼等亦由欲远离而独行。而以「\textbf{且当我}……」等教导这些学童举止。这里,\textbf{不要打断},即是说不要插入、不要评论。\textbf{请等},即请敬候,直至谈话终结,请保持默然之义。\end{enumerate}

\textbf{于是,婆罗门施罗便往世尊处走去,走到后,问候了世尊,彼此寒暄已,坐在一边。坐在一边的婆罗门施罗便在世尊身上寻找三十二大人相。婆罗门施罗在世尊身上看到了大部分三十二相,除了二处,于二处大人相疑惑、怀疑、未信解、未净信:即于阴马藏及广长舌。}

Atha kho Selo brāhmaṇo yena Bhagavā ten’upasaṅkami, upasaṅkamitvā Bhagavatā saddhiṃ sammodi, sammodanīyaṃ kathaṃ sāraṇīyaṃ vītisāretvā ekamantaṃ nisīdi. Ekamantaṃ nisinno kho Selo brāhmaṇo Bhagavato kāye dvattiṃsa-mahāpurisa-lakkhaṇāni samannesi. Addasā kho Selo brāhmaṇo Bhagavato kāye dvattiṃsa-mahāpurisa-lakkhaṇāni yebhuyyena ṭhapetvā dve, dvīsu mahāpurisalakkhaṇesu kaṅkhati vicikicchati nādhimuccati na sampasīdati— kosohite ca vatthaguyhe pahūtajivhatāya cā ti.

\begin{enumerate}\item \textbf{寻找},即寻觅。\textbf{大部分},即得见多数,未见少数。随后,为显明所未见者,便说\textbf{除了二处}。\textbf{疑惑},即生起疑惑,愿求道:哎!但愿我能见到!\textbf{怀疑},即从各处搜寻彼等时,不能得见。\textbf{未信解},即因此怀疑而不能决定。\textbf{未净信},即因此对世尊未得净喜「他圆满了相」。或者,以疑惑说极弱的疑虑,以怀疑说中度的,以未信解说强力的,以未净信说由此三法心中污浊之状。
\item \textbf{藏},即为膜鞘遮蔽。\textbf{阴},即生支\footnote{生支 \textit{aṅgajāta}:即男根,这里从旧译名。}。因为世尊如最胜大象般的阴马藏呈金色,与莲胎相同。由为衣服遮蔽故,未得见此,且由在口内,未得观察舌的广长之相,他便于此二相起了怀疑。\end{enumerate}

\textbf{于是,世尊便想:「这婆罗门施罗看到我大部分三十二大人相,除了二处,于二处大人相疑惑、怀疑、未信解、未净信:即于阴马藏及广长舌。」于是,世尊便作了如此的神变预作,好让婆罗门施罗看见世尊的阴马藏。然后,世尊伸出舌头,顺触、逆触两个耳孔,顺触、逆触两个鼻孔,乃至以舌覆盖整个前额。}

Atha kho Bhagavato etad ahosi: “passati kho me ayaṃ Selo brāhmaṇo dvattiṃsa-mahāpurisa-lakkhaṇāni yebhuyyena ṭhapetvā dve, dvīsu mahāpurisa-lakkhaṇesu kaṅkhati vicikicchati nādhimuccati na sampasīdati— kosohite ca vatthaguyhe pahūtajivhatāya cā” ti. Atha kho Bhagavā tathārūpaṃ iddhābhisaṅkhāraṃ abhisaṅkhāsi, yathā addasa Selo brāhmaṇo Bhagavato kosohitaṃ vatthaguyhaṃ. Atha kho Bhagavā jivhaṃ ninnāmetvā ubho pi kaṇṇasotāni anumasi paṭimasi, ubho pi nāsikasotāni anumasi paṭimasi, kevalam pi nalāṭamaṇḍalaṃ jivhāya chādesi.

\begin{enumerate}\item \textbf{如此}是什么样子?我们对此有何可言?这在龙军长老被弥兰陀王问及时已说:\begin{quoting}尊者龙军!世尊曾作恶行。\\什么?大王!\\令大众羞耻之处,他曾示与梵寿婆罗门及其弟子郁多罗、波婆利及其十六弟子婆罗门、施罗婆罗门及其三百学童,尊者!\\大王!世尊非示隐秘,世尊唯示其影,以神变预作已,穿著下衣,系好腰带,披覆上衣,唯显示影像,大王!\\若影像得见,是否为见呢?尊者!\\且住!大王!若有见到心形便能觉悟的有情,正等正觉者会取肉心而示之。\\你真聪明,尊者龙军!(弥兰陀王所问经第 4.3.3 段)\end{quoting}
\item \textbf{伸出},此中,当知以触碰耳孔表明长度,以触碰鼻孔表明薄度,以覆盖前额表明广度。\end{enumerate}

\textbf{于是,婆罗门施罗便想:「沙门乔达摩具足圆满的三十二大人相,非不圆满,但我不知道他是不是佛陀。我听闻年迈、高龄的婆罗门,老师和老师的老师说过『若他们是阿罗汉、正等正觉者,当他们自身的功德受到赞美时,会显示自身』,我何不面对沙门乔达摩,以合适的偈颂赞叹?」于是,婆罗门施罗便面对世尊,以合适的偈颂赞叹道:}

Atha kho Selassa brāhmaṇassa etad ahosi: “samannāgato kho samaṇo Gotamo dvattiṃsa-mahāpurisa-lakkhaṇehi paripuṇṇehi, no aparipuṇṇehi, no ca kho naṃ jānāmi ‘buddho vā no vā’. Sutaṃ kho pana metaṃ brāhmaṇānaṃ vuḍḍhānaṃ mahallakānaṃ ācariyapācariyānaṃ bhāsamānānaṃ: ‘ye te bhavanti arahanto sammāsambuddhā, te sake vaṇṇe bhaññamāne attānaṃ pātukarontī’ ti. Yan nūnāhaṃ samaṇaṃ Gotamaṃ sammukhā sāruppāhi gāthāhi abhitthaveyyan” ti. Atha kho Selo brāhmaṇo Bhagavantaṃ sammukhā sāruppāhi gāthāhi abhitthavi:

\subsection\*{\textbf{554} {\footnotesize 〔PTS 548〕}}

\textbf{「身体圆满,辉光美妙,出生良好,相貌可观,\\}
\textbf{「你肤色金黄,世尊!你齿牙洁白,具有精进。}

“Paripuṇṇakāyo suruci, sujāto cārudassano;\\
suvaṇṇavaṇṇo si Bhagavā, susukkadāṭho si vīriyavā. %\hfill\textcolor{gray}{\footnotesize 1}

\begin{enumerate}\item \textbf{身体圆满},即以诸相圆满及非劣的身材、肢体而身躯圆满。\textbf{辉光美妙},即身光善妙。\textbf{出生良好},即以高宽成就及形相成就而善出生。\textbf{相貌可观},即便长久观看也不满足、不厌逆,见之唯有喜乐、美好。\textbf{齿牙洁白},因为世尊的齿牙如月光一般,发出极白的光芒。\end{enumerate}

\begin{itemize}\item 案,\textbf{肤色金黄、齿牙洁白}为三十二相中的两个。\end{itemize}

\subsection\*{\textbf{555} {\footnotesize 〔PTS 549〕}}

\textbf{「出生良好之人所拥有的特相,\\}
\textbf{「一切大人相,都在你的身上。}

Narassa hi sujātassa, ye bhavanti viyañjanā;\\
sabbe te tava kāyasmiṃ, mahāpurisalakkhaṇā. %\hfill\textcolor{gray}{\footnotesize 2}

\subsection\*{\textbf{556} {\footnotesize 〔PTS 550〕}}

\textbf{「眼睛明净,脸庞圆满,高大、端正、具有光辉,\\}
\textbf{「在沙门僧团之中,如太阳般闪耀。}

Pasannanetto sumukho, brahā uju patāpavā;\\
majjhe samaṇasaṅghassa, ādicco va virocasi. %\hfill\textcolor{gray}{\footnotesize 3}

\subsection\*{\textbf{557} {\footnotesize 〔PTS 551〕}}

\textbf{「容貌美好的比丘,拥有金子般的皮肤,\\}
\textbf{「如是具最上肤色者,你为何现沙门相?}

Kalyāṇadassano bhikkhu, kañcanasannibhattaco;\\
kiṃ te samaṇabhāvena, evaṃ uttamavaṇṇino. %\hfill\textcolor{gray}{\footnotesize 4}

\begin{itemize}\item 案,\textbf{金子般的皮肤},即第 554 颂所说的「肤色金黄」,为三十二相之一。\end{itemize}

\subsection\*{\textbf{558} {\footnotesize 〔PTS 552〕}}

\textbf{「你应当成为王、转轮者、车乘之主,\\}
\textbf{「征服四方,作阎浮林的主宰。}

Rājā arahasi bhavituṃ, cakkavattī rathesabho;\\
cāturanto vijitāvī, Jambusaṇḍassa issaro. %\hfill\textcolor{gray}{\footnotesize 5}

\begin{enumerate}\item \textbf{阎浮林},即阎浮洲。\end{enumerate}

\subsection\*{\textbf{559} {\footnotesize 〔PTS 553〕}}

\textbf{「刹帝利和有采地的国王,都成为你的附庸,\\}
\textbf{「王中之王,人中的因陀,你应行王权!乔达摩!」}

Khattiyā bhogi-rājāno, anuyantā bhavantu te;\\
rājābhirājā manujindo, rajjaṃ kārehi Gotama”. %\hfill\textcolor{gray}{\footnotesize 6}

\subsection\*{\textbf{560} {\footnotesize 〔PTS 554〕}}

\textbf{「我是王,施罗!」世尊说,「无上的法王,\\}
\textbf{「我用法转轮,这轮不能倒转。」}

“Rājāham asmi Selā ti, \textit{(Bhagavā)} dhammarājā anuttaro;\\
dhammena cakkaṃ vattemi, cakkaṃ appaṭivattiyaṃ”. %\hfill\textcolor{gray}{\footnotesize 7}

\subsection\*{\textbf{561} {\footnotesize 〔PTS 555〕}}

\textbf{「你自称是等正觉,」婆罗门施罗说,「无上的法王,\\}
\textbf{「你说『我用法转轮』,乔达摩!}

“Sambuddho paṭijānāsi, \textit{(iti Selo brāhmaṇo)} dhammarājā anuttaro;\\
‘dhammena cakkaṃ vattemi’, iti bhāsasi Gotama. %\hfill\textcolor{gray}{\footnotesize 8}

\subsection\*{\textbf{562} {\footnotesize 〔PTS 556〕}}

\textbf{「谁是您的大将、追随大师的弟子?\\}
\textbf{「谁为你续转这转起的法轮?」}

Ko nu senāpati bhoto, sāvako satthu-r-anvayo;\\
ko te tam anuvatteti, dhammacakkaṃ pavattitaṃ”. %\hfill\textcolor{gray}{\footnotesize 9}

\subsection\*{\textbf{563} {\footnotesize 〔PTS 557〕}}

\textbf{「我所转之轮,施罗!」世尊说,「无上的法轮,\\}
\textbf{「舍利弗,承嗣如来者,能够续转。}

“Mayā pavattitaṃ cakkaṃ, \textit{(Selā ti Bhagavā)} dhammacakkaṃ anuttaraṃ;\\
sāriputto anuvatteti, anujāto Tathāgataṃ. %\hfill\textcolor{gray}{\footnotesize 10}

\begin{enumerate}\item 此时,尊者舍利弗坐在世尊的右侧,如积聚的金子般流放着光彩,为显示他,故世尊说此颂。\end{enumerate}

\subsection\*{\textbf{564} {\footnotesize 〔PTS 558〕}}

\textbf{「应证知的已证知,应修习的已修习,\\}
\textbf{「应舍弃的我已舍弃,所以我是佛陀,婆罗门!}

Abhiññeyyaṃ abhiññātaṃ, bhāvetabbañ ca bhāvitaṃ;\\
pahātabbaṃ pahīnaṃ me, tasmā Buddho’smi Brāhmaṇa. %\hfill\textcolor{gray}{\footnotesize 11}

\begin{enumerate}\item \textbf{应证知的}即明与解脱,而道谛、集谛是\textbf{应修习的、应舍弃的},然而由于以因成就果,它们的果,灭谛与苦谛也已说了,因此在这里也说了「应证得的已证得,应遍知的已遍知」。如是显示了四谛修习之果的明与解脱,「在觉悟了应觉悟的后,我成了佛陀」,以相应的因证明了佛性。\end{enumerate}

\subsection\*{\textbf{565} {\footnotesize 〔PTS 559〕}}

\textbf{「你应调伏对我的疑惑,你应信解,婆罗门!\\}
\textbf{「见到等正觉者们总是难得。}

Vinayassu mayi kaṅkhaṃ, adhimuccassu Brāhmaṇa;\\
dullabhaṃ dassanaṃ hoti, sambuddhānaṃ abhiṇhaso. %\hfill\textcolor{gray}{\footnotesize 12}

\subsection\*{\textbf{566} {\footnotesize 〔PTS 560〕}}

\textbf{「他们在世间确实总是难得出现,\\}
\textbf{「婆罗门!我是等正觉,无上的外科医生。}

Yesaṃ ve dullabho loke, pātubhāvo abhiṇhaso;\\
so’haṃ Brāhmaṇa sambuddho, sallakatto anuttaro. %\hfill\textcolor{gray}{\footnotesize 13}

\begin{enumerate}\item \textbf{外科医生},即贪箭等七箭的治疗者。\end{enumerate}

\begin{itemize}\item 菩提比丘:\textbf{七箭},即贪、嗔、痴、慢、见、忧、疑 \textit{kathaṅkathā}。\end{itemize}

\subsection\*{\textbf{567} {\footnotesize 〔PTS 561〕}}

\textbf{「已成为梵,无可比拟,摧伏魔军,\\}
\textbf{「臣服了一切敌人,我实欣喜,无处畏惧。」}

Brahmabhūto atitulo, Mārasenappamaddano;\\
sabbāmitte vasīkatvā, modāmi akutobhayo”. %\hfill\textcolor{gray}{\footnotesize 14}

\begin{enumerate}\item \textbf{已成为梵},即已成为最胜。\textbf{无可比拟},即越过相等、越过相似而无有相似之义。\textbf{摧伏魔军},即摧伏了如「爱欲是你的第一军……以及毁他」等所说的称为魔罗随从的魔军。\textbf{一切敌人},即蕴、烦恼、行作、死、天子魔罗等一切敌对。\end{enumerate}

\begin{itemize}\item 案,\textbf{爱欲是你的第一军……以及毁他},见至上经第 439~441 颂。\end{itemize}

\subsection\*{\textbf{568} {\footnotesize 〔PTS 562〕}}

\textbf{「诸君!你们倾听,如具眼者所说,\\}
\textbf{「这外科医生、大雄,如林中狮吼。}

“Imaṃ bhavanto nisāmetha, yathā bhāsati cakkhumā;\\
sallakatto mahāvīro, sīho va nadatī vane. %\hfill\textcolor{gray}{\footnotesize 15}

\subsection\*{\textbf{569} {\footnotesize 〔PTS 563〕}}

\textbf{「已成为梵、无可比拟、摧伏魔军者,\\}
\textbf{「谁见了能不净喜,即便是黑色出生?}

Brahmabhūtaṃ atitulaṃ, Mārasenappamaddanaṃ;\\
ko disvā na-ppasīdeyya, api kaṇhābhijātiko. %\hfill\textcolor{gray}{\footnotesize 16}

\begin{enumerate}\item \textbf{黑色出生},即于旃陀罗等低贱家族出生者。\end{enumerate}

\subsection\*{\textbf{570} {\footnotesize 〔PTS 564〕}}

\textbf{「若愿意的可跟随我,若不愿意的请离开,\\}
\textbf{「于此,我将出家,在胜慧者的跟前。」}

Yo maṃ icchati anvetu, yo vā n’icchati gacchatu;\\
idhāhaṃ pabbajissāmi, varapaññassa santike”. %\hfill\textcolor{gray}{\footnotesize 17}

\subsection\*{\textbf{571} {\footnotesize 〔PTS 565〕}}

\textbf{「如是,若您喜好正等正觉的教法,\\}
\textbf{「我们也将出家,在胜慧者的跟前。」}

“Evañ ce ruccati bhoto, Sammāsambuddhasāsane;\\
mayam pi pabbajissāma, varapaññassa santike”. %\hfill\textcolor{gray}{\footnotesize 18}

\subsection\*{\textbf{572} {\footnotesize 〔PTS 566〕}}

\textbf{这三百婆罗门合掌祈请:\\}
\textbf{「我们欲行梵行,世尊!在你的跟前。」}

Brāhmaṇā tisatā ime, yācanti pañjalīkatā;\\
“brahmacariyaṃ carissāma, Bhagavā tava santike”. %\hfill\textcolor{gray}{\footnotesize 19}

\subsection\*{\textbf{573} {\footnotesize 〔PTS 567〕}}

\textbf{「梵行已经善说,施罗!」世尊说,「是自见、无时的,\\}
\textbf{「对不放逸的学者,于此出家并非徒劳。」}

“Svākkhātaṃ brahmacariyaṃ, \textit{(Selā ti Bhagavā)} sandiṭṭhikam akālikaṃ;\\
yattha amoghā pabbajjā, appamattassa sikkhato” ti. %\hfill\textcolor{gray}{\footnotesize 20}

\begin{enumerate}\item 随后,因为施罗过去在莲花上世尊 \textit{Padumuttara} 的教法下是三百人中最上,与他们一起令人造了僧寮并行布施等福德,渐次经历了人天的成就,于最后有则作为他们的老师而出生,他们的业已向着解脱遍熟,且成为「来!比丘」状态的近依,所以,世尊令他们以「来!比丘」出家而说此颂。\textbf{自见},即现量。\textbf{无时},即与道无间即果的生起,非于时间间隔后当证得果。\textbf{于此},因为道梵行之相的出家,对不放逸的、不离于念的、三学的学者,不是徒劳。\end{enumerate}

\textbf{于是,婆罗门施罗与其随从在世尊跟前已得出家,已得具足。}

Alattha kho Selo brāhmaṇo sapariso Bhagavato santike pabbajjaṃ, alattha upasampadaṃ.

\begin{enumerate}\item 如是说已,世尊说道:「来!诸比丘!」他们全都持好衣钵,从空中前往问讯世尊。如是,结集者们说「\textbf{婆罗门施罗……已得具足}」。\end{enumerate}

\textbf{于是,萦发者翅宁是夜过后,在自己的草庵中令人准备了精美的硬食、软食,令人向世尊宣告时间:「是时,乔达摩君!食物已备好。」于是,世尊晨朝著了下衣,持了衣钵,往萦发者翅宁的草庵走去,走到后,与比丘僧团一起,坐在备好的坐处。}

Atha kho Keṇiyo jaṭilo tassā rattiyā accayena sake assame paṇītaṃ khādanīyaṃ bhojanīyaṃ paṭiyādāpetvā Bhagavato kālaṃ ārocāpesi: “kālo, bho Gotama, niṭṭhitaṃ bhattan” ti. Atha kho Bhagavā pubbaṇhasamayaṃ nivāsetvā pattacīvaramādāya yena Keṇiyassa jaṭilassa assamo ten’upasaṅkami, upasaṅkamitvā paññatte āsane nisīdi saddhiṃ bhikkhusaṅghena.

\textbf{于是,萦发者翅宁亲手以精美的硬食、软食满足、款待了佛陀为首的比丘僧团。于是,在世尊已足食,把手放开钵时,萦发者翅宁另取了低坐,坐在一边。世尊对坐在一边的萦发者翅宁以偈颂随喜:}

Atha kho Keṇiyo jaṭilo buddhappamukhaṃ bhikkhusaṅghaṃ paṇītena khādanīyena bhojanīyena sahatthā santappesi sampavāresi. Atha kho Keṇiyo jaṭilo Bhagavantaṃ bhuttāviṃ onītapattapāṇiṃ aññataraṃ nīcaṃ āsanaṃ gahetvā ekamantaṃ nisīdi. Ekamantaṃ nisinnaṃ kho Keṇiyaṃ jaṭilaṃ Bhagavā imāhi gāthāhi anumodi:

\subsection\*{\textbf{574} {\footnotesize 〔PTS 568〕}}

\textbf{「祭祀以火祭为上首,歌咏以颂诗为上首,\\}
\textbf{「王是人中上首,海是河的上首。}

“Aggihuttamukhā yaññā, sāvittī chandaso mukhaṃ;\\
rājā mukhaṃ manussānaṃ, nadīnaṃ sāgaro mukhaṃ. %\hfill\textcolor{gray}{\footnotesize 21}

\begin{itemize}\item 案,\textbf{颂诗} \textit{sāvittī},见孙陀利迦婆罗豆婆遮经第 461 颂注。\end{itemize}

\subsection\*{\textbf{575} {\footnotesize 〔PTS 569〕}}

\textbf{「月亮是星宿的上首,太阳是照耀的上首,\\}
\textbf{「对希求福德者,僧伽实是供养者的上首。」}

Nakkhattānaṃ mukhaṃ cando, ādicco tapataṃ mukhaṃ;\\
puññaṃ ākaṅkhamānānaṃ, saṅgho ve yajataṃ mukhan” ti. %\hfill\textcolor{gray}{\footnotesize 22}

\textbf{于是,世尊对萦发者翅宁以偈颂随喜已,从坐起而离开。}

Atha kho Bhagavā Keṇiyaṃ jaṭilaṃ imāhi gāthāhi anumoditvā uṭṭhāyāsanā pakkāmi.

\textbf{于是,尊者施罗与其随从独一、远离、不放逸、热忱、自励而住,此后不久……于是,尊者施罗与其随从成了众阿罗汉中的某个。}

Atha kho āyasmā Selo sapariso eko vūpakaṭṭho appamatto ātāpī pahitatto viharanto nacirass’eva…pe… aññataro kho pan’āyasmā Selo sapariso arahataṃ ahosi.

\textbf{于是,尊者施罗与其随从往世尊处走去,走到后,把衣偏覆一肩,向世尊合掌,以偈颂对世尊说:}

Atha kho āyasmā Selo sapariso yena Bhagavā ten’upasaṅkami, upasaṅkamitvā ekaṃsaṃ cīvaraṃ katvā yena Bhagavā ten’añjaliṃ paṇāmetvā Bhagavantaṃ gāthāya ajjhabhāsi:

\subsection\*{\textbf{576} {\footnotesize 〔PTS 570〕}}

\textbf{「自我们皈依,这是第八日,具眼者!\\}
\textbf{「以七夜,世尊!我们在你的教法中已得调御。}

“Yaṃ taṃ saraṇam āgamha, ito aṭṭhami Cakkhuma;\\
sattarattena Bhagavā, dant’amha tava sāsane. %\hfill\textcolor{gray}{\footnotesize 23}

\subsection\*{\textbf{577} {\footnotesize 〔PTS 571〕}}

\textbf{「你是佛陀,你是大师,你是征服魔罗者、牟尼,\\}
\textbf{「你已切断了随眠,已度,你使这人类得度。}

Tuvaṃ Buddho tuvaṃ Satthā, tuvaṃ Mārābhibhū muni;\\
tuvaṃ anusaye chetvā, tiṇṇo tāres’imaṃ pajaṃ. %\hfill\textcolor{gray}{\footnotesize 24}

\begin{itemize}\item 案,此颂及下颂同会堂经第 551~552 颂。\end{itemize}

\subsection\*{\textbf{578} {\footnotesize 〔PTS 572〕}}

\textbf{「你超越了所依,你破碎了诸漏,\\}
\textbf{「你是狮子,无所取著,舍弃了畏与怕。}

Upadhī te samatikkantā, āsavā te padālitā;\\
sīho si anupādāno, pahīnabhayabheravo. %\hfill\textcolor{gray}{\footnotesize 25}

\subsection\*{\textbf{579} {\footnotesize 〔PTS 573〕}}

\textbf{「这三百比丘合掌站立,\\}
\textbf{「请伸展双足!英雄!让龙象们礼拜大师。」}

Bhikkhavo tisatā ime, tiṭṭhanti pañjalīkatā;\\
pāde Vīra pasārehi, nāgā vandantu Satthuno” ti. %\hfill\textcolor{gray}{\footnotesize 26}

\begin{center}\vspace{1em}施罗经第七\\Selasuttaṃ sattamaṃ.\end{center}