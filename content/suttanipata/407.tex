\section{低舍弥勒经}

\begin{center}Tissametteyya Sutta\end{center}\vspace{1em}

\begin{enumerate}\item 据说,当世尊住在舍卫国时,名叫低舍、弥勒的两个朋友到了舍卫国。他们在哺时看见大众朝着祇园走去,就问「你们去哪里」,听到他们说「佛陀出现世间,他为了众人的利益而开示法,我们去祇园听法」,想「我们也去听」,便去了。他们听了开示不唐捐之法的世尊的法的开示后,在集会中即坐而沉思「这法不可能由住于俗家之中而圆满」。于是,在大众离去后,他们向世尊请求出家。世尊便命某个比丘「度他们出家」。他度他们出家后,说了皮五法的业处,准备去到林野而住。弥勒对低舍说:「朋友!亲教师去林野,我们也去吧!」低舍说「止!朋友!我想见世尊并闻法,你去吧」,便未去。弥勒与亲教师一起去到林野后,行沙门法,不久即与阿阇黎亲教师一起圆满了阿罗汉。而低舍的兄长因病去世了,他听到后便回了自己的村子,在那里,亲戚们诱惑他,便使他还俗。弥勒与阿阇黎亲教师一起来到舍卫国。这时,世尊出雨安居,在人间游行,渐次到了这村。于是弥勒礼敬世尊后,说「尊者!我在家的朋友现在此村,请等待片刻,怜悯故」,进村后,把他带到世尊跟前,站在一边,为了他的义利,以初颂问了世尊问题,世尊则以其余诸颂说了解答。这便是此经的缘起。\end{enumerate}

\subsection\*{\textbf{821} {\footnotesize 〔PTS 814〕}}

\textbf{「对从事淫欲者,」尊者低舍弥勒说,「请说其恼害!先生!\\}
\textbf{「听了你的教法后,我们将修学远离。」}

“­Methuna­m anu­yuttas­sa, \textit{(icc āyasmā Tisso Metteyyo)} vighātaṃ brūhi Mārisa;\\
sutvāna tava sāsanaṃ, viveke sikkhissāmase”. %\hfill\textcolor{gray}{\footnotesize 1}

\begin{itemize}\item 案,大义释中说低舍是长老之名,弥勒是长老之族姓,则似以低舍弥勒为一人,义注承其说,但云「\textbf{低舍}是长老之名,故以名称为低舍,\textbf{弥勒}为族姓,他以族姓为人所知,所以在缘起中说『名叫低舍、弥勒的两个朋友』,\textbf{我们将修学远离},是他为朋友请求法的开示,虽然他于学业已修学」,则认为低舍、弥勒为二人。原文中\textbf{尊者} \textit{āyasmā} 是单数,但\textbf{修学} \textit{sikkhissāmase} 却又是复数,殊费解。\end{itemize}

\subsection\*{\textbf{822} {\footnotesize 〔PTS 815〕}}

\textbf{「对从事淫欲者,弥勒!」世尊说,「这教法甚至都被忘记,\\}
\textbf{「且他邪修习,这于他是非圣。}

“­Methuna­m anu­yuttas­sa, \textit{(Metteyyā ti Bhagavā)} mussate vāpi sāsanaṃ;\\
micchā ca paṭipajjati, etaṃ tasmiṃ anāriyaṃ. %\hfill\textcolor{gray}{\footnotesize 2}

\begin{enumerate}\item \textbf{这教法甚至都被忘记},即教、行上的两重教法都被丧失。\textbf{这于他是非圣},即于此人,这是非圣,即邪行道。\end{enumerate}

\subsection\*{\textbf{823} {\footnotesize 〔PTS 816〕}}

\textbf{「若先前独自而行,(而今)沉湎淫欲,\\}
\textbf{「人们说他如失路的车乘,在世间是卑下的凡夫。}

Eko pubbe caritvāna, methunaṃ yo nisevati;\\
yānaṃ bhantaṃ va taṃ loke, hīnam āhu puthujjanaṃ. %\hfill\textcolor{gray}{\footnotesize 3}

\begin{enumerate}\item \textbf{先前独自而行},即先前由出家或由舍弃群体之义而独自居住。\textbf{人们说他如失路的车乘,在世间是卑下的凡夫},好比象车等的车乘,未调御、不直道而上坡,摔坏了车手,跌落于崖下,如是,此还俗之人由身恶行等不正而行,在地狱等中毁灭了自我,跌落于生之悬崖等,人们说如失路的车乘,且人们说是卑下的凡夫。\end{enumerate}

\subsection\*{\textbf{824} {\footnotesize 〔PTS 817〕}}

\textbf{「且先前的声誉和名声,他都已丧失,\\}
\textbf{「见到此后,他应修学,以舍弃淫欲。}

Yaso kitti ca yā pubbe, hāyate vāpi tassa sā;\\
etam pi disvā sikkhetha, methunaṃ vippahātave. %\hfill\textcolor{gray}{\footnotesize 4}

\subsection\*{\textbf{825} {\footnotesize 〔PTS 818〕}}

\textbf{「他被思惟占据,如同可怜之人在焦灼,\\}
\textbf{「听到他人的斥责,如此等者即生愧畏。}

Saṅkappehi pareto so, kapaṇo viya jhāyati;\\
sutvā paresaṃ nigghosaṃ, maṅku hoti tathāvidho. %\hfill\textcolor{gray}{\footnotesize 5}

\begin{enumerate}\item \textbf{占据},即具足。\textbf{他人的斥责},即亲教师等的责备之语。\textbf{即生愧畏},即生忧伤。\end{enumerate}

\subsection\*{\textbf{826} {\footnotesize 〔PTS 819〕}}

\textbf{「于是,受到他人言语的呵责,他就挥舞刀剑,\\}
\textbf{「这即他的大贪求,他落入妄语之中。}

Atha satthāni kurute, paravādehi codito;\\
esa khv-assa mahāgedho, mosavajjaṃ pagāhati. %\hfill\textcolor{gray}{\footnotesize 6}

\begin{enumerate}\item \textbf{刀剑},即身恶行等,因为它们由割伤自己与他人之义而被称为「刀剑」,尤其当被呵责时,他就挥舞妄语的刀剑,并说「我由这个原因而还俗」。由此故说\textbf{这即他的大贪求,他落入妄语之中}。\textbf{大贪求},即大束缚,指什么?即是落入妄语之中,当知这落入妄语便是他的大贪求。\end{enumerate}

\subsection\*{\textbf{827} {\footnotesize 〔PTS 820〕}}

\textbf{「被称为智者的人,决意独自而行,\\}
\textbf{「即便从事淫欲,也如钝人被摆布。}

Paṇḍito ti samaññāto, ekacariyaṃ adhiṭṭhito;\\
athāpi methune yutto, mando va parikissati. %\hfill\textcolor{gray}{\footnotesize 7}

\begin{enumerate}\item \textbf{如钝人被摆布},即行杀生等、体验此因而来的苦、寻求并守护财产,如同愚钝之人被摆布。\end{enumerate}

\subsection\*{\textbf{828} {\footnotesize 〔PTS 821〕}}

\textbf{「牟尼了知了这于此先前、之后的过患,\\}
\textbf{「应努力独自而行,不应沉湎淫欲。}

Etam ādīnavaṃ ñatvā, muni pubbāpare idha;\\
ekacariyaṃ daḷhaṃ kayirā, na nisevetha methunaṃ. %\hfill\textcolor{gray}{\footnotesize 8}

\begin{enumerate}\item \textbf{牟尼了知了这于此先前、之后的过患},即牟尼了知了在「且先前的声誉和名声,他都已丧失」等中所说的于此教法内从先前沙门的身份到之后还俗的身份的过患。\end{enumerate}

\subsection\*{\textbf{829} {\footnotesize 〔PTS 822〕}}

\textbf{「他唯应修学远离,这是圣者们的最上,\\}
\textbf{「不应由此认为是最胜,他即在涅槃的跟前。}

Vivekañ-ñeva sikkhetha, etaṃ ariyānam uttamaṃ;\\
na tena seṭṭho maññetha, sa ve nibbānasantike. %\hfill\textcolor{gray}{\footnotesize 9}

\begin{enumerate}\item \textbf{这是圣者们的最上},即此远离之行是佛等圣者们的最上,所以他唯应修学远离的意思。\textbf{不应由此认为是最胜},且不应由此远离而认为自己「我乃最胜」,即是说不应由此为傲。\end{enumerate}

\subsection\*{\textbf{830} {\footnotesize 〔PTS 823〕}}

\textbf{「对空无而行的牟尼、不关切爱欲者、\\}
\textbf{「已度过暴流者,系缚于爱欲的世人(徒有)羡慕。」}

Rittassa munino carato, kāmesu anapekkhino;\\
oghatiṇṇassa pihayanti, kāmesu gadhitā pajā” ti. %\hfill\textcolor{gray}{\footnotesize 10}

\begin{enumerate}\item \textbf{空无},即远离、去除了身恶行等。\textbf{系缚于爱欲的世人羡慕已度过暴流者},即执著于物欲的有情羡慕这已度过四暴流者,如同负债者之于无债者。当开示终了,低舍即得了须陀洹果,随后出家,证得了阿罗汉。\end{enumerate}

\begin{center}\vspace{1em}低舍弥勒经第七\\Tissametteyyasuttaṃ sattamaṃ.\end{center}