\section{低舍弥勒经}

\begin{center}Tissametteyya Sutta\end{center}\vspace{1em}

\begin{enumerate}\item 缘起为何?据说,当世尊住舍卫国时,名叫低舍、弥勒的两个朋友到了舍卫国。他们在哺时看见大众朝着祇园走去,就问:「你们去哪里?」随后,被他们告知「佛陀出现世间,他为了众人的利益开示法,我们去祇园听法」,便想「我们也去听」而去。他们听了开示不唐捐之法的世尊的法的开示后,在集会中即坐而沉思:「这法不可能以住于俗家之中而圆满。」
\item 于是,在大众离去后,他们便向世尊请求出家。世尊便命某比丘:「度他们出家。」他度他们出家后,授了皮五法的业处,就准备到林野居住。弥勒对低舍说:「朋友!亲教师去林野,我们也去吧!」低舍说「止!朋友!我渴望见到世尊并闻法,你去吧」,便未去。弥勒与亲教师一起去到林野后,行沙门法,不久即与阿阇黎亲教师一起圆满了阿罗汉。而低舍的兄长因病死去,他听到后便回了自己的村子,在那里,亲戚们诱惑他,便让他还了俗。
\item 弥勒与阿阇黎亲教师一起来到舍卫国。这时,世尊出雨安居,在人间游行,渐次到了这村。于是弥勒礼敬了世尊,说「尊者!我在家的朋友现在此村,请等待片刻,怜悯故」,进村后,便把他带到世尊跟前,站在一边,为了他的义利,以初颂问了世尊问题,世尊则为解答此,说了其余几颂。这即此经的缘起。\end{enumerate}

\subsection\*{\textbf{821} {\footnotesize 〔PTS 814〕}}

\textbf{「对从事淫欲者,」尊者低舍弥勒说,「请说困扰!先生!\\}
\textbf{「听了你的教法,我们将修学远离。」}

“­Methuna­m anu­yuttas­sa, \textit{(icc āyasmā Tisso Metteyyo)} vighātaṃ brūhi mārisa;\\
sutvāna tava sāsanaṃ, viveke sikkhissāmase”. %\hfill\textcolor{gray}{\footnotesize 1}

\begin{enumerate}\item 这里,\textbf{尊者}是敬称。\textbf{低舍}是这长老之名,因为他名为低舍,\textbf{弥勒}为族姓,且他以族姓为人所知,所以在缘起中说「名叫低舍、弥勒的两个朋友」。\textbf{困扰},即伤害。\textbf{说},即告知。\textbf{先生}是敬称,即是说离苦者。\textbf{听了你的教法},即听了你的言语。\textbf{我们将修学远离},是为朋友请求法的开示而说,而他于学业已修学。\end{enumerate}

\subsection\*{\textbf{822} {\footnotesize 〔PTS 815〕}}

\textbf{「对从事淫欲者,弥勒!」世尊说,「教法甚至都被忘记,\\}
\textbf{「且他邪行道,这于他是非圣。}

“­Methuna­m anu­yuttas­sa, \textit{(Metteyyā ti Bhagavā)} mussate vāpi sāsanaṃ;\\
micchā ca paṭipajjati, etaṃ tasmiṃ anāriyaṃ. %\hfill\textcolor{gray}{\footnotesize 2}

\begin{enumerate}\item \textbf{教法甚至都被忘记},即教理、行道上的两重教法都已丧失。\textbf{这于他是非圣},即于此人,这是非圣,即邪行道。\end{enumerate}

\subsection\*{\textbf{823} {\footnotesize 〔PTS 816〕}}

\textbf{「若先前独自而行,(而今)沉湎淫欲,\\}
\textbf{「人们说他如失路之车,是世间卑下的凡夫。}

Eko pubbe caritvāna, methunaṃ yo nisevati;\\
yānaṃ bhantaṃ va taṃ loke, hīnam āhu puthujjanaṃ. %\hfill\textcolor{gray}{\footnotesize 3}

\begin{enumerate}\item \textbf{先前独自而行},即以被称为出家或以舍离群体之义,先前独自居住。\textbf{人们说他如失路之车,是世间卑下的凡夫},好比象车等的车乘,未调御、不直道而上坡,摔坏了车手,跌落于崖下,如是,因这还俗之人以身恶行等不正而上坡,在地狱等中毁灭了自我,且跌落于生之悬崖等,人们说如失路之车,且是卑下的凡夫。\end{enumerate}

\subsection\*{\textbf{824} {\footnotesize 〔PTS 817〕}}

\textbf{「且先前的声誉和名声,他都已丧失,\\}
\textbf{「见到此后,他应修学,以舍弃淫欲。}

Yaso kitti ca yā pubbe, hāyate vāpi tassa sā;\\
etam pi disvā sikkhetha, methunaṃ vippahātave. %\hfill\textcolor{gray}{\footnotesize 4}

\begin{enumerate}\item \textbf{声誉和名声},即利养、恭敬和赞叹。\textbf{先前},即现出家相时。\textbf{他都已丧失},即当他还俗时,这声誉和名声都已丧失。\textbf{见到此后},即见到这先前的声誉和名声之相与后来的丧失后。\textbf{他应修学,以舍弃淫欲},即他应修学三学。什么原因?以舍弃淫欲,即是说为了舍弃淫欲。\end{enumerate}

\subsection\*{\textbf{825} {\footnotesize 〔PTS 818〕}}

\textbf{「他被思惟占据,如同可怜之人在焦灼,\\}
\textbf{「听到他人的斥责,这样的人即生愧畏。}

Saṅkappehi pareto so, kapaṇo viya jhāyati;\\
sutvā paresaṃ nigghosaṃ, maṅku hoti tathāvidho. %\hfill\textcolor{gray}{\footnotesize 5}

\begin{enumerate}\item 因为若不舍弃淫欲,「他被思惟占据……即生愧畏」。这里,\textbf{占据},即具足。\textbf{他人的斥责},即亲教师等的责备之语。\textbf{即生愧畏},即生忧虑。\end{enumerate}

\subsection\*{\textbf{826} {\footnotesize 〔PTS 819〕}}

\textbf{「于是,受到他人言语的呵责,他就挥舞刀剑,\\}
\textbf{「这即他的大贪求,他陷入妄语之中。}

Atha satthāni kurute, paravādehi codito;\\
esa khv assa mahāgedho, mosavajjaṃ pagāhati. %\hfill\textcolor{gray}{\footnotesize 6}

\begin{enumerate}\item 此后几颂的连结都自明。其中,\textbf{刀剑},即身恶行等。因为它们由割伤自己与他人之义而被称为「刀剑」。且尤其当被呵责时,他就挥舞妄语的刀剑,并说「我因这个原因而还俗」。因此故说「\textbf{这即他的大贪求,他陷入妄语之中}」。这里,\textbf{大贪求},即大束缚。若问是什么?即是陷入妄语之中,当知这陷入妄语便是他的大贪求。\end{enumerate}

\subsection\*{\textbf{827} {\footnotesize 〔PTS 820〕}}

\textbf{「决意独自而行,他被许为智者,\\}
\textbf{「但若从事淫欲,如钝人被摆布。}

Paṇḍito ti samaññāto, ekacariyaṃ adhiṭṭhito;\\
athāpi methune yutto, mando va parikissati. %\hfill\textcolor{gray}{\footnotesize 7}

\begin{enumerate}\item \textbf{如钝人被摆布},即当行杀生等、体验因此而来的苦、寻求并守护财产时,如愚钝之人般受苦。\end{enumerate}

\subsection\*{\textbf{828} {\footnotesize 〔PTS 821〕}}

\textbf{「牟尼了知了这之前、之后的过患,\\}
\textbf{「应努力独自而行,不应沉湎淫欲。}

Etam ādīnavaṃ ñatvā, muni pubbāpare idha;\\
ekacariyaṃ daḷhaṃ kayirā, na nisevetha methunaṃ. %\hfill\textcolor{gray}{\footnotesize 8}

\begin{enumerate}\item \textbf{牟尼了知了这之前、之后的过患},即牟尼了知了在以「且先前的声誉和名声,他都已丧失」开始所说的于此教法内从之前的沙门身份到之后的还俗身份的过患。\end{enumerate}

\subsection\*{\textbf{829} {\footnotesize 〔PTS 822〕}}

\textbf{「他唯应修学远离,这是圣者们的最上,\\}
\textbf{「不应以此认为是最胜,他便在涅槃的跟前。}

Vivekañ ñeva sikkhetha, etaṃ ariyānam uttamaṃ;\\
na tena seṭṭho maññetha, sa ve nibbānasantike. %\hfill\textcolor{gray}{\footnotesize 9}

\begin{enumerate}\item \textbf{这是圣者们的最上},即此远离之行是佛等圣者们的最上,所以他唯应修学远离之意。\textbf{不应以此认为是最胜},且不应以此远离认为自己「我乃最胜」,即是说不应以此为傲。\end{enumerate}

\subsection\*{\textbf{830} {\footnotesize 〔PTS 823〕}}

\textbf{「对空无而行的牟尼、不关切爱欲者、\\}
\textbf{「已度过暴流者,系缚于爱欲的世人(徒有)羡慕。」}

Rittassa munino carato, kāmesu anapekkhino;\\
oghatiṇṇassa pihayanti, kāmesu gadhitā pajā” ti. %\hfill\textcolor{gray}{\footnotesize 10}

\begin{enumerate}\item \textbf{空无},即远离、摆脱了身恶行等。\textbf{系缚于爱欲的世人羡慕已度过暴流者},即固著于物欲的有情羡慕这已度过四暴流者,如同负债者之于无债者,即以阿罗汉为顶点完成了开示。当开示终了,低舍即证须陀洹果,随后出家,证得了阿罗汉。\end{enumerate}

\begin{center}\vspace{1em}低舍弥勒经第七\\Tissametteyyasuttaṃ sattamaṃ.\end{center}