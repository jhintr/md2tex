\section{阿耆多学童问}

\begin{center}Ajita Māṇava Pucchā\end{center}\vspace{1em}

\begin{itemize}\item 案,经题中的\textbf{学童问},义注中均作\textbf{经},从此至\textbf{褐者学童问}皆同。\end{itemize}

\subsection\*{\textbf{1039} {\footnotesize 〔PTS 1032〕}}

\textbf{「世间被什么覆蔽?」尊者阿耆多说,「为何它不显明?\\}
\textbf{「你说什么是它的染著?什么是它的大怖畏?」}

“Kena’ssu nivuto loko, \textit{(icc āyasmā Ajito)} kena’ssu nappakāsati;\\
ki’ssābhilepanaṃ brūsi, kiṃ su tassa mahabbhayaṃ”. %\hfill\textcolor{gray}{\footnotesize 1}

\subsection\*{\textbf{1040} {\footnotesize 〔PTS 1033〕}}

\textbf{「世间被无明覆蔽,阿耆多!」世尊说,「由悭贪、放逸而不显明,\\}
\textbf{「我说渴望是染著,苦是它的大怖畏。」}

“Avijjāya nivuto loko, \textit{(Ajitā ti Bhagavā)} vevicchā pamādā nappakāsati;\\
jappābhilepanaṃ brūmi, dukkham assa mahabbhayaṃ”. %\hfill\textcolor{gray}{\footnotesize 2}

\begin{enumerate}\item \textbf{由悭贪、放逸而不显明},由悭吝之因、放逸之因而不显明,因为悭吝使其不以布施等功德显明,放逸(使其不)以戒等(显明)。\end{enumerate}

\subsection\*{\textbf{1041} {\footnotesize 〔PTS 1034〕}}

\textbf{「众流四处漂荡,」尊者阿耆多说,「什么是众流的障碍?\\}
\textbf{「请说众流的防护!众流应以什么来遮止?」}

“Savanti sabbadhi sotā, \textit{(icc āyasmā Ajito)} sotānaṃ kiṃ nivāraṇaṃ;\\
sotānaṃ saṃvaraṃ brūhi, kena sotā pidhiyyare”. %\hfill\textcolor{gray}{\footnotesize 3}

\begin{enumerate}\item \textbf{众流四处漂荡},即渴爱等的众流于一切色等处漂荡。\textbf{什么是障碍},即对于它们,什么是障盖、什么是守护?\textbf{请说防护},即请说它们的被称为障碍的防护,以此而问有余之舍弃。\textbf{众流应以什么来遮止},即众流以何法来遮止、截断,以此而问无余之舍弃。\end{enumerate}

\subsection\*{\textbf{1042} {\footnotesize 〔PTS 1035〕}}

\textbf{「世间的这些众流,阿耆多!」世尊说,「念是它们的障碍,\\}
\textbf{「我说众流的防护,它们应以慧来遮止。」}

“Yāni sotāni lokasmiṃ, \textit{(Ajitā ti Bhagavā)} sati tesaṃ nivāraṇaṃ;\\
sotānaṃ saṃvaraṃ brūmi, paññāy’ete pidhiyyare”. %\hfill\textcolor{gray}{\footnotesize 4}

\begin{enumerate}\item \textbf{念是它们的障碍},即与毗婆舍那相应、探求善法之趣的念,是这些众流的障碍。\textbf{应以慧来遮止},即于色等,这些众流应以有余量之通达无常等的道慧来完全地遮止。\end{enumerate}

\subsection\*{\textbf{1043} {\footnotesize 〔PTS 1036〕}}

\textbf{「即此慧与念,」尊者阿耆多说,「以及名色,先生!\\}
\textbf{「既然问到,请告诉我!这在何处灭去?」}

“Paññā c’eva sati yañ ca, \textit{(icc āyasmā Ajito)} nāmarūpañ ca Mārisa;\\
etaṃ me puṭṭho pabrūhi, katth’etaṃ uparujjhati”. %\hfill\textcolor{gray}{\footnotesize 5}

\begin{enumerate}\item 在问颂中,你所说的\textbf{慧与念},以及残余的\textbf{名色},这些全部在\textbf{何处}灭去。\end{enumerate}

\subsection\*{\textbf{1044} {\footnotesize 〔PTS 1037〕}}

\textbf{「你问的这问题,阿耆多!我要对你说,\\}
\textbf{「于此,名与色无余地灭去,\\}
\textbf{「由识的灭,它即在此灭去。」}

“Yam etaṃ pañhaṃ apucchi, Ajita taṃ vadāmi te;\\
yattha nāmañ ca rūpañ ca, asesaṃ uparujjhati;\\
viññāṇassa nirodhena, etth’etaṃ uparujjhati”. %\hfill\textcolor{gray}{\footnotesize 6}

\begin{enumerate}\item 在答颂中,因为慧与念都归于名法,所以不再个别叙述。其略义为,\textbf{由识的灭},即其同时,不先、不后,\textbf{它即在此灭去},即在识灭处灭去,由识灭而彼灭,即是说不过于此。\end{enumerate}

\subsection\*{\textbf{1045} {\footnotesize 〔PTS 1038〕}}

\textbf{「于此,那些已察知法者,与种种有学,\\}
\textbf{「既然问到,请贤者告诉我他们的威仪!先生!」}

“Ye ca saṅkhātadhammāse, ye ca sekhā puthū idha;\\
tesaṃ me nipako iriyaṃ, puṭṭho pabrūhi Mārisa”. %\hfill\textcolor{gray}{\footnotesize 7}

\begin{enumerate}\item 至此,已由「苦是它的大怖畏」阐明了苦谛,由「这些众流」阐明了集谛,由「它们应以慧来遮止」阐明了道谛,由「无余地灭去」阐明了灭谛,如是听闻了四圣谛后,未证得圣地,再问有学与无学的行道而说此颂。这里,\textbf{已察知法者},即以无常等审虑诸法者,于此是阿罗汉的同义语。\textbf{有学},即修学于戒等的其余圣补特伽罗。\textbf{种种},即众多七类人。\textbf{请贤者告诉我他们的威仪},即请贤明的智者您告诉我这些有学、无学的行道。\end{enumerate}

\subsection\*{\textbf{1046} {\footnotesize 〔PTS 1039〕}}

\textbf{「他不应贪求爱欲,他不应污浊其意,\\}
\textbf{「善巧于一切法,比丘应具念而游行。」}

“Kāmesu nābhigijjheyya, manasānāvilo siyā;\\
kusalo sabbadhammānaṃ, sato bhikkhu paribbaje” ti. %\hfill\textcolor{gray}{\footnotesize 8}

\begin{enumerate}\item 于是,因为有学应以欲贪盖为首,舍弃一切烦恼,故世尊以前半颂显明有学的行道。其义为,\textbf{他不应}以烦恼欲\textbf{贪求}于物\textbf{欲},且舍弃身恶行等污浊其意的法,\textbf{他不应污浊其意}。然而,因为无学由以无常等考量一切诸行等,而善巧于一切法,且以身随观念等而具念,由破碎了有身见等而得比丘性已,游行于一切威仪路,故以后半颂显明无学的行道。
\item 如是,世尊以阿罗汉为顶点完成了开示。当开示终了,阿耆多与一千弟子即住于阿罗汉性,而其余数千人生起了法眼。伴随着阿罗汉的证得,尊者阿耆多及其千名弟子的羚羊皮、萦发、树皮衣等都消失了,而全都持着神变所成的衣钵,发长二指,成为「来!比丘」,礼敬着世尊,合掌落坐。\end{enumerate}

\begin{center}\vspace{1em}阿耆多学童问第一\\Ajitamāṇavapucchā paṭhamā.\end{center}