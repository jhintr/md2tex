\section{那罗迦经}

\begin{center}Nālaka Sutta\end{center}\vspace{1em}

\subsection\*{\textbf{685} {\footnotesize 〔PTS 679〕}}

\textbf{阿私陀仙人在昼住处,见到庆喜、欢欣的三十三天众,\\}
\textbf{帝释因陀与衣服洁净的诸天持着白布,正热烈赞美着。\footnote{译文调整了行间的语序,分别对应巴利的 d-a-b-c。原文第二句的 Sakkañ ca,PTS 本作 sakkacca,译文即「……三十三天众恭敬了因陀,衣服洁净的诸天……」,若据语境并对照义注,当以缅甸本较妥。}}

Ānandajāte tidasagaṇe patīte, Sakkañ ca Indaṃ sucivasane ca deve;\\
dussaṃ gahetvā atiriva thomayante, Asito isi addasa divāvihāre. %\hfill\textcolor{gray}{\footnotesize 1}

\begin{enumerate}\item 缘起为何?据说,阿私陀仙人的外甥,名为那罗迦的苦行者,见到莲花上世尊的弟子行寂默的行道,便希求如是的状态,从此开始,圆满了十万劫的波罗蜜,在世尊转法轮后的第七天,便以「已知这」等二颂\footnote{即第 705~706 颂。}而问了寂默的行道。世尊则以「我将向你说明寂默」等方法对他解释\footnote{即第 707 颂及以下。}。而在世尊般涅槃后,由大迦叶尊者举行结集,阿难尊者被问及这寂默的行道,以及那罗迦因何、何时被激起而问了世尊,欲阐明这一切,便说了「阿私陀仙人」等二十序颂。这一切被称为「那罗迦经」。
\item 这里,\textbf{庆喜},即繁荣、增长。\textbf{欢欣},即满足。或者,庆喜即喜悦,欢欣即愉快。\textbf{衣服洁净},即衣服无染污。因为诸天的衣服来自如意树,不染尘埃。\textbf{持着白布},即举着天布,由与此世的白布相似而依世俗称为「白布」。\textbf{阿私陀仙人},仙人由黑的身色而得名如是\footnote{这是从语源上释名,阿私陀 \textit{Asita} 即「非白 \textit{a-sita}」。}。其余词义自明。
\item 而连结为:据说,他是净饭之父狮颚王的祭司,在净饭未灌顶时是其业师,灌顶时则仍为祭司。他早晚前去给侍国王,国王如少年时一样,不行跪拜,仅作合掌。据说这是已经灌顶的释迦诸王的法性。祭司因此嫌厌道:「大王!我将出家。」国王得知了他的决定,便请求:「那么,老师!当仍住在我的庭园内,好让我经常能看到你。」他答道「那就如是」,出家为苦行者,受着国王的护持,仍住在庭园内,在作了遍的预作后,便生起八等至和五神通。
\item 从此以后,他在王家用过餐后,便去到雪山或四大王天等的某个居处昼住。于是,一天,他去到三十三天的居处,进入宝殿,坐在天宝座上领受三摩地之乐,当哺食出起、站在殿门四处观察时,便见到帝释为首的诸天正在六十由旬的大道上嬉戏,挥舞着衣,说着赞美菩萨功德的话语。因此阿难尊者说「阿私陀仙人……赞美着」。\end{enumerate}

\subsection\*{\textbf{686} {\footnotesize 〔PTS 680〕}}

\textbf{见到欢喜、踊跃的诸天后,表了敬意,他便在那里说道:\\}
\textbf{「众天人为何异常喜形于色?缘何持着白布而高兴\footnote{高兴 \textit{ramayatha}:PTS 本作「挥舞 \textit{bhamayatha}」,似更佳。}?}

Disvāna deve muditamane udagge, cittiṃ karitvāna idam avoca tattha;\\
“kiṃ devasaṅgho atiriva kalyarūpo, dussaṃ gahetvā ramayatha kiṃ paṭicca. %\hfill\textcolor{gray}{\footnotesize 2}

\begin{enumerate}\item 随后,他如是「见到欢喜……而高兴」。这里,\textbf{踊跃},即身体雀跃。\textbf{表了敬意},即表了尊敬。\textbf{喜形于色},即满足之色。其余之义自明。\end{enumerate}

\subsection\*{\textbf{687} {\footnotesize 〔PTS 681〕}}

\textbf{「即便当时与阿修罗的战争,修罗战胜,阿修罗败北,\\}
\textbf{「那时也没有如这般身毛竖立,众神得见什么希有而喜悦?}

Yadā pi āsī asurehi saṅgamo, jayo surānaṃ asurā parājitā;\\
tadā pi n’etādiso lomahaṃsano, kim abbhutaṃ daṭṭhu marū pamoditā. %\hfill\textcolor{gray}{\footnotesize 3}

\begin{enumerate}\item 现在,「即便当时……」等颂的连结自明。而其词义,先就第一颂中的\textbf{修罗战胜},即诸天战胜。为显明此,当先知此前事:据说,帝释曾是三十三人之首、名为摩伽的学童,在摩竭陀国摩遮罗村居住,在圆满了七禁法\footnote{七禁法 \textit{vatapada}:杂阿含经第 1104 经作「受持七种受者……谓天帝释本为人时,供养父母及家诸尊长,和颜软语,不恶口,不两舌,常真实言,于悭吝世间虽在居家而不悭惜,行解脱施,勤施,常乐行施,施会供养,等施一切」,\textbf{相应部}第 11:11 禁法经则无「不恶口」,而作「不忿怒,若我生起忿怒,则能迅速调伏」。}后,便与随从一起转生到三十三天的居处。随后,先前的诸天说「做客的天子们来了,我们去恭敬他们」,便授以天莲花,并以一半国土相邀请。帝释不满足于一半国土,说服了自己的随从,一天,当他们酒醉后,便捉住他们的脚,扔到须弥山脚下。
\item 在须弥山的谷底,有一万由旬的阿修罗的居处对他们现起,装饰以遍覆的波利质多树和多彩的波吒梨树。随后,他们在恢复神智后,不见三十三天的居处,想到「哎!咄!我们因饮酒的过失而亡失,现在我们再不饮酒,饮非酒,我们现在不是修罗,我们现在成了阿修罗\footnote{修罗 \textit{sura} 与「酒 \textit{surā}」有字面的联系。}」。从此以后,即生起「阿修罗」之名,想道「噫!现在让我们和诸天一战」,便从四周爬上须弥山。
\item 随后,帝释出战,与阿修罗角力,再次(把阿修罗)扔到海里,教人塑了与自己相似的因陀像,安置在四门。随后,阿修罗们想「这帝释确实不放逸,总为守卫而立」,只好返回城中。随后,诸天为宣告自己的胜利,在大道上挥舞着衣,嬉戏庆祝。
\item 此处,阿私陀由于能忆念过去、未来四十劫,当转向于「过去因何也有这样的嬉戏」时,见到诸天在天与阿修罗的战争中得胜,便说:\begin{quoting}即便当时与阿修罗的战争,修罗战胜,阿修罗败北,\\那时也没有如这般身毛竖立,\end{quoting}即便那时也没有如这般身毛竖立的喜悦。\textbf{众神得见什么希有而喜悦},而今天,诸天见到了什么希有,竟如是喜悦?\end{enumerate}

\subsection\*{\textbf{688} {\footnotesize 〔PTS 682〕}}

\textbf{「他们吹哨、歌唱、演奏,他们击掌、舞蹈,\\}
\textbf{「我问你们,弥卢顶的居民,请快除去我的疑虑,诸君!」}

Seḷenti gāyanti ca vādayanti ca, bhujāni phoṭenti ca naccayanti ca;\\
pucchāmi vo’haṃ Merumuddhavāsine, dhunātha me saṃsayaṃ khippa mārisā”. %\hfill\textcolor{gray}{\footnotesize 4}

\begin{enumerate}\item 在第二颂中,\textbf{吹哨},即用嘴发出哨声。\textbf{歌唱}各种歌咏,\textbf{演奏}六万八千乐器。\textbf{击掌},即拍手。\textbf{我问你们},即便经自己转向也能知晓,为欲听闻他们的话语而问。\textbf{弥卢顶的居民},即在须弥山顶上的住者。因为须弥山的谷底是一万由旬的阿修罗的居处,中间是二千小洲围绕的四大洲,上部是一万由旬的三十三天的居处,所以诸天被称为「弥卢顶的居民」。\textbf{诸君}是称呼诸天,即是说无苦、无病。\end{enumerate}

\subsection\*{\textbf{689} {\footnotesize 〔PTS 683〕}}

\textbf{「这菩萨,无等的贵宝,为了人世间的利乐,出生\\}
\textbf{「在蓝毗尼地方的释氏村落,我们因此满足,异常喜形于色。}

“So bodhisatto ratanavaro atulyo, manussaloke hitasukhatthāya jāto;\\
Sakyāna gāme janapade Lumbineyye, ten’amha tuṭṭhā atiriva kalyarūpā. %\hfill\textcolor{gray}{\footnotesize 5}

\begin{enumerate}\item 然后,诸天对其告知此事,在所说的第三颂中,\textbf{菩萨},即能觉悟的有情,应去往正等正觉的有情。\textbf{贵宝},即作为高贵的珍宝。\textbf{我们因此满足},即以此原因,我们满足。据说,他们的意趣为:因为他证得佛性后,将开示如是之法,好让我们及其他天众证得学、无学之地,人类听闻其法后,若不能涅槃者,则行了布施等,将遍满天界。这里,「满足、喜形于色」二词虽然意义无别,当知是为了解答「众神得见什么希有而喜悦、众天人为何异常喜形于色」这两个问题而说。\end{enumerate}

\subsection\*{\textbf{690} {\footnotesize 〔PTS 684〕}}

\textbf{「他是一切有情之最上,至高之人,人中公牛,一切造物之最上,\\}
\textbf{「将在名为仙人的林中使轮转起,如同怒吼的狮子、有力的兽王。」}

So sabbasattuttamo aggapuggalo, narāsabho sabbapajānam uttamo;\\
vattessati cakkam Isivhaye vane, nadaṃ va sīho balavā migābhibhū”. %\hfill\textcolor{gray}{\footnotesize 6}

\begin{enumerate}\item 现在,在为显明之所以满足于菩萨出生之意趣而说的第四颂中,以\textbf{有情}摄人天,以\textbf{造物}摄其余趣,如是以二词显示五趣中的最胜之相。因为即便是具有无恐惧等德的狮子等畜生,他亦能超胜之,所以说是\textbf{造物之最上}。而在人天中,那些为自身的利益而行道的四补特伽罗之中,他是其中为双方利益而行道的\textbf{至高之人},由在人中与公牛相似故,为\textbf{人中公牛},因此,他们在赞美他时,还说了此二词。\end{enumerate}

\subsection\*{\textbf{691} {\footnotesize 〔PTS 685〕}}

\textbf{听闻此声,他急忙降下,然后进入净饭的居处,\\}
\textbf{在那里坐下后,对众释氏说:「童子在何处?我也欲见。」}

Taṃ saddaṃ sutvā turitam avasarī so, Suddhodanassa tada bhavanaṃ upāvisi;\\
nisajja tattha idam avocāsi Sakye, “kuhiṃ kumāro aham api daṭṭhukāmo”. %\hfill\textcolor{gray}{\footnotesize 7}

\begin{enumerate}\item 在第五颂中,\textbf{此声},即此诸天所说的言语之声。\end{enumerate}

\subsection\*{\textbf{692} {\footnotesize 〔PTS 686〕}}

\textbf{随后,众释氏将孩子示与名为阿私陀者,童子如炽热的金子\\}
\textbf{在坩埚里善加锤炼一般,光辉闪耀,肤色胜妙。\footnote{译文调整了行间的语序,分别对应巴利的 d-a-b-c。}}

Tato kumāraṃ jalitam iva suvaṇṇaṃ, ukkāmukhe va sukusalasampahaṭṭhaṃ;\\
daddallamānaṃ siriyā anomavaṇṇaṃ, dassesu puttaṃ Asitavhayassa Sakyā. %\hfill\textcolor{gray}{\footnotesize 8}

\begin{enumerate}\item 在第六颂中,\textbf{随后},即阿私陀话语的无间。\textbf{善加锤炼},即被极善巧的金匠锤炼,意即被锤炼者熔化。\textbf{闪耀},即发光。\textbf{名为阿私陀者},即名为阿私陀,又名黑天的仙人。\end{enumerate}

\subsection\*{\textbf{693} {\footnotesize 〔PTS 687〕}}

\textbf{见到童子如燃烧的火焰,如清净的众星之牛行于空中,\\}
\textbf{如照耀的太阳在秋日里破出云层,他庆喜,得广大喜。}

Disvā kumāraṃ sikhim iva pajjalantaṃ, tārāsabhaṃ va nabhasigamaṃ visuddhaṃ;\\
suriyaṃ tapantaṃ sarada-r-iv’abbhamuttaṃ, ānandajāto vipulam alattha pītiṃ. %\hfill\textcolor{gray}{\footnotesize 9}

\begin{enumerate}\item 在第七颂中,\textbf{众星之牛},即与众星之牛相似,意即月亮。\textbf{清净},即无有云翳等的杂染。\textbf{庆喜},即因仅仅听闻生起之喜而喜。\textbf{得喜},即见后再次得喜。\end{enumerate}

\subsection\*{\textbf{694} {\footnotesize 〔PTS 688〕}}

\textbf{众神在空中擎了多枝且千轮的伞盖,\\}
\textbf{金柄的拂尘上下翻飞,却不见握拂尘伞盖者。}

Anekasākhañ ca sahassamaṇḍalaṃ, chattaṃ marū dhārayum antalikkhe;\\
suvaṇṇadaṇḍā vītipatanti cāmarā, na dissare cāmarachattagāhakā. %\hfill\textcolor{gray}{\footnotesize 10}

\begin{enumerate}\item 此后,在为显示诸天总是恭敬菩萨而说的第八颂中,\textbf{多枝},即多肋。\textbf{千轮},即拥有赤金所造的千轮。\textbf{伞盖},即天白伞盖。\textbf{上下翻飞},即扇着身体,落下又举起。\end{enumerate}

\subsection\*{\textbf{695} {\footnotesize 〔PTS 689〕}}

\textbf{名为黑色光辉的萦发仙人见到如黄毯上的金饰一般,\\}
\textbf{及被擎在头顶的白伞盖,便心生踊跃、悦意而接住。}

Disvā jaṭī Kaṇhasirivhayo isi, suvaṇṇanikkhaṃ viya paṇḍukambale;\\
setañ ca chattaṃ dhariyanta muddhani, udaggacitto sumano paṭiggahe. %\hfill\textcolor{gray}{\footnotesize 11}

\begin{enumerate}\item 在第九颂中,\textbf{名为黑色光辉},即以「黑色」之词与「光辉」之词得称。据说,他还被称为、呼为「光辉的黑者」,即是说称呼。\textbf{黄毯},即染过的毯子。且此处,从主题上应说「童子」,或者应补充文本。
\item 且在前颂\footnote{即第 693 颂。}中,就未至伸手所及而说「见到」,然而此处为了接住,已被带至伸手所及,所以又有「见到」一词。或者,前者意在得到见的喜悦,故在颂末说「得广大喜」,此则意在接住,故在结尾说「悦意而接住」。且前者只与童子相关,此则还与白伞盖相关。
\item \textbf{见到},即见到如价值十万的犍陀罗黄毯上的金饰一般的童子,及「众神……伞盖」所说的被擎在头顶的白伞盖。而有些人说「这是就人类的伞盖而说的」。因为正如诸天,如是人类也手持伞盖、拂尘、孔雀翎、多罗扇、马尾麈趋近大人。即便如是,此语亦未有任何见长,所以如上所说便好。
\item \textbf{接住},即以双手接住。据说,人们带童子去礼拜仙人,而他的双足调转,站在仙人的头上。他见到这希有之事,便心生踊跃、悦意而接住。\end{enumerate}

\subsection\*{\textbf{696} {\footnotesize 〔PTS 690〕}}

\textbf{而接住释氏的头牛后,通晓相与颂诗者审视着,\\}
\textbf{心生净喜,便高声唱言:「这是无上的二足尊。」}

Paṭiggahetvā pana Sakyapuṅgavaṃ, jigīsako lakkhaṇamantapāragū;\\
pasannacitto giram abbhudīrayi, “anuttar’āyaṃ dvipadānam uttamo”. %\hfill\textcolor{gray}{\footnotesize 12}

\begin{enumerate}\item 在第十颂中,\textbf{审视},即寻求、遍求,即是说考察\footnote{审视 \textit{jigīsako}:Norman 说结合第 706 颂出现的 jigīsato,二词解作「渴求 \textit{desiring, longing for}」较妥。}。\textbf{通晓相与颂诗},即通晓相与吠陀。据说,在面对自己的大士的脚掌上,他见到轮(相),据此,当审视其它诸相,见到一切相的成就后,便了知「他必定会成佛」而如是说。\end{enumerate}

\subsection\*{\textbf{697} {\footnotesize 〔PTS 691〕}}

\textbf{然后,随念着自己的趣向,他脸色凝重,落下泪水,\\}
\textbf{众释氏见后,对悲泣的仙人说:「童子是否有什么障难?」}

Ath’attano gamanam anussaranto, akalyarūpo gaḷayati assukāni;\\
disvāna Sakyā isim avocuṃ rudantaṃ, “no ce kumāre bhavissati antarāyo”. %\hfill\textcolor{gray}{\footnotesize 13}

\begin{enumerate}\item 在第十一中,\textbf{自己的趣向},即以结生至无色趣。\textbf{他脸色凝重,落下泪水},即念及自己投生至无色,想「那时,我将不能听闻他法的开示了」,色有不满,为强力的忧愁所制而忧恼,滚落、落下泪水,文本也作 garayati。
\item 但是,如果他将心倾向于色有,不能投生到那里吗?为何如此悲泣?——并非不能投生,而是由于不善巧,他不知此规则。
\item 设问:即便如此,由为所得的等至镇伏故,生起忧恼是否仍不适当?——非也,由仅是镇伏故。因为以道的修习,烦恼正断,方不生起,而对得等至者,则会因强力的缘而生起。
\item 设问:当生起烦恼时,由退失禅那故,他如何能至无色趣?——由以少许努力即可再次证得故。因为对得等至者,当生起烦恼时,不至于有强力的违犯,烦恼激流甫一平息,以少许努力即可再次证此殊胜,难以了知「他们退失了殊胜」,他即是如此。\end{enumerate}

\subsection\*{\textbf{698} {\footnotesize 〔PTS 692〕}}

\textbf{仙人见后,对凝重的众释氏说:「我并非念及童子的不利,\\}
\textbf{「而且他也不会有障难,他非下劣之辈,请你们放心!}

Disvāna Sakye isi-m-avoca akalye, “nāhaṃ kumāre ahitam anussarāmi;\\
na cāpi-m-assa bhavissati antarāyo, na orak’āyaṃ adhimānasā bhavātha. %\hfill\textcolor{gray}{\footnotesize 14}

\begin{enumerate}\item 在第十二中,\textbf{他非下劣之辈},即他非下劣、有限之辈。他是就之后几颂中将说的佛陀之相而说。\end{enumerate}

\subsection\*{\textbf{699} {\footnotesize 〔PTS 693〕}}

\textbf{「这童子将得证至高的等觉,得见最上的清净,他将使法轮\\}
\textbf{「转起,他怜悯众人的利益,他的梵行将广为传播。}

Sambodhiyaggaṃ phusissat’āyaṃ kumāro, so dhammacakkaṃ paramavisuddhadassī;\\
vattessat’āyaṃ bahujanahitānukampī, vitthārik’assa bhavissati brahmacariyaṃ. %\hfill\textcolor{gray}{\footnotesize 15}

\begin{enumerate}\item 在第十三中,\textbf{至高的等觉},即一切知智。因为它以无颠倒而完全觉悟故被称为等觉,以处处无碍而为一切智的最上故被称为至高。\textbf{得证},即圆满。\textbf{得见最上的清净},即得见涅槃。因为它由究竟清净故为最上的清净。\textbf{梵行},即教法。\end{enumerate}

\subsection\*{\textbf{700} {\footnotesize 〔PTS 694〕}}

\textbf{「而我的寿命在此所剩无多,在此期间我将死亡,\\}
\textbf{「我将听不到无比坚韧者的法,因此我压抑、沉沦、痛苦。」}

Mamañ ca āyu na ciram idhāvaseso, ath’antarā me bhavissati kālakiriyā;\\
so’haṃ na sossaṃ asamadhurassa dhammaṃ, ten’amhi aṭṭo byasanaṅgato aghāvī”. %\hfill\textcolor{gray}{\footnotesize 16}

\begin{enumerate}\item 在第十四中,\textbf{在此期间},即是说在得证等觉之前。\textbf{无比坚韧},即无比精进。\textbf{压抑}即恼患,\textbf{沉沦}即失去快乐,\textbf{痛苦}即受苦,这全都是就生起忧恼而说。因为他因忧而恼患,由快乐的衰损而沉沦,即是说由失去快乐故,并因此心痛而痛苦。\end{enumerate}

\subsection\*{\textbf{701} {\footnotesize 〔PTS 695〕}}

\textbf{让众释氏生了广大之喜,梵行者便从后宫出来,\\}
\textbf{为怜悯自己的外甥,便激励以无比坚韧者的法。}

So Sākiyānaṃ vipulaṃ janetvā pītiṃ, antepuramhā niggamā brahmacārī;\\
so bhāgineyyaṃ sayaṃ anukampamāno, samādapesi asamadhurassa dhamme. %\hfill\textcolor{gray}{\footnotesize 17}

\begin{enumerate}\item 在第十五中,\textbf{出来},即离开。且如是离开后,\textbf{便激励自己的外甥},即是说自己姐妹的孩子,他了知到自己的少寿,而幺妹的孩子,学童那罗迦已积集了福德,以自力了知到「他长大后也会放逸」而怜悯他,便去到妹妹家:「那罗迦在哪里?」「尊者!他在外面玩耍。」他便命道:「把他带来!」即于彼时令其出家为苦行者而激励、教诫、训诫。\end{enumerate}

\subsection\*{\textbf{702} {\footnotesize 〔PTS 696〕}}

\textbf{「当你从别人听到『佛陀』这声音,『证等觉者开显法之道』,\\}
\textbf{「你应去到那里,遍问教义,在彼世尊处行梵行。」}

“Buddho ti ghosaṃ yada parato suṇāsi, sambodhipatto vivarati dhammamaggaṃ;\\
gantvāna tattha samayaṃ paripucchamāno, carassu tasmiṃ Bhagavati brahmacariyaṃ”. %\hfill\textcolor{gray}{\footnotesize 18}

\begin{enumerate}\item 如何?便说了第十六颂。这里,\textbf{法之道},即最上法涅槃之道,或作「至高之法 \textit{dhammaṃ aggaṃ}」,即与行道俱的涅槃。\textbf{那里},即他跟前。\textbf{梵行},即沙门法。\end{enumerate}

\subsection\*{\textbf{703} {\footnotesize 〔PTS 697〕}}

\textbf{经他这样心怀利益、于未来见最上清净者的训诫,\\}
\textbf{那罗迦积集了大量福德,期待胜者而别住,守护根门。}

Tenānusiṭṭho hitamanena tādinā, anāgate paramavisuddhadassinā;\\
so Nālako upacitapuññasañcayo, Jinaṃ patikkhaṃ parivasi rakkhitindriyo. %\hfill\textcolor{gray}{\footnotesize 19}

\begin{enumerate}\item 在第十七中,\textbf{这样},即于此安立,意即于此镇伏烦恼、得三摩地之时,镇伏烦恼、心得等持。\textbf{于未来见最上清净者},即由见到「这那罗迦未来时在世尊跟前将见到最上清净的涅槃」故,这仙人以此方法被称为「于未来见最上清净者」。\textbf{积集了大量福德},即从莲花上(世尊)开始,已行了大量福德。\textbf{别住},即出家后,以苦行而住。\textbf{守护根门},即守护耳根。据说,从此以后,他想「若入水后耳根退失,将无缘闻法」,便不潜入水中。\end{enumerate}

\subsection\*{\textbf{704} {\footnotesize 〔PTS 698〕}}

\textbf{听闻到胜者转动最胜之轮的声音,前往并见到了仙人中的牛王,既已净喜,\\}
\textbf{便问了高贵的牟尼最胜的寂默,当名为阿私陀者的教法来临时。}

Sutvāna ghosaṃ Jinavaracakkavattane, gantvāna disvā isinisabhaṃ pasanno;\\
moneyyaseṭṭhaṃ munipavaraṃ apucchi, samāgate Asitāvhayassa sāsane ti. %\hfill\textcolor{gray}{\footnotesize 20}

\begin{enumerate}\item 在第十八中,\textbf{听闻到声音},即那罗迦如是别住时,世尊渐次证得了等觉,在波罗奈转动法轮,他听闻到诸天欲求自己的义利,以「法轮已被世尊转动,彼正等正觉者、世尊已然出世」等方式前来宣告的\textbf{胜者转动最胜之轮}的声音。\textbf{前往并见到了仙人中的牛王},经七天,诸天兴起寂默的喧哗\footnote{即五种喧哗之一,见\textbf{吉祥经}注。},他在第七天到达仙人堕处,见到如牛王一般的仙人中的牛王,世尊则怀着「那罗迦将要前来,我将向他开示法」的意向坐在最胜佛座上。\textbf{既已净喜},即一见即心生净喜。\textbf{最胜的寂默},即是说最上智、道智。\textbf{当名为阿私陀者的教法来临时},即到达了阿私陀仙人教诫的时机。因为他训诫说「当他开显法之道时,你应去到那里,遍问教义,在彼世尊处行梵行」,而这便是此时,因此说「当名为阿私陀者的教法来临时」。其余于此自明。\end{enumerate}

\begin{center}\vspace{1em}序颂终\\Vatthugāthā niṭṭhitā.\end{center}

\subsection\*{\textbf{705} {\footnotesize 〔PTS 699〕}}

\textbf{「已知这阿私陀的话语属实,\\}
\textbf{「我问问你,乔达摩!已度一切法者。}

“Aññātam etaṃ vacanaṃ, Asitassa yathātathaṃ;\\
taṃ taṃ Gotama pucchāmi, sabbadhammāna pāraguṃ. %\hfill\textcolor{gray}{\footnotesize 21}

\begin{enumerate}\item 在二首问颂中,\textbf{已知这},即我已了知这。\textbf{属实},即无颠倒。意趣为何?阿私陀了知到「这童子将得证至高的等觉」,便对我说「当你从别人听到『佛陀』这声音,『证等觉者开显法之道』」,今天亲眼见到世尊,我已知这阿私陀的话语的确属实。\textbf{已度一切法者},即以雪山经中所说的六种行相\footnote{已度的六种行相,见\textbf{雪山经}第 169 颂注。}到达一切法的彼岸者。\end{enumerate}

\subsection\*{\textbf{706} {\footnotesize 〔PTS 700〕}}

\textbf{「对已步入无家、寻求行乞食者,\\}
\textbf{「牟尼!既然问到,请告诉我最上之法的寂默。」}

Anagāriy’upetassa, bhikkhācariyaṃ jigīsato;\\
muni pabrūhi me puṭṭho, moneyyaṃ uttamaṃ padaṃ”. %\hfill\textcolor{gray}{\footnotesize 22}

\begin{enumerate}\item \textbf{已步入无家},即出家之义。\textbf{寻求行乞食者}\footnote{寻求 \textit{jigīsato}:Norman 作「渴求」,见第 696 颂的注。},即遍求为圣者所习行的无染的乞食之行。\textbf{寂默},即牟尼的所有。\textbf{最上之法},即最上的行道。其余于此自明。\end{enumerate}

\subsection\*{\textbf{707} {\footnotesize 〔PTS 701〕}}

\textbf{「我将向你说明寂默,」世尊说,「难为、难以征服,\\}
\textbf{「噫!我将对你宣说,请约束自己!请你坚强!}

“Moneyyaṃ te upaññissaṃ, \textit{(iti Bhagavā)} dukkaraṃ durabhisambhavaṃ;\\
handa te naṃ pavakkhāmi, santhambhassu daḷho bhava. %\hfill\textcolor{gray}{\footnotesize 23}

\begin{enumerate}\item 于是,世尊既如是被问,便以「我将向你说明寂默」等方法向他解释了寂默的行道。这里,\textbf{说明},即开显、令知之义。\textbf{难为、难以征服},即是说难以作为,且当作为时难以克服、忍耐。此中的意趣为:我将令你知晓寂默,若它易于作为或征服,然而却是如此的难为、难以征服,从凡夫时开始,应不起染污心而行道。因为每个佛陀的每个弟子都如是作为、征服。
\item 如是,世尊显示了寂默的难为之相与难以征服之性,欲对他说,且令那罗迦生起勇猛,便说了后半颂。这里,\textbf{噫},即决定之义的不变词。\textbf{请约束自己},即请以能为难为的精进之支撑来支持自己。\textbf{请你坚强},即以能忍耐难以征服的不松懈之勇猛而坚定。这说的是什么?因为你已积集福德资粮,所以我下了决定,即便是如是的难为、难以征服,我将对你宣说这寂默,请约束自己!请你坚强!\end{enumerate}

\subsection\*{\textbf{708} {\footnotesize 〔PTS 702〕}}

\textbf{「在村中受到谩骂、礼拜,应保持同分,\\}
\textbf{「应守护意的嗔恚,平静、不高举而行。}

Samānabhāgaṃ kubbetha, gāme akkuṭṭha-vanditaṃ;\\
manopadosaṃ rakkheyya, santo anuṇṇato care. %\hfill\textcolor{gray}{\footnotesize 24}

\begin{enumerate}\item 如是,欲说最上损减的寂默仪法,在以约束、坚强敦促那罗迦后,首先,为显示舍弃与村落相关的烦恼,说了前半颂。这里,\textbf{同分},即平等分、如一、离于分别。
\item 现在,为显示如何保持同分的方法,说了后半颂。其义为:受到谩骂\textbf{应守护意的嗔恚},受礼拜则应\textbf{平静、不高举而行},即便受到国王的礼拜,也不应以「他在礼拜我」而生掉举。\end{enumerate}

\subsection\*{\textbf{709} {\footnotesize 〔PTS 703〕}}

\textbf{「在丛林中有种种出现,犹如火焰,\\}
\textbf{「女人们诱惑牟尼,莫要让她们诱惑你!}

Uccāvacā niccharanti, dāye aggisikhūpamā;\\
nāriyo muniṃ palobhenti, tā su taṃ mā palobhayuṃ. %\hfill\textcolor{gray}{\footnotesize 25}

\begin{enumerate}\item 现在,为显示舍弃与林野相关的烦恼,说了此颂。其义为:\textbf{在}被称为林野的\textbf{丛林中},\textbf{有}以可意、不可意而为\textbf{种种},即各种品类的所缘\textbf{出现},来至眼等的领域,且它们以生起热恼之义\textbf{犹如火焰}。或者,好比林中失火,火焰以各种品类高低出没,或有烟、或无烟、或青、或黄、或红、或小、或大,如是,在丛林中有种种所缘,以狮、虎、人、非人、种种鸟鸣、花、果、芽等类的种种品类出现,或恐怖、或诱人、有可恼、或使人痴迷。因此说「在丛林中有种种出现,犹如火焰」。
\item 如是,于种种出现的所缘中,任何来到园囿、林中的行者或日常行于林中的拾薪者等的女性,见到幽居者后,\textbf{女人们}以欢笑、交谈、悲泣、衣着不整等\textbf{诱惑牟尼,莫要让她们诱惑你},莫要让这些女人诱惑你,即是说你应如是作,好让她们莫诱惑你。\end{enumerate}

\subsection\*{\textbf{710} {\footnotesize 〔PTS 704〕}}

\textbf{「戒离淫欲法,舍弃了各种爱欲,\\}
\textbf{「不对立、不迷恋,对弱或强的生类,}

Virato methunā dhammā, hitvā kāme paropare;\\
aviruddho asāratto, pāṇesu tasathāvare. %\hfill\textcolor{gray}{\footnotesize 26}

\begin{enumerate}\item 如是,世尊对其显示了在村落与林野的行道的规则,现在,为显示戒律仪,说了以下二颂。这里,\textbf{舍弃了各种爱欲},即舍弃了除淫欲的其它善妙、不善妙的种种五欲。因为以舍弃此,戒离淫欲则能善成就。因此说「舍弃了各种爱欲」。这即此中的意趣。\end{enumerate}

\subsection\*{\textbf{711} {\footnotesize 〔PTS 705〕}}

\textbf{「以『他们如同我,我如同他们』,\\}
\textbf{「以自己作比方,不应杀,不应教人杀。}

‘Yathā ahaṃ tathā ete, yathā ete tathā ahaṃ’;\\
attānaṃ upamaṃ katvā, na haneyya na ghātaye. %\hfill\textcolor{gray}{\footnotesize 27}

\begin{enumerate}\item 而(上颂的)「不对立」等词与此中所说的「不应杀,不应教人杀」是为显示戒离杀生的成就而说的。其略释为:对他方的\textbf{生类不对立},对己方的\textbf{不迷恋},对一切俱爱、离爱的\textbf{弱或强的}生类,以欲活不欲死,以欲乐厌苦,以「\textbf{他们如同我}」的与己相同性,调伏对他们的对立,并仍以此方法,以「\textbf{我如同他们}」的与他人的相同性,调伏对自己的顺从,如是舍弃了两处的顺从与对立,因厌死而\textbf{以自己作比方},对生类中任何弱或强的生类\textbf{不应}以亲手实施等\textbf{杀},\textbf{不应}以命令等\textbf{教人杀}。\end{enumerate}

\subsection\*{\textbf{712} {\footnotesize 〔PTS 706〕}}

\textbf{「舍弃了凡夫所执著的希望与贪,\\}
\textbf{「具眼者能修行,能超越这地狱。}

Hitvā icchañ ca lobhañ ca, yattha satto puthujjano;\\
cakkhumā paṭipajjeyya, tareyya narakaṃ imaṃ. %\hfill\textcolor{gray}{\footnotesize 28}

\begin{enumerate}\item 如是,略说了以戒离淫欲、戒离杀生为首的别解脱律仪戒,以「舍弃了爱欲」等显示了根律仪,现在,为显示活命遍净,说了此颂。其义为:即此渴爱,由已得一便希望再次,已得二便希望三次,已得百千便希望更多,如是由希望未得的境域而说为「希望」,且即此对已得境域的贪婪为「贪」,\textbf{舍弃了}这\textbf{凡夫所执著的希望与贪},于此衣等的资具,凡夫以此希望与贪所执著、固著、束缚而住之处,双双舍弃已,当不为资具而违犯活命清净时,便以智眼而成\textbf{具眼者},\textbf{能修行}此寂默的行道。而如是修行者\textbf{能超越这地狱},即能超越这以难以填满之义而称为地狱的、作为邪命之因的对资具的渴爱,或者,即是说以此行道而能超越。\end{enumerate}

\subsection\*{\textbf{713} {\footnotesize 〔PTS 707〕}}

\textbf{「他应乏腹、节食、少欲、无贪,\\}
\textbf{「始终不饥于欲,则无欲、止息。}

Ūnūdaro mitāhāro, appicch’assa alolupo;\\
sadā icchāya nicchāto, aniccho hoti nibbuto. %\hfill\textcolor{gray}{\footnotesize 29}

\begin{enumerate}\item 如是显示了以舍弃对资具的渴爱为首的活命遍净,现在,以于饮食知量为首,为显示资具受用戒及与之相伴的直至证得阿罗汉的行道,说了此颂。其义为:于正当所得的无论何种衣等资具中,先说饮食,当饮食时,\textbf{他应}如\begin{quoting}少食四五口,汝即当饮水,\\勤修习比丘,实足以安住。(长老偈第 983 颂)\end{quoting}所说而\textbf{乏腹},而非如灌了风的布袋一般胀满了腹,即是说缘食后的倦怠带来昏沉睡眠。且当乏腹时,他应\textbf{节食},于食知量,以「不为嬉戏」等的省察,从功德及过失来限制食物。如是节食时,他还应以资具、头陀支、教法、证得等四种少欲而\textbf{少欲}。因为比丘行寂默的行道定然成如是少欲。
\item 这里,于各个资具以三种满足而知足为\textbf{资具少欲}。对持头陀支者,由不欲「让其他人知道我是俱头陀者」为\textbf{头陀支少欲}。对多闻者,由不欲「让其他人知道我是多闻者」为\textbf{教法少欲},如持中部的长老一般。对具足证得者,由不欲「让其他人知道我已证得这善法」为\textbf{证得少欲},当知这是证得阿罗汉以下的,因为这行道是为了证得阿罗汉。
\item 如是少欲,还应以阿罗汉道舍弃爱贪而成\textbf{无贪}。因为如是无贪,\textbf{始终不饥于欲,则无欲、止息},由饥于欲,有情如苦患于饥渴般不得满足,由不欲此希望而成无欲,且由无欲而成不饥、无苦患、得达最上的满足,如是由不饥而成止息,平息了一切烦恼热恼,如是当知此中应以逆序连结。\end{enumerate}

\subsection\*{\textbf{714} {\footnotesize 〔PTS 708〕}}

\textbf{「他行乞后,应去到林边,\\}
\textbf{「在树下安顿,牟尼入坐。}

Sa piṇḍacāraṃ caritvā, vanantam abhihāraye;\\
upaṭṭhito rukkhamūlasmiṃ, āsanūpagato muni. %\hfill\textcolor{gray}{\footnotesize 30}

\begin{enumerate}\item 如是论述了直至证得阿罗汉的行道,现在,为论述这行道比丘的以证得阿罗汉为究竟的受持头陀支及坐卧处仪法,说了以下二颂。这里,\textbf{他行乞后},即此比丘行乞后,或食事已毕。\textbf{应去到林边},即不为俗家的戏论所戏,唯应去到林中。\textbf{在树下安顿},或在树下而住。\textbf{入坐},即是说坐下。\textbf{牟尼},即行寂默的行道者。
\item 且此中,以「行乞」说常乞食支。而因为高贵的行乞者必是次第乞食者、一座食者、一钵食者、时后不食者,并受持三衣、粪扫衣,所以以此也说了六支。以「应去到林边」说阿练若住支,以「在树下安顿」说树下住支,以「入坐」说常坐不卧支。然后依次为随顺彼等,也说了露地住、随处住、冢间住支,如是,以此颂为那罗迦长老论述了十三头陀支。\end{enumerate}

\subsection\*{\textbf{715} {\footnotesize 〔PTS 709〕}}

\textbf{「他从事禅那,坚定,应乐于林边,\\}
\textbf{「应在树下禅修,令自己满足。}

Sa jhānapasuto dhīro, vanante ramito siyā;\\
jhāyetha rukkhamūlasmiṃ, attānam abhitosayaṃ. %\hfill\textcolor{gray}{\footnotesize 31}

\begin{enumerate}\item \textbf{他从事禅那,坚定},即他以生起未生起的禅那,以及以转向、入定、决意、出定、省察已生起的(禅那)从事禅那。\textbf{坚定},即具足坚毅。\textbf{应乐于林边},即应乐于林中,即是说不应乐于村边的坐卧处。\textbf{应在树下禅修,令自己满足},即不仅应从事于世间禅,而且即于此树下,以与须陀洹道等相应的出世间禅那,为令自己极满足而禅修。因为心因最高安息的出世间禅那而极满足,非因其它,因此说「令自己满足」。如是,以此颂的从事禅修论述了乐于林边的坐卧处及阿罗汉性。\end{enumerate}

\subsection\*{\textbf{716} {\footnotesize 〔PTS 710〕}}

\textbf{「随后,当夜晚逝去,他应去到村边,\\}
\textbf{「不应喜于招请,及从村里来的供养。}

Tato ratyā vivasāne, gāmantam abhihāraye;\\
avhānaṃ nābhinandeyya, abhihārañ ca gāmato. %\hfill\textcolor{gray}{\footnotesize 32}

\begin{enumerate}\item 现在,因为那罗迦长老听了这法的开示,去到林边,甚至没有食物也极勇猛于圆满行道,但没有食物则不能行沙门法。因为如此而行者不能转起活命,应不起烦恼遍求食物,这是此中的理法。所以,世尊为向他显示在随后的日子里应行乞但不应起烦恼,为论述以证得阿罗汉为究竟的行乞的仪法,说了以下六颂。
\item 这里,\textbf{随后},即「他行乞后,应去到林边」中所说的行乞、去到林边之后。\textbf{当夜晚逝去},即是说翌日。\textbf{他应去到村边},即在做完等正行的义务后,直至行乞之时,增长了远离,应以在往还的义务\footnote{往还的义务,见\textbf{犀牛角经}第 35 颂注。}中所说的方法,作意于业处去往村落。\textbf{不应喜于招请},即对「尊者!请来我们家中用餐」等的邀请,以及对以「给了还是没给,给了善妙的还是不善妙的」这样的寻(所得的)食物,圆满行道的比丘不应欢喜,即是说不应接受。然而,若他们使劲拿走钵并装满了布施,则在享用后应行沙门法,并不干扰头陀支,但不得因执取此而入村。\textbf{及从村里来的供养},即若入村后,即便他们带来一百盘食物,也不应于此欢喜,随后,不应接受哪怕一粒饭,而应顺次逐家行乞。\end{enumerate}

\subsection\*{\textbf{717} {\footnotesize 〔PTS 711〕}}

\textbf{「牟尼入村后,不应鲁莽地行于俗家,\\}
\textbf{「闭口不谈求食,不应说诱导的话语。}

Na munī gāmam āgamma, kulesu sahasā care;\\
ghāsesanaṃ chinnakatho, na vācaṃ payutaṃ bhaṇe. %\hfill\textcolor{gray}{\footnotesize 33}

\begin{enumerate}\item \textbf{牟尼入村后,不应鲁莽地行于俗家},即这为寂默而行道的牟尼去到村中时,不应鲁莽地行于俗家,即是说不应从事共同忧伤等不适当的俗家交际。\textbf{闭口不谈求食,不应说诱导的话语},即如闭口一般,不应以暗示、迂回、示相的表示\footnote{暗示、迂回、示相的表示,见\textbf{清净道论}·说戒品第 113 段及以下。}相诱导而说求食的话语。若他希望,正在生病者可以为了除病而说。或者,除为了住处以暗示、迂回、示相相诱导的表示外,无病者不应为了其余的资具而说任何。\end{enumerate}

\subsection\*{\textbf{718} {\footnotesize 〔PTS 712〕}}

\textbf{「『我得了任何便善哉,我未得也好』,\\}
\textbf{「他如如于这两者,即返回到树下。}

‘Alatthaṃ yad idaṃ sādhu, nālatthaṃ kusalaṃ’ iti;\\
ubhayen’eva so tādī, rukkhaṃ v’upanivattati. %\hfill\textcolor{gray}{\footnotesize 34}

\begin{enumerate}\item 此颂之义为:入村行乞,即便只得了任何少许,想「\textbf{我得了任何便善哉}」,当未得时,想「\textbf{我未得也好}」,也以为善妙,\textbf{他如如于}得与未得\textbf{这两者}而无变,\textbf{即返回到树下}。好比寻觅果实之人到了树下,无论得或未得果实,不受诱惑,不受打击,唯中舍而行,如是到了俗家,无论得或未得利养,唯中舍而行。\end{enumerate}

\subsection\*{\textbf{719} {\footnotesize 〔PTS 713〕}}

\textbf{「他以手持钵而行,不哑却被认为哑了,\\}
\textbf{「他不应轻蔑少施,不应鄙视施者。}

Sa pattapāṇi vicaranto, amūgo mūgasammato;\\
appaṃ dānaṃ na hīḷeyya, dātāraṃ nāvajāniyā. %\hfill\textcolor{gray}{\footnotesize 35}

\begin{enumerate}\item 此颂之义自明。\end{enumerate}

\subsection\*{\textbf{720} {\footnotesize 〔PTS 714〕}}

\textbf{「因为种种行道已由沙门阐明,\\}
\textbf{「他们不会两次去到彼岸,这也不会被觉知一次。}

Uccāvacā hi paṭipadā, samaṇena pakāsitā;\\
na pāraṃ diguṇaṃ yanti, na-y-idaṃ ekaguṇaṃ mutaṃ. %\hfill\textcolor{gray}{\footnotesize 36}

\begin{enumerate}\item 此颂的连结为:如是具足行乞的仪法者,不能满足于此,而应致力于行道。因为教法的精髓是行道,且「因为种种行道……觉知一次」。其义为:这道的\textbf{行道},依高低等类而有\textbf{种种},\textbf{已由}佛\textbf{沙门阐明}。因为乐行道速通达为高,苦行道迟通达为低,其它两者一支高一支低\footnote{苦行道迟通达等四种,见\textbf{清净道论}·说取业处品第 14 段及以下。},或者只有第一个为高,另三者均低。
\item 且以这样那样或高或低的行道,\textbf{他们不会两次去到彼岸},文本或作 duguṇaṃ,意即他们不会以一道两次去到涅槃。为什么?因为以此道舍断的烦恼不需被再次舍断,以此显明无有退法。\textbf{这也不会被觉知一次},即这彼岸一次也不会被证得。为什么?由以一道不能舍断一切烦恼故,显明唯以此一道则无有阿罗汉性。\footnote{此句费解。菩提比丘注 1696 的理解是,人不能以两种方式到达彼岸,但对所有人不是只有一种方式。Norman 在书中保留了 Miss Horner 的解释,即不能两次去到涅槃是因为舍断的烦恼无需被再次舍断,而涅槃却不止被觉知一次,而是四次,即分别在证得初果至四果时。Norman 自己给出的一种可能是,颂中的沙门不是佛陀,而是其他任何沙门,则意为其他沙门不能由两种极端去到涅槃,他们一次也无法到达。}\end{enumerate}

\subsection\*{\textbf{721} {\footnotesize 〔PTS 715〕}}

\textbf{「若比丘已无爱著、已截断了水流、\\}
\textbf{「舍断了应作与不应作,则无热恼。}

Yassa ca visatā natthi, chinnasotassa bhikkhuno;\\
kiccākiccappahīnassa, pariḷāho na vijjati. %\hfill\textcolor{gray}{\footnotesize 37}

\begin{enumerate}\item 现在,为显示行道的功德,说了此颂。其义为:\textbf{若}如是行道的\textbf{比丘}以此行道,由舍断了为百八种渴爱所缠绕的爱著而\textbf{已无爱著}、渴爱,以截断了烦恼之流而\textbf{已截断了水流},以舍断了善、不善而\textbf{舍断了应作与不应作},\textbf{则无}一丝贪生或嗔生的\textbf{热恼}。\end{enumerate}

\subsection\*{\textbf{722} {\footnotesize 〔PTS 716〕}}

\textbf{「我将向你说明寂默,应如剃刀的锋刃,\\}
\textbf{「用舌头抵住上颚,应自制于口腹。}

Moneyyaṃ te upaññissaṃ, khuradhārūpamo bhave;\\
jivhāya tālum āhacca, udare saññato siyā. %\hfill\textcolor{gray}{\footnotesize 38}

\begin{enumerate}\item 现在,因为听闻了这些偈颂后,那罗迦长老的心中生起「如果这些就是寂默,则易行而非难行,以少许困难即可圆满」,所以世尊为向他显示「寂默实是难为」,又再次说「我将向你说明寂默」等等。
\item 这里,\textbf{说明},即是说论述。\textbf{应如剃刀的锋刃},这是什么意思?行寂默的比丘应以剃刀的锋刃作譬而使用资具。好比去舔舐涂抹了蜜的刀刃时,防止舌被割伤,如是,在受用如法所得的资具时,应防止心生起烦恼。因为资具不易以遍净的方式去获得,并不易以无过的受用去受用,故而世尊便对依止资具多有所说。\textbf{用舌头抵住上颚,应自制于口腹},即用舌头顶住上颚,去除味爱,不使用以染污之道而来的资具,应自制于口腹。\end{enumerate}

\subsection\*{\textbf{723} {\footnotesize 〔PTS 717〕}}

\textbf{「心不应沉滞,且亦不应多虑,\\}
\textbf{「离生腥、无所依,以梵行为归宿。}

Alīnacitto ca siyā, na cāpi bahu cintaye;\\
nirāmagandho asito, brahmacariyaparāyaṇo. %\hfill\textcolor{gray}{\footnotesize 39}

\begin{enumerate}\item \textbf{心不应沉滞},即心应始终以修习善法、不停作为而不懈怠。\textbf{且亦不应多虑},即不应以亲族、国土、不死等寻多虑。\textbf{离生腥、无所依,以梵行为归宿},即已离烦恼,不以爱、见而依止于任何有,唯应以三学作为整个教法的梵行为归宿而修习。\end{enumerate}

\subsection\*{\textbf{724} {\footnotesize 〔PTS 718〕}}

\textbf{「应修学独坐,及沙门修行,\\}
\textbf{「独一被称为寂默,\\}
\textbf{「若你能乐于独一,则能照亮十方。}

Ekāsanassa sikkhetha, samaṇūpāsanassa ca;\\
ekattaṃ monam akkhātaṃ;\\
eko ce abhiramissasi, atha bhāhisi dasadisā. %\hfill\textcolor{gray}{\footnotesize 40}

\begin{enumerate}\item \textbf{独坐},即远离之坐。且此中以坐为首而说一切威仪路,当知即是说应于一切威仪路修学独一。且「独坐」是为格。\textbf{及沙门修行},即从事沙门应修行的三十八种所缘的修习,或作为沙门之修行的三十八种所缘\footnote{菩提比丘注 1699 云,四十业处中,光明遍、限定虚空遍的所缘等同于空无边处、识无边处的所缘,故说「三十八种所缘」。}。这也是为格,即是说为了修行。且此中,当知以独坐为身远离,以沙门修行为心远离。\textbf{独一被称为寂默},即如是以身、心远离,此独一被称为寂默。\textbf{若你能乐于独一},此句期待下颂,即与「则能照亮十方」相连\footnote{据此,则义注所见之本,下句当属下颂。}。\textbf{照亮},即知名,即是说修习此行道者,将在所有方向以声誉知名。\end{enumerate}

\subsection\*{\textbf{725} {\footnotesize 〔PTS 719〕}}

\textbf{「听闻了智者、禅修者、舍弃爱欲者的宣言,\\}
\textbf{「随后,致力于我者应更加培育惭与信。}

Sutvā dhīrānaṃ nighosaṃ, jhāyīnaṃ kāmacāginaṃ;\\
tato hiriñ ca saddhañ ca, bhiyyo kubbetha māmako. %\hfill\textcolor{gray}{\footnotesize 41}

\begin{enumerate}\item 而「听闻了智者」等四句之义为:\textbf{听闻了}以此声誉照亮十方的\textbf{智者、禅修者、舍弃爱欲者的宣言},然后,你不要因此犯掉举,\textbf{应更加培育惭与信},即因此声誉而惭,生起「此是出离之行道」之信,应更加增长行道。\textbf{致力于我者},因为如是方是我的弟子。\end{enumerate}

\subsection\*{\textbf{726} {\footnotesize 〔PTS 720〕}}

\textbf{「你们以河流来了知,较之渠与沟,\\}
\textbf{「小渠喧嚣着前行,大水默然前行。}

Taṃ nadīhi vijānātha, sobbhesu padaresu ca;\\
saṇantā yanti kusobbhā, tuṇhī yanti mahodadhī. %\hfill\textcolor{gray}{\footnotesize 42}

\begin{enumerate}\item \textbf{以河流},即我以「应更加培育惭与信」之说所说的「不应掉举」,你们也可以河流之例来了知,并以与之相反的\textbf{渠与沟}来了知,渠即水路,沟即穴隙。如何?\textbf{小渠喧嚣着前行,大水默然前行},因为小渠,即渠沟等类的所有小河喧嚣着,发出声响,高举前行,而恒河等类的大水默然前行,如是,以「我圆满了寂默」而高举者非致力于我者,致力于我者则生起惭与信,唯卑下其心。\end{enumerate}

\subsection\*{\textbf{727} {\footnotesize 〔PTS 721〕}}

\textbf{「欠缺者喧嚣,唯满盈者寂静,\\}
\textbf{「愚人譬如半桶,智者如满池。}

Yad ūnakaṃ taṃ saṇati, yaṃ pūraṃ santam eva taṃ;\\
aḍḍhakumbhūpamo bālo, rahado pūro va paṇḍito. %\hfill\textcolor{gray}{\footnotesize 43}

\begin{enumerate}\item 且更有「欠缺者……智者如满池」。这里,设若\textbf{愚人}以喧嚣\textbf{譬如半桶},\textbf{智者}以寂静\textbf{如满池},那么,为什么佛沙门如是忙碌于开示法,说了许多?与此相连,说了下颂。\end{enumerate}

\subsection\*{\textbf{728} {\footnotesize 〔PTS 722〕}}

\textbf{「当沙门说许多具有并伴随义利(的话),\\}
\textbf{「他知晓着而开示法,他知晓着而说许多。}

Yaṃ samaṇo bahuṃ bhāsati, upetaṃ atthasañhitaṃ;\\
jānaṃ so dhammaṃ deseti, jānaṃ so bahu bhāsati. %\hfill\textcolor{gray}{\footnotesize 44}

\begin{enumerate}\item 其义为:当佛\textbf{沙门说许多具有并伴随义利(的话)},具有义、具有法且伴随利益,不以高举,而是\textbf{他知晓着而开示法},即便整日开示也无戏论,因为他的所有语业与智相属。且如是开示时,以「这对你有利、这对你有利」的种种品类\textbf{知晓着而说许多},不仅是多说而已。\end{enumerate}

\subsection\*{\textbf{729} {\footnotesize 〔PTS 723〕}}

\textbf{「若知晓着而自制,知晓着而不说许多,\\}
\textbf{「这牟尼应得寂默,这牟尼证得了寂默。」}

Yo ca jānaṃ saṃyatatto, jānaṃ na bahu bhāsati;\\
sa munī monam arahati, sa munī monam ajjhagā” ti. %\hfill\textcolor{gray}{\footnotesize 45}

\begin{enumerate}\item 与末颂的连结为:如是具足一切知智的佛沙门知晓着而开示法,知晓着而说许多,而以此抉择分智\footnote{抉择分智 \textit{nibbedha-bhāgiya ñāṇa}:这是旧译,字面意思是「导向洞察之智」。}\textbf{知晓着}所开示之法\textbf{而自制,知晓着而不说许多,这牟尼应得寂默,这牟尼证得了寂默}。
\item 其义为:当知晓着这法而\textbf{自制}、守护心已,若所说的不给有情带来利乐,则\textbf{知晓着而不说许多}。如这般为寂默而行道的牟尼,\textbf{应得}被称为寂默的行道的\textbf{寂默}。且不仅只是应得,而且当知\textbf{这牟尼证得了}被称为阿罗汉道智的\textbf{寂默}。如此,便以阿罗汉为顶点完成了开示。
\item 那罗迦长老听后,便于三处少欲:于见、于闻、于问。因为他在开示的终了心得净喜,礼拜了世尊后即进入林中,不再生出「哎!我愿见世尊」的贪,此即他于见的少欲。同样,不生出「哎!我愿再次听闻法的开示」的贪,此即他于闻的少欲。同样,不生出「哎!我愿再次问寂默的行道」的贪,此即他于问的少欲。他既如是少欲,进入山脚下,在一片密林中不住两天,在一棵树下不坐两天,在一个村内不为乞食进入两天,如是从林到林、从树到树、从村到村地徘徊,行了随适的行道,便住于最上果。
\item 这里,因为圆满上等寂默的行道的比丘只能活七个月,圆满中等的七年,圆满低等的十六年,而他圆满了上等,所以七个月后,了知到寿行已尽,便在沐浴后著了下衣、系了腰带、披了双层的僧伽梨,面向十力五体投地,合掌已,背靠朱山而立,以无余依涅槃界而般涅槃。世尊了知他已般涅槃,便与比丘僧团一起去到那里,行了葬礼,教人拾了舍利、建了支提而返。\end{enumerate}

\begin{center}\vspace{1em}那罗迦经第十一\\Nālakasuttaṃ ekādasamaṃ.\end{center}