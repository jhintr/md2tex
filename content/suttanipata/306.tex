\section{会堂经}

\begin{center}Sabhiya Sutta\end{center}\vspace{1em}

\textbf{如是我闻。一时世尊住王舍城竹林喂松鼠处。尔时,先前曾是血亲的天人向游行者会堂提出问题:「会堂!若沙门、婆罗门能向你解释这些问题,你应在他跟前行梵行。」}

Evaṃ me sutaṃ— ekaṃ samayaṃ Bhagavā Rājagahe viharati Veḷuvane Kalandakanivāpe. Tena kho pana samayena Sabhiyassa paribbājakassa purāṇasālohitāya devatāya pañhā uddiṭṭhā honti: “yo te, Sabhiya, samaṇo vā brāhmaṇo vā ime pañhe puṭṭho byākaroti tassa santike brahmacariyaṃ careyyāsī” ti.

\begin{enumerate}\item 缘起为何?这在其因缘中已述。而在释义的过程中,其与先前相同者当知仍按先前所述之法。先前未述者,我们则避开意义自明之词,再作解释。
\item \textbf{竹林喂松鼠处}\footnote{竹林喂松鼠处:旧译作迦兰陀竹园。},「竹林」是这庭园之名。据说,它为群竹环绕,且有十八肘高的围垣,门阙的门上设有塔楼,碧色闪耀而怡人,因此称为「竹林」。且人们在此投喂饲料给松鼠,因此称为「喂松鼠处」。据说,过去某位国王来此庭园游玩,因饮酒而醉,便睡觉午休。他的仆从想「国王睡了」,便为花花果果所诱,四处散开。于是,黑蛇因酒香从某处树洞出来,朝国王而去。树天人见后,想「我要救王的命」,伪装成松鼠前往,在耳根处发出声响。国王醒转,黑蛇便退去。他看到它后,想「我的命是这松鼠救的」,便开始在此投喂松鼠,并宣布为庇护所。因之,从此便得称为「喂松鼠处」。
\item \textbf{游行者会堂},会堂是他的名字,游行者是指外道出家而说。\textbf{先前曾是血亲的天人},即非父非母,然而,由如其父其母般倾向其利益故,这天子被称为「先前曾是血亲的天人」。
\item 据说,在迦叶世尊般涅槃时所建的金支提内,三个族姓子在其亲传弟子跟前出家,获取了适于性行的业处,去到边鄙的国土,在阿兰若处行沙门法,且间或为了礼敬支提及听法而进城。后来,不愿片刻离开阿兰若,唯在那里不放逸而住,即便如是而住,仍不能证得任何殊胜。随后,他们想:「我们行乞食而希求活命,但因希求活命,不能证得出世间法,以凡夫死去实在是苦,噫!让我们扎个梯子,上到山里,不希求身命去行沙门法!」他们便这样做了。
\item 他们中的大长老由于具足近依,当天就得证具有六神通的阿罗汉。他以神变去到雪山,在阿耨达池洗了脸,到北俱卢行乞,食事已毕,再到另一个地方装满了钵,并取了阿耨达池的水和槟榔作的齿木,回到他们跟前说:「朋友!你们看我的威力!这是从北俱卢带来的乞食,这是从雪山带来的水和齿木,你们享用了这些再行沙门法,我会一直照顾你们。」他们听后,便对他说:「尊者!你们应作已作,仅与你们交谈对我们只是迁延,现在,请不要再到我们跟前来了!」他不能以任何方法让他们领受,便即离开。
\item 随后,过了两三天,他们中的一个成了五神通的阿那含。他也这样做,但被另一个拒绝,便也走了。他在拒绝后,继续精进,在从上山开始的第七天,未证得任何殊胜而死,转生到了天界。漏尽的长老也在同一天入般涅槃,阿那含则投生到了净居天。天子在六欲界天界顺逆地享受了天福,在我们世尊之时,从天界殁后,便在某个女游行者的胎内获取了结生。
\item 据说,她是某个刹帝利的女儿,父母想「让我们的女儿了解不同的教义」,便把她托付给一个游行者。他的一个游行者弟子和她一起邪行,她便因此怀孕。发现她怀孕后,众游行者便驱逐了她。她在去往某处时,便在中途的会堂里分娩,因此给他起名「\textbf{会堂}」。这会堂长大后,也出家为游行者,学习了种种典籍而成大论师,为论辩学说而在整个阎浮提巡行,却未见与自己相等的论师,便在城门外教人建了草庵,住在那里,教授刹帝利童子等学问。
\item 然后,世尊已转起无上法轮,渐次前往王舍城,住在竹林喂松鼠处,而会堂还不知道佛陀出世。于是,那净居天的梵天出定后,转向于「我以何威力证此殊胜」,忆起在迦叶世尊的教法下行沙门法及其友人,当转向于「其中一个已般涅槃,一个现在何处」时,便知晓「从天界殁后,生于阎浮提,还不知道佛陀已经出世」,想「噫!那我要催促他去亲近佛陀」,便准备了二十个问题,于夜分去到他的草庵,站在空中,唤到:「会堂!会堂!」
\item 他正睡觉,听到三次这声音便出来,见到光芒,便合掌而立。随后,梵天对他说:「会堂!我为你的义利带来了二十个问题,请你受持它们!若沙门、婆罗门能向你解释这些问题,你应在他跟前行梵行。」就此天子而说「先前曾是血亲的天人向游行者会堂提出问题」。\textbf{提出},即仅以略说,而非以分别。如是说已,会堂仅以一语就逐句受持了它们。然后,这梵天虽然知道佛陀出世,却并未告知,「当寻求义利时,游行者自会知晓大师,而除此之外的沙门婆罗门都是虚无」,以此意趣而说「会堂!若……行梵行」。
\item 然而,在注释「长老偈·四偈集」的「会堂长老譬喻」时,他们说:「他的母亲思惟了自己的邪行,便生嫌厌,生起禅那后,投生到梵界,这些问题是由这梵天提出的。」\end{enumerate}

\textbf{于是,游行者会堂在那天人跟前受持了这些问题,前往那些沙门婆罗门处,有僧团、有徒众、为徒众老师、知名、有名闻、为创教者、众所敬仰,例如富兰那·迦叶、末迦梨·瞿舍利子、阿耆多·翅舍钦婆罗、迦罗拘陀·迦栴延、先阇那·毗罗胝子、尼揵陀·若提子,走到后,向他们提出这些问题。他们不能解答游行者会堂提出的问题,不能解答而显出怨恨、嗔恨、不满,甚至还反问游行者会堂。}

Atha kho Sabhiyo paribbājako tassā devatāya santike te pañhe uggahetvā ye te samaṇabrāhmaṇā saṅghino gaṇino gaṇācariyā ñātā yasassino titthakarā sādhusammatā bahujanassa, seyyathidaṃ: Pūraṇo Kassapo, Makkhali Gosālo, Ajito Kesakambalo, Pakudho Kaccāno, Sañcayo Belaṭṭhaputto, Nigaṇṭho Nāṭaputto, te upasaṅkamitvā te pañhe pucchati. Te Sabhiyena paribbājakena pañhe puṭṭhā na sampāyanti, asampāyantā kopañ ca dosañ ca appaccayañ ca pātukaronti, api ca Sabhiyaṃ yeva paribbājakaṃ paṭipucchanti.

\begin{enumerate}\item \textbf{那些},即现在当说的泛指。\textbf{沙门婆罗门},即以行出家而得世间认可的沙门与婆罗门。\textbf{有僧团},即具徒众。\textbf{有徒众},即自称「我们是一切知者」的大师。\textbf{为徒众老师},即以略说、遍问等而为出家、在家众的老师。\textbf{知名},即有名,即是说著名、出名。\textbf{有名闻},即具足利养与随从。\textbf{创教者},即为追随彼等之见者所深入、潜入的外道见的创立者。\textbf{众所敬仰},即以「他们是善人」而为众人周知。\textbf{例如},即「请问他们有哪些」之义的不变词。
\item \textbf{富兰那}是名,\textbf{迦叶}是族姓。据说,他出身奴仆,以凑满一百奴仆而生,因此得名「富兰那」。逃走后,在裸行者中出家,便以「我是迦叶」声称族姓,且自称是一切知者。\textbf{末迦梨}是名,由出生于牛栏故,被称为\textbf{瞿舍利子}。据说,他也是出身奴仆,逃走后出家,自称是一切知者。\textbf{阿耆多}是名,以少欲而持用头发编的毯子,因此被称为\textbf{翅舍钦婆罗},他也自称是一切知者。
\item \textbf{迦罗拘陀}是名,\textbf{迦栴延}是族姓。以少欲,且认为水中有生命,拒绝沐浴、洗脸等,他也自称是一切知者。\textbf{先阇那}是名,而毗罗胝是其父,所以被称为\textbf{毗罗胝子},他也自称是一切知者。\textbf{尼揵陀}是出家名,\textbf{若提子}是以父名得称。据说「若提」是他父亲的名字,其子即若提子,他也自称是一切知者。他们都各有五百学生随从。
\item \textbf{这些问题},即这二十个问题。\textbf{他们},即那些六师。\textbf{不能解答},即不令成就。\textbf{怨恨}即心、心所的污浊,\textbf{嗔恨}即秽恶之心,且此二者即钝、利两类忿怒的同义语。\textbf{不满},即不喜,即是说忧虑。\textbf{显出},即以身语的转变显露、表明。\end{enumerate}

\textbf{于是,游行者会堂想:「那些尊敬的沙门婆罗门,有僧团、有徒众、为徒众老师、知名、有名闻、为创教者、众所敬仰,例如富兰那·迦叶……尼揵陀·若提子,他们不能解答我提出的问题,不能解答而显出怨恨、嗔恨、不满,甚至还于此反问我。我何不转回低处,去享受爱欲?」}

Atha kho Sabhiyassa paribbājakassa etad ahosi: “ye kho te bhonto samaṇabrāhmaṇā saṅghino gaṇino gaṇācariyā ñātā yasassino titthakarā sādhusammatā bahujanassa, seyyathidaṃ: Pūraṇo Kassapo…pe… Nigaṇṭho Nāṭaputto, te mayā pañhe puṭṭhā na sampāyanti, asampāyantā kopañ ca dosañ ca appaccayañ ca pātukaronti, api ca mañ ñev’ettha paṭipucchanti. Yan nūnāhaṃ hīnāyāvattitvā kāme paribhuñjeyyan” ti.

\begin{enumerate}\item \textbf{低处},即在家的状态。因为在家的状态相较于出家,由戒等德较低,或由受用低级的爱欲,被称为低处,而高处即出家。\textbf{转回},即退回。\textbf{享受爱欲},即受用爱欲。据说,他见到如此自称一切知者的出家人的虚妄而然。\end{enumerate}

\textbf{于是,游行者会堂想:「这沙门乔达摩也有僧团、有徒众、为徒众老师、知名、有名闻、为创教者、众所敬仰,我何不前往沙门乔达摩处,去问这些问题?」}

Atha kho Sabhiyassa paribbājakassa etad ahosi: “ayam pi kho samaṇo Gotamo saṅghī c’eva gaṇī ca gaṇācariyo ca ñāto yasassī titthakaro sādhusammato bahujanassa, yan nūnāhaṃ samaṇaṃ Gotamaṃ upasaṅkamitvā ime pañhe puccheyyan” ti.

\textbf{于是,游行者会堂想:「那些尊敬的沙门婆罗门衰老、年迈、高龄、迟暮、岁月已逝、上座、久住、出家已久,有僧团、有徒众、为徒众老师、知名、有名闻、为创教者、众所敬仰,例如富兰那·迦叶……尼揵陀·若提子,他们尚且不能解答我提出的问题,不能解答而显出怨恨、嗔恨、不满,甚至还于此反问我,沙门乔达摩将如何解答我提出的这些问题?毕竟沙门乔达摩年纪尚轻,且新近出家。」}

Atha kho Sabhiyassa paribbājakassa etad ahosi: “ye pi kho te bhonto samaṇabrāhmaṇā jiṇṇā vuḍḍhā mahallakā addhagatā vayo anuppattā therā rattaññū cirapabbajitā saṅghino gaṇino gaṇācariyā ñātā yasassino titthakarā sādhusammatā bahujanassa, seyyathidaṃ: Pūraṇo Kassapo…pe… Nigaṇṭho Nāṭaputto, te pi mayā pañhe puṭṭhā na sampāyanti, asampāyantā kopañ ca dosañ ca appaccayañ ca pātukaronti, api ca mañ ñev’ettha paṭipucchanti, kiṃ pana me samaṇo Gotamo ime pañhe puṭṭho byākarissati. Samaṇo hi Gotamo daharo c’eva jātiyā, navo ca pabbajjāyā” ti.

\textbf{于是,游行者会堂想:「然而,『年轻的』沙门不应被轻视、不应被轻蔑。即便年轻,沙门乔达摩也可能有大神变、大威力,我何不前往沙门乔达摩处问这些问题呢?」}

Atha kho Sabhiyassa paribbājakassa etad ahosi: “samaṇo kho ‘daharo’ ti na uññātabbo na paribhotabbo. Daharo pi ce sa samaṇo Gotamo mahiddhiko hoti mahānubhāvo, yan nūnāhaṃ samaṇaṃ Gotamaṃ upasaṅkamitvā ime pañhe puccheyyan” ti.

\begin{enumerate}\item \textbf{于是},以所生的寻思而返的\textbf{游行者会堂想},又如是再再审视:「\textbf{这沙门乔达摩……那些尊敬的沙门婆罗门……『年轻的』沙门不应被轻视……}」
\item 这里,\textbf{衰老}等词已述。\textbf{上座},即于自身的沙门法已至坚固的状态。\textbf{久住},即知宝\footnote{义注从语源上兼顾「时间 \textit{ratti}」和「宝 \textit{ratana}」来解释「久住 \textit{rattaññū}」。},如自称「我们了知涅槃之宝」,且为世间认可,或是晓得许多时间。\textbf{不应被轻视},即是说不应被作低去了知。\textbf{不应被轻蔑},即是说不应被如是接受:「他能知道什么?」\end{enumerate}

\textbf{于是,游行者会堂往王舍城出发游行。渐次游行到王舍城竹林喂松鼠处,往世尊处走去,走到后,问候了世尊,彼此寒暄已,坐在一边。坐在一边游行者会堂以偈颂对世尊说:}

Atha kho Sabhiyo paribbājako yena Rājagahaṃ tena cārikaṃ pakkāmi. Anupubbena cārikaṃ caramāno yena Rājagahaṃ Veḷuvanaṃ Kalandakanivāpo, yena Bhagavā ten’upasaṅkami, upasaṅkamitvā Bhagavatā saddhiṃ sammodi, sammodanīyaṃ kathaṃ sāraṇīyaṃ vītisāretvā ekamantaṃ nisīdi. Ekamantaṃ nisinno kho Sabhiyo paribbājako Bhagavantaṃ gāthāya ajjhabhāsi:

\subsection\*{\textbf{516} {\footnotesize 〔PTS 510〕}}

\textbf{「我带着困惑与疑问而来,」会堂说,「期待着问些问题,\\}
\textbf{「请了结它们!我提出的问题,请逐步、随法地向我解释!」}

“Kaṅkhī vecikicchī āgamaṃ, \textit{(iti Sabhiyo)} pañhe pucchituṃ abhikaṅkhamāno;\\
tes’antakaro bhavāhi pañhe me puṭṭho, anupubbaṃ anudhammaṃ byākarohi me”. %\hfill\textcolor{gray}{\footnotesize 1}

\begin{enumerate}\item 会堂问候世尊时,敬重世尊以容貌、调御、寂静与清净(所示)的一切知性,便离了掉举,而说「带着困惑与疑问」。这里,以「我能否得到这些解释」等对问题解释的疑惑而\textbf{带着困惑},以「这些问题的意义是什么」等疑问而\textbf{带着疑问}。或者,以弱力之疑困惑于这些问题的意义而\textbf{带着困惑},以强力省思仍有困扰、不能决定而\textbf{带着疑问}。\textbf{期待},即极其愿求。\textbf{了结它们},即了结这些问题。
\item 为显示「请您如是了结」,而说「我提出的问题……向我解释」。这里,\textbf{逐步},即问题的次第。\textbf{随法},即举出随适语义的经典。\end{enumerate}

\subsection\*{\textbf{517} {\footnotesize 〔PTS 511〕}}

\textbf{「你远道而来,会堂!」世尊说,「期待着问些问题,\\}
\textbf{「我会了结它们,你提出的问题,我会逐步、随法地向你解释。}

“Dūrato āgato si Sabhiya, \textit{(iti Bhagavā)} pañhe pucchituṃ abhikaṅkhamāno;\\
tes’antakaro bhavāmi pañhe te puṭṭho, anupubbaṃ anudhammaṃ byākaromi te. %\hfill\textcolor{gray}{\footnotesize 2}

\begin{enumerate}\item 据说,他从四处彷徨而来,道路七百由旬,因此世尊说「\textbf{你远道而来}」,或者,由从迦叶世尊的教法中而来故,便说他「你远道而来」。\end{enumerate}

\subsection\*{\textbf{518} {\footnotesize 〔PTS 512〕}}

\textbf{「请问我问题!会堂!任何你心中希望的,\\}
\textbf{「对各个问题,我都会为你了结!」}

Puccha maṃ Sabhiya pañhaṃ, yaṃ kiñci manas’icchasi;\\
tassa tass’eva pañhassa, ahaṃ antaṃ karomi te” ti. %\hfill\textcolor{gray}{\footnotesize 3}

\begin{enumerate}\item (世尊)以此颂作出一切知的邀请。\end{enumerate}

\textbf{于是,游行者会堂想:「这实在不思议!这实在未曾有!我在其他沙门婆罗门处,甚至都得不到许可,而沙门乔达摩却给了我这许可。」便心满意足、愉悦、踊跃、生起喜悦,问世尊问题:}

Atha kho Sabhiyassa paribbājakassa etad ahosi: “acchariyaṃ vata bho, abbhutaṃ vata bho, yaṃ vatāhaṃ aññesu samaṇabrāhmaṇesu okāsakammamattam pi nālatthaṃ taṃ me idaṃ samaṇena Gotamena okāsakammaṃ katan” ti, attamano pamudito udaggo pītisomanassajāto Bhagavantaṃ pañhaṃ apucchi:

\begin{enumerate}\item \textbf{心满意足},即以欢喜、愉悦、喜悦遍满于心。\textbf{踊跃},即身心高昂,但此词并不见于所有文本。现在,为显示之所以心满意足之法,而说\textbf{愉悦、生起喜悦}。\end{enumerate}

\subsection\*{\textbf{519} {\footnotesize 〔PTS 513〕}}

\textbf{「证得什么,人们称之为比丘?」会堂说,「因何而温顺?且如何人们称之为调御?\\}
\textbf{「如何被称为觉悟?我的所问,世尊!请解释!」}

“Kiṃpattinam āhu bhikkhunaṃ, \textit{(iti Sabhiyo)} sorataṃ kena kathañ ca dantam āhu;\\
buddho ti kathaṃ pavuccati, puṭṭho me Bhagavā byākarohi”. %\hfill\textcolor{gray}{\footnotesize 4}

\begin{enumerate}\item \textbf{温顺},即善寂静。文本也作 surata,即善节制之义。\textbf{觉悟},即觉醒\footnote{觉醒 \textit{vibuddha}:未见于词典,菩提比丘英译作 wide awake,这里据第 523 颂的义注,因与睡眠相对,译作「觉醒」。},或将由佛陀觉悟者。如是,会堂每颂各提四个问题,以五颂问了二十个问题,而世尊各以一颂来处理一个问题,以阿罗汉为顶点,以二十颂来解释。\end{enumerate}

\subsection\*{\textbf{520} {\footnotesize 〔PTS 514〕}}

\textbf{「以自己制造的道路,会堂!」世尊说,「得至般涅槃,已度疑惑,\\}
\textbf{「舍弃了离有与有,已立、灭尽再有,他即比丘。}

“Pajjena katena attanā, \textit{(Sabhiyā ti Bhagavā)} parinibbānagato vitiṇṇakaṅkho;\\
vibhavañ ca bhavañ ca vippahāya, vusitavā khīṇapunabbhavo sa bhikkhu. %\hfill\textcolor{gray}{\footnotesize 5}

\begin{enumerate}\item 这里,因为已破除烦恼的第一义比丘得至般涅槃,所以,为对其解释「证得什么,人们称之为比丘」之问,说了此颂。其义为:若以自己修习之道\textbf{得至般涅槃},得至烦恼的完全止息,且由得至般涅槃而\textbf{已度疑惑},\textbf{舍弃了}失败、成功、减损、增益、断、常、非福与福等类的\textbf{离有与有},住于道而\textbf{已立、灭尽再有},便应得这些称赞之语:\textbf{他即比丘}。\end{enumerate}

\subsection\*{\textbf{521} {\footnotesize 〔PTS 515〕}}

\textbf{「于一切处舍,具念,他在一切世间不伤害任何,\\}
\textbf{「已度的沙门不污浊,若他没有增盛,他为温顺。}

Sabbattha upekkhako satimā, na so hiṃsati kañci sabbaloke;\\
tiṇṇo samaṇo anāvilo, ussadā yassa na santi sorato so. %\hfill\textcolor{gray}{\footnotesize 6}

\begin{enumerate}\item 而因为从邪行善加节制且止息种种品类的烦恼,他即温顺,所以为显示此义,便以「于一切处舍」等方法,说了第二问的解释。其义为:若\textbf{于一切处}的色等所缘,以如「眼见色已,既不喜悦,也不忧伤」所转起的六支舍\footnote{六支舍,见\textbf{清净道论}·说地遍品第 157 段。}而\textbf{舍},以证得广大的念而\textbf{具念},\textbf{他在一切世间}、世间一切处\textbf{不伤害任何}弱强等类的有情,由度过暴流为\textbf{已度},由止息恶为\textbf{沙门},由舍弃污浊的思惟为\textbf{不污浊},而且,\textbf{若他没有}任何这些称为贪、嗔、痴、慢、见、烦恼、恶行等或粗或细的七种\textbf{增盛},\textbf{他}便以这些住于舍、广大念及不害,且以善节制邪行,以及此暴流等种种品类烦恼的止息而\textbf{为温顺}。\end{enumerate}

\subsection\*{\textbf{522} {\footnotesize 〔PTS 516〕}}

\textbf{「若其诸根已修,于内在及外在一切世间,\\}
\textbf{「突破了此世他世,等待时间,已修者为调御。}

Yass’indriyāni bhāvitāni, ajjhattaṃ bahiddhā ca sabbaloke;\\
nibbijjha imaṃ parañ ca lokaṃ, kālaṃ kaṅkhati bhāvito sa danto. %\hfill\textcolor{gray}{\footnotesize 7}

\begin{enumerate}\item 且因为已修诸根、不畏、不变者为调御,所以为显示此义,以此颂解释了第三问。其义为:\textbf{若其}眼等六\textbf{根}以行处的修习引入无常等三相,并以熏习的修习把握念正知之熏香而\textbf{已修},且它们正如以行处的修习\textbf{于内在}(已修),如是于\textbf{外在一切世间},当有诸根的欠缺或欠缺的发生,则于彼处即因贪等不成已修。
\item 如是\textbf{突破}、了知、通达\textbf{了此世他世},自身相续的蕴世间与他人相续的蕴世间,欲求不迟缓的死亡而\textbf{等待时间},期待、侍奉命尽的时间,不惧怕死亡。如长老说:\begin{quoting}我无惧于死,无欣欣于生。(长老偈第 20 颂)\\我不期待死,我不期待生,\\我等待时间,如雇工之于薪水。(长老偈第 606 颂)\end{quoting}\textbf{已修者为调御},即如是已修诸根者,他即为调御。\end{enumerate}

\subsection\*{\textbf{523} {\footnotesize 〔PTS 517〕}}

\textbf{「省思了全部思惟、轮回、亡殁与投生两者,\\}
\textbf{「离去尘垢、无秽、清净,证得生的灭尽者,他们称其为觉悟。」}

Kappāni viceyya kevalāni, saṃsāraṃ dubhayaṃ cutūpapātaṃ;\\
vigatarajam anaṅgaṇaṃ visuddhaṃ, pattaṃ jātikhayaṃ tam āhu buddhan” ti. %\hfill\textcolor{gray}{\footnotesize 8}

\begin{enumerate}\item 而因为所谓觉悟,即具足觉慧且从烦恼睡眠中觉醒,所以为显明此义,以此颂解释了第四问。这里,\textbf{思惟},即爱、见。两者由如是如是区分而被称为「思惟」。\textbf{省思},即以无常等把握。\textbf{全部},即整体。\textbf{轮回},即如\begin{quoting}诸蕴与界、处的次第\\无间断地转起,被称为轮回。\end{quoting}所说的被称为蕴等次第的轮回,且省思了全部轮回。至此说了对于作为诸蕴根本的业、烦恼以及对于诸蕴等三轮转的毗婆舍那。\textbf{亡殁与投生两者},即省思、了知了有情的亡殁、投生这两者之义。以此说了死生智。
\item \textbf{离去尘垢、无秽、清净},即由离去贪等尘垢、无有秽污、离去垢秽而离去尘垢、无秽、清净。\textbf{证得生的灭尽者},即证得涅槃者。\textbf{他们称其为觉悟},即由具足此出世间毗婆舍那、死生智、觉慧故,以及由以此离去尘垢等而从烦恼睡眠中觉醒故,以此行道而证得生的灭尽者,他们称之为觉悟。
\item 或者,\textbf{省思了全部劫波}\footnote{Kappa 兼有「思惟」与「劫波」之义,义注两存之。},即如\begin{quoting}于许多坏成劫,我曾在某处……(如是语第 99 经)\end{quoting}等方法省思之义。以此说了第一明。\textbf{轮回、亡殁与投生两者},即有情的死生及此二者的轮回,以\begin{quoting}确实,这些有情……(如是语第 99 经)\end{quoting}等方法省思之义。以此说了第二明。以其余说了第三明。因为以漏尽智而离去尘垢及证得涅槃。\textbf{他们称其为觉悟},即如是具足三明的觉慧者,他们称之为觉悟。\end{enumerate}

\textbf{于是,游行者会堂欢喜、随喜于世尊之所说,便心满意足、愉悦、踊跃、生起喜悦,进一步问世尊问题:}

Atha kho Sabhiyo paribbājako Bhagavato bhāsitaṃ abhinanditvā anumoditvā attamano pamudito udaggo pītisomanassajāto Bhagavantaṃ uttariṃ pañhaṃ apucchi:

\subsection\*{\textbf{524} {\footnotesize 〔PTS 518〕}}

\textbf{「证得什么,人们称之为婆罗门?」会堂说,「因何而为沙门?且如何为沐浴者?\\}
\textbf{「如何被称为龙象?我的所问,世尊!请解释!」}

“Kiṃpattinam āhu brāhmaṇaṃ, \textit{(iti Sabhiyo)} samaṇaṃ kena kathañ ca nhātako ti;\\
nāgo ti kathaṃ pavuccati, puṭṭho me Bhagavā byākarohi”. %\hfill\textcolor{gray}{\footnotesize 9}

\subsection\*{\textbf{525} {\footnotesize 〔PTS 519〕}}

\textbf{「排除了一切恶,会堂!」世尊说,「无垢,善等持,坚定,\\}
\textbf{「已超轮回,整全,他无所依,被称作如如,他即是梵。}

“Bāhitvā sabbapāpakāni, \textit{(Sabhiyā ti Bhagavā)} vimalo sādhusamāhito ṭhitatto;\\
saṃsāram aticca kevalī so, asito tādi pavuccate sa brahmā. %\hfill\textcolor{gray}{\footnotesize 10}

\begin{enumerate}\item 如是解答了第一颂所述的问题,在第二颂所述的问题中,因为证得梵的状态、最胜的状态者,即第一义婆罗门,排除了一切恶,所以为显示此义,以此颂解释了第一问。
\item 其义为:若以第四道\textbf{排除了一切恶、坚定},即是说住立,且由排除恶故\textbf{无垢},证得无垢、梵、最胜的状态,因以最上果定止息了引起定的散乱的烦恼尘垢而\textbf{善等持},以克服轮回之因而\textbf{已超轮回},以完结了应作而\textbf{整全},\textbf{他}由不依止于爱、见故而\textbf{无所依},由不为世间法所变故\textbf{被称作如如},如是应得称赞:\textbf{他即是梵}、他是婆罗门。\end{enumerate}

\subsection\*{\textbf{526} {\footnotesize 〔PTS 520〕}}

\textbf{「平静,舍弃了福与恶,离尘,了知了此世他世,\\}
\textbf{「已越过生死,如如者被如实称作沙门。}

Samitāvi pahāya puññapāpaṃ, virajo ñatvā imaṃ parañ ca lokaṃ;\\
jātimaraṇaṃ upātivatto, samaṇo tādi pavuccate tathattā. %\hfill\textcolor{gray}{\footnotesize 11}

\begin{enumerate}\item 而因为以平静了恶为沙门,以沐浴了恶为沐浴者,以不造作罪过被称为龙象,所以为显示此义,此后更以三颂解释了三问。这里,\textbf{平静},即以圣道平息了烦恼而住。\textbf{被如实地称作沙门},即这样的人被称作沙门。至此已解释了问题,其余则是为令会堂于此沙门生起敬意的称赞之语\footnote{义注之所以这样说,是因为「平静 \textit{samitāvi}」已经从语源上解释了「沙门 \textit{samaṇa}」。此经对字词的解释多是语源上的,如第 525 颂的排除与梵,此颂的平静与沙门,528 颂的罪过与龙象,531 颂的贮藏、切断与善,532 颂的淡黄与智者,535 颂的至于受、吠陀与通达诸明,537 颂的戒离、地狱与精进,541 颂的诸漏、执著与圣者,543 颂的驱除与游行者等,都有字面上的联系。}。因为若能平静,他便以不令福恶相续而\textbf{舍弃了福与恶},以离去尘垢而\textbf{离尘},以无常等\textbf{了知了此世他世,已越过生死}而为\textbf{如如者}。\end{enumerate}

\subsection\*{\textbf{527} {\footnotesize 〔PTS 521〕}}

\textbf{「清洗了于内在及外在一切世间的一切恶,\\}
\textbf{「于天人思惟之处,他不思惟,他们称其为沐浴者。}

Ninhāya sabbapāpakāni, ajjhattaṃ bahiddhā ca sabbaloke;\\
devamanussesu kappiyesu, kappaṃ n’eti tam āhu nhātako ti. %\hfill\textcolor{gray}{\footnotesize 12}

\begin{enumerate}\item 而在「清洗了……沐浴者」中,若\textbf{于}被称为\textbf{内在及外在}等\textbf{一切}的处\textbf{世间},对因内外所缘应得发生的\textbf{一切恶},以道智\textbf{清洗}、洗净已,以此清洗了恶之故,\textbf{于天人}以爱、见之思惟\textbf{思惟之处,他不思惟,他们称其为沐浴者},其义当作如是观。\end{enumerate}

\subsection\*{\textbf{528} {\footnotesize 〔PTS 522〕}}

\textbf{「在世间不造作任何罪过,舍离了一切结缚和束缚,\\}
\textbf{「于一切处不羁绊,解脱,如如者被如实称作龙象。」}

Āguṃ na karoti kiñci loke, sabbasaṃyoge visajja bandhanāni;\\
sabbattha na sajjatī vimutto, nāgo tādi pavuccate tathattā” ti. %\hfill\textcolor{gray}{\footnotesize 13}

\begin{enumerate}\item 在第四颂中,\textbf{在世间不造作任何罪过},即若在世间,不造作哪怕一点称为恶的罪过,便\textbf{被如实称作龙象}。至此已解释了问题,其余仍如先前之法,为称赞之语。因为若以道舍弃了罪过之故而不造作罪过,他便\textbf{舍离}、舍断了欲轭等一切轭,以及十种\textbf{结缚和}一切\textbf{束缚},\textbf{于一切处}的蕴等\textbf{不}以任何染著\textbf{羁绊},并以两种解脱\textbf{解脱}而为\textbf{如如者}。\end{enumerate}

\textbf{于是,游行者会堂……进一步问世尊问题:}

Atha kho Sabhiyo paribbājako…pe… Bhagavantaṃ uttariṃ pañhaṃ apucchi:

\subsection\*{\textbf{529} {\footnotesize 〔PTS 523〕}}

\textbf{「诸佛说谁是田地的胜者?」会堂说,「因何而为善?且如何为智者?\\}
\textbf{「如何被称名牟尼?我的所问,世尊!请解释!」}

“Kaṃ khettajinaṃ vadanti buddhā, \textit{(iti Sabhiyo)} kusalaṃ kena kathañ ca paṇḍito ti;\\
muni nāma kathaṃ pavuccati, puṭṭho me Bhagavā byākarohi”. %\hfill\textcolor{gray}{\footnotesize 14}

\subsection\*{\textbf{530} {\footnotesize 〔PTS 524〕}}

\textbf{「省思了全部田地,会堂!」世尊说,「天、人以及梵的田地,\\}
\textbf{「解脱于一切田地的根本束缚,如如者被如实称作田地的胜者。}

“Khettāni viceyya kevalāni, \textit{(Sabhiyā ti Bhagavā)} dibbaṃ mānusakañ ca brahmakhettaṃ;\\
sabbakhettamūlabandhanā pamutto, khettajino tādi pavuccate tathattā. %\hfill\textcolor{gray}{\footnotesize 15}

\begin{enumerate}\item 如是解答了第二颂所述的问题,在第三颂所述的问题中,因为「处」被称为\textbf{田地},如说:\begin{quoting}此眼、此眼处……此田地、此依处。(法集论第 596~598 段)\end{quoting}\textbf{战胜}、征服了这些,或者\textbf{省思}\footnote{义注将颂中的 viceyya 解释成「战胜 \textit{vijeyya}」或「省思 \textit{viceyya}」两者。}、以无常等相考察\textbf{了全部}而无余,且尤其对作为染著之因的\textbf{天、人以及梵的田地}——天有十二类处,人也同样,而梵的田地则为眼等十二类处中的六处\footnote{此处义注的原文疑误,说详菩提比丘注 1468。}——战胜或省思了这一切。而因为一切田地的根本束缚即无明、有爱等,所以便\textbf{解脱于一切田地的根本束缚}。如是,由战胜或省思这些田地故,得名\textbf{田地的胜者}\footnote{Norman 与菩提比丘将「田地的胜者 \textit{khettajina}」译作「田地的知者」,认为 jina < √jñā,并指出此词也见于薄伽梵歌第 13.1 颂,说详菩提比丘注 1469。},所以,以此颂解释了第一问。
\item 这里,有人从\begin{quoting}业是田地,识是种子,渴爱是润泽。(增支部第 3:77 经)\end{quoting}之语,而说业是田地。并解释此中的「天、人以及梵的田地」为:到达天的业为天的田地,到达人的业为人的田地,到达梵的业为梵的田地。其余唯如所述。\end{enumerate}

\subsection\*{\textbf{531} {\footnotesize 〔PTS 525〕}}

\textbf{「省思了全部贮藏,天、人以及梵的贮藏,\\}
\textbf{「解脱于一切贮藏的根本束缚,如如者被如实称作善。}

Kosāni viceyya kevalāni, dibbaṃ mānusakañ ca brahmakosaṃ;\\
sabbakosamūlabandhanā pamutto, kusalo tādi pavuccate tathattā. %\hfill\textcolor{gray}{\footnotesize 16}

\begin{enumerate}\item 而因为由贮藏与自身之义等同故,业被称为「贮藏」,且由切断、断绝彼等则成为善,所以为显示此义,以此颂解释了第二问。其义为:以世、出世间的毗婆舍那从境域及作用上以无常等相\textbf{省思了}被称为善、不善业的\textbf{全部贮藏},且尤其省思了作为染著之因的\textbf{天、人}的八欲界善思\textbf{以及梵的}的九广大善思之\textbf{贮藏}。随后,以此道的修习,\textbf{解脱于}无明、有爱等类的\textbf{一切贮藏的根本束缚},如是,由这些贮藏的切断\textbf{被如实称作善},且为\textbf{如如者}。
\item 或者,由剑鞘与有情及法的住所之义等同故\footnote{剑鞘的原文为 asikosa,与「贮藏 \textit{kosa}」有字面的联系,这里是以住所 \textit{nivāsa} 之义解释贮藏。},当知三有及十二处为「贮藏」。当如是作此中的连结。\end{enumerate}

\subsection\*{\textbf{532} {\footnotesize 〔PTS 526〕}}

\textbf{「省思了内在与外在两种淡黄,净慧者\\}
\textbf{「已越过黑白,如如者被如实称作智者。}

Dubhayāni viceyya paṇḍarāni, ajjhattaṃ bahiddhā ca suddhipañño;\\
kaṇhaṃ sukkaṃ upātivatto, paṇḍito tādi pavuccate tathattā. %\hfill\textcolor{gray}{\footnotesize 17}

\begin{enumerate}\item 且因为不仅因行动被称为智者\footnote{因行动 \textit{paṇḍati} 被称为智者 \textit{paṇḍita}:这里的解释也是语源上的。对「行动」的解释见\textbf{小诵}义注:以行动为智者,意即于现世来世之义利,以智之趣向而行。},更因至于、趋近淡黄,诉诸简择之慧被称为智者,所以为显示此义,以此颂解释了第三问。其义为:以无常等相\textbf{省思了内在与外在}如是\textbf{两种淡黄},即处,因为它们由本性遍净故,并因习俗而被称作如是,省思了彼等,以此行道而扫除尘垢故,\textbf{被如实称作净慧者、智者},因为他以慧而至于彼等淡黄。其余则是其称赞之语。因为他\textbf{已越过}被称为恶与福的\textbf{黑白}而为\textbf{如如者},所以如是称赞。\end{enumerate}

\subsection\*{\textbf{533} {\footnotesize 〔PTS 527〕}}

\textbf{「了知了不善人与善人的法,于内在及外在一切世间,\\}
\textbf{「为人天所应供养,已超染著与罗网,他是牟尼。」}

Asatañ ca satañ ca ñatvā dhammaṃ, ajjhattaṃ bahiddhā ca sabbaloke;\\
devamanussehi pūjanīyo, saṅgaṃ jālam aticca so munī” ti. %\hfill\textcolor{gray}{\footnotesize 18}

\begin{enumerate}\item 而因为说:\begin{quoting}寂默被称为智,即慧、证知……正见,具足此智者为牟尼。(大义释第 15 段)\end{quoting}所以为显示此义,以此颂解释了第四问。其义为:即此不善与善等类的不善人与善人的法,\textbf{于}此\textbf{内在及外在}这\textbf{一切世间},以简择之智\textbf{了知了不善人与善人的法},由了知此故,\textbf{已超}、已越贪等七种\textbf{染著与}爱、见等两种\textbf{罗网}而住,他由具足被称为寂默的简择之智而为\textbf{牟尼}。\textbf{为人天所应供养}则是其称赞之语。因为他由为漏尽牟尼故,值得人天的供养,所以如是称赞。\end{enumerate}

\textbf{于是,游行者会堂……进一步问世尊问题:}

Atha kho Sabhiyo paribbājako…pe… Bhagavantaṃ uttariṃ pañhaṃ apucchi:

\subsection\*{\textbf{534} {\footnotesize 〔PTS 528〕}}

\textbf{「证得什么,人们称之为通达诸明?」会堂说,「因何而为彻知?且如何为具精进?\\}
\textbf{「什么名为高贵?我的所问,世尊!请解释!」}

“Kiṃpattinam āhu vedaguṃ, \textit{(iti Sabhiyo)} anuviditaṃ kena kathañ ca viriyavā ti;\\
ājāniyo kin ti nāma hoti, puṭṭho me Bhagavā byākarohi”. %\hfill\textcolor{gray}{\footnotesize 19}

\subsection\*{\textbf{535} {\footnotesize 〔PTS 529〕}}

\textbf{「省思了全部吠陀,会堂!」世尊说,「凡属于沙门婆罗门的,\\}
\textbf{「于一切受离于贪染,已超一切吠陀,他即通达诸明。}

“Vedāni viceyya kevalāni, \textit{(Sabhiyā ti Bhagavā)} samaṇānaṃ yān’idh’atthi brāhmaṇānaṃ;\\
sabbavedanāsu vītarāgo, sabbaṃ vedam aticca vedagū so. %\hfill\textcolor{gray}{\footnotesize 20}

\begin{enumerate}\item 如是解答了第三颂所述的问题,在第四颂所述的问题中,因为若以四道智之吠陀而行、而至于烦恼的灭尽,他即名为第一义上的\textbf{通达诸明},且若对于一切\textbf{沙门婆罗门}被认为是典籍的\textbf{吠陀},唯由修习道的作用,以无常等\textbf{省思}已,于此以舍断欲贪而\textbf{已超}这\textbf{一切吠陀},若对缘于吠陀或从别处生起的受,\textbf{于}此\textbf{一切受离于贪染},所以为显示此义,不说「证得此」,而以此颂解释了第一问。
\item 或者,因为若以简择之慧省思了吠陀,于此以舍断欲贪已超一切吠陀而起,他即至于、知晓、超越被认为是典籍的吠陀,若于诸受离于贪染,他也至于、超越被认为是受的吠陀,而至于吠陀即通达诸明,所以亦为显示此义,不说「证得此」,而以此颂解释了第一问。\end{enumerate}

\subsection\*{\textbf{536} {\footnotesize 〔PTS 530〕}}

\textbf{「省思了戏论与名色,内在及外在疾病的根本,\\}
\textbf{「解脱于一切疾病的根本束缚,如如者被如实称作彻知。}

Anuvicca papañcanāmarūpaṃ, ajjhattaṃ bahiddhā ca rogamūlaṃ;\\
sabbarogamūlabandhanā pamutto, anuvidito tādi pavuccate tathattā. %\hfill\textcolor{gray}{\footnotesize 21}

\begin{enumerate}\item 而在第二问中,因为随觉被称为「\textbf{彻知}」,且他\textbf{省思了戏论与名色},以无常随观等省思、彻知了自身相续中爱、慢、见等戏论,以及以彼为缘的名色,且不仅对\textbf{内在},而是省思了\textbf{外在疾病的根本},即他人相续中此名色疾病根本的无明、有爱等,或即此戏论,以此修习,\textbf{解脱于一切疾病的根本束缚},或从一切,即从无明、有爱等类的疾病之根本束缚解脱,或即从此戏论解脱,所以为显示此义,以此颂解释了第二问。\end{enumerate}

\subsection\*{\textbf{537} {\footnotesize 〔PTS 531〕}}

\textbf{「于此戒离一切恶,已超地狱之苦,住于精进,\\}
\textbf{「他具精进、具精勤,如如者被如实称作智者。}

Virato idha sabbapāpakehi, nirayadukkhaṃ aticca viriyavāso;\\
so vīriyavā padhānavā, dhīro tādi pavuccate tathattā. %\hfill\textcolor{gray}{\footnotesize 22}

\begin{enumerate}\item 而在「且如何为具精进」中,因为以圣道\textbf{戒离一切恶}者,同样由戒离及未来不结生故,\textbf{已超地狱之苦}而住,\textbf{住于精进},以精进为家,这漏尽者值得被称为\textbf{具精进},所以为显示此义,以此颂解释了第三问。而\textbf{具精勤、智者、如如者}为其称赞之语。因为具精勤因道与禅那之精勤,智者因堪能消灭烦恼之敌,如如者因不变,所以如是称赞。其余当经连结而说。\end{enumerate}

\subsection\*{\textbf{538} {\footnotesize 〔PTS 532〕}}

\textbf{「若其切断束缚,内在及外在染著的根本,\\}
\textbf{「解脱于一切染著的根本束缚,如如者被如实称作高贵。}

Yass’assu lunāni bandhanāni, ajjhattaṃ bahiddhā ca saṅgamūlaṃ;\\
sabbasaṅgamūlabandhanā pamutto, ājāniyo tādi pavuccate tathattā” ti. %\hfill\textcolor{gray}{\footnotesize 23}

\begin{enumerate}\item 而在「什么名为高贵」中,因为舍断了一切邪曲、过失,知晓当作、不当作的马或象在世间被称为「高贵\footnote{高贵 \textit{ājāniya}:在这里形容马或象,即纯种之义,英译作 thoroughbred。}」,但它并未舍断彼等一切过失,而是漏尽者舍断了彼等,所以为显示其在第一义上值得说为「高贵」,以此颂解释了第四问。
\item 其义为:\textbf{若其切断},以慧剑斩断、破碎\textbf{内在及外在},即如是被称为内在及外在结缚的\textbf{束缚}。\textbf{染著的根本},即于彼彼依处,以未越过染著而为染著的根本。
\item 或者,若其切断贪等束缚,即内在及外在染著的根本,他从\textbf{一切},即从作为\textbf{染著的根本},或从作为一切染著的根本的\textbf{束缚解脱},\textbf{被如实称作高贵},而为\textbf{如如者}。\end{enumerate}

\textbf{于是,游行者会堂……进一步问世尊问题:}

Atha kho Sabhiyo paribbājako…pe… Bhagavantaṃ uttariṃ pañhaṃ apucchi:

\subsection\*{\textbf{539} {\footnotesize 〔PTS 533〕}}

\textbf{「证得什么,人们称之为闻解圣典?」会堂说,「因何而为圣者?且如何为具行者?\\}
\textbf{「什么名为游行者?我的所问,世尊!请解释!」}

“Kiṃpattinam āhu sottiyaṃ, \textit{(iti Sabhiyo)} ariyaṃ kena kathañ ca caraṇavā ti;\\
paribbājako kin ti nāma hoti, puṭṭho me Bhagavā byākarohi”. %\hfill\textcolor{gray}{\footnotesize 24}

\subsection\*{\textbf{540} {\footnotesize 〔PTS 534〕}}

\textbf{「听闻、证知了世间的一切法,会堂!」世尊说,「任何有过与无过,\\}
\textbf{「征服、无疑、解脱,于一切处无患,他们称其为闻解圣典。}

“Sutvā sabbadhammaṃ abhiññāya loke, \textit{(Sabhiyā ti Bhagavā)}\\
\makebox[2em]{} sāvajjānavajjaṃ yad atthi kiñci;\\
abhibhuṃ akathaṅkathiṃ vimuttaṃ, anighaṃ sabbadhi-m-āhu sottiyo ti. %\hfill\textcolor{gray}{\footnotesize 25}

\begin{enumerate}\item 如是解答了第四颂所述的问题,在第五颂所述的问题中,因为晓韵律者仅以研究颂诗所赞扬的「闻解圣典」,仅是世俗的闻解圣典,而圣者以多闻及洗刷恶则为第一义的闻解圣典,所以为显示此义,不说「证得此」,而以此颂解释了第一问。
\item 其义为:若在此\textbf{世间},以闻所成慧的作用\textbf{听闻},或以应作的作用听闻,以无常等\textbf{证知了任何有过与无过的}毗婆舍那所经验的\textbf{一切法},以此行道,征服了烦恼及烦恼所住之法,得称为\textbf{征服},他听闻、证知了世间任何有过与无过的一切法而征服,由具闻故,\textbf{他们称其为闻解圣典}。又因为他\textbf{无疑},从烦恼束缚\textbf{解脱},且\textbf{于一切处}、一切蕴处等法\textbf{无患}于贪等疾患,所以他无疑、解脱、于一切处无患,由洗刷恶故,他们也称其为闻解圣典。\end{enumerate}

\subsection\*{\textbf{541} {\footnotesize 〔PTS 535〕}}

\textbf{「斩断了诸漏与执著,这解者不再进入胎室,\\}
\textbf{「除去了三种想与泥沼,他不思惟,他们称其为圣者。}

Chetvā āsavāni ālayāni, vidvā so na upeti gabbhaseyyaṃ;\\
saññaṃ tividhaṃ panujja paṅkaṃ, kappaṃ n’eti tam āhu ariyo ti. %\hfill\textcolor{gray}{\footnotesize 26}

\begin{enumerate}\item 而因为由当为欲求利益之人亲近而为圣者,即当靠近之义,所以为显示其当亲近所藉的功德,以此颂解释了第二问。其义为:以慧剑\textbf{斩断了}四\textbf{漏}与二\textbf{执著},\textbf{这解者}、有智者、分别者、知晓四道者\textbf{不}因再有\textbf{再进入胎室},不进入任何母胎,\textbf{除去了}欲(嗔害)等类的\textbf{三种想与}被称为种种爱欲的\textbf{泥沼},\textbf{他不思惟}爱、见思惟中的任一,如是,\textbf{他们称其},即具足斩断诸漏等功德者,\textbf{为圣者}。
\item 或者,因为由远离恶以及不陷于不幸故而为圣者,所以仍为显示此义,以此颂解释了第二问。因为诸漏等恶法被认为不幸,且以此斩断、除去彼等,并不为彼等所动摇,则彼等便远离了他,且不陷于彼等,所以,以「恶法远离他」之义与「不陷于不幸」之义,他们称其为圣者,当知此中应如是连结。而在此异说中,「这解者不再进入胎室」只是称赞之语。\end{enumerate}

\subsection\*{\textbf{542} {\footnotesize 〔PTS 536〕}}

\textbf{「若于此在诸行中已达成就,善巧,始终知法,\\}
\textbf{「于一切处不羁绊,心已解脱,无有嗔恚,他是具行者。}

Yo idha caraṇesu pattipatto, kusalo sabbadā ājānāti dhammaṃ;\\
sabbattha na sajjati vimuttacitto, paṭighā yassa na santi caraṇavā so. %\hfill\textcolor{gray}{\footnotesize 27}

\begin{enumerate}\item 而在「如何为具行者」中,因为已达应以诸行成就者值得被说为「具行者」,所以为显示此,以此颂解释了第三问。这里,\textbf{若于此},即若于此教法中。\textbf{在诸行中},即于雪山经(第 163 颂)所说的戒等十五法中,依格作理由义。\textbf{已达成就},即已达应成就者,即是说他已达以行为由、以行为因、以行为缘应成就的阿罗汉。\textbf{他是具行者},即他以此诸行已达成就,即具行者。至此已解释了问题,其余则是其称赞之语。因为若以诸行已达成就,他即为\textbf{善巧}、娴熟,且\textbf{始终知}涅槃\textbf{法},因心总是倾向于涅槃,\textbf{于一切}蕴等\textbf{处不羁绊},并以二解脱\textbf{心已解脱,无有嗔恚}。\end{enumerate}

\subsection\*{\textbf{543} {\footnotesize 〔PTS 537〕}}

\textbf{「作为苦之异熟的业,无论上方、下方或四旁中间,\\}
\textbf{「遍知的行者驱除已,随后对伪善、慢、贪、忿怒,\\}
\textbf{「终结了名色,他们称其为已达成就的游行者。」}

Dukkhavepakkaṃ yad atthi kammaṃ, uddham adho tiriyaṃ vā pi majjhe;\\
paribbājayitvā pariññacārī, māyaṃ mānam atho pi lobhakodhaṃ;\\
pariyantam akāsi nāmarūpaṃ, taṃ paribbājakam āhu pattipattan” ti. %\hfill\textcolor{gray}{\footnotesize 28}

\begin{enumerate}\item 而因为以驱除业等名为游行者,所以为显示此义,以此颂解释了第四问。这里,苦即异熟,为\textbf{苦之异熟},由生起转起\footnote{转起 \textit{pavatti}:与结生相对而言。}之苦故,一切三界的业都被言及。\textbf{上方}指过去,\textbf{下方}指将来,\textbf{四旁中间}指现在。因其非上非下,而在四旁二者之间,因此说为「中间」。\textbf{驱除},即除去、清扫。\textbf{遍知的行者},即以智确定已而行者。以上即先前未及的释词。
\item 而其意趣与连结为:凡系属于三时而生苦的任何业,对彼等一切,以圣道枯竭了爱与无明之润泽,以不令结生而\textbf{驱除已},同样,由驱除故,由遍知此业而行,为\textbf{遍知的行者}。且不仅是业,\textbf{随后对}这些\textbf{伪善、慢、贪、忿怒}等法,也以断遍知遍知而行,\textbf{终结了名色},即终结、驱除了名色之义。以驱除这些业等,\textbf{他们称其为游行者}。\textbf{已达成就}则是其称赞之语。\end{enumerate}

\textbf{于是,游行者会堂欢喜、随喜于世尊之所说,心满意足、愉悦、踊跃、生起喜悦,便从坐起,把上衣偏覆一肩,向世尊合掌,面对世尊,以合适的偈颂赞叹:}

Atha kho Sabhiyo paribbājako Bhagavato bhāsitaṃ abhinanditvā anumoditvā attamano pamudito udaggo pītisomanassajāto uṭṭhāyāsanā ekaṃsaṃ uttarāsaṅgaṃ karitvā yena Bhagavā ten’añjaliṃ paṇāmetvā Bhagavantaṃ sammukhā sāruppāhi gāthāhi abhitthavi:

\subsection\*{\textbf{544} {\footnotesize 〔PTS 538〕}}

\textbf{「这六十三种依于沙门的论点,宏慧者!\\}
\textbf{「依于想的标记与想的异教,已经调伏,得度冥暗的暴流。}

“Yāni ca tīṇi yāni ca saṭṭhi, samaṇappavādasitāni bhūripañña;\\
saññakkhara-sañña-nissitāni, osaraṇāni vineyya oghatam’agā. %\hfill\textcolor{gray}{\footnotesize 29}

\begin{enumerate}\item 如是,会堂满意于问题的解释,在「这六十三种」等称赏之颂中,\textbf{异教},即浴场、津渡,即见之义。因为它们包含了「梵网经」中所说的六十二见,与有身见一起而成六十三,且因为它们依于作为其他外道沙门论点的典籍,是因受指示,而非因发生。而因发生,则有「女人、男人」等\textbf{想的标记}的世俗名称,以及愚人由邪寻思与传闻等生起的颠倒\textbf{想}「如此之我应当存在」,\textbf{依于}这两者生起,而非以自身的现量。而世尊\textbf{已经调伏}彼等,\textbf{得度冥暗的暴流},即得度、超越冥暗的暴流、黑暗的暴流。文本也作 oghantam agā,即得度暴流的边际。\end{enumerate}

\subsection\*{\textbf{545} {\footnotesize 〔PTS 539〕}}

\textbf{「你到达边际、到达苦的彼岸,你是阿罗汉、正等正觉,我想你是漏尽者,\\}
\textbf{「放光者、具觉者、广慧者,尽苦边者!你令我得度。}

Antagū si pāragū dukkhassa, arahāsi sammāsambuddho khīṇāsavaṃ taṃ maññe;\\
jutimā mutimā pahūtapañño, dukkhass’antakara atāresi maṃ. %\hfill\textcolor{gray}{\footnotesize 30}

\begin{enumerate}\item 随后,流转之苦的边际与彼岸涅槃,由于以其成就而无有苦,且由于为其对治,就此而说:\textbf{你到达边际、到达苦的彼岸}。或者,世尊由到达涅槃而为到达彼岸者,为称呼他而说:到达彼岸者!你到达苦的边际——当如此连接。完全地觉悟且由自己觉悟为\textbf{正等正觉}。\textbf{我想你},即极恭敬地说:「我认为是你,而非他人。」\textbf{放光者},即具足甚至能驱散他人黑暗的光。\textbf{具觉者},即具足堪能不由于他而知晓应知的觉、慧。\textbf{广慧者},即无尽慧者,此处意指一切知智。\textbf{尽苦边者},是为称呼而说。\textbf{你令我得度},即你令我得度疑惑。\end{enumerate}

\subsection\*{\textbf{546} {\footnotesize 〔PTS 540〕}}

\textbf{「当你了知了我的疑惑,你令我得度了疑,礼敬您!\\}
\textbf{「牟尼!于寂默之路已达成就者!无荒秽者!日种!你为温顺。}

Yaṃ me kaṅkhitam aññāsi, vicikicchā maṃ tārayi namo te;\\
muni monapathesu pattipatta, akhila Ādiccabandhu sorato si. %\hfill\textcolor{gray}{\footnotesize 31}

\begin{enumerate}\item 以下几颂是为致礼敬而说。这里,\textbf{疑惑},即就二十个问题之义而说。因为他以此而曾有疑惑。\textbf{寂默之路},即智之路。\end{enumerate}

\subsection\*{\textbf{547} {\footnotesize 〔PTS 541〕}}

\textbf{「我先前存在的疑惑,你已为我解释,具眼者!\\}
\textbf{「你确是牟尼、等正觉者,你已没有诸盖。}

Yā me kaṅkhā pure āsi, taṃ me byākāsi cakkhumā;\\
addhā munī si sambuddho, natthi nīvaraṇā tava. %\hfill\textcolor{gray}{\footnotesize 32}

\subsection\*{\textbf{548} {\footnotesize 〔PTS 542〕}}

\textbf{「而且你的一切苦恼都已破碎、清除,\\}
\textbf{「清凉,已达调御,具足坚毅,为真实而努力。}

Upāyāsā ca te sabbe, viddhastā vinaḷīkatā;\\
sītibhūto damappatto, dhitimā saccanikkamo. %\hfill\textcolor{gray}{\footnotesize 33}

\begin{enumerate}\item \textbf{清除},即除去根本,即是说断绝。\end{enumerate}

\subsection\*{\textbf{549} {\footnotesize 〔PTS 543〕}}

\textbf{「龙象!你这龙象、大雄之所说,\\}
\textbf{「一切诸天都随喜,包括那罗陀与波婆多。}

Tassa te nāga nāgassa, mahāvīrassa bhāsato;\\
sabbe devānumodanti, ubho Nārada-Pabbatā. %\hfill\textcolor{gray}{\footnotesize 34}

\begin{enumerate}\item \textbf{龙象……龙象},一为呼格,一与「所说……随喜」相连。文本省略了「法的开示」。\textbf{一切诸天},即空居者与地居者。\textbf{那罗陀与波婆多},据说他俩是具慧的天众,他俩也都随喜。他以净喜而说一切,以致礼敬。\end{enumerate}

\subsection\*{\textbf{550} {\footnotesize 〔PTS 544〕}}

\textbf{「礼敬您,高贵的人!礼敬您,最上的人!\\}
\textbf{「在俱有天的世间中,没有与你对等的人。}

Namo te purisājañña, namo te purisuttama;\\
sadevakasmiṃ lokasmiṃ, natthi te paṭipuggalo. %\hfill\textcolor{gray}{\footnotesize 35}

\begin{enumerate}\item 听闻了应当随喜、成就的解释,他便合掌而说\textbf{礼敬您}!\textbf{高贵的人},即人中具足出身者。\textbf{对等的人},即相似的人。\end{enumerate}

\subsection\*{\textbf{551} {\footnotesize 〔PTS 545〕}}

\textbf{「你是佛陀,你是大师,你是征服魔罗者、牟尼,\\}
\textbf{「你已切断了随眠,已度,你令这人类得度。}

Tuvaṃ buddho tuvaṃ satthā, tuvaṃ Mārābhibhū muni;\\
tuvaṃ anusaye chetvā, tiṇṇo tāres’imaṃ pajaṃ. %\hfill\textcolor{gray}{\footnotesize 36}

\begin{enumerate}\item \textbf{你}以通达四谛\textbf{是佛陀},以教授及带领商队是\textbf{大师},以征服四种魔罗是\textbf{征服魔罗者}。\textbf{牟尼},即佛牟尼。\end{enumerate}

\subsection\*{\textbf{552} {\footnotesize 〔PTS 546〕}}

\textbf{「你超越了依持,你破碎了诸漏,\\}
\textbf{「你是狮子,无取著,舍弃了畏与怕。}

Upadhī te samatikkantā, āsavā te padālitā;\\
sīho si anupādāno, pahīnabhayabheravo. %\hfill\textcolor{gray}{\footnotesize 37}

\begin{enumerate}\item \textbf{依持},即蕴、烦恼、种种爱欲、行作等四种\footnote{四种依持,见\textbf{有财者经}第 33 颂注。}。\end{enumerate}

\subsection\*{\textbf{553} {\footnotesize 〔PTS 547〕}}

\textbf{「好比美妙的白莲,不著于水,\\}
\textbf{「如是你不著于福与恶这两者,\\}
\textbf{「英雄!请伸展双足!会堂礼拜大师。」}

Puṇḍarīkaṃ yathā vaggu, toye na upalimpati;\\
evaṃ puññe ca pāpe ca, ubhaye tvaṃ na limpasi;\\
pāde vīra pasārehi, Sabhiyo vandati satthuno” ti. %\hfill\textcolor{gray}{\footnotesize 38}

\begin{enumerate}\item \textbf{美妙},即绝色。\textbf{不著于}世间的\textbf{福},即以不造作彼等,或以在未来不受用先前已作的果报,或以不染于作为其原由的爱、见。\textbf{礼拜大师},即如是说时,便握住双踝,五体投地地礼拜。\end{enumerate}

\textbf{于是,游行者会堂以头顶礼世尊的双足,对世尊说:「希有!尊者!……我皈依世尊、法与比丘僧,尊者!愿我能在世尊跟前出家,愿我能受具足!」}

Atha kho Sabhiyo paribbājako Bhagavato pādesu sirasā nipatitvā Bhagavantaṃ etad avoca: “abhikkantaṃ, bhante…pe… esāhaṃ Bhagavantaṃ saraṇaṃ gacchāmi dhammañ ca bhikkhusaṅghañ ca, labheyyāhaṃ, bhante, Bhagavato santike pabbajjaṃ, labheyyaṃ upasampadan” ti.

\textbf{「会堂!若先前为外道者,希求在此法律中出家、希求具足,他要别住四月,四个月后,由心已坚定的比丘们使其出家,使其受具足为比丘,然而于此,我也知道人的差别。」}

“Yo kho, Sabhiya, aññatitthiyapubbo imasmiṃ dhammavinaye ākaṅkhati pabbajjaṃ, ākaṅkhati upasampadaṃ, so cattāro māse parivasati, catunnaṃ māsānaṃ accayena āraddhacittā bhikkhū pabbājenti, upasampādenti bhikkhubhāvāya, api ca m’ettha puggalavemattatā viditā” ti.

\begin{enumerate}\item \textbf{心已坚定},即心已成就。\textbf{然而于此,我也知道人的差别},即于此,我也知道别住中外道之人的多样性,不是所有人都需要别住。那么谁不需要别住?事火的萦发者,出身释氏者,舍弃形相而来者,以及虽未舍弃而来但具足得道果之因者,如游行者会堂这样的。所以,世尊想「而你,会堂!没有需圆满外道义务的别住之因,我知道你希求义利,具足得道果之因」,便许可其出家而说「然而于此,我也知道人的差别」。\end{enumerate}

\textbf{「尊者!如果先前为外道者希求在此法律中出家、希求具足要别住四月,四个月后,由心已坚定的比丘们使其出家,使其受具足为比丘,我愿别住四年,四年后,请心已坚定的比丘们使我出家,使我受具足为比丘!」}

“Sace, bhante, aññatitthiyapubbā imasmiṃ dhammavinaye ākaṅkhantā pabbajjaṃ, ākaṅkhantā upasampadaṃ cattāro māse parivasanti, catunnaṃ māsānaṃ accayena āraddhacittā bhikkhū pabbājenti, upasampādenti bhikkhubhāvāya, ahaṃ cattāri vassāni parivasissāmi, catunnaṃ vassānaṃ accayena āraddhacittā bhikkhū pabbājentu upasampādentu bhikkhubhāvāyā” ti.

\begin{enumerate}\item 然而,会堂为显示自己的恭敬而说「\textbf{尊者!如果}……」。这一切及其它类此者,由意义自明及先前已述故,在此均不解释,因为可依所解释者得知。\end{enumerate}

\textbf{游行者会堂在世尊跟前已得出家,已得具足……尊者会堂便成了众阿罗汉中的某个。}

Alattha kho Sabhiyo paribbājako Bhagavato santike pabbajjaṃ alattha upasampadaṃ…pe… aññataro kho panāyasmā Sabhiyo arahataṃ ahosī ti.

\begin{center}\vspace{1em}会堂经第六\\Sabhiyasuttaṃ chaṭṭhaṃ.\end{center}