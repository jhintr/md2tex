\chapter{八颂品第四}

\section{爱欲经}


对于欲求着爱欲的人,若他于此成功,\hfill\textcolor{gray}{\footnotesize \textbf{773}} \\
有死者既得了所希望的,必然有喜意。


若对这欲求着、生起欲望的人,\hfill\textcolor{gray}{\footnotesize \textbf{774}} \\
那些爱欲消逝,他便如被箭射穿般恼坏。


若避开爱欲,如以足避开蛇头,\hfill\textcolor{gray}{\footnotesize \textbf{775}} \\
他具念,超越这对世间的爱著。


田地、物品、货币,或牛马、奴仆、\hfill\textcolor{gray}{\footnotesize \textbf{776}} \\
妇女、亲眷等种种爱欲,若人贪求,


则诸多无力压制他,诸多危难压迫他,\hfill\textcolor{gray}{\footnotesize \textbf{777}} \\
随后,苦追随他,如水之于漏船。


所以,常常具念之人应避开爱欲,\hfill\textcolor{gray}{\footnotesize \textbf{778}} \\
舍弃了这些,便能度过暴流,如汲水出船,到达彼岸。


\section{洞窟八颂经}


执著洞窟,为众多所覆蔽,持续沉溺于诱惑的人,\hfill\textcolor{gray}{\footnotesize \textbf{779}} \\
这样的人远于远离,因为在世间,爱欲不易舍弃。


源于希望,系缚于有之悦乐,这些难以解脱,因为不能由他人解脱,\hfill\textcolor{gray}{\footnotesize \textbf{780}} \\
关切着以后或以前,渴望着这些或先前的爱欲。


贪求爱欲、执著、愚痴,他们吝啬、住于不正、\hfill\textcolor{gray}{\footnotesize \textbf{781}} \\
陷入苦中,他们悲泣「我们殁后会成为什么」?


所以,人应唯于此修学,应了知世间任何的「不正」,\hfill\textcolor{gray}{\footnotesize \textbf{782}} \\
不应因此而行不正,因为智者们说「此生短暂」。


我看到世间这浑身颤栗着、陷于对诸有的渴爱的人类,\hfill\textcolor{gray}{\footnotesize \textbf{783}} \\
低贱的人们在死亡面前絮叨,不离对有与无有的渴爱。


看!于执为我处颤栗者,如少水枯流中的鱼,\hfill\textcolor{gray}{\footnotesize \textbf{784}} \\
见到此后,无我所者应无执于诸有而行。


对两端应调伏欲,遍知了触,无有贪求,\hfill\textcolor{gray}{\footnotesize \textbf{785}} \\
不做自责之事,智者不染于所见、所闻。


遍知了想,他能度过暴流,牟尼不染于资产,\hfill\textcolor{gray}{\footnotesize \textbf{786}} \\
拔出了箭,不放逸而行,不希求此世与他世。


\section{恶意八颂经}

有些恶意的在说,然后真实意的也在说,\hfill\textcolor{gray}{\footnotesize \textbf{787}} \\
但牟尼不参与生起的言论,所以牟尼无任何荒秽。


被欲引领、住于喜好者,如何能超越自己的见?\hfill\textcolor{gray}{\footnotesize \textbf{788}} \\
从事自身的完整,他只会说他所能了知的。


若人未被问及,却对别人说自己的戒禁,\hfill\textcolor{gray}{\footnotesize \textbf{789}} \\
若唯说自己自身,善人们说这是非圣法。


而寂静的比丘,内在寂静,于戒不夸耀「我是如此」,\hfill\textcolor{gray}{\footnotesize \textbf{790}} \\
若在世间已无任何增盛,善人们说这是圣法。


若其法遍计、造作,存有预设的不洁,\hfill\textcolor{gray}{\footnotesize \textbf{791}} \\
看到自身中的利益,依止缘于干扰的寂静。


见的住著实不易越过,于诸法抉择已即被摄取,\hfill\textcolor{gray}{\footnotesize \textbf{792}} \\
所以,人于这些住著,扬弃且执取法。


除遣者于世间任何的有与无有不存遍计的见,\hfill\textcolor{gray}{\footnotesize \textbf{793}} \\
除遣者舍弃了伪善与慢,他因何能达?他无牵涉。


牵涉者于诸法参与言论,他因何、如何能说无牵涉者?\hfill\textcolor{gray}{\footnotesize \textbf{794}} \\
因为他没有执取、扬弃,他即于此除遣了一切见。


\section{清净八颂经}


「我看见清净、最上的无病,人以所见而清净」,\hfill\textcolor{gray}{\footnotesize \textbf{795}} \\
如是证知,了知了最上,以随观清净,他认可智。


若人以所见而清净,或他以智舍弃苦,\hfill\textcolor{gray}{\footnotesize \textbf{796}} \\
则此有依持者被其它净化,如是语时,见显露他。


婆罗门不由其它,于所见、所闻、戒禁或所觉而说清净,\hfill\textcolor{gray}{\footnotesize \textbf{797}} \\
不染于福与恶,舍弃了执取,不于此造作。


舍弃了前者,束缚后者,跟随干扰,他们不能度脱染著,\hfill\textcolor{gray}{\footnotesize \textbf{798}} \\
他们捉取、扬弃,如猿猴放开又抓取枝条。


人自己受持了禁戒,随处而往,执著于想,\hfill\textcolor{gray}{\footnotesize \textbf{799}} \\
而知者以明体认了法,宏慧者不随处而往。


他平定了一切法,或任何所见、所闻、所觉,\hfill\textcolor{gray}{\footnotesize \textbf{800}} \\
这具见、坦荡而行者,在此世间,谁与同类?


他们不设想,不预设,他们不说「极度清净」,\hfill\textcolor{gray}{\footnotesize \textbf{801}} \\
舍离了系缚于取著的系缚,不对世间任何存有希望。


婆罗门越过界限,已知、已见,他无所摄取,\hfill\textcolor{gray}{\footnotesize \textbf{802}} \\
不染于贪,不染于离贪,他于此不执取更多。


\section{最上八颂经}


以「最上」住于诸见,人将其置于世间之上,\hfill\textcolor{gray}{\footnotesize \textbf{803}} \\
他说其余一切较此「卑下」,所以不能超越争论。


凡他在自身所见、所闻、戒禁或所觉中见到的利益,\hfill\textcolor{gray}{\footnotesize \textbf{804}} \\
他即于此执持之,而视其余的一切为下劣。


凡依止者视其余为卑下,善人们说这是系缚,\hfill\textcolor{gray}{\footnotesize \textbf{805}} \\
所以比丘不应依止所见、所闻、所觉或戒禁。


不应以智,或者以戒禁,在世间构建见,\hfill\textcolor{gray}{\footnotesize \textbf{806}} \\
不应表示自己相等,也不应认为卑下或殊胜。


舍弃了执取,无所取著,他也不依止智,\hfill\textcolor{gray}{\footnotesize \textbf{807}} \\
他在异议中不追随群体,他也不认可任何见。


若其于此两端、有与无有、此世或他世已无誓愿,\hfill\textcolor{gray}{\footnotesize \textbf{808}} \\
于诸法抉择已,他已无任何摄取的住著。


于此,他于所见、所闻或所觉已无些许遍计的想,\hfill\textcolor{gray}{\footnotesize \textbf{809}} \\
这无取于见的婆罗门,在此世间,谁与同类?


他们不设想,不预设,他们也不接受诸法,\hfill\textcolor{gray}{\footnotesize \textbf{810}} \\
婆罗门不被戒禁引领,已到彼岸,不再返回而如如。


\section{老经}


此生实在短暂,不及百年就要死去,\hfill\textcolor{gray}{\footnotesize \textbf{811}} \\
即便能活更久,仍会由衰老而死去。


人们忧伤于执为我者,因为资产无法永存,\hfill\textcolor{gray}{\footnotesize \textbf{812}} \\
看到了这分离确实存在,他不应居于俗家。


凡人以为「这是我的」,都随死亡被抛弃,\hfill\textcolor{gray}{\footnotesize \textbf{813}} \\
智者知晓此已,我的同仁不应倾向于我执。


好比醒来的人见不到梦中的所遇,\hfill\textcolor{gray}{\footnotesize \textbf{814}} \\
如是,他也见不到已死亡的爱人。


人们被见到、被听闻,他们的名字被称及,\hfill\textcolor{gray}{\footnotesize \textbf{815}} \\
而对亡者,唯有名字留存,可供谈论。


贪求于我所者不舍弃忧、悲、悭吝,\hfill\textcolor{gray}{\footnotesize \textbf{816}} \\
所以,得见安稳的牟尼舍弃了资产而行。


对内向而行、亲近远离坐处的比丘,\hfill\textcolor{gray}{\footnotesize \textbf{817}} \\
不在居处显示自身,人们说这对他是合适的。


牟尼不依于一切处,不喜,也无不喜,\hfill\textcolor{gray}{\footnotesize \textbf{818}} \\
悲泣与悭吝之于他,好比水不著于叶。


又好比水滴之于莲叶,好比水不著于莲花,\hfill\textcolor{gray}{\footnotesize \textbf{819}} \\
如是,牟尼不著于所见、所闻或所觉。


因为除遣者不因所见、所闻或所觉而思量,\hfill\textcolor{gray}{\footnotesize \textbf{820}} \\
不希望以其它而清净,因为他既不染著,也不离染。


\section{低舍弥勒经}


「对从事淫欲者,」尊者低舍弥勒说,「请说困扰!先生!\hfill\textcolor{gray}{\footnotesize \textbf{821}} \\
「听了你的教法,我们将修学远离。」


「对从事淫欲者,弥勒!」世尊说,「教法甚至都被忘记,\hfill\textcolor{gray}{\footnotesize \textbf{822}} \\
「且他邪行道,这于他是非圣。


「若先前独自而行,(而今)沉湎淫欲,\hfill\textcolor{gray}{\footnotesize \textbf{823}} \\
「人们说他如失路之车,是世间卑下的凡夫。


「且先前的声誉和名声,他都已丧失,\hfill\textcolor{gray}{\footnotesize \textbf{824}} \\
「见到此后,他应修学,以舍弃淫欲。


「他被思惟占据,如同可怜之人在焦灼,\hfill\textcolor{gray}{\footnotesize \textbf{825}} \\
「听到他人的斥责,这样的人即生愧畏。


「于是,受到他人言语的呵责,他就挥舞刀剑,\hfill\textcolor{gray}{\footnotesize \textbf{826}} \\
「这即他的大贪求,他陷入妄语之中。


「决意独自而行,他被许为智者,\hfill\textcolor{gray}{\footnotesize \textbf{827}} \\
「但若从事淫欲,如钝人被摆布。


「牟尼了知了这之前、之后的过患,\hfill\textcolor{gray}{\footnotesize \textbf{828}} \\
「应努力独自而行,不应沉湎淫欲。


「他唯应修学远离,这是圣者们的最上,\hfill\textcolor{gray}{\footnotesize \textbf{829}} \\
「不应以此认为是最胜,他便在涅槃的跟前。


「对空无而行的牟尼、不关切爱欲者、\hfill\textcolor{gray}{\footnotesize \textbf{830}} \\
「已度过暴流者,系缚于爱欲的世人(徒有)羡慕。」


\section{般修罗经}


他们说「唯于此清净」,说在其它的法中没有清净,\hfill\textcolor{gray}{\footnotesize \textbf{831}} \\
凡所依止,即此说是净,个个执著于各自的真实中。


他们渴求论议,投入集会后,互相认定对方是愚人,\hfill\textcolor{gray}{\footnotesize \textbf{832}} \\
他们依止不同,展开议论,渴求赞赏,自称是善巧。


在集会中进行论述,希望赞赏,变得焦虑,\hfill\textcolor{gray}{\footnotesize \textbf{833}} \\
而受驳斥便生愧畏,他因责备而被激怒,寻找破绽。


当判官们说他的论议缺损、被驳斥,\hfill\textcolor{gray}{\footnotesize \textbf{834}} \\
论败者便生悲忧,叹泣「他超过了我」。


于沙门众中生起的这些争论,其间有胜有负,\hfill\textcolor{gray}{\footnotesize \textbf{835}} \\
见到此后,应远离议论,因为除了赞赏利养,无他义利。


但或在集会之中发表论议后,于此受到赞赏,\hfill\textcolor{gray}{\footnotesize \textbf{836}} \\
他因此喜笑且高兴,达成了义利,如其心意。


这高兴便是其困扰之地,但他仍慢与傲慢地在论说,\hfill\textcolor{gray}{\footnotesize \textbf{837}} \\
见到此后,不应争论,因为善人们说不由此而清净。


好比以王家之食供给的英雄,咆哮而来,渴望敌手,\hfill\textcolor{gray}{\footnotesize \textbf{838}} \\
奔赴他的所在!英雄!先前就已没有了可战斗之事。


若执取见而争论,并说道「唯此真实」,\hfill\textcolor{gray}{\footnotesize \textbf{839}} \\
你应对他们说「于此生起的论议中,并无你的敌军」。


然而,若他们消灭了敌军而行,不以见抵制见,\hfill\textcolor{gray}{\footnotesize \textbf{840}} \\
你能从其中得到什么?般修罗!无物在此被他们执取为最上。


现在,你又开始寻思,心中思索着成见,\hfill\textcolor{gray}{\footnotesize \textbf{841}} \\
你已与除遣者相遇,却仍不堪有所进益。


\section{摩根提耶经}


「见到渴爱、不喜、贪染后,尚于淫欲毫无欲望,\hfill\textcolor{gray}{\footnotesize \textbf{842}} \\
「何况这充满屎尿者?我甚至不愿用脚去触碰她。」


「若你不希望这样的宝,为众王所愿求的女人,\hfill\textcolor{gray}{\footnotesize \textbf{843}} \\
「请宣说是怎样的见、戒禁、活命与有之投生!」


「于诸法抉择已,摩根提耶!」世尊说,「不被『我宣说此』摄取,\hfill\textcolor{gray}{\footnotesize \textbf{844}} \\
「且看见诸见而无取,当简别时,我得见内在寂静。」


「这些裁断、遍计,」摩根提耶说,「牟尼!无取于彼等,请说\hfill\textcolor{gray}{\footnotesize \textbf{845}} \\
「这『内在寂静』之义!智者们如何宣说?」


「不以见、闻、智,摩根提耶!」世尊说,「也不以戒禁而说清净,\hfill\textcolor{gray}{\footnotesize \textbf{846}} \\
「不以无见、无闻、无智、无戒、无禁,也不以此,\\
「对这些放弃而无取,寂静者无依止,不会渴望有。」


「若不以见、闻、智,」摩根提耶说,「也不以戒禁而说清净,\hfill\textcolor{gray}{\footnotesize \textbf{847}} \\
「不以无见、无闻、无智、无戒、无禁,也不以此,\\
「我认为实是愚痴之法,有些人以见认可清净。」


「依于见而追问,摩根提耶!」世尊说,「在摄取中陷入困惑,\hfill\textcolor{gray}{\footnotesize \textbf{848}} \\
「不能从中见到些许之想,所以,你视之为愚痴。


「若认为是同等、殊胜或低下,他因此而争论,\hfill\textcolor{gray}{\footnotesize \textbf{849}} \\
「当于三者无动摇,他便没有『同等、殊胜』。


「这婆罗门会说什么『真实』,或以何争论『虚妄』?\hfill\textcolor{gray}{\footnotesize \textbf{850}} \\
「若在其中没有相等或不等,他能与谁进行论议?


「舍弃了住处,无居所而流动,牟尼不在村中建立亲密,\hfill\textcolor{gray}{\footnotesize \textbf{851}} \\
「空乏爱欲,不作预设,他不会与人争吵辩论。


「独处者所据以在世间游行者,龙象不执取之而说,\hfill\textcolor{gray}{\footnotesize \textbf{852}} \\
「好比水生的带刺荷花,不染于水与淤泥,\\
「如是寂静、无求的牟尼,不染于爱欲与世间。


「通达诸明者非见行者,他不以觉生起慢,因为他不参与,\hfill\textcolor{gray}{\footnotesize \textbf{853}} \\
「不被业、也不被所闻引领,他不陷入住著。


「离想者没有系缚,慧解脱者没有愚痴,\hfill\textcolor{gray}{\footnotesize \textbf{854}} \\
「若执取想与见,他们便冲突着在世间游行。」


\section{前分离经}


「何等知见、何等戒,才能被称为寂静?\hfill\textcolor{gray}{\footnotesize \textbf{855}} \\
「乔达摩!请告诉我!当被问到这最上之人!」


「在分离之前离爱,」世尊说,「不依止前端,\hfill\textcolor{gray}{\footnotesize \textbf{856}} \\
「在中间不可估量,他没有预设。


「不忿怒,不惊怖,不吹嘘,不恶作,\hfill\textcolor{gray}{\footnotesize \textbf{857}} \\
「思而后言,不掉举,他实为制语的牟尼。


「不执于未来,不忧伤过去,\hfill\textcolor{gray}{\footnotesize \textbf{858}} \\
「于诸触得见远离,于诸见不被引领。


「内向,不诡诈,不渴望,不悭吝,\hfill\textcolor{gray}{\footnotesize \textbf{859}} \\
「不鲁莽,不嫌厌,且不从事诽谤。


「不享受愉悦,且不存傲慢,\hfill\textcolor{gray}{\footnotesize \textbf{860}} \\
「柔和,富有辩才,不信,不离染。


「不为欲求利养而修学,也不恼怒于无利养,\hfill\textcolor{gray}{\footnotesize \textbf{861}} \\
「不对立,也不因渴爱而贪求众味。


「舍,始终具念,在世间不认为是同等、\hfill\textcolor{gray}{\footnotesize \textbf{862}} \\
「殊胜或低下,他已没有增盛。


「他已没有依止,了知了法而无依止,\hfill\textcolor{gray}{\footnotesize \textbf{863}} \\
「他没有对有或离有的渴爱。


「我说他为寂静,不关切爱欲,\hfill\textcolor{gray}{\footnotesize \textbf{864}} \\
「他已没有系缚,已度过爱著。


「他没有孩子、牲畜、田地、物品,\hfill\textcolor{gray}{\footnotesize \textbf{865}} \\
「在他那里,没有执取或扬弃。


「凡夫以及沙门、婆罗门对他所说的,\hfill\textcolor{gray}{\footnotesize \textbf{866}} \\
「他不以为意,所以他于论议不动摇。


「离贪求,不悭吝,牟尼不说上等、\hfill\textcolor{gray}{\footnotesize \textbf{867}} \\
「同等或下等,无思惟者不起思惟。


「他在世间没有自我,也不因不存在者忧伤,\hfill\textcolor{gray}{\footnotesize \textbf{868}} \\
「不趣向诸法,他实被称为寂静者。」


\section{争辩争论经}


「争辩、争论从何而生?还有悲、忧以及悭吝、\hfill\textcolor{gray}{\footnotesize \textbf{869}} \\
「慢、傲慢以及诽谤,请您快说它们从何而生?」


「争辩、争论从喜爱而生,还有悲、忧以及悭吝、\hfill\textcolor{gray}{\footnotesize \textbf{870}} \\
「慢、傲慢以及诽谤,\\
「争辩、争论与悭吝相关,当争论生起便有诽谤。」


「那么世间的喜爱以何为因,还有行于世间的贪?\hfill\textcolor{gray}{\footnotesize \textbf{871}} \\
「人对未来的希望与达成以何为因?」


「世间的喜爱以欲为因,还有行于世间的贪,\hfill\textcolor{gray}{\footnotesize \textbf{872}} \\
「人对未来的希望与达成以此为因。」


「世间的欲以何为因?还有抉择从何而生?\hfill\textcolor{gray}{\footnotesize \textbf{873}} \\
「以及忿怒、妄语、疑,或是沙门所说的诸法?」


「凡在世间说『愉悦、不愉悦』,欲便依此而生,\hfill\textcolor{gray}{\footnotesize \textbf{874}} \\
「于诸色中见到了离有与有,人在世间做出抉择。


「忿怒、妄语、疑,这些法当二者存在时也是,\hfill\textcolor{gray}{\footnotesize \textbf{875}} \\
「有疑者应在智路上修学,法由沙门在了知后宣说。」


「愉悦、不愉悦以何为因?当什么不存在即无彼等?\hfill\textcolor{gray}{\footnotesize \textbf{876}} \\
「这『离有与有』之义,请对我说它以何为因?」


「愉悦、不愉悦以触为因,当触不存在即无彼等,\hfill\textcolor{gray}{\footnotesize \textbf{877}} \\
「这『离有与有』之义,我对你说,它以此为因。」


「那么世间的触以何为因?执取从何而生?\hfill\textcolor{gray}{\footnotesize \textbf{878}} \\
「当什么不存在即无我执?当什么灭去则诸触不触?」


「缘名与色而有触,以希求为因而有执取,\hfill\textcolor{gray}{\footnotesize \textbf{879}} \\
「当希求不存在即无我执,当诸色灭去则诸触不触。」


「体认何等者的色灭去?乐与苦又如何灭去?\hfill\textcolor{gray}{\footnotesize \textbf{880}} \\
「请对我说如何灭去!『我们应了知此』,这是我的心意。」


「非想之有想者,非异想之有想者,也非无想者,非灭想者,\hfill\textcolor{gray}{\footnotesize \textbf{881}} \\
「如是体认者的色即灭去,因为以想为因,即名为戏论。」


「我们所问的,您都已向我们解说,我们另有所问,请您快说!\hfill\textcolor{gray}{\footnotesize \textbf{882}} \\
「于此,是否有些智者说,至此已是夜叉最上的清净,\\
「还是说除此之外,另有其它?」


「于此,有些智者也说,至此已是夜叉最上的清净,\hfill\textcolor{gray}{\footnotesize \textbf{883}} \\
「然而其中有些,自称是善巧于无余依者,说尚有时日。


「了知了这些为『依止』,这牟尼了知了依止而省视,\hfill\textcolor{gray}{\footnotesize \textbf{884}} \\
「了知已即解脱,不落入争论,智者不入于有与无有。」


\section{小阵经}


「住于各自的见,善巧者们争执而说种种:\hfill\textcolor{gray}{\footnotesize \textbf{885}} \\
「『若如是知,他便明了法,若批评此,他即非整全』。


「他们如是争执而争论,并说『对方是愚人、不善巧』,\hfill\textcolor{gray}{\footnotesize \textbf{886}} \\
「此中何者是真实之论?因为他们全都自称是善巧者。」


「若不认可对方的法,便是愚人、野蛮、劣慧者,\hfill\textcolor{gray}{\footnotesize \textbf{887}} \\
「那么全部都是愚人、极劣慧者,因为他们全部都住于见。


「然而,如果以自己的见便洁净、净慧、善巧、具慧,\hfill\textcolor{gray}{\footnotesize \textbf{888}} \\
「那么其中无人是劣慧者,因为他们的见都同样完整。


「愚人们互相对彼此所说的,我不说『这是如实』,\hfill\textcolor{gray}{\footnotesize \textbf{889}} \\
「他们认为各自的见是真实,所以认定对方是愚人。」


「有些说是『真实、如实』的,其他人却说是『虚无、虚妄』,\hfill\textcolor{gray}{\footnotesize \textbf{890}} \\
「他们如是争执而争论,为什么众沙门说辞不一?」


「因为真实唯一,而非有二,于此了知的人不会争论,\hfill\textcolor{gray}{\footnotesize \textbf{891}} \\
「他们自己赞扬各种的真实,所以,众沙门说辞不一。」


「那为什么要说各种的真实?论说者们都自称善巧,\hfill\textcolor{gray}{\footnotesize \textbf{892}} \\
「是所闻的真实种类繁多,还是他们随念寻思?」


「真实并非种类繁多、在世间是常,除非因想,\hfill\textcolor{gray}{\footnotesize \textbf{893}} \\
「于诸见遍计寻思已,他们说『真实、虚妄』之二元法。


「所见、所闻、戒禁、或所觉,依于这些便显露轻侮,\hfill\textcolor{gray}{\footnotesize \textbf{894}} \\
「立于抉择便作讪笑,并说『对方是愚人、不善巧』。


「正因为认定对方是愚人,便以此说自己是善巧,\hfill\textcolor{gray}{\footnotesize \textbf{895}} \\
「他自称自己本人为善巧,轻侮别人,说着这些。


「他以过误之见而完整,以慢而迷醉,自认为圆满,\hfill\textcolor{gray}{\footnotesize \textbf{896}} \\
「自己在心中为自己灌顶,因为他的见是那么完整。


「若以对方的言语便低劣,则自身也一并是劣慧者,\hfill\textcolor{gray}{\footnotesize \textbf{897}} \\
「若他自己通达诸明,是智者,则沙门众中无人是愚人。


「若宣说除此之外的法,他们便有违清净、非整全,\hfill\textcolor{gray}{\footnotesize \textbf{898}} \\
「如是众外道宣说种种,因为他们被自见之贪染所染著。


「他们争论『唯于此清净』,说在其它的法中没有清净,\hfill\textcolor{gray}{\footnotesize \textbf{899}} \\
「如是众外道执著种种,于此努力宣扬着自己的路。


「而努力宣扬着自己的路,他在此会认定哪个对方是愚人?\hfill\textcolor{gray}{\footnotesize \textbf{900}} \\
「当说对方是愚人、非清净法时,他自己会引起纠纷。


「立于抉择,以自身度量,他在世间愈加陷入争论,\hfill\textcolor{gray}{\footnotesize \textbf{901}} \\
「舍弃了一切抉择,人在世间便不再制造纠纷。」


\section{大阵经}


「凡是那些住于见者,争论道『唯此真实』,\hfill\textcolor{gray}{\footnotesize \textbf{902}} \\
「他们全都引来责备,还是于此也受到赞赏?」


「而这实微少,不足以平息,我说有两种争论的果,\hfill\textcolor{gray}{\footnotesize \textbf{903}} \\
「见到此后,观照着不争之地的安稳,他不应争论。


「凡是那些共许、凡俗的,知者不入于所有这些,\hfill\textcolor{gray}{\footnotesize \textbf{904}} \\
「无牵涉者不堪忍耐所见、所闻,如何能生牵涉?


「以戒为最高者以自制说清净,受持了禁戒而护持,\hfill\textcolor{gray}{\footnotesize \textbf{905}} \\
「『我们唯于此修学,然后便能清净』,陷入于有,自称善巧。


「如果丧失戒禁,他因违犯了业而颤栗,\hfill\textcolor{gray}{\footnotesize \textbf{906}} \\
「他渴望、愿求清净,如离家出行者失去了商队。


「舍弃了一切戒禁,以及有过、无过的业,\hfill\textcolor{gray}{\footnotesize \textbf{907}} \\
「不愿求清净、不清净,应戒离而行,无取于寂静。


「依于苦行或厌离,抑或所见、所闻、所觉,\hfill\textcolor{gray}{\footnotesize \textbf{908}} \\
「上流者们赞叹清净,未离于对有或无有的渴爱。


「愿求者有渴望,或颤栗于遍计,\hfill\textcolor{gray}{\footnotesize \textbf{909}} \\
「于此无亡殁与转生者,他为何颤栗,又渴望何方?」


「有些说是『最上』的法,但其他人却说是『卑下』,\hfill\textcolor{gray}{\footnotesize \textbf{910}} \\
「此中何者是真实之论?因为他们全都自称是善巧者。」


「因为他们说自己的法圆满,便说他人的法卑下,\hfill\textcolor{gray}{\footnotesize \textbf{911}} \\
「他们如是争执而争论,说各自共许的真实。


「如果因被对方蔑视便卑下,则诸法中便没有殊胜的,\hfill\textcolor{gray}{\footnotesize \textbf{912}} \\
「因为他们都说其他的法低下,努力宣扬着自己。


「好比他们赞赏自己的路,同样也供养他们的正法,\hfill\textcolor{gray}{\footnotesize \textbf{913}} \\
「那么一切论说都将是如实的,因为清净对他们唯是各别的。


「婆罗门不受他人引领,于诸法抉择已不被摄取,\hfill\textcolor{gray}{\footnotesize \textbf{914}} \\
「所以,超越了争论,因为他不视其他法为更胜。


「『我知、我见,此即如是』,有些人以见认可清净,\hfill\textcolor{gray}{\footnotesize \textbf{915}} \\
「如果他看到了,这对他有什么?他们忽略后,以其它说清净。


「当人看时,便能见名色,既见已,便能知晓唯有这些,\hfill\textcolor{gray}{\footnotesize \textbf{916}} \\
「让他随意看多或少,因为善巧者们说不能以此而清净。


「住著论者实在不易调伏,预设着遍计的见,\hfill\textcolor{gray}{\footnotesize \textbf{917}} \\
「对所依止的,自许清净者于此说是净,他于此作如是见。


「婆罗门经省察,不起思惟,不流于见,也不缚于智,\hfill\textcolor{gray}{\footnotesize \textbf{918}} \\
「且他了知凡俗的共许后,舍,让他人执取。


「舍离了系缚,牟尼于此世间的种种争论,不随流于众,\hfill\textcolor{gray}{\footnotesize \textbf{919}} \\
「于诸不寂静中,他寂静、舍,无取,让他人执取。


「舍弃了过去诸漏,不造新者,不随欲而往,也非住著论者,\hfill\textcolor{gray}{\footnotesize \textbf{920}} \\
「这智者解脱于见,不染于世间,不谴责自己。


「他平定了一切法,或任何所见、所闻、所觉,\hfill\textcolor{gray}{\footnotesize \textbf{921}} \\
「这牟尼放下重担,已解脱,不思惟,不抑止,不愿求。」


\section{迅速经}


「我问你,日种!远离以及寂静的境地,大仙!\hfill\textcolor{gray}{\footnotesize \textbf{922}} \\
「如何见后,比丘便涅槃,于此世间都无所取?」


「名为戏论的根本,」世尊说,「『我是』等一切,他应以智慧止息,\hfill\textcolor{gray}{\footnotesize \textbf{923}} \\
「凡是内在的渴爱,调伏彼等已,应始终具念修学。


「任何他所证知的法,内在的或外在的,\hfill\textcolor{gray}{\footnotesize \textbf{924}} \\
「不应以此强势,因为这不是善人们所说的寂灭。


「他不应以此认为更优、更劣,或是相同,\hfill\textcolor{gray}{\footnotesize \textbf{925}} \\
「为种种形相所触时,他不应将自己区分。


「他唯应寂止其内在,比丘不应从别处寻求寂静,\hfill\textcolor{gray}{\footnotesize \textbf{926}} \\
「对于内在寂静者,没有执取,又何来扬弃?


「好比在大海中间,不起波浪而住立,\hfill\textcolor{gray}{\footnotesize \textbf{927}} \\
「如是他应住立不动,比丘不应生起任何增盛。」


「开眼者已解说了调伏危难的亲证之法,\hfill\textcolor{gray}{\footnotesize \textbf{928}} \\
「大德!请宣说行道、波罗提木叉和三摩地!」


「切勿以眼摇曳,于村谈应遮耳,\hfill\textcolor{gray}{\footnotesize \textbf{929}} \\
「不应贪求众味,也不应执取世间任何为我所。


「当他为触所触时,比丘不应起任何悲伤,\hfill\textcolor{gray}{\footnotesize \textbf{930}} \\
「不应渴望有,于诸恐怖也不应震颤。


「然后,食物、饮品、硬食、衣服等,\hfill\textcolor{gray}{\footnotesize \textbf{931}} \\
「获得后不应积贮,未获得这些时也不应焦渴。


「应禅修,不应游步,应戒离恶作,不应放逸,\hfill\textcolor{gray}{\footnotesize \textbf{932}} \\
「然后,比丘应居于安静的坐卧处。


「不应多睡眠,应保持警觉,热忱,\hfill\textcolor{gray}{\footnotesize \textbf{933}} \\
「应舍弃倦怠、伪善、戏笑、嬉戏、淫欲以及严饰。


「不应使用阿闼婆、梦、相,以及星占,\hfill\textcolor{gray}{\footnotesize \textbf{934}} \\
「我的弟子不应从事鸣声、助孕、医疗。


「比丘不应震颤于责备,被赞赏时不应高举,\hfill\textcolor{gray}{\footnotesize \textbf{935}} \\
「应去除贪以及悭吝、忿怒与诽谤。

「不应从事买卖,比丘在任何处不应引来指责,\hfill\textcolor{gray}{\footnotesize \textbf{936}} \\
「且在村中不应冒犯,不应为好乐利养而与人闲谈。


「比丘既不应夸耀,也不应说有企图的话,\hfill\textcolor{gray}{\footnotesize \textbf{937}} \\
「不应变得鲁莽,不应谈论争议之论。


「不应堕入妄语,不应知而行狡诈,\hfill\textcolor{gray}{\footnotesize \textbf{938}} \\
「不应以活命、慧、戒禁蔑视他人。

「当被激怒时,听到沙门或凡夫们的众多话语后,\hfill\textcolor{gray}{\footnotesize \textbf{939}} \\
「不应以恶口回答他们,因为善人们不制造敌对。


「且知晓了这法,比丘审视着,应始终具念而修学,\hfill\textcolor{gray}{\footnotesize \textbf{940}} \\
「了知了寂灭为寂静,在乔达摩的教法中不应放逸。


「因为他是征服者,不被征服,得见亲证之法,而非传闻,\hfill\textcolor{gray}{\footnotesize \textbf{941}} \\
「所以,应当在彼世尊的教法内不放逸,始终礼敬而随学。」


\section{执杖经}


从执杖产生怖畏,请看人的纠纷!\hfill\textcolor{gray}{\footnotesize \textbf{942}} \\
我将宣说如我所经历的悚惧。


看到颤栗的人类,好比少水中的鱼,\hfill\textcolor{gray}{\footnotesize \textbf{943}} \\
看到彼此的敌对,怖畏便进入了我。


世间无处坚实,一切方向动荡,\hfill\textcolor{gray}{\footnotesize \textbf{944}} \\
希求着自己的居处,我不见未被居住者。


但最后却看到敌对,我生起了不喜,\hfill\textcolor{gray}{\footnotesize \textbf{945}} \\
然后,我见到其中的箭,难以察觉、依附于心。


被这箭射中者,四处彷徨,\hfill\textcolor{gray}{\footnotesize \textbf{946}} \\
拔出了这箭,他不再奔突、不再沉沦。


于此,众学被诵出,\hfill\textcolor{gray}{\footnotesize \textbf{947}} \\
凡在世间被结缚者,不应从事于其中,\\
突破了一切爱欲,应修学自己的涅槃。


他应真实,不鲁莽,不伪善,去除诽谤,\hfill\textcolor{gray}{\footnotesize \textbf{948}} \\
不忿怒,牟尼应超越贪之恶与悭贪。


他应忍耐睡眠、倦怠、昏沉,不应放逸而住,\hfill\textcolor{gray}{\footnotesize \textbf{949}} \\
存意涅槃的人,不应住于傲慢。


不应堕入妄语,不应爱执于色,\hfill\textcolor{gray}{\footnotesize \textbf{950}} \\
且应遍知慢,应戒离暴力而行。


不应喜于故旧,不应偏爱新者,\hfill\textcolor{gray}{\footnotesize \textbf{951}} \\
于正消逝者不应忧伤,不应束缚于钩牵。


我说贪求为大暴流,我说渴望为奔流,\hfill\textcolor{gray}{\footnotesize \textbf{952}} \\
所缘为震动,而爱欲的泥沼难以超越。


牟尼不偏离真实,婆罗门立于高地,\hfill\textcolor{gray}{\footnotesize \textbf{953}} \\
他舍遣了一切,他实被称为寂静者。


他实为知者,他通达诸明,了知了法而无依止,\hfill\textcolor{gray}{\footnotesize \textbf{954}} \\
他在世间举止正当,于此无所羡慕。


若于此超越了爱欲,世间难以超越的染著,\hfill\textcolor{gray}{\footnotesize \textbf{955}} \\
他便不再忧伤,不再忧虑,截断了水流,没有束缚。


让先前的凋萎,你切莫有任何后来,\hfill\textcolor{gray}{\footnotesize \textbf{956}} \\
如果你不执取中间,你将寂静而行。


于一切名色,没有执为我者,\hfill\textcolor{gray}{\footnotesize \textbf{957}} \\
且不因不存在而忧伤,他在世间便不衰损。


他没有任何「这是我的」,抑或「他人的」,\hfill\textcolor{gray}{\footnotesize \textbf{958}} \\
他找不到执为我者,不忧伤「这不是我的」。


他不苛刻,不贪求,不动摇,于一切处平等,\hfill\textcolor{gray}{\footnotesize \textbf{959}} \\
当被问及,我说这即是不动摇者的利益。


不动摇的了知者,已无任何的行作,\hfill\textcolor{gray}{\footnotesize \textbf{960}} \\
他戒离各种努力,于一切处见安稳。


牟尼不说同等、下等、上等,\hfill\textcolor{gray}{\footnotesize \textbf{961}} \\
他寂静,离于悭吝,不执取,不扬弃。


\section{舍利弗经}


「我此前未见过,」尊者舍利弗说,「也未从任何人听闻,\hfill\textcolor{gray}{\footnotesize \textbf{962}} \\
「如是言语善妙的大师,从兜率天来的众主。


「如同为俱有天的世间所见,具眼者\hfill\textcolor{gray}{\footnotesize \textbf{963}} \\
「驱散了一切暗冥,他独自证得喜乐。


「为了众多在此被束缚者,我带着问题,前往\hfill\textcolor{gray}{\footnotesize \textbf{964}} \\
「这无依止、如如、无诡诈、来作众主的佛陀。


「对于生起嫌厌的比丘,亲近空旷的坐处,\hfill\textcolor{gray}{\footnotesize \textbf{965}} \\
「或树下、塚间,或群山的洞窟,


「种种的卧处,那里有多么可怕?\hfill\textcolor{gray}{\footnotesize \textbf{966}} \\
「在无声的坐卧处,比丘为何能不颤抖?


「对于前往未至之方者,世间有多少危难\hfill\textcolor{gray}{\footnotesize \textbf{967}} \\
「比丘应在边鄙的坐卧处去克服?


「他的言路应如何?他在此的行处应如何?\hfill\textcolor{gray}{\footnotesize \textbf{968}} \\
「自励的比丘的戒禁应如何?


「他专一、贤明、具念,受持何学\hfill\textcolor{gray}{\footnotesize \textbf{969}} \\
「能驱除自身的垢秽,如同锻工之于银?」


「舍利弗!对于生起嫌厌、」世尊说,「若亲近空旷的坐卧处、\hfill\textcolor{gray}{\footnotesize \textbf{970}} \\
「欲求等觉者,我将对你说安乐与随法,如同了知者。


「坚定、具念、具制限而行的比丘不应怖畏五种怖畏,\hfill\textcolor{gray}{\footnotesize \textbf{971}} \\
「虻、蛾、蛇、人的攻击与四足者。


「不应惊怖于异法者,即便见到了其中诸多可怕,\hfill\textcolor{gray}{\footnotesize \textbf{972}} \\
「然后,他应克服其它危难,追随着善。


「为疾患、饥饿所感,应忍耐寒冷、炎热,\hfill\textcolor{gray}{\footnotesize \textbf{973}} \\
「他为种种这些所感,无家者勇猛精进已,应努力作为。


「不应盗窃,不应妄语,对弱者、强者应以慈遍满,\hfill\textcolor{gray}{\footnotesize \textbf{974}} \\
「凡所了知的意的扰动,应以『这是黑分』驱散之。


「不应沦于忿怒、傲慢的控制,应掘断其根而立,\hfill\textcolor{gray}{\footnotesize \textbf{975}} \\
「然后,克服喜爱或不喜爱时,应完全克服。


「以智慧为先导,善妙、欢喜者应镇伏这些危难,\hfill\textcolor{gray}{\footnotesize \textbf{976}} \\
「他应忍耐边鄙卧处的不喜,他应忍耐四种悲法:


「『我将吃什么,或我将在哪吃,我睡得很苦,今天将在哪睡』,\hfill\textcolor{gray}{\footnotesize \textbf{977}} \\
「有学、无居所而行者,应调伏这些悲寻。


「适时地获得了食物与衣服,为了于此知足,他应知量,\hfill\textcolor{gray}{\footnotesize \textbf{978}} \\
「他守护此等,在村中自制而行,即便被激怒也不应说恶语。


「目光下视,且不游步,应从事禅那,常事醒觉,\hfill\textcolor{gray}{\footnotesize \textbf{979}} \\
「等持于舍,他应断绝寻、意乐、恶作。


「当被言语呵责,具念者应欢喜,应破除对同梵行的荒秽,\hfill\textcolor{gray}{\footnotesize \textbf{980}} \\
「他应说善语,而不过分,不应存心于闲谈。


「然后,在世间有五尘,对彼等具念,为了调伏而修学,\hfill\textcolor{gray}{\footnotesize \textbf{981}} \\
「能忍耐对于色、声、味、香、触的贪染。


「应调伏对这些法的欲,比丘具念、善解脱心,\hfill\textcolor{gray}{\footnotesize \textbf{982}} \\
「他时常正当地审视着法,成就专一,他便能破除暗冥。」
