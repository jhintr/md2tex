\section{摩根提耶经}

\begin{center}Māgaṇḍiya Sutta\end{center}\vspace{1em}

\begin{enumerate}\item 缘起为何?一时,世尊住舍卫国,晨朝以佛眼观察世间,见到俱卢国斑御的村民,名叫摩根提耶的婆罗门及其妻子证阿罗汉的近依后,即从舍卫国去到那里,在离斑御不远的某处密林坐下,放出金光。摩根提耶也正好去到那里洗脸,看到金光后,想「这是什么」,四下观望,见到世尊后,即生悦意。据说,他的女儿也是金黄的肤色,许多刹帝利童子等向她求婚而未得。婆罗门有这样的主张「我将只给你拥有金黄肤色的沙门」。他看见世尊后,便动了心:「他和我女儿的肤色相同,我要把女儿给他。」他急忙回家,对婆罗门尼说:「夫人!夫人!我看见和女儿相同肤色的人,快装扮女儿!我们带去见他!」婆罗门尼便以香水沐浴了女儿,以衣服、鲜花等装扮好,正好到了世尊行乞的时间。
\item 于是,世尊入斑御乞食。他们也带了女儿,去到世尊的坐处。在那里未见到世尊,婆罗门尼就四处观望,看到世尊坐处的草垫。而由决意之力,诸佛的坐处及足迹皆不紊乱。她便对婆罗门说:「婆罗门!这是他的草垫?」「唯!夫人!」「那么,婆罗门!我们此行将不会成功了。」「夫人!为什么?」「看!婆罗门!这草垫不紊乱,这不是受用爱欲者所用的。」婆罗门便说:「夫人!当寻求吉祥时,莫说些不祥的!」婆罗门尼又四处观望,看到世尊的足迹,便对婆罗门说:「婆罗门!这是他的足迹?」「唯!夫人!」「看这足迹!婆罗门!这有情并非贪求爱欲者。」「夫人!你怎么知道?」为显示自己的智力,她便说颂:\begin{quoting}染著者的足迹曲起,嗔恚者的足迹尾长,\\愚痴者的足迹急压,去蔽者的足迹如斯。(清净道论·说取业处品第 88 段)\end{quoting}
\item 然而,他们的这谈话被打断了,此时,世尊食事已毕,回到密林。婆罗门尼看到世尊饰以最上之相、散发一寻光辉的形相后,便对婆罗门说:「婆罗门!你看见的就是他吗?」「唯!夫人!」「此行肯定不会成功了,像这样的人受用爱欲,无有是处。」他们正如是对话时,世尊便在草垫上坐下了。于是,婆罗门左手拉着女儿,右手拿着水罐,靠近世尊后说「出家先生!你和这女孩都是金黄的肤色,她对你是合适的,我把她给你作妻子去抚养」,想要给他,便去到世尊跟前而立。世尊好似未与婆罗门交谈,而是与另一人一起谈话般,说了此颂。\end{enumerate}

\subsection\*{\textbf{842} {\footnotesize 〔PTS 835〕}}

\textbf{「见到渴爱、不喜、贪染后,尚于淫欲毫无欲望,\\}
\textbf{「何况这充满屎尿者?我甚至不愿用脚去触碰她。」}

“Disvāna Taṇhaṃ Aratiṃ Ragañ ca, nāhosi chando api methunasmiṃ;\\
kim ev’idaṃ muttakarīsapuṇṇaṃ, pādā pi naṃ samphusituṃ na icche”. %\hfill\textcolor{gray}{\footnotesize 1}

\begin{enumerate}\item 这里,在牧羊人的尼拘律树下,\textbf{见到}化作种种形相而来的魔罗的女儿\textbf{渴爱、不喜、贪染后},我\textbf{尚于淫欲毫无}些许\textbf{欲望},\textbf{何况}见到\textbf{这充满屎尿}的女孩的形相?于一切处\textbf{我甚至不愿用脚去触碰她},如何能与之共处?\end{enumerate}

\subsection\*{\textbf{843} {\footnotesize 〔PTS 836〕}}

\textbf{「若你不希望这样为众多国王追求的女宝,\\}
\textbf{「请宣说是何等的见、戒禁、活命、有之转生!」}

“Etādisañ ce ratanaṃ na icchasi, nāriṃ narindehi bahūhi patthitaṃ;\\
diṭṭhigataṃ sīlavataṃ nu jīvitaṃ, bhavūpapattiñ ca vadesi kīdisaṃ”. %\hfill\textcolor{gray}{\footnotesize 2}

\subsection\*{\textbf{844} {\footnotesize 〔PTS 837〕}}

\textbf{「于诸法抉择已,摩根提耶!」世尊说,「他即不会执取『我宣说此』,\\}
\textbf{「且于众见知晓而无取,当简别时,得见内在的寂静。」}

“‘Idaṃ vadāmī’ ti na tassa hoti, \textit{(Māgaṇḍiyā ti Bhagavā)} dhammesu niccheyya samuggahītaṃ;\\
passañ ca diṭṭhīsu anuggahāya, ajjhattasantiṃ pacinaṃ adassaṃ”. %\hfill\textcolor{gray}{\footnotesize 3}

\subsection\*{\textbf{845} {\footnotesize 〔PTS 838〕}}

\textbf{「抉择于遍计后,」摩根提耶说,「牟尼!你说即于其无取,\\}
\textbf{「这『内在的寂静』之义,智者们如何宣说?」}

“Vinicchayā yāni pakappitāni, \textit{(iti Māgaṇḍiyo)} te ve Munī brūsi anuggahāya;\\
‘ajjhattasantī’ ti yam etam atthaṃ, kathaṃ nu dhīrehi paveditaṃ taṃ”. %\hfill\textcolor{gray}{\footnotesize 4}

\subsection\*{\textbf{846} {\footnotesize 〔PTS 839〕}}

\textbf{「不由见、闻、智,摩根提耶!」世尊说,「也不由戒禁而说清净,\\}
\textbf{「不由无见、无闻、无智、无戒、无禁,也不由此(而说),\\}
\textbf{「除遣了这些而无取,寂静者无所依,不会渴望有。」}

“Na diṭṭhiyā na sutiyā na ñāṇena, \textit{(Māgaṇḍiyā ti Bhagavā)} sīlabbatenāpi na suddhim āha;\\
adiṭṭhiyā assutiyā añāṇā, asīlatā abbatā no pi tena;\\
ete ca nissajja anuggahāya, santo anissāya bhavaṃ na jappe”. %\hfill\textcolor{gray}{\footnotesize 5}

\begin{enumerate}\item \textbf{不由见}等,即拒斥了见、闻、八等至之智、外道之戒禁。\textbf{不由无见而说},即我不以无十事正见而说,同样,\textbf{无闻}即无九分教,\textbf{无智}即无业自性及谛随顺之智,\textbf{无戒}即无别解脱律仪,\textbf{无禁}即无头陀支之禁。\textbf{除遣了这些而无取},即由除去前面「见」等种类的黑分法而除遣,由无关切而从事后面「无见」等种类的白分法而无取。\textbf{寂静者无所依,不应渴望有},即以此修习止息了贪等的寂静者,不依于眼等任何法,乃至连一有也不羡慕、不希求,此即「内在的寂静」之意。\end{enumerate}

\subsection\*{\textbf{847} {\footnotesize 〔PTS 840〕}}

\textbf{「若不由见、闻、智,」摩根提耶说,「也不由戒禁而说清净,\\}
\textbf{「不由无见、无闻、无智、无戒、无禁,也不由此(而说),\\}
\textbf{「我以为实是愚痴之法,有些人以见认可清净。」}

“No ce kira diṭṭhiyā na sutiyā na ñāṇena, \textit{(iti Māgaṇḍiyo)} sīlabbatenāpi na suddhim āha;\\
adiṭṭhiyā assutiyā añāṇā, asīlatā abbatā no pi tena;\\
maññām’ahaṃ momuham eva dhammaṃ, diṭṭhiyā eke paccenti suddhiṃ”. %\hfill\textcolor{gray}{\footnotesize 6}

\subsection\*{\textbf{848} {\footnotesize 〔PTS 841〕}}

\textbf{「依于见而追问,摩根提耶!」世尊说,「即于执取陷入痴迷,\\}
\textbf{「不能从中见到些许之想,所以,你认为是愚痴。}

“Diṭṭhañ ca nissāya anupucchamāno, \textit{(Māgaṇḍiyā ti Bhagavā)} samuggahītesu pamoham āgā;\\
ito ca nāddakkhi aṇum pi saññaṃ, tasmā tuvaṃ momuhato dahāsi. %\hfill\textcolor{gray}{\footnotesize 7}

\begin{enumerate}\item \textbf{不能从}我所说的内在寂静、从修习或从法的开示\textbf{中见到些许}相应\textbf{之想},由此原因,\textbf{你认为}这法\textbf{是愚痴}。\end{enumerate}

\subsection\*{\textbf{849} {\footnotesize 〔PTS 842〕}}

\textbf{「若以为是同等、殊胜或低下,将由此而争论,\\}
\textbf{「当于三者无动摇,他便没有『同等、殊胜』。}

Samo visesī uda vā nihīno, yo maññati so vivadetha tena;\\
tīsu vidhāsu avikampamāno, ‘samo visesī’ ti na tassa hoti. %\hfill\textcolor{gray}{\footnotesize 8}

\subsection\*{\textbf{850} {\footnotesize 〔PTS 843〕}}

\textbf{「这婆罗门为何会说『真实』,或与谁争论『虚妄』?\\}
\textbf{「若于其中没有是或非,他能与谁进行论议?}

‘Saccan’ ti so brāhmaṇo kiṃ vadeyya, ‘musā’ ti vā so vivadetha kena;\\
yasmiṃ samaṃ visamaṃ vā pi natthi, sa kena vādaṃ paṭisaṃyujeyya. %\hfill\textcolor{gray}{\footnotesize 9}

\subsection\*{\textbf{851} {\footnotesize 〔PTS 844〕}}

\textbf{「舍弃了住处,无居所而行,牟尼不在村中建立亲密,\\}
\textbf{「空乏爱欲,无有预设,他不会与人相辩论。}

Okaṃ pahāya aniketasārī, gāme akubbaṃ muni santhavāni;\\
kāmehi ritto apurakkharāno, kathaṃ na viggayha janena kayirā. %\hfill\textcolor{gray}{\footnotesize 10}

\subsection\*{\textbf{852} {\footnotesize 〔PTS 845〕}}

\textbf{「独处者在世间所据以游行者,龙象不执取之而宣说,\\}
\textbf{「好比水生的带刺莲花,不染于水与淤泥,\\}
\textbf{「如是寂静、无求的牟尼,不染于爱欲与世间。}

Yehi vivitto vicareyya loke, na tāni uggayha vadeyya nāgo;\\
jalambujaṃ kaṇḍaka-vārijaṃ yathā, jalena paṅkena c’anūpalittaṃ;\\
evaṃ munī santivādo agiddho, kāme ca loke ca anūpalitto. %\hfill\textcolor{gray}{\footnotesize 11}

\subsection\*{\textbf{853} {\footnotesize 〔PTS 846〕}}

\textbf{「通达诸明者不以见、不以觉生起慢,因为他不关切,\\}
\textbf{「也不被业、不被所闻引领,他不陷入住著。}

Na vedagū diṭṭhiyāyako na mutiyā, sa mānam eti na hi tammayo so;\\
na kammunā no pi sutena neyyo, anūpanīto sa nivesanesu. %\hfill\textcolor{gray}{\footnotesize 12}

\begin{itemize}\item 案,\textbf{diṭṭhiyāyako} 从 PTS 本作 \textit{diṭṭhiyā}。\end{itemize}

\subsection\*{\textbf{854} {\footnotesize 〔PTS 847〕}}

\textbf{「离想者没有系缚,慧解脱者没有愚痴,\\}
\textbf{「执取想与见者,他们冲突着,在世间游行。」}

Saññāvirattassa na santi ganthā, paññāvimuttassa na santi mohā;\\
saññañ ca diṭṭhiñ ca ye aggahesuṃ, te ghaṭṭayantā vicaranti loke” ti. %\hfill\textcolor{gray}{\footnotesize 13}

\begin{enumerate}\item \textbf{离想者},即由以出离想为前分的修习而舍弃爱欲等想者,这是指俱分解脱的奢摩他行者。\textbf{慧解脱者},即由以毗婆舍那为前分的修习而解脱于一切烦恼者,这是指纯观行者。\textbf{执取想与见者,他们冲突着},执取欲想等想者多为在家人,互相起爱欲的冲突,执取见者多为出家人,互相起法的冲突。当开示终了,婆罗门和婆罗门尼出家已,证得了阿罗汉。\end{enumerate}

\begin{center}\vspace{1em}摩根提耶经第九\\Māgaṇḍiyasuttaṃ navamaṃ.\end{center}