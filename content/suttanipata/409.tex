\section{摩根提耶经}

\begin{center}Māgaṇḍiya Sutta\end{center}\vspace{1em}

\begin{enumerate}\item 缘起为何?一时,世尊住舍卫国,晨朝以佛眼观察世间,见到俱卢国斑御的村民,名叫摩根提耶的婆罗门及其妻子证阿罗汉的近依后,即从舍卫国去到那里,在离斑御不远的某处密林坐下,放出金光。摩根提耶也正好去到那里洗脸,看到金光后,想「这是什么」,四下观望,见到世尊后,即生悦意。据说,他的女儿也是金黄的肤色,许多刹帝利童子等向她求婚而未得。婆罗门有这样的主张「我将只给你拥有金黄肤色的沙门」。他看见世尊后,便动了心:「他和我女儿的肤色相同,我要把女儿给他。」他急忙回家,对婆罗门尼说:「夫人!夫人!我看见和女儿相同肤色的人,快装扮女儿!我们带去见他!」婆罗门尼便以香水沐浴了女儿,以衣服、鲜花等装扮好,正好到了世尊行乞的时间。
\item 于是,世尊入斑御乞食。他们也带了女儿,去到世尊的坐处。在那里未见到世尊,婆罗门尼就四处观望,看到世尊坐处的草垫。而由决意之力,诸佛的坐处及足迹皆不紊乱。她便对婆罗门说:「婆罗门!这是他的草垫?」「唯!夫人!」「那么,婆罗门!我们此行将不会成功了。」「夫人!为什么?」「看!婆罗门!这草垫不紊乱,这不是受用爱欲者所用的。」婆罗门便说:「夫人!当寻求吉祥时,莫说些不祥的!」婆罗门尼又四处观望,看到世尊的足迹,便对婆罗门说:「婆罗门!这是他的足迹?」「唯!夫人!」「看这足迹!婆罗门!这有情并非贪求爱欲者。」「夫人!你怎么知道?」为显示自己的智力,她便说颂:\begin{quoting}染著者的足迹曲起,嗔恚者的足迹尾长,\\愚痴者的足迹急压,去蔽者的足迹如斯。(清净道论·说取业处品第 88 段)\end{quoting}
\item 然而,他们的这谈话被打断了,此时,世尊食事已毕,回到密林。婆罗门尼看到世尊饰以最上之相、散发一寻光辉的形相后,便对婆罗门说:「婆罗门!你看见的就是他吗?」「唯!夫人!」「此行肯定不会成功了,像这样的人受用爱欲,无有是处。」他们正如是对话时,世尊便在草垫上坐下了。于是,婆罗门左手拉着女儿,右手拿着水罐,靠近世尊后说「出家先生!你和这女孩都是金黄的肤色,她对你是合适的,我把她给你作妻子去抚养」,想要给他,便去到世尊跟前而立。世尊好似未与婆罗门交谈,而是与另一人一起谈话般,说了此颂。\end{enumerate}

\subsection\*{\textbf{842} {\footnotesize 〔PTS 835〕}}

\textbf{「见到渴爱、不喜、贪染后,尚于淫欲毫无欲望,\\}
\textbf{「何况这充满屎尿者?我甚至不愿用脚去触碰她。」}

“Disvāna Taṇhaṃ Aratiṃ Ragañ ca, nāhosi chando api methunasmiṃ;\\
kim ev’idaṃ muttakarīsapuṇṇaṃ, pādā pi naṃ samphusituṃ na icche”. %\hfill\textcolor{gray}{\footnotesize 1}

\begin{enumerate}\item 其义为:在牧羊人的尼拘律树下,\textbf{见到}化作种种形相而来、令人垂涎的魔罗之女\textbf{渴爱、不喜、贪染后},我\textbf{尚于淫欲毫无}些许\textbf{欲望},\textbf{何况}见到\textbf{这充满屎尿}的女孩的形相?于一切处\textbf{我甚至不愿用脚去触碰她},如何能与之共处?\end{enumerate}

\subsection\*{\textbf{843} {\footnotesize 〔PTS 836〕}}

\textbf{「若你不希望这样的宝,为众王所愿求的女人,\\}
\textbf{「请宣说是怎样的见、戒禁、活命与有之投生!」}

“Etādisañ ce ratanaṃ na icchasi, nāriṃ narindehi bahūhi patthitaṃ;\\
diṭṭhigataṃ sīlavataṃ nu jīvitaṃ, bhavūpapattiñ ca vadesi kīdisaṃ”. %\hfill\textcolor{gray}{\footnotesize 2}

\begin{enumerate}\item 随后,摩根提耶为问「所谓出家人,即舍弃了人间的爱欲,为了天界的爱欲而出家,而他连天界的爱欲和这女宝也不希望,那他的见是什么」,说了第二颂。这里,\textbf{这样的宝}是就天界的女宝而说,\textbf{女人}是就自己的女儿。\textbf{请宣说是怎样的有之投生},即请宣说自己的有之投生是怎样的。\end{enumerate}

\subsection\*{\textbf{844} {\footnotesize 〔PTS 837〕}}

\textbf{「于诸法抉择已,摩根提耶!」世尊说,「不被『我宣说此』摄取,\\}
\textbf{「且看见诸见而无取,当简别时,我得见内在寂静。」}

“‘Idaṃ vadāmī’ ti na tassa hoti, \textit{(Māgaṇḍiyā ti Bhagavā)} dhammesu niccheyya samuggahītaṃ;\\
passañ ca diṭṭhīsu anuggahāya, ajjhattasantiṃ pacinaṃ adassaṃ”. %\hfill\textcolor{gray}{\footnotesize 3}

\begin{enumerate}\item 此后的二颂以答问转起,连结自明。其中第一颂的略义为:摩根提耶!\textbf{于}六十二见的\textbf{诸法抉择已},我\textbf{不}如「唯此真实,余皆虚妄」等\textbf{被「我宣说此」摄取}。什么原因?因为我\textbf{看见诸见}中的过患,不取任何见,\textbf{当简别时,我得见}因贪等的寂静而被称为\textbf{内在寂静}的涅槃。\end{enumerate}

\subsection\*{\textbf{845} {\footnotesize 〔PTS 838〕}}

\textbf{「这些裁断、遍计,」摩根提耶说,「牟尼!无取于彼等,请说\\}
\textbf{「这『内在寂静』之义!智者们如何宣说?」}

“Vinicchayā yāni pakappitāni, \textit{(iti Māgaṇḍiyo)} te ve munī brūsi anuggahāya;\\
‘ajjhattasantī’ ti yam etam atthaṃ, kathaṃ nu dhīrehi paveditaṃ taṃ”. %\hfill\textcolor{gray}{\footnotesize 4}

\begin{enumerate}\item 第二颂的略义为:\textbf{这些}见,由为彼彼有情抉择而把握故,被称为\textbf{裁断},且由自己的缘,以行作之相等方法被称为\textbf{遍计}。你,\textbf{牟尼}!不取\textbf{彼等}成见之法,\textbf{请说这「内在寂静」之义},请告知我,\textbf{智者们如何宣说},智者们如何阐明该词?\end{enumerate}

\subsection\*{\textbf{846} {\footnotesize 〔PTS 839〕}}

\textbf{「不以见、闻、智,摩根提耶!」世尊说,「也不以戒禁而说清净,\\}
\textbf{「不以无见、无闻、无智、无戒、无禁,也不以此,\\}
\textbf{「对这些放弃而无取,寂静者无依止,不会渴望有。」}

“Na diṭṭhiyā na sutiyā na ñāṇena, \textit{(Māgaṇḍiyā ti Bhagavā)} sīlabbatenāpi na suddhim āha;\\
adiṭṭhiyā assutiyā añāṇā, asīlatā abbatā no pi tena;\\
ete ca nissajja anuggahāya, santo anissāya bhavaṃ na jappe”. %\hfill\textcolor{gray}{\footnotesize 5}

\begin{enumerate}\item 于是,世尊为显示智者们以之阐明该词的方法及其对立,说了此颂。这里,以\textbf{不以见}等,拒斥了见、闻、八等至之智、外道之戒禁。在「\textbf{而说清净}」中所说的「说」字在一切处与「不」结合,作了人称的转换,当知义为「我不以见而说清净」等。且这里,\textbf{不以无见而说},即我不以除十事正见而说。同样,\textbf{无闻},即除九分教,\textbf{无智},即除业自性及谛随顺之智,\textbf{无戒},即除别解脱律仪,\textbf{无禁},即除头陀支之禁。\textbf{也不以此},当知义为「我不仅以其中之一的见等而说」等。
\item \textbf{对这些放弃而无取},即以除去前面「见」等类的黑分法而放弃,以不参与得至后面「无见」等类的白分法而无取。\textbf{寂静者无依止,不会渴望有},即以此行道止息了贪等的寂静者,不依止眼等中的任何法,乃至连一有也不羡慕、不愿求,此即「内在寂静」之意。\end{enumerate}

\subsection\*{\textbf{847} {\footnotesize 〔PTS 840〕}}

\textbf{「若不以见、闻、智,」摩根提耶说,「也不以戒禁而说清净,\\}
\textbf{「不以无见、无闻、无智、无戒、无禁,也不以此,\\}
\textbf{「我认为实是愚痴之法,有些人以见认可清净。」}

“No ce kira diṭṭhiyā na sutiyā na ñāṇena, \textit{(iti Māgaṇḍiyo)} sīlabbatenāpi na suddhim āha;\\
adiṭṭhiyā assutiyā añāṇā, asīlatā abbatā no pi tena;\\
maññām’ahaṃ momuham eva dhammaṃ, diṭṭhiyā eke paccenti suddhiṃ”. %\hfill\textcolor{gray}{\footnotesize 6}

\begin{enumerate}\item 如是说已,摩根提耶因未了解言语之义,说了此颂。这里,「见」等仍如前述,但他是就黑分而说两处。而「说」字与「若不」结合,其义视如「若不说、若不论」。\textbf{愚痴},即极度愚昧,或迷惑。\textbf{认可},即了知。\end{enumerate}

\subsection\*{\textbf{848} {\footnotesize 〔PTS 841〕}}

\textbf{「依于见而追问,摩根提耶!」世尊说,「在摄取中陷入困惑,\\}
\textbf{「不能从中见到些许之想,所以,你视之为愚痴。}

“Diṭṭhañ ca nissāya anupucchamāno, \textit{(Māgaṇḍiyā ti Bhagavā)} samuggahītesu pamoham āgā;\\
ito ca nāddakkhi aṇum pi saññaṃ, tasmā tuvaṃ momuhato dahāsi. %\hfill\textcolor{gray}{\footnotesize 7}

\begin{enumerate}\item 于是,世尊为拒斥其依于见而问,说了此颂。其义为:你,摩根提耶!\textbf{依于见}再再而问,你被这些见摄取,正是\textbf{在}这些\textbf{摄取中陷入困惑},\textbf{不能从}我所说的内在寂静、从行道,或从法的开示\textbf{中见到些许}相应\textbf{之想},因此原因,\textbf{你视}这法\textbf{为愚痴}。\end{enumerate}

\subsection\*{\textbf{849} {\footnotesize 〔PTS 842〕}}

\textbf{「若认为是同等、殊胜或低下,他因此而争论,\\}
\textbf{「当于三者无动摇,他便没有『同等、殊胜』。}

Samo visesī uda vā nihīno, yo maññati so vivadetha tena;\\
tīsu vidhāsu avikampamāno, ‘samo visesī’ ti na tassa hoti. %\hfill\textcolor{gray}{\footnotesize 8}

\begin{enumerate}\item 如是,显示了摩根提耶在摄取中因困惑的争论之过,现在,为显示在此及其余诸法中自己已离困惑的离争论性,说了此颂。其义为:\textbf{若}如是以三种慢或见\textbf{认为},\textbf{他因此}慢、此见或此补特伽罗\textbf{而争论}。但若如我等,于此\textbf{三者无动摇,他便没有「同等、殊胜」},文本省略了「且没有低下」。\end{enumerate}

\subsection\*{\textbf{850} {\footnotesize 〔PTS 843〕}}

\textbf{「这婆罗门会说什么『真实』,或以何争论『虚妄』?\\}
\textbf{「若在其中没有相等或不等,他能与谁进行论议?}

‘Saccan’ ti so brāhmaṇo kiṃ vadeyya, ‘musā’ ti vā so vivadetha kena;\\
yasmiṃ samaṃ visamaṃ vā pi natthi, sa kena vādaṃ paṭisaṃyujeyya. %\hfill\textcolor{gray}{\footnotesize 9}

\begin{enumerate}\item 更有「这婆罗门……」。其义为:\textbf{这}如此舍弃了慢与见者,如我等者,以排除了恶等方法为\textbf{婆罗门},\textbf{会说什么}「唯此\textbf{真实}」,会说什么依处?或者以何原因说「我的真实,你的\textbf{虚妄}」,\textbf{以何}慢、见、补特伽罗\textbf{争论}?\textbf{若在}如我等的漏尽者\textbf{中没有}以「我是相同」转起的\textbf{相等},\textbf{或}以另二者之相转起的\textbf{不等},\textbf{他能与}俱慢等中的\textbf{谁进行论议}、反驳?\end{enumerate}

\subsection\*{\textbf{851} {\footnotesize 〔PTS 844〕}}

\textbf{「舍弃了住处,无居所而流动,牟尼不在村中建立亲密,\\}
\textbf{「空乏爱欲,不作预设,他不会与人争吵辩论。}

Okaṃ pahāya aniketasārī, gāme akubbaṃ muni santhavāni;\\
kāmehi ritto apurakkharāno, kathaṃ na viggayha janena kayirā. %\hfill\textcolor{gray}{\footnotesize 10}

\begin{enumerate}\item 这样的人不是一向「舍弃了住处……」吗?这里,\textbf{舍弃了住处},即于此,以舍弃欲贪抛弃了以色为依处等的识的场所。\textbf{无居所而流动},即不以渴爱流动于色相的居所等。\textbf{牟尼不在村中建立亲密},即不在村中建立与俗家的亲密。\textbf{空乏爱欲},即于爱欲,以无欲贪之相,与一切爱欲分离。\textbf{不作预设},即不转生为将来的自体。\textbf{他不会与人争吵辩论},即他不会与人谈论诤论。\end{enumerate}

\subsection\*{\textbf{852} {\footnotesize 〔PTS 845〕}}

\textbf{「独处者所据以在世间游行者,龙象不执取之而说,\\}
\textbf{「好比水生的带刺荷花,不染于水与淤泥,\\}
\textbf{「如是寂静、无求的牟尼,不染于爱欲与世间。}

Yehi vivitto vicareyya loke, na tāni uggayha vadeyya nāgo;\\
jalambujaṃ kaṇḍakavārijaṃ yathā, jalena paṅkena c’anūpalittaṃ;\\
evaṃ munī santivādo agiddho, kāme ca loke ca anūpalitto. %\hfill\textcolor{gray}{\footnotesize 11}

\begin{enumerate}\item 他这样的「独处者……」。这里,\textbf{所据以},即所据以之见。\textbf{独处者游行},即空乏者游行。\textbf{龙象不执取之而说},即以「不造作罪过\footnote{即\textbf{会堂经}第 528 颂。}」等方法的龙象,不执取这些见而说。\textbf{水生},即在水中而生的\textbf{带刺}的茎的\textbf{荷花},即是说莲花。\textbf{好比不染于水与淤泥},即好比这莲花不染于水与淤泥。\textbf{如是寂静、无求的牟尼},即如是内在寂静的牟尼,以无有贪求而无求。\textbf{不染于爱欲与世间},即不以两种涂抹染于两种爱欲与苦处等的世间。\end{enumerate}

\subsection\*{\textbf{853} {\footnotesize 〔PTS 846〕}}

\textbf{「通达诸明者非见行者,他不以觉生起慢,因为他不参与,\\}
\textbf{「不被业、也不被所闻引领,他不陷入住著。}

Na vedagū diṭṭhiyāyako na mutiyā, sa mānam eti na hi tammayo so;\\
na kammunā no pi sutena neyyo, anūpanīto sa nivesanesu. %\hfill\textcolor{gray}{\footnotesize 12}

\begin{enumerate}\item 更有「通达诸明者……」。这里,\textbf{通达诸明者非见行者},即如我等通达四道者,不是见行者,或不以见而行,或不认可其为实质。这里的语义为:以「行」为行者,以具格之见而行为见行者,以业格之义的属格而行,亦为见行者。\textbf{他不以觉生起慢},他也不以所觉的色等类的觉生起慢。\textbf{因为他不参与},以爱、见参与者即志在于彼,而他并不如此。\textbf{不被业、也不被所闻引领},即他不被福行等的业,或所闻清净等的所闻引领。\textbf{他不陷入住著},即他由舍弃两种牵涉故,于一切爱、见的住著不陷入。\end{enumerate}

\subsection\*{\textbf{854} {\footnotesize 〔PTS 847〕}}

\textbf{「离想者没有系缚,慧解脱者没有愚痴,\\}
\textbf{「若执取想与见,他们便冲突着在世间游行。」}

Saññāvirattassa na santi ganthā, paññāvimuttassa na santi mohā;\\
saññañ ca diṭṭhiñ ca ye aggahesuṃ, te ghaṭṭayantā vicaranti loke” ti. %\hfill\textcolor{gray}{\footnotesize 13}

\begin{enumerate}\item 且像这样的「离想者……」。这里,\textbf{离想者},即以出离想为前导的修习而舍弃爱欲等想者,以此词指俱分解脱的奢摩他行者。\textbf{慧解脱者},即以毗婆舍那为前导的修习而解脱一切烦恼者,以此词指干观行者。\textbf{若执取想与见,他们便冲突着在世间游行},执取欲想等的想者多为在家人,互相起爱欲的冲突而行,而执取见者多为出家人,互相起法的冲突而行。此中其余未述者,当依已述者可知。当开示终了,婆罗门和婆罗门尼出家后,证得了阿罗汉。\end{enumerate}

\begin{center}\vspace{1em}摩根提耶经第九\\Māgaṇḍiyasuttaṃ navamaṃ.\end{center}