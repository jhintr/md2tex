\section{小阵经}

\begin{center}Cūḷabyūha Sutta\end{center}\vspace{1em}

\begin{enumerate}\item 缘起为何?仍在此大集会中,有些天人生起「所有这些成见者都说『我们是正确的』,那么这些正确是只建立在自己的见上,还是也把握了其它的见」之心,为阐明此义,仍以先前的方法,让相佛问自己问题后而说。\end{enumerate}

\subsection\*{\textbf{885} {\footnotesize 〔PTS 878〕}}

\textbf{「住于各自的见,善巧者们争执而说种种:\\}
\textbf{「『若如是知,他便明了法,若批评此,他即非整全』。}

“Sakaṃ sakaṃ diṭṭhi paribbasānā, viggayha nānā kusalā vadanti;\\
‘yo evaṃ jānāti sa vedi dhammaṃ, idaṃ paṭikkosam akevalī so’. %\hfill\textcolor{gray}{\footnotesize 1}

\begin{enumerate}\item 这里,开头两颂是问颂。其中,\textbf{住于各自的见},即居于自身的见。\textbf{善巧者们争执而说种种},即强力地把握见后,于此认定「我们是善巧者」,宣说种种,而非一辞。\textbf{若如是知,他便明了法,若批评此,他即非整全},即他们说:就此见而言,若他如是知,他便明了法,但批评此,即是低劣。\end{enumerate}

\subsection\*{\textbf{886} {\footnotesize 〔PTS 879〕}}

\textbf{「他们如是争执而争论,并说『对方是愚人、不善巧』,\\}
\textbf{「此中何者是真实之论?因为他们全都自称是善巧者。」}

Evam pi viggayha vivādayanti, ‘bālo paro akkusalo’ ti cāhu;\\
sacco nu vādo katamo imesaṃ, sabbe va h’īme kusalā vadānā”. %\hfill\textcolor{gray}{\footnotesize 2}

\begin{enumerate}\item \textbf{愚人},即低劣。\textbf{不善巧},即不明了。\end{enumerate}

\subsection\*{\textbf{887} {\footnotesize 〔PTS 880〕}}

\textbf{「若不认可对方的法,便是愚人、野蛮、劣慧者,\\}
\textbf{「那么全部都是愚人、极劣慧者,因为他们全部都住于见。}

“Parassa ce dhammam anānujānaṃ, bālo mako hoti nihīnapañño;\\
sabbe va bālā sunihīnapaññā, sabbe v’ime diṭṭhi paribbasānā. %\hfill\textcolor{gray}{\footnotesize 3}

\begin{enumerate}\item 现在是三首答颂。它们以后半对阵前半所说之义,以此阵之故,并由较后经量小之故,此经得名「小阵」。这里,\textbf{对方的法},即对方的见。\textbf{全部都是愚人},意即当如是时,他们全都是愚人。什么原因?\textbf{因为他们全部都住于见}。\end{enumerate}

\subsection\*{\textbf{888} {\footnotesize 〔PTS 881〕}}

\textbf{「然而,如果以自己的见便洁净、净慧、善巧、具慧,\\}
\textbf{「那么其中无人是劣慧者,因为他们的见都同样完整。}

Sandiṭṭhiyā c’eva na vīvadātā, saṃsuddhapaññā kusalā mutīmā;\\
na tesaṃ koci parihīnapañño, diṭṭhī hi tesam pi tathā samattā. %\hfill\textcolor{gray}{\footnotesize 4}

\begin{enumerate}\item \textbf{然而,如果以自己的见便……具慧},即以自身的见便不洁净,不洁白、仍有杂染,如果他们净慧、善巧、具慧。或者,文本也作 sandiṭṭhiyā ce pana vīvadātā,其义为:然而,如果以自己的见便洁白、净慧、善巧、具慧\footnote{义注对上半颂给出两种读法,第一种颇费解,所以这里的偈颂采取第二种读法来译,Norman 与菩提比丘的英译也如此。}。\textbf{其中无人},即当如是时,其中连一个劣慧者也没有。什么原因?\textbf{因为他们的见都同样完整}\footnote{完整 \textit{samatta}:也可译作「被采纳」,二英译本即作 adopted,下第 896 颂同。},和另外的人一样。\end{enumerate}

\subsection\*{\textbf{889} {\footnotesize 〔PTS 882〕}}

\textbf{「愚人们互相对彼此所说的,我不说『这是如此』,\\}
\textbf{「他们认为各自的见是真实,所以认定对方是愚人。」}

Na vāham ‘etaṃ tathiyan’ ti brūmi, yam āhu bālā mithu aññamaññaṃ;\\
sakaṃ sakaṃ diṭṭhim akaṃsu saccaṃ, tasmā hi ‘bālo’ ti paraṃ dahanti”. %\hfill\textcolor{gray}{\footnotesize 5}

\begin{enumerate}\item 此颂的略义为:那些俩俩\textbf{互相对彼此}说是「愚人」的,\textbf{我不说这是如此}、如实。什么原因?因为他们全都\textbf{认为各自的见}「唯此\textbf{真实},余皆虚妄」。且以此因,\textbf{认定对方是愚人}。且此中的文本作 tathiyan, kathivan\footnote{PTS 本作 tathivan。} 两读。\end{enumerate}

\subsection\*{\textbf{890} {\footnotesize 〔PTS 883〕}}

\textbf{「有些说是『真实、如实』的,其他人却说是『虚无、虚妄』,\\}
\textbf{「他们如是争执而争论,为什么众沙门说辞不一?」}

“Yam āhu ‘saccaṃ tathiyan’ ti eke, tam āhu aññe ‘tucchaṃ musā’ ti;\\
evam pi vigayha vivādayanti, kasmā na ekaṃ samaṇā vadanti”. %\hfill\textcolor{gray}{\footnotesize 6}

\begin{enumerate}\item 在此问颂中,即\textbf{有些说是「真实、如实」的}见。\end{enumerate}

\subsection\*{\textbf{891} {\footnotesize 〔PTS 884〕}}

\textbf{「因为真实唯一,而非有二,于此了知的人不会争论,\\}
\textbf{「他们自己赞扬各种的真实,所以,众沙门说辞不一。」}

“Ekañ hi saccaṃ na dutīyam atthi, yasmiṃ pajā no vivade pajānaṃ;\\
nānā te saccāni sayaṃ thunanti, tasmā na ekaṃ samaṇā vadanti”. %\hfill\textcolor{gray}{\footnotesize 7}

\begin{enumerate}\item 在此答颂中,\textbf{真实唯一},即灭,或道。\textbf{于此了知的人不会争论},即于此真实了知的人不会争论。\end{enumerate}

\subsection\*{\textbf{892} {\footnotesize 〔PTS 885〕}}

\textbf{「那为什么要说各种的真实?论说者们都自称善巧,\\}
\textbf{「是所闻的真实种类繁多,还是他们随念寻思?」}

“Kasmā nu saccāni vadanti nānā, pavādiyāse kusalā vadānā;\\
saccāni sutāni bahūni nānā, udāhu te takkam anussaranti”. %\hfill\textcolor{gray}{\footnotesize 8}

\begin{enumerate}\item 在此问颂中,\textbf{论说者们},即论议者们。\textbf{还是他们随念寻思},即还是这些论议者只是跟随自己的寻思?\end{enumerate}

\subsection\*{\textbf{893} {\footnotesize 〔PTS 886〕}}

\textbf{「真实并非种类繁多、在世间是常,除非因想,\\}
\textbf{「于诸见遍计寻思已,他们说『真实、虚妄』之二元法。}

“Na h’eva saccāni bahūni nānā, aññatra saññāya niccāni loke;\\
takkañ ca diṭṭhīsu pakappayitvā, ‘saccaṃ musā’ ti dvayadhammam āhu. %\hfill\textcolor{gray}{\footnotesize 9}

\begin{enumerate}\item 在此答颂中,\textbf{在世间是常,除非因想},即除非仅仅因想,把握所执取者是常。\textbf{于诸见遍计寻思已},即于诸见,仅生起自己的邪思惟。而因为于诸见生起寻者,也生起见,所以在「义释」中说「生起、完全生起见」等等\footnote{即\textbf{大义释}第 121 段。}。\end{enumerate}

\subsection\*{\textbf{894} {\footnotesize 〔PTS 887〕}}

\textbf{「所见、所闻、戒禁、或所觉,依于这些便显露轻侮,\\}
\textbf{「立于抉择便作讪笑,并说『对方是愚人、不善巧』。}

Diṭṭhe sute sīlavate mute vā, ete ca nissāya vimānadassī;\\
vinicchaye ṭhatvā pahassamāno, ‘bālo paro akkusalo’ ti cāha. %\hfill\textcolor{gray}{\footnotesize 10}

\begin{enumerate}\item 现在,为显示在如是不实的繁多真实中,仅随念寻思的成见者的邪行道,说了以下几颂。这里,\textbf{所见},意即所见为清净,\textbf{所闻}等仿此。\textbf{依于这些便显露轻侮},即依于这些见,便显现出被称为(不)净相\footnote{原文作净相 \textit{suddhibhāva},「不」字据 PTS 本所引的缅甸本 B\textsuperscript{a} 补,说详菩提比丘注 1953。}的轻侮、不敬。\textbf{立于抉择便作讪笑},即如是显露轻侮者,还立于此见的抉择,生起满意、讪笑,作如是说:「对方低劣且无知。」\end{enumerate}

\subsection\*{\textbf{895} {\footnotesize 〔PTS 888〕}}

\textbf{「正因为认定对方是愚人,便以此说自己是善巧,\\}
\textbf{「他自称自己本人为善巧,轻侮别人,说着这些。}

Yen’eva ‘bālo’ ti paraṃ dahāti, tenātumānaṃ ‘kusalo’ ti cāha;\\
sayam attanā so kusalā vadāno, aññaṃ vimāneti tad eva pāva. %\hfill\textcolor{gray}{\footnotesize 11}

\begin{enumerate}\item 当如是时,「正因为……」。这里,\textbf{轻侮},即指责。\textbf{说着这些},即说着这些见的言说,或此人。\end{enumerate}

\subsection\*{\textbf{896} {\footnotesize 〔PTS 889〕}}

\textbf{「他以过误之见而完整,以慢而迷醉,自认为圆满,\\}
\textbf{「自己在心中为自己灌顶,因为他的见是那么完整。}

Atisāradiṭṭhiyā so samatto, mānena matto paripuṇṇamānī;\\
sayam eva sāmaṃ manasābhisitto, diṭṭhī hi sā tassa tathā samattā. %\hfill\textcolor{gray}{\footnotesize 12}

\begin{enumerate}\item 此颂之义为:\textbf{他}如是\textbf{以}此违越相的\textbf{过误之见而完整}、圆满、膨胀,并\textbf{以}此见之\textbf{慢而迷醉}「我圆满、整全」,如是\textbf{自认为圆满},\textbf{自己在心中为自己灌顶}「我是智者」。什么原因?\textbf{因为他的见是那么完整}。\end{enumerate}

\subsection\*{\textbf{897} {\footnotesize 〔PTS 890〕}}

\textbf{「若以对方的言语便低劣,则自身也一并是劣慧者,\\}
\textbf{「若他自己通达诸明,是智者,则沙门众中无人是愚人。}

Parassa ce hi vacasā nihīno, tumo sahā hoti nihīnapañño;\\
atha ce sayaṃ vedagū hoti dhīro, na koci bālo samaṇesu atthi. %\hfill\textcolor{gray}{\footnotesize 13}

\begin{enumerate}\item 此颂的连结及语义为:还有什么?凡是立于抉择作讪笑,并说「对方是愚人、不善巧」者,\textbf{若以}那\textbf{对方的言语},他便被其称为\textbf{低劣},\textbf{则自身也一并是劣慧者},他也与之一并是劣慧者,因为他也说其为「愚人」。于是,他的言语便非标准,但是,\textbf{若他自己}本身\textbf{通达诸明}且\textbf{是智者},当如是时,\textbf{则沙门众中无人是愚人},因为他们也全都以自己的希望是智者。\end{enumerate}

\subsection\*{\textbf{898} {\footnotesize 〔PTS 891〕}}

\textbf{「若宣说除此之外的法,他们便有违清净、非整全,\\}
\textbf{「如是众外道宣说种种,因为他们被自见之贪染所染著。}

Aññaṃ ito yābhivadanti dhammaṃ, aparaddhā suddhim akevalī te;\\
evam pi titthyā puthuso vadanti, sandiṭṭhirāgena hi te’bhirattā. %\hfill\textcolor{gray}{\footnotesize 14}

\begin{enumerate}\item 此颂的连结及语义为:当如是说「若他自己通达诸明,是智者,则沙门众中无人是愚人」时,有些人会想:为什么?此处当答:因为\textbf{若宣说除此之外的法,他们便有违清净、非整全,如是众外道宣说种种},即是说,若宣说除此之外的见,他们便有违、迷失清净之道,且不整全,如是众外道由此宣说种种。但设问:为什么他们如是说?\textbf{因为他们被自见之贪染所染著},即是说因为被自己的见之贪染所染著。\end{enumerate}

\subsection\*{\textbf{899} {\footnotesize 〔PTS 892〕}}

\textbf{「他们争论『唯于此清净』,说在其它的法中没有清净,\\}
\textbf{「如是众外道执著种种,于此努力宣扬着自己的路。}

‘Idh’eva suddhiṃ’ iti vādayanti, nāññesu dhammesu visuddhim āhu;\\
evam pi titthyā puthuso niviṭṭhā, sakāyane tattha daḷhaṃ vadānā. %\hfill\textcolor{gray}{\footnotesize 15}

\begin{enumerate}\item 而如是染著者,「他们争论……」。这里,\textbf{自己的道},即自己的道。\end{enumerate}

\subsection\*{\textbf{900} {\footnotesize 〔PTS 893〕}}

\textbf{「而努力宣扬着自己的路,他在此会认定哪个对方是愚人?\\}
\textbf{「当说对方是愚人、非清净法时,他自己会引起纠纷。}

Sakāyane vā pi daḷhaṃ vadāno, kam ettha ‘bālo’ ti paraṃ daheyya;\\
sayaṃ va so medhagam āvaheyya, paraṃ vadaṃ bālam asuddhidhammaṃ. %\hfill\textcolor{gray}{\footnotesize 16}

\begin{enumerate}\item 且如是,在这些努力宣扬者中的任何外道,\textbf{而努力宣扬着自己的路,他在此会认定哪个对方是愚人}?略言之,即于此被称为常、断,或详言之,即于非有、自在因、决定等类的自己的入处,努力宣扬「唯此真实」者,在此见中,会如法地视哪个对方是愚人?按其理解,难道不全都是智者、善行道者吗?且当如是时,\textbf{当说对方是愚人、非清净法时,他自己会引起纠纷},他在说对方「这是愚人、非清净法」时,便与自己引起争辩。为什么?因为按其理解,全都是智者、善行道者。\end{enumerate}

\subsection\*{\textbf{901} {\footnotesize 〔PTS 894〕}}

\textbf{「立于抉择,以自身度量,他在世间愈加陷入争论,\\}
\textbf{「舍弃了一切抉择,人在世间便不再制造纠纷。」}

Vinicchaye ṭhatvā sayaṃ pamāya, uddhaṃsa lokasmiṃ vivādam eti;\\
hitvāna sabbāni vinicchayāni, na medhagaṃ kubbati jantu loke” ti. %\hfill\textcolor{gray}{\footnotesize 17}

\begin{enumerate}\item 如是,在一切处,\textbf{立于抉择,以自身度量,他在世间愈加陷入争论},立于见,且以自己的导师等度量,他更加陷入争论。而如是了知了抉择中的过患,以圣道\textbf{舍弃了一切抉择,人在世间便不再制造纠纷},即以阿罗汉为顶点完成了开示。当开示终了,仍与前分离经所说的一样,而有现观。\end{enumerate}

\begin{center}\vspace{1em}小阵经第十二\\Cūḷabyūhasuttaṃ dvādasamaṃ.\end{center}