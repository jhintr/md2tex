\section{小阵经}

\begin{center}Cūḷabyūha Sutta\end{center}\vspace{1em}

\begin{enumerate}\item 亦在此大集会中,有些天人生起「一切具见者都说『我们是正确的』,那么这些正确是只建立在自己的见上,还是也把握了其它的见」之心,为显明其义,以如前所述的方法,使相佛问自己问题后而说。\end{enumerate}

\begin{itemize}\item 案,据义注,此经第 885~886、890、892 等四颂是发问,其余是回答。\end{itemize}

\subsection\*{\textbf{885} {\footnotesize 〔PTS 878〕}}

\textbf{「住于各自的见而争执,善巧者们宣说种种:\\}
\textbf{「『如是知者,即了知了法,若非难此,他便非整全』。}

“Sakaṃ sakaṃ diṭṭhi paribbasānā, viggayha nānā kusalā vadanti;\\
‘yo evaṃ jānāti sa vedi dhammaṃ, idaṃ paṭikkosam akevalī so’. %\hfill\textcolor{gray}{\footnotesize 1}

\subsection\*{\textbf{886} {\footnotesize 〔PTS 879〕}}

\textbf{「他们如是争执而争论,并说『对方是愚人、不善巧』,\\}
\textbf{「此中何者是真实之论?因为他们全都宣称是善巧者。」}

Evam pi viggayha vivādayanti, ‘bālo paro akkusalo’ ti cāhu;\\
sacco nu vādo katamo imesaṃ, sabbe va h’īme kusalā vadānā”. %\hfill\textcolor{gray}{\footnotesize 2}

\subsection\*{\textbf{887} {\footnotesize 〔PTS 880〕}}

\textbf{「若不认可对方的法,便是愚人、野蛮、劣慧者,\\}
\textbf{「那么全部都是愚人、极劣慧者,因为他们全部都住于见。}

“Parassa ce dhammam anānujānaṃ, bālo mako hoti nihīnapañño;\\
sabbe va bālā sunihīnapaññā, sabbe v’ime diṭṭhi paribbasānā. %\hfill\textcolor{gray}{\footnotesize 3}

\begin{enumerate}\item 此颂以后半颂对阵前半颂所说之义,以此阵之故,并由较后经为短,此经得名「小阵」。\textbf{对方的法},即对方的见。\end{enumerate}

\subsection\*{\textbf{888} {\footnotesize 〔PTS 881〕}}

\textbf{「若以自己的见而洁净、净慧、善巧、具慧,\\}
\textbf{「那么其中无人是劣慧者,因为他们的见都同样是完整的。}

Sandiṭṭhiyā c’eva na vīvadātā, saṃsuddhapaññā kusalā mutīmā;\\
na tesaṃ koci parihīnapañño, diṭṭhī hi tesam pi tathā samattā. %\hfill\textcolor{gray}{\footnotesize 4}

\begin{itemize}\item 案,第一句按义注给出的另一读法 \textit{sandiṭṭhiyā \textbf{ce pana} vīvadātā} 来翻译,Norman 与菩提比丘的英译也是如此,也符合上面所说的对阵之法。\end{itemize}

\subsection\*{\textbf{889} {\footnotesize 〔PTS 882〕}}

\textbf{「我不说『这是如实』,如愚人们互相对彼此所说,\\}
\textbf{「他们认为各自的见是真实,所以认定对方是『愚人』。」}

Na vāham ‘etaṃ tathiyan’ ti brūmi, yam āhu bālā mithu aññamaññaṃ;\\
sakaṃ sakaṃ diṭṭhim akaṃsu saccaṃ, tasmā hi ‘bālo’ ti paraṃ dahanti”. %\hfill\textcolor{gray}{\footnotesize 5}

\subsection\*{\textbf{890} {\footnotesize 〔PTS 883〕}}

\textbf{「有些说是『真实、如实』的,其他人却说是『虚无、虚妄』,\\}
\textbf{「他们如是争执而争论,为何众沙门说辞不一?」}

“Yam āhu ‘saccaṃ tathiyan’ ti eke, tam āhu aññe ‘tucchaṃ musā’ ti;\\
evam pi vigayha vivādayanti, kasmā na ekaṃ samaṇā vadanti”. %\hfill\textcolor{gray}{\footnotesize 6}

\subsection\*{\textbf{891} {\footnotesize 〔PTS 884〕}}

\textbf{「因为真实唯一,而非有二,于此了知的人不会争论,\\}
\textbf{「他们宣扬各种自己的真实,所以,众沙门说辞不一。」}

“Ekañ hi saccaṃ na dutīyam atthi, yasmiṃ pajā no vivade pajānaṃ;\\
nānā te saccāni sayaṃ thunanti, tasmā na ekaṃ samaṇā vadanti”. %\hfill\textcolor{gray}{\footnotesize 7}

\subsection\*{\textbf{892} {\footnotesize 〔PTS 885〕}}

\textbf{「那么,为何宣扬各种真实?论说者们都自称善巧,\\}
\textbf{「是所闻的真实种类繁多,还是他们随念寻思?」}

“Kasmā nu saccāni vadanti nānā, pavādiyāse kusalā vadānā;\\
saccāni sutāni bahūni nānā, udāhu te takkam anussaranti”. %\hfill\textcolor{gray}{\footnotesize 8}

\subsection\*{\textbf{893} {\footnotesize 〔PTS 886〕}}

\textbf{「真实并非种类繁多,除非以想(执取)世间是常,\\}
\textbf{「于诸见遍计寻思已,他们说『真实、虚妄』之二元法。}

“Na h’eva saccāni bahūni nānā, aññatra saññāya niccāni loke;\\
takkañ ca diṭṭhīsu pakappayitvā, ‘saccaṃ musā’ ti dvayadhammam āhu. %\hfill\textcolor{gray}{\footnotesize 9}

\subsection\*{\textbf{894} {\footnotesize 〔PTS 887〕}}

\textbf{「所见、所闻、戒禁、或所觉,依于这些便显露轻侮,\\}
\textbf{「立于抉择便作讪笑,并说『对方是愚人、不善巧』。}

Diṭṭhe sute sīlavate mute vā, ete ca nissāya vimānadassī;\\
vinicchaye ṭhatvā pahassamāno, ‘bālo paro akkusalo’ ti cāha. %\hfill\textcolor{gray}{\footnotesize 10}

\subsection\*{\textbf{895} {\footnotesize 〔PTS 888〕}}

\textbf{「正因为认定对方是『愚人』,便以此说自己是『善巧』,\\}
\textbf{「他宣称自己本人为善巧,轻侮别人,说着这些。}

Yen’eva ‘bālo’ ti paraṃ dahāti, tenātumānaṃ ‘kusalo’ ti cāha;\\
sayam attanā so kusalā vadāno, aññaṃ vimāneti tad-eva pāva. %\hfill\textcolor{gray}{\footnotesize 11}

\subsection\*{\textbf{896} {\footnotesize 〔PTS 889〕}}

\textbf{「他以过误之见而盛满,以慢而迷醉,自认为圆满,\\}
\textbf{「自己在心中为自己灌顶,因为他的见是那么完整。}

Atisāradiṭṭhiyā so samatto, mānena matto paripuṇṇamānī;\\
sayam eva sāmaṃ manasābhisitto, diṭṭhī hi sā tassa tathā samattā. %\hfill\textcolor{gray}{\footnotesize 12}

\subsection\*{\textbf{897} {\footnotesize 〔PTS 890〕}}

\textbf{「若以对方的言语即低劣,则(对方)自身也同样是劣慧者,\\}
\textbf{「若他自己通达诸明,是智者,则沙门众中无人是愚人。}

Parassa ce hi vacasā nihīno, tumo sahā hoti nihīnapañño;\\
atha ce sayaṃ vedagū hoti dhīro, na koci bālo samaṇesu atthi. %\hfill\textcolor{gray}{\footnotesize 13}

\subsection\*{\textbf{898} {\footnotesize 〔PTS 891〕}}

\textbf{「若宣说除此之外的法,他们便有违清净、非整全,\\}
\textbf{「如是众外道宣说种种,因为他们以贪染自身的见而染著。}

Aññaṃ ito yābhivadanti dhammaṃ, aparaddhā suddhim akevalī te;\\
evam pi titthyā puthuso vadanti, sandiṭṭhirāgena hi te’bhirattā. %\hfill\textcolor{gray}{\footnotesize 14}

\subsection\*{\textbf{899} {\footnotesize 〔PTS 892〕}}

\textbf{「他们争论『唯于此清净』,说在其它的法中没有清净,\\}
\textbf{「如是众外道种种住立,于此努力宣扬着自己的道。}

‘Idh’eva suddhiṃ’ iti vādayanti, nāññesu dhammesu visuddhim āhu;\\
evam pi titthyā puthuso niviṭṭhā, sakāyane tattha daḷhaṃ vadānā. %\hfill\textcolor{gray}{\footnotesize 15}

\subsection\*{\textbf{900} {\footnotesize 〔PTS 893〕}}

\textbf{「或努力宣扬着自己的道,他在此会认定哪个对方是『愚人』?\\}
\textbf{「当说对方是愚人、非清净法时,他自己会引起纠纷。}

Sakāyane vā pi daḷhaṃ vadāno, kam ettha ‘bālo’ ti paraṃ daheyya;\\
sayaṃ va so medhagam āvaheyya, paraṃ vadaṃ bālam asuddhidhammaṃ. %\hfill\textcolor{gray}{\footnotesize 16}

\subsection\*{\textbf{901} {\footnotesize 〔PTS 894〕}}

\textbf{「立于抉择,度量着自身,他在世间愈加陷入争论,\\}
\textbf{「舍弃了一切抉择,人在世间便不再制造纠纷。」}

Vinicchaye ṭhatvā sayaṃ pamāya, uddhaṃsa lokasmiṃ vivādam eti;\\
hitvāna sabbāni vinicchayāni, na medhagaṃ kubbati jantu loke” ti. %\hfill\textcolor{gray}{\footnotesize 17}

\begin{center}\vspace{1em}小阵经第十二\\Cūḷabyūhasuttaṃ dvādasamaṃ.\end{center}