\section{至上经}

\begin{center}Padhāna Sutta\end{center}\vspace{1em}

\subsection\*{\textbf{428} {\footnotesize 〔PTS 425〕}}

\textbf{我自励于至上,朝着尼连禅河,\\}
\textbf{极努力而禅修,为证离轭安稳。}

Taṃ maṃ padhānapahitattaṃ, nadiṃ Nerañjaraṃ pati;\\
viparakkamma jhāyantaṃ, yogakkhemassa pattiyā. %\hfill\textcolor{gray}{\footnotesize 1}

\begin{enumerate}\item 缘起为何?尊者阿难以「我将向至上前行,于此我意喜乐」终结了出家经。世尊坐在香房内,便想:「我六年间愿求至上所行的难行之事,今天要对众比丘说说。」于是,便从香房出来,坐于佛座,从「我自励于至上」开始,说了此经。
\item 这里,\textbf{taṃ maṃ} 二词唯指认自我\footnote{Norman 说即 so ahaṃ 的业格形式。此颂的「我」及其修饰词均为业格,作下颂的宾语。}。\textbf{自励于至上},即向涅槃遣送心,或遍舍自体。\textbf{朝着尼连禅河},即说明目标\footnote{目标 \textit{lakkhaṇa}:通常译作「相」,说详菩提比丘注 1333。}。因为自励于至上的目标是尼连禅河,因此在这里作业格,其义为「向尼连禅河」,即是说在尼连禅河畔。\textbf{极努力},即极其勇猛。\textbf{禅修},即从事无息禅\footnote{无息禅 \textit{appāṇakajjhāna}:菩提比丘注 1334,指停止以口、鼻、耳等呼吸的禅修,见\textbf{中部}·大萨遮经。}。\textbf{为证离轭安稳},即为了证得离于四轭的安稳涅槃。\end{enumerate}

\subsection\*{\textbf{429} {\footnotesize 〔PTS 426〕}}

\textbf{不解脱者前来,说着哀怜的话:\\}
\textbf{「你面黄肌瘦,死亡在你跟前。}

Namucī karuṇaṃ vācaṃ, bhāsamāno upāgami;\\
“kiso tvam asi dubbaṇṇo, santike maraṇaṃ tava. %\hfill\textcolor{gray}{\footnotesize 2}

\begin{enumerate}\item \textbf{不解脱者},即魔罗。因为他不解脱欲从其境域出离的人天,为彼等制造障碍,所以被称为「不解脱者」。\textbf{哀怜的话},即与怜悯相应的话,\textbf{说着}这慨叹\textbf{前来}。
\item 那为什么前来?据说,大人有一天想道:「一直寻求食物来期待活命,却不能以期待活命来证得不死。」随后即行断食,因此面黄肌瘦。于是,魔罗恐惧道「还不知晓此道能不能觉悟,就作极凌厉的苦行,他终将超越我的境域,我要去说些如此这般,予以阻止」,便即前来。因此便说:\textbf{你面黄肌瘦,死亡在你跟前}。\end{enumerate}

\subsection\*{\textbf{430} {\footnotesize 〔PTS 427〕}}

\textbf{「一千分去死,一分是你的活命,\\}
\textbf{「活着!先生!活命更好,活着你将造作福德。}

Sahassabhāgo maraṇassa, ekaṃso tava jīvitaṃ;\\
jīva bho jīvitaṃ seyyo, jīvaṃ puññāni kāhasi. %\hfill\textcolor{gray}{\footnotesize 3}

\begin{enumerate}\item 且如是说已,然后,为宣告此死亡在跟前之状,说了「\textbf{一千分去死,一分是你的活命}」。其义为:存在一千分,即一千分。这是什么?省略了文本「死亡之缘」。这即是说:这无息禅等的一千分是你的死亡之缘,相较于此,对你仅有一分能活命,死亡如是在你跟前。
\item 如是宣告了死亡在跟前之状,然后,为振奋其活命,说了「\textbf{活着!先生!活命更好}」。若问如何更好?\textbf{活着你将造作福德}。\end{enumerate}

\subsection\*{\textbf{431} {\footnotesize 〔PTS 428〕}}

\textbf{「且你行梵行,并供奉火供,\\}
\textbf{「能积累广大的福德,为何追求至上?}

Carato ca te brahmacariyaṃ, aggihuttañ ca jūhato;\\
pahūtaṃ cīyate puññaṃ, kiṃ padhānena kāhasi. %\hfill\textcolor{gray}{\footnotesize 4}

\begin{enumerate}\item 然后,为显示自己所认可的福德,说了此颂。这里的\textbf{梵行}是就时不时离淫欲而说的,即众苦行者所行者。其余于此自明。\end{enumerate}

\subsection\*{\textbf{432} {\footnotesize 〔PTS 429〕}}

\textbf{「通往至上之路难行、难为、难以征服。」\\}
\textbf{魔罗说着这些偈颂,便站在佛陀跟前。}

Duggo maggo padhānāya, dukkaro durabhisambhavo”;\\
imā gāthā bhaṇaṃ Māro, aṭṭhā Buddhassa santike. %\hfill\textcolor{gray}{\footnotesize 5}

\begin{enumerate}\item 为令无欲于至上,说了这半颂。这里,当如是知晓其义:由受持无息禅等故而苦于前行,即\textbf{难行},由以受苦之身心而为故,即\textbf{难为},由不能以像这样在死亡跟前证得,即\textbf{难以征服}。
\item 此后,结集者们说了这半颂:\textbf{魔罗说着这些偈颂,便站在佛陀跟前}。也有人说整颂都是。但我们认为世尊为说明自己,如他人一般,说了此中所有种种。\end{enumerate}

\subsection\*{\textbf{433} {\footnotesize 〔PTS 430〕}}

\textbf{对这如是论说的魔罗,世尊说到:\\}
\textbf{「放逸的眷属!恶者!为这义利来到此处。}

Taṃ tathāvādinaṃ Māraṃ, Bhagavā etad abravi;\\
“pamattabandhu pāpima, yen’atthena idhāgato. %\hfill\textcolor{gray}{\footnotesize 6}

\begin{enumerate}\item 在第六颂中,\textbf{为这义利},此中的意趣即:你,恶者!为他人制造障碍及为自身的义利而来。其余自明。\end{enumerate}

\subsection\*{\textbf{434} {\footnotesize 〔PTS 431〕}}

\textbf{「对我而言,福德没有丝毫义利,\\}
\textbf{「魔罗应对那些福德对其有义利的人去说。}

Aṇumatto pi puññena, attho mayhaṃ na vijjati;\\
yesañ ca attho puññena, te Māro vattum arahati. %\hfill\textcolor{gray}{\footnotesize 7}

\begin{enumerate}\item 为反驳「活着你将造作福德」之语,说了此颂。这里的\textbf{福德}是就魔罗所说的趣向流转的福德而说的。其余自明。\end{enumerate}

\subsection\*{\textbf{435} {\footnotesize 〔PTS 432〕}}

\textbf{「有信,同样有精进,我还有慧,\\}
\textbf{「对如是自励的我,你为何追问活着?}

Atthi saddhā tathā viriyaṃ, paññā ca mama vijjati;\\
evaṃ maṃ pahitattam pi, kiṃ jīvam anupucchasi. %\hfill\textcolor{gray}{\footnotesize 8}

\begin{enumerate}\item 现在,就「一分是你的活命」之语,为威吓魔罗,说了此颂。这里的意趣为:咄!魔罗!若人不信无上寂静高贵的境界,或虽信但懈怠,或虽信、发起精进但恶慧,你可藉追问他活命而闪耀,但我于无上寂静高贵的境界\textbf{有}决定之\textbf{信},\textbf{同样有}被称为身心不弛懈之勇猛的\textbf{精进},\textbf{我还有}金刚一样的\textbf{慧},\textbf{对如是自励}、最上意乐\textbf{的我},\textbf{你为何追问活着},你为什么问活命?
\item 以「我还有慧」中的「还 \textit{ca}」字,而有念与定。当有如是时,圆满涅槃所需具足的五根,我不缺其中任一根,对如是自励的我,你为何追问活着?难道不是说「不如生一日,励力行精进、具慧修禅定、得见生灭法\footnote{即\textbf{法句}·千品第 111~113 颂。}」吗?\end{enumerate}

\subsection\*{\textbf{436} {\footnotesize 〔PTS 433〕}}

\textbf{「这风都能使河水之流干涸,\\}
\textbf{「为何不能枯竭自励之我的血液?}

Nadīnam api sotāni, ayaṃ vāto visosaye;\\
kiñ ca me pahitattassa, lohitaṃ n’upasussaye. %\hfill\textcolor{gray}{\footnotesize 9}

\begin{enumerate}\item 如是威吓了魔罗,为显示身心所转起者,说了以下三颂。其义自明,而其意趣之解释为:\textbf{这}在我身体内,由无息禅的精进之力所等起的\textbf{风}转起,\textbf{都能使}世间恒河、阎摩那等\textbf{河水之流干涸},\textbf{为何不能枯竭}如是\textbf{自励之我的}四升之量的\textbf{血液}\footnote{据菩提比丘注 1339,体重 70 公斤的常人约有 5 公升的血液,则这里的一升 \textit{nāḷi} 约合 1.25 公升。}?\end{enumerate}

\subsection\*{\textbf{437} {\footnotesize 〔PTS 434〕}}

\textbf{「血液枯竭时,胆汁和痰也枯竭,\\}
\textbf{「肌肉消尽时,心更加净喜,\\}
\textbf{「我的念、慧、定更加住立。}

Lohite sussamānamhi, pittaṃ semhañ ca sussati;\\
maṃsesu khīyamānesu, bhiyyo cittaṃ pasīdati;\\
bhiyyo sati ca paññā ca, samādhi mama tiṭṭhati. %\hfill\textcolor{gray}{\footnotesize 10}

\begin{enumerate}\item 且不独只有我的血液枯竭,而是当这\textbf{血液枯竭时},停滞或流动等类在体内循行的\textbf{胆汁和}覆蔽饮食等的四升之量的\textbf{痰},甚至等量的尿与食素\textbf{也枯竭}。且当它们枯竭时,肌肉也消尽,当我的\textbf{肌肉}如是渐渐\textbf{消尽时,心更加净喜},它不因此而消沉。你不知晓这样的心,却只看到身体,便说「你面黄肌瘦,死亡在你跟前」。且不独我的心净喜,而且\textbf{我的念、慧、定更加住立},竟无一丝放逸、愚痴或心的散乱。\end{enumerate}

\subsection\*{\textbf{438} {\footnotesize 〔PTS 435〕}}

\textbf{「如是而住的我,达到最剧烈的受,\\}
\textbf{「心不希求爱欲,请看有情的清净!}

Tassa m’evaṃ viharato, pattass’uttamavedanaṃ;\\
kāmesu nāpekkhate cittaṃ, passa sattassa suddhataṃ. %\hfill\textcolor{gray}{\footnotesize 11}

\begin{enumerate}\item \textbf{如是而住的我},举凡沙门、婆罗门所能感受到的猛烈的受,无论过去、未来或现时,\textbf{达到}作为彼等例证的\textbf{最剧烈的受}。好比某些人的心在遇到苦时希求乐,冷时希求热,热时希求冷,饥时希求食,渴时希求水,如是,\textbf{心不希求}种种五欲中的任一\textbf{爱欲}。我的心不以「哎!我吃完美食,就躺在舒适的卧处」这样的行相生起,你,魔罗!\textbf{请看有情的清净}!\end{enumerate}

\subsection\*{\textbf{439} {\footnotesize 〔PTS 436〕}}

\textbf{「爱欲是你的第一军,第二叫不乐,\\}
\textbf{「你的第三是饥渴,第四叫作渴爱,}

Kāmā te paṭhamā senā, dutiyā arati vuccati;\\
tatiyā khuppipāsā te, catutthī taṇhā pavuccati. %\hfill\textcolor{gray}{\footnotesize 12}

\begin{enumerate}\item 如是显示了自身的清净,对以「我要遮止他」而来的魔罗,为破碎其心愿,先宣告了魔军,再显示不可被其战胜之状,说了「爱欲是你的第一军」为首的六颂。
\item 这里,因为最开始,对于诸物欲的烦恼\textbf{爱欲}迷惑作为在家的有情,在征服彼等后,对已达出家相者,于边鄙的坐卧处,或于某某增上善法生起\textbf{不乐}。如说:\begin{quoting}朋友!乐于出家,实在难为。(相应部第 38:16 经)\end{quoting}随后,由依赖他人活命故,\textbf{饥渴}逼迫,而对受彼逼迫者,遍求之\textbf{渴爱}疲乏其心。\end{enumerate}

\subsection\*{\textbf{440} {\footnotesize 〔PTS 437〕}}

\textbf{「你的第五是昏沉睡眠,第六叫恐怖,\\}
\textbf{「你的第七是疑,覆藏、顽固是你的第八,}

Pañcamaṃ thinamiddhaṃ te, chaṭṭhā bhīrū pavuccati;\\
sattamī vicikicchā te, makkho thambho te aṭṭhamo. %\hfill\textcolor{gray}{\footnotesize 13}

\begin{enumerate}\item 然后,\textbf{昏沉睡眠}步入那些心已疲乏者。随后,对未证得殊胜、住在难以征服的林野与偏远森林的坐卧处者,被称为恐惧的\textbf{恐怖}产生。对这些惧怕、焦虑者,对长期未尝远离之味的住者,于行道生起\textbf{疑}:也许这不是道?对去除彼而住者,因证得些许的殊胜,慢、\textbf{覆藏、顽固}产生。\end{enumerate}

\subsection\*{\textbf{441} {\footnotesize 〔PTS 438〕}}

\textbf{「利养、名闻、恭敬,与邪得的声誉,\\}
\textbf{「自赞,并且毁他,}

Lābho siloko sakkāro, micchāladdho ca yo yaso;\\
yo c’attānaṃ samukkaṃse, pare ca avajānati. %\hfill\textcolor{gray}{\footnotesize 14}

\begin{enumerate}\item 对连彼等也去除而住者,随后,依于证得更多殊胜,\textbf{利养、恭敬、名闻}生起。痴迷于利养等者,宣说相似法而获得\textbf{邪声誉},住立于此者,则以出身等\textbf{自赞,并且毁他},所以,当知爱欲等的初军等之相。\end{enumerate}

\subsection\*{\textbf{442} {\footnotesize 〔PTS 439〕}}

\textbf{「不解脱者!这些是你的军队,黑暗的攻击者,\\}
\textbf{「怯懦者无法战胜它,而战胜已则得快乐。}

Esā Namuci te senā, Kaṇhassābhippahārinī;\\
na naṃ asūro jināti, jetvā ca labhate sukhaṃ. %\hfill\textcolor{gray}{\footnotesize 15}

\begin{enumerate}\item 如是指出这十种军队后,因为它资助由具足黑法而为「黑暗」的不解脱者,所以,指认其为「你的军队」,说了此颂。这里,\textbf{攻击者},即沙门、婆罗门的杀戮者、毁灭者、障碍制造者之义。\textbf{怯懦者无法战胜它,而战胜已则得快乐},如是顾及身体及性命的怯懦之人无法战胜你的军队,而勇士则能战胜,且战胜已,得证道乐与果乐。\end{enumerate}

\subsection\*{\textbf{443} {\footnotesize 〔PTS 440〕}}

\textbf{「若戴着文阇草,我的活命会成为羞耻!\\}
\textbf{「若被战胜而活,我死在战场更好。}

Esa muñjaṃ parihare, dhir atthu mama jīvitaṃ;\\
saṅgāme me mataṃ seyyo, yañ ce jīve parājito. %\hfill\textcolor{gray}{\footnotesize 16}

\begin{enumerate}\item 且因为获得快乐,所以为愿求快乐,我也\textbf{愿戴着文阇草}\footnote{这里,颂中的译文与义注所解释的正相反,说明如下:首先,口衔文阇草被认为是投降的行为,而非义注中所说的「不撤退之相」,Pischel、Norman 持此观点,并有 Winternitz 引用 Laghvarthanīti 之句为证:「他不应杀害用口齿衔着文阇草的人。」其次,「戴着 \textit{parihare}」作为祈愿语气,可译作「若戴着」或「愿戴着」。所以,这里在颂中作「若戴着」,意即不愿戴,而在义注中作「愿戴着」,与之后的解释相符。}。奔赴战场而不撤退的人们,为令人知晓自己的不撤退之相,会在头首、旗帜或武器上绑扎文阇草,请如是受持我:他也戴着这(文阇草)。
\item 被你的军队战胜,\textbf{我的活命会成为羞耻},所以,请如是受持:\textbf{若被战胜而活,我死在战场更好},意即若被这活命战胜而活,较此活命,我与你,为诸正行道者制造障碍者一起,死在战场更好。\end{enumerate}

\subsection\*{\textbf{444} {\footnotesize 〔PTS 441〕}}

\textbf{「沉沦于此,许多沙门婆罗门未被看见,\\}
\textbf{「不知晓这善行者们所行之道。}

Pagāḷh’ettha na dissanti, eke samaṇabrāhmaṇā;\\
tañ ca maggaṃ na jānanti, yena gacchanti subbatā. %\hfill\textcolor{gray}{\footnotesize 17}

\begin{enumerate}\item 设问:为什么死了更好?因为「许多沙门婆罗门……之道」。\textbf{沉沦于此},陷入、没入于你爱欲为首、自赞毁他为末的军队,\textbf{许多沙门婆罗门未被看见},未以戒等功德显耀,如入暗冥。当他们如是沉沦时,即便有时以「善哉!信」等方式浮起,如落水之人浮起,由被这军队淹没故,也同样\textbf{不知晓这}一切佛、辟支佛等\textbf{善行者们所行}的安稳、趣向涅槃\textbf{之道}。听完此颂,魔罗未发一言,便即离开。\end{enumerate}

\subsection\*{\textbf{445} {\footnotesize 〔PTS 442〕}}

\textbf{「看到周围的军队,跟随驾乘的魔罗,\\}
\textbf{「我前去应战,他无法将我驱离此处!}

Samantā dhajiniṃ disvā, yuttaṃ Māraṃ savāhanaṃ;\\
yuddhāya paccuggacchāmi, mā maṃ ṭhānā acāvayi. %\hfill\textcolor{gray}{\footnotesize 18}

\begin{enumerate}\item 当他离开后,从这行苦行中未证得些许殊胜的大士渐渐想到「是否有其它觉悟之道」等等,吃了粗粝的食物,恢复了力气,在毗舍佉满月日的清晨享用了善生的乳糜,坐在贤林昼住,于此生起八等至,便度过了白天,晡时,朝大菩提座前行,在菩提树下敷上由平安布施的八把草,为一万世界的天人所恭敬、尊重,决意于四支精进:\begin{quoting}宁愿让皮肤筋骨残留,\\让体内的血肉无余地干枯。\end{quoting}立下誓言「现在,不证菩提,我将不破此跏趺」,坐于不可战胜的跏趺。
\item 恶者魔罗了知后,想「今天,悉达多立誓而坐,而就在今天,现在,这誓言当被制止」,召集了从菩提座绵延至轮围山,十二由旬宽、上至九由旬高的魔军,骑上一百五十由旬身量的象王「山带」,幻化出千臂,持种种武器,说着「捉!杀!打」,幻化出如在「旷野经」中所说品类的雨,它们碰到大人,便于此变成所说的品类。随后,便以金刚钩击打象的面瘤,引它冲向大人,说道:「从跏趺起来!悉达多君!」大人说「我不起!魔罗」,看着周围的军队,说了以下几颂。
\item 这里,\textbf{跟随},即被发动。\textbf{驾乘},即伴着象王山带。\textbf{我前去},即我将上前面对,且他唯以威力,而非以身。为什么?\textbf{他无法将我驱离此处},即是说魔罗无法使我从这不可战胜的跏趺坐处动摇。\end{enumerate}

\subsection\*{\textbf{446} {\footnotesize 〔PTS 443〕}}

\textbf{「俱有天的世间不能征服你的军队,\\}
\textbf{「而我将以慧粉碎它们,如以石头粉碎生钵。}

Yaṃ te taṃ nappasahati, senaṃ loko sadevako;\\
taṃ te paññāya bhecchāmi, āmaṃ pattaṃ va asmanā. %\hfill\textcolor{gray}{\footnotesize 19}

\begin{enumerate}\item \textbf{不能征服},即不堪忍受,或不能战胜。\textbf{生钵},即粘土类的土器。\textbf{石头},即岩石。其余于此自明。\end{enumerate}

\subsection\*{\textbf{447} {\footnotesize 〔PTS 444〕}}

\textbf{「控制了思惟,念也善住立,\\}
\textbf{「愿我从国游行至国,调伏众多弟子!}

Vasīkaritvā saṅkappaṃ, satiñ ca sūpatiṭṭhitaṃ;\\
raṭṭhā raṭṭhaṃ vicarissaṃ, sāvake vinayaṃ puthū. %\hfill\textcolor{gray}{\footnotesize 20}

\begin{enumerate}\item 现在,为显示「粉碎了你的魔军,随后,战场的胜者、成就法王灌顶者,我将行此」,说了此颂。这里,\textbf{控制了思惟},即以道的修习,舍弃了一切邪思惟,唯以转起正思惟而控制了思惟。\textbf{念也善住立},即于身等四处,善加住立自身的念。如是控制思惟、善住立念者,\textbf{愿我从国游行至国,调伏众多}人天等类的\textbf{弟子}。\end{enumerate}

\subsection\*{\textbf{448} {\footnotesize 〔PTS 445〕}}

\textbf{「他们不放逸、自励,我的教法的行者\\}
\textbf{「将不如你愿而行,所到之处即不忧伤。」}

Te appamattā pahitattā, mama sāsanakārakā;\\
akāmassa te gamissanti, yattha gantvā na socare”. %\hfill\textcolor{gray}{\footnotesize 21}

\begin{enumerate}\item 然后,\textbf{他们}经我调伏,\textbf{不放逸……即不忧伤},意即此涅槃、不死。
\item 于是,魔罗听了这些偈颂,便说:「看到这些徒众,比丘!你不害怕吗?」「唯!魔罗!我不害怕。」「为什么你不害怕?」「由已作布施等波罗蜜的福德故。」「谁知道你曾作这布施等?」「作证在此有什么用?恶者!况且我仅在一期生命中,曾为毗输安多罗而行布施,以此威力,这大地七次以六种方式作了震动,便是作证。」如是说已,大地便以水为边界而震动,释放出恐怖的声音,魔罗听后,如遭雷劈般恐惧,偃伏了旗帜,便与随从一起逃逸。于是,大人以三夜分证了三明,当明相出现时,便发此慨叹:\begin{quoting}经多生轮回……一切爱灭尽。(法句·老品第 153~154 颂)\end{quoting}\end{enumerate}

\subsection\*{\textbf{449} {\footnotesize 〔PTS 446〕}}

\textbf{「七年间,我步步跟随着世尊,\\}
\textbf{「没等到具念的等正觉的陷落。}

“Satta vassāni Bhagavantaṃ, anubandhiṃ padāpadaṃ;\\
otāraṃ nādhigacchissaṃ, sambuddhassa satīmato. %\hfill\textcolor{gray}{\footnotesize 22}

\begin{enumerate}\item 魔罗因慨叹的声音而返回,想「他自称『我是佛陀』,噫!我就跟随他去观察等正行,如果有任何身、语的失误,我将恼乱他」,便于先前的菩萨地跟随了六年之后,又跟随了证得觉悟者一年。随后,未观察到世尊有任何的失误,便说了这些失望的偈颂。这里,\textbf{陷落},即瑕疵、缺陷。\end{enumerate}

\subsection\*{\textbf{450} {\footnotesize 〔PTS 447〕}}

\textbf{「脂肪色泽的石头,乌鸦围绕而飞:\\}
\textbf{「也许在此我们能发现嫩肉,也许会有美味。}

Medavaṇṇaṃ va pāsāṇaṃ, vāyaso anupariyagā;\\
ap’ettha muduṃ vindema, api assādanā siyā. %\hfill\textcolor{gray}{\footnotesize 23}

\begin{enumerate}\item \textbf{脂肪色泽},即如同脂肪团。\textbf{围绕而飞},即周匝而行。\end{enumerate}

\subsection\*{\textbf{451} {\footnotesize 〔PTS 448〕}}

\textbf{「于彼未得美味,乌鸦便从此离开,\\}
\textbf{「如同乌鸦琢了石头,我们嫌厌了乔达摩而去。」}

Aladdhā tattha assādaṃ, vāyas’etto apakkami;\\
kāko va selam āsajja, nibbijjāpema Gotamaṃ”. %\hfill\textcolor{gray}{\footnotesize 24}

\begin{enumerate}\item 其余于此自明。而其连结为:\textbf{七年间},我们\textbf{跟随世尊},期待陷落,处处不离一步,可即便如是跟随,也\textbf{没等到陷落}。我就好比乌鸦,把\textbf{脂肪色泽的石头}作脂肪想,以喙从侧面琢击,当未得美味,想「\textbf{也许在此我们能发现嫩肉,也许}从这边\textbf{会有美味}」,如是从四周琢击,\textbf{围绕而飞},于处处\textbf{未得美味},便想「这真是石头」,嫌厌而去,如是,我对世尊,于身业等处,以自己的小慧之喙琢击,从四周围绕而飞,想「也许在某处我们能发现不遍净的身正行等的嫩肉之相,也许从某处会有美味」,而现在我们未得美味,\textbf{如同乌鸦琢了石头,我们嫌厌了乔达摩而去},琢了之后,嫌厌而去。\end{enumerate}

\subsection\*{\textbf{452} {\footnotesize 〔PTS 449〕}}

\textbf{他被忧伤击溃,琵琶从腋间滑落,\\}
\textbf{那夜叉意志消沉,即于此处隐没。}

Tassa sokaparetassa, vīṇā kacchā abhassatha;\\
tato so dummano yakkho, tatth’ev’antaradhāyathā ti. %\hfill\textcolor{gray}{\footnotesize 25}

\begin{enumerate}\item 据说,魔罗如是说时,因七年疲乏而无果,便生起深深的忧伤,因此而肢体消沉,名为「木瓜之黄」的琵琶从腋间掉落——它由善巧者演奏一次,便在四月间发出甜美之声——帝释捡到后,便给了五顶\footnote{五顶 \textit{Pañcasikha}:为乾闼婆乐神。},而他竟未觉知其已掉落。因此世尊说了此颂,有人说是结集者所说,但我们对此不认可。\end{enumerate}

\begin{center}\vspace{1em}至上经第二\\Padhānasuttaṃ dutiyaṃ.\end{center}