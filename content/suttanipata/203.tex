\section{惭经}

\begin{center}Hiri Sutta\end{center}\vspace{1em}

\begin{enumerate}\item 缘起为何?世尊未出世时,在舍卫国有某位富有的大财主,拥有八十俱胝的财产,其独子可爱、悦意。他以种种品类的乐事、资助抚养其长大,如待天童一般,却尚未把所有权交付予他,便与妻子同时死了。随后,司库便在这学童的父母身后打开内室,付以所有权,说道:「主人!这是你父母的所有,这是祖父、曾祖父的所有,这来自家族七代。」学童见了财产便想:「唯有这财产可见,而由以积聚者却不可得见,一切唯归制于死,且往者从此带不走任何便往,如是,他世应舍弃资产而往,除了善行,不能带走任何而往,我何不遍舍这财产,去取能带走的善行之财?」
\item 他每天派发百千,便又想:「这财产太多!岂能以如是少量遍舍?我何不行大布施?」他便告知国王:「大王!我家有如许财产,我想行大布施,善哉!大王!请在城内布告!」国王便照做。他装满每位来者的器皿,以七天布施了所有财产,施后便想:「如是作了大遍舍后,已不适合居家,我何不出家?」随后便对随从说起此事。他们说「主人!莫要想『财产耗尽』,我们少时就能以种种方法积累财产」,以种种方式请求。他却不顾他们的请求,出家为苦行者。
\item 这里,有八种苦行者:携妻子者、搜罗者、时到者、未火煮者、拳石者、齿断者、落果者、脱茎者等。这里,\textbf{携妻子者},即与孩子妻妾一起出家,以耕、商等营生的萦发者翅宁等\footnote{萦发者翅宁 \textit{Keṇiya-jaṭila}:见于\textbf{施罗经}。}。\textbf{搜罗者},即在城门建了草庵,在那里教授刹帝利、婆罗门等童子技艺,拒绝金钱而受纳胡麻、米粒等允许的物品,他们较携妻子者更胜。\textbf{时到者},即在食时,获取到来之食而存活者,他们较搜罗者更胜。\textbf{未火煮者},即吃未经火煮的叶、果等而存活者,他们较时到者更胜。\textbf{拳石者},即拿了拳头大的石头,或拿了任何斧、刀等巡游,当饥饿时,便从所遇之树取了皮吃,决意布萨支,培育四梵住,他们较未火煮者更胜。\textbf{齿断者},即连拳头大的石头也不拿而行,当饥饿时,便从所遇之树用牙齿啃了吃皮,决意布萨支,培育梵住,他们较拳石者更胜。\textbf{落果者},即依天然的湖泊或密林而住,只吃彼处水中的莲藕,或密林里花期时的花、果期时的果,当无花果时,乃至吃彼处树上的苔藓而住,而不会为了食物去到别处,决意布萨支,培育梵住,他们较齿断者更胜。\textbf{脱茎者},即只吃从茎上脱落掉到地上的叶子,其余则与前者相同,他们为一切之最胜。
\item 而这婆罗门族姓子想「在苦行者出家中,我要作最高的出家」,便出家为脱茎者,在雪山中越过了两三座山,建了草庵后定居。于是,世尊出世,转起最上法轮,渐次到了舍卫国,住舍卫国祇树给孤独园。尔时,一个住在舍卫国的人为寻觅旃檀心材等到了他的草庵,礼拜后站在一边。他见到后,便问:「你从哪里来?」「从舍卫国,尊者!」「那里有什么消息吗?」「那里,尊者!人们不放逸,行布施等的福德。」「听了谁的教诫?」「佛世尊的。」苦行者因听闻佛陀之声而惊叹:「先生!你说『佛陀』?」以生腥经中所说的方法问了三次后心满意足「即便此声也难得」,欲到世尊跟前,便想:「空手去到佛陀跟前不太妥,我应该带点什么再去。」但又想:「诸佛不重财利,噫!我带上法的礼物去吧!」便准备了四个问题:\begin{quoting}不应亲近怎样的朋友?应亲近怎样的朋友?\\应从事怎样的努力?什么是味中最上?\end{quoting}
\item 他带着这些问题,便朝中国出发,渐次到了舍卫国,进入衹园。尔时,世尊正为开示法而坐于坐处。他见到世尊,未予顶礼就站在一边。世尊便以「仙人!还好吗」等方法问候。他以「乔达摩君!还好」等方法答复后,想「如果他是佛陀,就会以言语解答以意提出的问题」,便唯以意问了世尊这些问题。世尊为解答婆罗门提出的初问,便先说了两颂半。\end{enumerate}

\subsection\*{\textbf{256} {\footnotesize 〔PTS 253〕}}

\textbf{摆脱着惭,嫌厌着,说着「我是你的」,\\}
\textbf{而不承担堪能的工作,应知「他不是我的」。}

Hiriṃ tarantaṃ vijigucchamānaṃ, “tavāham asmi” iti bhāsamānaṃ;\\
sayhāni kammāni anādiyantaṃ, “n’eso maman” ti iti naṃ vijaññā. %\hfill\textcolor{gray}{\footnotesize 1}

\begin{enumerate}\item 其义为:\textbf{摆脱着惭},即越过惭而无惭、无羞。\textbf{嫌厌着},即如见到不净一般。因为无惭者嫌厌惭,如见不净,因此不追随、不执著于此,因此而说「嫌厌着」。\textbf{说着「我是你的」},即以「兄弟!我是你的朋友,欲你得利、快乐,为了你的义利,宁可舍弃我的生命」等方法而说。
\item \textbf{而不承担堪能的工作},且即便如是说已,却连其堪能、能做的工作也不承担,不为去做而受持。\textbf{应知「他不是我的」},有智之士应知晓这样的人「他是假装的朋友、他不是我的朋友」。\end{enumerate}

\subsection\*{\textbf{257} {\footnotesize 〔PTS 254〕}}

\textbf{若对朋友说了爱语而不遵行,\\}
\textbf{智者便知晓(他)言而无行。}

Ananvayaṃ piyaṃ vācaṃ, yo mittesu pakubbati;\\
akarontaṃ bhāsamānaṃ, parijānanti paṇḍitā. %\hfill\textcolor{gray}{\footnotesize 2}

\begin{enumerate}\item \textbf{不遵行},即不依所说的「我会施、我会做」之事而随行。\textbf{若对朋友说了爱语},即以过去、未来之语句来欢迎,以无义之事来摄受,只以字句的影像对朋友展现爱语。\textbf{智者便知晓(他)言而无行},即智者便确定而知晓这样不行其言、只以言语发声者为「仅言者、非友、假装的朋友」。\end{enumerate}

\subsection\*{\textbf{258} {\footnotesize 〔PTS 255〕}}

\textbf{他不是朋友:若总是留心,担心背叛,唯挑剔瑕疵,\\}
\textbf{若在此倚靠,如孩子在胸前,若不被他人分裂,他才是朋友。}

Na so mitto yo sadā appamatto, bhedāsaṅkī randham evānupassī;\\
yasmiñ ca seti urasīva putto, sa ve mitto yo parehi abhejjo. %\hfill\textcolor{gray}{\footnotesize 3}

\begin{enumerate}\item \textbf{他不是朋友:若总是留心,担心背叛,唯挑剔瑕疵},若唯担心背叛,总是因受到甜蜜的对待留心而住,对任何因失念、未作意的所作或是因无智的未作,唯以「当他指责我时,我就以此叱责他」挑剔瑕疵,则他不是可亲近的朋友。
\item 如是,世尊解答了「不应亲近怎样的朋友」这一初问,为解答第二个而说后半颂。其义为:\textbf{若在此}朋友处,朋友以倚靠而随入其心,好比\textbf{孩子在}父亲的\textbf{胸前},不会存有「当我在他胸前倚靠时,会生起苦或不适」等的疑虑,无有担心而倚靠,如是他以朋友之相于妻妾、财产、性命等寄予信任,无有担心而\textbf{倚靠}。\textbf{若}即便\textbf{被他人}说了百种原因、千种原因也\textbf{不分裂},\textbf{他才是}可以亲近的\textbf{朋友}。\end{enumerate}

\subsection\*{\textbf{259} {\footnotesize 〔PTS 256〕}}

\textbf{承担起为人的责任,希求果报者培育\\}
\textbf{生起欢喜、带来赞赏、快乐之因。\footnote{这里的译文颠倒了上下两行的顺序。}}

Pāmujjakaraṇaṃ ṭhānaṃ, pasaṃsāvahanaṃ sukhaṃ;\\
phalānisaṃso bhāveti, vahanto porisaṃ dhuraṃ. %\hfill\textcolor{gray}{\footnotesize 4}

\begin{enumerate}\item 如是,世尊解答了第二问「应亲近怎样的朋友」,为解答第三个而说此颂。其义为:\textbf{因}即原因。那这是什么?即精进。因为它由生起与法相关的喜悦之乐,而被称为「生起欢喜」。如说:\begin{quoting}诸比丘!若于善说的法律发起精进者,便住于乐。(增支部第 1:319 经)\end{quoting}由带来最初的人天之乐,到最后的涅槃之乐,因接近果而为\textbf{快乐}。\textbf{培育},即增长。\textbf{承担起为人的责任},即负起与人相称的重担而住,培育被称为正精勤的精进之因,应从事这样的努力。\end{enumerate}

\subsection\*{\textbf{260} {\footnotesize 〔PTS 257〕}}

\textbf{已饮了远离之味与寂止之味,\\}
\textbf{他无有恐怖、无有恶,饮着法喜之味。}

Pavivekarasaṃ pitvā, rasaṃ upasamassa ca;\\
niddaro hoti nippāpo, dhammapītirasaṃ pivan ti. %\hfill\textcolor{gray}{\footnotesize 5}

\begin{enumerate}\item 如是,世尊解答了第三问「应从事怎样的努力」,为解答第四个而说此颂。这里,由从远离烦恼而生,\textbf{远离}被称为最上果,其味则以乐味之义,与乐相应。由止息烦恼而生,或由被称为涅槃的寂止所缘,\textbf{寂止}亦然。由不离圣道、在被称为涅槃的法中生起的喜味,\textbf{法喜之味}亦然。已饮了这远离之味与寂止之味,饮着法喜之味,\textbf{他无有恐怖、无有恶},饮后以无有烦恼热恼而无有恐怖,且饮时由舍弃恶而无有恶,所以此为「味中最上」。而有人则连接为「以禅那、涅槃、省察以及身、心、依持之远离,而有远离之味等此三法」,仍以前者为善。
\item 如是,世尊为解答第四问,以阿罗汉为顶点完成了开示。在开示终了,婆罗门在世尊跟前出了家,不久便成了证无碍解的阿罗汉。\end{enumerate}

\begin{center}\vspace{1em}惭经第三\\Hirisuttaṃ tatiyaṃ.\end{center}