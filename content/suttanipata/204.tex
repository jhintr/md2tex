\section{吉祥经}

\begin{center}Maṅgala Sutta\end{center}\vspace{1em}

\begin{enumerate}\item 缘起为何?据说,在阎浮提,大众在各地的城门、议事厅、会堂等处聚集,施以金钱,教人演说劫持悉多\footnote{劫持悉多 \textit{Sītāharaṇa}:即罗摩衍那中的故事。}等种种品类的外道言谈,一一言谈经四月方止。一天,那里便发起了吉祥的言谈:「到底什么是吉祥?什么是见吉祥、闻吉祥、觉吉祥?谁知道吉祥?」
\item 于是,名为见吉祥的人便说:「我知道吉祥,所见是世间的吉祥,所见,即共许为祥瑞的色。比如于此,有人按时起来,见到犀鸟,或木橘的幼树,或孕妇,或庄严齐整的孩童,或盛满的水罐,或新鲜的鲑鱼,或纯种的马,或纯种马车,或公牛,或母牛,或棕色的母牛,或见到其它任何这样共许为祥瑞的色,这被称为见吉祥。」有些人接受了他的话,有些则不接受。那些不接受者便与他争论。
\item 于是,名为闻吉祥的人便说:「这眼,先生!既能见净,也见不净,对善妙、不善妙,可意、不可意也同样。如果以之所见为吉祥,一切都会吉祥,所以所见并非吉祥,而所闻则是吉祥,所闻,即共许为祥瑞的声。比如于此,有人按时起来,听到『已增长』,或『正增长』,或『圆满』,或『鬼宿』\footnote{鬼宿 \textit{phussa}:亦作月名,音译为「弗沙」,在今十二月~一月间,这里似作星宿为妥。},或『欢喜』,或『幸运』,或『幸运增长』,或『今天好星象、好寸阴、好日头、好吉祥』,或任何这样共许为祥瑞的声音,这被称为闻吉祥。」也有些人接受了他的话,有些则不接受。那些不接受者便与他争论。
\item 于是,名为觉吉祥的人便说:「这耳,先生!能听善、不善、可意、不可意。如果以之所闻为吉祥,一切都会吉祥,所以所闻并非吉祥,而所觉则是吉祥,所觉,即共许为祥瑞的香、味、触。比如于此,有人按时起来,嗅到荷花香等的花香,或嚼余甘子的齿木,或触摸大地,或触摸新鲜的谷物、湿润的牛粪、乌龟、一担胡麻、花、果,或适当涂抹细腻的粘土,或著细腻的衣布,或戴细腻的头巾,或嗅、尝、触任何这样共许为祥瑞的香、味、触,这被称为觉吉祥。」也有些人接受了他的话,有些则不接受。
\item 那里,见吉祥无法说服闻吉祥、觉吉祥,他俩中的某个也不能说服另两个。且在这些人中,若接受了见吉祥的话,则认为「唯所见是吉祥」,若接受了闻吉祥、觉吉祥的话,则认为「唯所闻、唯所觉是吉祥」。如是,这吉祥的言谈充斥于整个阎浮提。
\item 于是,在整个阎浮提,人们团团聚集后便思惟吉祥:「到底什么是吉祥?」守护这些人的天人听闻这言谈后,便也同样思惟。地居天是这些天人的朋友,听闻后便也同样思惟,而空居天是这些天人的朋友、四大王天是空居天的朋友……以此方法,直至阿迦腻吒天是善见天的朋友,听闻后,阿迦腻吒天便也同样团团聚集,思惟吉祥。
\item 如是,一万轮围的一切处便生起了吉祥的思惟。且它一经生起,便得不到「这是吉祥、这是吉祥」的裁断,住于裁断十二年。一切人、天与婆罗门,除圣弟子外,依所见、所闻、所觉分裂成三种,连一个能以「唯此是吉祥」如实至于究竟的也没有,世间便生起吉祥的喧哗。
\item \textbf{喧哗有五种}:劫的喧哗、转轮王的喧哗、佛陀的喧哗、吉祥的喧哗、寂默的喧哗。这里,欲界天人披头散发,泪流满面,以手拭泪,穿著红衣,持极扭曲的外表,在人道上徘徊宣告:「经百千年,劫起将生,此世间将毁灭,大海会干涸,而此大地与须弥山王会燃烧毁灭,世间消亡,直至梵界,先生们!请修习慈!先生们!请修习悲、喜、舍!请侍奉母亲!请侍奉父亲!请尊敬族中长者!警觉!莫放逸!」此即\textbf{劫的喧哗}。仍是欲界天人在人道上徘徊宣告:「经百年,转轮王将出现于世!」此即\textbf{转轮王的喧哗}。
\item 而净居天人庄严以梵璎珞、裹以梵头巾已,生起喜悦,述说佛陀的功德,在人道上徘徊宣告:「经千年,佛陀将出现于世!」此即\textbf{佛陀的喧哗}。仍是净居天人,了知众人之心后,在人道上徘徊宣告:「经十二年,正等正觉者将谈论吉祥!」此即\textbf{吉祥的喧哗}。仍是净居天人在人道上徘徊宣告:「经七年,某位比丘会与世尊相遇,问寂默的行道!」此即\textbf{寂默的喧哗}\footnote{即\textbf{经集}·那罗迦经。}。在这五种喧哗中,在以所见之吉祥等三种而分裂的人天中,此吉祥的喧哗便在世间生起。
\item 于是,当人天反复简别而不得吉祥时,经十二年,三十三天的天人便聚集相会,如是共相思惟:「先生们!好比家主之于家内之人,村主之于村民,国王之于一切人等,如是诸天因陀帝释为我等最上、最胜——他以福德、光辉、自在、智慧为二种天界的君主,我们何不去问诸天因陀帝释此义?」他们便前往帝释的跟前,诸天因陀帝释的身体以随适于此时的著装、璎珞而灿烂,为二又二分之一俱胝的天女所围绕,坐在波利质多树下的黄毯最胜座上,他们礼敬已站在一边,便说:「大人!你要知道,现今发起了吉祥之问,一些说所见即吉祥,一些说所闻即吉祥,一些说所觉即吉祥,我们和其他人对此未得究竟,善哉!请您确切地给予解答!」
\item 天王天性聪明,便说:「这吉祥的言谈最初从哪里发起的?」他们便说:「我们,陛下!从四大王天听闻。」随后,四大王天从空居天,空居天从地居天,地居天从守护人类之天,守护人类之天则说:「是在人世间发起的。」
\item 于是,诸天因陀帝释便问:「正等正觉者住在哪里?」他们便说:「人世间,陛下!」他说:「有谁去问了彼世尊?」「没人,陛下!」「先生们!难道你们想丢开火而取萤火之光?越过能开示无余之吉祥的彼世尊,想到本该提问的我?去!先生们!我们去问彼世尊,我们定能得到祥瑞的解答!」便命令一位天子:「你去问世尊!」这天子便以随适于此时的庄严庄严了自己,如发出闪电一般,为天众所围绕,来到了祇林大寺,礼敬了世尊后站在一边,为问吉祥之问而以偈颂发言。世尊为对其解答此问,便说了此经。\end{enumerate}

\textbf{如是我闻。一时世尊住舍卫国祇树给孤独园。于是,有一容貌殊胜的天人在深夜中照亮了整座祇园,往世尊处走去,走到后,礼敬了世尊,站在一边。然后,这位站在一边的天人以偈颂对世尊说:}

Evaṃ me sutaṃ— ekaṃ samayaṃ Bhagavā Sāvatthiyaṃ viharati Jetavane Anāthapiṇḍikassa ārāme. Atha kho aññatarā devatā abhikkantāya rattiyā abhikkantavaṇṇā kevalakappaṃ Jetavanaṃ obhāsetvā yena Bhagavā ten’upasaṅkami, upasaṅkamitvā Bhagavantaṃ abhivādetvā ekamantaṃ aṭṭhāsi. Ekamantaṃ ṭhitā kho sā devatā Bhagavantaṃ gāthāya ajjhabhāsi:

\begin{enumerate}\item 这里,\textbf{如是我闻}等义,已在耕田婆罗豆婆遮经注中略述,而欲求其详者,当以中部义注·破除戏论中所说的方法把握。且在耕田婆罗豆婆遮经中所说的「住摩竭陀南山的一那罗婆罗门村」,此处为「住舍卫国祇树给孤独园」。所以,这里我们将以「舍卫国」开始,解释先前未曾释词者。
\item 此即:\textbf{舍卫国},即如是之名的城市。据说,它曾是名为舍卫的仙人的住处,所以,好比 Kusamba 的住处被称为㤭赏弥 \textit{Kosambī},Kākaṇḍa 的住处被称为 Kākaṇḍī,如是以阴性被称为舍卫国。而古人们则解释说:因为当商队在此处相遇,被问及「有什么物品」时,他们便说「全都有 \textit{sabbam atthi}」,所以依于此语而被称为舍卫国 \textit{Sāvatthī}。在此舍卫国,即以此显明其行处村落。
\item 祇陀是王子之名,因为由他种植栽培,此祇陀之林而为\textbf{祇树}。因在此存有给予众孤独者的团食而为\textbf{给孤独}。即在给孤独长者以遍舍五十四俱胝所建的园林之义,以此显明其随适于出家众的住处空间。
\item \textbf{于是},即不间断之义、显示另一场景之义的不变词。以此显示犹未间断时,在世尊的寺庙处发生了这另一场景。那是什么?即「有一天人」等等。这里,\textbf{有一},即不定的指称。因为他的名字、族姓未明,所以说是「有一」。\textbf{天人}即天,此词男女通用。而在此处,他唯是男性天子,而以通用之名得称「天人」。
\item \textbf{深夜}中的「深 \textit{abhikkanta}」字见于灭尽、善妙、殊胜、随喜等处。这里,在\begin{quoting}尊者!夜已深,已过初夜,比丘僧团已久坐。请世尊为众比丘诵波罗提木叉!(增支部第 8:20 经)\end{quoting}等处为灭尽,在\begin{quoting}他在此四人中更善妙、更高贵。(增支部第 4:100 经)\end{quoting}等处为善妙,在\begin{quoting}谁礼拜我的双足?以神变、以辉光闪亮,\\以殊胜的容貌在一切方向照耀?(天宫事第 857 颂)\end{quoting}等处为殊胜,在\begin{quoting}希有 \textit{abhikkantaṃ}!乔达摩君!希有!乔达摩君!(增支部第 2:16 经)\end{quoting}等处为随喜。而此处则为灭尽,即是说在此灭尽之夜、消尽之夜。
\item \textbf{容貌殊胜}中的「深 \textit{abhikkanta}\footnote{此处字面应作「殊胜」,为显示与上一段的「深」字为同一字,故也写作「深」。}」字为殊胜,而「色 \textit{vaṇṇa}\footnote{此处字面应作「容貌」,为显示其基本义,故作「色」。}」字见于肤色、赞美、家族品第、原因、形状、度量、色处等处。这里,在\begin{quoting}你肤色金黄,世尊!(经集第 554 颂)\end{quoting}等处为肤色,在\begin{quoting}长者!你何时收集了沙门乔达摩的这些赞美?(中部·优波离经第 77 段)\end{quoting}等处为赞美,在\begin{quoting}乔达摩君!有这四种种姓 \textit{vaṇṇa}。(长部·起世因本经第 115 段)\end{quoting}等处为家族品第,在\begin{quoting}那么,以何原因被称为偷香?(相应部第 9:14 经)\end{quoting}等处为原因,在\begin{quoting}化现作大象王之色。(相应部第 4:2 经)\end{quoting}等处为形状,在\begin{quoting}钵有三色。\end{quoting}等处为度量,在\begin{quoting}色、香、味、食素。\end{quoting}等处为色处。而此处当视作肤色,即是说容貌殊胜、肤色殊胜。
\item \textbf{整座}中的「整」字有无余、多数、不杂、不多过、坚定、离系等多种含义。如在\begin{quoting}纯一 \textit{kevala} 满净之梵行。(长部·阿摩昼经第 255 段)\end{quoting}等处为无余之义,在\begin{quoting}整个鸯伽、摩揭陀的人拿了许多食物,想要前往。(大品第 43 段)\end{quoting}等处为多数,在\begin{quoting}纯大 \textit{kevala} 苦蕴集起。(分别论第 225 段)\end{quoting}等处为不杂,在\begin{quoting}确实,此尊者仅仅 \textit{kevala} 是依于信……(大品第 244 段)\end{quoting}等处为不多过,在\begin{quoting}尊者!尊者阿耨楼陀的门人,名为跋西迦者,完全地 \textit{kevalakappaṃ} 住于破僧。(增支部第 4:243 经)\end{quoting}等处为坚定,在\begin{quoting}被称为整全 \textit{kevalī}、已立、最上之人。(相应部第 22:57 经)\end{quoting}等处为离系。而此处当为无余之义。
\item 而此「座 \textit{kappa}」字有可信、惯例、时间、施设、截断、异说、藉口、全部等多种含义\footnote{\textbf{犀牛角经}第 35 颂的义注解释 kappa 为「相似 \textit{paṭibhāga}」。}。如在\begin{quoting}此乔达摩君可信,好比阿罗汉、正等正觉者般。(中部·大萨遮经第 387 段)\end{quoting}等处为可信之义,在\begin{quoting}我允许,诸比丘!以五种沙门作净 \textit{samaṇakappa} 来受用果实。(小品第 250 段)\end{quoting}等处为惯例,在\begin{quoting}我以之常时而住。(中部·大萨遮经第 387 段)\end{quoting}等处为时间,在\begin{quoting}尊者劫波 \textit{Kappa} 说……(经集第 1099 颂)\end{quoting}等处为施设,在\begin{quoting}业已庄严,截断须发。(本生第 22:1368 颂)\end{quoting}等处为截断,在\begin{quoting}许可二指净 \textit{dvaṅgulakappa}。(小品第 446 段)\end{quoting}等处为异说,在\begin{quoting}最好去躺下。(增支部第 8:80 经)\end{quoting}等处为藉口,在\begin{quoting}照亮了整座竹林。(相应部第 2:13 经)\end{quoting}等处为全部。而此处当为全部之义。因为此中的「整座祇园」,当视为无余、全部的祇园之义。\textbf{照亮},即以光充满,如月如日,使之一片光照、一片光明之义。
\item \textbf{往世尊处走去},即依格之义的具格\footnote{这是指 yena 一词。},此处当视为「因为世尊在彼,便去向彼处」之义。或者亦可视为「因为世尊为人天所应前往,便以此因而前往」之义。那以何原因,世尊为应当前往?以旨在证得种种品类的功德、殊胜,如鸟群以旨在受用甘甜的果实之于常年结果的大树一般。走去,即是说到达。
\item \textbf{走到后},即显明前往之终了。或者,即是说如是到达已,随后去到被称为世尊附近的距离更近的位置。\textbf{礼敬了世尊},即礼拜、躬身、敬礼了世尊。\textbf{一边},即表示状态的中性词,即是说一处空间、一侧,或为依格之义的业格。\textbf{站},即排除坐等,从事站、站立之义。那如何站为站在一边?\begin{quoting}不在前,不在后,亦不近、不远,\\非于沟渠、逆风,亦非低处高处,\\避开这些过失,便说是站在一边。\end{quoting}
\item 那么,他为什么唯是站,而非坐?为欲轻快地退却故。因为天人出于某些用意来到人世间,好比洁净之人来到粪坑一样。就彼等的天性,从百由旬开外,人世间便因恶臭而厌逆,彼等不乐此处,因此他行了前来的义务后,为欲轻快地退却便不坐。且为了去除行等威仪的疲乏而坐,对于天人,并无此疲乏,所以便也不坐。且众大弟子围绕世尊而站,他尊敬彼等,便也不坐。复次,唯为尊重世尊,他便不坐。因为对于欲坐的天人,坐处便会出现,而他并不求此,连坐下的心也不起,便站在一边。
\item \textbf{然后,这位站在一边的天人},即出于如此等原因而站在一边的这位天人。\textbf{以偈颂对世尊说},即对世尊以限定的音节、句式组织之语而说之义。\end{enumerate}

\subsection\*{\textbf{261} {\footnotesize 〔PTS 258〕}}

\textbf{众多天人与人都曾思惟吉祥,\\}
\textbf{希求幸福,请您说说最上的吉祥。}

“Bahū devā manussā ca, maṅgalāni acintayuṃ;\\
ākaṅkhamānā sotthānaṃ, brūhi maṅgalam uttamaṃ”. %\hfill\textcolor{gray}{\footnotesize 1}

\begin{enumerate}\item 这里,\textbf{众多},即指称不定之数,以此来说数百、数千、数百千。以嬉戏为\textbf{天人},即以种种五欲游戏,或以自身的光辉而闪耀之义。且依共许、投生、清净而有三种天人。如说:\begin{quoting}天人,有三种天人:共许的天人、投生的天人、清净的天人。这里,共许的天人,即国王、王后、王子等。投生的天人,即自四大王天以上的天人。清净的天人,即阿罗汉。(小义释)\end{quoting}这里指的是其中投生的天人。摩奴的后代为\textbf{人}。而古人们说,因意的增盛而为人。他们依阎浮提人、西瞿耶尼人、北俱卢人、东毗提诃人而有四种。这里指的是阎浮提人。有情以此而吉祥者为\textbf{吉祥}\footnote{有情以此而吉祥者为吉祥 \textit{maṅgalan ti imehi sattā ti maṅgalāni}:PTS 本断句作 maṃ galanti imehi sattā ti maṅgalāni,可译为「有情以此滋益我者为吉祥」,聊备一说。},且臻于繁荣、增长之义。
\item \textbf{希求},即希望、愿求、渴望。\textbf{幸福},即平安的状态,即是说一切现世、来世的净美、善妙、可爱、如法的存在者。\textbf{请说说},即请开示、说明、宣说、揭示、分别、显明。\textbf{吉祥},即繁荣之因、增长之因、一切成就之因。\textbf{最上},即殊胜、高贵、带来一切世间的利乐。这是偈颂的逐字释词。
\item 而其聚合之义为:一万轮围中为欲听闻吉祥之问的众天人聚集在这一轮围中,化作十、二十、三十、四十、五十、六十、七十乃至八十仅占据一毫尖端的微细之身,世尊则以辉光超越了一切天、魔、梵而闪耀,坐于设好的最上佛座,他们围绕而立,这天子见到后,又以心了知到此时未能到来的整个阎浮提人的心思,为拔除一切人天的疑箭,便说:「众多天人与人都曾思惟吉祥,希求幸福,希望自己平安,请您说说最上的吉祥!我的提问经彼等天人的允许,且为摄受人类,请依于对我等的怜悯,说说为我们全体带来究竟利乐的最上的吉祥,世尊!」\end{enumerate}

\subsection\*{\textbf{262} {\footnotesize 〔PTS 259〕}}

\textbf{不亲近愚人,且亲近智者,\\}
\textbf{供养应供者,这是最上的吉祥。}

“Asevanā ca bālānaṃ, paṇḍitānañ ca sevanā;\\
pūjā ca pūjaneyyānaṃ, etaṃ maṅgalam uttamaṃ. %\hfill\textcolor{gray}{\footnotesize 2}

\begin{enumerate}\item 如是,听闻了天子的此语后,世尊便说了此颂。这里,\textbf{不亲近},即不结交、不承事。\textbf{愚人},以存活、呼吸为愚人,意即仅以吸气、呼气而活,而非以慧命。\textbf{智者},以前往为智者,意即于现世、来世的义利,以智趣\footnote{智趣 \textit{ñāṇagati}:这里的「趣」当亦为智慧之义,见\textbf{施罗经}的义注。}而前往。\textbf{亲近},即结交、承事、与之为友、与之亲密。\textbf{供养},即恭敬、尊重、尊敬、礼拜。\textbf{应供者},即值得供养者。\textbf{这是最上的吉祥},即此不亲近愚人、亲近智者、供养应供者,结合这一切后而说「这是最上的吉祥」。即是说,你所问的「请您说说最上的吉祥」,此中,请先把握「这是最上的吉祥」。这是此颂的释词。
\item 而其释义当如是了知:如是,听闻了天子的此语后,世尊便说了此颂。这里,因为有四种谈论:被问之谈、未问之谈、俱适用之谈、无适用之谈。这里,\begin{quoting}我问你,广慧的乔达摩!如何行事的弟子才是善的?(经集第 379 颂)\end{quoting}以及\begin{quoting}先生!你如何得度暴流?(相应部第 1:1 经)\end{quoting}等处被问及而谈为\textbf{被问之谈}。\begin{quoting}他人说是乐的,圣者们说是苦。(经集第 768 颂)\end{quoting}等处未被问及,唯以自身的意乐而谈为\textbf{未问之谈}。诸佛的一切谈论,以\begin{quoting}诸比丘!我开示俱因之法。(增支部第 3:126 经)\end{quoting}之语,为\textbf{俱适用之谈}。在此教法内没有\textbf{无适用之谈}。如是,在这些谈论中,由世尊为天子所问而谈,为被问之谈。
\item 且在被问之谈中,好比黠慧之人善巧于道,善巧于非道,在被问及道时,先宣说应舍弃者,再宣说应把握者:「在某处有两叉路,于此舍左取右!」如是于应亲近、不应亲近中,先宣说不应亲近者,再宣说应亲近者。且世尊如善巧于道者一般。如说:\begin{quoting}低舍!善巧于道者,即如来、阿罗汉、正等正觉者的同义语。(相应部第 22:84 经)\end{quoting}因为他\begin{quoting}善巧于此世间,善巧于后世间,善巧于死境,善巧于不死境,善巧于魔境,善巧于非魔境。(中部·小牧牛经第 352 段)\end{quoting}所以,先宣说不应亲近者而宣说应亲近者,便说「不亲近愚人,且亲近智者」。因为先应舍弃之道好似不应亲近、不应承事的愚人,随后应把握之道好似应亲近、应承事的智者。
\item 那为什么世尊在谈论吉祥时,先谈论不亲近愚人及亲近智者?当答:因为人天以亲近愚人而执取此所见等中的吉祥之见,而彼实非吉祥,所以,为指责彼等破坏此世义利与他世义利之与不善知识的交际,并为赞叹成就两者义利之与善知识的交际,先谈论不亲近愚人及亲近智者。
\item 这里,\textbf{愚人}者,即任何具足杀生等不善业道的有情,彼等当以三种行相而知。如经说:\begin{quoting}诸比丘!有这三种愚人的愚人相。(增支部第 3:3 经)\end{quoting}复次,当知即富楼那迦叶等六师,提婆达多、瞿迦梨、迦留罗提舍、骞荼达婆、三闻达多\footnote{瞿迦梨 \textit{Kokālika}、迦留罗提舍 \textit{Kaṭamodakatissa}、骞荼达婆 \textit{Khaṇḍadeviyāputta}、三闻达多 \textit{Samuddadatta}:四人即提婆达多的伴党,这里的译名从四分律。}、学童女罗望子\footnote{学童女罗望子,见\textbf{雪山经}第 155 颂的义注。}等,以及过去时长明的兄弟\footnote{长明的兄弟 \textit{Dīghavidassa bhātā}:此人未见于三藏及注疏,待考。}等与其他类似的愚痴有情。
\item 他们如被火点燃的炭木,以自身的恶执败坏自己及践行自己的话语者,好比长明的兄弟经历四佛,以六十由旬之量的自体仰面坠落,在大地狱中煎熬,且好比赞同其见的五百家族,同样投生为其伴侣,在地狱中煎熬。因为如说:\begin{quoting}此即如,诸比丘!从苇舍或草舍发起之火,甚至能烧尽尖顶、内外抹面、无风、门闩合上、窗户紧闭者,如是,诸比丘!任何生起的怖畏,这一切都由愚人生起,非由智者。任何生起的祸害……任何生起的危险……非由智者。如此,诸比丘!愚人带有怖畏,智者则无怖畏,愚人带有祸害,智者则无祸害,愚人带有危险,智者则无危险。(增支部第 3:1 经)\end{quoting}
\item 复次,愚人如同臭鱼,而亲近彼者则如同裹臭鱼的叶包,为有智者所弃置、嫌厌。如说:\begin{quoting}若人以固沙草叶捆缚臭鱼,\\则固沙草也散发腐臭,如是即亲近愚人。(如是语第 76 经)\end{quoting}且无称智者在受惠于诸天因陀帝释时如是说:\begin{quoting}不应见、不应听愚人,且不应与愚人共住,\\不应与愚人语谈,且不应当赞同。\\愚人对你做了什么?迦叶!请说原因!\\迦叶!你因何不愿见到愚人?\\恶慧者导向非理,怂恿非务,\\难引导至更胜,受到正说却愤怒,\\他不知调伏,善哉不得见彼。(本生第 13:90~92 颂)\end{quoting}
\item 如是,世尊以一切行相指责亲近愚人,说了不亲近愚人为吉祥后,现在为赞叹亲近智者而说亲近智者为吉祥。这里,\textbf{智者}者,即任何具足戒离杀生等十善业道的有情,彼等当以三种行相而知。如经说:\begin{quoting}诸比丘!有这三种智者的智者相。(增支部第 3:3 经)\end{quoting}复次,当知即佛、辟支佛、八十大声闻及其他如来的弟子,以及善导、大典尊、毗荼罗、萨罗绷伽、大药、闻酒、尼弥国王、铁屋童子、无称智者等智者。
\item 他们如怖畏中的守护,如黑暗中明灯,如受制于饥渴等苦时获得饮食,堪能摧破践行自己话语者的一切怖畏、祸害、危险。因为如同藉由如来,无量阿僧祇的人天得证漏尽、住于梵界、住于天界、投生于善趣世间。于舍利弗长老处心得净喜,以四资具给侍长老已,八千家转生天界,于大目犍连、大迦叶等所有大声闻处也同样。善导大师的弟子,有些投生梵界,有些投生为他化自在天的伴侣……长者、大财主家的伴侣。如说:\begin{quoting}诸比丘!从智者无有怖畏,从智者无有祸害,从智者无有危险。(增支部第 3:1 经)\end{quoting}
\item 复次,智者如同山马茶花鬘等香物,而亲近彼者则如包裹山马茶花鬘等香物的叶子,为有智者所可敬、可爱。如说:\begin{quoting}若人以叶捆缚山马茶,\\叶子便散发芳香,如是即亲近智者。(如是语第 76 经)\end{quoting}且无称智者在受惠于诸天因陀帝释时如是说:\begin{quoting}应见智者,应听智者,应与智者共住,\\应与智者语谈,且应当赞同。\\智者对你做了什么?迦叶!请说原因!\\迦叶!你因何而愿见到智者?\\善慧导向如理,不怂恿非务,\\易引导至更胜,受到正说而不怒,\\他了知调伏,善哉与彼相会。(本生第 13:94~96 颂)\end{quoting}
\item 如是,世尊以一切行相赞叹亲近智者,说了亲近智者为吉祥后,现在为赞叹对以此不亲近愚人及亲近智者而渐次至于应供之状态的供养,而说「供养应供者,这是最上的吉祥」。这里,\textbf{应供者},即由除去一切过失、由具足一切功德的诸佛世尊,其后则是诸辟支佛与圣弟子。因为对彼等的供养,即便少量,也会致于长时的利乐,制鬘者善意、末利夫人等可作此处的例证。
\item 这里,我们仅说一个例证。据说,世尊在一天上午著了下衣,持了衣钵,便入王舍城行乞。于是,制鬘者善意正拿着摩揭陀国王频婆娑罗的花而行,便见到世尊到了城门,明净、可喜、严饰以三十二大人相与八十随形好,闪耀着佛陀的光辉。他见后便想:「国王拿到花,会给一百或一千,而这最多只是此世的快乐,但供养世尊则是不可量、阿僧祇的果报,带来长时的利乐,噫!我以这些花供养世尊!」心生净喜,取了一束花,朝世尊抛去,花从空中而去,在世尊上方成了花鬘的华盖而住。制鬘者见此威神,心更净喜,又抛了一束花,它们去后成了花鬘的外围而住。如是,他抛了八束花,它们去后成了花的重阁而住。世尊仿佛在重阁之内,大众云集。世尊看着制鬘者,便显露微笑。阿难长老想「诸佛不会无因无缘显露微笑」,便问微笑之因。世尊说道:「阿难!这制鬘者以此供养的威力,于百千劫轮回于人天,最终将成名为善意自在的辟支佛。」并在言辞的终了,为开示法而说此颂:\begin{quoting}若彼作善业,作已不追悔,\\欢喜而愉悦,应得受异熟。(法句第 68 颂)\end{quoting}在颂的终了,八万四千生类得了法的现观。如是,当知即便少量的供养也会带来长时的利乐。
\item 且这只是财的供养,遑论行道的供养?至于那些族姓子以皈依、领受学处、受持布萨支,及以四遍净戒与自身的功德供养世尊,谁能描述彼等供养的果报?因为彼等被称为以最上的供养供养如来。如说:\begin{quoting}阿难!若比丘,或比丘尼,或优婆塞,或优婆夷,法随法行而住,正当行道,随法而行,他便以最上的供养恭敬、尊重、奉事、供养、崇拜如来。\end{quoting}据此,亦当知供养辟支佛、圣弟子所带来的利乐。
\item 复次,在家人中的幼者应供养长兄长姊,孩子应供养父母,媳妇应供养丈夫、舅姑,当知如是亦为此中的应供者。因为供养彼等由被称为善法及由为寿等增长之因而为吉祥。如说:\begin{quoting}他们应敬母、敬父、敬沙门、梵行、尊敬族中长者,以受持此善法,彼等将转起。因受持这些善法,他们将增长寿,他们将增长美……等等。\end{quoting}
\item 如是,以此颂说了不亲近愚人、亲近智者、供养应供者等三种吉祥。这里,当知不亲近愚人以庇护缘于亲近愚人之怖畏等,由为两处世间的利益之因而为吉祥,亲近智者及供养应供者以在彼等「果报之荣耀」的注释中所说的方法,由为涅槃与善趣之因而为吉祥。此后则不再显示大纲,凡出现吉祥之处,我们将令其明确,并分别其吉祥性。\end{enumerate}

\subsection\*{\textbf{263} {\footnotesize 〔PTS 260〕}}

\textbf{住于适宜处,过去曾培福,\\}
\textbf{自身正誓愿,这是最上的吉祥。}

Patirūpadesavāso ca, pubbe ca katapuññatā;\\
attasammāpaṇidhi ca, etaṃ maṅgalam uttamaṃ. %\hfill\textcolor{gray}{\footnotesize 3}

\begin{enumerate}\item 如是,世尊仅以「请说说最上的吉祥」被请求一次,却如被乞求少许的大施主、广博之人一般,以一颂说了三种吉祥,更有甚者,以诸天之欲闻,以诸多吉祥之存在,以欲敦促彼彼有情于处处随顺于彼彼之吉祥,又开始以此颂等的诸颂而说多种吉祥。
\item 这里,先于初颂,\textbf{适宜},即适当。\textbf{处},即村、镇、城、国土等任何有情居住的空间。\textbf{住},即在此居住。\textbf{过去},即先前在过去生中。\textbf{曾培福},即已积累善。\textbf{自身},即是说心,或整个自体。\textbf{正誓愿},即其自身的正誓愿,即是说确立、建立。余如前述。以上是释词。
\item 而其释义当知如是:\textbf{适宜处}者,即于此处有四众居住,转起布施等福行之事,显明大师的九分教。这里的住处,由对有情为福行之缘而被称为「吉祥」。进入僧伽罗岛的渔夫可为此处的例证。
\item 另一法:适宜处者,即世尊的菩提座处,转法轮处,在十二由旬的会众中摧破一切外道的觉想已、示现双神变的茎芒果树下处\footnote{茎芒果 \textit{kaṇḍamba} 树下处:亦在舍卫城外,旧译名待考。},从天下降处\footnote{从天下降处:即僧迦尸 \textit{Saṅkassa},见\textbf{舍利弗经}的义注。},或其它在舍卫国、王舍城等佛陀的住处。这里的住处,由对有情为获得六无上\footnote{六无上 \textit{cha-anuttariya}:据菩提比丘注 1074,即见无上、闻无上、利养无上、学无上、敬事无上、随念无上等,见\textbf{增支部}第 6:30 经。}之缘而被称为「吉祥」。
\item 另一法:东方有镇名为迦瞻伽罗,其西为摩诃萨罗,随后即边境,其内即中国。东南方有河名萨罗罗婆提,随后即边境,其内即中国。南方有镇名白耳环,随后即边境,其内即中国。西方有婆罗门村名祭柱,随后即边境,其内即中国。北方有山名香根草幢,随后即边境,其内即中国。此中土长三百由旬,宽二百五十,周匝九百由旬,这名为适宜处。
\item 此中,在四大洲与二千小洲中行使自在主权的转轮王出现,圆满了一阿僧祇又十万劫波罗蜜后,舍利弗、大目犍连等大弟子出现,圆满了二阿僧祇又十万劫波罗蜜后,众辟支佛出现,圆满了四、八或十六阿僧祇又十万劫波罗蜜后,众正等正觉者出现。这里,众有情受了转轮王的教诫,住于五戒,以天为归宿,住于辟支佛的教诫也同样,而住于正等正觉者及弟子的教诫,则以天或涅槃为归宿。所以,这里的住处,由为这些成就之缘而被称为「吉祥」。
\item \textbf{过去曾培福}者,即在过去生中,就佛、辟支佛、漏尽者已积累善,此亦为吉祥。为什么?当面见到佛、辟支佛,以在诸佛或佛弟子的面前听闻,即便在四句偈的终了,也能证得阿罗汉。且若人过去曾作服务、增盛善根,他便以此善根令毗婆舍那生起,证得漏尽,好比国王摩诃劫宾那及正妃。因此说过去曾培福为「吉祥」。
\item \textbf{自身正誓愿}者,于此,有些人将自身的恶戒住立于戒,将不信住立于信的成就,将悭住立于舍的成就。这被称为「自身正誓愿」,且此为吉祥。为什么?由为舍弃现法、来世的敌对以及证得种种利益之因故。
\item 如是,以此颂也说了住于适宜处、过去曾培福、自身正誓愿等三种吉祥,且已在各处分别彼等的吉祥性。\end{enumerate}

\subsection\*{\textbf{264} {\footnotesize 〔PTS 261〕}}

\textbf{博学,技艺,且善学律仪,\\}
\textbf{若善说话语,这是最上的吉祥。}

Bāhusaccañ ca sippañ ca, vinayo ca susikkhito;\\
subhāsitā ca yā vācā, etaṃ maṅgalam uttamaṃ. %\hfill\textcolor{gray}{\footnotesize 4}

\begin{enumerate}\item 现在,在此颂中,\textbf{博学},即多闻之状。\textbf{技艺},即精通任何的手艺。\textbf{律仪},即身、语、心的调伏。\textbf{善学},即善加修学。\textbf{善说},即善加言说。\textbf{若},即不定的指称。\textbf{话语},即言词、言路。余如前述。以上是释词。
\item 而释义当知如是:\textbf{博学}者,即任何以\begin{quoting}持闻,即闻的积集。(增支部第 4:22 经)\end{quoting}与\begin{quoting}于此,诸比丘!一人多闻于契经、应颂、记说……(增支部第 4:6 经)\end{quoting}等方法所说明的大师教法之受持,其由为舍弃不善、证得善之因,并由为证得第一义谛之因,被称为「吉祥」。如世尊说:\begin{quoting}诸比丘!具闻的圣弟子舍弃不善,培育善,舍弃有过,培育无过,保持自身之净。(增支部第 7:67 经)\end{quoting}又说:\begin{quoting}他考察所持诸法之义,由察义而认可诸法,当认可法时便生起欲,生欲者勇猛,勇猛者衡量,衡量者精勤,精勤者以身作证第一义谛,且以慧通达而得见之。(中部·瞻翅经第 432 段)\end{quoting}复次,在家博学之无过者,当知由带来两处世间的利乐而为吉祥。
\item \textbf{技艺}者,即在家技艺与出家技艺。这里,在家技艺,即制摩尼、制金等不妨碍他人、避开不善的业,其由带来此世的义利而为吉祥。出家技艺,即筹划、缝补衣等的沙门资具之准备,任何以\begin{quoting}于此,诸比丘!任何对同梵行者的高下应作,比丘于此机敏……(增支部第 10:17 经)\end{quoting}等方法所随处赞叹,以及所说的「作为依怙之法」,当知由为自己与他人带来两处世间的利乐而为吉祥。
\item \textbf{律仪}者,即在家律仪与出家律仪。这里,在家律仪,即戒离十不善道,于此,其以不违犯杂染及确定正行之功德为善学,由带来两处世间的利乐而为吉祥。出家律仪,即于七聚罪的不违犯,其亦以所说的方法为善学。或者,四遍净戒为出家律仪,其以住立于此而证阿罗汉之修学为善学,当知由为证得世、出世间乐之因而为吉祥。
\item \textbf{善说话语}者,即无有妄语等过失的话语。如说:\begin{quoting}诸比丘!具足四支的话语为善说。\end{quoting}或者,非绮语的话语即为善说。如说:\begin{quoting}善人们说善说为最上,说法而非非法,这是第二,\\说可爱而非不可爱,这是第三,说真实而非虚妄,这是第四。(经集第 453 颂)\end{quoting}当知亦由带来两处世间的利乐而为吉祥。且因为它已系属于律仪,所以律仪应以不包摄此的律仪摄来把握,否则以此烦劳所为何来?于此,当知对他人作法的开示之话语为「善说话语」。因为好比「住于适宜处」,如是由为两处世间的利乐与证得涅槃之缘而被称为吉祥。且说:\begin{quoting}佛陀所说的安稳言语,是为了证得涅槃,\\为尽苦的边际,那确是言语中的最上者。(经集第 457 颂)\end{quoting}
\item 如是,以此颂说了博学、技艺、善学律仪、善说话语等四种吉祥,且已在各处分别彼等的吉祥性。\end{enumerate}

\subsection\*{\textbf{265} {\footnotesize 〔PTS 262〕}}

\textbf{给侍父母,摄护妻儿,\\}
\textbf{营生无惑,这是最上的吉祥。}

Mātāpitu upaṭṭhānaṃ, puttadārassa saṅgaho;\\
anākulā ca kammantā, etaṃ maṅgalam uttamaṃ. %\hfill\textcolor{gray}{\footnotesize 5}

\begin{enumerate}\item 现在,在此颂中,母与父为\textbf{父母}。孩子与妻妾为\textbf{妻儿}。摄受即\textbf{摄护}。业即\textbf{营生}。余如前述。这是释词。
\item 而释义当知如是:\textbf{母}者,即生母,\textbf{父}也同样。\textbf{给侍}者,即以洗足、按摩、涂身、沐浴并以给予四资具而提供资助。这里,因为父母给予孩子许多资助,欲求孩子的义利,怜爱孩子,见到孩子在外玩耍后浑身坌尘归来,便掸去尘土,嗅吻其头,生起爱怜,即便孩子顶戴父母百年,也不能回报他们。又因为他们是抚养者、养育者、此世间的展示者、共许为梵、共许为启蒙老师,所以对他们的资助带来此世的赞叹与后世的天乐,因此被称为吉祥。如世尊说:\begin{quoting}父母被称为梵、启蒙老师,\\且怜爱子嗣,应受儿辈的供养。\\所以,智者应礼敬、恭敬他们,\\以食物、以饮品、以衣服、以卧处,\\以涂身、以沐浴及以洗足。\\由此敬事于父母,智者们\\在此世赞叹他,死后则在天界欢喜。(增支部第 3:31 经)\end{quoting}
\item 另一法:给侍者,即扶养、行义务、维持家族世系等五种,当知其由为遮止恶等五种现法利益之因而为吉祥。如世尊说:\begin{quoting}长者子!孩子当以五处给侍东方之父母:受到扶养,我当扶养他们,我当对他们行义务,我当维持家族世系,我当获得遗产,或者,我当对已逝世者给予供养。长者子!受到孩子以此五处给侍的东方之父母,以五处怜爱孩子:遮止于恶,令住于善,令学技艺,令与适宜的妻子结合,适时赠与遗产。(长部·尸迦罗经第 267 段)\end{quoting}
\item 复次,若以令生净喜、令受持戒、出家等三事给侍父母者,彼即给侍父母者之最上。这给侍父母,作为父母已予资助的返助,由为多种现法及来世义利的足处而被称为吉祥。
\item \textbf{妻儿}之中,由自己所生的儿子或女儿都算作孩子。妻妾,即二十种妻子\footnote{这里及下引\textbf{长部}·尸迦罗经第 269 段中的「妻子 \textit{bhariyā}」,字面意思为「受扶养者」。}中的任一种妻子。\textbf{摄护},即以尊敬等给予资助。当知其由为善加整饬之行事等现法利益之因而为吉祥。如世尊说:\begin{quoting}当知西方为妻儿。(长部·尸迦罗经第 266 段)\end{quoting}以「妻子」一词包摄此中所示的妻儿后:\begin{quoting}长者子!丈夫当以五处给侍西方之妻子:以尊敬,以不鄙视,以不通奸,以舍遣主宰,以提供严具。长者子!受到丈夫以此五处给侍的西方之妻子,以五处怜爱丈夫:行事善加整饬,摄受仆从,不通奸,守护积贮,机敏、不怠惰于一切义务。(长部·尸迦罗经第 269 段)\end{quoting}
\item 或者,此为另一法:\textbf{摄护},即以如法的布施、爱语、利行而摄受。此即是:在布萨日布施薪水,在节日令观节庆,在吉祥日行吉祥,教诫、教授现法、来世的义利等。当知其以所说的方式,由为现法利益之因、来世利益之因,且为诸天亦应礼敬之因,而为吉祥。如诸天之因陀帝释说:\begin{quoting}若在家的优婆塞们造作福德、具戒,\\如法养育妻妾,我将礼敬彼等,摩多梨\footnote{摩多梨 \textit{Mātalī}:即帝释的御者。}!(相应部第 11:18 经)\end{quoting}
\item \textbf{营生无惑}者,即以知时、适宜行、不怠惰、具足奋起与精进、无有灾厄而免于违时、不适宜行、不行、弛懈行等惑乱之状的耕田、护牛、商贩等营生。这些,以自身或妻儿或奴仆工人的堪任而得从事,由为现法中财产、谷物的增长与获得之因,被称为吉祥。如世尊说:\begin{quoting}行事得体、负责的奋起者得到财产。(经集第 189 颂)\end{quoting}以及\begin{quoting}非以在白昼惯于睡眠,以厌恶在夜晚起身,\\以始终迷醉上瘾,堪能安居俗家。\\他们说「太冷、太热、这太晚」,\\义利便避开如此舍弃营生的学童。\\若于此视寒暑不多于草芥,\\尽为人的义务,他便不为乐所弃。(长部·尸迦罗经第 253 段)\end{quoting}以及\begin{quoting}若收集财富,一如蜜蜂逡巡,\\财富便趋聚拢,如蚁垤堆起。(长部·尸迦罗经第 265 段)\end{quoting}如是等等。
\item 如是,以此颂也说了给侍母、给侍父、摄护妻儿、营生无惑等四种吉祥,或分摄护妻儿为二而作五种,或合给侍父母为一而作三种,且已在各处分别彼等的吉祥性。\end{enumerate}

\subsection\*{\textbf{266} {\footnotesize 〔PTS 263〕}}

\textbf{布施,法行,摄护亲族,\\}
\textbf{诸业无过,这是最上的吉祥。}

Dānañ ca dhammacariyā ca, ñātakānañ ca saṅgaho;\\
anavajjāni kammāni, etaṃ maṅgalam uttamaṃ. %\hfill\textcolor{gray}{\footnotesize 6}

\begin{enumerate}\item 现在,在此颂中,以之被给予者为\textbf{布施},即是说自己的所有被供与他人。法之行,或不离于法之行,为\textbf{法行}。以「他们是我们的」而被认出者为\textbf{亲族}。没有过失为\textbf{无过},即是说不被非难、不被指责。余如前述。这是释词。
\item 而释义当知如是:\textbf{布施}者,即指定他人,以善觉为先导的遍舍食物等十布施物之思,或是与之相应的无贪。因为以无贪供与他人此物,因此说「以之被给予者为布施」。由其为证得众人所喜爱、悦意等现法、来世之果的殊胜之因,被称为吉祥。\begin{quoting}狮子!施者、施主为众人所喜爱、悦意。(增支部第 5:34 经)\end{quoting}如是等经,可在此被忆念。
\item 另一法:布施者有两种:财施与法施。这里,财施,即所说的品类。而欲求他人的利益,由正等正觉者所宣说的此世、他世苦尽乐来之法的开示,为法施。且在这两种布施中,此为最上。如说:\begin{quoting}诸施法施胜,诸味法味胜,\\诸喜法喜胜,除爱胜诸苦。(法句第 354 颂)\end{quoting}这里,财施的吉祥性已述。而法施,因为是体验义等功德的足处,所以被称为吉祥。如世尊说:\begin{quoting}诸比丘!比丘如是如是详细地对他人开示如所闻、如所学之法,他便如是如是于此法体验义并体验法。(增支部第 5:26 经)\end{quoting}如是等等。
\item \textbf{法行}者,即十善业道之行。如说:\begin{quoting}诸长者!有三种身之法行、平等行。\end{quoting}如是等等。当知此法行由为投生天界之因而为吉祥。如世尊说:\begin{quoting}诸长者!因法行、平等行,如是于此一些有情身坏死后,投生善趣天界。(中部·萨犁耶经第 441 段)\end{quoting}
\item \textbf{亲族}者,即母方或父方直至祖上七代相关者。当他们因破财或患病受到打击,来到自己身边时,尽力以衣食、财产、谷物等摄护,由为证得赞叹等现法及趣向善趣等来世的殊胜之因,被称为吉祥。
\item \textbf{诸业无过}者,即受持布萨支、担当执事、培植园林、修桥等身语意的善行之业。因为这些由为证得种种品类的利乐之因,被称为吉祥。且\begin{quoting}毗舍佉!有此可能,此间的女子或男子遵行具足八支的布萨已,身坏死后,会投生为四大王天的伴侣。(增支部第 8:43 经)\end{quoting}如是等经,可在此被忆念。
\item 如是,以此颂说了布施、法行、摄护亲族、诸业无过等四种吉祥,且已在各处分别彼等的吉祥性。\end{enumerate}

\subsection\*{\textbf{267} {\footnotesize 〔PTS 264〕}}

\textbf{回避、戒离于恶,克制饮酒,\\}
\textbf{于诸法不放逸,这是最上的吉祥。}

Āratī viratī pāpā, majjapānā ca saṃyamo;\\
appamādo ca dhammesu, etaṃ maṅgalam uttamaṃ. %\hfill\textcolor{gray}{\footnotesize 7}

\begin{enumerate}\item 现在,在此颂中,\textbf{回避},即停止。\textbf{戒离},即戒除,或有情以之戒除者为戒离。\textbf{恶},即不善。以可迷醉之义为\textbf{酒}。\textbf{诸法},即诸善。余如前述。这是释词。
\item 而释义当知如是:\textbf{回避}者,即见恶中的过患而意有不乐。\textbf{戒离}者,即由身、语以业门戒除。且此戒离者,有既得离、受持离、正断离等三种。
\item 这里,若缘于族姓子自身的出身、族姓或种姓,以「若我杀此生类、取不与物,这于我不适宜」等方法,由既得之依处而戒离,名为\textbf{既得离}。以受持学处而转起者,名为\textbf{受持离},从其转起开始,族姓子不行杀生等。与圣道相应者,名为\textbf{正断离},从其转起开始,圣弟子的五种怖畏和怨敌\footnote{五种怖畏和怨敌:即对五戒的违犯,见\textbf{增支部}第 10:92 经。}便已止息。
\item \textbf{恶}者,即凡如\begin{quoting}长者子!杀生是业烦恼,不与取……欲邪行……妄语(是业烦恼)。\end{quoting}所详述,以\begin{quoting}杀生、不与取、妄语,及所说的\\通奸,智者不予赞叹。(长部·尸迦罗经第 245 段)\end{quoting}一颂所摄的被称为业烦恼的四种不善。回避、戒离于这一切,由为现法、来世舍弃怖畏和怨敌等以及证得种种殊胜之因,被称为吉祥。\begin{quoting}长者子!圣弟子戒离杀生。\end{quoting}如是等经,可在此被忆念。
\item \textbf{克制饮酒}者,即先前所说\footnote{这是指\textbf{小诵}·十学处的义注。}的「远离放逸之因的谷酒、果酒、麻醉品」的同义语。而因为饮麻醉品者不知义、不知法,对母亲构成障碍,对父亲、佛、辟支佛、如来弟子构成障碍,于现法遭谴责,于来世至恶趣,于后后成疯狂,克制饮酒者则止息这些过失,得达与之相反的功德的成就,所以当知此克制饮酒为吉祥。
\item \textbf{于诸}善\textbf{法不放逸}者,以\begin{quoting}或者,于善法的修习不恭敬行、不常恒行、不注意行、行为迟缓、丢弃欲、丢弃责任、不习行、不修习、不多作、不决意、不践行、放逸,像这样的放逸、成为放逸、已放逸性,这被称为放逸。(分别论第 846 段)\end{quoting}所说的放逸的对立,据义当知为于诸善法的不离念。其由为证得种种品类的善之因,以及为证得不死之因,被称为吉祥。这里,\begin{quoting}不放逸、精勤者……(增支部第 5:26 经)\end{quoting}与\begin{quoting}无逸不死道。(法句第 21 颂)\end{quoting}如是等大师的教法可被忆念。
\item 如是,以此颂说了戒离于恶、克制饮酒、于诸善法不放逸等三种吉祥,且已在各处分别彼等的吉祥性。\end{enumerate}

\subsection\*{\textbf{268} {\footnotesize 〔PTS 265〕}}

\textbf{尊重,谦逊,知足,知恩,\\}
\textbf{时常闻法,这是最上的吉祥。}

Gāravo ca nivāto ca, santuṭṭhi ca kataññutā;\\
kālena dhammassavanaṃ, etaṃ maṅgalam uttamaṃ. %\hfill\textcolor{gray}{\footnotesize 8}

\begin{enumerate}\item 现在,在此颂中,\textbf{尊重},即尊重状。\textbf{谦逊},即举止谦卑。\textbf{知足},即满足。了知所作为\textbf{知恩}。余如前述。这是释词。
\item 而释义当知如是:\textbf{尊重}者,即于应予尊重的佛、辟支佛、如来弟子、阿阇梨、亲教师、父母、兄姊等随适地行尊重、作尊重、俱尊重性。此尊重,因为是趣向善趣等之因,如说:\begin{quoting}尊重应尊重者,奉事应奉事者,供养应供养者。他以此业,以如是成就,以如是受持,身坏死后,投生善趣天界。若身坏……不投生……,若回到人界,则再生之处为上等家族。(中部·小业分别经第 295 段)\end{quoting}以及\begin{quoting}诸比丘!有七不退法。何等为七?尊重大师……(增支部第 7:32 经)\end{quoting}等,所以被称为吉祥。
\item \textbf{谦逊}者,卑下其意,谦逊其行,具足此的人折服了慢、折服了傲,如同擦脚布、如同断了角的公牛、如同拔了牙的蛇,柔和、亲近、乐于共语,此即谦逊。其由为获得名誉等的功德之因,被称为吉祥。且说:\begin{quoting}行为谦逊而不傲慢,如此者得到名誉。(长部·尸迦罗经第 273 段)\end{quoting}如是等等。
\item \textbf{知足}者,即满足于无论何种资具。此有十二种,即于衣,有随所得而满足、随力而满足、随适合而满足等三种,如是于食等。
\item 此为其类别的解释:兹有比丘得衣,或善妙,或不善妙,他即以此存活,不希求其它,即便得到也不接受,此即其于衣\textbf{随所得而满足}。然而,他生了病,当披重衣时弯腰或疲劳,他与同分的比丘交换已,以轻者而存活便即满足,此即其于衣\textbf{随力而满足}。另一比丘有上好资具的利养,他得了衣钵等中某一高价的衣已,想「这适合长久出家且多闻的长老们」,便布施给他们,自己则从尘堆或其它任何处搜集了弊坏衣,做成僧伽梨,穿了便即满足,此即其于衣\textbf{随适合而满足}。
\item 兹有比丘得食,或粗鄙,或上好,他即以此存活,不希求其它,即便得到也不接受,此即其于食\textbf{随所得而满足}。然而,他生了病,吃了粗鄙的食物便病患加剧,他将其给予同分比丘,受用其手中的酥、蜜、乳等已,为行沙门法而满足,此即其于食\textbf{随力而满足}。另一比丘得了上好的食物,想「这食物适合长久出家,及其他没有上好食物便无法存活的同梵行者们」,便布施给他们,自己则行乞,受用混合之食便即满足,此即其于食\textbf{随适合而满足}。
\item 兹有比丘得至坐卧处,他即以此满足,再至其它更善妙处也不接受,此即其于坐卧处\textbf{随所得而满足}。然而,他生了病,当住在无风的坐卧处,便极为胆汁病等所恼,他将其给予同分比丘,在其所在的有风、寒凉的坐卧处住下,为行沙门法而满足,此即其于坐卧处\textbf{随力而满足}。另一比丘即便得至善妙的坐卧处也不领受,想「善妙的坐卧处是放逸之因,于此坐者陷入昏沉睡眠,且为睡眠征服而醒来者现起欲寻」,他拒绝此已,在露地、树下、茅蓬的任何处住下便即满足,此即其于坐卧处\textbf{随适合而满足}。
\item 兹有比丘得药,或呵利勒,或阿摩勒,他即以此存活,不希求以其它所得的酥、蜜、糖等,即便得到也不接受,此即其于疾病资具\textbf{随所得而满足}。而欲求油的病人得了糖,他将其给予同分比丘,将其手中的油作了药,为行沙门法而满足,此即其于疾病资具\textbf{随力而满足}。另一比丘当被招呼「尊者!你想要什么就拿」时,于一器皿内置有腐尿呵利勒,于一置有四蜜,若此两种中的任一都能止息其病,而他想到「腐尿呵利勒者,为佛等所赞叹,且说\begin{quoting}『依腐尿药出家,于此汝应尽寿命坚持』(大品第 128 段)」,\end{quoting}便拒绝四蜜之药,当以尿呵利勒作了药时,便是最上的满足,此即其于疾病资具\textbf{随适合而满足}。
\item 如是,这所有类别的满足被称为知足。其由为证得舍弃过求、恶求、奢求等恶法之因,以及为善趣之因、圣道之资粮、游行四方等之因,当知为吉祥。且说:\begin{quoting}游行四方者无有障碍,随所遇而知足。(经集第 42 颂)\end{quoting}如是等等。
\item \textbf{知恩}者,即对或少或多以任何方式所作的资助,以再再忆念之相而了知。复次,由庇护地狱等苦,福德即是生类的众多资助,由此,忆念彼等的资助当知为知恩。其由为证得为善人所赞叹的种种品类的殊胜之因,被称为吉祥。且说:\begin{quoting}诸比丘!此二类人世间难得,何等为二?先行者,与知恩明恩者。(增支部第 2:120 经)\end{quoting}
\item \textbf{时常闻法}者,即若于彼时有掉举俱行之心,或为欲寻等中的任一所征服,便在此时为去除彼等而闻法。另有人说,每五日闻法,名时常闻法。如尊者阿那律说:\begin{quoting}我们每五天,尊者!整夜为论法而共坐。(中部·小牛角林经第 327 段)\end{quoting}复次,若于彼时能前往善知识处,为去除自身的疑惑而闻法,便在此时闻法,当知亦为时常闻法。如说:\begin{quoting}他们时常前往,遍询、遍问。(长部·十上经第 358 段)\end{quoting}等等。此时常闻法,由为证得舍弃盖、四种利益、漏尽等种种殊胜之因,当知为吉祥。如说:\begin{quoting}诸比丘!若于彼时,圣弟子注意、作意、收摄一切思,倾耳闻法,则于此时,他无有五盖。(相应部第 46:38 经)\end{quoting}以及\begin{quoting}诸比丘!对以耳随行、(以言惯习、以意审虑、以见)善通达之法,四种利益可期。(增支部第 4:191 经)\end{quoting}以及\begin{quoting}诸比丘!此四法,当时常正当修习、正当随转,渐次至于漏尽。何等为四?时常闻法……(增支部第 4:147 经)\end{quoting}
\item 如是,以此颂说了尊重、谦逊、知足、知恩、时常闻法等五种吉祥,且已在各处分别彼等的吉祥性。\end{enumerate}

\subsection\*{\textbf{269} {\footnotesize 〔PTS 266〕}}

\textbf{忍耐,易教,得见沙门,\\}
\textbf{时常论法,这是最上的吉祥。}

Khantī ca sovacassatā, samaṇānañ ca dassanaṃ;\\
kālena dhammasākacchā, etaṃ maṅgalam uttamaṃ. %\hfill\textcolor{gray}{\footnotesize 9}

\begin{enumerate}\item 现在,在此颂中,以顺从地接受,易对其语为易语,易语之业为\textbf{易教}。由止息烦恼为\textbf{沙门}。余如前述。这是释词。
\item 而释义当知如是:\textbf{忍耐}者,即承受之忍耐,具足此的比丘,对以十骂詈事骂詈,或以杀戮、捆缚等伤害的人,若不闻、若不见,保持不变,如忍辱宗者。如说:\begin{quoting}过去,他曾是阐明忍辱的沙门,\\迦尸王割截他,仍以忍辱而住。(本生第 4:51 颂)\end{quoting}或者,以无有更多罪过而作意为善待,如尊者富楼那。如说:\begin{quoting}尊者!如果输那人骂詈、指责我,我于此将如是:「这些输那人贤善,这些输那人甚善——他们竟没有以掌击我。」(中部·教诫富楼那经第 396 段)\end{quoting}
\item 且具足此者应为仙人赞叹,如萨罗绷伽仙人说:\begin{quoting}杀了忿恨,再不忧伤,仙人们赞叹舍弃覆藏者,\\忍耐所有人说的恶语,善人说这是最上的忍耐。(本生第 17:64 颂)\end{quoting}应为诸天赞叹,如诸天因陀帝释说:\begin{quoting}若善人实为有力,忍受弱者,\\他们说这是最高的忍耐——弱者总是忍耐。(相应部第 11:4 经)\end{quoting} 应为诸佛赞叹,如世尊说:\begin{quoting}能忍骂与打,而无有瞋恨,\\具忍力强军,是谓婆罗门。(法句第 399 颂)\end{quoting}而此忍耐由为证得此世的上述赞叹及其它功德之因,当知为吉祥。
\item \textbf{易教}者,即当被如法地说教时,不回避、不默然、不思功德过失,先事以极度的尊敬、尊重、谦卑,而道「善哉」之言。其由为从同梵行者跟前获得教诫、教授之因,以及为舍弃过失、证得功德之因,被称为吉祥。
\item \textbf{得见沙门}者,即凡前往、给侍、随念、听闻、得见已止息烦恼、已修习身戒心慧\footnote{身、戒、心、慧:这里的译文从 PTS 本,原作「身、语、心、慧」,说详菩提比丘注 1093。}、具足最上的调御与奢摩他的出家人,这一切,由开示的缺略,都被称为「得见」。当知其为吉祥。为什么?由资助良多故。且说:\begin{quoting}诸比丘!我说即便见到这些比丘也资助良多。(如是语第 104 经)\end{quoting}等等。
\item 因为欲求利益的族姓子见到具戒的比丘来到家门,若有可施,便应尽力以可施敬候,若无可施,便应五体投地而礼拜。当此不可得时,应合掌礼敬,当此亦不可得时,应以净喜之心、喜爱之眼注视。因为仅以基于得见的福德,便数千生无有眼的病、热、肿、疹,眼睛有明净的五色光芒,如宝殿中敞开的摩尼窗一般,百千劫在人天有一切成就的利养。
\item 且作为生而有慧的人,以正当转起的得见沙门所成的福德能领受如此异熟之成就,此非希有,于此,即便对畜生,他们也赞叹仅仅以信所生的得见沙门的如是异熟之成就:\begin{quoting}圆眼的猫头鹰,在吠提耶山长久居住,\\这枭实在快乐,得见时起的最胜佛陀。\\心于我、于无上的比丘僧团净喜,\\它历百千劫,不会至恶趣。\\从天界殁后,为善业激荡,\\将成无尽智,以「喜悦」闻名。(中部义注第 1.144 段)\end{quoting}
\item \textbf{时常论法}者,即在黄昏,或在黎明,二学经的比丘讨论某一经,持律的论律,学阿毗达摩的论阿毗达摩,诵本生的论本生,学义注的论义注,或为去除为退缩、掉举、疑所据之心,在彼彼时讨论,此即时常论法。其由为精通阿笈摩等功德之因,被称为吉祥。
\item 如是,以此颂说了忍耐、易教、得见沙门、时常论法等四种吉祥,且已在各处分别彼等的吉祥性。\end{enumerate}

\subsection\*{\textbf{270} {\footnotesize 〔PTS 267〕}}

\textbf{苦行,梵行,得见圣谛,\\}
\textbf{证悟涅槃,这是最上的吉祥。}

Tapo ca brahmacariyañ ca, ariyasaccāna dassanaṃ;\\
nibbānasacchikiriyā ca, etaṃ maṅgalam uttamaṃ. %\hfill\textcolor{gray}{\footnotesize 10}

\begin{enumerate}\item 现在,在此颂中,消磨恶不善法为\textbf{苦行}。众梵之行为\textbf{梵行},即是说最胜行。从丛林出离为\textbf{涅槃}。余如前述。这是释词。
\item 而释义当知如是:\textbf{苦行}者,即根律仪,由消磨贪忧等故,或精进,由消磨懈怠故。因为具足此的人被称为热忱者。其由为舍弃贪等、获得禅那等之因,当知为吉祥。
\item \textbf{梵行}者,即戒离淫欲、沙门法、教法、道等的同义语。因为如在\begin{quoting}舍弃非梵行而成梵行者。(长部·沙门果经第 194 段)\end{quoting}等中戒离淫欲被称为梵行,在\begin{quoting}朋友!是否在我们的世尊处建立梵行?(中部·传车经第 257 段)\end{quoting}等中为沙门法,在\begin{quoting}波旬!我此梵行若不兴盛、不繁荣、不为众人广为传播,我不会般涅槃。(长部·大般涅槃经第 168 段)\end{quoting}等中为教法,在\begin{quoting}比丘!此八圣道即梵行,此即是正见……(相应部第 5:6 经)\end{quoting}等中为道。而在此处,由下文以得见圣谛摄道故,其余一切都合适。且其由为证得层层更上的种种殊胜之因,当知为吉祥。
\item \textbf{得见圣谛}者,即以现观在「童子问\footnote{童子问 \textit{Kumārapañha}:即\textbf{小诵}第四经。菩提比丘据 PTS 参校的 S\textsuperscript{gn} 本,作「清净道论」,见其注 1096。}」中所说之义的四圣谛而见道。其由为度越轮回之苦之因,被称为吉祥。
\item \textbf{证悟涅槃}者,此处的「涅槃」意指阿罗汉果,因为它由出离于以五趣之丛林而被称为丛林的渴爱,也被称为涅槃。其证得或省察被称为「证悟」。而以得见圣谛证悟其它涅槃的成就,则非此处的意思。如是,此证悟涅槃由为现法乐住等之因,当知为吉祥。
\item 如是,以此颂也说了苦行、梵行、得见圣谛、证悟涅槃等四种吉祥,且已在各处分别彼等的吉祥性。\end{enumerate}

\subsection\*{\textbf{271} {\footnotesize 〔PTS 268〕}}

\textbf{为世间法所触,其心不动摇,\\}
\textbf{无忧、离尘、安稳,这是最上的吉祥。}

Phuṭṭhassa lokadhammehi, cittaṃ yassa na kampati;\\
asokaṃ virajaṃ khemaṃ, etaṃ maṅgalam uttamaṃ. %\hfill\textcolor{gray}{\footnotesize 11}

\begin{enumerate}\item 现在,在此颂中,\textbf{世间法},即是说只要世间转起,便不停歇之法。\textbf{心},即意。\textbf{其},即新学、中座或上座。\textbf{不动摇},即不震动、不颤动。\textbf{无忧},即离忧、拔出忧箭。\textbf{安稳},即无畏、无祸害。余如前述。这是释词。
\item 而释义当知如是:「\textbf{为世间法所触,其心不动摇}」者,即为得失等八世间法所触、所淹没者的心不动摇、不震动、不颤动。其心由带来不可为任何所动摇的出世间的状态,当知为吉祥。那么谁的心为此等所触而不动摇?阿罗汉漏尽者,而非其他。如说:\begin{quoting}好比整块的岩石,不因风动摇,\\如是色、味、声、香、触及全部\\可意、不可意法,不能撼动如如者,\\心住立、解脱,且随观其灭。(增支部第 6:55 经)\end{quoting}
\item \textbf{无忧}者,即漏尽者之心。因为以\begin{quoting}忧,生忧,忧状,内忧,内遍忧,心之遍烧状。(分别论第 237 段)\end{quoting}等方法所说为忧,由无有此为无忧。有人说为涅槃,此因前句而不适用。且好比无忧,如是离尘、安稳等也是漏尽者之心。因为由离去贪嗔痴之尘为\textbf{离尘},且由离于四轭之安稳为\textbf{安稳}。虽然它以彼彼行相,于彼彼转起的刹那,依所述而把握为三种,由带来不转之蕴等世间最上的状态,及带来应供养等状态,当知为吉祥。
\item 如是,以此颂说了不为八世间法所动摇之心、无忧之心、离尘之心、安稳之心等四种吉祥,且已在各处分别彼等的吉祥性。\end{enumerate}

\subsection\*{\textbf{272} {\footnotesize 〔PTS 269〕}}

\textbf{已作此等者,随处皆不败,\\}
\textbf{随处平安行,这是他们最上的吉祥。}

Etādisāni katvāna, sabbattha-m-aparājitā;\\
sabbattha sotthiṃ gacchanti, taṃ tesaṃ maṅgalam uttaman” ti. %\hfill\textcolor{gray}{\footnotesize 12}

\begin{enumerate}\item 如是,世尊以「不亲近愚人」等十颂谈论了三十八种吉祥,现在为赞美这些由自己所说的吉祥,说了这末颂。
\item 其释义为:\textbf{此等},即这些如这般由我所说的不亲近愚人等。\textbf{随处皆不败},即在一切处,于蕴、烦恼、行作、天子魔罗等的四种怨敌,不为任一所败,即是说自己战胜了四种魔罗。且当知此中的 m 仅作词的连接。
\item \textbf{随处平安行},即作了此等吉祥、不为四种魔罗所败已,于一切此世、他世及站立、经行等,平安地前行,由无有以亲近愚人等将生的诸漏、困扰、热恼,平安地前行,即是说无祸害、无苦痛、安稳、无畏而行。且当知此中的鼻音 \textit{ṃ} 是为了易于结颂而说。
\item \textbf{这是他们最上的吉祥},世尊便以此颂句完成了开示。如何?如是,天子!若作此等者,因为随处平安行,所以此不亲近愚人等虽有三十八种,是那些作此等者最上、最胜、最高贵的吉祥,请把握!
\item 如是,当世尊完成开示的终了,百千俱胝的天人圆满了阿罗汉,证须陀洹、斯陀含、阿那含果者之数不可计算。于是,世尊在第二天告阿难尊者:「阿难!是夜有一天人前来我处,问了吉祥之问,于是我对他说了三十八种吉祥,阿难!受持此吉祥法门!受持已,告知诸比丘!」长老受持已,便告知了诸比丘。当知它经阿阇梨辗转持来,流传至今,如是,此梵行便兴盛、繁荣、为众广播周知,乃至为人天善加宣说。
\item 现在,为仍于这些吉祥熟识、熟练其智,从头开始连结:如是,这些欲求此世、他世、出世之乐的有情,舍弃了亲近愚人,依止智者,供养应供者,以住于适宜处及过去曾培福,被激荡于善的转起,对自身正誓愿已,自体为博学、技艺、律仪所庄严,而说随适于律仪的善说,只要不舍弃俗家身份,便以给侍父母清理旧债,以摄护妻儿承担新债,以无惑的营生得至财富、谷物等的兴盛,以布施得财的坚实,以法行得命的坚实,以摄护亲族行自家的利益,以无过之业行别家的利益,以戒离于恶避免别人的伤害,以克制饮酒避免自己的伤害,以于诸法不放逸增长善品已,以所增长的善性舍弃俗家相而住于出家相,于佛、佛弟子、亲教师、阿阇梨等以尊重、谦逊完成义务,以知足舍弃资具的贪求,以知恩住于善人之地,以闻法舍弃心的退缩,以忍耐克服一切危难,以易教令自己有依怙,以得见沙门而见行道之从事,以论法于可疑之法去除疑惑,以根律仪之苦行成就戒清净,以沙门法之梵行成就心清净,且随后成就另四清净,以此行道证得作为得见圣谛之法门的智见清净,证悟被称为阿罗汉果的涅槃。证悟此者,心如须弥山之于风雨般,不为八世间法所动摇,而成无忧、离尘、安稳。且安稳者,于一切处不为任一所败,并于一切处平安地前行。因此世尊便说:\begin{quoting}已作此等者,随处皆不败,\\随处平安行,这是他们最上的吉祥。\end{quoting}\end{enumerate}

\begin{center}\vspace{1em}吉祥经第四\\Maṅgalasuttaṃ catutthaṃ.\end{center}