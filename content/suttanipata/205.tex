\section{针毛经}

\begin{center}Sūciloma Sutta\end{center}\vspace{1em}

\textbf{如是我闻\footnote{此经旧译见杂阿含经第 1324 经、别译杂阿含经第 323 经。\textbf{针毛},别译杂阿含经作箭毛。\textbf{粗夜叉},杂阿含经作炎鬼,别译作炙夜叉。}。一时世尊住在伽耶石榻针毛夜叉的居处。尔时,粗夜叉与针毛夜叉在世尊不远处经过。于是,粗夜叉对针毛夜叉说:「这是沙门。」「这不是沙门,这是伪沙门,很快我就知道他是沙门还是伪沙门。」}

Evaṃ me sutaṃ— ekaṃ samayaṃ Bhagavā Gayāyaṃ viharati Ṭaṅkitamañce Sūcilomassa yakkhassa bhavane. Tena kho pana samayena Kharo ca yakkho Sūcilomo ca yakkho Bhagavato avidūre atikkamanti. Atha kho Kharo yakkho Sūcilomaṃ yakkhaṃ etad avoca: “eso samaṇo” ti. “N’eso samaṇo, samaṇako eso, yāvāhaṃ jānāmi yadi vā so samaṇo, yadi vā so samaṇako” ti.

\begin{enumerate}\item 缘起为何?其缘起将以释义的方式显明。且在释义中,「如是我闻」等之义已述。而「住在伽耶石榻针毛夜叉的居处」中,何为伽耶?何为石榻?且世尊为什么住在此夜叉的居处?当答:\textbf{伽耶}既是村,也是津渡,在此两者都合适。因为住在伽耶村不远处,可被称为「住在伽耶」,石榻便在此村附近不远的村口前,而住在伽耶津也可被称为「住在伽耶」,石榻便在伽耶津。\textbf{石榻},即在四块岩石之上架起宽广的岩石所做的岩床。夜叉的居处便依于此,如旷野的居处。
\item 或者,因为世尊在这天黎明时分,从大悲等至出起,以佛眼观察世间,便见到针毛与粗毛两夜叉须陀洹果的近依,所以持了衣钵,在日出前便前往津渡的所在——虽然那地方被从四面八方聚集之人的唾涕等种种不净所流污——坐在针毛夜叉的居处。因此说「一时世尊住在伽耶石榻针毛夜叉的居处」。
\item \textbf{尔时},即世尊住在彼处时。\textbf{粗夜叉与针毛夜叉在世尊不远处经过},这些夜叉是谁?又为什么经过?当答:他俩中,首先,一个在过去未问便拿了僧团的油涂抹自己的身体。他以此业堕地狱后,转生在伽耶池岸夜叉的胎中,仍以此业的异熟肢体丑陋,皮肤好似盖瓦,触之粗砺。据说,当他欲令他人恐惧时,便竖起盖瓦似的皮层,令其恐惧。如是,他由粗触故,便得名\textbf{粗夜叉}。
\item 另一则在迦叶世尊时曾是优婆塞,月中八日去到寺庙听法。一天,当在僧园门口宣告闻法时,他正在自己的田地里嬉戏,听到布告后,想「要是我去洗澡会很久」,便以脏污之体进了布萨堂,在昂贵的地毯上不敬地躺下睡着了。相应部诵者则说他实是比丘,而非优婆塞。他以此业及其它的业堕地狱后,转生在伽耶池岸的夜叉胎中。他以此业剩余的异熟而丑恶,且身上有针样的体毛。如是,他由针样的体毛故,便得名\textbf{针毛}。他俩为至自己的行处,从居处出发,须臾即至,仍以来路返回,前往另一方向时,在世尊不远处经过。
\item \textbf{于是,粗夜叉……},为什么他俩这样说?粗(夜叉)见到沙门般的人后便说了,但针毛却如是主张「害怕的人就不是沙门,由假装沙门而为伪沙门」,所以思量世尊这样的「\textbf{这不是沙门,这是伪沙门}」,但乍说之后,又欲考察,便说「\textbf{很快我就知道……}」。\end{enumerate}

\textbf{于是,针毛夜叉往世尊处走去,走到后,把身体靠近世尊。然后,世尊便移开身体。于是,针毛夜叉对世尊说:「沙门!你怕我吗?」「我并不怕你,朋友!而是你的触碰粗恶。」}

Atha kho Sūcilomo yakkho yena Bhagavā ten’upasaṅkami, upasaṅkamitvā Bhagavato kāyaṃ upanāmesi. Atha kho Bhagavā kāyaṃ apanāmesi. Atha kho Sūcilomo yakkho Bhagavantaṃ etad avoca: “bhāyasi maṃ, samaṇā” ti? “Na khvāhaṃ taṃ, āvuso, bhāyāmi, api ca te samphasso pāpako” ti.

\begin{enumerate}\item \textbf{于是},即如是说了之后。从「\textbf{针毛夜叉}」到「\textbf{而是你的触碰粗恶}」之义明了,只是其中的「\textbf{身体……世尊}」为把自己的身体靠近世尊,当知如是连结。\end{enumerate}

\textbf{「沙门!我将问你问题,如果你不能向我解答,我就扰乱你的心识,撕碎你的心脏,捉住脚抛到恒河对岸去。」「朋友!我实不见在这俱有天、魔、梵、沙门婆罗门、天人的人世间,有人能扰乱我的心识、撕碎心脏、捉住脚抛到恒河对岸去的,但是,朋友!问你所愿吧!」于是,针毛夜叉以偈颂对世尊说:}

“Pañhaṃ taṃ, samaṇa, pucchissāmi, sace me na byākarissasi, cittaṃ vā te khipissāmi, hadayaṃ vā te phālessāmi, pādesu vā gahetvā pāragaṅgāya khipissāmī” ti. “Na khvāhaṃ taṃ, āvuso, passāmi sadevake loke samārake sabrahmake sassamaṇabrāhmaṇiyā pajāya sadevamanussāya yo me cittaṃ vā khipeyya hadayaṃ vā phāleyya pādesu vā gahetvā pāragaṅgāya khipeyya, api ca tvaṃ, āvuso, puccha yad ākaṅkhasī” ti. Atha kho Sūcilomo yakkho Bhagavantaṃ gāthāya ajjhabhāsi:

\begin{enumerate}\item 随后,见到世尊不害怕,他便说了「\textbf{沙门!我将问你问题……}」等。什么原因?因为他想:「他作为人,却不怕我这如是粗砺的非人之触,噫!我要问他佛陀境域内的问题!他对此肯定无法解说,随后我将如是恼乱他!」世尊听后,便说了「\textbf{朋友!我实不见……}」等。这一切当以在旷野经中所说的方法,以一切行相了知。\textbf{于是,针毛夜叉以偈颂对世尊说}了下颂。\end{enumerate}

\subsection\*{\textbf{273} {\footnotesize 〔PTS 270〕}}

\textbf{贪嗔由何为因?不乐、乐与汗毛竖立由何而生?\\}
\textbf{寻由何起(驱散)意,如同孩童驱散乌鸦?}

“Rāgo ca doso ca kutonidānā, aratī ratī lomahaṃso kutojā;\\
kuto samuṭṭhāya mano vitakkā, kumārakā dhaṅkam iv’ossajanti”. %\hfill\textcolor{gray}{\footnotesize 1}

\begin{enumerate}\item 这里,\textbf{贪嗔}即所述之法。\textbf{由何} \textit{kuto},当知即以 to 替代体格,且复合时无有间断。或者,\textbf{因},即出生、出现之义,所以「由何为因」即是说由何出生、由何出现。\textbf{不乐、乐与汗毛竖立由何而生},即如\begin{quoting}于边鄙的坐卧处,或于某某增上善法,不乐、不乐性、不喜、不愉快、不满、不安。(分别论第 856 段)\end{quoting}所分别的不乐,以及于种种五欲的乐、由令起汗毛竖立而归于「汗毛竖立」的心的恐惧,他问这三法由何而生、由何所生?
\item \textbf{由何起},即由何生起。\textbf{意},即善心。\textbf{寻},即于蛇经(第 7 颂)所说的欲寻等九种。\textbf{如同孩童驱散乌鸦},他问:好比正在嬉戏的村童,把乌鸦用绳捆住脚后驱散、抛掷,如是,不善寻由何而起,驱散善意?\end{enumerate}

\subsection\*{\textbf{274} {\footnotesize 〔PTS 271〕}}

\textbf{贪嗔由此为因,不乐、乐与汗毛竖立由此而生,\\}
\textbf{寻由此起(驱散)意,如同孩童驱散乌鸦。}

“Rāgo ca doso ca itonidānā, aratī ratī lomahaṃso itojā;\\
ito samuṭṭhāya mano vitakkā, kumārakā dhaṅkam iv’ossajanti. %\hfill\textcolor{gray}{\footnotesize 2}

\begin{enumerate}\item 于是,世尊为向他解答这些问题,说了第二颂。这里,\textbf{此}是就自体而说的。因为贪嗔以自体为因,且不乐、乐与汗毛竖立由自体而生,欲寻等不善寻仍由自体而起,驱散善意,因此为驳斥其它如「原质\footnote{原质 \textit{prakṛti/pakati}:即数论的术语,与「原人 \textit{puruṣa/purisa}」相对。}」等为原因而说「\textbf{由此为因、由此而生、由此起}」。且此中的语法当知仍以前颂所说的方法而成就。\end{enumerate}

\subsection\*{\textbf{275} {\footnotesize 〔PTS 272〕}}

\textbf{从黏腻而生,从自我而成,如同榕树的茎生,\\}
\textbf{各各纠缠于爱欲,如同藤蔓蔓延于林中。}

Snehajā attasambhūtā, nigrodhasseva khandhajā;\\
puthū visattā kāmesu, māluvā va vitatā vane. %\hfill\textcolor{gray}{\footnotesize 3}

\begin{enumerate}\item 如是解答了这些问题后,现在,为总结在「由此为因」等中所说的「由自体为因、由自体而生、由自体起」之义而说「\textbf{从黏腻而生,从自我而成}」。因为以贪为初、以寻为后的这一切,以渴爱之黏腻而生,且同样,在生时,于自我——分为五取蕴的自体之基——而成,因此说「从黏腻而生,从自我而成」。
\item 现在,为阐明此义而举譬「\textbf{如同榕树的茎生}」。这里,在茎中所生者为茎生,即气根的同义语。这说的是什么?好比榕树名为茎生的气根,在有水味之黏腻时则生,且在生时,即于此榕树的各个枝条上而成,如是,这些贪等也在有内在的渴爱之黏腻时则生,且在生时,即于此自体的各个眼等类的门、所缘、依处而成。所以,当知此即「彼等由自体为因、由自体而生、由自体起」。
\item 而这是余下一颂半的整体释义:且如是从自我而成者,彼等\textbf{各各纠缠于爱欲}。因为贪依种种五欲等,嗔依嫌恨事等,不乐等依彼彼类等等,此一切烦恼于一切处各有多种行相,依于依处、门、所缘等,于彼彼物欲如是如是纠缠、固著、附著、交织已而住。如同什么?\textbf{如同藤蔓蔓延于林中},好比林中蔓延的藤蔓,于彼彼树的枝叉等纠缠、固著、附著、交织已而住。\end{enumerate}

\subsection\*{\textbf{276} {\footnotesize 〔PTS 273〕}}

\textbf{若知晓那由何为因,他们便除遣之,夜叉!听!\\}
\textbf{他们度过这先前未度的难度的暴流,无有再有。}

Ye naṃ pajānanti yatonidānaṃ, te naṃ vinodenti suṇohi yakkha;\\
te duttaraṃ ogham imaṃ taranti, atiṇṇapubbaṃ apunabbhavāyā” ti. %\hfill\textcolor{gray}{\footnotesize 4}

\begin{enumerate}\item 如是于各类物欲纠缠的烦恼聚,若知晓那由何为因,他们便除遣之,夜叉!听!这里,\textbf{由何为因},即表示状态的中性词。以此显明什么?\textbf{若}有情如是\textbf{知晓那}烦恼聚「由何为因而生起」,则\textbf{他们便}在了知其「于为渴爱之黏腻所黏的自体而生起」后,以过患随观等修习的智火烧尽,且\textbf{除遣}、舍弃、去除此渴爱之黏腻,\textbf{夜叉}!\textbf{听}我们的这善说!如是,此中以了知自体显明苦之遍知,且以除遣渴爱之黏腻的贪等烦恼聚显明集之舍断。
\item 且若除遣此者,\textbf{他们度过这先前未度的难度的暴流,无有再有}。以此显明道之修习与灭之证得。因为若除遣此烦恼聚,他们必然修习道——没有无道之修习便除遣烦恼者——且若修习道,他们度过这以自然之智难度的欲暴流等四种暴流——因为道之修习便是度暴流。\textbf{先前未度},即先前以长久的旅途,甚至以梦,也未曾超越。\textbf{无有再有},即涅槃。
\item 如是,听到此显明四谛之颂,以善修习之慧践行「闻已持法、考察所持诸法之义」等论,此二夜叉友人即在颂的终了住于须陀洹果,皆成净喜,肤色金黄,严饰以天的庄严。\end{enumerate}

\begin{center}\vspace{1em}针毛经第五\\Sūcilomasuttaṃ pañcamaṃ.\end{center}