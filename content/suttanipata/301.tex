\chapter{大品第三}

\section{出家经}

\begin{center}Pabbajjā Sutta\end{center}\vspace{1em}

\subsection\*{\textbf{408} {\footnotesize 〔PTS 405〕}}

\textbf{我将宣扬出家:具眼者如何出家,\\}
\textbf{他如何经审视而选择了出家。}

Pabbajjaṃ kittayissāmi, yathā pabbaji cakkhumā;\\
yathā vīmaṃsamāno so, pabbajjaṃ samarocayi. %\hfill\textcolor{gray}{\footnotesize 1}

\begin{enumerate}\item 缘起为何?据说,当世尊住舍卫国时,尊者阿难想到:「舍利弗等大弟子的出家事已被宣扬,众比丘和众优婆塞都知晓,而世尊的却未被宣扬,我何不宣扬之?」他坐在祇园寺的坐处,拿了彩绘的扇子,为对众比丘宣扬世尊的出家,说了此经。
\item 这里,因为宣扬出家应宣扬如何出家,而宣扬如何出家应宣扬如何经审视而选择了出家,所以在说了「我将宣扬出家」后,说了「如何出家」等。\textbf{具眼者},即具足五眼之义。其余在初颂中自明。\end{enumerate}

\subsection\*{\textbf{409} {\footnotesize 〔PTS 406〕}}

\textbf{「这居家险迫,是尘垢之处,\\}
\textbf{「而出家闲旷」,见到如此,他便出了家。}

“Sambādho’yaṃ gharāvāso, rajassāyatanaṃ” iti;\\
“abbhokāso va pabbajjā”, iti disvāna pabbaji. %\hfill\textcolor{gray}{\footnotesize 2}

\begin{enumerate}\item 现在,为阐明「如何经审视」之义,说了此颂。\textbf{险迫},即因妻儿等的压迫与烦恼的压迫,而无有行善的机会。\textbf{尘垢之处},即贪等尘垢的发生处,如剑浮阇\footnote{剑浮阇 \textit{Kamboja} 多产马,故义注如是说。}等为马等之处。\textbf{闲旷},即相对于所说的险迫,如天空般敞开。
\item \textbf{见到如此,他便出了家},即如是为病、老、死善加激励的心审视了居家、出家的过患、利益,行了大出离,在无劣河畔以剑断发,直至须发至于适宜沙门的二指之量,在接受梵天伽吒迦罗\footnote{伽吒迦罗 \textit{Ghaṭikāra},见\textbf{中部}第 81 经。}供养的八资具后,不由任何人教授「当如是著下衣、披上衣」,而为数千生转起的自身出家习行所教,便出了家。即是说著一下衣,作一郁多罗僧,一衣置于肩,陶钵挂于肩,便决意了出家相。其余于此自明。\end{enumerate}

\subsection\*{\textbf{410} {\footnotesize 〔PTS 407〕}}

\textbf{出家后,他以身避免了恶业,\\}
\textbf{舍弃了语恶行,清净了活命。}

Pabbajitvāna kāyena, pāpakammaṃ vivajjayi;\\
vacīduccaritaṃ hitvā, ājīvaṃ parisodhayi. %\hfill\textcolor{gray}{\footnotesize 3}

\begin{enumerate}\item 如是宣扬了世尊的出家,随后,为阐明离了无劣河畔的出家行道与趣于至上,说了「出家后,他以身」等一切。这里,\textbf{以身避免了恶业},即避免了三种身恶行。\textbf{语恶行},即四种语恶行。\textbf{清净了活命},即舍弃了邪命,唯转起正命。\end{enumerate}

\subsection\*{\textbf{411} {\footnotesize 〔PTS 408〕}}

\textbf{佛陀到了王舍城,摩竭陀的山栏城,\\}
\textbf{去行乞食,显现出高贵相。}

Agamā Rājagahaṃ Buddho, Magadhānaṃ Giribbajaṃ;\\
piṇḍāya abhihāresi, ākiṇṇavaralakkhaṇo. %\hfill\textcolor{gray}{\footnotesize 4}

\begin{enumerate}\item 如是清净了以活命为第八的戒后,\textbf{佛陀}从无劣河畔以七天\textbf{到了}三十由旬外的\textbf{王舍城}。这里,虽然当到了王舍城时还未成佛,但由作了佛陀的宿行,故可以这样说,如同「于此国王出生,于此得了王权」等世间习俗之语。\textbf{摩竭陀},即是说摩竭陀国的城市。\textbf{山栏城}也是它的名称。因为它在名为般择婆、耆阇崛、吠婆罗、仙人吞、方广的五山之中,如同牛栏而住,所以被称为「山栏城」。
\item \textbf{去行乞食},即为食物在此城游行。据说,他站在城门,便想:「如果我告诉国王频婆娑罗自己到来,『净饭的儿子,名为悉达多的王子来了』,他会为我带来许多资具。然而,宣告后接受资具,这不适合出家的我,噫!我去行乞。」便披了天赐的粪扫衣,拿了陶钵,从东门入城,逐家行乞。因此尊者阿难说「去行乞食」。\textbf{显现出高贵相},如内在的高贵相流露于身体,或者,即广大的高贵相。因为广大也被称为「显现」。如说:\begin{quoting}显现粗鲁之人,如被涂抹了尿布。(本生第 6:118 颂)\end{quoting}即「广大粗鲁」之义。\end{enumerate}

\subsection\*{\textbf{412} {\footnotesize 〔PTS 409〕}}

\textbf{站在高楼上的频婆娑罗看到了他,\\}
\textbf{见到相的具足,便说了此义:}

Tam addasā Bimbisāro, pāsādasmiṃ patiṭṭhito;\\
disvā lakkhaṇasampannaṃ, imam atthaṃ abhāsatha. %\hfill\textcolor{gray}{\footnotesize 5}

\begin{enumerate}\item \textbf{看到了他},随后,据说,之前七天在城中庆祝了节日,而在这天便鸣鼓:「节日已过,应从事工作。」于是,大众聚集在朝廷,国王也想「我将安排工作」,打开窗子,当检阅军队时,便看到去行乞食的大士。因此尊者阿难说「站在高楼上的频婆娑罗看到了他」。\textbf{便说了此义},即对大臣们说了此义。\end{enumerate}

\subsection\*{\textbf{413} {\footnotesize 〔PTS 410〕}}

\textbf{「诸君!请注意他!英俊、硕大、明净,\\}
\textbf{「具足行,眼见一寻之地,}

“Imaṃ bhonto nisāmetha, abhirūpo brahā suci;\\
caraṇena ca sampanno, yugamattañ ca pekkhati. %\hfill\textcolor{gray}{\footnotesize 6}

\begin{enumerate}\item 现在,为显示对那些大臣所说之义,说了此颂。这里,\textbf{他},即国王显示菩萨。\textbf{诸君},即称呼大臣们。\textbf{注意},即看。\textbf{英俊},即身材、肢体可观。\textbf{硕大},即具足高大、宽阔。\textbf{明净},即肤色遍净。\end{enumerate}

\subsection\*{\textbf{414} {\footnotesize 〔PTS 411〕}}

\textbf{「目光下视,具念,他不像来自卑贱的家族,\\}
\textbf{「让王使们速去:比丘将去何方?」}

Okkhittacakkhu satimā, nāyaṃ nīcakulā-m-iva;\\
rājadūtā’bhidhāvantu, kuhiṃ bhikkhu gamissati”. %\hfill\textcolor{gray}{\footnotesize 7}

\begin{enumerate}\item \textbf{不像来自卑贱的家族},即不像从卑贱的家族出家之义。\textbf{比丘将去何方},即以此意趣而说——让王使们速速前去了知:这比丘将去何方?现今将住何处?因为我们欲见。\end{enumerate}

\subsection\*{\textbf{415} {\footnotesize 〔PTS 412〕}}

\textbf{那些受遣的王使们便从后紧随:\\}
\textbf{「比丘将去何方?将住何处?」}

Te pesitā rājadūtā, piṭṭhito anubandhisuṃ;\\
“kuhiṃ gamissati bhikkhu, kattha vāso bhavissati”. %\hfill\textcolor{gray}{\footnotesize 8}

\subsection\*{\textbf{416} {\footnotesize 〔PTS 413〕}}

\textbf{次第行乞,守护根门,善加防护,\\}
\textbf{正知、忆念,他很快便装满了钵。}

Sapadānaṃ caramāno, guttadvāro susaṃvuto;\\
khippaṃ pattaṃ apūresi, sampajāno paṭissato. %\hfill\textcolor{gray}{\footnotesize 9}

\begin{enumerate}\item 以目光下视\textbf{守护根门},以具念\textbf{善加防护}。或者,以具念守护根门,以端正地持僧伽梨善加防护。\textbf{他很快便装满了钵},他由正知及忆念故,不获取更多,想「至此已足」,以意乐之盛满,很快便装满了钵。\end{enumerate}

\subsection\*{\textbf{417} {\footnotesize 〔PTS 414〕}}

\textbf{行乞之后,牟尼出了城,\\}
\textbf{去往般择婆山:他将住在此。}

Piṇḍacāraṃ caritvāna, nikkhamma nagarā muni;\\
Paṇḍavaṃ abhihāresi, ettha vāso bhavissati. %\hfill\textcolor{gray}{\footnotesize 10}

\begin{enumerate}\item \textbf{牟尼},由为寂默而行道故,即便未证得牟尼性,也称为牟尼。或者,以世间的习俗,因为世人也说未成就寂默的出家人为牟尼。\textbf{去往般择婆山},即上到此山。据说,他问人们:「在此城内,出家人住在何处?」于是,人们便告诉他:「在般择婆上,面朝东方的山坡。」所以,他便去往这般择婆山,作如是想「他将住在此」。\end{enumerate}

\subsection\*{\textbf{418} {\footnotesize 〔PTS 415〕}}

\textbf{见到已进入住处,三个使者便就近坐下,\\}
\textbf{而其中一个返回,告知国王道:}

Disvāna vāsūpagataṃ, tayo dūtā upāvisuṃ;\\
tesu eko va āgantvā, rājino paṭivedayi. %\hfill\textcolor{gray}{\footnotesize 11}

\subsection\*{\textbf{419} {\footnotesize 〔PTS 416〕}}

\textbf{「大王!这比丘在般择婆山的东坡,\\}
\textbf{「在山洞中,如老虎、公牛、狮子而坐。」}

“Esa bhikkhu mahārāja, Paṇḍavassa puratthato;\\
nisinno byaggh’usabho va, sīho va girigabbhare”. %\hfill\textcolor{gray}{\footnotesize 12}

\begin{enumerate}\item \textbf{在山洞中,如老虎、公牛、狮子},即在山的洞窟中,如老虎、如公牛、如狮子而坐之义。因为这三者最胜,已离怖畏与恐惧而坐在山洞中,所以如是举譬。\end{enumerate}

\subsection\*{\textbf{420} {\footnotesize 〔PTS 417〕}}

\textbf{听了使者的话,刹帝利便以祥瑞的车乘\\}
\textbf{匆忙出发,前往般择婆山。}

Sutvāna dūtavacanaṃ, bhaddayānena khattiyo;\\
taramānarūpo niyyāsi, yena Paṇḍavapabbato. %\hfill\textcolor{gray}{\footnotesize 13}

\begin{enumerate}\item \textbf{祥瑞的车乘},即象、马、车、轿等最上的车乘。\end{enumerate}

\subsection\*{\textbf{421} {\footnotesize 〔PTS 418〕}}

\textbf{驶尽车道后,刹帝利从车上下来,\\}
\textbf{徒步前往,靠近了他便坐下。}

Sa yānabhūmiṃ yāyitvā, yānā oruyha khattiyo;\\
pattiko upasaṅkamma, āsajja naṃ upāvisi. %\hfill\textcolor{gray}{\footnotesize 14}

\begin{enumerate}\item \textbf{驶尽车道后},即行至象、马等车乘所能行之地。\textbf{靠近},即到达、行至附近之义。\end{enumerate}

\subsection\*{\textbf{422} {\footnotesize 〔PTS 419〕}}

\textbf{落了坐,国王随即问候,\\}
\textbf{寒暄之后,他说了此义:}

Nisajja rājā sammodi, kathaṃ sāraṇīyaṃ tato;\\
kathaṃ so vītisāretvā, imam atthaṃ abhāsatha. %\hfill\textcolor{gray}{\footnotesize 15}

\subsection\*{\textbf{423} {\footnotesize 〔PTS 420〕}}

\textbf{「你青春、年少,正是初出的青年,\\}
\textbf{「具足肤色、高大,像刹帝利出身。}

“Yuvā ca daharo cāsi, paṭhamuppattiko susu;\\
vaṇṇārohena sampanno, jātimā viya khattiyo. %\hfill\textcolor{gray}{\footnotesize 16}

\begin{enumerate}\item \textbf{青春},具足青春。\textbf{年少},即以出生而言幼弱。\textbf{正是初出的青年},即此二者殊胜:你青春,即当青春时,为初出的青年,以初出的青春之力而奋起\footnote{此句按 PTS 本译出。},你年少,即当年少时,便看似有力的青年。\end{enumerate}

\subsection\*{\textbf{424} {\footnotesize 〔PTS 421〕}}

\textbf{「当闪耀在军队的前列,在象群之前,\\}
\textbf{「我赐财富,你当享用!当被问及,请告知出身!」}

Sobhayanto anīkaggaṃ, nāgasaṅghapurakkhato;\\
dadāmi bhoge bhuñjassu, jātiṃ akkhāhi pucchito”. %\hfill\textcolor{gray}{\footnotesize 17}

\begin{enumerate}\item \textbf{军队的前列},即部队、军阵之首。\textbf{我赐财富,你当享用},当知此中如是连接:即在鸯伽与摩竭陀,但凡你想要的财富,我都会赐予你,当你闪耀在军队的前列,在象群之前,你当享用!\end{enumerate}

\subsection\*{\textbf{425} {\footnotesize 〔PTS 422〕}}

\textbf{「前方的国土,国王!在雪山山坡,\\}
\textbf{「具足财产和勇气,在㤭萨罗世居。}

“Ujuṃ janapado rāja, Himavantassa passato;\\
dhanaviriyena sampanno, Kosalesu niketino. %\hfill\textcolor{gray}{\footnotesize 18}

\begin{enumerate}\item \textbf{前方的国土}等,据说,经如是说「我赐财富,你当享用!当被问及,请告知出身」,大人便想:「如果我希求王位,四大王天等都会以各自的王位邀请我,或者住在家中就能作转轮王,然而这国王却不知晓而如是说,噫!我来让他知晓!」便抬起臂膀,指着来时的方向,说了「前方的国土」等。
\item 这里,说\textbf{在雪山山坡},即显示谷物的成就,无有欠缺。因为依靠雪山,即便生长在岩石间隙的大娑罗树,也以五种增长\footnote{以五种增长:据菩提比丘注 1325,见\textbf{增支部}第 5:40 经,即以枝叶、以皮、以外皮、以边材、以心材。}而增长,何况播种在田间的谷物?说\textbf{具足财产和勇气\footnote{勇气 \textit{viriya}:通常译作「精进」,这里义注用「英雄、勇者 \textit{vīra}」来解释,故作此译。}},即显示不欠缺七宝,及其为其他国王不可思议的由英雄人物所治理之状。说\textbf{在㤭萨罗世居},即相对于新王之相。因为新王不可说是「世居」,而对从初时开始,依传承在此国土居住者,则被称为「世居」。且净饭王便是如此,就此而说「在㤭萨罗世居」,以此显明传承而来的财富的成就。\end{enumerate}

\subsection\*{\textbf{426} {\footnotesize 〔PTS 423〕}}

\textbf{「族名为太阳,生名为释迦,\\}
\textbf{「我从这家族出家,不愿求爱欲。}

Ādiccā nāma gottena, Sākiyā nāma jātiyā;\\
tamhā kulā pabbajito’mhi, na kāme abhipatthayaṃ. %\hfill\textcolor{gray}{\footnotesize 19}

\begin{enumerate}\item 至此已显明自身财富的成就,先以「族名为太阳,生名为释迦」告知了出身的成就,再反驳国王所说的「我赐财富,你当享用」,说了「我从这家族出家,不愿求爱欲」。此中的意趣为:要是我愿求爱欲,就不会丢开如此具足财产和勇气、八万二千英雄人物辈出的家族而出家。\end{enumerate}

\subsection\*{\textbf{427} {\footnotesize 〔PTS 424〕}}

\textbf{「见到爱欲的过患,视出离为安稳,\\}
\textbf{「我将向至上前行,于此我意喜乐。」}

Kāmesv ādīnavaṃ disvā, nekkhammaṃ daṭṭhu khemato;\\
padhānāya gamissāmi, ettha me rañjatī mano” ti. %\hfill\textcolor{gray}{\footnotesize 20}

\begin{enumerate}\item 如是反驳了国王的话语,随后,为显示自己的出家之因,说了「见到爱欲的过患,视出离为安稳」,此应与「我(从这家族)出家」相连。这里,\textbf{视},即见到。此中其余及前几颂中凡未考察者,当知由彼等一切之义自明故,不予考察。
\item 如是说了自己的出家之因,为告知国王欲向至上前行,说了「我将向至上\footnote{至上 \textit{padhāna} 的常见义项是精勤,亦有主要、首要之义,这里从义注的解释而作「至上」。}前行,于此我意喜乐」。其义为:因为,大王!我视出离为安稳而出家,所以,为希求这第一义的出离、涅槃之不死、以一切法中最上之义的至上,我将为了至上前行,我意喜乐于此至上,非于爱欲。
\item 如是说已,据说,国王便对菩萨说:「尊者!先前有所耳闻:据说,净饭王的儿子,悉达多王子在见到四征兆\footnote{四征兆:即老人、病人、死尸、出家人。}后出了家,将会成佛。尊者!我见到你的决心,如是净喜:你一定会圆满佛性!善哉!尊者!证得佛性后,请先进入我的领地!」\end{enumerate}

\begin{center}\vspace{1em}出家经第一\\Pabbajjāsuttaṃ paṭhamaṃ.\end{center}