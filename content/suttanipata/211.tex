\section{罗睺罗经}

\begin{center}Rāhula Sutta\end{center}\vspace{1em}

\subsection\*{\textbf{338} {\footnotesize 〔PTS 335〕}}

\textbf{「是否因为经常共住,你不轻视智者?\\}
\textbf{「为人类持炬者,是否受到你的尊敬?」}

“Kacci abhiṇhasaṃvāsā, nāvajānāsi paṇḍitaṃ;\\
ukkādhāro manussānaṃ, kacci apacito tayā”. %\hfill\textcolor{gray}{\footnotesize 1}

\begin{enumerate}\item 缘起为何?世尊于正等正觉现等觉已,从菩提座渐次前往迦毗罗卫,在那里,罗睺罗童子向他乞求遗产,说「沙门!请给我遗产」,世尊便命舍利弗长老「教罗睺罗童子出家」,这些都应以(律藏)犍度义注(大品第 105 段)中所说之法把握。如是出家已,等到了年龄,罗睺罗童子便仍随舍利弗长老受具戒,而大目犍连长老则为其羯磨阿阇黎。世尊想「这童子具足出身等,莫让他因出身、族姓、家族、容貌之美等而起慢、起㤭」,从少年时起,乃至未证得圣地,为作教诫便经常说此经。所以在经的最后说「如此,世尊常常以这些偈颂教诫尊者罗睺罗」。
\item 这里,第一颂的略义为:「罗睺罗!你是否因经常共住,不以出身等的任一事轻蔑智者?由持慧灯、法开示之灯的\textbf{为人类持炬者,是否受到你的尊敬},是否始终受到你的恭敬?」这是就尊者舍利弗说的。\end{enumerate}

\subsection\*{\textbf{339} {\footnotesize 〔PTS 336〕}}

\textbf{「因为经常共住,我不轻视智者,\\}
\textbf{「为人类持炬者,始终受到我的尊敬。」}

“Nāhaṃ abhiṇhasaṃvāsā, avajānāmi paṇḍitaṃ;\\
ukkādhāro manussānaṃ, niccaṃ apacito mayā”.\footnote{PTS 本在此颂下标有「序颂 \textit{Vatthugāthā}」。} %\hfill\textcolor{gray}{\footnotesize 2}

\begin{enumerate}\item 如是说已,尊者罗睺罗为显示「世尊!我并不如下人一般,因共住而起慢、起㤭」,便说了此答颂。其义自明。\end{enumerate}

\subsection\*{\textbf{340} {\footnotesize 〔PTS 337〕}}

\textbf{「舍弃了可爱、悦意的种种五欲,\\}
\textbf{「因信出家已,当成得尽苦边者!}

“Pañca kāmaguṇe hitvā, piyarūpe manorame;\\
saddhāya gharā nikkhamma, dukkhass’antakaro bhava. %\hfill\textcolor{gray}{\footnotesize 3}

\begin{enumerate}\item 随后,世尊为对其更作教诫,说了余下的几颂。这里,因为种种五欲对有情可爱、生喜,极为有情所希望、希求,且能愉悦其意,而尊者罗睺罗舍弃彼等,因信出家,不因国王、盗贼的召唤,不因债务、怖畏的逼迫,不受活命的驱使,所以世尊以「\textbf{舍弃了可爱、悦意的种种五欲,因信出家已}」鼓舞他后,为敦促与此出离相适的行道而说「\textbf{当成得尽苦边者}」。
\item 这里,若问:「难道尊者不是希求遗产,受迫而出家?那为什么世尊还说『因信出家』?」由信解出离故。因为此尊者长时信解出离,在见到莲花上正等正觉的儿子,名为优波离婆多的沙弥后,作为名为僧佉的龙王,于七日间作布施,希求如是之相,从此以后,具足愿求、具足志向,圆满了十万劫的波罗蜜,投生于最后有。而世尊如是了知其信解出离,因为此智为如来(十)力之一种,所以便说「因信出家」。或者,此中的意趣为:已长时因信出家,现在当成得尽苦边者!\end{enumerate}

\subsection\*{\textbf{341} {\footnotesize 〔PTS 338〕}}

\textbf{「应亲近善知识,与边鄙、远离、\\}
\textbf{「少愦闹的住处!于饮食当知量!}

Mitte bhajassu kalyāṇe, pantañ ca sayanāsanaṃ;\\
vivittaṃ appanigghosaṃ, mattaññū hohi bhojane. %\hfill\textcolor{gray}{\footnotesize 4}

\begin{enumerate}\item 现在,为他从头开始显示得尽流转之苦的行道,先说了此颂。这里,以戒等为主导者,名为善知识,亲近彼等能随戒等增长,如依于雪山,大沙罗树随根等增长一般。因此说「\textbf{应亲近善知识}」。\textbf{边鄙、远离、少愦闹的住处},即若住处为边鄙——即在远处,远离——即少聚集,少愦闹——即于彼处以鹿、猪等声而生起林野想,则应亲近这样的住处。\textbf{于饮食当知量},即应知量,应知晓接受之量与受用之量之义。
\item 这里,以知晓接受之量,当所施少量、施主欲少施时,唯应取少量,当所施少量、施主欲多施时,唯应取少量,而当所施较多、施主欲少施时,唯应取少量,当所施较多、施主欲多施时,唯应知晓自身之力而取,且世尊唯赞叹适量。以知晓受用之量,则应经如理作意——如孩子之肉、如给车轴涂油——而受用饮食。\end{enumerate}

\subsection\*{\textbf{342} {\footnotesize 〔PTS 339〕}}

\textbf{「衣服、乞食,以及资具、住处,\\}
\textbf{「莫于这些生渴爱!莫再来世间!}

Cīvare piṇḍapāte ca, paccaye sayanāsane;\\
etesu taṇhaṃ mākāsi, mā lokaṃ punar āgami. %\hfill\textcolor{gray}{\footnotesize 5}

\begin{enumerate}\item 如是,以上颂敦促了作为梵行之资助的亲近善知识,以及激励了以坐卧处、饮食为首的资具受用之遍净戒,现在,因为以对衣服等的渴爱而成邪命,所以在遮止此后,为激励活命遍净戒而说此颂。
\item 这里,\textbf{资具},即疾病的资具。\textbf{这些},即这些衣服等比丘生起渴爱的四事。\textbf{莫生渴爱},即当以「为遮蔽羞处等,此四资具乃为久病之人的救治,乃为这如朽屋般极弱身躯之支持」等方法见到过患时,莫生渴爱!即是说应不令生、不令起而住。什么原因?\textbf{莫再来世间}!因为于这些生渴爱者,便为渴爱所牵引,再次来到此世间。你莫于这些生渴爱,当如是时,便不会再来此世间。
\item 如是说已,尊者罗睺罗想「世尊对我说『莫于衣服生渴爱』」,便受持了与衣服相关的二头陀支——粪扫衣支、三衣支,想「世尊对我说『莫于乞食生渴爱』」,便受持了与乞食相关的五头陀支——常乞食支、次第乞食支、一座食支、一钵食支、时后不食支,想「世尊对我说『莫于坐卧处生渴爱』」,便受持了与坐卧处相关的六头陀支——阿练若住支、露地住支、树下住支、随处住支、冢间住支、常坐不卧支,想「世尊对我说『莫于疾病的资具生渴爱』」,便于一切资具以随所得、随力、随适合等三种满足而知足,如易语的族姓子,顺从地接受此教诫。\end{enumerate}

\subsection\*{\textbf{343} {\footnotesize 〔PTS 340〕}}

\textbf{「防护于波罗提木叉与五根,\\}
\textbf{「应作身至念,应多多厌逆!}

Saṃvuto pātimokkhasmiṃ, indriyesu ca pañcasu;\\
sati kāyagatā ty atthu, nibbidābahulo bhava. %\hfill\textcolor{gray}{\footnotesize 6}

\begin{enumerate}\item 如是,世尊在对尊者罗睺罗激励了活命遍净戒后,现在,为激励其余的戒及止观,说了此颂。这里,在「\textbf{防护于波罗提木叉}」中,文本省略了「应当 \textit{bhavassu}」,或者当知与末句的「应 \textit{bhava}」相连,第二句也同样。如是,便以此二语激励了别解脱律仪戒与根律仪戒。且此中,以显然故只说了五根,但由相而言,当知第六(意根)也被提及。
\item \textbf{应作身至念},即如是,你既住立于四遍净戒,应作、应为四界差别、四种正知\footnote{四种正知:即有益正知、适宜正知、行处正知、无痴正知。}、入出息念、食厌想修习等类的身至念\footnote{这里的身至念是就广义而说的,见\textbf{清净道论}·说随念业处品第 42~43 段。},即应修习此之义。\textbf{应多多厌逆},即应多多不满于轮回之流转,于一切世间成不喜想\footnote{于一切世间成不喜想:据菩提比丘注 1201,此修习见\textbf{增支部}第 10:60 Girimānanda 经:「凡于世间的取著,心的住著、执持、随眠,比丘舍弃彼等,无取而住。」}之义。\end{enumerate}

\subsection\*{\textbf{344} {\footnotesize 〔PTS 341〕}}

\textbf{「回避净的、伴有贪染的相!\\}
\textbf{「于不净修习心,一境、善等持!}

Nimittaṃ parivajjehi, subhaṃ rāgūpasañhitaṃ;\\
asubhāya cittaṃ bhāvehi, ekaggaṃ susamāhitaṃ. %\hfill\textcolor{gray}{\footnotesize 7}

\begin{enumerate}\item 至此,已显明抉择分的近行地,现在,为显明安止地而说此颂。这里,\textbf{相},即可成为贪染之处的净相。因此在其后特别说明「\textbf{净的、伴有贪染的}」。\textbf{回避},即以不作意而遍舍。\textbf{于不净修习心},即应如是修习心,以便让对有识或无识之身的不净修习得以成就。\textbf{一境、善等持},即以近行定而一境,以安止定而善等持。其义为:应如是修习之,让你的心成为这般。\end{enumerate}

\subsection\*{\textbf{345} {\footnotesize 〔PTS 342〕}}

\textbf{「且应修习无相!舍弃慢的随眠!\\}
\textbf{「随后,因慢的止息,寂静得行。」}

Animittañ ca bhāvehi, mānānusayam ujjaha;\\
tato mānābhisamayā, upasanto carissasī” ti. %\hfill\textcolor{gray}{\footnotesize 8}

\begin{enumerate}\item 如是,已显明安止地,为显明毗婆舍那而说此颂。这里,\textbf{且应修习无相},即是说,如是以抉择分的定等持之心应修习毗婆舍那。因为毗婆舍那以「无常随观智从常相解脱,即无相解脱」等方法,以无取于贪相等,得称为无相。如说:\begin{quoting}朋友!我由不作意一切相,具足无相的心定而住。朋友!我以此住而住,而有随相之识。(相应部第 40:9 经)\end{quoting}
\item \textbf{舍弃慢的随眠},即以此无相修习而得无常想已,如\begin{quoting}弥醯!无常想者得起无我想,无我想者得成就我慢的去除。(增支部第 9:3 经)\end{quoting}等,渐次舍弃、舍断、遍舍慢的随眠之义。\textbf{随后,因慢的止息,寂静得行},即然后,如是因以圣道而得慢的止息、尽、灭、舍弃、舍遣而寂静、寂灭、清凉、无有一切恼患与热恼,直至以无余依涅槃界般涅槃,将以空、无相、无愿中的任一果定而行、而住,即以阿罗汉为顶点完成了开示。\end{enumerate}

\textbf{如此,世尊常常以这些偈颂教诫尊者罗睺罗。}

Itthaṃ sudaṃ Bhagavā āyasmantaṃ Rāhulaṃ imāhi gāthāhi abhiṇhaṃ ovadatī ti.

\begin{enumerate}\item 此后的「如此,世尊……」等是结集者的话语。这里,\textbf{如此},即是说如是。其余之义于此自明。且尊者罗睺罗经如是教诫,当解脱所需成熟之法至于成熟时,便在「小教诫罗睺罗经\footnote{即\textbf{中部}第 147 经。}」的终了,与数千天人一起,住于阿罗汉。\end{enumerate}

\begin{center}\vspace{1em}罗睺罗经第十一\\Rāhulasuttaṃ ekādasamaṃ.\end{center}