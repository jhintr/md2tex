\section{船经}

\begin{center}Nāvā Sutta\end{center}\vspace{1em}

\begin{enumerate}\item \textbf{法经} \textit{Dhammasutta},也称为\textbf{船经}。缘起为何?此经是就尊者舍利弗长老而说。这于此是略说,而详说当知从二上首弟子的缘起开始。此即是:据说,世尊未出世时,二上首弟子圆满了一阿僧祇又十万劫的波罗蜜后,转生到天界。其中的第一个殁后——在王舍城不远处,有名为优波提舍村的婆罗门采地,那里,拥有五百六十俱胝财产的家主有名为色质的妻子——便在其胎内获取了结生。第二个——也在其不远处,有名为拘律陀村的婆罗门采地,那里,拥有同样财产的家主有名为目犍连尼的妻子——便在其胎内在同一天获取了结生。
\item 如是,他俩在同一天获取结生并出胎。仍在同一天,他俩中的一个由生在优波提舍村得名\textbf{优波提舍},一个由生在拘律陀村得名\textbf{拘律陀}。这俩朋友一起玩着泥巴,便渐渐长大,且一一有五百学童随从。他俩去庭园或河津时,便带上随从前往,其一以五百金轿,其二以五百纯种马车。
\item 那时,王舍城时时有山顶祭庆。在晡时时分的城中,住在整个鸯伽、摩揭陀的知名刹帝利孩童等聚集后,坐在善加敷设的床椅上,观赏祭庆的盛况。于是,这俩朋友与其随从一起便去到那里,坐在设好的坐处。随后,优波提舍观赏着祭庆的盛况,见到大众聚集后,便想:「这么些人,未及百年就将死去。」死亡好像已经朝他而来,停在额边一般。拘律陀也如是。于种种品类舞者的舞蹈,他俩的心竟不屑一顾,而是生起了悚惧。
\item 于是,当祭庆结束,二友人与自身的随从随着众人离开时,拘律陀便问优波提舍:「兄弟!为什么看了舞蹈等,你却毫不欢喜呢?」他告知其经过后,也同样反问了他。他也告知其自身的经过,说道:「来!兄弟!我们出家后,寻求不死吧!」优波提舍同意道:「善哉!兄弟!」
\item 随后,二人舍弃家财后,再次来到王舍城。尔时,名为先阇那的游行者在王舍城定居。他俩在他跟前与五百学童一起出了家,仅几天就学会了三吠陀和所有游行者的教义。他俩在考察其典籍的初、中、后时,没看到后分,便问老师:「这些典籍有初、中,却没有后分——即没有较此典籍所能证的更上应证。」他也说:「我也未见到这样的后分。」他俩便说:「那么,我们去寻求这后分。」老师便对他俩说:「你们随所乐去寻求吧!」
\item 如是,他俩为其许可,彷徨于寻求不死,便在整个阎浮提知名。刹帝利智者等,当被他俩提问时,不能作更多的解答。但当说到「优波提舍、拘律陀」时,却没人会说「他俩是谁?我们不认识」,他俩便如是著名。
\item 当他俩如是游行于寻求不死时,我们的世尊出世后,转起了无上法轮,渐次到达王舍城。而这些游行者游行了整个阎浮提后,别提不死,连后分之问的解答也没得到,再次到了王舍城。于是,尊者马胜晨朝著了下衣等等,直至他俩出家的一切,当以「出家犍度」中所提及之法详述\footnote{即\textbf{律藏}·大品第 60 段及以下。}。
\item 如是出家已,在这二友人中,尊者舍利弗以半月证得了声闻波罗蜜智。他当与马胜长老一起住于同一寺庙时,给侍完世尊,便立即前去给侍长老,尊重道:「此尊者是我的启蒙老师,我依于彼而了知世尊的教法。」而当未与马胜长老一起住于同一寺庙时,便观察了长老所住的方向,五体投地礼拜,合掌礼敬。
\item 某些比丘见后,便发起谈论:「舍利弗已是上首弟子,还礼敬方位,到今天,我说,还未舍弃婆罗门见。」于是,世尊以天耳界听闻了这闲谈,便显明自身仍坐于设好的最上佛座,告诸比丘:「诸比丘!你们现今为何谈论而共坐?」他们便宣说了经过。随后,世尊说:「诸比丘!舍利弗并未礼敬方位,而是礼拜、礼敬、礼遇依于彼而了知教法的自己的老师,诸比丘!舍利弗是尊师者。」为开示法,便对聚集在此者说了此经。\end{enumerate}

\subsection\*{\textbf{319} {\footnotesize 〔PTS 316〕}}

\textbf{人对能从其了知法者,应如诸天对因陀般尊敬,\\}
\textbf{那受到尊敬、对其心生净喜的多闻者会阐明法。}

Yasmā hi dhammaṃ puriso vijaññā, Indaṃ va naṃ devatā pūjayeyya;\\
so pūjito tasmi pasannacitto, bahussuto pātukaroti dhammaṃ. %\hfill\textcolor{gray}{\footnotesize 1}

\begin{enumerate}\item 这里,\textbf{人对能从其了知法者},即人对能从其了知、知晓、明了三藏之类的圣典之法,或听闻圣典后应证的九出世间之类的通达之法的补特伽罗。文本也作 yassa,语义相同\footnote{这是说颂中的 yasmā 也作 yassa。}。\textbf{应如诸天对因陀般尊敬},即好比诸天在两处天界尊敬诸天因陀帝释,如是,此人对此补特伽罗,应按时起身,尽解鞋带等的一切义务而尊敬、恭敬、尊重之。
\item 什么原因?\textbf{那受到尊敬……阐明法},那老师受到如是尊敬,便对弟子心生净喜,多闻于圣典、通达者会以开示及行道来阐明、开示圣典之法及听闻开示后如教诫应证的通达之法,或者,以开示阐明圣典之法,以譬喻阐明自身所证的通达之法。\end{enumerate}

\subsection\*{\textbf{320} {\footnotesize 〔PTS 317〕}}

\textbf{智者留心、留意于此,践行着法之随法,\\}
\textbf{不放逸者结交此等,成为有智、明辨、微妙。}

Tad aṭṭhikatvāna nisamma dhīro, dhammānudhammaṃ paṭipajjamāno;\\
viññū vibhāvī nipuṇo ca hoti, yo tādisaṃ bhajati appamatto. %\hfill\textcolor{gray}{\footnotesize 2}

\begin{enumerate}\item \textbf{智者留心、留意于此},即堪能理解的智者,留心、听闻于为如是净喜的老师所阐明的法。\textbf{践行着法之随法},即修习着由随顺出世间法而作为随法的毗婆舍那。\textbf{成为有智、明辨、微妙},以证得被称为裁量的慧而为有智,阐明后能令他人明了,以堪能令人了知而为明辨,以通达最微细之义而为微妙。\textbf{不放逸者结交此等},即不放逸者于彼更净喜已,结交此等如先前所说品类的多闻者。\end{enumerate}

\subsection\*{\textbf{321} {\footnotesize 〔PTS 318〕}}

\textbf{亲近着下劣、愚痴、不达义利、妒忌者,\\}
\textbf{于此便不明了法,未度疑惑,趣近死亡。}

Khuddañ ca bālaṃ upasevamāno, anāgatatthañ ca usūyakañ ca;\\
idh’eva dhammaṃ avibhāvayitvā, avitiṇṇakaṅkho maraṇaṃ upeti. %\hfill\textcolor{gray}{\footnotesize 3}

\begin{enumerate}\item 如是赞叹亲近智者之师后,现在为非难亲近愚痴之师,便说了此颂。这里,\textbf{下劣},即具足下劣的身业等。由无有慧而为\textbf{愚痴}。\textbf{不达义利},即未证得圣典、通达之义利。\textbf{妒忌},即因嫉妒心而不能忍受弟子的进步。词句上余皆自明。
\item 而旨趣上,若老师有众多衣等的利养,却不能给予弟子以衣等,且于法施也不能一言于无常、苦、无我,由具足下劣等法而为下劣、愚痴、不达义利、妒忌,亲近此老师,以\begin{quoting}若人以固沙草叶捆缚臭鱼……(如是语第 76 经)\end{quoting}所说之法,则自己也成愚痴。所以,当知其义为:于此教法,连任何少量的圣典之法或通达之法也不明了、不知晓,则不得度脱对于法的疑惑,趣近死亡。\end{enumerate}

\subsection\*{\textbf{322} {\footnotesize 〔PTS 319〕}}

\textbf{好比人落入了河中,水流洪大湍急,\\}
\textbf{他且随流漂没,如何能将他人渡过?}

Yathā naro āpagam otaritvā, mahodakaṃ salilaṃ sīghasotaṃ;\\
so vuyhamāno anusotagāmī, kiṃ so pare sakkhati tārayetuṃ. %\hfill\textcolor{gray}{\footnotesize 4}

\begin{enumerate}\item 现在,为令此义明了,便说了以下二颂。这里,\textbf{水流},即四处流溢,即是说广大。文本也作 saritaṃ,语义相同。\textbf{湍急},即运来可运者,即是说具有冲力。此中的 \textbf{kiṃ so},由已经以「他漂没 \textit{so vuyhamāno}」中的 so 字表示此人,故(第四句的)so 字只是不变词,即是说 kiṃ su。好比说:\begin{quoting}我必将不存,我必将亡失。\footnote{引文中的「必 \textit{so}」即表示 so 字作不变词解,但所引中部的正文为 vinassissāmi nāmassu, nassu nāma bhavissāmī。}(中部·蛇喻经第 242 段)\end{quoting}\end{enumerate}

\subsection\*{\textbf{323} {\footnotesize 〔PTS 320〕}}

\textbf{同样,不明了法,不倾听多闻者的语义,\\}
\textbf{自身且无知、未度疑惑,如何能使他人理解?}

Tath’eva dhammaṃ avibhāvayitvā, bahussutānaṃ anisāmay’atthaṃ;\\
sayaṃ ajānaṃ avitiṇṇakaṅkho, kiṃ so pare sakkhati nijjhapetuṃ. %\hfill\textcolor{gray}{\footnotesize 5}

\begin{enumerate}\item \textbf{法},即先前所说的二种法。词句上余皆自明。
\item 而旨趣上,好比任何人落入了所说品类的河中,为此河水随流漂没,随流而行,如何能将希求彼岸的他人带至彼岸?文本也作 sakkati。同样,于两种法,自己不能以慧明了,且未在多闻者跟前倾听其义,由自身的不明了及无知、不倾听而未度疑惑,如何能使他人理解、得见?如是当知此中之义。且\begin{quoting}纯陀!他自己尚且沉溺……(中部·减损经第 87 段)\end{quoting}等经句在此可被忆念。\end{enumerate}

\subsection\*{\textbf{324} {\footnotesize 〔PTS 321〕}}

\textbf{又如登上坚固的船,具备浆舵,\\}
\textbf{知晓此理、善巧、具慧,他于此能渡脱旁人。}

Yathā pi nāvaṃ daḷham āruhitvā, phiyen’arittena samaṅgibhūto;\\
so tāraye tattha bahū pi aññe, tatrūpayaññū kusalo mutīmā. %\hfill\textcolor{gray}{\footnotesize 6}

\begin{enumerate}\item 如是,说了令「愚人因亲近愚人,不能使他人理解」之义明了的譬喻,现在,为令「在『不放逸者结交此等』中所说的智者能令他人理解」之义明了,便说了以下二颂。
\item 这里,\textbf{浆},即勺状的板。\textbf{舵},即竹杖。\textbf{于此},即于此船。\textbf{知晓此理},即以知晓此船驶往、驶离等的理法、以道的引领而知晓理法。以业已修学、以极善巧的手艺为\textbf{善巧}。以能对治生起的祸害为\textbf{具慧}。\end{enumerate}

\subsection\*{\textbf{325} {\footnotesize 〔PTS 322〕}}

\textbf{如是,若通达诸明、修己、多闻、具不动之法,\\}
\textbf{他既遍知,故能使具足倾听及基础的他人理解。}

Evam pi yo vedagu bhāvitatto, bahussuto hoti avedhadhammo;\\
so kho pare nijjhapaye pajānaṃ, sotāvadhānūpanisūpapanne. %\hfill\textcolor{gray}{\footnotesize 7}

\begin{enumerate}\item \textbf{通达诸明},即由被称为吠陀\footnote{吠陀、明的原文均为 veda,一为音译,一为意译。}的四道智而到达。\textbf{修己},即以此道的修习,心已修习。\textbf{多闻},即如先前所述。\textbf{具不动之法},即自性不可为八世间法所动摇。\textbf{具足倾听及基础},即具足倾听及道果的近依。其余的句义明了。旨趣与连结亦能以前法可知,不饶详繁。\end{enumerate}

\subsection\*{\textbf{326} {\footnotesize 〔PTS 323〕}}

\textbf{所以,一定要结交既有智、又多闻的善人,\\}
\textbf{知晓义利并践行,了知法,他便获得快乐。}

Tasmā have sappurisaṃ bhajetha, medhāvinañ c’eva bahussutañ ca;\\
aññāya atthaṃ paṭipajjamāno, viññātadhammo sa sukhaṃ labhethā ti. %\hfill\textcolor{gray}{\footnotesize 8}

\begin{enumerate}\item 如是,说了令「智者能令他人理解」之义明了的譬喻,为敦促此亲近智者,便说了这末颂。
\item 于此,其略义为:因为具足近依者能以亲近智者圆满殊胜,\textbf{所以,一定要结交善人}。应结交什么样的善人?\textbf{既有智、又多闻},以慧的成就为有智,以所说品类的二种闻为多闻。因为结交此等者,\textbf{知晓}其所说之法的\textbf{义利},且如是了知已,如教诫而\textbf{践行},以此行道,依于通达而\textbf{了知法},\textbf{他便获得}、证得、圆满道、果、涅槃等类的出世间的\textbf{快乐},即以阿罗汉为顶点完成了开示。\end{enumerate}

\begin{center}\vspace{1em}船经第八\\Nāvāsuttaṃ aṭṭhamaṃ.\end{center}