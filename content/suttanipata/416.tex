\section{舍利弗经}

\begin{center}Sāriputta Sutta\end{center}\vspace{1em}

\begin{enumerate}\item \textbf{舍利弗经},也称\textbf{长老问经}\footnote{Dharmānanda Kosambi 认为此经即阿育王 Calcutta-Bairāṭ 敕令中的「\textbf{优波提舍问} \textit{Upatisapasine}」,优波提舍者,舍利弗之名也。}。缘起为何?此经的缘起为:王舍城的商人首先得了旃檀木材,将此旃檀木材所制的钵竖立于空中,尊者宾头卢·婆罗豆婆遮以神变拿到了钵。基于此事,制止弟子们作神变。外道们欲与世尊比试神变。作神变。世尊去往舍卫国。外道跟随。在舍卫国,波斯匿王至佛所。瘤芒果树显现。为战胜外道,遮止四众热衷于作神变。作双神变。世尊作神变后,去往三十三天的居处,于此三月间开示法。为尊者大目犍连\footnote{PTS 本作「尊者阿那律」。}祈请,从天界降至僧迦尸城。且在上述事件期间,详述了诸多本生,直至为一万轮围的天人供养,世尊由摩尼所造的阶梯降至僧迦尸城,站在阶梯的台阶上说:\begin{quoting}智者修禅定,喜出家寂静,\\正念正觉者,天人所敬爱。(法句·佛陀品第 181 颂)\end{quoting}即如此法句颂所说。
\item 而尊者舍利弗率先顶礼了站在阶梯台阶上的世尊,随后是莲花色比丘尼,然后是众人。于此,世尊便想:「在此集会中,目犍连以神变为上首而知名,阿那律以天眼,富楼那以论法,但此集会尚不知舍利弗以何功德为上首,我何不以智慧功德显扬舍利弗?」于是,问了长老问题。长老对世尊所问的凡夫之问、有学之问及无学之问等的一切皆作了解答。此时,人们便知其「以智慧为上首」。于是,世尊说「舍利弗不唯现在以智慧为上首,在过去也以智慧为上首」,引出了本生。
\item 在过去,有一千多个仙人以林中的根果为食,住在山麓。他们的老师生了病,他们便去护持。最长的弟子说「我去拿合适的药,你们当不放逸地护持老师」,去往人境。当他尚未归来,老师就死了。众弟子觉得他「现在将要死去」,便问他关于等至的事。他就空无边处等至说「什么也没有」,众弟子便认为「老师什么成就也没有」。于是,最长的弟子取了药回来,看到他已死去,便说:「你们问了老师些什么吗?」「唯!我们问了,他说『什么也没有』,老师什么也没证得。」「老师说『什么也没有』,便是证得了空无边处,老师应受恭敬。」即说颂:\begin{quoting}即便聚集千余,众无慧者将悲泣百年,\\而一有慧之人即更胜,他能了知所说之义。(本生第 1:99 颂)\end{quoting}
\item 当世尊说此本生时,尊者舍利弗为了与自己共住的五百比丘的义利,为问适宜的坐卧处、行处、戒禁等,以「我此前未见过」的赞颂为首,说了八颂。世尊为解答其义,说了随后的几颂。\end{enumerate}

\subsection\*{\textbf{962} {\footnotesize 〔PTS 955〕}}

\textbf{「我此前未见过,」尊者舍利弗说,「也未从任何人听闻,\\}
\textbf{「如是言语善妙的大师,从兜率天来的众主。}

“Na me diṭṭho ito pubbe, \textit{(icc āyasmā Sāriputto)} na suto uda kassaci;\\
evaṃ vagguvado satthā, Tusitā gaṇi-m-āgato. %\hfill\textcolor{gray}{\footnotesize 1}

\begin{enumerate}\item 这里,\textbf{此前},即在此降至僧迦尸城之前。\textbf{从兜率天来的众主},即从兜率天众下堕而至母胎故为从兜率天来,由为众人老师故为众主。或者,从以知足之义被称为兜率的天界来作众主,或来作知足的阿罗汉的众主。\end{enumerate}

\subsection\*{\textbf{963} {\footnotesize 〔PTS 956〕}}

\textbf{「如同为俱有天的世间所见,具眼者\\}
\textbf{「驱散了一切暗冥,他独自证得喜乐。}

Sadevakassa lokassa, yathā dissati cakkhumā;\\
sabbaṃ tamaṃ vinodetvā, eko va ratim ajjhagā. %\hfill\textcolor{gray}{\footnotesize 2}

\begin{enumerate}\item 第二颂中,\textbf{如同为俱有天的世间所见},即好比为俱有天的世间,亦为人们所见。或者,如同所见,即真实、无颠倒地被见。\textbf{具眼者},即最上眼者。\textbf{独自},即因被称为出家等而独自。\textbf{喜乐},即出离之喜乐等。\end{enumerate}

\subsection\*{\textbf{964} {\footnotesize 〔PTS 957〕}}

\textbf{「为了众多在此被束缚者,我带着问题,前往\\}
\textbf{「这无依止、如如、无诡诈、来作众主的佛陀。\footnote{此颂的译文调换了上下两行。}}

Taṃ Buddhaṃ asitaṃ tādiṃ, akuhaṃ gaṇim āgataṃ;\\
bahūnam idha baddhānaṃ, atthi pañhena āgamaṃ. %\hfill\textcolor{gray}{\footnotesize 3}

\begin{enumerate}\item 第三颂中,\textbf{众多在此被束缚者},即众多在此的刹帝利等学生。因为学生的行为仰赖老师故,被称为「被束缚」。\textbf{我带着问题前往},即我希求问题而来,或为希求者以问题而来,或存在问题而来。\end{enumerate}

\subsection\*{\textbf{965} {\footnotesize 〔PTS 958〕}}

\textbf{「对于生起嫌厌的比丘,亲近空旷的坐处,\\}
\textbf{「或树下、塚间,或群山的洞窟,}

Bhikkhuno vijigucchato, bhajato rittam āsanaṃ;\\
rukkhamūlaṃ susānaṃ vā, pabbatānaṃ guhāsu vā. %\hfill\textcolor{gray}{\footnotesize 4}

\begin{enumerate}\item 第四颂中,\textbf{生起嫌厌},即为生等逼迫。\textbf{空旷的坐处},即远离的床椅。\textbf{群山的洞窟},即应连结为:亲近在群山洞窟的空旷坐处。\end{enumerate}

\subsection\*{\textbf{966} {\footnotesize 〔PTS 959〕}}

\textbf{「种种的卧处,那里有多么可怕?\\}
\textbf{「在无声的坐卧处,比丘为何能不颤抖?}

Uccāvacesu sayanesu, kīvanto tattha bheravā;\\
yehi bhikkhu na vedheyya, nigghose sayanāsane. %\hfill\textcolor{gray}{\footnotesize 5}

\begin{enumerate}\item 第五颂中,\textbf{种种},即卑劣、高贵。\textbf{卧处},即寺庙等的坐卧处。\textbf{那里有多么可怕},即那里有多少怖畏之因。文本也作 kuvanto,其义为「唧唧啾啾」,但不能与前后相连。\end{enumerate}

\subsection\*{\textbf{967} {\footnotesize 〔PTS 960〕}}

\textbf{「对于前往未至之方者,世间有多少危难\\}
\textbf{「比丘应在边鄙的坐卧处去克服?}

Katī parissayā loke, gacchato agataṃ disaṃ;\\
ye bhikkhu abhisambhave, pantamhi sayanāsane. %\hfill\textcolor{gray}{\footnotesize 6}

\begin{enumerate}\item 第六颂中,\textbf{多少危难},即多少灾祸。\textbf{未至之方},即涅槃。因为它由先前未曾到达故为未至,同样,由可被指明故为方向,因此说是「未至之方」。\textbf{克服},即征服。\textbf{边鄙},即边界。\end{enumerate}

\subsection\*{\textbf{968} {\footnotesize 〔PTS 961〕}}

\textbf{「他的言路应如何?他在此的行处应如何?\\}
\textbf{「自励的比丘的戒禁应如何?}

Ky āssa byappathayo assu, ky āss’assu idha gocarā;\\
kāni sīlabbatān’āssu, pahitattassa bhikkhuno. %\hfill\textcolor{gray}{\footnotesize 7}

\begin{enumerate}\item 第七颂中,\textbf{他的言路应如何},即他的言语应该是怎样的。\end{enumerate}

\subsection\*{\textbf{969} {\footnotesize 〔PTS 962〕}}

\textbf{「他专一、贤明、具念,受持何学\\}
\textbf{「能驱除自身的垢秽,如同锻工之于银?」}

Kaṃ so sikkhaṃ samādāya, ekodi nipako sato;\\
kammāro rajatasseva, niddhame malam attano”. %\hfill\textcolor{gray}{\footnotesize 8}

\begin{enumerate}\item 第八颂中,\textbf{专一、贤明},即一境心、智者。\end{enumerate}

\subsection\*{\textbf{970} {\footnotesize 〔PTS 963〕}}

\textbf{「舍利弗!对于生起嫌厌、」世尊说,「若亲近空旷的坐卧处、\\}
\textbf{「欲求等觉者,我将对你说安乐与随法,如同了知者。}

“Vijigucchamānassa yad idaṃ phāsu, \textit{(Sāriputtā ti Bhagavā)} rittāsanaṃ sayanaṃ sevato ce;\\
sambodhikāmassa yathānudhammaṃ, taṃ te pavakkhāmi yathā pajānaṃ. %\hfill\textcolor{gray}{\footnotesize 9}

\begin{enumerate}\item 如是,尊者舍利弗以三颂称赞了世尊,以五颂为了五百学生的义利问了坐卧处、行处、戒禁等,世尊为显扬此义,以「对于生起嫌厌」等方法开始解答。这里,先说初颂之义为:\textbf{对于}因生等\textbf{生起嫌厌,若亲近空旷的坐卧处、欲求等觉}的比丘,舍利弗!\textbf{我将对你说安乐与随法},即安住与随法,\textbf{如同了知者},如同了知者所说,我即如是说。\end{enumerate}

\subsection\*{\textbf{971} {\footnotesize 〔PTS 964〕}}

\textbf{「坚定、具念、具制限而行的比丘不应怖畏五种怖畏,\\}
\textbf{「虻、蛾、蛇、人的攻击与四足者。}

Pañcannaṃ dhīro bhayānaṃ na bhāye, bhikkhu sato sapariyantacārī;\\
ḍaṃsādhipātānaṃ sarīsapānaṃ, manussaphassānaṃ catuppadānaṃ. %\hfill\textcolor{gray}{\footnotesize 10}

\begin{enumerate}\item 第二颂中,\textbf{具制限而行},即于戒等四种制限\footnote{四种制限:\textbf{义释}说即别解脱律仪、根律仪、饮食知量、常事醒觉等。}而行。\textbf{虻、蛾},即褐蝇与其它的蝇。因为其它的蝇从各处飞过后叮咬,所以称为蛾\footnote{这是语源上的解释:飞过 \textit{adhipatitvā} 和 蛾 \textit{adhipāta} 有字面的联系。}。\textbf{人的攻击},即盗贼等的攻击。\end{enumerate}

\subsection\*{\textbf{972} {\footnotesize 〔PTS 965〕}}

\textbf{「不应惊怖于异法者,即便见到了其中诸多可怕,\\}
\textbf{「然后,他应克服其它危难,追随着善。}

Paradhammikānam pi na santaseyya, disvā pi tesaṃ bahubheravāni;\\
athāparāni abhisambhaveyya, parissayāni kusalānuesī. %\hfill\textcolor{gray}{\footnotesize 11}

\begin{enumerate}\item 第三颂中,\textbf{异法者},即除七同法\footnote{七同法:即比丘、比丘尼、式叉摩那、沙弥、沙弥尼、优婆塞、优婆夷。}的所有外人。\textbf{追随着善},即追求着善法。\end{enumerate}

\subsection\*{\textbf{973} {\footnotesize 〔PTS 966〕}}

\textbf{「为疾患、饥饿所感,应忍耐寒冷、炎热,\\}
\textbf{「他为种种这些所感,无家者勇猛精进已,应努力作为。}

Ātaṅkaphassena khudāya phuṭṭho, sītaṃ athuṇhaṃ adhivāsayeyya;\\
so tehi phuṭṭho bahudhā anoko, viriyaṃ parakkamma daḷhaṃ kareyya. %\hfill\textcolor{gray}{\footnotesize 12}

\begin{enumerate}\item 第四颂中,\textbf{疾患},即疾病。\textbf{他为种种这些所感},即当他为这些多种疾患的行相所感时。\textbf{无家者},即无行作识等的空间者。\end{enumerate}

\subsection\*{\textbf{974} {\footnotesize 〔PTS 967〕}}

\textbf{「不应盗窃,不应妄语,对弱者、强者应以慈遍满,\\}
\textbf{「凡所了知的意的扰动,应以『这是黑分』驱散之。}

Theyyaṃ na kāre na musā bhaṇeyya, mettāya phasse tasathāvarāni;\\
yad āvilattaṃ manaso vijaññā, ‘kaṇhassa pakkho’ ti vinodayeyya. %\hfill\textcolor{gray}{\footnotesize 13}

\begin{enumerate}\item 如是,解答了以「对于生起嫌厌的比丘」等三颂所问之义,现在,为解答以「他的言路应如何」等方法所问者,说了「不应盗窃」等。这里,\textbf{凡所了知的意的扰动},即凡所了知的心的扰动,对这一切\textbf{应以「这是黑分」驱散之}。\end{enumerate}

\subsection\*{\textbf{975} {\footnotesize 〔PTS 968〕}}

\textbf{「不应沦于忿怒、傲慢的控制,应掘断其根而立,\\}
\textbf{「然后,克服喜爱或不喜爱时,应完全克服。}

Kodhātimānassa vasaṃ na gacche, mūlam pi tesaṃ palikhañña tiṭṭhe;\\
atha ppiyaṃ vā pana appiyaṃ vā, addhā bhavanto abhisambhaveyya. %\hfill\textcolor{gray}{\footnotesize 14}

\begin{enumerate}\item \textbf{应掘断其根而立},即应掘断此忿怒、傲慢的无明等根而立。\textbf{克服……时,应完全克服},即如是克服喜爱或不喜爱时,唯应完全克服,即非于此有所懈怠地努力之意\footnote{原文 bhavanto 费解,义注释作「克服 \textit{abhibhavanto}」,Norman 和菩提比丘都支持与前词相连,作 addhā-bhavanto,认为是 addhabhavanto 由诗律而拖长了元音,则下半颂的译文可作「当主宰时,他能克服喜爱或不喜爱」,这里的译文仍从义注。}。\end{enumerate}

\subsection\*{\textbf{976} {\footnotesize 〔PTS 969〕}}

\textbf{「以智慧为先导,善妙、欢喜者应镇伏这些危难,\\}
\textbf{「他应忍耐边鄙卧处的不喜,他应忍耐四种悲法:}

Paññaṃ purakkhatvā kalyāṇapīti, vikkhambhaye tāni parissayāni;\\
aratiṃ sahetha sayanamhi pante, caturo sahetha paridevadhamme. %\hfill\textcolor{gray}{\footnotesize 15}

\begin{enumerate}\item \textbf{他应忍耐四种悲法},即他应忍耐下颂所说的可悲之法。\end{enumerate}

\subsection\*{\textbf{977} {\footnotesize 〔PTS 970〕}}

\textbf{「『我将吃什么,或我将在哪吃,我睡得很苦,今天将在哪睡』,\\}
\textbf{「有学、无居所而行者,应调伏这些悲寻。}

‘Kiṃ sū asissāmi kuva vā asissaṃ, dukkhaṃ vata settha kv ajja sessaṃ’;\\
ete vitakke paridevaneyye, vinayetha sekho aniketacārī. %\hfill\textcolor{gray}{\footnotesize 16}

\begin{enumerate}\item \textbf{我睡得很苦,今天将在哪睡},即我昨夜睡得很苦,今晚我将在哪睡。\textbf{这些寻},即这二种乞食相关、二种坐卧处相关的四寻。\textbf{无居所而行},即无障碍而行、离渴爱而行。\end{enumerate}

\subsection\*{\textbf{978} {\footnotesize 〔PTS 971〕}}

\textbf{「适时地获得了食物与衣服,为了于此知足,他应知量,\\}
\textbf{「他守护此等,在村中自制而行,即便被激怒也不应说恶语。}

Annañ ca laddhā vasanañ ca kāle, mattaṃ so jaññā idha tosanatthaṃ;\\
so tesu gutto yatacāri gāme, rusito pi vācaṃ pharusaṃ na vajjā. %\hfill\textcolor{gray}{\footnotesize 17}

\begin{enumerate}\item \textbf{适时地},意即在乞食时或在衣时,如法、公正地\textbf{获得了}被称为乞食的\textbf{食物}与被称为衣的\textbf{衣服}。\textbf{他应知量},即在接受和受用时,他应知量。\textbf{于此},即于此教法,或这只是不变词。\textbf{为了知足},即为了满足,即是说为了此义而知量。\textbf{他守护此等},即此比丘守护这些资具。\textbf{自制而行},即自制而住,即是说守护威仪、守护身语意门等。文本也作 yaticārī,亦是此义。\textbf{被激怒},即是说被刺激。\end{enumerate}

\subsection\*{\textbf{979} {\footnotesize 〔PTS 972〕}}

\textbf{「目光下视,且不游步,应从事禅那,常事醒觉,\\}
\textbf{「等持于舍,他应断绝寻、意乐、恶作。}

Okkhittacakkhu na ca pādalolo, jhānānuyutto bahujāgar’assa;\\
upekkham ārabbha samāhitatto, takkāsayaṃ kukkucciy’ūpachinde. %\hfill\textcolor{gray}{\footnotesize 18}

\begin{enumerate}\item \textbf{从事禅那},即以生起未生起的及习行已生起的而从事于禅那。\textbf{等持于舍},即增长第四禅的舍已,等持其心。\textbf{寻、意乐、恶作},即欲寻等的寻,以及此寻之意乐的欲想等,以及手的不安等的恶作。\end{enumerate}

\subsection\*{\textbf{980} {\footnotesize 〔PTS 973〕}}

\textbf{「当被言语呵责,具念者应欢喜,应破除对同梵行的荒秽,\\}
\textbf{「他应说善语,而不过分,不应存心于闲谈。}

Cudito vacībhi satimābhinande, sabrahmacārīsu khilaṃ pabhinde;\\
vācaṃ pamuñce kusalaṃ nātivelaṃ, janavādadhammāya na cetayeyya. %\hfill\textcolor{gray}{\footnotesize 19}

\begin{enumerate}\item \textbf{当被言语呵责,具念者应欢喜},即当被亲教师等的言语呵责时,应具念而欢喜这呵责。\textbf{应说善语},即应说能生起智的言语。\textbf{过分},即不说过分的言语,即超过时间的边界及戒的边界者。\textbf{存心},即令心生起。\end{enumerate}

\subsection\*{\textbf{981} {\footnotesize 〔PTS 974〕}}

\textbf{「然后,在世间有五尘,对彼等具念,为了调伏而修学,\\}
\textbf{「能忍耐对于色、声、味、香、触的贪染。}

Athāparaṃ pañca rajāni loke, yesaṃ satīmā vinayāya sikkhe;\\
rūpesu saddesu atho rasesu, gandhesu phassesu sahetha rāgaṃ. %\hfill\textcolor{gray}{\footnotesize 20}

\begin{enumerate}\item \textbf{五尘},即色贪等的五尘。\textbf{对彼等具念,为了调伏而修学},即对彼等建立念已,为了调伏而修学三学。因为如是修学者,\textbf{能忍耐对于色、声、味、香、触的贪染},而非他人。\end{enumerate}

\subsection\*{\textbf{982} {\footnotesize 〔PTS 975〕}}

\textbf{「应调伏对这些法的欲,比丘具念、善解脱心,\\}
\textbf{「他时常正当地审视着法,成就专一,他便能破除暗冥。」}

Etesu dhammesu vineyya chandaṃ, bhikkhu satimā suvimuttacitto;\\
kālena so sammā dhammaṃ parivīmaṃsamāno, ekodibhūto vihane tamaṃ so” ti. %\hfill\textcolor{gray}{\footnotesize 21}

\begin{enumerate}\item 随后,他为了调伏彼等而渐次修学,「应调伏……」。这里,\textbf{这些},即色等。\textbf{他时常正当地审视着法},即这比丘在以「当心掉举时,即是三摩地之时」等方法所说之时,以无常等方法审视着一切有为法。\textbf{成就专一,他便能破除暗冥},即他一境心,便能破除一切愚痴等的暗冥,于此更无疑惑。其余一切处皆自明。
\item 如是,世尊以阿罗汉为顶点完成了开示。当开示终了,五百比丘即证阿罗汉,且三十俱胝之数的天、人得了法的现观。\end{enumerate}

\begin{center}\vspace{1em}舍利弗经第十六\\Sāriputtasuttaṃ soḷasamaṃ.\end{center}

\begin{center}\vspace{1em}八颂品第四\\Aṭṭhakavaggo catuttho.\end{center}

\textbf{其总颂曰:}

\begin{quoting}爱欲、洞窟、恶意,清净、最上与老,\\弥勒、般修罗,摩根提与前分离,\\争辩与二阵,还有迅速,\\最好的执杖经,以长老问为十六,\\以上这些经,都是八颂品者。\end{quoting}

Tass’uddānaṃ –

\begin{quoting}Kāmaṃ Guhañ ca Duṭṭhā ca, Suddhañ ca Paramā Jarā;\\Metteyyo ca Pasūro ca, Māgaṇḍi Purābhedanaṃ;\\Kalahaṃ dve ca Byūhāni, punad eva Tuvaṭṭakaṃ;\\Attadaṇḍavaraṃ suttaṃ, Therapuṭṭhena soḷasa;\\iti etāni suttāni, sabbān’Aṭṭhakavaggikā ti.\end{quoting}

%\begin{flushright}乙巳重阳二稿\end{flushright}