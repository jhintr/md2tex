\section{舍利弗经}

\begin{center}Sāriputta Sutta\end{center}\vspace{1em}

\begin{enumerate}\item 经题亦作\textbf{长老问经}。此经的缘起。王舍城的富翁在得到旃檀木材后,将以此旃檀木材制成的钵竖立于空中,宾头卢·婆罗豆婆遮以神通拿到了钵,基于此事,(世尊)制止弟子们作神通。外道们欲与世尊比试神变。作神变。世尊去往舍卫国。外道跟随。在舍卫国,波斯匿王至佛所,Gaṇḍamba 树显现。为战胜外道,遮止四众作神变。作双神变。世尊作神变后,去往三十三天的居处,于此三月间开示法。由大目犍连尊者的祈请,从天界降至僧迦尸城 \textit{Saṅkassa}。在上述事件期间,详述了众多本生,直至为一万轮围的天人供养,世尊从摩尼所造的阶梯降至僧迦尸城后,站在阶梯的台阶上,说了法句(佛陀品第 181 颂):\begin{quoting}智者修禅定,喜出家寂静,\\正念正觉者,天人所敬爱。\end{quoting}
\item 舍利弗尊者率先礼敬了站在阶梯台阶上的世尊,随后是莲花色比丘尼 \textit{Uppalavaṇṇā},然后是众人。世尊想「在此集会中,目犍连以神通为上首而知名,阿那律以天眼,富楼那以论法,但此集会尚不知舍利弗以何功德为上首,我何不以智慧功德显扬舍利弗」。于是,问了长老问题。长老对世尊所问的凡夫之问、有学之问及无学之问等的一切皆作了解答。此时,人们便知道是「以智慧为上首」。于是,世尊说「舍利弗不唯现在以智慧为上首,在过去也以智慧为上首」,引出了本生。
\item 在过去,有一千多个仙人以林中的根果为食,住在山脚下。他们的老师生了病,而有侍奉的义务。最年长的弟子说「我去拿合适的药,你们当不放逸地侍奉老师」,去往人境。在他尚未归来时,老师便已死了。众弟子想他「现在将要死去」,便问他关于等至的事。他以空无边处等至而说「什么也没有」。众弟子便认为「老师什么成就也没有」。于是,最年长的弟子取了药回来,看到他已死去,便说「你们问了老师什么」,「唯!我们问了,他说『什么也没有』,老师什么也没证得」。「老师说『什么也没有』,便是证得了空无边处,老师应当受到恭敬」,即说颂:\begin{quoting}即便聚集千余,众无慧者将悲泣百年,\\而一有慧之人即更胜,他能了知所说之义。\end{quoting}当世尊说此本生时,舍利弗尊者为了与自己共住的五百比丘的义利,以此赞颂之颂为首,为问适宜的坐卧处、行处、戒禁等而说了八颂。\end{enumerate}

\begin{itemize}\item 案,Dharmānanda Kosambi 认为此经即阿育王 Calcutta-Bairāṭ 敕令中的\textbf{优波提舍问} \textit{Upatisapasine},优波提舍者,舍利弗之名也。\end{itemize}

\subsection\*{\textbf{962} {\footnotesize 〔PTS 955〕}}

\textbf{「我此前未见过,」尊者舍利弗说,「也未从任何人听闻,\\}
\textbf{「如是言语善妙的大师,从兜率天而来的众主。}

“Na me diṭṭho ito pubbe, \textit{(icc āyasmā Sāriputto)} na suto uda kassaci;\\
evaṃ vagguvado satthā, Tusitā gaṇi-m āgato. %\hfill\textcolor{gray}{\footnotesize 1}

\subsection\*{\textbf{963} {\footnotesize 〔PTS 956〕}}

\textbf{「好比具眼者,为俱有天的世间所见,\\}
\textbf{「驱散了一切暗冥,他独自证得喜乐。}

Sadevakassa lokassa, yathā dissati cakkhumā;\\
sabbaṃ tamaṃ vinodetvā, eko va ratim ajjhagā. %\hfill\textcolor{gray}{\footnotesize 2}

\subsection\*{\textbf{964} {\footnotesize 〔PTS 957〕}}

\textbf{「为了众多于此被束缚者,我带着问题,前往\\}
\textbf{「这无所依、如如、无诡诈、来作众主的佛陀。}

Taṃ Buddhaṃ asitaṃ tādiṃ, akuhaṃ gaṇim āgataṃ;\\
bahūnam idha baddhānaṃ, atthi pañhena āgamaṃ. %\hfill\textcolor{gray}{\footnotesize 3}

\begin{itemize}\item 案,译文的顺序为 c-d-a-b。\end{itemize}

\subsection\*{\textbf{965} {\footnotesize 〔PTS 958〕}}

\textbf{「对于生起厌离的比丘,亲近空旷的坐处,\\}
\textbf{「或树下、塚间,或山中的洞窟,}

Bhikkhuno vijigucchato, bhajato rittam āsanaṃ;\\
rukkhamūlaṃ susānaṃ vā, pabbatānaṃ guhāsu vā. %\hfill\textcolor{gray}{\footnotesize 4}

\subsection\*{\textbf{966} {\footnotesize 〔PTS 959〕}}

\textbf{「种种的卧处,那里有多么可怕?\\}
\textbf{「在无声的坐卧处,比丘为何能无动摇?}

Uccāvacesu sayanesu, kīvanto tattha bheravā;\\
yehi bhikkhu na vedheyya, nigghose sayanāsane. %\hfill\textcolor{gray}{\footnotesize 5}

\subsection\*{\textbf{967} {\footnotesize 〔PTS 960〕}}

\textbf{「对于前往未至之方者,世间有多少危难\\}
\textbf{「比丘应在边鄙的坐卧处去克服?}

Katī parissayā loke, gacchato agataṃ disaṃ;\\
ye bhikkhu abhisambhave, pantamhi sayanāsane. %\hfill\textcolor{gray}{\footnotesize 6}

\begin{enumerate}\item \textbf{未至之方},即涅槃,由先前未曾到达,故为未至,由可被指明,故为方向,所以这(涅槃)被称为未至之方。\end{enumerate}

\subsection\*{\textbf{968} {\footnotesize 〔PTS 961〕}}

\textbf{「他的言路应如何?他的行处于此应如何?\\}
\textbf{「自励的比丘的戒禁应如何?}

Ky-āssa byappathayo assu, ky-āss’assu idha gocarā;\\
kāni sīlabbatān’āssu, pahitattassa bhikkhuno. %\hfill\textcolor{gray}{\footnotesize 7}

\subsection\*{\textbf{969} {\footnotesize 〔PTS 962〕}}

\textbf{「他专一、贤明、具念,受持何学\\}
\textbf{「能驱除自身的垢秽,如同锻工之于银?」}

Kaṃ so sikkhaṃ samādāya, ekodi nipako sato;\\
kammāro rajatasseva, niddhame malam attano”. %\hfill\textcolor{gray}{\footnotesize 8}

\subsection\*{\textbf{970} {\footnotesize 〔PTS 963〕}}

\textbf{「舍利弗!对于生起厌离、」世尊说,「若亲近空旷的坐卧处、\\}
\textbf{「欲求等觉者,我将对你说安乐与随法,如同了知者那样。}

“Vijigucchamānassa yad-idaṃ phāsu, \textit{(Sāriputtā ti Bhagavā)} rittāsanaṃ sayanaṃ sevato ce;\\
sambodhikāmassa yathānudhammaṃ, taṃ te pavakkhāmi yathā pajānaṃ. %\hfill\textcolor{gray}{\footnotesize 9}

\begin{itemize}\item 案,\textbf{随法} \textit{yathānudhammaṃ},Norman 将其作为\textbf{欲求等觉}的状语,即「如法地欲求等觉」,菩提比丘将其作为\textbf{说}的状语,即「如法地说」,这里从义注所说,即将其与\textbf{安乐}并列,作为\textbf{说}的宾语。\end{itemize}

\subsection\*{\textbf{971} {\footnotesize 〔PTS 964〕}}

\textbf{「坚定、具念、具制限而行的比丘不应怖畏五种怖畏,\\}
\textbf{「虻、蛾、蛇、人的攻击与四足者。}

Pañcannaṃ dhīro bhayānaṃ na bhāye, bhikkhu sato sapariyantacārī;\\
Ḍaṃsādhipātānaṃ sarīsapānaṃ, manussaphassānaṃ catuppadānaṃ. %\hfill\textcolor{gray}{\footnotesize 10}

\begin{enumerate}\item \textbf{具制限而行},即于戒等四种制限而行。\textbf{人的攻击},即盗贼等的攻击。\end{enumerate}

\begin{itemize}\item 案,义释说\textbf{四种制限},即别解脱律仪、根律仪、饮食知量、常事醒觉等。\end{itemize}

\subsection\*{\textbf{972} {\footnotesize 〔PTS 965〕}}

\textbf{「不应惊怖于持异法者,即便见到了其中诸多可怕,\\}
\textbf{「然后,他应克服其它危难,追随着善。}

Paradhammikānam pi na santaseyya, disvā pi tesaṃ bahubheravāni;\\
athāparāni abhisambhaveyya, parissayāni kusalānuesī. %\hfill\textcolor{gray}{\footnotesize 11}

\begin{enumerate}\item \textbf{持异法者},即除七同法外的所有外道。\end{enumerate}

\begin{itemize}\item 菩提比丘:\textbf{七同法},即比丘、比丘尼、式叉摩那、沙弥、沙弥尼、优婆塞、优婆夷。\end{itemize}

\subsection\*{\textbf{973} {\footnotesize 〔PTS 966〕}}

\textbf{「为病患、饥饿所感,应忍耐寒冷、炎热,\\}
\textbf{「他为种种这些所感,无家者努力精进已,应坚定地作为。}

Ātaṅkaphassena khudāya phuṭṭho, sītaṃ athuṇhaṃ adhivāsayeyya;\\
so tehi phuṭṭho bahudhā anoko, viriyaṃ parakkamma daḷhaṃ kareyya. %\hfill\textcolor{gray}{\footnotesize 12}

\subsection\*{\textbf{974} {\footnotesize 〔PTS 967〕}}

\textbf{「不应盗窃,不应妄语,对弱者、强者应以慈遍满,\\}
\textbf{「凡所了知的意的扰动,应以『这是黑分』驱散之。}

Theyyaṃ na kāre na musā bhaṇeyya, mettāya phasse tasathāvarāni;\\
yad-āvilattaṃ manaso vijaññā, ‘kaṇhassa pakkho’ ti vinodayeyya. %\hfill\textcolor{gray}{\footnotesize 13}

\subsection\*{\textbf{975} {\footnotesize 〔PTS 968〕}}

\textbf{「不应沦于忿怒、傲慢的控制,应掘其根已而立,\\}
\textbf{「然后,克服着喜爱或不喜爱,应完全克服。}

Kodhātimānassa vasaṃ na gacche, mūlam pi tesaṃ palikhañña tiṭṭhe;\\
atha ppiyaṃ vā pana appiyaṃ vā, addhā bhavanto abhisambhaveyya. %\hfill\textcolor{gray}{\footnotesize 14}

\begin{itemize}\item 案,\textbf{克服着} \textit{bhavanto} 费解,义注释作 \textit{abhibhavanto},Norman 和菩提比丘都支持与前词相连,作 \textit{addhā-bhavanto},认为是 \textit{addhabhavanto} 由诗律而拖长了元音,但其义词典也不载,这里仍从义注。\end{itemize}

\subsection\*{\textbf{976} {\footnotesize 〔PTS 969〕}}

\textbf{「以智慧为先导,善妙、欢喜者应镇伏这些危难,\\}
\textbf{「他应忍耐边鄙卧处的不喜,他应忍耐四种悲法:}

Paññaṃ purakkhatvā kalyāṇapīti, vikkhambhaye tāni parissayāni;\\
aratiṃ sahetha sayanamhi pante, caturo sahetha paridevadhamme. %\hfill\textcolor{gray}{\footnotesize 15}

\begin{enumerate}\item \textbf{四种悲法},即下颂所说的可悲之法。\end{enumerate}

\subsection\*{\textbf{977} {\footnotesize 〔PTS 970〕}}

\textbf{「『我将吃些什么,或我将在哪里吃,我睡得很苦,今天将在哪里睡』,\\}
\textbf{「有学、无居所而行者,应调伏这些悲寻。}

‘Kiṃ sū asissāmi kuva vā asissaṃ, dukkhaṃ vata settha kv-ajja sessaṃ’;\\
ete vitakke paridevaneyye, vinayetha sekho aniketacārī. %\hfill\textcolor{gray}{\footnotesize 16}

\begin{enumerate}\item \textbf{无居所而行},即无障碍而行、无渴爱而行。\end{enumerate}

\subsection\*{\textbf{978} {\footnotesize 〔PTS 971〕}}

\textbf{「适时地获得了食物与衣服,为了于此知足,他应知量,\\}
\textbf{「他守护此等,在村中自制而行,即便被激怒也不应说恶语。}

Annañ ca laddhā vasanañ ca kāle, mattaṃ so jaññā idha tosanatthaṃ;\\
so tesu gutto yatacāri gāme, rusito pi vācaṃ pharusaṃ na vajjā. %\hfill\textcolor{gray}{\footnotesize 17}

\begin{enumerate}\item \textbf{他守护此等},即此比丘守护这些资具。\textbf{自制而行},即自制而住,守护威仪、守护身语意门等。\end{enumerate}

\subsection\*{\textbf{979} {\footnotesize 〔PTS 972〕}}

\textbf{「目光下视,且不游步,应从事禅那,常事醒觉,\\}
\textbf{「等持于舍,他应断绝寻、意乐、恶作。}

Okkhittacakkhu na ca pādalolo, jhānānuyutto bahujāgar’assa;\\
upekkham ārabbha samāhitatto, takkāsayaṃ kukkucciy’ūpachinde. %\hfill\textcolor{gray}{\footnotesize 18}

\begin{enumerate}\item \textbf{从事禅那},以生起未生起的及习行已生起的而从事于禅那。\textbf{等持于舍},即增长第四禅的舍已,等持其心。\textbf{寻、意乐、恶作},即欲寻等的寻,以及作为这寻的意乐的欲想等,以及手的恶作等的恶作。\end{enumerate}

\subsection\*{\textbf{980} {\footnotesize 〔PTS 973〕}}

\textbf{「当受言语呵责,具念者应欢喜,应破除对同梵行者的荒秽,\\}
\textbf{「他应说善语,而不过分,不应存心于闲谈。}

Cudito vacībhi satimābhinande, sabrahmacārīsu khilaṃ pabhinde;\\
vācaṃ pamuñce kusalaṃ nātivelaṃ, janavādadhammāya na cetayeyya. %\hfill\textcolor{gray}{\footnotesize 19}

\begin{enumerate}\item \textbf{当受言语呵责,具念者应欢喜},当受亲教师等的言语呵责,应具念而欢喜于这呵责。\textbf{应说善语},即应说能生起智慧的语言。\textbf{过分},即超越时间的边界及戒的边界。\end{enumerate}

\subsection\*{\textbf{981} {\footnotesize 〔PTS 974〕}}

\textbf{「然后,在世间有五尘,为调伏它们,具念者应修学,\\}
\textbf{「应忍耐对于色、声、味、香、触等的贪染。}

Athāparaṃ pañca rajāni loke, yesaṃ satīmā vinayāya sikkhe;\\
rūpesu saddesu atho rasesu, gandhesu phassesu sahetha rāgaṃ. %\hfill\textcolor{gray}{\footnotesize 20}

\begin{enumerate}\item \textbf{五尘},即色贪等的五尘。\end{enumerate}

\subsection\*{\textbf{982} {\footnotesize 〔PTS 975〕}}

\textbf{「应调伏对这些法的欲,比丘具念、善解脱心,\\}
\textbf{「他时常正当地审视着法,成就专一,他便能破除暗冥。」}

Etesu dhammesu vineyya chandaṃ, bhikkhu satimā suvimuttacitto;\\
kālena so sammā dhammaṃ parivīmaṃsamāno, ekodibhūto vihane tamaṃ so” ti. %\hfill\textcolor{gray}{\footnotesize 21}

\begin{enumerate}\item \textbf{他时常正当地审视着法},这比丘当在如「当心掉举时,即是(修习)三摩地的时间」中所说的时间,以无常等方法审视着一切有为法。\textbf{成就专一,他便能破除暗冥},他一境性心,便能破除一切愚痴等的暗冥,于此更无疑惑。如是,世尊以阿罗汉为顶点而完成了开示。当开示终了,五百比丘即证阿罗汉,三十俱胝之数的天、人得了法的现观。\end{enumerate}

\begin{center}\vspace{1em}舍利弗经第十六\\Sāriputtasuttaṃ soḷasamaṃ.\end{center}

\begin{center}\vspace{1em}八颂品第四\\Aṭṭhakavaggo catuttho.\end{center}