\section{胜利经}

\begin{center}Vijaya Sutta\end{center}\vspace{1em}

\begin{enumerate}\item \textbf{难陀经},也称\textbf{胜利经}、\textbf{身离欲经}。缘起为何?据说,此经曾在两处叙述,所以有两重缘起。
\item 这里,当世尊渐次到达迦毗罗卫,调伏了诸释种,度了难陀等出家,听许女人出家时,难陀长老的姊妹难陀、释迦王安稳的女儿绝色难陀、倾国难陀等三难陀便出了家。尔时,世尊住舍卫国。绝色难陀甚是绝色、可观、明艳,因此得名「绝色难陀」。倾国难陀也未见到与自己的容貌相等的。她俩都由容貌的㤭慢不去给侍世尊「世尊贬抑、指责容貌,以多种方法显示容貌的过患」,乃至都不愿见。
\item 若如是不净喜,为什么要出家?由无归宿。因为绝色难陀的丈夫,释迦王子,在婚礼当天便死了,于是父母便强令其出家。倾国难陀当难陀尊者证得阿罗汉时便无意乐「我的丈夫、母亲大波阇波提与其他亲属都已出家,无诸亲属,居家生活实苦」,由于居家生活不得乐味而出家,而非由信。
\item 于是,世尊了知到她们智的成熟,便命令大波阇波提:「让所有比丘尼依次前来受教诫!」她们在轮到自己时,派遣别人。随后,世尊说:「当轮到时,应自己前来,不得派遣别人。」于是,某天,绝色难陀便去了。世尊以化现之色令其惊怖,并以法句偈\begin{quoting}此城骨所建……(法句第 11:150 颂)\end{quoting}与长老尼偈\begin{quoting}难陀!看这积聚!恼患、不净、腐臭,\\渗漏着、滴淌着,为愚人所愿求。(长老尼偈第 19 颂)\\且应修习无相!舍弃慢的随眠!\\随后,因慢的止息,寂静得行。(长老尼偈第 20 颂、经集第 345 颂)\end{quoting}渐次令住于阿罗汉。
\item 于是,某天,舍卫国的居民们作了饭前的布施,受持布萨,整理衣裳,持了花、香等,为闻法去到祇园,在闻法的终了礼拜了世尊,进入城内。比丘尼僧团也在听了法论后,到达比丘尼的住处。在那里,人们和众比丘尼赞美着世尊。因为在世间共住的四量者\footnote{四量者,见\textbf{增支部}第 4:65 经。}之中,见了正等正觉而不净喜者是不存在的。以色为量之人在见到世尊以相装饰、杂以随形好、环绕以辉煌光鬘的一寻身光、如为庄严世间而生起的色后而净喜,以声为量者在听闻数百个本生中的名声,以及具足八支\footnote{具足八支的梵音:即清晰、明了、美妙、和雅、饱满、不散、深沉、浑厚,见\textbf{长部}·阇尼沙经、大典尊经等。}、如迦陵频伽般甜美发声的梵音后而净喜,以俭为量者在见到衣钵等的俭约或难行之事的俭约后而净喜,以法为量者在考察戒蕴等中的任一法蕴后而净喜,所以,在一切处赞美着世尊。倾国难陀到达比丘尼的住处后,也听到她们以种种方法对世尊的赞美,想要接近世尊,便告知了众比丘尼。众比丘尼即带了她去世尊处。
\item 世尊事先便得知她会到来,如同想要以刺取刺、以楔出楔的男子般,为了以容貌调伏对容貌的㤭慢,以自身的神变之力,化出十五六岁年纪且极可观的女子站在一侧扇扇。难陀与众比丘尼一起到来后,礼拜了世尊,在比丘尼僧团中坐下,见到世尊从脚掌至发尖容貌的成就,又再见到那站在世尊一侧化现者的容貌「哎!这女子真美」,便舍弃了对自己容貌的㤭慢,沉迷于她的容貌。
\item 随后,世尊将此女子的年龄变为二十岁再予显示。因为女人唯十六岁的年纪光彩照人,而非以上。于是,见到她容貌的减损后,难陀对其容貌的欲贪便变薄了。随后,世尊将其变为尚未生育的容貌、一旦经历生育的容貌、中年妇女的容貌、老年妇女的容貌,如是直到百岁,弯腰拄杖,斑痕丛生,当难陀仍在观看时,便显示了其死亡,以及膨胀等类、为群鸦等围绕啄食、恶臭而嫌厌的状态。难陀见此次第后,想「如是如是,对我也好,对他人也好,此次第是一切共通的」,便安立于无常想,并据此而于苦、无我想,三有便如炽燃,成无所皈依而现起。
\item 于是,世尊了知到「难陀的心跃入业处」,因顺适于她,说了以下几颂:\begin{quoting}难陀!看这积聚!恼患、不净、腐臭,\\渗漏着、滴淌着,为愚人所愿求。(长老尼偈第 19 颂)\\此即如彼,彼即如此,\\从空观诸界!莫再来世间!\\于诸有离欲已,寂静得行。\end{quoting}在颂的终了,难陀便住立于须陀洹果。于是,世尊为其得证更高的道,论及附以空性的毗婆舍那业处,说了此经。此即是其一重缘起。
\item 而当世尊住于王舍城时,曾在衣犍度中详说经过的妓女娑罗婆蒂之女,耆婆的幼妹,名为室利者,在母亲去世、得了其位后,于\begin{quoting}以不忿胜忿……(法句第 17:223 颂)\end{quoting}一颂之事中,鄙视商人之女富楼那并请求世尊原谅,在听了法的开示后,成为须陀洹,设了八天的常食。某个常食者比丘对她便起了贪染,在法句偈之事\footnote{此事见\textbf{法句}第 147 颂义注。}中说他甚至无法饮食,绝食而卧。就在他这样躺卧时,室利竟死去,成了夜摩居处中善夜摩的王后。
\item 于是,世尊在遮止其遗体火化、令国王弃置遗体于新坟后,为比丘僧团所随从,前往观看,并带上了那比丘,国王及城民们也都前往。那里,人们说:「先前以一千零八(硬币)也难得见室利,现而今,没人愿以一硬币去看她。」室利天女也为五百车所随从,去到那里。世尊便在那里为给集会者作法的开示,说了此经,并为教诫那比丘,说了这\begin{quoting}观此粉饰身……(法句第 11:147 颂)\end{quoting}的法句偈。此即是其第二重缘起。\end{enumerate}

\subsection\*{\textbf{195} {\footnotesize 〔PTS 193〕}}

\textbf{走着或是站着,坐着或是躺着,\\}
\textbf{弯曲或伸展,这是身体的运动。}

Caraṃ vā yadi vā tiṭṭhaṃ, nisinno uda vā sayaṃ;\\
samiñjeti pasāreti, esā kāyassa iñjanā. %\hfill\textcolor{gray}{\footnotesize 1}

\begin{enumerate}\item 这里,\textbf{走着},即整个色身以朝着欲去方位的活动而行。\textbf{或是站着},即此或是以起立而站。\textbf{坐着或是躺着},即此以下半身之弯曲、上半身之直立的状态而坐,或以横向伸展而躺。\textbf{弯曲或伸展},即弯曲或伸展彼彼关节。\textbf{这是身体的运动},即这一切都是此俱识之身体的运动、摇动、变动,其中没有任何他者在走或在伸展。
\item 而且,所谓「我走」,即在心生起时,以之等起的风界遍满身体,因此而有朝着欲去方位的活动,即在方位之变换处显现色的变换之义,因此称为「走着」。同样,所谓「我站」,即在心生起时,以之等起的风界遍满身体,因此而有直立,即以向更高之处显现色之义,因此称为「站着」。同样,所谓「我坐」,即在心生起时,以之等起的风界遍满身体,因此而有下半身之弯曲、上半身之直立,即以如此的状态显现色之义,因此称为「坐着」。同样,所谓「我躺」,即在心生起时,以之等起的风界遍满身体,因此而有横向之伸展,即以如此的状态显现色之义,因此称为「躺着」。
\item 且如是任何名为某某的尊者走着或是站着,坐着或是躺着,即是说他于彼彼威仪以弯曲或伸展彼彼关节的方式弯曲或伸展的意思。且因为在弯曲或伸展之心生起时,它便以所说的方法而存在,所以,这是身体的运动,其中没有任何他者,无有任何或走或伸展的有情或人,而仅仅是:\begin{quoting}出于心之种种,风便成种种,\\由于风之种种,身体的运动便成种种。\end{quoting}这即此中的第一义。
\item 如是,在此颂中,因为以长时运用一种威仪而有对身体的压迫,且为了除此而作威仪的更换,所以,世尊以「走着或是」等显示被威仪遮蔽的苦相。同样,由在行走时即无站立等,当说这一切行走等类「这是身体的运动」时,显示被相续遮蔽的无常相,由彼彼和合而起,当以拒斥我而说「这是身体的运动」时,显示被我想密集遮蔽的无我相。\end{enumerate}

\subsection\*{\textbf{196} {\footnotesize 〔PTS 194〕}}

\textbf{骨腱相连,涂以皮肉,\\}
\textbf{身体为表皮所遮蔽,不能如实得见。}

Aṭṭhi-nahāru-saṃyutto, taca-maṃsāvalepano;\\
chaviyā kāyo paṭicchanno, yathābhūtaṃ na dissati. %\hfill\textcolor{gray}{\footnotesize 2}

\begin{enumerate}\item 如是,在以显示三相论述了空性的业处后,又为显明俱识者与无识者的不净,开始了此颂。
\item 其义为:且此「这是身体的运动」中的身体,由在清净道论对三十二行相的注解中以色、形、方位、处所、界限等\footnote{三十二行相的注解,见\textbf{清净道论}·说随念业处品第 83~138 段及以下。}及以不造作的方法\footnote{不造作的方法:据菩提比丘注 856,如\textbf{清净道论}·说定品第 47~80 段所示。}所阐明的三百六十块骨与九百块腱相连,而为\textbf{骨腱相连},由涂以仍于彼处所阐明的脚趾尖皮等的皮与分解为九百片的肉,而为\textbf{涂以皮肉},当知最为恶臭嫌厌。
\item 且此处当知什么?若不是这(身体)为从中人的全身剥聚仅有枣核之量、薄如蝇翼的表皮所遮蔽,如屋壁之为青等色彩?然而,这\textbf{身体为}如是之薄的\textbf{表皮所遮蔽},\textbf{不能}为无有慧眼的愚痴凡夫\textbf{如实得见}。
\item 那么对他而言,连为外皮所沾染,被称为最为嫌厌之法的皮\footnote{皮与表皮:皮 \textit{taco} 在内,唯白色,表皮 \textit{chavi} 在外,呈黑、褐、黄等色,见\textbf{清净道论}·说随念业处品第 93 段。},连为皮所包裹,凡是分解为如\begin{quoting}九百片肉,涂抹在躯干上,\\如种种蛆群聚集的粪坑般腐臭。\end{quoting}所说的九百块肉,连为肉所涂抹,凡是以\begin{quoting}在一寻的躯干内,有九百块腱,\\束缚着骨聚,如房舍之为藤蔓。\end{quoting}所说的腱,连为腱所敷布,依次树立的腐臭、恶臭的三百六十块骨,都不能如实得见。\end{enumerate}

\subsection\*{\textbf{197} {\footnotesize 〔PTS 195〕}}

\textbf{充以小肠,充以胃,肝脏、膀胱,\\}
\textbf{与心、肺、肾、脾,}

Antapūro udarapūro, yakanapeḷassa vatthino;\\
hadayassa papphāsassa, vakkassa pihakassa ca. %\hfill\textcolor{gray}{\footnotesize 3}

\begin{enumerate}\item 由于不执取这薄如蝇翼的表皮,而是对由为沾染于外皮的皮所包裹而对这整个世间未予显明、俱种种品类、内在于躯骸、最为不净、恶臭、嫌厌者,以慧眼通达已,当如是观「充以小肠,充以胃……胆汁、膏」。
\item 这里,\textbf{胃},即胃中物的同义语,以所处之名而称为「胃」。\textbf{膀胱},即尿,以位置接近而称为「膀胱」。\textbf{充}是统摄词,所以当如「充以肝脏、充以膀胱」般连结。\textbf{心}等处仿此。且这小肠等一切,仍当以清净道论所说的方法,即以色、形、方位、处所、界限等及以不造作的方法了知。\end{enumerate}

\subsection\*{\textbf{198} {\footnotesize 〔PTS 196〕}}

\textbf{以及涕、唾、汗、脂肪,\\}
\textbf{与血、关节滑液、胆汁、膏。}

Siṅghāṇikāya kheḷassa, sedassa ca medassa ca;\\
lohitassa lasikāya, pittassa ca vasāya ca. %\hfill\textcolor{gray}{\footnotesize 4}

\subsection\*{\textbf{199} {\footnotesize 〔PTS 197〕}}

\textbf{另外,从其九孔总有不净流出,\\}
\textbf{眼眵从眼中,耳垢从耳中,}

Ath’assa navahi sotehi, asucī savati sabbadā;\\
akkhimhā akkhigūthako, kaṇṇamhā kaṇṇagūthako. %\hfill\textcolor{gray}{\footnotesize 5}

\begin{enumerate}\item 如是,世尊在显示了内在于躯骸者「其中无有任何一物能被把握为与珍珠摩尼相等者,相反,这身体充满不净」后,现在,以分泌出躯骸者令其明了,并摄受了先前之所说,为显示内在于躯骸者,说了这两颂。
\item 这里,\textbf{另外},即指出方法的变换,即是说以另一方法来观不净的意思。\textbf{其},即此身体。\textbf{九孔},即双眼窍、耳窍、鼻窍、口、肛道、尿道。\textbf{不净流出},即对一切世间明了、俱种种品类、最为恶臭嫌厌的不净流出、流露、滴淌,而无任何沉香旃檀等种种香,或摩尼珍珠等种种宝。\textbf{总有},即在或夜或昼、或上午或下午、或站或走的一切时。
\item 若问「这不净是什么」,即\textbf{眼眵从眼中}等。因为从其两眼窍流出如脱落的皮肉似的眼眵,从耳窍流出如尘泥似的耳垢,从鼻窍流出如脓似的涕。\end{enumerate}

\subsection\*{\textbf{200} {\footnotesize 〔PTS 198〕}}

\textbf{涕从鼻中,有时从口吐出\\}
\textbf{胆汁和痰,汗污则从身上。}

Siṅghāṇikā ca nāsato, mukhena vamat’ekadā;\\
pittaṃ semhañ ca vamati, kāyamhā sedajallikā. %\hfill\textcolor{gray}{\footnotesize 6}

\begin{enumerate}\item \textbf{从口吐出},若问「吐出什么」,\textbf{有时}是\textbf{胆汁},即当尚未停滞的胆汁受到扰动时,将其吐出的意思。\textbf{和痰},即不仅是胆汁,有时也吐出存于胃膜内一钵之量的痰。且此仍当从色等,以清净道论所说的方法\footnote{见\textbf{清净道论}·说随念业处品第 128 段。}了知。以「和痰」中的「和」字显示吐出痰及其它这样的胃中物、血等不净。
\item 如是,在显示七门中的不净吐秽后,知时、知人、知众的世尊不再以明言涉及此外的二门,为以另外的方法显示全身的不净流出,而说「汗污则从身上」。这里,\textbf{汗污},即汗与盐渍、泥垢等类的污秽,它与「总有流出」连在一起。\end{enumerate}

\subsection\*{\textbf{201} {\footnotesize 〔PTS 199〕}}

\textbf{另外,其头颅充满了脑,\\}
\textbf{出于无明,愚人以净思量它。}

Ath’assa susiraṃ sīsaṃ, matthaluṅgassa pūritaṃ;\\
subhato naṃ maññati bālo, avijjāya purakkhato. %\hfill\textcolor{gray}{\footnotesize 7}

\begin{enumerate}\item 如是,正好比在烹煮食物时,米粒之垢及水垢随浮沫一起沸起,涂抹锅盖,向外溢出,同样,在以业生之火烹煮饮食等类的食物时,饮食等之垢沸起,以「眼眵从眼中」等类泌出,涂抹眼等,向外溢出,在依此显示了此身不净的状态后,现在,对世间共许为最上部位的头首,认可其为极其殊胜者,甚至对应予礼拜者不行礼拜,世尊为以无实性及不净性显示其不净的状态,说了此颂。
\item 这里,\textbf{颅},即窍。\textbf{充满了脑},即如盛满了凝乳的瓠瓜般,盛满了脑。且此脑仍当以清净道论所说的方法了知。\textbf{愚人以净思量它},即以恶思惟思惟的愚人以净思量这盛满如是种种腐臭的身体,且以爱、见、慢三者之思量,思量为洁净、可意、可爱、适意。为什么?因为\textbf{出于无明},即出于、迫于、起于遮蔽四谛的愚痴,被令「如是持取、如是执著、如是思量」之意。且看!这无明有多引致非利!\end{enumerate}

\subsection\*{\textbf{202} {\footnotesize 〔PTS 200〕}}

\textbf{而当其死去,躺着,膨胀,青瘀,\\}
\textbf{被丢弃在塚间,亲戚们不再关切。}

Yadā ca so mato seti, uddhumāto vinīlako;\\
apaviddho susānasmiṃ, anapekkhā honti ñātayo. %\hfill\textcolor{gray}{\footnotesize 8}

\begin{enumerate}\item 如是,在以俱识者显示了不净后,现在为以无识者显示,或者,因为即便如转轮王的身体也盛满如所说的腐臭,所以,在显示了一切品类成就状态中的不净后,现在为显示毁亡状态,说了此颂。
\item 其义为:像这样的身体,当因寿、暖、识的离去而\textbf{死去},如鼓风的布袋般\textbf{膨胀},以色的剥落而\textbf{青瘀},\textbf{在塚间}徒然如被抛弃的木块般\textbf{被丢弃}而\textbf{躺着},于是,\textbf{亲戚们}便决然\textbf{不再关切}「他现在再也起不来了」。
\item 这里,以「死去」显示无常性,以「躺着」显示不活跃,并以此二者敦促舍弃对寿命和力量的㤭慢。以「膨胀」显示形状的毁亡,以「青瘀」显示外皮的毁亡,并以此二者敦促舍弃对容貌的㤭慢,及舍弃缘于肤色之美的慢。以「被丢弃」显示无可执取的状态,以「在塚间」显示内在不应承受的嫌厌状态,且以此二者敦促舍弃对我所的执取及净想。以「亲戚们不再关切」显示不可回报的状态,并以此敦促舍弃对随从的㤭慢。\end{enumerate}

\subsection\*{\textbf{203} {\footnotesize 〔PTS 201〕}}

\textbf{狗、豺、狼、蛆都来啖食他,\\}
\textbf{乌鸦、秃鹫也来啄食,还有其它生类。}

Khādanti naṃ suvānā ca, siṅgālā ca vakā kimī;\\
kākā gijjhā ca khādanti, ye c’aññe santi pāṇino. %\hfill\textcolor{gray}{\footnotesize 9}

\begin{enumerate}\item 如是,在上颂以未破坏的无识者显示了不净后,现在,为以破坏显示,说了此颂。
\item 这里,\textbf{还有其它},即还有其它乌鸦、鹰鹯等食腐的\textbf{生类},它们也都来啄食之义。其余自明。\end{enumerate}

\subsection\*{\textbf{204} {\footnotesize 〔PTS 202〕}}

\textbf{听闻了佛语,俱智的比丘于此\\}
\textbf{遍知了它,因为他如实地得见。}

Sutvāna buddhavacanaṃ, bhikkhu paññāṇavā idha;\\
so kho naṃ parijānāti, yathābhūtañ hi passati. %\hfill\textcolor{gray}{\footnotesize 10}

\begin{enumerate}\item 如是,在以「走着或是」等空性业处的方法,及「骨腱相连」等俱识者的不净、「而当其死去躺着」等无识者的不净显示了身体后,如是于无有常、乐、我且极不净的身体,以「出于无明,愚人以净思量它」阐明了愚人的行为,并以无明为首显示了流转,现在,为了显示智者的行为,及以遍知为首显示还灭,开始了此颂。
\item 这里,\textbf{听闻},即如理倾听。\textbf{佛语},即引致身离欲的佛语。\textbf{比丘},即有学或凡夫。\textbf{俱智},即具足被称为智的于无常等品类转起的毗婆舍那。\textbf{于此},即于教法中。
\item \textbf{遍知了它},即以三遍知遍知此身体。如何?正好比善巧的商人观察了货物「有这,还有这」,在考量「当取走这些,会有这些盈利」后即照做,再取了连带盈利的本金,弃置那货物,如是,当以智眼观察「骨腱等,还有这些发毛等」时,以知遍知遍知,当考量「这些法是无常、苦、无我」时,以度遍知遍知,如是度已,当证得圣道时,于彼以舍弃欲贪,以断遍知遍知。或者,当以俱识者、无识者的不净见时,以知遍知遍知,当以无常等见时,以度遍知遍知,随后,以阿罗汉道除去欲贪已,当舍弃此时,以断遍知遍知。
\item 若问「为什么他如是遍知」,\textbf{因为他如实地得见}。且当此义成就时,唯以「俱智」等,因为听闻了佛语而成俱智,又因为这身体即便对一切人明了,未听闻佛语则不能遍知,所以,为显示此智之因,以及对此外之人不能如是而见,而说「听闻了佛语」。由为难陀比丘尼与颠倒心之比丘发起开示,以及由最上会众、由显示证得此行道的比丘的状态而说「比丘」。\end{enumerate}

\subsection\*{\textbf{205} {\footnotesize 〔PTS 203〕}}

\textbf{此即如彼,彼即如此,\\}
\textbf{于内在及外在,他能于身离欲。}

Yathā idaṃ tathā etaṃ, yathā etaṃ tathā idaṃ;\\
ajjhattañ ca bahiddhā ca, kāye chandaṃ virājaye. %\hfill\textcolor{gray}{\footnotesize 11}

\begin{enumerate}\item 现在,在「因为他如实地得见」中,为显示见者如何如实得见,说了此颂。
\item 其义为:好比这俱识者之不净,由寿、暖、识之未离而行住坐卧,如是,现今这在塚间躺着的无识者在先前彼等法未离时也曾是。又好比这现今已死之身,由彼等法之离去而不得行住坐卧,如是,这俱识者在彼等法离去时也将是。
\item 又好比这俱识者现今尚未死去在塚间躺着,未至膨胀等的状态,如是,这现今已死之身先前也曾是。然而,好比这现今无识者之不净,死去而在塚间躺着,且至膨胀等的状态,如是,这俱识者也将是。
\item 这里,\textbf{此即如彼},即将已死之身与自己置于同等,舍弃对外在的嗔。\textbf{彼即如此},即将自己与已死之身置于同等,舍弃对内在的贪。当舍弃以之平等对待两者的行相时,便舍弃了于两处的痴。
\item 如是,在以如实知见完成了对前分中不善根的舍弃后,因为如是行道的比丘渐次证得了阿罗汉道,堪能对一切欲贪离欲,所以说\textbf{他能于身离欲},省略的文本为「如是行道的比丘渐次」。\end{enumerate}

\subsection\*{\textbf{206} {\footnotesize 〔PTS 204〕}}

\textbf{这俱智的比丘于此已离欲贪,\\}
\textbf{得证不死、寂静、涅槃、不殁的境地。}

Chandarāgaviratto so, bhikkhu paññāṇavā idha;\\
ajjhagā amataṃ santiṃ, nibbānaṃ padam accutaṃ. %\hfill\textcolor{gray}{\footnotesize 12}

\begin{enumerate}\item 如是,在显示了有学地后,现在,为显示无学地,说了此颂。
\item 其义为:\textbf{这比丘}以阿罗汉道智而\textbf{俱智},证得了与道无间的果,于是,由舍弃了一切欲贪而\textbf{已离欲贪},\textbf{得证}以无死的状态或以胜妙之义而\textbf{不死}、由止息一切行而\textbf{寂静}、由无有被称为渴爱的丛林而\textbf{涅槃}、由无有亡殁状态而\textbf{不殁}的受赞叹的\textbf{境地}。或者,当知这比丘以阿罗汉道智而俱智,住立于与道无间的果,便名为已离欲贪,且得证所说品类的境地。以此显明「此为其所舍弃,此为以舍而得」。\end{enumerate}

\subsection\*{\textbf{207} {\footnotesize 〔PTS 205〕}}

\textbf{这两足者不净,恶臭,备受爱护,\\}
\textbf{充满种种腐臭,散发于处处。}

Dvipādako’yaṃ asuci, duggandho parihīrati;\\
nānākuṇapaparipūro, vissavanto tato tato. %\hfill\textcolor{gray}{\footnotesize 13}

\begin{enumerate}\item 如是,在以俱识者、无识者论述了不净业处及结果后,又为以简略的开示指责作为如是大功德之障碍的放逸住,说了以下两颂。
\item 这里,虽然无足者等的身体也为不净,然而于此,依统摄或依高贵的部分,或者因为其它不净生物的身体以盐、酸等调和后,被呈作人类的食物,而人类的身体则不然,所以为显示其更为不净的状态,而说\textbf{两足者}。
\item \textbf{这},即显示人类的身体。\textbf{恶臭,备受爱护},即当恶臭时,以花、香等装扮而备受爱护。\textbf{充满种种腐臭},即盛满发等多种品类的腐臭。\textbf{散发于处处},即对努力以花、香等遮蔽者,其精进为无果,仍从九门散发着唾涕等,且从毛孔散发着汗污。\end{enumerate}

\subsection\*{\textbf{208} {\footnotesize 〔PTS 206〕}}

\textbf{若以这样的身体自视高标,\\}
\textbf{或鄙视他人,除无知见,还能为何?}

Etādisena kāyena, yo maññe uṇṇametave;\\
paraṃ vā avajāneyya, kim aññatra adassanā ti. %\hfill\textcolor{gray}{\footnotesize 14}

\begin{enumerate}\item 这里,现在请看:\textbf{若以这样的身体},即男子、女子或任何愚人以爱、见、慢的思量,或以「我、我所、常」等方法\textbf{自视高标},\textbf{或}以出身等\textbf{鄙视他人},置自身于高处,\textbf{除无知见,还能为何}?除了未以圣道知见圣谛,他还有什么别的如是高标或鄙视的原因呢?
\item 在开示的终了,难陀比丘尼陷入悚惧「哎!咄!我真是愚人,不曾前来给侍单单为我转起如是种种法的开示的世尊」。且经如是悚惧、念持此法的开示,即以此业处,数日之内便证了阿罗汉。
\item 而在第二处,据说,在开示的终了,八万四千生类得了法的现观,室利天女证得阿那含果,且此比丘也住于须陀洹果。\end{enumerate}

\begin{center}\vspace{1em}胜利经第十一\\Vijayasuttaṃ ekādasamaṃ.\end{center}

%\begin{flushright}甲辰五月十九二稿\end{flushright}