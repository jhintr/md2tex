\section{摩伽经}

\begin{center}Māgha Sutta\end{center}\vspace{1em}

\textbf{如是我闻。一时世尊住王舍城耆阇崛山。于是,摩伽学童往世尊处走去,走到后,问候了世尊,彼此寒暄已,坐在一边。坐在一边的摩伽学童对世尊说:}

Evaṃ me sutaṃ— ekaṃ samayaṃ Bhagavā Rājagahe viharati Gijjhakūṭe pabbate. Atha kho Māgho māṇavo yena Bhagavā ten’upasaṅkami, upasaṅkamitvā Bhagavatā saddhiṃ sammodi, sammodanīyaṃ kathaṃ sāraṇīyaṃ vītisāretvā ekamantaṃ nisīdi. Ekamantaṃ nisinno kho Māgho māṇavo Bhagavantaṃ etad avoca:

\begin{enumerate}\item 缘起为何?这在其因缘中已述。因为这摩伽学童是施者、施主,他想到:「施与到来的穷人和旅人是否有大果报呢?我要去问问沙门乔达摩,据说沙门乔达摩知道过去、未来和现在。」他便前往世尊处提问,而世尊便据问题向他作了解答。在汇集了结集者、婆罗门、世尊等三者的话后,此即是「摩伽经」。
\item 这里,\textbf{王舍城},即如是名称的城市。因为由曼达多、大典尊等(王)所拥有故,被称为王舍城。人们还以其它方式来解释,何妨?这只是那城市的名称。然而,它只在佛时和转轮王时才是城市,在其余时间则空如而为夜叉占据,作为它们的春林\footnote{春林 \textit{vasantavana}:菩提比丘注 1411 引\textbf{律藏疏}云,即夜叉春时游戏之处。}而存在。如是显示了行处村落后,便说所住之处:\textbf{耆阇崛山}。当知其因群鹫居于其巅,或其山巅肖似灵鹫,所以被称为「耆阇崛\footnote{耆阇崛 \textit{Gijjhakūṭa}:意译即鹫峰。}」。
\item \textbf{于是……说},此中,\textbf{摩伽}是这婆罗门的名字。\textbf{学童},即以未越过弟子的状态而得称,即便年岁已老。也有人说是因先前的习行,如褐者学童\footnote{褐者学童 \textit{Piṅgiya māṇava}:即\textbf{彼岸道品}中的十六学童之一。}。因为他即使已有百二十岁,仍因先前的习行被认作「褐者学童」。其余已述。\end{enumerate}

\textbf{「乔达摩君!我是施者、施主、慷慨的应请者,我如法地寻求财富,如法地寻求财富后,把如法所获、如法所得的财富,布施给一人,布施给二、三、四、五、六、七、八、九、十人,布施给二十、三十、四十、五十人,布施给一百人,布施给更多。乔达摩君!我如是布施、如是供献,能带来许多福德吗?」}

“Ahañ hi, bho Gotama, dāyako dānapati vadaññū yācayogo, dhammena bhoge pariyesāmi, dhammena bhoge pariyesitvā dhammaladdhehi bhogehi dhammādhigatehi ekassa pi dadāmi dvinnam pi tiṇṇam pi catunnam pi pañcannam pi channam pi sattannam pi aṭṭhannam pi navannam pi dasannam pi dadāmi, vīsāya pi tiṃsāya pi cattālīsāya pi paññāsāya pi dadāmi, satassa pi dadāmi, bhiyyo pi dadāmi. Kaccāhaṃ, bho Gotama, evaṃ dadanto evaṃ yajanto bahuṃ puññaṃ pasavāmī” ti?

\begin{enumerate}\item \textbf{乔达摩君!我是……能带来许多福德吗},此中,\textbf{施者、施主},若因受命而布施别人的财产,他即是施者,但由对此所施并无所有权,故非施主,而他唯布施自己的财产,因此说「乔达摩君!我是施者、施主」。这即此中之义,但在别处,有以「间或被悭吝克服者为施者,未被克服者为施主」等方法而说者,这也合适。\textbf{慷慨的},即甫一开口,我就因明确个体的特点或因把握众多资助之相,知晓乞求者的话语:「他需要此,他需要彼。」\textbf{应请者},即适于乞求。显明「若一见到乞求者,便现高傲,说粗恶语等,他即非应请者,而我不如此」。
\item \textbf{如法},即避免了不与取、虚诳、欺诈等而行乞、乞求之义。因为乞求是婆罗门寻求财富之法,且这些乞求者从欲行摄受的他人所得的财富名为如法所获、如法所得,而他即如是寻求而得,因此说「\textbf{如法地寻求财富……如法所得的财富}」。\textbf{布施给更多},即布施给较此更上,无有限量,于此显示以所获的财富之量布施。\end{enumerate}

\textbf{「确实,学童!你如是布施、如是供献,能带来许多福德。学童!若施者、施主、慷慨的应请者,如法地寻求财富,如法地寻求财富后,把如法所获、如法所得的财富,布施给一人……布施给一百人,布施给更多,他能带来许多福德。」}

“Taggha tvaṃ, māṇava, evaṃ dadanto evaṃ yajanto bahuṃ puññaṃ pasavasi. Yo kho, māṇava, dāyako dānapati vadaññū yācayogo, dhammena bhoge pariyesati, dhammena bhoge pariyesitvā dhammaladdhehi bhogehi dhammādhigatehi ekassa pi dadāti…pe… satassa pi dadāti, bhiyyo pi dadāti, bahuṃ so puññaṃ pasavatī” ti.

\begin{enumerate}\item \textbf{确实},即必然之语的不变词。因为布施,乃至施与畜生,必然为一切佛、辟支佛、声闻所赞赏。如说:\begin{quoting}布施在一切处被称赞,布施不为任何人谴责。\end{quoting}所以,世尊也必然赞赏他,而说「\textbf{确实,学童!你……带来许多福德}」。余义自明。\end{enumerate}

\textbf{于是,摩伽学童以偈颂对世尊说:}

Atha kho Māgho māṇavo Bhagavantaṃ gāthāya ajjhabhāsi:

\begin{enumerate}\item 当如是为世尊说到「他能带来许多福德」时,婆罗门欲闻应供者的供养清净,便进一步问了世尊,因此结集者们说:「\textbf{于是,摩伽学童以偈颂对世尊说。}」其义已述。\end{enumerate}

\subsection\*{\textbf{492} {\footnotesize 〔PTS 487〕}}

\textbf{「我问慷慨的乔达摩,」摩伽学童说,「身著袈裟,无家而行,\\}
\textbf{「若应请者、施主、在家人,希望福德、希求福德而作供奉,\\}
\textbf{「于此布施饮食给他人,供奉者的供品如何得到净化?」}

“Pucchām’ahaṃ Gotamaṃ vadaññuṃ, \textit{(iti Māgho māṇavo)} kāsāyavāsiṃ agahaṃ carantaṃ;\\
yo yācayogo dānapati gahaṭṭho, puññatthiko yajati puññapekkho;\\
dadaṃ paresaṃ idha annapānaṃ, kathaṃ hutaṃ yajamānassa sujjhe”. %\hfill\textcolor{gray}{\footnotesize 1}

\begin{enumerate}\item 而在「我问」等颂中,\textbf{慷慨},即知语,即是说以一切行相知晓有情所说之语的意趣。\textbf{得到净化},即藉由应供者而得成清净、大果报。而此中的连结为:若应请者、施主、在家人希望福德,以布施饮食给他人而作供奉,不是仅把祭品投入火中,且唯希求福德,而非希求回报、好的名称声望等,他这样供奉者的供品如何能得净化?\end{enumerate}

\subsection\*{\textbf{493} {\footnotesize 〔PTS 488〕}}

\textbf{「若应请者、施主、在家人,摩伽!」世尊说,「希望福德、希求福德而作供奉,\\}
\textbf{「于此布施饮食给他人,这样的人当藉由应供者成功。」}

“Yo yācayogo dānapati gahaṭṭho, \textit{(Māghā ti Bhagavā)} puññatthiko yajati puññapekkho;\\
dadaṃ paresaṃ idha annapānaṃ, ārādhaye dakkhiṇeyyebhi tādi”. %\hfill\textcolor{gray}{\footnotesize 2}

\begin{enumerate}\item \textbf{这样的人当藉由应供者成功},即这样的应请者当藉由应供者成功、成就、清净,令此供品有大果报,而非其它之义。以此作「供奉者的供品如何得到净化」的解答。\end{enumerate}

\subsection\*{\textbf{494} {\footnotesize 〔PTS 489〕}}

\textbf{「若应请者、施主、在家人,」摩伽学童说,「希望福德、希求福德而作供奉,\\}
\textbf{「于此布施饮食给他人,世尊!请告知我应供者!」}

“Yo yācayogo dānapati gahaṭṭho, \textit{(iti Māgho māṇavo)} puññatthiko yajati puññapekkho;\\
dadaṃ paresaṃ idha annapānaṃ, akkhāhi me Bhagavā dakkhiṇeyye”. %\hfill\textcolor{gray}{\footnotesize 3}

\begin{enumerate}\item \textbf{世尊!请告知我应供者},此中当知如是连结:若应请者布施以供奉他人,世尊!请告知我应供者。\end{enumerate}

\subsection\*{\textbf{495} {\footnotesize 〔PTS 490〕}}

\textbf{「若无取著而行于世间,无所牵绊、整全、克己,\\}
\textbf{「希求福德的婆罗门若欲供奉,应适时给予他们供品。\footnote{此经第 495~509 的后半颂,同\textbf{孙陀利迦婆罗豆婆遮经}第 468~471 的后半颂,唯彼处作「祭祀、祭品」,此处则随上下文作「供奉、供品」。}}

“Ye ve asattā vicaranti loke, akiñcanā kevalino yatattā;\\
kālena tesu habyaṃ pavecche, yo brāhmaṇo puññapekkho yajetha. %\hfill\textcolor{gray}{\footnotesize 4}

\begin{enumerate}\item 然后,世尊为以种种品类的方法对其阐明应供者,说了以「若无取著」开头的几颂。这里,\textbf{无取著},即不以贪等染著而固著。\textbf{整全},即已完结了义务。\textbf{克己},即守护心。\end{enumerate}

\subsection\*{\textbf{496} {\footnotesize 〔PTS 491〕}}

\textbf{「若一切结缚与束缚已断,调御、解脱、无患、无待,\\}
\textbf{「希求福德的婆罗门若欲供奉,应适时给予他们供品。}

Ye sabbasaṃyojanabandhanacchidā, dantā vimuttā anīghā nirāsā;\\
kālena tesu habyaṃ pavecche, yo brāhmaṇo puññapekkho yajetha. %\hfill\textcolor{gray}{\footnotesize 5}

\begin{enumerate}\item 以无上调御而\textbf{调御},以慧与心的解脱而\textbf{解脱},以无未来流转之苦而\textbf{无患},以无现今的烦恼而\textbf{无待}。\end{enumerate}

\subsection\*{\textbf{497} {\footnotesize 〔PTS 492〕}}

\textbf{「若一切结缚已解脱,调御、解脱、无患、无待,\\}
\textbf{「希求福德的婆罗门若欲供奉,应适时给予他们供品。}

Ye sabbasaṃyojanavippamuttā, dantā vimuttā anīghā nirāsā;\\
kālena tesu habyaṃ pavecche, yo brāhmaṇo puññapekkho yajetha. %\hfill\textcolor{gray}{\footnotesize 6}

\begin{enumerate}\item 而此颂的第二颂,当知以显示修习威力的方法而说。此经可以为证:\begin{quoting}诸比丘!对从事修习而安住的比丘,即便不如是希望:「哎!愿我的心以无取从诸漏解脱!」然后,他的心也以无取从诸漏解脱。(增支部第 7:71 经)\end{quoting}\end{enumerate}

\subsection\*{\textbf{498} {\footnotesize 〔PTS 493〕}}

\textbf{「舍弃了贪、嗔、痴,漏尽、梵行已立,\\}
\textbf{「希求福德的婆罗门若欲供奉,应适时给予他们供品。}

Rāgañ ca dosañ ca pahāya mohaṃ, khīṇāsavā vūsitabrahmacariyā;\\
kālena tesu habyaṃ pavecche, yo brāhmaṇo puññapekkho yajetha. %\hfill\textcolor{gray}{\footnotesize 7}

\begin{enumerate}\item \textbf{舍弃了贪……伪善与慢\footnote{即第 499 颂。}……不陷于渴爱\footnote{即第 501 颂。}},即不倾向于欲贪等。\end{enumerate}

\subsection\*{\textbf{499} {\footnotesize 〔PTS 494\textit{a-d}〕}}

\textbf{「伪善与慢不住于彼,漏尽、梵行已立,\\}
\textbf{「希求福德的婆罗门若欲供奉,应适时给予他们供品。\footnote{PTS 本取 499、500 的第一句,与「希求福德的婆罗门若欲供奉,应适时给予他们供品」一起,作为其第 494 颂,若据 498 颂的义注来看,当是。}}

Yesu na māyā vasati na māno, khīṇāsavā vūsitabrahmacariyā;\\
kālena tesu habyaṃ pavecche, yo brāhmaṇo puññapekkho yajetha. %\hfill\textcolor{gray}{\footnotesize 8}

\subsection\*{\textbf{500} {\footnotesize 〔PTS 494\textit{e-h}〕}}

\textbf{「若离贪、无我所、无待,漏尽、梵行已立,\\}
\textbf{「希求福德的婆罗门若欲供奉,应适时给予他们供品。}

Ye vītalobhā amamā nirāsā, khīṇāsavā vūsitabrahmacariyā;\\
kālena tesu habyaṃ pavecche, yo brāhmaṇo puññapekkho yajetha. %\hfill\textcolor{gray}{\footnotesize 9}

\subsection\*{\textbf{501} {\footnotesize 〔PTS 495〕}}

\textbf{「他们确实不陷于渴爱,已度暴流,无我所而行,\\}
\textbf{「希求福德的婆罗门若欲供奉,应适时给予他们供品。}

Ye ve na taṇhāsu upātipannā, vitareyya oghaṃ amamā caranti;\\
kālena tesu habyaṃ pavecche, yo brāhmaṇo puññapekkho yajetha. %\hfill\textcolor{gray}{\footnotesize 10}

\subsection\*{\textbf{502} {\footnotesize 〔PTS 496〕}}

\textbf{「他们对世上任何,对此世或他世的有与无有,没有渴爱,\\}
\textbf{「希求福德的婆罗门若欲供奉,应适时给予他们供品。}

Yesaṃ taṇhā natthi kuhiñci loke, bhavābhavāya idha vā huraṃ vā;\\
kālena tesu habyaṃ pavecche, yo brāhmaṇo puññapekkho yajetha. %\hfill\textcolor{gray}{\footnotesize 11}

\begin{enumerate}\item \textbf{渴爱},即色爱等六种。\textbf{有与无有}\footnote{有与无有,见\textbf{蛇经}第 6 颂的注。},即常或断,或者说是「有的无有」,即是说再有的转生\footnote{再有的转生 \textit{punabbhavābhinibbatti}:PTS 本作「再有的不转生 \textit{punabbhavānabhi°}」。}。\textbf{此世或他世},即对「世上任何」的详说。\end{enumerate}

\subsection\*{\textbf{503} {\footnotesize 〔PTS 497〕}}

\textbf{「舍弃了爱欲,无家而行,善加自制,如梭子般正直,\\}
\textbf{「希求福德的婆罗门若欲供奉,应适时给予他们供品。\footnote{此颂及下颂全同\textbf{孙陀利迦婆罗豆婆遮经}第 469~470 颂。}}

Ye kāme hitvā agahā caranti, susaññatattā tasaraṃ va ujjuṃ;\\
kālena tesu habyaṃ pavecche, yo brāhmaṇo puññapekkho yajetha. %\hfill\textcolor{gray}{\footnotesize 12}

\subsection\*{\textbf{504} {\footnotesize 〔PTS 498〕}}

\textbf{「离于贪染,善等持诸根,如月亮解脱于罗睺的束缚,\\}
\textbf{「希求福德的婆罗门若欲供奉,应适时给予他们供品。}

Ye vītarāgā susamāhitindriyā, cando va Rāhuggahaṇā pamuttā;\\
kālena tesu habyaṃ pavecche, yo brāhmaṇo puññapekkho yajetha. %\hfill\textcolor{gray}{\footnotesize 13}

\subsection\*{\textbf{505} {\footnotesize 〔PTS 499〕}}

\textbf{「平静,离于贪染而无忿恨,于此舍弃已,他们没有趣向,\\}
\textbf{「希求福德的婆罗门若欲供奉,应适时给予他们供品。}

Samitāvino vītarāgā akopā, yesaṃ gatī natthi’dha vippahāya;\\
kālena tesu habyaṃ pavecche, yo brāhmaṇo puññapekkho yajetha. %\hfill\textcolor{gray}{\footnotesize 14}

\begin{enumerate}\item \textbf{离于贪染\footnote{即第 504 颂。}……平静},即令烦恼止息之义。且由平静故,\textbf{离于贪染而无忿恨}。\textbf{于此舍弃已……},即是说舍弃了此世间的现在诸蕴,没有此后的趣向。在此之后,有人说这颂尚有「\textbf{舍弃了爱欲,无家而行,善加自制,如梭子般正直}\footnote{同第 503 颂的上半颂。}」。\end{enumerate}

\subsection\*{\textbf{506} {\footnotesize 〔PTS 500〕}}

\textbf{「无余舍弃了生死,超越一切疑惑,\\}
\textbf{「希求福德的婆罗门若欲供奉,应适时给予他们供品。}

Jahitvā jātimaraṇaṃ asesaṃ, kathaṅkathiṃ sabbam upātivattā;\\
kālena tesu habyaṃ pavecche, yo brāhmaṇo puññapekkho yajetha. %\hfill\textcolor{gray}{\footnotesize 15}

\subsection\*{\textbf{507} {\footnotesize 〔PTS 501〕}}

\textbf{「若以自作洲,行于世间,无所牵绊,于一切处解脱,\\}
\textbf{「希求福德的婆罗门若欲供奉,应适时给予他们供品。}

Ye attadīpā vicaranti loke, akiñcanā sabbadhi vippamuttā;\\
kālena tesu habyaṃ pavecche, yo brāhmaṇo puññapekkho yajetha. %\hfill\textcolor{gray}{\footnotesize 16}

\begin{enumerate}\item \textbf{以自作洲},即唯于自身的功德为自己作洲而行,即是说漏尽者。\end{enumerate}

\subsection\*{\textbf{508} {\footnotesize 〔PTS 502〕}}

\textbf{「若于此,如其所是地了知,『这是最后,而无再有』,\\}
\textbf{「希求福德的婆罗门若欲供奉,应适时给予他们供品。}

Ye h’ettha jānanti yathā tathā idaṃ, ‘ayam antimā natthi punabbhavo’ ti;\\
kālena tesu habyaṃ pavecche, yo brāhmaṇo puññapekkho yajetha. %\hfill\textcolor{gray}{\footnotesize 17}

\begin{enumerate}\item 其义为:\textbf{若于此}蕴、处等的相续,\textbf{如其}蕴、处等\textbf{所是地了知},即觉知其本来的自性,以无常等了知「\textbf{这是最后,而无再有}」,即他们如是了知:这是我们的最后生,现在无有再有。\end{enumerate}

\subsection\*{\textbf{509} {\footnotesize 〔PTS 503〕}}

\textbf{「若通达诸明,乐于禅那,具念,得证等觉,众所皈依,\\}
\textbf{「希求福德的婆罗门若欲供奉,应适时给予他供品。」}

Yo vedagū jhānarato satīmā, sambodhipatto saraṇaṃ bahūnaṃ;\\
kālena tamhi habyaṃ pavecche, yo brāhmaṇo puññapekkho yajetha”. %\hfill\textcolor{gray}{\footnotesize 18}

\begin{enumerate}\item 现在,世尊就自己说了此颂。这里,\textbf{具念},即具足六常住念\footnote{六常住念 \textit{cha-satata-vihāra-sati}:见\textbf{增支部}第 4:195 经。}。\textbf{得证等觉},即得证一切知性。\textbf{众所皈依},即因破除众天人的怖畏而作为皈依。\end{enumerate}

\subsection\*{\textbf{510} {\footnotesize 〔PTS 504〕}}

\textbf{「确实,我的问题并非徒劳,」摩伽学童说,「世尊告知了我应供者,\\}
\textbf{「于此,你如其所是地了知,因为这法已如是为你所知。}

“Addhā amoghā mama pucchanā ahu, \textit{(iti Māgho māṇavo)} akkhāsi me Bhagavā dakkhiṇeyye;\\
tvañ h’ettha jānāsi yathā tathā idaṃ, tathā hi te vidito esa dhammo. %\hfill\textcolor{gray}{\footnotesize 19}

\begin{enumerate}\item 如是听闻了应供者,婆罗门心满意足,说了此颂。这里,\textbf{于此,你如其所是地了知},即于此世间,你如其所是地了知这一切应知,即是说确切地了知、如其所然地了知。\textbf{因为这法已如是为你所知},即因为这法界已为你善通达,由善通达此故,你能了知你所希望者。\end{enumerate}

\subsection\*{\textbf{511} {\footnotesize 〔PTS 505〕}}

\textbf{「若应请者、施主、在家人,希望福德、希求福德而作供奉,\\}
\textbf{「于此布施饮食给他人,世尊!请告知我供奉的成就。」}

Yo yācayogo dānapati gahaṭṭho, puññatthiko yajati puññapekkho;\\
dadaṃ paresaṃ idha annapānaṃ, akkhāhi me Bhagavā yaññasampadaṃ”. %\hfill\textcolor{gray}{\footnotesize 20}

\begin{enumerate}\item 如是,这婆罗门赞叹了世尊,知晓了由应供者具足的供奉之成就,还欲听闻施者具足的六支圆满的供奉之成就,便以此颂进一步发问。此处,其连结为:若应请者以布施他人而作供奉,世尊!请告知我其供奉的成就。\end{enumerate}

\subsection\*{\textbf{512} {\footnotesize 〔PTS 506〕}}

\textbf{「你若作供奉,当供奉时,摩伽!」世尊说,「于一切处,应使心净喜,\\}
\textbf{「供奉者的供品是所缘,住立于此后,舍弃过失。}

“Yajassu yajamāno, \textit{(Māghā ti Bhagavā)} sabbattha ca vippasādehi cittaṃ;\\
ārammaṇaṃ yajamānassa yañño, ettha patiṭṭhāya jahāti dosaṃ. %\hfill\textcolor{gray}{\footnotesize 21}

\begin{enumerate}\item 于是,世尊以两颂向其告知。这里,其意义的连结为:\textbf{你若作供奉,摩伽!当供奉时,于一切处,应使心净喜},在三时中,应使心净喜。如是,具足以\begin{quoting}在布施前欢喜,在布施时使心净喜,\\布施后心满意足,这是供奉之成就。(增支部第 6:37 经)\end{quoting}所说的供奉之成就的供奉得成。这里,设问:如何心得净喜?以舍弃过失。如何舍弃过失?以供奉的所缘。
\item 因为\textbf{供奉者的供品是所缘,住立于此后,舍弃过失},因为以对众有情的慈为先导、以正见之光破除愚痴之暗的心,供奉者的供品——被称为所施——是所缘,他住立于以所缘转起的供品,舍弃以所施为缘的贪、以接受者为缘的忿与此二者之因的痴等如是三种过失。\end{enumerate}

\subsection\*{\textbf{513} {\footnotesize 〔PTS 507〕}}

\textbf{「他离于贪染,去除嗔恨,培育着无量的慈心,\\}
\textbf{「昼夜持续而不放逸,向一切方向无量地遍满。」}

So vītarāgo pavineyya dosaṃ, mettaṃ cittaṃ bhāvayam appamāṇaṃ;\\
rattindivaṃ satatam appamatto, sabbā disā pharati appamaññaṃ”. %\hfill\textcolor{gray}{\footnotesize 22}

\begin{enumerate}\item 他如是于财富\textbf{离于贪染},且于有情\textbf{去除嗔恨},唯由舍弃这些,断除五盖者渐次以遍满无量有情,或以对一有情无余遍满,\textbf{培育着}近行、安止等\textbf{无量的慈心},又为使修习广大,\textbf{昼夜持续而}于一切威仪路\textbf{不放逸},\textbf{向一切方向无量地遍满}这称为慈的禅那。\end{enumerate}

\subsection\*{\textbf{514} {\footnotesize 〔PTS 508〕}}

\textbf{「谁被净化、被解脱、被束缚?以何缘由去往梵界?\\}
\textbf{「牟尼!既然问到,请对无知的我说!因为就我今天所见,世尊就是梵天作证,\\}
\textbf{「因为你对我们,真的等同于梵天,如何投生梵界?放光者!」}

“Ko sujjhati muccati bajjhatī ca, ken’attanā gacchati brahmalokaṃ;\\
ajānato me muni brūhi puṭṭho, Bhagavā hi me sakkhi brahm’ajja diṭṭho;\\
tuvañ hi no brahmasamo si saccaṃ, kathaṃ upapajjati brahmalokaṃ jutima”. %\hfill\textcolor{gray}{\footnotesize 23}

\begin{enumerate}\item 然后,婆罗门不知这慈「即是梵界之道」,听闻了这完全超过了自身境界的慈修习,更加对世尊生起对一切知者的尊敬,由自身决意于梵界故,且认为唯投生梵界为清净与解脱,为问梵界之道,说了此颂。
\item 且此处,就造作趣向梵界的福德,而说\textbf{谁被净化、被解脱},就不造作而说\textbf{被束缚}。\textbf{真的},即就世尊的同梵性,极其恭敬地发誓。\textbf{如何投生},即仍极其恭敬地再次发问。\textbf{放光者},即称呼世尊。\end{enumerate}

\subsection\*{\textbf{515} {\footnotesize 〔PTS 509〕}}

\textbf{「若以三种供奉的成就而作供奉,摩伽!」世尊说,「这样的人当藉由应供者成功,\\}
\textbf{「如是正当地供奉已,应请者投生到梵界,我说。」}

“Yo yajati tividhaṃ yaññasampadaṃ, \textit{(Māghā ti Bhagavā)} ārādhaye dakkhiṇeyyebhi tādi;\\
evaṃ yajitvā sammā yācayogo, upapajjati brahmalokan ti brūmī” ti. %\hfill\textcolor{gray}{\footnotesize 24}

\begin{enumerate}\item 这里,因为若比丘以慈生起三、四禅后,以此为基础而修观,则证阿罗汉,他即被净化、被解脱,且如此便不能去往梵界。然而,若以慈生起三、四禅后,以「这等至实是寂静」等方法味著于彼,他即被束缚,且不退失禅那者以此禅那去往梵界。所以,世尊不认可被净化、被解脱者之趣向梵界,便不提及此补特伽罗,为显示被束缚者以此禅那趣往梵界,对婆罗门以适当的方法说了此颂。
\item 这里,\textbf{三种},是就三时的净喜而说,以此从施者的角度显示三支。\textbf{这样的人当藉由应供者成功},即这样圆满了三种成就的人,当藉由应供的漏尽者,成就、圆满三种供奉的成就,以此从受者的角度显示三支。\textbf{如是正当地供奉已,应请者},即如是以慈禅为足处,正当地作具足六支的供奉,这应请者以此六支供奉为近依的慈禅而\textbf{投生到梵界,我说},即教导婆罗门,完成了开示。其余一切颂中之义自明。且此后仍如先前所述。\end{enumerate}

\textbf{如是说已,摩伽学童对世尊说:「希有!乔达摩君!……从今起,尽寿命,请乔达摩君受持我皈依!」}

Evaṃ vutte, Māgho māṇavo Bhagavantaṃ etad avoca: “abhikkantaṃ, bho Gotama…pe… ajjatagge pāṇupetaṃ saraṇaṃ gatan” ti.

\begin{center}\vspace{1em}摩伽经第五\\Māghasuttaṃ pañcamaṃ.\end{center}