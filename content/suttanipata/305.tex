\section{摩伽经}

\textbf{如是我闻。一时世尊住王舍城耆阇崛山。于是,摩伽学童往世尊处走去,走到后,问候了世尊,彼此寒暄已,坐在一边。坐在一边的摩伽学童对世尊说:}

\begin{enumerate}\item 缘起为何?这在其因缘中已述。因为这摩伽学童是施者、施主,他想到:「施与到来的穷人和旅人是否有大果报呢?我要去问问沙门乔达摩,据说沙门乔达摩知道过去、未来和现在。」他便前往世尊处提问,而世尊便据问题向他作了解答。在汇集了结集者、婆罗门、世尊等三者的话后,此即是「摩伽经」。
\item 这里,\textbf{王舍城},即如是名称的城市。因为由曼达多、大典尊等(王)所拥有故,被称为王舍城。人们还以其它方式来解释,何妨?这只是那城市的名称。然而,它只在佛时和转轮王时才是城市,在其余时间则空如而为夜叉占据,作为它们的春林\footnote{春林 \textit{vasantavana}:菩提比丘注 1411 引\textbf{律藏疏}云,即夜叉春时游戏之处。}而存在。如是显示了行处村落后,便说所住之处:\textbf{耆阇崛山}。当知其因群鹫居于其巅,或其山巅肖似灵鹫,所以被称为「耆阇崛\footnote{耆阇崛 \textit{Gijjhakūṭa}:意译即鹫峰。}」。
\item \textbf{于是……说},此中,\textbf{摩伽}是这婆罗门的名字。\textbf{学童},即以未越过弟子的状态而得称,即便年岁已老。也有人说是因先前的习行,如褐者学童\footnote{褐者学童 \textit{Piṅgiya māṇava}:即\textbf{彼岸道品}中的十六学童之一。}。因为他即使已有百二十岁,仍因先前的习行被认作「褐者学童」。其余已述。\end{enumerate}

\textbf{「乔达摩君!我是施者、施主、慷慨的应请者,我如法地寻求财富,如法地寻求财富后,把如法所获、如法所得的财富,施与一人,施与二、三、四、五、六、七、八、九、十人,施与二十、三十、四十、五十人,施与一百人,施与更多。乔达摩君!我如是布施、如是供奉,能带来许多福德吗?」}

\begin{enumerate}\item \textbf{乔达摩君!我是……能带来许多福德吗},此中,\textbf{施者、施主},若因受命而布施别人的财产,他即是施者,但由对此所施并无所有权,故非施主,而他唯布施自己的财产,因此说「乔达摩君!我是施者、施主」。这即此中之义,但在别处,有以「间或被悭吝克服者为施者,未被克服者为施主」等方法而说者,这也合适。\textbf{慷慨的},即甫一开口,我就因明确个体的特点或因把握众多资助之相,知晓乞求者的话语:「他需要此,他需要彼。」\textbf{应请者},即适于乞求。显明「若一见到乞求者,便现高傲,说粗恶语等,他即非应请者,而我不如此」。
\item \textbf{如法},即避免了不与取、虚诳、欺诈等而行乞、乞求之义。因为乞求是婆罗门寻求财富之法,且这些乞求者从欲行摄受的他人所得的财富名为如法所获、如法所得,而他即如是寻求而得,因此说「\textbf{如法地寻求财富……如法所得的财富}」。\textbf{施与更多},即施与较此更上,无有限量,于此显示以所获的财富之量布施。\end{enumerate}

\textbf{「确实,学童!你如是布施、如是供奉,能带来许多福德。学童!若施者、施主、慷慨的应请者,如法地寻求财富,如法地寻求财富后,把如法所获、如法所得的财富,施与一人……施与一百人,施与更多,他能带来许多福德。」}

\begin{enumerate}\item \textbf{确实},即必然之语的不变词。因为布施,乃至施与畜生,必然为一切佛、辟支佛、声闻所赞赏。如说:\begin{quoting}布施在一切处被称赞,布施不为任何人谴责。\end{quoting}所以,世尊也必然赞赏他,而说「\textbf{确实,学童!你……带来许多福德}」。余义自明。\end{enumerate}

\textbf{于是,摩伽学童以偈颂对世尊说:}

\begin{enumerate}\item 当如是为世尊说到「他能带来许多福德」时,婆罗门欲闻应供者的供养清净,便进一步问了世尊,因此结集者们说:「\textbf{于是,摩伽学童以偈颂对世尊说。}」其义已述。\end{enumerate}

\subsection\*{\textbf{492} {\footnotesize 〔PTS 487〕}}

\textbf{「我问慷慨的乔达摩,」摩伽学童说,「身著袈裟,无家而行,\\}
\textbf{「若应请者、施主、在家人,希望福德、希求福德而作供奉,\\}
\textbf{「于此布施饮食给他人,供奉者的供品如何得到净化?」}

\begin{enumerate}\item 而在「我问」等颂中,\textbf{慷慨},即知语\footnote{慷慨 \textit{vadaññū}:义注解作「知语」,是认为该词来自 vada + jñā,但 Norman 说该词其实来自梵语 vadānya,即慷慨之义。},即是说以一切行相知晓有情所说之语的意趣。\textbf{得到净化},即藉由应供者而得成清净、大果报。而此中的连结为:若应请者、施主、在家人希望福德,以布施饮食给他人而作供奉,不是仅把祭品投入火中,且唯希求福德,而非希求回报、好的名称声望等,他这样供奉者的供品如何能得净化?\end{enumerate}

\subsection\*{\textbf{493} {\footnotesize 〔PTS 488〕}}

\textbf{「若应请者、施主、在家人,摩伽!」世尊说,「希望福德、希求福德而作供奉,\\}
\textbf{「于此布施饮食给他人,这样的人当藉由应供者成功。」}

\begin{enumerate}\item \textbf{这样的人当藉由应供者成功},即这样的应请者当藉由应供者成功、成就、清净,令此供品有大果报,而非其它之义。以此作「供奉者的供品如何得到净化」的解答。\end{enumerate}

\subsection\*{\textbf{494} {\footnotesize 〔PTS 489〕}}

\textbf{「若应请者、施主、在家人,」摩伽学童说,「希望福德、希求福德而作供奉,\\}
\textbf{「于此布施饮食给他人,世尊!请告知我应供者!」}

\begin{enumerate}\item \textbf{世尊!请告知我应供者},此中当知如是连结:若应请者布施以供奉他人,世尊!请告知我应供者。\end{enumerate}

\subsection\*{\textbf{495} {\footnotesize 〔PTS 490〕}}

\textbf{「若无取著而行于世间,无所牵绊、整全、克己,\\}
\textbf{「希求福德的婆罗门若欲供奉,应适时给予他们供品。\footnote{此经第 495~509 的后半颂,同\textbf{孙陀利迦婆罗豆婆遮经}第 468~471 的后半颂,唯彼处作「祭祀、祭品」,此处则随上下文作「供奉、供品」。}}

\begin{enumerate}\item 然后,世尊为以种种品类的方法对其阐明应供者,说了以「若无取著」开头的几颂。这里,\textbf{无取著},即不以贪等染著而固著。\textbf{整全},即已完结了义务。\textbf{克己},即守护心。\end{enumerate}

\subsection\*{\textbf{496} {\footnotesize 〔PTS 491〕}}

\textbf{「若一切结缚与束缚已断,调御、解脱、无患、无待,\\}
\textbf{「希求福德的婆罗门若欲供奉,应适时给予他们供品。}

\begin{enumerate}\item 以无上调御而\textbf{调御},以慧与心的解脱而\textbf{解脱},以无未来流转之苦而\textbf{无患},以无现今的烦恼而\textbf{无待}。\end{enumerate}

\subsection\*{\textbf{497} {\footnotesize 〔PTS 492〕}}

\textbf{「若一切结缚已解脱,调御、解脱、无患、无待,\\}
\textbf{「希求福德的婆罗门若欲供奉,应适时给予他们供品。}

\begin{enumerate}\item 而此颂的第二颂,当知以显示修习威力的方法而说。此经可以为证:\begin{quoting}诸比丘!对从事修习而安住的比丘,即便不如是希望:「哎!愿我的心以无取从诸漏解脱!」然后,他的心也以无取从诸漏解脱。(增支部第 7:71 经)\end{quoting}\end{enumerate}

\subsection\*{\textbf{498} {\footnotesize 〔PTS 493〕}}

\textbf{「舍弃了贪、嗔、痴,漏尽、梵行已立,\\}
\textbf{「希求福德的婆罗门若欲供奉,应适时给予他们供品。}

\begin{enumerate}\item \textbf{舍弃了贪……伪善与慢\footnote{即第 499 颂。}……不陷于渴爱\footnote{即第 501 颂。}},即不倾向于欲贪等。\end{enumerate}

\subsection\*{\textbf{499} {\footnotesize 〔PTS 494\textit{a-d}〕}}

\textbf{「伪善与慢不住于彼,漏尽、梵行已立,\\}
\textbf{「希求福德的婆罗门若欲供奉,应适时给予他们供品。\footnote{PTS 本取 499、500 的第一句,与「希求福德的婆罗门若欲供奉,应适时给予他们供品」一起,作为其第 494 颂,若据 498 颂的义注来看,当是。}}

\subsection\*{\textbf{500} {\footnotesize 〔PTS 494\textit{e-h}〕}}

\textbf{「若离贪、无我所、无待,漏尽、梵行已立,\\}
\textbf{「希求福德的婆罗门若欲供奉,应适时给予他们供品。}

\subsection\*{\textbf{501} {\footnotesize 〔PTS 495〕}}

\textbf{「他们确实不陷于渴爱,已度暴流,无我所而行,\\}
\textbf{「希求福德的婆罗门若欲供奉,应适时给予他们供品。}

\subsection\*{\textbf{502} {\footnotesize 〔PTS 496〕}}

\textbf{「他们对世上任何,对此世或他世的有与无有,没有渴爱,\\}
\textbf{「希求福德的婆罗门若欲供奉,应适时给予他们供品。}

\begin{enumerate}\item \textbf{渴爱},即色爱等六种。\textbf{有与无有}\footnote{有与无有,见\textbf{蛇经}第 6 颂的译注。},即常或断,或者说是「有的无有」,即是说再有的转生\footnote{再有的转生 \textit{punabbhavābhinibbatti}:PTS 本作「再有的不转生 \textit{punabbhavānabhi°}」。}。\textbf{此世或他世},即对「世上任何」的详说。\end{enumerate}

\subsection\*{\textbf{503} {\footnotesize 〔PTS 497〕}}

\textbf{「舍弃了爱欲,无家而行,善加自制,如梭子般正直,\\}
\textbf{「希求福德的婆罗门若欲供奉,应适时给予他们供品。\footnote{此颂及下颂全同\textbf{孙陀利迦婆罗豆婆遮经}第 469~470 颂。}}

\subsection\*{\textbf{504} {\footnotesize 〔PTS 498〕}}

\textbf{「离于贪染,善等持诸根,如月亮解脱于罗睺的束缚,\\}
\textbf{「希求福德的婆罗门若欲供奉,应适时给予他们供品。}

\subsection\*{\textbf{505} {\footnotesize 〔PTS 499〕}}

\textbf{「平静,离于贪染而无忿恨,于此舍弃已,他们没有趣向,\\}
\textbf{「希求福德的婆罗门若欲供奉,应适时给予他们供品。}

\begin{enumerate}\item \textbf{离于贪染\footnote{即第 504 颂。}……平静},即令烦恼止息之义。且由平静故,\textbf{离于贪染而无忿恨}。\textbf{于此舍弃已……},即是说舍弃了此世间的现在诸蕴,没有此后的趣向。在此之后,有人说这颂尚有「\textbf{舍弃了爱欲,无家而行,善加自制,如梭子般正直}\footnote{同第 503 颂的前半颂。}」。\end{enumerate}

\subsection\*{\textbf{506} {\footnotesize 〔PTS 500〕}}

\textbf{「无余舍弃了生死,超越一切疑惑,\\}
\textbf{「希求福德的婆罗门若欲供奉,应适时给予他们供品。}

\subsection\*{\textbf{507} {\footnotesize 〔PTS 501〕}}

\textbf{「若以自作洲,行于世间,无所牵绊,于一切处解脱,\\}
\textbf{「希求福德的婆罗门若欲供奉,应适时给予他们供品。}

\begin{enumerate}\item \textbf{以自作洲},即唯于自身的功德为自己作洲而行,即是说漏尽者。\end{enumerate}

\subsection\*{\textbf{508} {\footnotesize 〔PTS 502〕}}

\textbf{「若于此,如其所是地了知,『这是最后,而无再有』,\\}
\textbf{「希求福德的婆罗门若欲供奉,应适时给予他们供品。}

\begin{enumerate}\item 其义为:\textbf{若于此}蕴、处等的相续,\textbf{如其}蕴、处等\textbf{所是地了知},即觉知其本来的自性,以无常等了知「\textbf{这是最后,而无再有}」,即他们如是了知:这是我们的最后生,现在无有再有。\end{enumerate}

\subsection\*{\textbf{509} {\footnotesize 〔PTS 503〕}}

\textbf{「若通达诸明,乐于禅那,具念,得证等觉,众所皈依,\\}
\textbf{「希求福德的婆罗门若欲供奉,应适时给予他供品。」}

\begin{enumerate}\item 现在,世尊就自己说了此颂。这里,\textbf{具念},即具足六常住念\footnote{六常住念 \textit{cha-satata-vihāra-sati}:见\textbf{增支部}第 4:195 经。}。\textbf{得证等觉},即得证一切知性。\textbf{众所皈依},即因破除众天人的怖畏而作为皈依。\end{enumerate}

\subsection\*{\textbf{510} {\footnotesize 〔PTS 504〕}}

\textbf{「确实,我的问题并非徒劳,」摩伽学童说,「世尊告知了我应供者,\\}
\textbf{「于此,你如其所是地了知,因为这法已如是为你所知。}

\begin{enumerate}\item 如是听闻了应供者,婆罗门心满意足,说了此颂。这里,\textbf{于此,你如其所是地了知},即于此世间,你如其所是地了知这一切应知,即是说确切地了知、如其所然地了知。\textbf{因为这法已如是为你所知},即因为这法界已为你善通达,由善通达此故,你能了知你所希望者。\end{enumerate}

\subsection\*{\textbf{511} {\footnotesize 〔PTS 505〕}}

\textbf{「若应请者、施主、在家人,希望福德、希求福德而作供奉,\\}
\textbf{「于此布施饮食给他人,世尊!请告知我供奉的成就。」}

\begin{enumerate}\item 如是,这婆罗门赞叹了世尊,知晓了由应供者具足的供奉之成就,还欲听闻施者具足的六支圆满的供奉之成就,便以此颂进一步发问。此处,其连结为:若应请者以布施他人而作供奉,世尊!请告知我其供奉的成就。\end{enumerate}

\subsection\*{\textbf{512} {\footnotesize 〔PTS 506〕}}

\textbf{「你若作供奉,当供奉时,摩伽!」世尊说,「于一切处,应使心净喜,\\}
\textbf{「供奉者的供奉是所缘,住立于此后,舍弃过失。}

\begin{enumerate}\item 于是,世尊以两颂向其告知。这里,其意义的连结为:\textbf{你若作供奉,摩伽!当供奉时,于一切处,应使心净喜},在三时中,应使心净喜。如是,具足以\begin{quoting}在布施前欢喜,在布施时使心净喜,\\布施后心满意足,这是供奉之成就。(增支部第 6:37 经)\end{quoting}所说的供奉之成就的供奉得成。这里,设问:如何心得净喜?以舍弃过失。如何舍弃过失?以供奉的所缘。
\item 因为\textbf{供奉者的供奉是所缘,住立于此后,舍弃过失},因为以对众有情的慈为先导、以正见之光破除愚痴之暗的心,供奉者的供奉——被称为所施——是所缘,他住立于以所缘转起的供奉,舍弃以所施为缘的贪、以接受者为缘的忿与此二者之因的痴等如是三种过失。\end{enumerate}

\subsection\*{\textbf{513} {\footnotesize 〔PTS 507〕}}

\textbf{「他离于贪染,去除嗔恨,培育着无量的慈心,\\}
\textbf{「昼夜持续而不放逸,向一切方向无量地遍满。」}

\begin{enumerate}\item 他如是于财富\textbf{离于贪染},且于有情\textbf{去除嗔恨},唯由舍弃这些,断除五盖者渐次以遍满无量有情,或以对一有情无余遍满,\textbf{培育着}近行、安止等\textbf{无量的慈心},又为使修习广大,\textbf{昼夜持续而}于一切威仪路\textbf{不放逸},\textbf{向一切方向无量地遍满}这称为慈的禅那。\end{enumerate}

\subsection\*{\textbf{514} {\footnotesize 〔PTS 508〕}}

\textbf{「谁被净化、被解脱、被束缚?以何缘由去往梵界?\\}
\textbf{「牟尼!既然问到,请对无知的我说!因为就我今天所见,世尊就是梵天作证,\\}
\textbf{「因为你对我们,真的等同于梵天,如何投生梵界?放光者!」}

\begin{enumerate}\item 然后,婆罗门不知这慈「即是梵界之道」,听闻了这完全超过了自身境域的慈修习,更加对世尊生起对一切知者的尊敬,由自身决意于梵界故,且认为唯投生梵界为清净与解脱,为问梵界之道,说了此颂。
\item 且此处,就造作趣向梵界的福德,而说\textbf{谁被净化、被解脱},就不造作而说\textbf{被束缚}。\textbf{真的},即就世尊的同梵性,极其恭敬地发誓。\textbf{如何投生},即仍极其恭敬地再次发问。\textbf{放光者},即称呼世尊。\end{enumerate}

\subsection\*{\textbf{515} {\footnotesize 〔PTS 509〕}}

\textbf{「若以三种供奉的成就而作供奉,摩伽!」世尊说,「这样的人当藉由应供者成功,\\}
\textbf{「如是正当地供奉已,应请者投生到梵界,我说。」}

\begin{enumerate}\item 这里,因为若比丘以慈生起三、四禅后,以此为基础而修观,则证阿罗汉,他即被净化、被解脱,且如此便不能去往梵界。然而,若以慈生起三、四禅后,以「这等至实是寂静」等方法味著于彼,他即被束缚,且不退失禅那者以此禅那去往梵界。所以,世尊不认可被净化、被解脱者之趣向梵界,便不提及此补特伽罗,为显示被束缚者以此禅那趣往梵界,对婆罗门以适当的方法说了此颂。
\item 这里,\textbf{三种},是就三时的净喜而说,以此从施者的角度显示三支。\textbf{这样的人当藉由应供者成功},即这样圆满了三种成就的人,当藉由应供的漏尽者,成就、圆满三种供奉的成就,以此从受者的角度显示三支。\textbf{如是正当地供奉已,应请者},即如是以慈禅为足处,正当地作具足六支的供奉,这应请者以此六支供奉为近依的慈禅而\textbf{投生到梵界,我说},即教导婆罗门,完成了开示。其余一切颂中之义自明。且此后仍如先前所述。\end{enumerate}

\textbf{如是说已,摩伽学童对世尊说:「希有!乔达摩君!……从今起,尽寿命,请乔达摩君受持我皈依为优婆塞!」}

