\section{瞿迦梨经}

\textbf{如是我闻\footnote{此经旧译见杂阿含经第 1278 经,这里的人名、地狱名等从旧译。}。一时世尊住舍卫国祇树给孤独园。于是,瞿迦梨比丘往世尊处走去,走到后,礼敬了世尊,坐在一边。坐在一边的瞿迦梨比丘对世尊说:「尊者!舍利弗、目犍连是恶欲者,受制于恶欲。」}

\begin{enumerate}\item 缘起为何?此经的缘起唯将于释义中揭晓。且在释义中,\textbf{如是我闻}等已述。而在「\textbf{于是,瞿迦梨……}」等中,谁是这瞿迦梨?他又为什么前往?当答:据说,在瞿迦梨国的瞿迦梨城,瞿迦梨长者之子出家后,在父亲派人修建的寺内定居,名为「小瞿迦梨」,而非提婆达多的弟子,被认作「大瞿迦梨」的婆罗门之子。
\item 据说,世尊住于舍卫城时,二上首弟子与五百比丘一起在人间游行,当接近雨安居开始时,二人想要独居,便遣散了众比丘,持了自己的衣钵,到达此国中的此城,去到此寺。于此,他们与瞿迦梨庆慰已,便对他说:「朋友!我们将在此居住三个月,切莫要告诉任何人!」他答道「善哉」,三个月过后的一天,便早早进城宣告:「你们还不知道二上首弟子来此安居,没人以资具邀请他们!」城民们说:「尊者!你为什么不告诉我们呢?」「难道还要告诉?难道你们没看见两个比丘住着,这俩不是上首弟子吗?」
\item 他们便迅速集合,带来了熟酥、砂糖、衣布等,放在瞿迦梨跟前。他便想:「上首弟子最少欲,得知『以特意的言语而来的利养』则不受用,若不受用则会说『去给原住者』,噫!我让人拿着利养前去!」他便这样做了。二长老看了,得知由特意的言语而来,想「这些资具不适合我们,也不适合瞿迦梨」,便没说「去给原住者」,丢弃后即离开。因此,瞿迦梨生起忧恼:「如何自己不接受,也不让人给我呢?」
\item 二人便来到世尊跟前。而世尊自恣已,若自己不去人间游行,则派遣二上首弟子,说:\begin{quoting}二比丘!为众人的利益去游行!(律藏·大品第 32 段)\end{quoting}等等。这是诸如来的习行。尔时,(世尊)自己不想去,于是便再次派遣他们:「二比丘!你们去游行!」他们与五百比丘一起游行,渐次去到此国中的此城。城民们认得是长老,便准备了供品和资具,在城中搭了帐篷布施供品,并为二长老授予资具。二长老接受后,给了比丘僧团。
\item 瞿迦梨看到后,便想:「他们先前少欲,现在则为贪所胜而生恶欲,先前看起来如少欲、知足、远离一般,他们是恶欲者、现不实功德者、恶比丘。」他去往二长老那里,说「朋友!你们先前像是少欲、知足、远离一般,而现在则成了恶比丘」,便持了衣钵,立刻匆匆离去,想「我要把这事告诉世尊」,朝舍卫国的方向而行,渐次去往世尊处。即此瞿迦梨,以此原因前往,因此说「\textbf{于是,瞿迦梨比丘往世尊处走去}」等等。\end{enumerate}

\textbf{如是说已,世尊对瞿迦梨比丘说:「莫要如是,瞿迦梨!莫要如是,瞿迦梨!应使心净喜于舍利弗、目犍连,瞿迦梨!舍利弗、目犍连是端严者。」第二次……第三次,瞿迦梨比丘对世尊说:「尊者!虽然对我来说,世尊可信、可靠,然而,舍利弗、目犍连实是恶欲者,受制于恶欲。」第三次,世尊对瞿迦梨比丘说:「莫要如是,瞿迦梨!莫要如是,瞿迦梨!应使心净喜于舍利弗、目犍连,瞿迦梨!舍利弗、目犍连是端严者。」}

\begin{enumerate}\item 世尊一见他匆匆而来,经转向便知:「他前来是想骂詈上首弟子。」且当再转向于「能否遮止」时,便见到「不能,前来冒犯了二长老后,定会投生到红莲花地狱」。然而,即便见到如是,为免于「即便听到叱责舍利弗、目犍连也不加遮止」等他人的指责,并为显示指责圣者的大罪,便以「莫要如是」等方法遮止了三次。
\item 这里,\textbf{莫要如是},即莫作是说、莫作是讲之义。\textbf{端严},即善习。\textbf{可信},即作为信的归趣,即是说带来净喜。\textbf{可靠},即作为支持,即是说带来决定。\end{enumerate}

\textbf{于是,瞿迦梨比丘从坐起,礼敬了世尊,右绕而去。而离去后不久,瞿迦梨比丘全身就遍满了芥子大的疹子,长成芥子大后,变成了绿豆大,长成绿豆大后,变成了鹰嘴豆大,长成鹰嘴豆大后,变成了枣核大,长成枣核大后,变成了枣子大,长成枣子大后,变成了余甘子大,长成余甘子大后,变成了未熟的木橘大,长成未熟的木橘大后,变成了木橘大,长成木橘大后即破裂,流出脓血。于是,瞿迦梨比丘即由此病死去。且死去的瞿迦梨比丘于舍利弗、目犍连心怀嫌恨,投生到红莲花地狱。}

\begin{enumerate}\item \textbf{全身就遍满了},即连发尖之量的地方也不留,在全身破了核后,以生发的疹子遍满。这里,因为以佛陀的威力,这样的业在诸佛面前未给予异熟,但在一离开视线所及便给予,所以在他离去后不久便起了疹子。因此说「\textbf{而离去后不久,瞿迦梨……}」。
\item 那么,若问:为什么他不就留在那里?以业的威力。因为给予机会的业势必成熟,使他不能留在那里。他被业的威力所逼迫,从坐起而去。
\item 当它们破裂时,全身如菠萝蜜成熟一般。他因肢体腐烂而遭厄难,沉沦于苦,便躺在祇园入口处。于是,来往听法的人们见到他后,都说:「可耻!瞿迦梨!可耻!瞿迦梨!行为不当,你就以自己的口而遭厄难。」护卫的天人听到后也说了声「可耻」,空居天听到护卫的天人后也如此,直至阿迦腻吒的居处,都生起了一声「可耻」。
\item 且当时,名为都卢的比丘是瞿迦梨的亲教师,证得阿那含果后便转生于净居天,他也从等至出起,听到这「可耻」,便前来教诫瞿迦梨,为令他于舍利弗、目犍连心生净喜。瞿迦梨仍不接受他的话语,反而冒犯了他,死后便投生到红莲花地狱。因此说「\textbf{于是,瞿迦梨比丘即由此病……红莲花地狱}」。\end{enumerate}

\textbf{于是,容貌殊胜的娑婆主梵天在深夜中照亮了整座祇园,往世尊处走去,走到后,礼敬了世尊,站在一边。然后,这位站在一边的娑婆主梵天对世尊说:「尊者!瞿迦梨比丘已死,而且,尊者!死去的瞿迦梨比丘于舍利弗、目犍连心怀嫌恨,已投生到红莲花地狱。」娑婆主梵天说了这些,说完这些,礼敬了世尊,右绕后,即于彼处隐没。}

\begin{enumerate}\item 这梵天是谁?又为什么前往世尊处说了这些?他在迦叶世尊的教法下,是名为 娑诃迦的比丘,成阿那含后生于净居天,在那里人们就认他作「娑婆主梵天」。而他想「我当前往世尊处提及红莲花地狱,随后,世尊将告知诸比丘,善巧于接续谈话的比丘将问彼处的寿量,世尊为作宣说,将阐明指责圣者的过患」,以此原因前往世尊处,说了这些。世尊正是如此而为,某比丘也作了提问,且因此问,便说了「\textbf{比丘!好比……}」等等。\end{enumerate}

\textbf{于是,世尊在是夜过后,告诸比丘:「诸比丘!是夜,娑婆主梵天在深夜中……诸比丘!娑婆主梵天说了这些,说完这些,右绕我后,即于彼处隐没。」如是说已,某比丘对世尊说:「尊者!红莲花地狱中的寿量有多长?」「长哉!比丘!红莲花地狱中的寿量,不能简单地用『若干年、若干百年、若干千年、若干百千年』来计算。」「尊者!那可以打个比方吗?」}

\textbf{「可以,比丘!」世尊说,「比丘!好比装在㤭萨罗大车上的二十石芝麻,随后,有人每过一百年取走一粒芝麻,比丘!以此方式,这装在㤭萨罗大车上的二十石芝麻会更快些趋于耗尽,而非一阿浮陀地狱。比丘!如是二十阿浮陀地狱为一尼罗浮陀地狱,二十尼罗浮陀地狱为一阿波波地狱,二十阿波波地狱为一阿吒吒地狱,二十阿吒吒地狱为一阿休休地狱,二十阿休休地狱为一白睡莲地狱,二十白睡莲地狱为一香睡莲地狱,二十香睡莲地狱为一青莲花地狱,二十青莲花地狱为一白莲花地狱,二十白莲花地狱为一红莲花地狱。比丘!而瞿迦梨比丘于舍利弗、目犍连心怀嫌恨,已投生到红莲花地狱。」世尊说了这些。善逝说罢,大师进一步说:}

\begin{enumerate}\item 这里,\textbf{二十石},即以四摩竭陀合为一㤭萨罗合,以此四合为一升,四升为一斗,四斗为一斛,四斛为一石\footnote{这里借用古代的计量单位来翻译 pattha, āḷhaka, doṇa, mānikā, khāri,依次作「合、升、斗、斛、石」,读者略其进制可也。},即以此石作二十石。\textbf{阿浮陀地狱},并没有单独名为阿浮陀的地狱,而是就在无间地狱中,以阿浮陀之数煎熬之处为「阿浮陀地狱」,\textbf{尼罗浮陀}等处也是如此。
\item 这里,亦当如是知此年数:正如一千万为一俱胝,如是一千万俱胝为一般俱胝,一千万般俱胝为一俱胝般俱胝,一千万俱胝般俱胝为一那由他,一千万那由他为一尼那由他,一千万尼那由他为一阿浮陀,随后,二十倍为一尼罗浮陀,一切处都是如此。而有些人说「彼彼处由悲伤之各异、惩处之各异而得此等名称」,另有人说「这些即是寒冷地狱」。
\item \textbf{进一步},是就与显明此义与特殊之义相关的偈颂而说。因为依于文本,在所说的二十颂中,唯有「十万又」一颂\footnote{即第 666 颂。}为显明此义,其余唯显明特殊之义,而最后的二颂在「大义注」中未予裁断,因此我们说「二十颂中」。\end{enumerate}

\subsection\*{\textbf{663} {\footnotesize 〔PTS 657〕}}

\textbf{「对于已生之人,有斧生其口中,\\}
\textbf{「愚人说恶语时,以之砍伤自己。}

\begin{enumerate}\item 这里,\textbf{斧},即以砍伤自己之义而与斧相似的粗恶之语。\textbf{砍伤},即掘断被称为善根的自身之根。\end{enumerate}

\subsection\*{\textbf{664} {\footnotesize 〔PTS 658〕}}

\textbf{「若赞美应受责备者,或责备应受赞美者,\\}
\textbf{「他便以口积累厄运,以此厄运,他不得快乐。}

\begin{enumerate}\item \textbf{或责备应受赞美者},即对以最上之义应受赞美之人,他提出恶欲等指责之。\textbf{厄运},即罪过。\end{enumerate}

\subsection\*{\textbf{665} {\footnotesize 〔PTS 659〕}}

\textbf{「若在投骰子时输财,这厄运是小事,\\}
\textbf{「哪怕是全部,哪怕连自己一起,\\}
\textbf{「若对善逝们生起恶意,这才是更大的厄运。}

\begin{enumerate}\item \textbf{在投骰子时},即在赌博嬉戏投骰子时。\textbf{哪怕是全部,哪怕连自己一起},即哪怕是自己全部的财产,连同自己一起。\textbf{善逝们},即由善妙而行及由行至善妙之处而得名「善逝」的佛、辟支佛、声闻等。\textbf{生起恶意},即是说对他而言,这恶意才是更大的厄运。\end{enumerate}

\subsection\*{\textbf{666} {\footnotesize 〔PTS 660〕}}

\textbf{「十万又三十六尼罗浮陀,及五阿浮陀\footnote{这里的尼罗浮陀、阿浮陀都作为数字解,但和经中先前所述的寿量不合,姑且存疑。},\\}
\textbf{「谴责圣者之人以恶的语、意针对已,进入地狱。}

\begin{enumerate}\item 为什么?因为「十万又……进入地狱」,因为以年数计,有若许时间,\textbf{谴责圣者之人以恶的语、意针对已,进入地狱}这些时间,即是说在那里煎熬。这是略举红莲花地狱中的寿量。\end{enumerate}

\subsection\*{\textbf{667} {\footnotesize 〔PTS 661〕}}

\textbf{「不实语者进入地狱,或做了却说『我没做』的也是,\\}
\textbf{「这两者死后都一样,在别处成为下劣之业的人。}

\begin{enumerate}\item 现在,为以别的方法分别「若对善逝们生起恶意,这才是更大的厄运」之义,说了此颂。这里,\textbf{不实语者},即因指责圣者作虚妄语者。\textbf{地狱},即红莲花等。\textbf{死后都一样},即从此离开后,都一样投生到地狱。\textbf{别处},即别的世间。\end{enumerate}

\subsection\*{\textbf{668} {\footnotesize 〔PTS 662〕}}

\textbf{「若冒犯无恶之人、清净无秽之人,\\}
\textbf{「恶便落回到这愚人,如逆风扬起细尘。}

\begin{enumerate}\item 且更有「若冒犯……」。这里,当知以无有恶意为\textbf{无恶},以无有无明之垢为\textbf{清净},以无有恶欲为\textbf{无秽}。或者此中也可如是连结:由无恶故清净,由清净故无秽。\end{enumerate}

\subsection\*{\textbf{669} {\footnotesize 〔PTS 663〕}}

\textbf{「若从事于各种贪,以言语指责他人,\\}
\textbf{「无信、贪婪、不慷慨、悭吝,从事于诽谤。}

\begin{enumerate}\item 如是证明了对善逝们的恶意有更大的厄运,现在则说名为「遮止事颂\footnote{遮止事颂 \textit{vāritavatthugāthā}:锡兰本与 PTS 作「急速事颂 \textit{turita°}」。}」的十四颂。据说,由大目犍连尊者为教诫将死的瞿迦梨而说,有些则说「由大梵天」。其中,为与此经统摄到一起,而有此总说:「若从事于各种贪」等等。
\item 这里,在第一颂中,先说由被指认为「各种\footnote{guṇa 有「各种」的意思,见\textbf{犀牛角经}第 50 颂的译注。}」故,或由多次转起故,贪本身即\textbf{各种贪},即渴爱的同义语。\textbf{不慷慨}\footnote{义注将「慷慨」解作「知语」,见\textbf{摩伽经}第 492 颂的译注。},即不知语,甚至不接受世尊的教诫。\textbf{悭吝},即以五种悭吝。\textbf{从事于诽谤},即欲分裂上首弟子。其余自明。
\item 这即是说:朋友瞿迦梨!若像你这样从事于各种贪,无信、贪婪、不慷慨、悭吝,从事于诽谤,他就以言语指责其他不应受指责之人,因此我说「险口者」等如下三颂。\end{enumerate}

\subsection\*{\textbf{670} {\footnotesize 〔PTS 664〕}}

\textbf{「险口者!不实者!非圣者!杀胎者!恶者!作恶作者!\\}
\textbf{「人边者!厄运者!贱生者!不要在此多说!你是堕地狱者。}

\begin{enumerate}\item 这是其中非自明之词的语义:\textbf{险口者},即口不正者。\textbf{不实者},即偏离实际、虚妄语者。\textbf{非圣者},即不善人。\textbf{杀胎者},即杀害生命者、摧毁增长者。\textbf{人边者},即边际之人。\textbf{厄运者},即不幸之人。\textbf{贱生者},即佛陀的贱生子。\end{enumerate}

\subsection\*{\textbf{671} {\footnotesize 〔PTS 665〕}}

\textbf{「你反坌尘垢以致不利,你呵责善人,犯罪者!\\}
\textbf{「行了许多恶行,你会长时去向堕处。}

\begin{enumerate}\item \textbf{反坌尘垢},即对自身抛撒烦恼尘垢。\textbf{堕处},即坑渊,文本也作 papātaṃ,其义相同,也作 papadaṃ,意即大地狱。\end{enumerate}

\subsection\*{\textbf{672} {\footnotesize 〔PTS 666〕}}

\textbf{「因为没有人的业会消失,它一来,主人就得到它,\\}
\textbf{「在他世,愚钝的犯罪者在自身中见到苦。}

\begin{enumerate}\item 在「\textbf{它一来}」中,「一 \textit{ha}」字是不变词,「它 \textit{taṃ}」即此善不善业,或者,hataṃ 即已至、已行道,即已积集之义。\textbf{主人},即由造作而为这业的主人。因为他就得到它,即是说他的业不会消失。且因为他得到,所以\textbf{在他世,愚钝的犯罪者……苦}。\end{enumerate}

\subsection\*{\textbf{673} {\footnotesize 〔PTS 667〕}}

\textbf{「他去到铁矛、带有刃口的铁枪击打之处,\\}
\textbf{「然后,食物是与之相当的炽热铁丸。}

\begin{enumerate}\item 现在,为阐明愚钝者所见之苦,说了此颂。这里,先说前半颂之义:就此\textbf{铁矛击打之处},\textbf{他去到}如世尊所说的\begin{quoting}诸比丘!狱吏们对其行以名为「五种捆缚」之刑。(中部·贤愚经第 250 段)\end{quoting}之处,且当如是去时,被令卧在炽燃的铜地上,被狱吏们在五处捶打炽热的被称为标枪的\textbf{带有刃口的铁枪}而去,就此,世尊说:\begin{quoting}他们在手中钻入炽热的铁枪。(同上)\end{quoting}等等。
\item 随后是另半颂:在那里煎熬了若干千年后,为经历剩余的果报,渐次来到灰水河畔,就此而说:\begin{quoting}他们在口中投入炽热的铁丸,在口中洒以炽热的铜水。(同上)\end{quoting}这里,\textbf{铁},即铜。\textbf{丸},即状如木橘。且当知此中以铁摄铜水,以其它摄铁丸。\textbf{与之相当},即随适于所作的业。\end{enumerate}

\subsection\*{\textbf{674} {\footnotesize 〔PTS 668〕}}

\textbf{「当说时,不甜蜜地说,不热心,不作为救护而来,\\}
\textbf{「他们躺在炭火的卧具上,进入烈火燃烧之处。}

\begin{enumerate}\item 在随后的几颂中,\textbf{不甜蜜},即\textbf{当}狱吏们\textbf{说}「捉、打」等\textbf{时},不\textbf{说}甜蜜之语。\textbf{不热心},即不笑脸相对急忙赶来,不笑脸相迎,即是说唯带着厄难前来。\textbf{不作为救护而来},即不是作为救护、庇护、皈依而来,即是说捉着、打着而来。
\item \textbf{他们躺在炭火的卧具上},即当上到炭火之山,他们要在炭火的卧具上躺上数千年。\textbf{烈火燃烧之处},即周遭燃烧,并在所有方向烧有烈火。\textbf{进入},即当被抛进大地狱即沉没。大地狱,即(增支部)所说的「四隅」,站在一百由旬之外,见者的双眼灼裂。\end{enumerate}

\subsection\*{\textbf{675} {\footnotesize 〔PTS 669〕}}

\textbf{「且被网罩住,在那里用铁锤击打,\\}
\textbf{「他们去到弥漫如大雾般的盲目的暗黑。}

\begin{enumerate}\item \textbf{且被网罩住},即被铁网包裹后,如猎鹿人击打鹿一般,这是(中部)天使经中未提及的刑罚。\textbf{他们去到盲目的暗黑},即他们去往以令不可见为\textbf{盲目}、由浓厚的黑暗而被称为\textbf{暗黑}的名为「烟唤」的地狱。据说,他们在那里嗅到刺鼻的烟后,双眼裂开,因此说是「盲目」。\textbf{弥漫如大雾般},即这盲目的暗黑如大雾般弥漫之义。文本也作 vitthatañ。这也是天使经中未提及的刑罚。\end{enumerate}

\subsection\*{\textbf{676} {\footnotesize 〔PTS 670〕}}

\textbf{「然后,进入烈火燃烧的铜制的釜中,\\}
\textbf{「在其中长时地煎熬,在火堆中翻滚。}

\begin{enumerate}\item 这铜釜以大地为边界,四那由他及二十万由旬深,其中充满了铜。\textbf{翻滚},即是说忽上忽下地行进,煎熬至生起泡沫。当知这唯如天使经中所述。\end{enumerate}

\subsection\*{\textbf{677} {\footnotesize 〔PTS 671〕}}

\textbf{「然后,在脓血混杂中,犯罪者在那里被煎熬,\\}
\textbf{「举凡所到之处,在那里,触碰都是折磨。}

\begin{enumerate}\item \textbf{在脓血混杂中},即在脓血混杂的铜釜中。\textbf{所到},即所行。文本也作 abhiseti,即举凡所依、所靠之处之义。\textbf{折磨},即受逼恼。文本也作 kilijjati,即成腐烂之义。\textbf{触碰},即被这脓血所触。这也是天使经中未提及的刑罚。\end{enumerate}

\subsection\*{\textbf{678} {\footnotesize 〔PTS 672〕}}

\textbf{「在虫聚的水中,犯罪者在那里被煎熬,\\}
\textbf{「连可去的堤岸都没有,因为整个锅全一样。}

\begin{enumerate}\item \textbf{虫聚},即虫的住处。在天使经中,这铜釜也被称为「粪地狱」,在那里,蚊子般的生类咬开跌落者的皮肤等,噬其骨髓。\textbf{连可去的堤岸都没有},即没有可以离开的堤岸。文本也作 tīravam atthi,其义相同。\textbf{因为整个锅全一样},因为由于被倒置在此釜的上半部分,整个便器各处都一样,所以没有可以离开的堤岸。\end{enumerate}

\subsection\*{\textbf{679} {\footnotesize 〔PTS 673〕}}

\textbf{「他们进入锋利的剑叶林,肢体被完全割截,\\}
\textbf{「用钩子钩住舌头,反复切割后再击打。}

\begin{enumerate}\item \textbf{剑叶林}唯如天使经中所述。因为它从远处看,如迷人的芒果林,于是,地狱众生因贪进入其中,随后,它们的叶子因风摇落,割截肢体,因此说「\textbf{他们进入……肢体被完全割截}」。\textbf{用钩子钩住舌头,反复切割后再击打},狱吏们迅速跑进那剑叶林中,用钩子拖出跌落的妄语地狱众生的舌头,好比人们在地上敷展新鲜的兽皮,用木桩捶打,如是捶打已,用斧劈开,切出一块块后击打,而被切出的块又再再生长。文本也作 āracayāracayā,即反复回转之义。这也是天使经中未提及的刑罚。\end{enumerate}

\subsection\*{\textbf{680} {\footnotesize 〔PTS 674〕}}

\textbf{「然后,他们去到带着刃口、带着刀锋的危险的灰河,\\}
\textbf{「愚钝的作恶者作了恶,在那里坠落。}

\begin{enumerate}\item \textbf{灰河},即天使经中所说的河:大灰水河。据说,它视如恒河,水流洪大。于是,地狱众生想着「我们要洗澡、我们要喝水」,便坠落其中。\textbf{带着刃口、带着刀锋},据说,此河上下两岸,如依次置有带着刃口的刀锋而立,因此说其「带着刃口、带着刀锋」。因希望水,\textbf{他们去到}这带着刃口、带着刀锋之中,即跟随之义。且当如是而去时,为恶业逼迫,\textbf{愚钝}的愚人\textbf{在那里坠落}之义。\end{enumerate}

\subsection\*{\textbf{681} {\footnotesize 〔PTS 675〕}}

\textbf{「黑色与斑点的狗和渡鸦群撕咬着那里的悲泣者,\\}
\textbf{「还有贪婪的豺,而秃鹫和鸦则啄击着。}

\begin{enumerate}\item \textbf{黑色与斑点},这应与\textbf{狗}结合,即是说黑色与斑点色的狗在撕咬。\textbf{渡鸦群},即黑色的乌鸦群。\textbf{贪婪},即强烈地生起贪求,有人说作「大鹫 \textit{mahāgijjhā}」。\textbf{鸦},即非黑色的乌鸦。这也是天使经中未提及的刑罚,而在那里已述者则在此未提及,当知这些作为此等的前后分亦已提及。\end{enumerate}

\subsection\*{\textbf{682} {\footnotesize 〔PTS 676〕}}

\textbf{「于此,这犯罪之人所得的生活实在是艰难,\\}
\textbf{「所以于此,在余下的生命中人应尽义务,切莫放逸!}

\begin{enumerate}\item 现在,在显示了所有地狱生活后,为作教诫,说了此颂。其义为:\textbf{于此}地狱,\textbf{这犯罪之人所得的}种种品类刑罚的\textbf{生活实在是艰难},\textbf{所以于此,在余下的生命中},即因命相续的存在,尚住立于此世间时,\textbf{人应}以履行皈依等的善业而\textbf{尽义务}。且当行义务时,唯应以常恒之行而行,\textbf{切莫放逸},不应犯须臾的放逸。
\item 这是此中积累的释义。而因为所说之余的字词由所说之理及意义自明之故,极易知晓,所以未作逐词的解释。\end{enumerate}

\subsection\*{\textbf{683} {\footnotesize 〔PTS 677〕}}

\textbf{「陷入红莲花地狱者,已由知者们计算其芝麻量,\\}
\textbf{「计有五那由他俱胝,以及另外的一千二百俱胝。}

\subsection\*{\textbf{684} {\footnotesize 〔PTS 678〕}}

\textbf{「于此所说的地狱之苦,当在那里长久居住,\\}
\textbf{「所以,在清净、端严的善德者中,应常常守护语意。」}

