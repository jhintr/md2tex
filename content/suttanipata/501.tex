\chapter{彼岸道品第五}

\section{序颂}

\begin{center}Vatthugāthā\end{center}\vspace{1em}

\begin{enumerate}\item 其缘起为:据说,在过去,有一个住在波罗奈的木匠,在其师承中独一无二,他的十六个学生各有一千个弟子。如是,这一万六千又十七个师弟全都依波罗奈谋生,去到山边取了木材,即于此制成种种楼阁的构件,绑在筏上,沿恒河运到波罗奈,如果国王需要,他们就拼接成一层乃至七层的楼阁交给国王,若不然,就卖给别人,养育妻儿。
\item 然后有一天,他们的老师想「不能总是靠木匠活谋生,因为到了老年,这活就难做了」,便告诸弟子:「徒儿们!你们拿些无花果树等不太硬的木材来!」他们答道「善哉」,就拿来了。他用这些做了木鸟,进入其中,开启了机关。木鸟如金翅鸟王一般,跃入空中,在林上飞行后,降落在弟子们跟前。于是,老师对学生们说:「徒儿们!造了这样的木艇,便能得到整个阎浮提洲的统治,徒儿们!你们也去造这些,得了统治后,我们将能活命,靠木匠手艺而活太苦。」他们这样做完,便报告给老师。随后,老师对他们说:「徒儿们!我们去夺取哪个国家呢?」「波罗奈国,老师!」「别!徒儿们!别琢磨这个,因为我们即便夺取了它,也免不了『木匠王、木匠小王』这样的木匠称呼,阎浮提甚大,我们去别处!」
\item 随后,他们带着妻儿登上木艇,全副武装,朝着雪山而去,进入雪山中某个城市后,即现身在国王的住处。他们在此夺取了统治,将老师灌顶为国王。他即以「木艇国王」而知名,而这城市也由于被他夺取而得名「木艇城」,整个王国同样也此。木艇国王如法而治,十六个学生小王被任命为大臣,也同样如此。这王国被国王以四摄事所摄受,而极度富有、繁荣、免于灾厄。市民、国民极度珍重国王和王众:「我们所得的国王贤明,王众们贤明。」
\item 然后有一天,商人们从中土带了货物来到木艇城,并带了礼物,觐见了国王。国王便问「你们从哪里来」等一切。「陛下!从波罗奈。」他于此问了一切本末,说:「请向你们的国王致以我的友于之意。」他们领命:「善哉!」他给了他们行资,到出发时,再申敬意,便即放行。他们到了波罗奈,便告诉国王。国王命人击鼓「从今起,我免除从木艇国来的商旅的税收」,想「让木艇成为我的朋友」,两人便成了未见面的朋友。
\item 木艇也在自己的城内命人击鼓「从今起,我免除从波罗奈来的商旅的税收,且应给他们行资」。随后,波罗奈王遣使送信:「若在你的国土内发生任何可观或可闻的希有之事,也让我们与观、与闻。」他也同样遣使回信。他们如是订好协议后,在春天的某个时节,木艇得到一些极昂贵、极柔软的毛毯,有着如同朝日的光辉般的色泽。国王见到这些后,想「送给我的朋友」,命象牙匠们雕了八个象牙盒,将毛毯放入这些盒中,命胶师们造了外围相同大小的胶球,再将八个胶球放入箱中,用布包裹好,封以王家玺印,派遣大臣道:「交给波罗奈国王!」并送信说:「这礼物应在城市中央、为大臣围观而欣赏。」
\item 他们去后,便交给波罗奈国王。他命人宣读了信,召集众大臣后,在城市中央的王家庭院里揭开封印、拆除包裹、打开箱子,看见八个胶球后,想「我的朋友像给玩耍胶球的孩子们一样,派人送给我胶球」而羞恼,在自己的坐处击碎了一个胶球。胶立即裂开,象牙盒开了口子,分成两半。他看到里面的毛毯后,打开了其它的,也都是同样。每条毛毯长十六肘、宽八肘。当毛毯展开时,王家庭院像是被阳光照耀到一样。大众们看到后,弹指挥衣,高兴地说道:「未见面的朋友、木艇国王送给我们国王这样的礼物,交这样的朋友是合适的。」
\item 国王命人传唤仲裁官,让他们对每条毛毯估价,全都是无价之宝。随后,他想:「后送者应该比先送礼物者送得更多,但朋友送给我的礼物是无价的,那么,我该送什么给朋友呢?」而在那时,迦叶世尊已出世,住于波罗奈。然后,国王想到:「没有比三宝更高贵的宝物了,噫!我派人告诉朋友三宝的出世吧。」他命人将\begin{quoting}佛已出世,为了一切生命的利益,\\法已出世,为了一切生命的快乐,\\僧已出世,即是无上的福田。\end{quoting}一颂以及从阿罗汉乃至一比丘的行道用天然的朱砂写在金箔上,放到七宝制成的箱子里,把这箱子放到摩尼制成的箱子里,摩尼的放到琥珀的里面,琥珀的放到红宝石的里面,红宝石的放到金子的里面,金子的放到银子的里面,银子的放到象牙的里面,象牙的放到心木里面,把心木制成的箱子放到柜子里,用白布包裹、封印好柜子,命人在最光鲜的大象上置以金旗、饰以金饰、覆以金网,施以台座后,将柜子放在台座上,命人持白伞,作一切香花等供养,伴着一切铙钹鼓奏,歌诵着数百赞诗,命人严饰道路,直至自己的国境,亲率而往。站在那里后,送礼给邻国国王道:「应以如是的恭敬致送这礼物。」那些国王们听后,都逆路迎接,一直送至木艇的国境。
\item 木艇听闻后,也逆路迎接,同样供养,请进城里,召集了大臣和市民,在王家庭院拆除包裹、打开柜子,看到柜子里的箱子,顺次打开所有箱子后,看到金箔上的信而高兴「我的朋友送来了百千劫难遇的宝贝礼物」,想到「我们真是闻所未闻,『佛已出世』,我何不前去见佛并闻法」,便告诉大臣们:「据说佛法僧宝已出世,你们认为该怎么办?」他们说:「大王!请您在此等候,我们前去了解事情的经过。」
\item 随后,十六个大臣为一万六千人所随从,礼拜了国王「如果佛陀出世,就不再相见,如未出世,我们就回来」而离去。国王的外甥随后也顶礼了国王,说:「我也要去。」「亲爱的!你在那里得知佛陀出世后,就再回来告诉我!」他答道「善哉」便出发。他们全都在一切处只休息一晚,便到达波罗奈。但在他们尚未到达时,世尊便入灭了。他们在整个寺庙彷徨,见到亲传弟子后,便问:「谁是佛陀?佛陀在哪里?」弟子们便告诉他们:「佛陀已入灭。」他们悲伤道「哎!我们远道而来,竟然不得一见」,便问:「那么,尊者!世尊有给过什么教诫吗?」「唯!优婆塞!有的,应皈依三宝,应受持五戒,应奉行八支具足的布萨,应给予布施,应出家。」他们听后,除了外甥,全都出了家。
\item 外甥取了受用舍利,便朝着木艇国出发。受用舍利,即菩提树、衣钵等等。他还带了一位行法、持法、持律的长老出发,渐次回到城内,告诉国王「佛陀已出世,且已入灭」,宣告了世尊给予的教诫。国王亲近长老、闻法已,命人造寺、建支提、种菩提树,皈依三宝、五戒等常戒,奉行八支具足的布萨,给予布施等,直至寿尽,转生在欲界天界。那一万六千人出家后,以凡夫死去,仍成为这国王的随从。
\item 他们在天界度过了一佛的间隔,当我们的世尊尚未出世时,从天界下堕,老师出生为波斯匿王之父的祭司之子,名为「波婆利」,具足三大人相,通晓三吠陀,在父亲身后成为祭司。其余的一万六千又十六个也转生在舍卫国的婆罗门家族。其中十六个年长的弟子在波婆利跟前学习技艺,另外的一万六千个则在他们跟前,如是,他们全都再次会合。
\item 大㤭萨罗王死后,他们将波斯匿灌顶为国王。波婆利则成为他的祭司。国王将父亲给予的及其它财富都给了波婆利,因为他还在孩童时,便在他跟前学习过技艺。随后,波婆利告诉国王:「大王!我将出家。」「老师!你在这里,如同我的父亲一样,请别出家!」「止!大王!我将出家。」国王不能遮止,便请求:「为了早晚我能得见,请在王家园林里出家!」老师便为一万六千人所随从,与十六学生一起,出家为苦行僧,住在王家园林里。国王以四事护持,早晚前去给持。
\item 然后有一天,弟子们对老师说:「住在城市附近实为大障碍,老师!让我们到无人聚集的空旷处,住在边鄙的坐卧处实对出家人有大助益。」老师接受道「善哉」,便告诉国王。国王经三次遮止而不能遮止,便给予二十万钱币,命令二位大臣:「众仙人想住在哪里,便在那里建造草庵后布施。」随后,老师为一万六千又十六个萦发者所随从,为二大臣所资助,从北方国土出发,往南方国土而去。尊者阿难摄取了此义,当结集时,提起彼岸道品的因缘,说了以下几颂。\end{enumerate}

\subsection\*{\textbf{983} {\footnotesize 〔PTS 976〕}}

\textbf{从㤭萨罗怡人的城市来到了南方之地,\\}
\textbf{婆罗门通晓颂诗,愿求着无所有。}

Kosalānaṃ purā rammā, agamā dakkhiṇāpathaṃ;\\
ākiñcaññaṃ patthayāno, brāhmaṇo mantapāragū. %\hfill\textcolor{gray}{\footnotesize 1}

\begin{enumerate}\item 这里,\textbf{㤭萨罗的城市},即㤭萨罗国的城市,即是说舍卫国。\textbf{无所有},即无所有性,即是说远离资产与器具。\end{enumerate}

\subsection\*{\textbf{984} {\footnotesize 〔PTS 977〕}}

\textbf{他在阿萨迦的境内、阿罗迦的接壤处\\}
\textbf{居住,在乔陀婆利岸边,靠着采集与果实。}

So Assakassa visaye, Aḷakassa samāsane;\\
vasi Godhāvarīkūle, uñchena ca phalena ca. %\hfill\textcolor{gray}{\footnotesize 2}

\begin{enumerate}\item \textbf{他在阿萨迦的境内、阿罗迦的接壤处},即此婆罗门在阿萨迦与阿罗迦两国\footnote{两国:PTS 本作「两案达罗国 \textit{Andhaka-rājānaṃ}」。}接壤之境的附近国土,意即在两国之间。\textbf{在乔陀婆利岸边},即在乔陀婆利河的岸边。在乔陀婆利分叉后,形成了三由旬之量的三角洲,全都为林檎树林所覆盖,先前萨罗绷伽\footnote{萨罗绷伽,见\textbf{旷野经}缘起部分的注。}等所居之处,即指此地。据说,他(波婆利)看到这地方后,便告知二大臣:「这是先前沙门的住处,适合出家人。」二大臣为了占有此地,给了阿萨迦王、阿罗迦王各十万钱币,他们便给了此地及另外二由旬之量,共五由旬之量的地方。据说,此地在两国国界之间。二大臣教人于此建造了草庵,并让人从舍卫国带来别的财物,建了行处村落,便离开了。\textbf{靠着采集与果实},即靠着搜罗与林木的根果。\end{enumerate}

\subsection\*{\textbf{985} {\footnotesize 〔PTS 978〕}}

\textbf{就在这附近,有一个大村庄,\\}
\textbf{随后,他以所得的收入举行了大祭祀。}

Tass’eva upanissāya, gāmo ca vipulo ahu;\\
tato jātena āyena, mahāyaññam akappayi. %\hfill\textcolor{gray}{\footnotesize 3}

\begin{enumerate}\item 所以说「就在这附近,有一个大村庄」。这里,\textbf{这},即这乔陀婆利岸边,或这婆罗门,且此属格作业格义,即在这附近之义。\textbf{随后,他以所得的收入举行了大祭祀},即在这村中,以耕田等而有百千的收入,地主们收了,便去到阿萨迦王的跟前说:「请陛下接受收入!」他说:「我不该接受,只应供给老师!」老师自己也不收,便举行了布施的祭祀。如是,他便年年都给予布施。\end{enumerate}

\subsection\*{\textbf{986} {\footnotesize 〔PTS 979〕}}

\textbf{举行完大祭祀,便又进入了草庵,\\}
\textbf{当他回去时,另一个婆罗门来到。}

Mahāyaññaṃ yajitvāna, puna pāvisi assamaṃ;\\
tasmiṃ paṭipaviṭṭhamhi, añño āgañchi brāhmaṇo. %\hfill\textcolor{gray}{\footnotesize 4}

\begin{enumerate}\item 此颂之义为:他如是年年布施祭祀,有一年,\textbf{举行完大祭祀},从这村离开,\textbf{便又进入了草庵},且进入茅蓬后,他便坐而细思「所施甚善」。\textbf{当他}如是\textbf{回去时},年轻的婆罗门尼不愿做家务活,说「婆罗门!这波婆利在乔陀婆利岸边每年派发百千,去!从他求上五百,带个女仆给我」,派了\textbf{另一个婆罗门来到}。\end{enumerate}

\subsection\*{\textbf{987} {\footnotesize 〔PTS 980〕}}

\textbf{脚已磨破,口干齿污,头面蒙尘,\\}
\textbf{且他走向他后,乞求五百钱。}

Ugghaṭṭapādo tasito, paṅkadanto rajassiro;\\
so ca naṃ upasaṅkamma, satāni pañca yācati. %\hfill\textcolor{gray}{\footnotesize 5}

\begin{enumerate}\item \textbf{脚已磨破},即因为行路而摩擦脚底,或踵接踵、踝碰踝、膝撞膝磨破了脚。\end{enumerate}

\subsection\*{\textbf{988} {\footnotesize 〔PTS 981〕}}

\textbf{波婆利看到他后,请他坐下,\\}
\textbf{问罢安乐与健康,说了这样的话:}

Tam enaṃ Bāvarī disvā, āsanena nimantayi;\\
sukhañ ca kusalaṃ pucchi, idaṃ vacanam abravi. %\hfill\textcolor{gray}{\footnotesize 6}

\begin{enumerate}\item \textbf{问罢安乐与健康},即问了安乐与健康:「婆罗门!你是否安乐?是否健康?」\end{enumerate}

\subsection\*{\textbf{989} {\footnotesize 〔PTS 982〕}}

\textbf{「我所能布施之物,我已全部派发,\\}
\textbf{「许可我!婆罗门!我没有五百钱。」}

“Yaṃ kho mama deyyadhammaṃ, sabbaṃ visajjitaṃ mayā;\\
anujānāhi me brahme, natthi pañcasatāni me”. %\hfill\textcolor{gray}{\footnotesize 7}

\begin{enumerate}\item \textbf{许可},即准许、相信。\end{enumerate}

\subsection\*{\textbf{990} {\footnotesize 〔PTS 983〕}}

\textbf{「如果对我的乞求,您不作施舍,\\}
\textbf{「在第七天,让你的头裂成七分!」}

“Sace me yācamānassa, bhavaṃ nānupadassati;\\
sattame divase tuyhaṃ, muddhā phalatu sattadhā”. %\hfill\textcolor{gray}{\footnotesize 8}

\subsection\*{\textbf{991} {\footnotesize 〔PTS 984〕}}

\textbf{诡诈者造作后,他扬言恐吓,\\}
\textbf{听到他的这话,波婆利生起苦恼。}

Abhisaṅkharitvā kuhako, bheravaṃ so akittayi;\\
tassa taṃ vacanaṃ sutvā, Bāvarī dukkhito ahu. %\hfill\textcolor{gray}{\footnotesize 9}

\begin{enumerate}\item \textbf{造作},即是说拾了牛粪、野花、亚香茅草等,急急忙忙去到波婆利的草庵门口,以牛粪涂地,撒开花朵,铺好草后,用长口水瓶的水洗了左脚,走了七步,摸着自己的脚底,作了这样的诡诈事。\textbf{他扬言恐吓},即宣扬了生起怖畏的话,意即说了上一颂。\textbf{苦恼},即忧虑。\end{enumerate}

\subsection\*{\textbf{992} {\footnotesize 〔PTS 985〕}}

\textbf{他焦灼、不食,被忧箭射中,\\}
\textbf{然后,对这样的心,意便不喜于禅那。}

Ussussati anāhāro, sokasallasamappito;\\
atho pi evaṃ cittassa, jhāne na ramatī mano. %\hfill\textcolor{gray}{\footnotesize 10}

\begin{enumerate}\item \textbf{焦灼},即想着「他的这话说不定会成真」而焦灼。\end{enumerate}

\subsection\*{\textbf{993} {\footnotesize 〔PTS 986〕}}

\textbf{看到惧怕、苦恼,天人怀着善意,\\}
\textbf{走向波婆利,说了这样的话:}

Utrastaṃ dukkhitaṃ disvā, devatā atthakāminī;\\
Bāvariṃ upasaṅkamma, idaṃ vacanam abravi. %\hfill\textcolor{gray}{\footnotesize 11}

\begin{enumerate}\item \textbf{天人},即住在草庵的天人。\end{enumerate}

\subsection\*{\textbf{994} {\footnotesize 〔PTS 987〕}}

\textbf{「他不了知头,他是希求财产的诡诈者,\\}
\textbf{「对于头或头落,他并没有智。」}

“Na so muddhaṃ pajānāti, kuhako so dhanatthiko;\\
muddhani muddhapāte vā, ñāṇaṃ tassa na vijjati”. %\hfill\textcolor{gray}{\footnotesize 12}

\subsection\*{\textbf{995} {\footnotesize 〔PTS 988〕}}

\textbf{「那么您知道,请告诉我!当被问及\\}
\textbf{「头与头裂,让我们听你的话!」}

“Bhotī carahi jānāsi, taṃ me akkhāhi pucchitā;\\
muddhaṃ muddhādhipātañ ca, taṃ suṇoma vaco tava”. %\hfill\textcolor{gray}{\footnotesize 13}

\begin{enumerate}\item \textbf{那么您知道},即如果您知道。\textbf{头裂},即头落。\end{enumerate}

\subsection\*{\textbf{996} {\footnotesize 〔PTS 989〕}}

\textbf{「我对此也不知,我于此没有智,\\}
\textbf{「对于头与头裂,于此唯是胜者们的知见。」}

“Aham p’etaṃ na jānāmi, ñāṇam ettha na vijjati;\\
muddhani muddhādhipāte ca, Jinānaṃ h’ettha dassanaṃ”. %\hfill\textcolor{gray}{\footnotesize 14}

\subsection\*{\textbf{997} {\footnotesize 〔PTS 990〕}}

\textbf{「那么在这地轮上,谁会知道\\}
\textbf{「头与头裂?请告诉我!天人!」}

“Atha ko carahi jānāti, asmiṃ pathavimaṇḍale;\\
muddhaṃ muddhādhipātañ ca, taṃ me akkhāhi devate”. %\hfill\textcolor{gray}{\footnotesize 15}

\subsection\*{\textbf{998} {\footnotesize 〔PTS 991〕}}

\textbf{「先前从迦毗罗卫出离,世间的导师,\\}
\textbf{「甘蔗王的后裔,释迦之子,放光者。}

“Purā Kapilavatthumhā, nikkhanto lokanāyako;\\
apacco Okkākarājassa, Sakyaputto pabhaṅkaro. %\hfill\textcolor{gray}{\footnotesize 16}

\begin{enumerate}\item \textbf{先前},即在二十九岁的青春时节。当波婆利婆罗门住在乔陀婆利岸边八年后,佛陀便出世。\textbf{后裔},即后代。\end{enumerate}

\subsection\*{\textbf{999} {\footnotesize 〔PTS 992〕}}

\textbf{「婆罗门!他实为等觉者,通晓一切法,\\}
\textbf{「得证一切神通力,于一切法具眼,\\}
\textbf{「得证一切业的灭尽,在依持的灭尽处解脱。}

So hi brāhmaṇa sambuddho, sabbadhammāna pāragū;\\
sabbābhiññābalappatto, sabbadhammesu cakkhumā;\\
sabbakammakkhayaṃ patto, vimutto upadhikkhaye. %\hfill\textcolor{gray}{\footnotesize 17}

\begin{enumerate}\item \textbf{得证一切神通力},即得证一切神通之力,或得证一切神通与力。\textbf{解脱},即作为所缘后,从转起处解脱其心。\end{enumerate}

\subsection\*{\textbf{1000} {\footnotesize 〔PTS 993〕}}

\textbf{「他是佛陀、世尊、具眼者,在世间开示法,\\}
\textbf{「你去后若是问他,他会为你解答。」}

Buddho so Bhagavā loke, dhammaṃ deseti cakkhumā;\\
taṃ tvaṃ gantvāna pucchassu, so te taṃ byākarissati”. %\hfill\textcolor{gray}{\footnotesize 18}

\subsection\*{\textbf{1001} {\footnotesize 〔PTS 994〕}}

\textbf{听到「等觉者」之语,波婆利起了踊跃,\\}
\textbf{他的忧伤变薄了,并得了广大的喜。}

“Sambuddho” ti vaco sutvā, udaggo Bāvarī ahu;\\
sok’assa tanuko āsi, pītiñ ca vipulaṃ labhi. %\hfill\textcolor{gray}{\footnotesize 19}

\subsection\*{\textbf{1002} {\footnotesize 〔PTS 995〕}}

\textbf{这波婆利心满意足、踊跃、充满喜悦,问那天人:\\}
\textbf{「世间的依怙在何村何镇,或在何国土?\\}
\textbf{「去到那里后,我们好看看等觉者、两足尊。」}

So Bāvarī attamano udaggo, taṃ devataṃ pucchati vedajāto;\\
“katamamhi gāme nigamamhi vā pana, katamamhi vā janapade lokanātho;\\
yattha gantvāna passemu, sambuddhaṃ dvipaduttamaṃ”. %\hfill\textcolor{gray}{\footnotesize 20}

\subsection\*{\textbf{1003} {\footnotesize 〔PTS 996〕}}

\textbf{「广慧、最上宏慧的胜者在舍卫国、㤭萨罗的宫城,\\}
\textbf{「这释迦之子离于重担、无漏,人中公牛知道头裂。」}

“Sāvatthiyaṃ Kosalamandire Jino, pahūtapañño varabhūrimedhaso;\\
so Sakyaputto vidhuro anāsavo, muddhādhipātassa vidū narāsabho”. %\hfill\textcolor{gray}{\footnotesize 21}

\begin{enumerate}\item \textbf{广慧},即大慧。\textbf{最上宏慧},即最高的广慧,或为生物喜爱的最上智慧。\textbf{离于重担},即是说无比。\end{enumerate}

\subsection\*{\textbf{1004} {\footnotesize 〔PTS 997〕}}

\textbf{随后,他便招呼学生,通晓颂诗的众婆罗门:\\}
\textbf{「来!学童们!我将宣告,请听我的话!}

Tato āmantayī sisse, brāhmaṇe mantapārage;\\
“etha māṇavā akkhissaṃ, suṇātha vacanaṃ mama. %\hfill\textcolor{gray}{\footnotesize 22}

\begin{enumerate}\item \textbf{通晓颂诗},即通晓吠陀。\end{enumerate}

\subsection\*{\textbf{1005} {\footnotesize 〔PTS 998〕}}

\textbf{「这在世间总是难得出现、\\}
\textbf{「以等觉者著名者,他现在已在世间出现,\\}
\textbf{「速速去到舍卫国,看看两足尊!」}

Yass’eso dullabho loke, pātubhāvo abhiṇhaso;\\
sv ājja lokamhi uppanno, sambuddho iti vissuto;\\
khippaṃ gantvāna Sāvatthiṃ, passavho dvipaduttamaṃ”. %\hfill\textcolor{gray}{\footnotesize 23}

\subsection\*{\textbf{1006} {\footnotesize 〔PTS 999〕}}

\textbf{「看到后,我们怎么就知道是『佛陀』?婆罗门!\\}
\textbf{「请告诉无知的我们,好让我们认识他!」}

“Kathaṃ carahi jānemu, disvā ‘Buddho’ ti brāhmaṇa;\\
ajānataṃ no pabrūhi, yathā jānemu taṃ mayaṃ”. %\hfill\textcolor{gray}{\footnotesize 24}

\subsection\*{\textbf{1007} {\footnotesize 〔PTS 1000〕}}

\textbf{「因为在颂诗中流传有大人相,\\}
\textbf{「且三十二种被完整、依次地解释。}

“Āgatāni hi mantesu, mahāpurisalakkhaṇā;\\
dvattiṃsāni ca byākkhātā, samattā anupubbaso. %\hfill\textcolor{gray}{\footnotesize 25}

\begin{enumerate}\item \textbf{解释},即论述,即是说详述。\textbf{完整},即是说圆满。\end{enumerate}

\subsection\*{\textbf{1008} {\footnotesize 〔PTS 1001〕}}

\textbf{「若他身上存在这些大人相,\\}
\textbf{「他唯有两种趣向,实无第三种。}

Yass’ete honti gattesu, mahāpurisalakkhaṇā;\\
dve yeva tassa gatiyo, tatiyā hi na vijjati. %\hfill\textcolor{gray}{\footnotesize 26}

\subsection\*{\textbf{1009} {\footnotesize 〔PTS 1002〕}}

\textbf{「如果他居家,则能征服这大地,\\}
\textbf{「不以杖,不以刀,而以法教训。}

Sace agāraṃ āvasati, vijeyya pathaviṃ imaṃ;\\
adaṇḍena asatthena, dhammena-m-anusāsati. %\hfill\textcolor{gray}{\footnotesize 27}

\subsection\*{\textbf{1010} {\footnotesize 〔PTS 1003〕}}

\textbf{「而如果他出家,从家至于非家,\\}
\textbf{「便是去蔽的等觉者,无上的阿罗汉。}

Sace ca so pabbajati, agārā anagāriyaṃ;\\
vivaṭṭacchado sambuddho, arahā bhavati anuttaro. %\hfill\textcolor{gray}{\footnotesize 28}

\subsection\*{\textbf{1011} {\footnotesize 〔PTS 1004〕}}

\textbf{「出身、族姓与相,颂诗、学生,还有\\}
\textbf{「头与头裂,你们唯当以意去问。}

Jātiṃ gottañ ca lakkhaṇaṃ, mante sisse punāpare;\\
muddhaṃ muddhādhipātañ ca, manasā yeva pucchatha. %\hfill\textcolor{gray}{\footnotesize 29}

\begin{enumerate}\item \textbf{出身、种姓与相},即「出生多久了」等我的出身、族姓与(大人)相。\textbf{颂诗、学生},即我所熟习的吠陀、我的学生。\textbf{你们唯当以意去问},即这七个问题,你们唯当以心去问。\end{enumerate}

\subsection\*{\textbf{1012} {\footnotesize 〔PTS 1005〕}}

\textbf{「如果他是佛陀、所见无碍者,\\}
\textbf{「他会用语言回答以意所提的问题。」}

Anāvaraṇadassāvī, yadi buddho bhavissati;\\
manasā pucchite pañhe, vācāya vissajessati”. %\hfill\textcolor{gray}{\footnotesize 30}

\subsection\*{\textbf{1013} {\footnotesize 〔PTS 1006〕}}

\textbf{听到波婆利的话后,十六个婆罗门学生:\\}
\textbf{阿耆多、低舍弥勒、富楼那,还有慈达,}

Bāvarissa vaco sutvā, sissā soḷasa brāhmaṇā;\\
Ajito Tissametteyyo, Puṇṇako atha Mettagū. %\hfill\textcolor{gray}{\footnotesize 31}

\begin{enumerate}\item \textbf{低舍弥勒}只是一人,他以名与族姓得称。\end{enumerate}

\subsection\*{\textbf{1014} {\footnotesize 〔PTS 1007〕}}

\textbf{净洗、优波湿婆与难陀,还有黄金,\\}
\textbf{可教、劫波二人,与智者胶耳,}

Dhotako Upasīvo ca, Nando ca atha Hemako;\\
Todeyya-Kappā dubhayo, Jatukaṇṇī ca Paṇḍito. %\hfill\textcolor{gray}{\footnotesize 32}

\subsection\*{\textbf{1015} {\footnotesize 〔PTS 1008〕}}

\textbf{善器与生起,和布萨罗婆罗门,\\}
\textbf{有智的空王,与大仙褐者,}

Bhadrāvudho Udayo ca, Posālo cāpi brāhmaṇo;\\
Mogharājā ca medhāvī, Piṅgiyo ca mahā isi. %\hfill\textcolor{gray}{\footnotesize 33}

\subsection\*{\textbf{1016} {\footnotesize 〔PTS 1009〕}}

\textbf{全都有各自的徒众,闻名于一切世间,\\}
\textbf{是禅者,乐于禅那,智者,受过去习气的熏习,}

Paccekagaṇino sabbe, sabbalokassa vissutā;\\
jhāyī jhānaratā dhīrā, pubbavāsanavāsitā. %\hfill\textcolor{gray}{\footnotesize 34}

\begin{enumerate}\item \textbf{有各自的徒众},即各个都具徒众。\textbf{受过去习气的熏习},即过去曾在迦叶世尊的教法内出家,心为往还的义务\footnote{往还的义务,见\textbf{犀牛角经}第 35 颂注。}的福德习气所熏习。\end{enumerate}

\subsection\*{\textbf{1017} {\footnotesize 〔PTS 1010〕}}

\textbf{礼拜了波婆利,并且右绕了他,\\}
\textbf{全都萦发、持羚羊皮,朝北方出发。}

Bāvariṃ abhivādetvā, katvā ca naṃ padakkhiṇaṃ;\\
jaṭājinadharā sabbe, pakkāmuṃ uttarāmukhā. %\hfill\textcolor{gray}{\footnotesize 35}

\subsection\*{\textbf{1018} {\footnotesize 〔PTS 1011〕}}

\textbf{(途经)阿罗迦的波提吒那,然后是大箭城、\\}
\textbf{优禅尼以及乔那陀、吠提舍、名为婆那萨者,}

Aḷakassa Patiṭṭhānaṃ, puri Māhissatiṃ tadā;\\
Ujjeniñ cāpi Gonaddhaṃ, Vedisaṃ Vanasavhayaṃ. %\hfill\textcolor{gray}{\footnotesize 36}

\begin{enumerate}\item \textbf{大箭城},即名为大箭的城,即是说城市。意即进入此城,一切处都如是。\textbf{乔那陀},即乔陀城之名。\textbf{名为婆那萨者},即被称为波婆那城者,有些作 Vanasāvatthin。\end{enumerate}

\subsection\*{\textbf{1019} {\footnotesize 〔PTS 1012〕}}

\textbf{㤭赏弥,沙计多,与最上之城舍卫国,\\}
\textbf{私多毗,迦毗罗卫,与拘尸那罗宫城,}

Kosambiñ cāpi Sāketaṃ, Sāvatthiñ ca puruttamaṃ;\\
Setabyaṃ Kapilavatthuṃ, Kusinārañ ca mandiraṃ. %\hfill\textcolor{gray}{\footnotesize 37}

\begin{enumerate}\item 如是,从婆那萨到㤭赏弥,再从㤭赏弥到沙计多\footnote{沙计多 \textit{Sāketa} 在 Ayojjhā/Ayodhyā 附近,被认为是佛陀时代印度的六座大城之一,属㤭萨罗,其余五座分别是瞻婆 \textit{Campā}、王舍城、舍卫城、㤭赏弥、波罗奈,亦见于\textbf{老经}的因缘中。},据说,这十六萦发者所得的随从有六由旬之量。于是,世尊想「波婆利的萦发者正吸引着大众前来,但他们的诸根尚未成熟,此地也不适宜,而摩竭陀国土上的石支提却适宜他们,因为当我在那里开示法,大众将得法的现观,且经过所有城市,会有更多的人们前来」,便为比丘僧团围绕,从舍卫国去往王舍城方向。这些萦发者到了舍卫国,进入寺庙后,审视着「谁是佛陀,佛陀在哪里」,去到香房下,看到世尊的足迹,由\begin{quoting}染著者的足迹曲起……去蔽者的足迹如斯。(清净道论·说取业处品第 88 段)\end{quoting}得出结论「佛陀是一切智者」。世尊渐次经过私多毗、迦毗罗卫等城,吸引着大众,去到石支提。萦发者也即刻离开舍卫国,经过所有这些城市,到达了石支提。因此说「\textbf{㤭赏弥,沙计多,与最上之城舍卫国,私多毗,迦毗罗卫}」等等。\end{enumerate}

\subsection\*{\textbf{1020} {\footnotesize 〔PTS 1013〕}}

\textbf{财富城波婆,毗舍离,摩竭陀城,\\}
\textbf{与怡人、悦意的石支提。}

Pāvañ ca Bhoganagaraṃ, Vesāliṃ Māgadhaṃ puraṃ;\\
Pāsāṇakaṃ cetiyañ ca, ramaṇīyaṃ manoramaṃ. %\hfill\textcolor{gray}{\footnotesize 38}

\begin{enumerate}\item 这里,\textbf{摩竭陀城},意即王舍城。\textbf{石支提}在大石之上,过去曾是天人的住处,虽然在世尊出世后成为寺庙,它仍因过去的习俗被称为「石支提」。\end{enumerate}

\subsection\*{\textbf{1021} {\footnotesize 〔PTS 1014〕}}

\textbf{如口干者之于寒泉,如商人之于大利,\\}
\textbf{如为暑热炙灼者之于凉荫,他们匆匆登山。}

Tasito v’udakaṃ sītaṃ, mahālābhaṃ va vāṇijo;\\
chāyaṃ ghammābhitatto va, turitā pabbatam āruhuṃ. %\hfill\textcolor{gray}{\footnotesize 39}

\begin{enumerate}\item \textbf{如口干者之于寒泉},即这些萦发者匆匆追随世尊,白天走着(世尊)晚上走过的路,晚上走着白天走过的路,听到「世尊在此」,便极欢喜、愉悦地登上了这支提,因此说\textbf{他们匆匆登山}。\end{enumerate}

\subsection\*{\textbf{1022} {\footnotesize 〔PTS 1015〕}}

\textbf{此时,世尊置身比丘僧团之前,\\}
\textbf{对诸比丘开示法,如林中狮吼。}

Bhagavā tamhi samaye, bhikkhusaṅghapurakkhato;\\
bhikkhūnaṃ dhammaṃ deseti, sīho va nadatī vane. %\hfill\textcolor{gray}{\footnotesize 40}

\subsection\*{\textbf{1023} {\footnotesize 〔PTS 1016〕}}

\textbf{阿耆多见到了佛陀,如百道光芒\footnote{百道光芒 \textit{sataraṃsiṃ}:PTS 本作「光芒直射 \textit{vitaraṃsi}」。}的太阳,\\}
\textbf{好比十五之夜趋近圆满的月亮。}

Ajito addasa Buddhaṃ, sataraṃsiṃ va bhāṇumaṃ;\\
candaṃ yathā pannarase, pāripūriṃ upāgataṃ. %\hfill\textcolor{gray}{\footnotesize 41}

\subsection\*{\textbf{1024} {\footnotesize 〔PTS 1017〕}}

\textbf{然后,看到他的身体和圆满的特相,\\}
\textbf{站在一边,振奋,在意中问了问题:}

Ath’assa gatte disvāna, paripūrañ ca byañjanaṃ;\\
ekamantaṃ ṭhito haṭṭho, manopañhe apucchatha. %\hfill\textcolor{gray}{\footnotesize 42}

\begin{enumerate}\item \textbf{站在一边,振奋},即看到世尊坐在这石支提上由帝释天所造的大亭子里,当世尊以「众仙人尚可忍耐否」等方法问候时,自己也以「乔达摩君尚可忍耐否」等承迎后,最长的弟子阿耆多站在一边,心怀振奋,\textbf{在意中问了问题}。\end{enumerate}

\subsection\*{\textbf{1025} {\footnotesize 〔PTS 1018〕}}

\textbf{「关于出身,请说!请说族姓及其相!\\}
\textbf{「请说颂诗的成就!婆罗门教授几多?」}

“Ādissa jammanaṃ brūhi, gottaṃ brūhi salakkhaṇaṃ;\\
mantesu pāramiṃ brūhi, kati vāceti brāhmaṇo”. %\hfill\textcolor{gray}{\footnotesize 43}

\begin{enumerate}\item 这里,\textbf{关于},即如关于「几岁」。\textbf{出身},即问「请说:我们老师的出身」。\textbf{成就},即所到的终点。\end{enumerate}

\subsection\*{\textbf{1026} {\footnotesize 〔PTS 1019〕}}

\textbf{「年寿一百二十,他族姓波婆利,\\}
\textbf{「他身上有三相,通晓三种吠陀,}

“Vīsaṃ vassasataṃ āyu, so ca gottena Bāvarī;\\
tīṇi’ssa lakkhaṇā gatte, tiṇṇaṃ vedāna pāragū. %\hfill\textcolor{gray}{\footnotesize 44}

\subsection\*{\textbf{1027} {\footnotesize 〔PTS 1020〕}}

\textbf{「于相、传承及其词汇、仪轨,\\}
\textbf{「教授五百,于自法已达成就。」}

Lakkhaṇe itihāse ca, sanighaṇḍu-sakeṭubhe;\\
pañca satāni vāceti, sadhamme pāramiṃ gato”. %\hfill\textcolor{gray}{\footnotesize 45}

\begin{enumerate}\item \textbf{于相},即于大人相,意即于此及于此后的传承等的全部,或者引出下句后,应与「已达成就」相连。\textbf{教授五百},即亲自教授五百生性怠惰、恶慧的学童颂诗。\textbf{自法},即一些婆罗门的法,即是说三明的教法。\end{enumerate}

\subsection\*{\textbf{1028} {\footnotesize 〔PTS 1021〕}}

\textbf{「波婆利之相的简择,最上之人!\\}
\textbf{「断除疑惑者!请阐明!莫使我们有疑惑!」}

“Lakkhaṇānaṃ pavicayaṃ, Bāvarissa naruttama;\\
kaṅkhacchida pakāsehi, mā no kaṅkhāyitaṃ ahu”. %\hfill\textcolor{gray}{\footnotesize 46}

\begin{enumerate}\item \textbf{相的简择},即相的细节,他问道:「他身上有哪三相?」\end{enumerate}

\subsection\*{\textbf{1029} {\footnotesize 〔PTS 1022〕}}

\textbf{「以舌覆面,眉间有毫,\\}
\textbf{「阴马藏,当知如是!学童!」}

“Mukhaṃ jivhāya chādeti, uṇṇ’assa bhamukantare;\\
kosohitaṃ vatthaguyhaṃ, evaṃ jānāhi māṇava”. %\hfill\textcolor{gray}{\footnotesize 47}

\subsection\*{\textbf{1030} {\footnotesize 〔PTS 1023〕}}

\textbf{尚未听到任何所问,却听到问题被解答,\\}
\textbf{所有人都充满喜悦、合掌,思忖道:}

Pucchañ hi kiñci asuṇanto, sutvā pañhe viyākate;\\
vicinteti jano sabbo, vedajāto katañjalī. %\hfill\textcolor{gray}{\footnotesize 48}

\subsection\*{\textbf{1031} {\footnotesize 〔PTS 1024〕}}

\textbf{「到底是谁,是天、梵、还是善生之主因陀\footnote{善生之主、因陀都是帝释的称号。}\\}
\textbf{「以意问了这些问题?他在应答谁?」}

“Ko nu devo vā brahmā vā, Indo vā pi Sujampati;\\
manasā pucchi te pañhe, kam etaṃ paṭibhāsati”. %\hfill\textcolor{gray}{\footnotesize 49}

\begin{enumerate}\item \textbf{他在应答谁},即他在应答天等中哪个补特伽罗的这问语。\end{enumerate}

\subsection\*{\textbf{1032} {\footnotesize 〔PTS 1025〕}}

\textbf{「波婆利遍问头与头裂,\\}
\textbf{「请解答!世尊!调伏我们的疑惑!仙人!」}

“Muddhaṃ muddhādhipātañ ca, Bāvarī paripucchati;\\
taṃ byākarohi Bhagavā, kaṅkhaṃ vinaya no ise”. %\hfill\textcolor{gray}{\footnotesize 50}

\begin{enumerate}\item 如是,婆罗门听到五个问题的解答,为问余下的两个,说了「波婆利遍问头与头裂……」。\end{enumerate}

\subsection\*{\textbf{1033} {\footnotesize 〔PTS 1026〕}}

\textbf{「当知无明是头,而明是头裂,\\}
\textbf{「伴以信、念、定、欲与精进。」}

“Avijjā muddhā ti jānāhi, vijjā muddhādhipātinī;\\
saddhā-sati-samādhīhi, chanda-viriyena saṃyutā”. %\hfill\textcolor{gray}{\footnotesize 51}

\begin{enumerate}\item 于是,世尊对其解答这些,说了此颂。这里,因为作为对四谛无知的无明为轮回之首,所以说「\textbf{无明是头}」。又因为阿罗汉道之\textbf{明},具足与其俱生的\textbf{信、念、定}、欲作之\textbf{欲与精进},由诸根聚于一味故,能令此头裂,所以说「\textbf{明是头裂}」等等。\end{enumerate}

\subsection\*{\textbf{1034} {\footnotesize 〔PTS 1027〕}}

\textbf{随后,学童为大喜悦裹挟,\\}
\textbf{把羚羊皮偏覆一肩,以头顶礼双足:}

Tato vedena mahatā, santhambhitvāna māṇavo;\\
ekaṃsaṃ ajinaṃ katvā, pādesu sirasā pati. %\hfill\textcolor{gray}{\footnotesize 52}

\begin{enumerate}\item \textbf{随后,学童为大喜悦}……,于是,听了这问题的解答,被生起的大喜悦裹挟而不退缩,即身心至于踊跃之义。\end{enumerate}

\subsection\*{\textbf{1035} {\footnotesize 〔PTS 1028〕}}

\textbf{「波婆利婆罗门,与其学生一起,先生!\\}
\textbf{「心怀踊跃而欢喜,礼拜双足,具眼者!」}

“Bāvarī brāhmaṇo bhoto, saha sissehi mārisa;\\
udaggacitto sumano, pāde vandati cakkhuma”. %\hfill\textcolor{gray}{\footnotesize 53}

\begin{enumerate}\item 且顶礼后,说了「\textbf{波婆利}……」一颂。\end{enumerate}

\subsection\*{\textbf{1036} {\footnotesize 〔PTS 1029〕}}

\textbf{「愿波婆利婆罗门与诸学生幸福!\\}
\textbf{「也愿你幸福!愿你长命!学童!}

“Sukhito Bāvarī hotu, saha sissehi brāhmaṇo;\\
tvañ cāpi sukhito hohi, ciraṃ jīvāhi māṇava. %\hfill\textcolor{gray}{\footnotesize 54}

\begin{enumerate}\item 于是,世尊为怜悯彼,说了此颂。\end{enumerate}

\subsection\*{\textbf{1037} {\footnotesize 〔PTS 1030〕}}

\textbf{「波婆利的与你的,或所有人的一切疑虑,\\}
\textbf{「都有机会,请问任何心中所希望的!」}

Bāvarissa ca tuyhaṃ vā, sabbesaṃ sabbasaṃsayaṃ;\\
katāvakāsā pucchavho, yaṃ kiñci manas’icchatha”. %\hfill\textcolor{gray}{\footnotesize 55}

\begin{enumerate}\item 且说后,以「\textbf{波婆利的}……」作了一切智的邀请。这里,\textbf{所有人},即全体一万六千人。\end{enumerate}

\subsection\*{\textbf{1038} {\footnotesize 〔PTS 1031〕}}

\textbf{得到等觉者的许可,坐下后合了掌,\\}
\textbf{阿耆多于此便问了如来第一个问题。}

Sambuddhena katokāso, nisīditvāna pañjalī;\\
Ajito paṭhamaṃ pañhaṃ, tattha pucchi Tathāgataṃ. %\hfill\textcolor{gray}{\footnotesize 56}

\begin{enumerate}\item \textbf{于此便问了如来},即于此石支提上,或于此会众中,或于这些邀请中,阿耆多便问了第一个问题。其余一切颂中皆自明。\end{enumerate}

\begin{center}\vspace{1em}序颂终\\Vatthugāthā niṭṭhitā.\end{center}