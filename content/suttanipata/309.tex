\section{婆悉吒经}

\begin{center}Vāseṭṭha Sutta\end{center}\vspace{1em}

\textbf{如是我闻\footnote{此经即\textbf{中部}第 98 经,旧译见长阿含经卷第十六·三明经。这里的译名,如婆悉吒、伊车能伽罗、多梨车等从三明经。}。一时世尊住伊车能伽罗的伊车能伽罗林。尔时,有众多非常有名的富裕婆罗门在伊车能伽罗定居,即婆罗门旃基、婆罗门多梨车、婆罗门莲卧、婆罗门生闻、婆罗门可教,及其他非常有名的富裕婆罗门。}

Evaṃ me sutaṃ— ekaṃ samayaṃ Bhagavā Icchānaṅgale viharati Icchānaṅgalavanasaṇḍe. Tena kho pana samayena sambahulā abhiññātā abhiññātā brāhmaṇamahāsālā Icchānaṅgale paṭivasanti, seyyathidaṃ— Caṅkī brāhmaṇo, Tārukkho brāhmaṇo, Pokkharasāti brāhmaṇo, Jāṇussoṇi brāhmaṇo, Todeyyo brāhmaṇo, aññe ca abhiññātā abhiññātā brāhmaṇamahāsālā.

\begin{enumerate}\item 缘起为何?这在其因缘中已述。而我们将避开已述与意义自明之词,再作其释义。\textbf{伊车能伽罗}是村名。富裕婆罗门中的\textbf{旃基}、\textbf{多梨车}、\textbf{可教}是习俗的名字,\textbf{莲卧}、\textbf{生闻}则是关于特相的。据说,前者转生在雪山腹地池塘的莲花中,某个苦行者采了这莲花,看到躺卧于此的男婴,抚养长大后,便示与国王,由其卧于莲花,便给他起名「莲卧」。后者则是关于地位的特相。据说,王室祭司的地位得名「生闻\footnote{生闻 \textit{Jāṇussoṇi}:亦见于\textbf{中部}第 4 经等处,译名从杂阿含经第 1041 经。}」,他便因此为众所知。
\item 所有这些及其他非常有名的富裕婆罗门为什么在伊车能伽罗定居?为诵习、研究吠陀。据说,那时在㤭萨罗国,通吠陀的婆罗门为诵习及研究吠陀,在此村中集会,因此,他们便常常从自己的采地前来定居于此。\end{enumerate}

\textbf{于是,当学童婆悉吒、婆罗豆婆遮徒步随行、游荡时,便发起这闲谈:「先生!如何才是婆罗门?」}

Atha kho Vāseṭṭha-Bhāradvājānaṃ māṇavānaṃ jaṅghāvihāraṃ anucaṅkamantānaṃ anuvicarantānaṃ ayam antarākathā udapādi: “kathaṃ, bho, brāhmaṇo hotī” ti?

\begin{enumerate}\item \textbf{这闲谈},即是说当他们游荡,在谈论与自己朋友适合的言谈时,便在此言谈之间、中间,发起这另一言谈。\end{enumerate}

\textbf{学童婆罗豆婆遮如是说:「先生!只要父母双方出身良好,乃至祖上七代腹胎纯正,不因出身论而受排斥、责难,如此,他就是婆罗门。」学童婆悉吒如是说:「先生!只要具戒、具足仪法,如此,他就是婆罗门。」学童婆罗豆婆遮不能说服学童婆悉吒,学童婆悉吒也不能说服学童婆罗豆婆遮。}

Bhāradvājo māṇavo evam āha: “yato kho, bho, ubhato sujāto hoti mātito ca pitito ca saṃsuddhagahaṇiko yāva sattamā pitāmahayugā akkhitto anupakkuṭṭho jātivādena, ettāvatā kho bho brāhmaṇo hotī” ti. Vāseṭṭho māṇavo evam āha: “yato kho, bho, sīlavā ca hoti vatasampanno ca, ettāvatā kho, bho, brāhmaṇo hotī” ti. N’eva kho asakkhi Bhāradvājo māṇavo Vāseṭṭhaṃ māṇavaṃ saññāpetuṃ, na pana asakkhi Vāseṭṭho māṇavo Bhāradvājaṃ māṇavaṃ saññāpetuṃ.

\begin{enumerate}\item \textbf{腹胎纯正},即胎室纯正,在纯正的婆罗门尼的胎室内转生之意。在「平等消化的腹胎」等中,胃中之火被称为「腹胎」,但在此是指母胎。\textbf{乃至七代},即以母亲的母亲、父亲的父亲如是上溯,乃至七生。且此中,祖父祖母为祖,同样,外祖父外祖母为外祖,祖与外祖仍为祖,众祖的年代为\textbf{祖上}之代。代,即寿量,在此只是一种表达,而从意义上说,祖上即祖上之代。\textbf{不受排斥},即不被任何人就出身而蔑视——他是什么?\textbf{不受责难},即前代不因出身染污论受到责难。\textbf{具足仪法},即具足正行\footnote{正行 \textit{ācāra},见\textbf{清净道论}·说戒品第 44 段。}。\textbf{说服},即令知、令觉,即是说令其无言\footnote{无言 \textit{niruttaraṃ}:原文作「无间断 \textit{nirantaraṃ}」,今从 PTS 本。}。\end{enumerate}

\textbf{于是,学童婆悉吒对学童婆罗豆婆遮说:「婆罗豆婆遮君!这沙门乔达摩、释迦子、从释迦族出家者,住在伊车能伽罗的伊车能伽罗林,关于这乔达摩君,有如是的善称:『彼……佛、世尊』。让我们去!婆罗豆婆遮君!我们去往沙门乔达摩处,去到后,我们问问沙门乔达摩此义。如沙门乔达摩对我们所解释的,我们便这样受持之。」「如是,先生!」学童婆罗豆婆遮答学童婆悉吒。}

Atha kho Vāseṭṭho māṇavo Bhāradvājaṃ māṇavaṃ āmantesi: “ayaṃ kho, bho Bhāradvāja, samaṇo Gotamo Sakyaputto Sakyakulā pabbajito Icchānaṅgale viharati Icchānaṅgalavanasaṇḍe, taṃ kho pana bhavantaṃ Gotamaṃ evaṃ kalyāṇo kittisaddo abbhuggato: ‘iti pi…pe… buddho bhagavā’ ti. Āyāma, bho Bhāradvāja, yena samaṇo Gotamo ten’upasaṅkamissāma, upasaṅkamitvā samaṇaṃ Gotamaṃ etam atthaṃ pucchissāma. Yathā no samaṇo Gotamo byākarissati tathā naṃ dhāressāmā” ti. “Evaṃ, bho” ti kho Bhāradvājo māṇavo Vāseṭṭhassa māṇavassa paccassosi.

\textbf{于是,学童婆悉吒、婆罗豆婆遮往世尊处走去,走到后,问候了世尊,彼此寒暄已,坐在一边。坐在一边的学童婆悉吒以偈颂对世尊说:}

Atha kho Vāseṭṭha-Bhāradvājā māṇavā yena Bhagavā ten’upasaṅkamiṃsu, upasaṅkamitvā Bhagavatā saddhiṃ sammodiṃsu, sammodanīyaṃ kathaṃ sāraṇīyaṃ vītisāretvā ekamantaṃ nisīdiṃsu. Ekamantaṃ nisinno kho Vāseṭṭho māṇavo Bhagavantaṃ gāthāhi ajjhabhāsi:

\subsection\*{\textbf{600} {\footnotesize 〔PTS 594〕}}

\textbf{「我俩都被认为、且自认是三明者,\\}
\textbf{「我是莲卧的学童、他是多梨车的。}

“Anuññātapaṭiññātā, tevijjā mayam asm’ubho;\\
ahaṃ Pokkharasātissa, Tārukkhassāyaṃ māṇavo. %\hfill\textcolor{gray}{\footnotesize 1}

\begin{enumerate}\item \textbf{被认为、且自认},即我们被老师们如是认可「你们是三明者」,且我们自己也这么认为之义。\textbf{我是莲卧的学童、他是多梨车的},即以「我是莲卧的最长的弟子、上首学生,他是多梨车的」之意而说,为显明老师的成就及自己的成就。\end{enumerate}

\subsection\*{\textbf{601} {\footnotesize 〔PTS 595〕}}

\textbf{「凡众三明者所说的,我们于此整全,\\}
\textbf{「我们通句读、晓文法,在读诵上和老师一样。}

Tevijjānaṃ yad akkhātaṃ, tatra kevalino smase;\\
padaka’sma veyyākaraṇā, jappe ācariyasādisā. %\hfill\textcolor{gray}{\footnotesize 2}

\begin{enumerate}\item \textbf{众三明者},即众三吠陀者。\textbf{整全},即已达究竟。现在,为详绎其整全性而说「我们通句读……一样」。这里,\textbf{在读诵上},即在吠陀上。\end{enumerate}

\subsection\*{\textbf{602} {\footnotesize 〔PTS 596〕}}

\textbf{「我们之间于出身论存在争论,乔达摩!\\}
\textbf{「婆罗豆婆遮说由出身就是婆罗门,\\}
\textbf{「而我说由业,如是当知!具眼者!}

Tesaṃ no jātivādasmiṃ, vivādo atthi Gotama;\\
‘jātiyā brāhmaṇo hoti’, Bhāradvājo iti bhāsati;\\
ahañ ca ‘kammunā’ brūmi, evaṃ jānāhi cakkhuma. %\hfill\textcolor{gray}{\footnotesize 3}

\begin{enumerate}\item 现在,为以统贯之语显示此而说「而我说由业」。\footnote{此句原文在对「业」的解释之后,恐是错简,今置于前。}\textbf{业},即十种善业道之业。因为就前七种身语业而说「先生!只要具戒」,就三种意业而说「具足仪法」,因为具足此即具足正行。\end{enumerate}

\subsection\*{\textbf{603} {\footnotesize 〔PTS 597〕}}

\textbf{「我们双方都不能说服彼此,\\}
\textbf{「便前来问先生,以等正觉著名者。}

Te na sakkoma saññāpetuṃ, aññamaññaṃ mayaṃ ubho;\\
bhavantaṃ puṭṭhum āgamhā, sambuddhaṃ iti vissutaṃ. %\hfill\textcolor{gray}{\footnotesize 4}

\subsection\*{\textbf{604} {\footnotesize 〔PTS 598〕}}

\textbf{「好比月亮已无亏缺,众人前往合掌,\\}
\textbf{「如是在世间,我们顶礼礼敬乔达摩。}

Candaṃ yathā khayātītaṃ, pecca pañjalikā janā;\\
vandamānā namassanti, evaṃ lokasmi Gotamaṃ. %\hfill\textcolor{gray}{\footnotesize 5}

\begin{enumerate}\item \textbf{已无亏缺},即欠缺的状态已过,即圆满之义。\end{enumerate}

\subsection\*{\textbf{605} {\footnotesize 〔PTS 599〕}}

\textbf{「我们问乔达摩,在世间出现的眼目,\\}
\textbf{「由出身就是婆罗门,抑或由业而成,\\}
\textbf{「请告诉无知的我们,好让我们懂得婆罗门!」}

Cakkhuṃ loke samuppannaṃ, mayaṃ pucchāma Gotamaṃ;\\
jātiyā brāhmaṇo hoti, udāhu bhavati kammunā;\\
ajānataṃ no pabrūhi, yathā jānemu brāhmaṇaṃ”. %\hfill\textcolor{gray}{\footnotesize 6}

\begin{enumerate}\item \textbf{在世间出现的眼目},即在无明黑暗的世间中,驱散这黑暗,以对世间显示现法等的义利而作为眼目出现者。\end{enumerate}

\subsection\*{\textbf{606} {\footnotesize 〔PTS 600〕}}

\textbf{「我将对你们解释,婆悉吒!」世尊说,「按照次第、如其所是地,\\}
\textbf{「生类的种类分别,因为种类各不相同。}

“Tesaṃ vo ahaṃ byakkhissaṃ, \textit{(Vāseṭṭhā ti Bhagavā)} anupubbaṃ yathātathaṃ;\\
jātivibhaṅgaṃ pāṇānaṃ, aññamaññā hi jātiyo. %\hfill\textcolor{gray}{\footnotesize 7}

\begin{enumerate}\item 如是赞叹已,世尊受婆悉吒祈请,亦为摄受两人,说了「我将对你们解释」等。这里,\textbf{按照次第},当知此中的意趣为:请先停止婆罗门的思惟!我将从昆虫、蚱蜢、草木开始按照次第解释,如是,二学童将被详说调伏。\textbf{种类分别},即种类的详说。\textbf{因为种类各不相同},即彼彼生类的种类各不相同,品类各异之义。\end{enumerate}

\subsection\*{\textbf{607} {\footnotesize 〔PTS 601〕}}

\textbf{「你们也知道草木,虽然它们并不自认,\\}
\textbf{「它们的特征与生俱来,因为种类各不相同。}

Tiṇarukkhe pi jānātha, na cāpi paṭijānare;\\
liṅgaṃ jātimayaṃ tesaṃ, aññamaññā hi jātiyo. %\hfill\textcolor{gray}{\footnotesize 8}

\begin{enumerate}\item 随后,在当说的生类的种类分别中,先就无执取者开始说「你们也知道草木」。若问:这是为了什么?为了易于了知执取者。因为当于无执取者的种类分别已把握时,则于执取者便更为明了。
\item 这里,\textbf{草}者内柔外刚,所以棕榈、椰子等归于草类,\textbf{木}则外柔内刚,草与木为草木,为以复数业格显示彼等,而说「你们也知道草木」。\textbf{虽然它们并不自认},即虽不如是自认「我们是草、我们是木」。\textbf{特征与生俱来},即便不自认,它们与生俱来的形状与自己种属的草等相似。什么原因?\textbf{因为种类各不相同},因为有些是草类,有些是木类,且在草中,有些是棕榈类,有些是椰子类,当如是详述。
\item 以此显明什么?这多样的种类,即便没有自己的自认或他人的指认,也会以异于其它种类的特质被把握。且如果由出身就成婆罗门,那么他即便没有自己的自认或他人的指认,也应能以异于刹帝利或吠舍、首陀罗的特质被把握,但却不能被把握,所以不由出身就是婆罗门。我们会以言词的区分进一步在「如在这些种类中」一颂中\footnote{即第 613 颂。}揭示此义。\end{enumerate}

\subsection\*{\textbf{608} {\footnotesize 〔PTS 602〕}}

\textbf{「随后,昆虫、蚱蜢,乃至于蝼蚁,\\}
\textbf{「它们的特征与生俱来,因为种类各不相同。}

Tato kīṭe paṭaṅge ca, yāva kuntha-kipillike;\\
liṅgaṃ jātimayaṃ tesaṃ, aññamaññā hi jātiyo. %\hfill\textcolor{gray}{\footnotesize 9}

\begin{enumerate}\item 如是显示了无执取者中的种类差别,为显示执取者中的而说此颂。\textbf{乃至于蝼蚁},即以蝼蚁为边界之义。\end{enumerate}

\subsection\*{\textbf{609} {\footnotesize 〔PTS 603〕}}

\textbf{「你们也知道四足者,小的和大的,\\}
\textbf{「它们的特征与生俱来,因为种类各不相同。}

Catuppade pi jānātha, khuddake ca mahallake;\\
liṅgaṃ jātimayaṃ tesaṃ, aññamaññā hi jātiyo. %\hfill\textcolor{gray}{\footnotesize 10}

\begin{enumerate}\item \textbf{小的},即松鼠\footnote{原文作「黑色 \textit{kāḷa}」,疑是「松鼠 \textit{kāḷakā}」,见 PTS 本脚注。}、变色龙等。\textbf{大的},即兔、猫等。彼等一切各有多种。\end{enumerate}

\subsection\*{\textbf{610} {\footnotesize 〔PTS 604〕}}

\textbf{「你们也知道以腹为足、长背的蛇,\\}
\textbf{「它们的特征与生俱来,因为种类各不相同。}

Pādūdare pi jānātha, urage dīghapiṭṭhike;\\
liṅgaṃ jātimayaṃ tesaṃ, aññamaññā hi jātiyo. %\hfill\textcolor{gray}{\footnotesize 11}

\begin{enumerate}\item \textbf{长背},因为蛇从头到尾都是背,因此被称为「长背」。彼等也以毒蛇等类而有多种品类。\end{enumerate}

\subsection\*{\textbf{611} {\footnotesize 〔PTS 605〕}}

\textbf{「随后,你们也知道水生的鱼,以水为行处,\\}
\textbf{「它们的特征与生俱来,因为种类各不相同。}

Tato macche pi jānātha, odake vārigocare;\\
liṅgaṃ jātimayaṃ tesaṃ, aññamaññā hi jātiyo. %\hfill\textcolor{gray}{\footnotesize 12}

\begin{enumerate}\item 鱼也以鲑鱼等类而有多种品类。\end{enumerate}

\subsection\*{\textbf{612} {\footnotesize 〔PTS 606〕}}

\textbf{「随后,你们也知道鸟,凭羽翼而行于空中,\\}
\textbf{「它们的特征与生俱来,因为种类各不相同。}

Tato pakkhī pi jānātha, pattayāne vihaṅgame;\\
liṅgaṃ jātimayaṃ tesaṃ, aññamaññā hi jātiyo. %\hfill\textcolor{gray}{\footnotesize 13}

\begin{enumerate}\item \textbf{鸟},因为有翼而被称为「鸟」。彼等也以乌鸦等类而有多种品类。\end{enumerate}

\subsection\*{\textbf{613} {\footnotesize 〔PTS 607〕}}

\textbf{「如在这些种类中,存在种种与生俱来的特征,\\}
\textbf{「如是在人类中,不存在种种与生俱来的特征。}

Yathā etāsu jātīsu, liṅgaṃ jātimayaṃ puthu;\\
evaṃ natthi manussesu, liṅgaṃ jātimayaṃ puthu. %\hfill\textcolor{gray}{\footnotesize 14}

\begin{enumerate}\item 如是显示了以陆、水、空为行处的生类的种类差别,现在,为表明所显示的意趣,说了此颂。其义可由先前所说意趣的解释略知。\end{enumerate}

\subsection\*{\textbf{614} {\footnotesize 〔PTS 608〕}}

\textbf{「不是以发、以头、以耳、以眼,\\}
\textbf{「不是以口、以鼻、以唇或以眉,}

Na kesehi na sīsena, na kaṇṇehi na akkhibhi;\\
na mukhena na nāsāya, na oṭṭhehi bhamūhi vā. %\hfill\textcolor{gray}{\footnotesize 15}

\begin{enumerate}\item 而为亲自显示此中当详说者,说了「不是以发」等等。这里的连结为:所说的「在人类中,不存在种种与生俱来的特征」,当知即如是不存在,即\textbf{不是以发}。因为没有定则说「婆罗门有这样的发,刹帝利有那样的」,如象、马、鹿等。一切当以此法连结。\end{enumerate}

\subsection\*{\textbf{615} {\footnotesize 〔PTS 609〕}}

\textbf{「不是以颈、以肩、以腹、以背,\\}
\textbf{「不是以臀、以胸、以阴、以媾\footnote{Norman 说在梵本「\textbf{天譬喻} \textit{Divyāvadāna}」中,methune 作 methunaiḥ,则这里的 sambādhe, methune 也是复数具格的形式,并将两者分别译作「女阴、睾丸」,理由是 methuna 间有「双生」之义,故这里有可能是指睾丸,但同时又说 PED 认为 sambādha 可兼指男女的生殖器,如此则 methuna 应指交媾,唯一的问题是与颂中其它的词均指身体部位不相类。},}

Na gīvāya na aṃsehi, na udarena na piṭṭhiyā;\\
na soṇiyā na urasā, na sambādhe na methune. %\hfill\textcolor{gray}{\footnotesize 16}

\subsection\*{\textbf{616} {\footnotesize 〔PTS 610〕}}

\textbf{「不是以手、以足,或以指、以甲,\\}
\textbf{「不是以胫、以股,或以色、以声,\\}
\textbf{「并没有如其它种类中与生俱来的特征。}

Na hatthehi na pādehi, nāṅgulīhi nakhehi vā;\\
na jaṅghāhi na ūrūhi, na vaṇṇena sarena vā;\\
liṅgaṃ jātimayaṃ n’eva, yathā aññāsu jātisu. %\hfill\textcolor{gray}{\footnotesize 17}

\begin{enumerate}\item \textbf{并没有如其它种类中与生俱来的特征},当知此即对所说之义的总结。其连结为:正因为在人类中,不存在头发等种种与生俱来的特征,所以当知「在婆罗门等类的人类中,并没有如其它种类中与生俱来的特征」。\end{enumerate}

\subsection\*{\textbf{617} {\footnotesize 〔PTS 611〕}}

\textbf{「这于人类,在个别的身体上不存在,\\}
\textbf{「而人类中的差别,以名称而被言说。}

Paccattañ ca sarīresu, manussesv etaṃ na vijjati;\\
vokārañ ca manussesu, samaññāya pavuccati. %\hfill\textcolor{gray}{\footnotesize 18}

\begin{enumerate}\item 现在,既然不存在种类的差别,为显示「婆罗门、刹帝利」这多样性如何产生,说了此颂。其义为:\textbf{这}如畜生一般,唯以母胎成就的头发等形象的多样性,\textbf{于人类}的婆罗门等中,在各自的\textbf{身体上不存在}。而既然这不存在,则「婆罗门、刹帝利」等多样性的约定方式,即这\textbf{人类中的差别,以名称而被言说},仅以习俗被言说。\end{enumerate}

\subsection\*{\textbf{618} {\footnotesize 〔PTS 612〕}}

\textbf{「在人类中,凡是依护牛而活,\\}
\textbf{「婆悉吒!如是当知他是耕者,不是婆罗门。}

Yo hi koci manussesu, gorakkhaṃ upajīvati;\\
evaṃ Vāseṭṭha jānāhi, kassako so na brāhmaṇo. %\hfill\textcolor{gray}{\footnotesize 19}

\begin{enumerate}\item 至此,世尊折服了婆罗豆婆遮的论说,现在,若以出身就是婆罗门,则即便破失了活命、戒、正行也还是婆罗门,而因为古婆罗门以及世间的其他智者不赞成其婆罗门性,所以,为了支持婆悉吒的论说,说了「在人类中」等八颂。这里,\textbf{护牛},即护田,即是说耕作。因为地被称为「牛」,而田即其分类。\end{enumerate}

\subsection\*{\textbf{619} {\footnotesize 〔PTS 613〕}}

\textbf{「在人类中,凡是以种种技艺而活,\\}
\textbf{「婆悉吒!如是当知他是匠人,不是婆罗门。}

Yo hi koci manussesu, puthusippena jīvati;\\
evaṃ Vāseṭṭha jānāhi, sippiko so na brāhmaṇo. %\hfill\textcolor{gray}{\footnotesize 20}

\begin{enumerate}\item \textbf{种种技艺},即纺织业等多种技艺。\end{enumerate}

\subsection\*{\textbf{620} {\footnotesize 〔PTS 614〕}}

\textbf{「在人类中,凡是依买卖而活,\\}
\textbf{「婆悉吒!如是当知他是商人,不是婆罗门。}

Yo hi koci manussesu, vohāraṃ upajīvati;\\
evaṃ Vāseṭṭha jānāhi, vāṇijo so na brāhmaṇo. %\hfill\textcolor{gray}{\footnotesize 21}

\subsection\*{\textbf{621} {\footnotesize 〔PTS 615〕}}

\textbf{「在人类中,凡是以侍奉他人而活,\\}
\textbf{「婆悉吒!如是当知他是仆人,不是婆罗门。}

Yo hi koci manussesu, parapessena jīvati;\\
evaṃ Vāseṭṭha jānāhi, pessiko so na brāhmaṇo. %\hfill\textcolor{gray}{\footnotesize 22}

\subsection\*{\textbf{622} {\footnotesize 〔PTS 616〕}}

\textbf{「在人类中,凡是依不与取而活,\\}
\textbf{「婆悉吒!如是当知他是贼人,不是婆罗门。}

Yo hi koci manussesu, adinnaṃ upajīvati;\\
evaṃ Vāseṭṭha jānāhi, coro eso na brāhmaṇo. %\hfill\textcolor{gray}{\footnotesize 23}

\subsection\*{\textbf{623} {\footnotesize 〔PTS 617〕}}

\textbf{「在人类中,凡是依射艺而活,\\}
\textbf{「婆悉吒!如是当知是战士,不是婆罗门。}

Yo hi koci manussesu, issatthaṃ upajīvati;\\
evaṃ Vāseṭṭha jānāhi, yodhājīvo na brāhmaṇo. %\hfill\textcolor{gray}{\footnotesize 24}

\begin{enumerate}\item \textbf{射艺},即以武器而活,即是说箭与刀。\end{enumerate}

\subsection\*{\textbf{624} {\footnotesize 〔PTS 618〕}}

\textbf{「在人类中,凡是以祭祀而活,\\}
\textbf{「婆悉吒!如是当知他是祭司,不是婆罗门。}

Yo hi koci manussesu, porohiccena jīvati;\\
evaṃ Vāseṭṭha jānāhi, yājako eso na brāhmaṇo. %\hfill\textcolor{gray}{\footnotesize 25}

\subsection\*{\textbf{625} {\footnotesize 〔PTS 619〕}}

\textbf{「在人类中,凡是食禄村庄与王国,\\}
\textbf{「婆悉吒!如是当知他是国王,不是婆罗门。}

Yo hi koci manussesu, gāmaṃ raṭṭhañ ca bhuñjati;\\
evaṃ Vāseṭṭha jānāhi, rājā eso na brāhmaṇo. %\hfill\textcolor{gray}{\footnotesize 26}

\subsection\*{\textbf{626} {\footnotesize 〔PTS 620〕}}

\textbf{「而我不说从胎所生、源于母亲者即是婆罗门,\\}
\textbf{「若他有所牵绊,只名为敬语者,\\}
\textbf{「无牵绊、无执取,我说他是婆罗门。\footnote{第 626~653 颂同\textbf{法句}·婆罗门品第 396~423 颂。}}

Na cāhaṃ brāhmaṇaṃ brūmi, yonijaṃ mattisambhavaṃ;\\
bhovādi nāma so hoti, sace hoti sakiñcano;\\
akiñcanaṃ anādānaṃ, tam ahaṃ brūmi brāhmaṇaṃ. %\hfill\textcolor{gray}{\footnotesize 27}

\begin{enumerate}\item 如是以婆罗门教及世间习俗证明了破失活命、戒、正行者的非婆罗门性,故不是由出身为婆罗门,而是由功德为婆罗门,所以,无论于何族姓出身的有德者,他即是婆罗门,这于此是正理。如是,从意义既推出此正理,为再以言词的区分阐明此正理,说了此颂。
\item 其义为:而若于四胎中任何处所生者,或此处特别地,若源于被称为婆罗门的母亲,他即\textbf{从胎所生、源于母亲者}。且这以\begin{quoting}从两边善生。(长部·种德经第 303 段)\end{quoting}等方法被婆罗门谈论的称作婆罗门之遍净生产之道的母胎,以及以「腹胎纯正」谈论的母亲的成就,由从此出生、发源故,也被称为「从胎所生、源于母亲者」。\textbf{我不说}此从胎所生、源于母亲者仅以从胎所生、源于母亲\textbf{即是婆罗门}。
\item 为什么?因为由区别于其他仅说「先生、先生」之语但却有所牵绊者之故,\textbf{若他有所牵绊,只名为敬语者}\footnote{敬语者 \textit{bhovādī}:字面意思为「说着『先生、您』的人」,叶均译作「说菩者」,也极形象。}。而无论于何族姓出身者,以无有贪等的牵绊而无牵绊,并以舍遣一切抓取而无执取,\textbf{无牵绊、无执取,我说他是婆罗门}。为什么?因为已除去恶。\end{enumerate}

\subsection\*{\textbf{627} {\footnotesize 〔PTS 621〕}}

\textbf{「切断了一切结缚,他不再恐惧,\\}
\textbf{「超越执著、离轭,我说他是婆罗门。}

Sabbasaṃyojanaṃ chetvā, yo ve na paritassati;\\
saṅgātigaṃ visaṃyuttaṃ, tam ahaṃ brūmi brāhmaṇaṃ. %\hfill\textcolor{gray}{\footnotesize 28}

\begin{enumerate}\item 且更有以「切断了一切结缚」为首的二十七颂。这里,\textbf{一切结缚},即十种结缚\footnote{十种结缚,见\textbf{正游行经}第 366 颂的注。}。\textbf{不再恐惧},即不因爱而恐惧。在「我说他……」中,即\textbf{我说他},这由超越贪等执著而\textbf{超越执著}、以无有四轭而\textbf{离轭}者\textbf{是婆罗门}之义。\end{enumerate}

\subsection\*{\textbf{628} {\footnotesize 〔PTS 622〕}}

\textbf{「切断了带、纽、缰与辔,\\}
\textbf{「已拔出闩,已觉悟,我说他是婆罗门。}

Chetvā naddhiṃ varattañ ca, sandānaṃ sahanukkamaṃ;\\
ukkhittapalighaṃ buddhaṃ, tam ahaṃ brūmi brāhmaṇaṃ. %\hfill\textcolor{gray}{\footnotesize 29}

\begin{enumerate}\item \textbf{带}即以束缚之相转起的忿恨,\textbf{纽}即以捆缚之相转起的渴爱,\textbf{缰与辔}即与随眠之辔相伴的六十二见之缰,切断了这一切而住,由拔出无明之闩而\textbf{已拔出闩},由觉悟四谛而\textbf{已觉悟},我说是婆罗门之义。\end{enumerate}

\subsection\*{\textbf{629} {\footnotesize 〔PTS 623〕}}

\textbf{「无嗔者忍受骂詈、殴打、捆缚,\\}
\textbf{「具忍力者,具强军者,我说他是婆罗门。}

Akkosaṃ vadhabandhañ ca, aduṭṭho yo titikkhati;\\
khantībalaṃ balānīkaṃ, tam ahaṃ brūmi brāhmaṇaṃ. %\hfill\textcolor{gray}{\footnotesize 30}

\begin{enumerate}\item \textbf{无嗔者}如是为十种骂詈事所\textbf{骂詈}、为手掌等所击打、为锁链等所\textbf{捆缚},能不忿怒而忍耐,由具足忍力为\textbf{具忍力者},由具足因再再生起而如军队般的忍力之强军为\textbf{具强军者},我说如这般者是婆罗门之义。\end{enumerate}

\subsection\*{\textbf{630} {\footnotesize 〔PTS 624〕}}

\textbf{「无忿怒,具仪法,具戒,无增盛,\\}
\textbf{「调御,最后身,我说他是婆罗门。}

Akkodhanaṃ vatavantaṃ, sīlavantaṃ anussadaṃ;\\
dantaṃ antimasārīraṃ, tam ahaṃ brūmi brāhmaṇaṃ. %\hfill\textcolor{gray}{\footnotesize 31}

\begin{enumerate}\item \textbf{具仪法}即具足头陀行,以四遍净戒为\textbf{具戒},以无有渴爱的增盛\footnote{在\textbf{会堂经}第 521 颂注中说有七种增盛,但没有此处提到的「渴爱」。}为\textbf{无增盛},以调御六根为\textbf{调御},以自体住于终点为\textbf{最后身},我说他是婆罗门之义。\end{enumerate}

\subsection\*{\textbf{631} {\footnotesize 〔PTS 625〕}}

\textbf{「如莲叶上的水珠,如锥尖上的芥子,\\}
\textbf{「若不著于爱欲者,我说他是婆罗门。}

Vāri pokkharapatte va, āragge-r-iva sāsapo;\\
yo na limpati kāmesu, tam ahaṃ brūmi brāhmaṇaṃ. %\hfill\textcolor{gray}{\footnotesize 32}

\begin{enumerate}\item \textbf{若不著于},即如是,若不著于内在的两种爱欲,不安立于此爱欲者,我说他是婆罗门之义。\end{enumerate}

\subsection\*{\textbf{632} {\footnotesize 〔PTS 626〕}}

\textbf{「若唯于此了知自身之苦的灭尽,\\}
\textbf{「放下重担、离轭,我说他是婆罗门。}

Yo dukkhassa pajānāti, idh’eva khayam attano;\\
pannabhāraṃ visaṃyuttaṃ, tam ahaṃ brūmi brāhmaṇaṃ. %\hfill\textcolor{gray}{\footnotesize 33}

\begin{enumerate}\item \textbf{苦}即蕴之苦,\textbf{放下重担}即舍离蕴之重担,从四轭或一切烦恼\textbf{离轭},我说他是婆罗门之义。\end{enumerate}

\subsection\*{\textbf{633} {\footnotesize 〔PTS 627〕}}

\textbf{「深慧,有智,熟知道与非道,\\}
\textbf{「证得最上义利,我说他是婆罗门。}

Gambhīrapaññaṃ medhāviṃ, maggāmaggassa kovidaṃ;\\
uttamattham anuppattaṃ, tam ahaṃ brūmi brāhmaṇaṃ. %\hfill\textcolor{gray}{\footnotesize 34}

\begin{enumerate}\item \textbf{深慧}即具足于甚深的蕴等转起的慧,以法滋益的慧为\textbf{有智},以「此向恶趣、此向善趣、此是涅槃之道、此是非道」等如是娴熟于道、非道为\textbf{熟知道与非道},\textbf{证得}被称为阿罗汉的\textbf{最上义利},我说他是婆罗门之义。\end{enumerate}

\subsection\*{\textbf{634} {\footnotesize 〔PTS 628〕}}

\textbf{「不与在家人及出家人两者交际,\\}
\textbf{「无家而行,少欲,我说他是婆罗门。}

Asaṃsaṭṭhaṃ gahaṭṭhehi, anāgārehi cūbhayaṃ;\\
anokasārim appicchaṃ, tam ahaṃ brūmi brāhmaṇaṃ. %\hfill\textcolor{gray}{\footnotesize 35}

\begin{enumerate}\item 以无有见、闻、交谈、受用、身体的接触为\textbf{不交际}\footnote{五种交际,见\textbf{犀牛角经}第 36 颂注。},\textbf{无家而行}即无执著而行,我说如这般者是婆罗门之义。\end{enumerate}

\subsection\*{\textbf{635} {\footnotesize 〔PTS 629〕}}

\textbf{「对弱的与强的生物放下了棍杖,\\}
\textbf{「若不杀、不教人杀,我说他是婆罗门。}

Nidhāya daṇḍaṃ bhūtesu, tasesu thāvaresu ca;\\
yo na hanti na ghāteti, tam ahaṃ brūmi brāhmaṇaṃ. %\hfill\textcolor{gray}{\footnotesize 36}

\begin{enumerate}\item \textbf{放下},即丢弃、搁置。\textbf{弱的与强的},即以爱与害怕为弱的,以无有爱而坚定为强的。\textbf{若不杀},即若如是对一切有情以离去嗔恚而丢弃了棍杖,既不亲自杀,也不教他人杀,我说他是婆罗门之义。\end{enumerate}

\subsection\*{\textbf{636} {\footnotesize 〔PTS 630〕}}

\textbf{「敌对者中无敌对者,持棍杖者中止息者,\\}
\textbf{「有执取者中无执取者,我说他是婆罗门。}

Aviruddhaṃ viruddhesu, attadaṇḍesu nibbutaṃ;\\
sādānesu anādānaṃ, tam ahaṃ brūmi brāhmaṇaṃ. %\hfill\textcolor{gray}{\footnotesize 37}

\begin{enumerate}\item \textbf{无敌对者},即在因嫌恨而\textbf{敌对}的世间大众中,以无有嫌恨而为无敌对者,当手中有棍杖、刀剑,或即便没有,由不戒离攻击他人的\textbf{持棍杖}的人众中,\textbf{止息}而丢弃棍杖者,由取著五蕴为「我、我所」的\textbf{有执取者}中,无有此取著为\textbf{无执取者},我说如这般者是婆罗门之义。\end{enumerate}

\subsection\*{\textbf{637} {\footnotesize 〔PTS 631〕}}

\textbf{「若其贪、嗔、慢、覆藏都已脱落,\\}
\textbf{「如芥子从锥尖一般,我说他是婆罗门。}

Yassa rāgo ca doso ca, māno makkho ca pātito;\\
sāsapo-r-iva āraggā, tam ahaṃ brūmi brāhmaṇaṃ. %\hfill\textcolor{gray}{\footnotesize 38}

\begin{enumerate}\item \textbf{从锥尖},即他的这些贪等,与这以他人功德伪装为相的覆藏,如芥子从锥尖一般脱落,好比芥子不能安立于锥尖,如是不能住立于心,我说他是婆罗门之义。\end{enumerate}

\subsection\*{\textbf{638} {\footnotesize 〔PTS 632〕}}

\textbf{「能说不粗俗、有内容、真实之语,\\}
\textbf{「不以之冒犯任何人,我说他是婆罗门。}

Akakkasaṃ viññāpaniṃ, giraṃ saccam udīraye;\\
yāya nābhisaje kañci, tam ahaṃ brūmi brāhmaṇaṃ. %\hfill\textcolor{gray}{\footnotesize 39}

\begin{enumerate}\item \textbf{不粗俗},即不粗恶。\textbf{有内容},即有意义的内容。\textbf{真实},即存在。\textbf{不冒犯},即不以此语言以激怒他人的方式教人固著\footnote{义注之所以这样说,是这里的「冒犯 \textit{abhisaje}」兼有固著的意思,详见菩提比丘注 1594。}。唯漏尽者会说这样的话,所以我说他是婆罗门之义。\end{enumerate}

\subsection\*{\textbf{639} {\footnotesize 〔PTS 633〕}}

\textbf{「若于此,或长或短、或细或粗、净与不净,\\}
\textbf{「不取世间所不与者,我说他是婆罗门。}

Yo’dha dīghaṃ va rassaṃ vā, aṇuṃ thūlaṃ subhāsubhaṃ;\\
loke adinnaṃ nādiyati, tam ahaṃ brūmi brāhmaṇaṃ. %\hfill\textcolor{gray}{\footnotesize 40}

\begin{enumerate}\item 衣服、璎珞等中\textbf{或长或短},摩尼、珍珠等中\textbf{或细或粗},以高价、低价为\textbf{净与不净},若人于此\textbf{世间不取}他人所拥有者,我说他是婆罗门之义。\end{enumerate}

\subsection\*{\textbf{640} {\footnotesize 〔PTS 634〕}}

\textbf{「于此世及他世,若其不存希望,\\}
\textbf{「离欲、离轭,我说他是婆罗门。}

Āsā yassa na vijjanti, asmiṃ loke paramhi ca;\\
nirāsāsaṃ visaṃyuttaṃ, tam ahaṃ brūmi brāhmaṇaṃ. %\hfill\textcolor{gray}{\footnotesize 41}

\begin{enumerate}\item \textbf{离欲}即离渴爱,\textbf{离轭}即离一切烦恼,我说他是婆罗门之义。\end{enumerate}

\subsection\*{\textbf{641} {\footnotesize 〔PTS 635〕}}

\textbf{「若其不存执著,已知而无疑,\\}
\textbf{「证得不死之地,我说他是婆罗门。}

Yassālayā na vijjanti, aññāya akathaṅkathī;\\
amatogadham anuppattaṃ, tam ahaṃ brūmi brāhmaṇaṃ. %\hfill\textcolor{gray}{\footnotesize 42}

\begin{enumerate}\item \textbf{执著},即渴爱。\textbf{已知而无疑},即如实已知八事,对八事的疑惑去疑\footnote{八事的疑惑,见\textbf{宝经}第 233 颂的注。}。\textbf{证得不死之地},即跃入了不死涅槃而证得,我说他是婆罗门之义。\end{enumerate}

\subsection\*{\textbf{642} {\footnotesize 〔PTS 636〕}}

\textbf{「若于此超越福与恶两者的染著,\\}
\textbf{「无忧、离尘、清净,我说他是婆罗门。}

Yo’dha puññañ ca pāpañ ca, ubho saṅgam upaccagā;\\
asokaṃ virajaṃ suddhaṃ, tam ahaṃ brūmi brāhmaṇaṃ. %\hfill\textcolor{gray}{\footnotesize 43}

\begin{enumerate}\item \textbf{两者},即舍弃了福与恶两者之义。\textbf{染著},即贪等类的染著。\textbf{超越},即越过。我说他(无)以流转根本之忧而\textbf{无忧},以无有内在贪尘等而\textbf{离尘},以离随烦恼而\textbf{清净},即是婆罗门之义。\end{enumerate}

\subsection\*{\textbf{643} {\footnotesize 〔PTS 637〕}}

\textbf{「如月一般离垢,清净、明净而不污浊,\\}
\textbf{「灭尽了喜与有,我说他是婆罗门。}

Candaṃ va vimalaṃ suddhaṃ, vippasannam anāvilaṃ;\\
nandībhavaparikkhīṇaṃ, tam ahaṃ brūmi brāhmaṇaṃ. %\hfill\textcolor{gray}{\footnotesize 44}

\begin{enumerate}\item \textbf{离垢},即无有黑云等垢。\textbf{清净},即离随烦恼。\textbf{明净},即心净喜。\textbf{不污浊},即无烦恼污浊。\textbf{灭尽了喜与有},即于三有灭尽了渴爱\footnote{灭尽了喜与有:这里义注的解释和\textbf{雪山经}第 177 颂的不同,颂中的译文仍与第 177 颂保持一致。},我说他是婆罗门之义。\end{enumerate}

\subsection\*{\textbf{644} {\footnotesize 〔PTS 638〕}}

\textbf{「若超越了这险路、难路、轮回与愚痴,\\}
\textbf{「已度,已到彼岸,禅修者不动、无疑,\\}
\textbf{「以无所取著而止息,我说他是婆罗门。}

Yo’maṃ palipathaṃ duggaṃ, saṃsāraṃ moham accagā;\\
tiṇṇo pāraṅgato jhāyī, anejo akathaṅkathī;\\
anupādāya nibbuto, tam ahaṃ brūmi brāhmaṇaṃ. %\hfill\textcolor{gray}{\footnotesize 45}

\begin{enumerate}\item \textbf{若}比丘越过了\textbf{这}贪之\textbf{险路},以及烦恼之\textbf{难路}、\textbf{轮回}之流转、未通达四谛之\textbf{愚痴},则\textbf{已度}四暴流,到达彼岸,以二种禅那\textbf{禅修者}以无有渴爱而\textbf{不动},以无有疑惑而\textbf{无疑},以无有取著而无所取著,以烦恼寂灭而\textbf{止息},我说他是婆罗门之义。\end{enumerate}

\subsection\*{\textbf{645} {\footnotesize 〔PTS 639〕}}

\textbf{「若于此舍弃了爱欲,无家游行,\\}
\textbf{「灭尽了爱欲与有,我说他是婆罗门。}

Yo’dha kāme pahantvāna, anāgāro paribbaje;\\
kāmabhavaparikkhīṇaṃ, tam ahaṃ brūmi brāhmaṇaṃ. %\hfill\textcolor{gray}{\footnotesize 46}

\begin{enumerate}\item \textbf{若}人\textbf{于此}世间舍弃了两种\textbf{爱欲},则\textbf{无家}而游行,既灭尽了爱欲,又灭尽了有,我说他是婆罗门之义。\end{enumerate}

\subsection\*{\textbf{646} {\footnotesize 〔PTS 640〕}}

\textbf{「若于此舍弃了渴爱,无家游行,\\}
\textbf{「灭尽了渴爱与有,我说他是婆罗门。}

Yo’dha taṇhaṃ pahantvāna, anāgāro paribbaje;\\
taṇhābhavaparikkhīṇaṃ, tam ahaṃ brūmi brāhmaṇaṃ. %\hfill\textcolor{gray}{\footnotesize 47}

\begin{enumerate}\item \textbf{}若于此\textbf{世间舍弃了对六门的}渴爱\textbf{,以不希求居家而}无家\textbf{游行,且由灭尽了渴爱与有而}灭尽了渴爱与有\textit{},我说他是婆罗门之义。\end{enumerate}

\subsection\*{\textbf{647} {\footnotesize 〔PTS 641〕}}

\textbf{「舍弃了人类之轭,超越了天界之轭,\\}
\textbf{「离于一切轭,我说他是婆罗门。}

Hitvā mānusakaṃ yogaṃ, dibbaṃ yogaṃ upaccagā;\\
sabbayogavisaṃyuttaṃ, tam ahaṃ brūmi brāhmaṇaṃ. %\hfill\textcolor{gray}{\footnotesize 48}

\begin{enumerate}\item \textbf{人类之轭},即人类的寿量与五种爱欲。\textbf{天界之轭}亦然。\textbf{超越了},即若舍弃了人类之轭,越过了天界,以离于一切四轭,我说他是婆罗门之义。\end{enumerate}

\subsection\*{\textbf{648} {\footnotesize 〔PTS 642〕}}

\textbf{「舍弃了乐与不乐,得成清凉,无有依持,\\}
\textbf{「征服一切世间的英雄,我说他是婆罗门。}

Hitvā ratiñ ca aratiṃ, sītibhūtaṃ nirūpadhiṃ;\\
sabbalokābhibhuṃ vīraṃ, tam ahaṃ brūmi brāhmaṇaṃ. %\hfill\textcolor{gray}{\footnotesize 49}

\begin{enumerate}\item \textbf{乐},即种种五欲之乐。\textbf{不乐},即不安于住在林野。\textbf{得成清凉},即止息。\textbf{无有依持}\footnote{依持,见\textbf{有财者经}第 33 颂注。},即离随烦恼。\textbf{英雄},即我说如这般征服了一切蕴世间而住、具精进者是婆罗门之义。\end{enumerate}

\subsection\*{\textbf{649} {\footnotesize 〔PTS 643〕}}

\textbf{「若完全知晓了有情的亡殁与投生,\\}
\textbf{「无著、善逝、觉悟,我说他是婆罗门。}

Cutiṃ yo vedi sattānaṃ, upapattiñ ca sabbaso;\\
asattaṃ sugataṃ buddhaṃ, tam ahaṃ brūmi brāhmaṇaṃ. %\hfill\textcolor{gray}{\footnotesize 50}

\begin{enumerate}\item \textbf{若知晓了},即若以一切行相令有情的死亡与结生明了而知晓,我说他以不固著而\textbf{无著},由以行道善往而\textbf{善逝},以觉悟四谛而\textbf{觉悟},是婆罗门之义。\end{enumerate}

\subsection\*{\textbf{650} {\footnotesize 〔PTS 644〕}}

\textbf{「诸天、乾闼婆及人都不知其趣向,\\}
\textbf{「漏尽的阿罗汉,我说他是婆罗门。}

Yassa gatiṃ na jānanti, devā gandhabbamānusā;\\
khīṇāsavaṃ arahantaṃ, tam ahaṃ brūmi brāhmaṇaṃ. %\hfill\textcolor{gray}{\footnotesize 51}

\begin{enumerate}\item 这些诸天等不知其趣向,我说他以诸漏已尽而\textbf{漏尽},由回避烦恼而为\textbf{阿罗汉},是婆罗门之义。\end{enumerate}

\subsection\*{\textbf{651} {\footnotesize 〔PTS 645〕}}

\textbf{「若其前、后、中间都没有牵绊,\\}
\textbf{「无牵绊、无执取,我说他是婆罗门。}

Yassa pure ca pacchā ca, majjhe ca natthi kiñcanaṃ;\\
akiñcanaṃ anādānaṃ, tam ahaṃ brūmi brāhmaṇaṃ. %\hfill\textcolor{gray}{\footnotesize 52}

\begin{enumerate}\item \textbf{前}即过去诸蕴,\textbf{后}即将来,\textbf{中间}即现在。\textbf{牵绊},即他对这些处没有被称为爱执的牵绊。我说他以(无)贪的牵绊等而\textbf{无牵绊},以无有任何执取而\textbf{无执取},是婆罗门之义。\end{enumerate}

\subsection\*{\textbf{652} {\footnotesize 〔PTS 646〕}}

\textbf{「公牛,高贵者,英雄,大仙,战胜者,\\}
\textbf{「不动,沐浴者,已觉悟,我说他是婆罗门。}

Usabhaṃ pavaraṃ vīraṃ, mahesiṃ vijitāvinaṃ;\\
anejaṃ nhātakaṃ buddhaṃ, tam ahaṃ brūmi brāhmaṇaṃ. %\hfill\textcolor{gray}{\footnotesize 53}

\begin{enumerate}\item 以不颤抖而与公牛一般为\textbf{公牛},以最上之义为\textbf{高贵},以精进的成就为\textbf{英雄},由寻求大的戒蕴等为\textbf{大仙},由战胜三种魔罗\footnote{据菩提比丘注 1605,即除死以外的烦恼、蕴、天子三种魔罗。}为\textbf{战胜者},以洗去烦恼为\textbf{沐浴者},以觉悟四谛为\textbf{已觉悟},我说如这般者为婆罗门之义。\end{enumerate}

\subsection\*{\textbf{653} {\footnotesize 〔PTS 647〕}}

\textbf{「若知晓先前的住处,得见天界与苦处,\\}
\textbf{「然后证得生的灭尽,我说他是婆罗门。\footnote{此颂较\textbf{法句}第 423 颂少了「业已完成无上智,一切圆满成就者」两句。}}

Pubbenivāsaṃ yo vedi, saggāpāyañ ca passati;\\
atho jātikkhayaṃ patto, tam ahaṃ brūmi brāhmaṇaṃ. %\hfill\textcolor{gray}{\footnotesize 54}

\begin{enumerate}\item \textbf{若}令\textbf{先前的住处}明了而知晓,以天眼\textbf{得见}二十六种天世间之\textbf{天界}及四种\textbf{苦处},\textbf{然后证得}被称为\textbf{生的灭尽}的阿罗汉,我说他是婆罗门之义。\end{enumerate}

\subsection\*{\textbf{654} {\footnotesize 〔PTS 648〕}}

\textbf{「这世间遍计的姓名、种姓只是名称,\\}
\textbf{「从共许产生,随处被遍计。}

Samaññā h’esā lokasmiṃ, nāmagottaṃ pakappitaṃ;\\
sammuccā samudāgataṃ, tattha tattha pakappitaṃ. %\hfill\textcolor{gray}{\footnotesize 55}

\begin{enumerate}\item 如是,世尊从功德说了婆罗门,为显示「那些执著于『由出身就是婆罗门』者,他们不知这只是习俗,且他们的这个见为恶见」,说了以下二颂。
\item 其义为:当知如「他是婆罗门、刹帝利、婆罗豆婆遮、婆悉吒」等\textbf{这世间遍计的姓名、种姓只是名称}\footnote{遍计 \textit{pakappita}:这里随旧译的译名,意即「预设、刻意创造」,相对于自然形成而言。},只是概念、习俗。为什么?因为\textbf{从共许产生},由许可而来。因为在其出生时,\textbf{随处被}亲戚、血亲\textbf{遍计}。若其不被如是遍计,无人看见任何人能知晓「他是婆罗门」或「婆罗豆婆遮」。\end{enumerate}

\subsection\*{\textbf{655} {\footnotesize 〔PTS 649〕}}

\textbf{「无知者的成见已长时随眠,\\}
\textbf{「唯无知者说『由出身就是婆罗门』。}

Dīgharattam anusayitaṃ, diṭṭhigatam ajānataṃ;\\
ajānantā no pabruvanti, ‘jātiyā hoti brāhmaṇo’. %\hfill\textcolor{gray}{\footnotesize 56}

\begin{enumerate}\item 如是,且对此遍计,\textbf{无知者的成见已长时随眠},不知「姓名、种姓」的有情在心中已长时随眠成见,由其随眠,\textbf{无知}此姓名、种姓者便\textbf{说由出身就是婆罗门},即是说唯有无知者作如是说。\end{enumerate}

\subsection\*{\textbf{656} {\footnotesize 〔PTS 650〕}}

\textbf{「不由出身而是婆罗门,不由出身而非婆罗门,\\}
\textbf{「由业而是婆罗门,由业而非婆罗门。}

Na jaccā brāhmaṇo hoti, na jaccā hoti abrāhmaṇo;\\
kammunā brāhmaṇo hoti, kammunā hoti abrāhmaṇo. %\hfill\textcolor{gray}{\footnotesize 57}

\begin{enumerate}\item 如是,已显示了「那些执著于『由出身就是婆罗门』者,他们不知这只是习俗,且他们的这个见为恶见」,现在,为直接驳斥出身论,并为阐述业论,说了此颂。\end{enumerate}

\subsection\*{\textbf{657} {\footnotesize 〔PTS 651〕}}

\textbf{「由业而是耕者,由业而是匠人,\\}
\textbf{「由业而是商人,由业而是仆人。}

Kassako kammunā hoti, sippiko hoti kammunā;\\
vāṇijo kammunā hoti, pessiko hoti kammunā. %\hfill\textcolor{gray}{\footnotesize 58}

\begin{enumerate}\item 这里,为详明「由业而是婆罗门,由业而非婆罗门」半颂之义,而说「由业而是耕者」等等。这里,\textbf{业},即现在耕田等所生的思业。\end{enumerate}

\subsection\*{\textbf{658} {\footnotesize 〔PTS 652〕}}

\textbf{「由业而是贼人,由业而是战士,\\}
\textbf{「由业而是祭司,由业而是国王。}

Coro pi kammunā hoti, yodhājīvo pi kammunā;\\
yājako kammunā hoti, rājā pi hoti kammunā. %\hfill\textcolor{gray}{\footnotesize 59}

\subsection\*{\textbf{659} {\footnotesize 〔PTS 653〕}}

\textbf{「见缘起者、熟知业与异熟者、\\}
\textbf{「智者,如是如实地得见此业。\footnote{此颂的译文将上下两行倒置。}}

Evam etaṃ yathābhūtaṃ, kammaṃ passanti paṇḍitā;\\
paṭiccasamuppādadassā, kammavipākakovidā. %\hfill\textcolor{gray}{\footnotesize 60}

\begin{enumerate}\item \textbf{见缘起者},即如「因此缘而成如是」的见缘起者。\textbf{熟知业与异熟者},即如「由业而生于应尊敬或鄙视的族姓,其它贵贱的属性也在贵贱之业成熟时而成」等善巧于业与异熟。\end{enumerate}

\subsection\*{\textbf{660} {\footnotesize 〔PTS 654〕}}

\textbf{「世间由业转起,人类由业转起,\\}
\textbf{「有情系缚于业,如行进之车的车辖。}

Kammunā vattati loko, kammunā vattati pajā;\\
kammanibandhanā sattā, rathassāṇīva yāyato. %\hfill\textcolor{gray}{\footnotesize 61}

\begin{enumerate}\item 颂中\textbf{由业转起}的\textbf{世间、人类、有情}之义唯一,仅词各异。且此中,当知以第一句遮止\begin{quoting}存在梵、大梵……最胜者、创造者、控制者、已生及将生者之父。(长部·梵网经第 42 段)\end{quoting}等见。因为世间由业转起,投生至彼彼诸趣,何来其创造者?以第二句显示「如是由业投生已,在转起中,也由过去、现在等类的业而转起,经历着苦乐、遭遇着贵贱等而转起」。以第三句唯总结此义「如是于一切处,有情系缚于业,唯受缚于业转起,无有它处」。以第四句用譬喻阐明此义,\textbf{如行进之车的车辖}。好比车辖是行进之车的系缚,非不系缚于此而行,如是业是世间投生、转起的系缚,非不系缚于此而投生、转起。\end{enumerate}

\subsection\*{\textbf{661} {\footnotesize 〔PTS 655〕}}

\textbf{「以苦行、以梵行、以自制、以调御,\\}
\textbf{「以此而成婆罗门,此是最上婆罗门。}

Tapena brahmacariyena, saṃyamena damena ca;\\
etena brāhmaṇo hoti, etaṃ brāhmaṇam uttamaṃ. %\hfill\textcolor{gray}{\footnotesize 62}

\begin{enumerate}\item 现在,因为世间如是系缚于业,所以为显示以最胜之业而成最胜,说了以下二颂。这里,\textbf{苦行},即根律仪。\textbf{梵行},即依于学的其它最胜行。\textbf{自制},即戒。\textbf{调御},即慧。\textbf{以此},即以最胜义的作为梵的业\textbf{而成婆罗门}。为什么?因为\textbf{此是最上婆罗门},即是说因为此业是最上的婆罗门性。文本亦作「梵 \textit{brahmānam}」,其义即导向梵,即是说导向、带来、给予梵的状态。\end{enumerate}

\subsection\*{\textbf{662} {\footnotesize 〔PTS 656〕}}

\textbf{「具足三明,寂静,灭尽再有,\\}
\textbf{「婆悉吒!如是当知对有识者,是梵、帝释。」}

Tīhi vijjāhi sampanno, santo khīṇapunabbhavo;\\
evaṃ Vāseṭṭha jānāhi, Brahmā Sakko vijānatan” ti. %\hfill\textcolor{gray}{\footnotesize 63}

\begin{enumerate}\item 在第二颂中,\textbf{寂静},即止息了烦恼。\textbf{梵、帝释},即是说婆悉吒!如是当知这样的人不止是婆罗门,对于有识的智者,他还是梵与帝释。其余唯如所述。\end{enumerate}

\textbf{如是说已,学童婆悉吒、婆罗豆婆遮对世尊说:「希有!乔达摩君!……从今起,尽寿命,请乔达摩君受持我们皈依为优婆塞!」}

Evaṃ vutte, Vāseṭṭha-Bhāradvājā māṇavā Bhagavantaṃ etad avocuṃ: “abhikkantaṃ, bho Gotama…pe… upāsake no bhavaṃ Gotamo dhāretu ajjatagge pāṇupete saraṇaṃ gate” ti.

\begin{center}\vspace{1em}婆悉吒经第九\\Vāseṭṭhasuttaṃ navamaṃ.\end{center}