\section{如法经}

\begin{center}Dhammika Sutta\end{center}\vspace{1em}

\textbf{如是我闻。一时世尊住舍卫国祇树给孤独园。尔时,优婆塞如法与五百优婆塞一起,往世尊处走去,走到后,礼敬了世尊,坐在一边。坐在一边的优婆塞如法以偈颂对世尊说:}

Evaṃ me sutaṃ— ekaṃ samayaṃ Bhagavā Sāvatthiyaṃ viharati Jetavane Anāthapiṇḍikassa ārāme. Atha kho Dhammiko upāsako pañcahi upāsakasatehi saddhiṃ yena Bhagavā ten’upasaṅkami, upasaṅkamitvā Bhagavantaṃ abhivādetvā ekamantaṃ nisīdi. Ekamantaṃ nisinno kho Dhammiko upāsako Bhagavantaṃ gāthāhi ajjhabhāsi:

\begin{enumerate}\item 缘起为何?据说,当世尊世主住世时,有位名为如法的优婆塞,名行相符。据说,他具足皈依、具足戒、多闻、持三藏、为阿那含、得神通、是空行者。其随从为五百优婆塞,他们也都如此。一天,这持布萨者幽居宴坐,在中夜将尽时分,便起如是寻思:「我何不去问在家与出家的行道?」他为五百优婆塞随从,便去到世尊处,问了此义,而世尊则为其作答。这里,与先前解释相同者,当以所述之理可知,我们将解释先前未(述者)。\end{enumerate}

\subsection\*{\textbf{379} {\footnotesize 〔PTS 376〕}}

\textbf{「我问你,宏慧的乔达摩!如何行事的弟子才是善的?\\}
\textbf{「无论他从家至于非家,或是居家的优婆塞。}

“Pucchāmi taṃ Gotama bhūripañña, kathaṅkaro sāvako sādhu hoti;\\
yo vā agārā anagāram eti, agārino vā pan’upāsakāse. %\hfill\textcolor{gray}{\footnotesize 1}

\begin{enumerate}\item 这里,先就初颂中的\textbf{如何行事},即如何做、如何行道。\textbf{善},即善妙、无过、成就义利。其余之义自明。而其连结为:无论他从家至于、出家为非家,或是居家的优婆塞,于这两种弟子中,如何行事的弟子才是善的。\end{enumerate}

\subsection\*{\textbf{380} {\footnotesize 〔PTS 377〕}}

\textbf{「因为你知晓俱有天的世间的趣向和归宿,\\}
\textbf{「且无有等同,见微妙义者!他们说你是高贵的佛陀。}

Tuvañ hi lokassa sadevakassa, gatiṃ pajānāsi parāyaṇañ ca;\\
na c’atthi tulyo nipuṇatthadassī, tuvañ hi Buddhaṃ pavaraṃ vadanti. %\hfill\textcolor{gray}{\footnotesize 2}

\begin{enumerate}\item 现在,为显明被如是提问的世尊堪能作答,说了以下两颂。这里,\textbf{趣向}即意乐之趣向。\textbf{归宿},即结果。或者,趣向即地狱等的五类,归宿即离于趣向的别径,脱离趣向的般涅槃。\textbf{且无有等同},即无有与你相等者。\end{enumerate}

\subsection\*{\textbf{381} {\footnotesize 〔PTS 378〕}}

\textbf{「你证得了一切智,为怜悯有情而阐明法,\\}
\textbf{「你是去蔽者、一切眼者,无垢者在一切世间闪耀。}

Sabbaṃ tuvaṃ ñāṇam avecca dhammaṃ, pakāsesi satte anukampamāno;\\
vivaṭṭacchado’si samantacakkhu, virocasi vimalo sabbaloke. %\hfill\textcolor{gray}{\footnotesize 3}

\begin{enumerate}\item \textbf{你证得了一切智,为怜悯有情而阐明法},即你,世尊!凡是应知者,都已无余地证得、通达,为怜悯有情而阐明一切智与法。凡对彼彼有益者,你都会对其解释、开示,即是说对你而言,不存在老师的秘传。\textbf{无垢者闪耀},即如除去烟尘等的月一般,无垢者以无有贪等的垢秽而闪耀。其余之义于此自明。\end{enumerate}

\subsection\*{\textbf{382} {\footnotesize 〔PTS 379〕}}

\textbf{「名为伊罗婆那的象王来到你的面前,听闻『胜者』后,\\}
\textbf{「他也向你讨教,听后说了『善哉』,喜形于色而离开。}

Āgañchi te santike nāgarājā, Erāvaṇo nāma ‘Jino’ ti sutvā;\\
so pi tayā mantayitvājjhagamā, ‘sādhū’ ti sutvāna patītarūpo. %\hfill\textcolor{gray}{\footnotesize 4}

\begin{enumerate}\item 现在,为赞叹世尊,先称誉世尊此时对其开示法的天子,说了以下两颂。这里,\textbf{名为伊罗婆那的象王},据说,这名为伊罗婆那的天子,形随所欲,住在天宫。当帝释来庭园嬉戏时,便将躯体化作一百五十由旬,变出三十三瘤,成了名为伊罗婆那的大象。其一一瘤中各有二牙,一一牙上各有七池,一一池中各有七莲,一一莲上各有七花,一一花间各有七叶,一一叶上各有七天女舞蹈,即著名的帝释天女——莲天女,且在「天宫事」中提起:\begin{quoting}众娴习的女子在莲上盘旋。(天宫事第 1034 颂)\end{quoting}
\item 而在此三十三瘤的中间,有名为善见的瘤,量有三十由旬,于此,一由旬之量的摩尼宝座被敷展在三由旬高的花亭中,诸天因陀帝释便在此为众天女围绕,享受天福。而当诸天因陀帝释从庭园嬉戏返回时,便又收回此形,成为天子——就此而说「名为伊罗婆那的象王来到你的面前」。
\item \textbf{听闻『胜者』后},即以『此世尊战胜了恶法』而如是听闻。\textbf{他也向你讨教},即与你一起商讨,意即提了问题。\textbf{听后说了『善哉』,喜形于色而离开},即听了这问题后,心满意足,说了『善哉,尊者』,以满意之状而离开之义。\end{enumerate}

\subsection\*{\textbf{383} {\footnotesize 〔PTS 380〕}}

\textbf{「王者毗沙门·俱鞞罗也亲近你而遍问法,\\}
\textbf{「你回答了他的提问,智者!他听后也喜形于色。}

Rājā pi taṃ Vessavaṇo Kuvero, upeti dhammaṃ paripucchamāno;\\
tassāpi tvaṃ pucchito brūsi dhīra, so cāpi sutvāna patītarūpo. %\hfill\textcolor{gray}{\footnotesize 5}

\begin{enumerate}\item 此中,当知这夜叉以享乐之义而为\textbf{王者},在毗沙那王城执掌王权而为\textbf{毗沙门},以先前的名字而为\textbf{俱鞞罗}。据说,他曾是名为俱鞞罗的富裕婆罗门,造了布施等福后,转生为毗沙那王城的君主,所以被称为「俱鞞罗·毗沙门」。且在(长部)稻竿经中说:\begin{quoting}先生!俱鞞罗大王的王城名为毗沙那,所以俱鞞罗大王被称为「毗沙门」。(长部·稻竿经第 291 段)\end{quoting}其余于此自明。
\item 这里,若问:那为什么住在更远的三十三天居处的伊罗婆那先来,毗沙门后来,就住在同城的优婆塞最后?他又如何得知彼等的到来,以之而如是说?当答:据说,那时,毗沙门上到以数千珊瑚为座的十二由旬的妇运\footnote{妇运 \textit{Nārivāhana}:为毗沙门的车名。},放出珊瑚鸟,为一万俱胝的夜叉围绕,想「我要去问世尊问题」,便避开处于空中的宫殿,沿路而来,到了竹节城中难陀母优婆夷的住处上方。优婆夷有此威德:诸戒遍净,常离非时食,受持三藏,住阿那含果。她在那时开了窗,为消夏而站在风凉之处,以齐整的文句、甜美的声音诵「八颂·彼岸道品\footnote{即\textbf{经集}第四、五两品。}」。毗沙门便在那里驻车,直至优婆夷以\begin{quoting}当住于摩竭陀石支提时,世尊说了这些,对十六个随侍的婆罗门……(经集·彼岸道赞颂)\end{quoting}说了结语,听完全部后,在品的终了,伸长了如金鼓一般的脖子,鼓掌欢呼:「善哉!善哉!姊妹!」
\item 她说:「谁在这里?」「姊妹!我是毗沙门。」据说,优婆夷先成了须陀洹,毗沙门后成,他便依法就同胞之相,以姊妹之语对待此优婆夷。且当被优婆夷说到「非时,贤首之弟!你宜知时」时,便说:「姊妹!我对你净喜,我行净喜之相。」「那么,贤首!工人们不能收割我田里成熟的稻子,请命令你的随从!」他说了「善哉!姊妹」,便命令众夜叉。他们装满了一千二百五十座粮仓。从此以后,不曾欠缺的粮仓,在世间便被指认为「如同难陀母的粮仓」。毗沙门装满粮仓后,便前往世尊处。世尊说:「你非时而来。」于是,他便告知了世尊经过。以此原因,毗沙门虽然住在更近的四大王天的居处,却后来。而伊罗婆那却无任何梗阻,因此他更先来。
\item 但此优婆塞虽已是阿那含,且天性为一食者,同样在那时,想「是布萨日」,便决意了布萨支,在晡时整理衣裳,为五百优婆塞所随从,前往祇林听闻法的开示,回到自家后,为这些优婆塞讲论皈依、戒、布萨之功德等类的优婆塞法,便遣散了这些优婆塞。而就在他家,已为他们在单独的内室设好五百张足高拳肘\footnote{足高拳肘 \textit{muṭṭhi-hattha-ppamāṇa-pādaka}:足即床足,菩提比丘注 1285 云,此词意在说明布萨支中的「不坐卧高床」。}、受许可的床。他们进入各自的内室后,入等至而坐,优婆塞便也如是。
\item 且尔时,在舍卫城中,住有五百七十万户人家,以人数计,则有十八俱胝的人。因此,在初夜,由象马人鼓之声等,舍卫城如大海一般,一片声响。在随后的中夜,这声响平息。此时,优婆塞从等至出起,转向于自身的功德,想「我以此道乐、果乐得乐而住,此乐依何而得?依于世尊」,便对世尊心生净喜,转向于「世尊现今如何而住」,便以天眼看到伊罗婆那、毗沙门,以天耳界听到法的开示,以他心智了知彼等之心净喜,想:「我何不也去问世尊利益两者的行道?」所以,虽然他住在同一城内,却最后来,且如是知晓彼等前来。因此说「名为伊罗婆那……他听后也喜形于色」。\end{enumerate}

\subsection\*{\textbf{384} {\footnotesize 〔PTS 381〕}}

\textbf{「任何习于思辨的外道,无论是邪命者还是尼乾陀,\\}
\textbf{「全都无法以慧超越你,如站立者之于急速而行者。}

Ye kec’ime titthiyā vādasīlā, Ājīvakā vā yadi vā Nigaṇṭhā;\\
paññāya taṃ nātitaranti sabbe, ṭhito vajantaṃ viya sīghagāmiṃ. %\hfill\textcolor{gray}{\footnotesize 6}

\begin{enumerate}\item 现在,为以较此(教法)外的世间共许的沙门、婆罗门的高贵之相赞叹世尊,说了以下两颂。这里,\textbf{外道}\footnote{菩提比丘注 1288 云:外道一词的字面意思是指河流的津渡,也许与度过轮回之河有关。},即于难陀·婆蹉、商祇遮等三位先人创教者\footnote{\textbf{中部}第 36 经提到三个邪命者的先人是难陀·婆蹉 \textit{Nanda Vaccha}、祇舍·商祇遮 \textit{Kisa Saṅkicca}、末迦梨·瞿舍利 \textit{Makkhali Gosāla}。}所创的外道之见中所生,于彼等教法中出家的富兰那等六师\footnote{六师外道的译名,杂阿含经第 105 经中作富兰那·迦叶、末迦梨·瞿舍利子、阿耆多·翅舍钦婆罗、迦罗拘陀·迦栴延、先阇那·毗罗胝子、尼揵陀·若提子。}。这里,若提子为\textbf{尼乾陀},其余为\textbf{邪命者}。为显示彼等全体,而说\textbf{任何习于思辨的外道},即以「我们是正行道、别人是邪行道」而习于造作思辨,以唇剑刺戳世间而行。「无论是邪命者……」,即对这些在一处略说者分解后显示。\textbf{全都},即包摄其他任何的外道弟子等而说。\textbf{如站立者之于行者},意即好比某些没有目的的站立者,无法超越\textbf{急速}而行之人,如是,以无有慧的目的,这些站立者不能觉悟彼彼义类,无法超越极速行的世尊。\end{enumerate}

\subsection\*{\textbf{385} {\footnotesize 〔PTS 382〕}}

\textbf{「任何习于思辨的年长婆罗门,与任何存在的婆罗门,\\}
\textbf{「以及其他自认为是论师者,全都靠你获得义利。}

Ye kec’ime brāhmaṇā vādasīlā, vuddhā cāpi brāhmaṇā santi keci;\\
sabbe tayi atthabaddhā bhavanti, ye cāpi aññe vādino maññamānā. %\hfill\textcolor{gray}{\footnotesize 7}

\begin{enumerate}\item \textbf{任何习于思辨的年长婆罗门},至此显示旃基、多梨车、莲卧、生闻等\footnote{旃基、多梨车、莲卧、生闻:这些人亦见于\textbf{经集}·婆悉吒经。}。\textbf{与任何存在的婆罗门},以此可得任何但凡存在的中年、少年婆罗门,如是显示马刈\footnote{马刈,见\textbf{中部}·马刈经。}、婆悉吒、阿摩昼\footnote{阿摩昼,见\textbf{长部}·阿摩昼经。}、郁多罗学童\footnote{郁多罗学童,见\textbf{长部}·弊宿经。}等。\textbf{获得义利},即以「或许他能解答此问、断此疑惑」等获得义利。\textbf{以及其他},即其他\textbf{自认为}「我们\textbf{是论师}」的无量刹帝利、智者、婆罗门、梵、天、夜叉等,他们也全都靠你获得义利。\end{enumerate}

\subsection\*{\textbf{386} {\footnotesize 〔PTS 383〕}}

\textbf{「因为这法微妙且乐,它由你,世尊!所善宣说,\\}
\textbf{「全都想听闻它,当被问及,请对我们说它!最胜的佛陀!}

Ayañ hi dhammo nipuṇo sukho ca, yo’yaṃ tayā Bhagavā suppavutto;\\
tam eva sabbe pi sussūsamānā, taṃ no vada pucchito buddhaseṭṭha. %\hfill\textcolor{gray}{\footnotesize 8}

\begin{enumerate}\item 如是以种种方式赞叹世尊已,现在,仍如法赞叹他后,为请求法论,说了以下两颂。这里,\textbf{这法},是就三十七菩提分法而说。\textbf{微妙},即精致、难以通达。\textbf{乐},即当通达时,能带来出世间之乐,所以,由带来乐故,被称为乐。\textbf{所善宣说},即所善开示。\textbf{想听闻},即我们愿欲听闻之义。\textbf{请对我们说它},即请对我们说这法,文本也作 tvaṃ no,即「请你对我们说」之义。\end{enumerate}

\subsection\*{\textbf{387} {\footnotesize 〔PTS 384〕}}

\textbf{「所有这些共坐的比丘,与众优婆塞都同样想听,\\}
\textbf{「让他们听无垢者所随觉的法!如诸天听婆娑婆\footnote{婆娑婆 \textit{Vāsava}:即帝释的称呼之一。}的善说。」}

Sabbe p’ime bhikkhavo sannisinnā, upāsakā cāpi tath’eva sotuṃ;\\
suṇantu dhammaṃ vimalenānubuddhaṃ, subhāsitaṃ Vāsavasseva devā”. %\hfill\textcolor{gray}{\footnotesize 9}

\begin{enumerate}\item \textbf{所有这些比丘},即在此刻而坐者,据说有五百比丘,他显示彼等并作请求。\textbf{与众优婆塞},即显示自己的随从及他人。其余于此自明。\end{enumerate}

\subsection\*{\textbf{388} {\footnotesize 〔PTS 385〕}}

\textbf{「请听我!诸比丘!我让你们闻除遣之法,你们都应奉行之,\\}
\textbf{「见到义利的具慧者,应从事这随顺出家的威仪路。}

“Suṇātha me bhikkhavo sāvayāmi vo, dhammaṃ dhutaṃ tañ ca carātha sabbe;\\
iriyāpathaṃ pabbajitānulomikaṃ, sevetha naṃ atthadaso mutīmā. %\hfill\textcolor{gray}{\footnotesize 10}

\begin{enumerate}\item 于是,世尊为先显示出家的行道,在召唤诸比丘后,先说了此颂。这里,\textbf{除遣之法,你们都应奉行之},以除遣烦恼而为除遣,即是说我让你们闻这般除遣烦恼的行道之法,对这因我得闻者,你们都应奉行、行道,莫作放逸。\textbf{威仪路},即行等四种。\textbf{随顺出家},即适宜沙门、与念正知相应。另有人说唯以在林野从事业处所转起者。\textbf{见到义利},即随观利益。\textbf{具慧者},即具觉者。于此颂中,其余自明。\end{enumerate}

\subsection\*{\textbf{389} {\footnotesize 〔PTS 386〕}}

\textbf{「比丘不应在非时游行,而应适时在村中行乞,\\}
\textbf{「因为执著系缚非时行者,所以诸佛不在非时游行。}

No ve vikāle vicareyya bhikkhu, gāme ca piṇḍāya careyya kāle;\\
akālacāriñ hi sajanti saṅgā, tasmā vikāle na caranti buddhā. %\hfill\textcolor{gray}{\footnotesize 11}

\begin{enumerate}\item \textbf{不在非时},即如是从事随顺出家的威仪路的比丘,尚不应在非时——就越过日中而言——游行,而应于适当之\textbf{时在村中行乞}。什么原因?\textbf{因为执著系缚非时行者},即贪之执著等众多执著系缚、包裹、攥紧、紧跟非时而行的补特伽罗。\textbf{所以诸佛不在非时游行},所以,那些觉悟四谛的圣人不在非时行乞。
\item 据说,尔时尚未制定非时食学处,所以此处唯以法的开示对众凡夫显示过患,说了此颂。然而,诸圣者从得道起便戒离,此是法性。\end{enumerate}

\subsection\*{\textbf{390} {\footnotesize 〔PTS 387〕}}

\textbf{「那些让有情迷醉的色、声、味、香、触,\\}
\textbf{「应调伏对这些法的欲,他应按时进早餐。}

Rūpā ca saddā ca rasā ca gandhā, phassā ca ye sammadayanti satte;\\
etesu dhammesu vineyya chandaṃ, kālena so pavise pātarāsaṃ. %\hfill\textcolor{gray}{\footnotesize 12}

\begin{enumerate}\item 如是,在遮止非时游行后,为显示「适时游行者亦应如是游行」,说了此颂。其义为:凡是\textbf{色}等,都令生种种品类的迷醉,\textbf{让有情迷醉},对于此等,应以在(中部)乞食清净经等中所说的方法除去\textbf{欲}后,于适当之\textbf{时进早餐}。且此中,以「早上应食」为早餐,即乞食之名。而可得(乞食)之处,因与之相关,在此也被称为「早餐」。当知此中之义为:应去往可得乞食之处。\end{enumerate}

\subsection\*{\textbf{391} {\footnotesize 〔PTS 388〕}}

\textbf{「比丘按时得了团食,独自回返,应坐于幽僻处,\\}
\textbf{「内省,不应向外用意,端摄自体。}

Piṇḍañ ca bhikkhu samayena laddhā, eko paṭikkamma raho nisīde;\\
ajjhattacintī na mano bahiddhā, nicchāraye saṅgahitattabhāvo. %\hfill\textcolor{gray}{\footnotesize 13}

\begin{enumerate}\item 如是入(村)\footnote{义注将上颂的「进早餐」解释为入村乞食,故在这里将「进」译作「入(村)」。},\begin{quoting}比丘按时得了团食,独自回返,应坐于幽僻处,\\内省,不应向外用意,端摄自体。\end{quoting}这里,\textbf{团食},即混合之食。因为它从各处汇集,以混同之义,被称为团食。\textbf{按时},即日中以内之时。\textbf{独自回返},即成就身远离,无伴而返。\textbf{内省},即引入三相,思惟蕴相续。\textbf{不应向外用意},即不应以贪而将心追逐外在的色等。\textbf{端摄自体},即善加摄持心。\end{enumerate}

\subsection\*{\textbf{392} {\footnotesize 〔PTS 389〕}}

\textbf{「若他与弟子,或其他人,或任何比丘交谈,\\}
\textbf{「应谈论殊胜的法,不诽谤,也不指责他人。}

Sace pi so sallape sāvakena, aññena vā kenaci bhikkhunā vā;\\
dhammaṃ paṇītaṃ tam udāhareyya, na pesuṇaṃ no pi parūpavādaṃ. %\hfill\textcolor{gray}{\footnotesize 14}

\begin{enumerate}\item 且如是而住者,\begin{quoting}若他与弟子,或其他人,或任何比丘交谈,\\应谈论殊胜的法,不诽谤,也不指责他人。\end{quoting}这说的是什么?这瑜伽行者,若与任何前来欲闻的弟子,或与任何其他外道、在家人等,或与在此出家的比丘一起交谈,则应谈论殊胜的法——与道、果等相关者或十论事之类,以无热之义而为殊胜的法,且不应谈论其它哪怕少许的诽谤之语或指责他人之语。\end{enumerate}

\subsection\*{\textbf{393} {\footnotesize 〔PTS 390〕}}

\textbf{「因为有些人反对言论,我们不赞叹这些小慧者,\\}
\textbf{「执著从此系缚他们,因为他们把心放逐到远处。}

Vādañ hi eke paṭiseniyanti, na te pasaṃsāma parittapaññe;\\
tato tato ne pasajanti saṅgā, cittañ hi te tattha gamenti dūre. %\hfill\textcolor{gray}{\footnotesize 15}

\begin{enumerate}\item 现在,为显示指责他人的过失,说了此颂。其义为:于此,有些愚痴之人\textbf{反对}、抵制与指责他人有关的种种品类诤论之类的\textbf{言论},好似欲斗者迎着军队前行。我们\textbf{不赞叹这些}低劣之慧。什么原因?\textbf{执著从此系缚他们},因为争论之执著从此言路出发,系缚、紧跟像他们这样的人。什么原因执著?\textbf{因为他们把心放逐到远处},因为当他们反对时,便将心放逐,远离于奢摩他、毗婆舍那所到之处。\end{enumerate}

\subsection\*{\textbf{394} {\footnotesize 〔PTS 391〕}}

\textbf{「团食、寺庙、坐卧具与洗除僧伽梨尘土的水,\\}
\textbf{「听闻了善逝开示的法,胜慧的弟子经省思而受用。}

Piṇḍaṃ vihāraṃ sayanāsanañ ca, āpañ ca saṅghāṭirajūpavāhanaṃ;\\
sutvāna dhammaṃ Sugatena desitaṃ, saṅkhāya seve varapaññasāvako. %\hfill\textcolor{gray}{\footnotesize 16}

\begin{enumerate}\item 如是显示了小慧者的转起,现在,为显示大慧者的转起,说了此颂。这里,\textbf{寺庙}即住处,\textbf{坐卧具}即床椅,以三词唯说坐卧处。\textbf{洗除僧伽梨尘土},即清洗僧伽梨不净的尘垢等。\textbf{听闻了善逝开示的法},即听闻了世尊在「一切漏的防护\footnote{即\textbf{中部}第 2 经。}」等中,以\begin{quoting}经如理省思而受用衣服:仅为防寒……\end{quoting}等方法所开示的法,\textbf{胜慧的弟子经省思而受用}。如是,此处以团食指食物,以寺庙等指坐卧处,以水为目显示疾病的资具,以僧伽梨显示衣服,省思这四种资具,即胜慧的弟子以\begin{quoting}仅为此身之住立……\end{quoting}等方法省察已而受用,胜慧如来的弟子——有学,或凡夫,或径指阿罗汉——堪能受用。因为他即所说的「四依」:\begin{quoting}有经省思受用者,有经省思忍受者,有经省思回避者,有经省思除去者。(中部·那罗迦波那经第 168 段)\end{quoting}\end{enumerate}

\subsection\*{\textbf{395} {\footnotesize 〔PTS 392〕}}

\textbf{「所以,团食、坐卧具与洗除僧伽梨尘土的水,\\}
\textbf{「对于这些法不染的比丘,好比莲花上的水珠。}

Tasmā hi piṇḍe sayanāsane ca, āpe ca saṅghāṭirajūpavāhane;\\
etesu dhammesu anūpalitto, bhikkhu yathā pokkhare vāribindu. %\hfill\textcolor{gray}{\footnotesize 17}

\begin{enumerate}\item 且胜慧的弟子经省思此而受用,\textbf{所以,团食……好比莲花上的水珠},当知如是。\end{enumerate}

\subsection\*{\textbf{396} {\footnotesize 〔PTS 393〕}}

\textbf{「我对你们说在家的义务,如是行事的弟子才是善的,\\}
\textbf{「因为这全部的比丘法不能为有资产者触及。}

Gahaṭṭhavattaṃ pana vo vadāmi, yathākaro sāvako sādhu hoti;\\
na h’esa labbhā sapariggahena, phassetuṃ yo kevalo bhikkhudhammo. %\hfill\textcolor{gray}{\footnotesize 18}

\begin{enumerate}\item 如是,为显示漏尽的行道,以阿罗汉为顶点完成了出家的行道,现在,为显示在家的行道,先说了此颂。这里,先就初颂中的\textbf{弟子},即在家弟子。余皆自明。
\item 而其连结为:此前由我所说的全部、未混、完整、圆满的比丘法,它不能为有田地、物品等资产者触及,不能证得。\end{enumerate}

\subsection\*{\textbf{397} {\footnotesize 〔PTS 394〕}}

\textbf{「不应伤害生命,不应教人伤害,不应认可其他伤害者,\\}
\textbf{「对一切生物放下了棍杖,凡存在于世间的强者和弱者。}

Pāṇaṃ na hane na ca ghātayeyya, na cānujaññā hanataṃ paresaṃ;\\
sabbesu bhūtesu nidhāya daṇḍaṃ, ye thāvarā ye ca tasā santi loke. %\hfill\textcolor{gray}{\footnotesize 19}

\begin{enumerate}\item 如是在遮止此比丘法后,唯为显示在家法,说了此颂。这里,以前半说三边\footnote{三边 \textit{tikoṭi}:即不应自作、不应教作、不应认可。}遍净的戒离杀生,以后半说利益有情的行道。且此中的第三句已在犀牛角经(第 35 颂)、第四句中的强弱等类已在慈经第 146 颂从一切品类给予解释。其余之义自明。
\item 而其连结当按逆序:对弱强的一切生物放下了棍杖,不应伤害,不应教人伤害,不应认可。或者,在「放下了棍杖」后,应引入补充的文本「应转起」,否则后者无法与前者相连。\end{enumerate}

\subsection\*{\textbf{398} {\footnotesize 〔PTS 395〕}}

\textbf{「此后,有觉知的弟子应避免任何场所的任何未给予物,\\}
\textbf{「不应教人取,也不应认可取者,能避免一切未给予物。}

Tato adinnaṃ parivajjayeyya, kiñci kvaci sāvako bujjhamāno;\\
na hāraye harataṃ nānujaññā, sabbaṃ adinnaṃ parivajjayeyya. %\hfill\textcolor{gray}{\footnotesize 20}

\begin{enumerate}\item 如是,在显示第一学处后,现在为显示第二学处,说了此颂。这里,\textbf{任何},即或少或多。\textbf{任何场所},即村落或林野。\textbf{弟子},即在家弟子。\textbf{有觉知},即知晓「这是他人的财产」。\textbf{能避免一切未给予物},即显示如是行道者能避免一切未给予物,而非其它。此中其余已述且自明。\end{enumerate}

\subsection\*{\textbf{399} {\footnotesize 〔PTS 396〕}}

\textbf{「智者应避免非梵行,如避免燃烧的火坑,\\}
\textbf{「而不能行梵行者,不应侵犯他人的妻妾。}

Abrahmacariyaṃ parivajjayeyya, aṅgārakāsuṃ jalitaṃ va viññū;\\
asambhuṇanto pana brahmacariyaṃ, parassa dāraṃ na atikkameyya. %\hfill\textcolor{gray}{\footnotesize 21}

\begin{enumerate}\item 如是,在显示第二学处的三边遍净后,为从高贵的部分起\footnote{从高贵的部分起:即先说避免非梵行,再说避免欲邪行。}显示第三,说了此颂。\end{enumerate}

\subsection\*{\textbf{400} {\footnotesize 〔PTS 397〕}}

\textbf{「若有人去往会堂,或去往集会,不应对人妄语,\\}
\textbf{「不应教人说,不应认可说者,能避免一切非实。}

Sabhaggato vā parisaggato vā, ekassa v’eko na musā bhaṇeyya;\\
na bhāṇaye bhaṇataṃ nānujaññā, sabbaṃ abhūtaṃ parivajjayeyya. %\hfill\textcolor{gray}{\footnotesize 22}

\begin{enumerate}\item 现在,为显示第四学处,说了此颂。这里,\textbf{去往会堂},即去往议事厅等。\textbf{去往集会},即去往团体中。此中其余已述且自明。\end{enumerate}

\subsection\*{\textbf{401} {\footnotesize 〔PTS 398〕}}

\textbf{「若在家人乐于此法,便不应饮用麻醉与饮品,\\}
\textbf{「了知到这『导致疯狂』,不应教人饮,不应认可饮者。}

Majjañ ca pānaṃ na samācareyya, dhammaṃ imaṃ rocaye yo gahaṭṭho;\\
na pāyaye pivataṃ nānujaññā, ‘ummādanantaṃ’ iti naṃ viditvā. %\hfill\textcolor{gray}{\footnotesize 23}

\begin{enumerate}\item 如是,在显示第四学处的三边遍净后,为显示第五,说了此颂。这里,\textbf{麻醉与饮品}是为了易于结颂而如是说,其义即「不应饮用麻醉品」。\textbf{此法},即此戒离麻醉品之法。\textbf{导致疯狂},即以疯狂为终了。因为麻醉品的最轻异熟,即导致人的疯狂。此中其余已述且自明。\end{enumerate}

\subsection\*{\textbf{402} {\footnotesize 〔PTS 399〕}}

\textbf{「因为愚人们因迷醉而作恶,还教其他放逸的人也作,\\}
\textbf{「应回避这致人疯狂、愚痴、愚人爱悦的非福处。}

Madā hi pāpāni karonti bālā, kārenti c’aññe pi jane pamatte;\\
etaṃ apuññāyatanaṃ vivajjaye, ummādanaṃ mohanaṃ bālakantaṃ. %\hfill\textcolor{gray}{\footnotesize 24}

\begin{enumerate}\item 如是,在显示第五学处的三边遍净后,现在,先显示前几学处中唯麻醉品造成杂染与破坏,为敦促更努力地戒离,说了此颂。这里,\textbf{因迷醉},迷醉之因。\textbf{因为} \textit{hi},即补足语句的不变词。\textbf{作恶},即作杀生等所有不善。\textbf{致人疯狂、愚痴},即在他世疯狂,此世愚痴。其余之义自明。\end{enumerate}

\subsection\*{\textbf{403} {\footnotesize 〔PTS 400〕}}

\textbf{「不应伤害生命,不应取不予物,不应妄语,不应饮麻醉,\\}
\textbf{「应戒离非梵行、淫欲,不应在夜间受用非时食,}

Pāṇaṃ na hane na cādinnam ādiye, musā na bhāse na ca majjapo siyā;\\
abrahmacariyā virameyya methunā, rattiṃ na bhuñjeyya vikālabhojanaṃ. %\hfill\textcolor{gray}{\footnotesize 25}

\begin{enumerate}\item 至此,已显示了在家弟子的常戒,现在,为显示布萨支,说了以下两颂。这里,\textbf{非梵行},即作为非最胜之行。\textbf{淫欲},溺于淫欲法。\textbf{不应在夜间受用非时食},即不应在夜间受用,也不应在白天受用过时之食。\end{enumerate}

\subsection\*{\textbf{404} {\footnotesize 〔PTS 401〕}}

\textbf{「不应持花鬘,不应涂香,应睡在床、地面、卧具上,\\}
\textbf{「他们称这为八支布萨,由行到苦边的佛陀阐发。}

Mālaṃ na dhāre na ca gandham ācare, mañce chamāyaṃ va sayetha santhate;\\
etañ hi aṭṭhaṅgikam āh’uposathaṃ, Buddhena dukkhantagunā pakāsitaṃ. %\hfill\textcolor{gray}{\footnotesize 26}

\begin{enumerate}\item \textbf{不应涂香},当知此中以香亦包摄油、粉等。\textbf{床},即受许可的床。\textbf{卧具},即以草垫等受许可的敷物所敷者。然而,为长毛毯等所敷\textbf{地面}亦适合。\textbf{八支},即如五支的乐器,不缺一支。\textbf{行到苦边},即行到流转之苦的边际。其余于此自明。他们也说后半为结集者所说。\end{enumerate}

\subsection\*{\textbf{405} {\footnotesize 〔PTS 402〕}}

\textbf{「此后,遵守布萨,在半月的十四、十五、八日,\\}
\textbf{「与神变半月,意怀净喜,八支具足而完整。}

Tato ca pakkhass’upavass’uposathaṃ, cātuddasiṃ pañcadasiñ ca aṭṭhamiṃ;\\
pāṭihāriyapakkhañ ca pasannamānaso, aṭṭhaṅgupetaṃ susamattarūpaṃ. %\hfill\textcolor{gray}{\footnotesize 27}

\begin{enumerate}\item 如是显示了布萨支,现在,为显示布萨之时,说了此颂。这里,\textbf{此后},即补足语句的不变词。\textbf{遵守布萨,在半月的},当如是与后句连结:在半月的\textbf{十四、十五、八日},在这三天遵守布萨,度过、住于这八支布萨。\textbf{与神变半月},此中,雨季开始前分的阿沙陀月\footnote{阿沙陀月:即现在的六~七月。}、雨季内的三个月、迦底迦月\footnote{迦底迦月:即现在的十~十一月。}等这五个月被称为神变半月。另有人说,仅指阿沙陀、迦底迦、颇求那\footnote{颇求那月:即现在的二~三月。}等三个月。另有人说,为半月布萨日的前后日,即每半月被称为第十三、第一、第七、第九的各四天。可随所好而取。而对欲求福德者,一切皆可说。当如是连接:在此神变半月,\textbf{意怀净喜},\textbf{完整}、圆满、不舍一日地遵守\textbf{八支具足}的布萨。\end{enumerate}

\subsection\*{\textbf{406} {\footnotesize 〔PTS 403〕}}

\textbf{「此后,早晨,遵守布萨者应以饮食对比丘僧团\\}
\textbf{「如其所应地分享,智者心怀净喜而随喜。}

Tato ca pāto upavutthuposatho, annena pānena ca bhikkhusaṅghaṃ;\\
pasannacitto anumodamāno, yathārahaṃ saṃvibhajetha viññū. %\hfill\textcolor{gray}{\footnotesize 28}

\begin{enumerate}\item 如是显示了布萨之时,现在,为显示在这些时日遵守布萨所当作者,说了此颂。且此中,\textbf{此后},即补足语句的不变词,或为无间之义,即是说「然后」。\textbf{早晨},即翌日的前分。\textbf{食},即粥食等。\textbf{饮},即八种饮品\footnote{八种饮品:据菩提比丘注 1309,即芒果汁、香蕉汁、蜂蜜汁、葡萄汁、藕汁,等等,见\textbf{律藏}。}。\textbf{随喜},即总是喜悦之义。\textbf{如其所应},即随适自己,即是说如其能力。\textbf{分享},即分配、侍奉。余皆自明。\end{enumerate}

\subsection\*{\textbf{407} {\footnotesize 〔PTS 404〕}}

\textbf{「他应如法赡养父母,应从事合法的贸易,\\}
\textbf{「这在家人坚持、不放逸,得至名为自身光芒的天人。」}

Dhammena mātāpitaro bhareyya, payojaye dhammikaṃ so vaṇijjaṃ;\\
etaṃ gihī vattayam appamatto, Sayampabhe nāma upeti deve” ti. %\hfill\textcolor{gray}{\footnotesize 29}

\begin{enumerate}\item 如是,说了遵守布萨者的义务,现在,先谈了尽寿命的重务及活命遍净,为显示以此行道可证之处,说了此颂。这里,\textbf{如法},即以如法所得之财。\textbf{赡养},即养育。\textbf{合法的贸易},即避开贩卖人口、武器、毒品、肉、酒这五种非法贸易,其余的合法贸易。且此中,以贸易摄耕田、牧牛等别的合法职业。其余之义自明。
\item 而此连结为:这具足常戒、布萨戒、布施等法的圣弟子,应从事合法的贸易,随后,应以由不离于法而为如法所得之财赡养父母。然后,这\textbf{在家人}如是\textbf{不放逸},\textbf{坚持}着从开始所说的义务,身坏后,\textbf{得至}、追随、紧跟\textbf{名为自身光芒的天人}——凡以自己的光芒驱散黑暗,以发光故得名「自身光芒」的六欲界天——转生于彼等转生之处。\end{enumerate}

\begin{center}\vspace{5em}如法经第十四\\Dhammikasuttaṃ cuddasamaṃ.\end{center}

\textbf{其总颂曰:}

\begin{quoting}宝、生腥与惭,吉祥、针毛,\\法行、婆罗门、船、何戒与起身,\\还有罗睺罗、劫波以及游行,\\与如法,知者以「小品」称十四。\end{quoting}

Tass’uddānaṃ —

\begin{quoting}Ratanāmagandho Hiri ca, Maṅgalaṃ Sūcilomena;\\Dhammacariyañ ca Brāhmaṇo, Nāvā Kiṃsīlam Uṭṭhānaṃ;\\Rāhulo puna Kappo ca, Paribbājaniyaṃ tathā;\\Dhammikañ ca viduno āhu, Cūḷavaggan ti cuddasā ti.\end{quoting}

\begin{center}\vspace{1em}小品第二\\Cūḷavaggo dutiyo.\end{center}

%\begin{flushright}乙巳小暑二稿\end{flushright}