\section{清净八颂经}

\begin{center}Suddhaṭṭhaka Sutta\end{center}\vspace{1em}

\begin{enumerate}\item 据说,在过去迦叶世尊时,住在波罗奈的某个地主为采购货物,以五百车去往边境。在那里与守林人交了朋友,并给了他礼物,问到「朋友!你之前是否见过旃檀髓」,在得到答复「是的,大人」后,便与他一起进入了旃檀林,把所有车子都装满了旃檀髓,并对守林人说「朋友!你要来波罗奈时,请拿些旃檀髓再来」,便回了波罗奈。后来,这守林人拿了旃檀髓去了他家。他见到后,尽了一切款待,哺时,命人碾碎旃檀髓,装在盒子里,说「去,朋友!洗了澡再来」,派他的仆人和他一起去了浴场。那时正值波罗奈的节日。于是,波罗奈的居民在早上布施后,晚上穿了干净的衣服,拿了花鬘、芳香,去礼拜迦叶世尊的大支提。守林人看见他们后,问「大众去哪里」,听到「去寺庙礼拜支提」后,自己也去了。在那里,见到人们以雄黄、红砷等种种方式供养支提,不懂怎样画画,他取了旃檀,在大支提的金砖上作了铜钵大小的曼陀罗。于是,在日出时分,太阳的光芒出现。他看见后,净喜并发愿「凡我出生之处,愿这样的光芒现于我胸」。他死后投生于三十三天,光芒便现于其胸,他的胸轮如月轮一般照耀,于是他们便称呼他「月光天子」。
\item 他以此成就在六天界顺逆地度过了一佛的间隔,当我们的世尊出世时,投生在舍卫国富裕的婆罗门家中,而其胸前也同样有月轮般的光轮。在命名日,众婆罗门为其祈福,看见这曼陀罗后,惊异道「这童子有幸运福德之相」,便取名「月光」。他成年后,众婆罗门便带走并装扮他,令穿上红色的披风、登上车,供养他「这是大梵天」,宣告说「谁看见月光,谁就能得到名誉、财富等,且来世上升天界」,在村、镇、王畿巡游。所到之处,越来越多的人们前来「据说这名叫月光的先生,谁看见他,谁就能得到名誉、财富、天界」,整个瞻部洲震动。众婆罗门不向空手而来的人们显示(月光),仅在收取了百千(资财)后向来者显示。如是,众婆罗门带着月光巡游,逐步来到舍卫国。
\item 那时,世尊已转起无上的法轮,渐次来到舍卫国,住在舍卫国祇树给孤独园,为众人的利益而开示法。当月光到了舍卫国,他如小河入海般不彰,也没人提起「月光」。他在哺时看见大众拿着花鬘、芳香朝着祇园走去,便问「你们去哪里」,听到他们说「佛陀出现世间,他为了众人的利益而开示法,我们去祇园听法」,他也被婆罗门众簇拥着去往那里。那时,世尊正坐在法堂的无上佛座上。月光走近世尊后,善语慰问,坐在一边,而他的光芒瞬间消失。因为在佛光附近八十手的范围内,别的光芒无法闪耀。他坐下后,想「我的光芒消失了」,便起身,起身后准备离开。于是有人对他说「月光君!你是害怕沙门乔达摩而离开吗」。「我不是害怕而离开,而是因其威力,我的光不能成就」,再次在世尊前坐下后,看了从脚掌直到发尖的形相、光芒和(大人)相的成就,极度净喜道「沙门乔达摩有大威力,我胸前生起的光芒其量甚微,众婆罗门尚且带着我巡游了整个瞻部洲,如是具足无上之相成就的沙门乔达摩却毫无慢的生起,确实,这具足无劣之德者能成为人天之师」,礼拜了世尊,请求出家。世尊便命某位长老「度他出家」。他度其出家后,说了皮五法的业处。他即开始修观,不久即证阿罗汉,而以「月光长老」知名。关于他,诸比丘发起讨论「朋友!是否谁看见月光,谁就能得名誉、财富,或上升天界,或以见眼门之色而圆满清净」,在此缘由下,世尊说了此经。\end{enumerate}

\subsection\*{\textbf{795} {\footnotesize 〔PTS 788〕}}

\textbf{「我看见清净、最上的无病,人以所见而清净」,\\}
\textbf{如是证知,了知了最上,「随观清净」,他认可智。}

“Passāmi suddhaṃ paramaṃ arogaṃ, diṭṭhena saṃsuddhi narassa hoti”;\\
evābhijānaṃ “paraman” ti ñatvā, suddhānupassī ti pacceti ñāṇaṃ. %\hfill\textcolor{gray}{\footnotesize 1}

\begin{enumerate}\item 这里,第一颂的意思是:诸比丘!不是以这样的所见而清净,而且,具见的愚人看到了由烦恼垢秽而不清净、由不离烦恼之病而有病的月光婆罗门或其他这样的人后,证知「\textbf{我看见清净、最上的无病},且\textbf{人以所见而清净}」,他\textbf{如是证知},\textbf{了知了}这所见为\textbf{最上},于此所见\textbf{随观清净},\textbf{他认可}此所见为道\textbf{智}。\end{enumerate}

\subsection\*{\textbf{796} {\footnotesize 〔PTS 789〕}}

\textbf{若人以所见而清净,或他以智而舍弃苦,\\}
\textbf{则此有所依者被其它净化,如是语时,见显露他。}

Diṭṭhena ce suddhi narassa hoti, ñāṇena vā so pajahāti dukkhaṃ;\\
aññena so sujjhati sopadhīko, diṭṭhī hi naṃ pāva tathā vadānaṃ. %\hfill\textcolor{gray}{\footnotesize 2}

\begin{enumerate}\item 但这并非道智,故说第二颂。其意为:\textbf{若人以}称为所见之色的\textbf{所见而}得烦恼的\textbf{清净},\textbf{或}若\textbf{他以}此\textbf{智而舍弃}生等之\textbf{苦},在如是情况下,他\textbf{被}除圣道外的\textbf{其它}非清净之道\textbf{净化},由贪等所依而为\textbf{有所依者},但如是等人未被净化,所以\textbf{如是语时,见显露他},当他以「世间是常」等方式如是言说时,正是此见说明这随顺于见者:「他是邪见者」。\end{enumerate}

\subsection\*{\textbf{797} {\footnotesize 〔PTS 790〕}}

\textbf{婆罗门不由其它而于所见、所闻、戒禁或所觉说是清净,\\}
\textbf{不染于福与恶,舍弃了所得,不于此造作。}

Na brāhmaṇo aññato suddhim āha, diṭṭhe sute sīlavate mute vā;\\
puññe ca pāpe ca anūpalitto, attañjaho na-y-idha pakubbamāno. %\hfill\textcolor{gray}{\footnotesize 3}

\begin{enumerate}\item 由排除了恶而为\textbf{婆罗门},这以道得证漏尽的漏尽婆罗门,\textbf{不由}除圣道智的\textbf{其它}生起的邪智\textbf{而于}被称为公认祥瑞之色的\textbf{所见}、被称为如是种类之声的\textbf{所闻}、被称为不违犯的\textbf{戒}、象禁等类的\textbf{禁}及地等类的\textbf{所觉说是清净}。其它(后半颂)是为了赞扬这婆罗门而说。因为他\textbf{不染于}三界的\textbf{福与}一切\textbf{恶},由舍弃了我见或任何执取而\textbf{舍弃了所得},不作福行等而\textbf{不于此造作},所以如是赞叹他。\end{enumerate}

\subsection\*{\textbf{798} {\footnotesize 〔PTS 791〕}}

\textbf{舍弃了前者,系缚后者,溺于贪扰,他们不能度脱染著,\\}
\textbf{他们执取、放弃,如猿猴放开又捉取枝条。}

Purimaṃ pahāya aparaṃ sitāse, ejānugā te na taranti saṅgaṃ;\\
te uggahāyanti nirassajanti, kapīva sākhaṃ pamuñcaṃ gahāyaṃ. %\hfill\textcolor{gray}{\footnotesize 4}

\subsection\*{\textbf{799} {\footnotesize 〔PTS 792〕}}

\textbf{人自己受持了禁戒,随处而往,执著于想,\\}
\textbf{而智者以明体认了法,宏慧者不随处而往。}

Sayaṃ samādāya vatāni jantu, uccāvacaṃ gacchati saññasatto;\\
vidvā ca vedehi samecca dhammaṃ, na uccāvacaṃ gacchati bhūripañño. %\hfill\textcolor{gray}{\footnotesize 5}

\begin{enumerate}\item \textbf{禁戒},即象禁等。\textbf{随处},即往返、从贱至贵,或从(此)师至(彼)师。\textbf{执著于想},即执著于欲想等。\textbf{而智者以明体认了法},即知第一义的阿罗汉以四道智之明证得了四谛法。\end{enumerate}

\subsection\*{\textbf{800} {\footnotesize 〔PTS 793〕}}

\textbf{他破除了一切法,或任何所见、所闻、所觉,\\}
\textbf{这具见、坦荡而行者,于此世间,谁与同类?}

Sa sabbadhammesu visenibhūto, yaṃ kiñci diṭṭhaṃ va sutaṃ mutaṃ vā;\\
tam eva dassiṃ vivaṭaṃ carantaṃ, kenīdha lokasmi vikappayeyya. %\hfill\textcolor{gray}{\footnotesize 6}

\begin{enumerate}\item \textbf{这具见},即这如是具清净之见者。\textbf{坦荡而行者},即以离于爱等的蔽覆而坦荡前行。\end{enumerate}

\subsection\*{\textbf{801} {\footnotesize 〔PTS 794〕}}

\textbf{他们不造作,不偏好,他们不说「极度清净」,\\}
\textbf{舍离了被系缚于取著的系缚,不希求世间任何地方。}

Na kappayanti na purekkharonti, “accantasuddhī” ti na te vadanti;\\
ādānaganthaṃ gathitaṃ visajja, āsaṃ na kubbanti kuhiñci loke. %\hfill\textcolor{gray}{\footnotesize 7}

\begin{enumerate}\item 由证得第一义极度清净,\textbf{他们不说}非极度清净的无作或常见为\textbf{极度清净}。\end{enumerate}

\subsection\*{\textbf{802} {\footnotesize 〔PTS 795〕}}

\textbf{越过界限的婆罗门已知、已见,他无所摄取,\\}
\textbf{不染于贪,不染于离贪,于此他更无所执取。}

Sīmātigo brāhmaṇo tassa natthi, ñatvā va disvā va samuggahītaṃ;\\
na rāgarāgī na virāgaratto, tassīdha natthi param uggahītan ti. %\hfill\textcolor{gray}{\footnotesize 8}

\begin{center}\vspace{1em}清净八颂经第四\\Suddhaṭṭhakasuttaṃ catutthaṃ.\end{center}