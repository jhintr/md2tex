\section{清净八颂经}

\begin{center}Suddhaṭṭhaka Sutta\end{center}\vspace{1em}

\begin{enumerate}\item 缘起为何?据说,在过去迦叶世尊时,住在波罗奈的某个地主为了采购货物,以五百车去往边境。在那里与巡林人交了朋友,献给他礼物,便问到:「兄弟!你之前是否见过旃檀髓?」在得到答复「唯!大人」后,便与他一起进入了旃檀林,把所有车子都装满了旃檀髓,并对巡林人说「兄弟!你来波罗奈时,请拿些旃檀髓再来」,便回了波罗奈。后来,这巡林人便拿了旃檀髓去到他家。他见到后,尽了一切款待,哺时,命人碾碎旃檀髓,装满箱子,说「去,兄弟!洗了澡再来」,便派他的仆人和他一起去浴场。
\item 那时正值波罗奈的节日。于是,波罗奈的居民在早上布施后,晚上穿了干净的衣服,拿了花鬘、芳香等,前去礼拜迦叶世尊的大支提。这巡林人看见他们后,便问:「大众去哪里?」听闻「去寺庙礼拜支提」后,自己也去了。在那里,见到人们以雌黄、雄黄等种种品类供养支提,他不懂怎样绘画,便取了旃檀,在大支提的金砖上作了铜钵大小的曼陀罗。于是,在日出时分,太阳的光芒出现在那里。他见之净喜,并发愿:「凡我转生之处,愿这样的光芒现于我胸。」他死后转生于三十三天,光芒便现于其胸,他的胸轮如月轮一般照耀,人们便以「月光天子」称呼他。
\item 他以此成就在六天界顺逆地抛掷了一佛的间隔,当我们的世尊出世时,便转生在舍卫国的富裕婆罗门家中,而其胸前也同样有月轮般的光轮。且在命名日,众婆罗门为其祈福,看见这曼陀罗后,惊异道「这童子有幸运福德之相」,便取名「月光」。他成年后,众婆罗门便带走并装扮他,让他穿上红色的披风、登上车,以「这是大梵天」供养他,宣言道「谁看见月光,谁就能得到名誉、财产等,且来世去往天界」,在村、镇、王畿巡游。所到之处,越来越多的人们前来「据说这名叫月光的先生,谁看见他,谁就能得到名誉、财产、天界」,整个瞻部洲震动。众婆罗门不向空手而来的人们显示,仅在收取百千(钱)后向来者显示。如是,众婆罗门带着月光巡游,逐步到了舍卫国。
\item 那时,世尊已转起无上的法轮,渐次到了舍卫国,住舍卫国祇树给孤独园,为众人的利益开示法。当月光到达舍卫国,他如小河入海般不彰,也没人说起「月光」。他在哺时看见大众拿着花鬘、芳香朝着祇园走去,便问:「你们去哪里?」听到他们说「佛陀出现世间,他为众人的利益开示法,我们去祇园听法」,他也被婆罗门众簇拥着去到那里。那时,世尊正坐在法堂的最上佛座上。月光走近世尊后,善语慰问,坐在一边,而他的光芒瞬间消失。因为在佛光附近八十肘的范围内,别的光芒无法克服。他刚坐下,想「我的光芒消失了」便起身,且起身后准备离开。
\item 于是,有人对他说:「月光君!你是害怕沙门乔达摩而离开吗?」「我不是害怕而离开,而是因其威力,我的光不能成就。」再次在世尊前坐下后,看了从脚掌直到发尖的形相、光芒和(大人)相等的成就,极度净喜道「沙门乔达摩有大威德,我胸前生起的光芒其量甚微,众婆罗门尚且带着我巡游了整个瞻部洲,如是具足高贵之相成就的沙门乔达摩却毫无慢的生起,确实,这具足最上之德者能成为人天之师」,便礼拜世尊,请求出家。
\item 世尊便命某位长老「度他出家」。他度其出家后,告知了皮五法的业处。他即开始修观,不久即证阿罗汉,而以「月光长老」知名。比丘们就他发起谈论:「朋友!是否谁看见月光,谁就能得名誉、财产,或去往天界,或因见此眼门之色就能圆满清净?」世尊便在此事由下,说了此经。\end{enumerate}

\subsection\*{\textbf{795} {\footnotesize 〔PTS 788〕}}

\textbf{「我看见清净、最上的无病,人以所见而清净」,\\}
\textbf{如是证知,了知了最上,以随观清净,他认可智。}

“Passāmi suddhaṃ paramaṃ arogaṃ, diṭṭhena saṃsuddhi narassa hoti”;\\
evābhijānaṃ “paraman” ti ñatvā, suddhānupassī ti pacceti ñāṇaṃ. %\hfill\textcolor{gray}{\footnotesize 1}

\begin{enumerate}\item 这里,先说第一颂之义为:诸比丘!不是以这样的所见而清净,而是看到由烦恼垢秽而不清净、由不离烦恼之病而有病的月光婆罗门或其他这样的人后,具见的愚人证知「\textbf{我看见清净、最上的无病},且\textbf{人以}此被称为\textbf{所见}的知见\textbf{而清净}」,他\textbf{如是证知},\textbf{了知了}这知见为\textbf{最上},当于此知见\textbf{随观清净}时,\textbf{他认可}此知见为道\textbf{智}。\end{enumerate}

\subsection\*{\textbf{796} {\footnotesize 〔PTS 789〕}}

\textbf{若人以所见而清净,或他以智舍弃苦,\\}
\textbf{则此有依持者被其它净化,如是语时,见显露他。}

Diṭṭhena ce suddhi narassa hoti, ñāṇena vā so pajahāti dukkhaṃ;\\
aññena so sujjhati sopadhīko, diṭṭhī hi naṃ pāva tathā vadānaṃ. %\hfill\textcolor{gray}{\footnotesize 2}

\begin{enumerate}\item 但这并非道智,因此说了第二颂。其义为:\textbf{若人以}被称为见色的\textbf{所见而}得烦恼的\textbf{清净},\textbf{或}若\textbf{他以}此\textbf{智舍弃}生等之\textbf{苦},当如是时,他只是\textbf{被}除圣道外的\textbf{其它}非清净之道\textbf{净化},由贪等依持,仍作为\textbf{有依持者}被净化,但这样的人未被净化,所以\textbf{如是语时,见显露他},当他以「世间是常」等方式如是如是语时,正是此见说明这随顺见者:他是邪见者。\end{enumerate}

\subsection\*{\textbf{797} {\footnotesize 〔PTS 790〕}}

\textbf{婆罗门不由其它,于所见、所闻、戒禁或所觉而说清净,\\}
\textbf{不染于福与恶,舍弃了执取,不于此造作。}

Na brāhmaṇo aññato suddhim āha, diṭṭhe sute sīlavate mute vā;\\
puññe ca pāpe ca anūpalitto, attañjaho na-y-idha pakubbamāno. %\hfill\textcolor{gray}{\footnotesize 3}

\begin{enumerate}\item 第三颂。其义为:由排除了恶而为\textbf{婆罗门},这以道得证漏尽的漏尽婆罗门,\textbf{不由}除圣道智的\textbf{其它}生起的邪智,\textbf{于}被称为共许为祥瑞之色的\textbf{所见}、被称为如是种类之声的\textbf{所闻}、被称为不违犯的\textbf{戒}、象禁等类的\textbf{禁}及地等类的\textbf{所觉而说清净}。
\item 其余则是为了赞扬这婆罗门而说。因为他\textbf{不染于}三界之\textbf{福与}一切\textbf{恶},由舍弃了我见或任何执持而\textbf{舍弃了执取},由不造作福行等被称为\textbf{不于此造作},所以为如是赞叹他而说。且当知一切应与前句相连。\end{enumerate}

\subsection\*{\textbf{798} {\footnotesize 〔PTS 791〕}}

\textbf{舍弃了前者,束缚后者,跟随干扰,他们不能度脱染著,\\}
\textbf{他们捉取、扬弃,如猿猴放开又抓取枝条。}

Purimaṃ pahāya aparaṃ sitāse, ejānugā te na taranti saṅgaṃ;\\
te uggahāyanti nirassajanti, kapīva sākhaṃ pamuñcaṃ gahāyaṃ. %\hfill\textcolor{gray}{\footnotesize 4}

\begin{enumerate}\item 如是,以「不染于福与恶,舍弃了执取,不于此造作,婆罗门不由其它而说清净」说了婆罗门不由其它而说清净,现在,为显示那些由其它而说清净的持见者之不排除此见,说了此颂。
\item 其义为:因为\textbf{他们}虽然由其它而说清净,由未舍弃见故,犹有抓取和放下,因此,\textbf{舍弃了}大师等的\textbf{前者,束缚后者,跟随}、被征服于被称为\textbf{干扰}的渴爱,\textbf{不能度脱}贪等类的\textbf{染著},而未度脱彼则\textbf{捉取、扬弃}彼彼法,如\textbf{猿猴}之于\textbf{枝条}。\end{enumerate}

\subsection\*{\textbf{799} {\footnotesize 〔PTS 792〕}}

\textbf{人自己受持了禁戒,随处而往,执著于想,\\}
\textbf{而知者以明体认了法,宏慧者不随处而往。}

Sayaṃ samādāya vatāni jantu, uccāvacaṃ gacchati saññasatto;\\
vidvā ca vedehi samecca dhammaṃ, na uccāvacaṃ gacchati bhūripañño. %\hfill\textcolor{gray}{\footnotesize 5}

\begin{enumerate}\item 第五颂的连结为:凡如「如是语时,见显露他」所说者,他自己受持。这里,\textbf{禁戒},即象禁等。\textbf{随处},即反复,或从低至高、从大师到大师等。\textbf{执著于想},即固著于欲想等。\textbf{而知者以明体认了法},即知第一义的阿罗汉以四道智之明证得了四谛法。其余自明。\end{enumerate}

\subsection\*{\textbf{800} {\footnotesize 〔PTS 793〕}}

\textbf{他平定了一切法,或任何所见、所闻、所觉,\\}
\textbf{这具见、坦荡而行者,在此世间,谁与同类?}

Sa sabbadhammesu visenibhūto, yaṃ kiñci diṭṭhaṃ va sutaṃ mutaṃ vā;\\
tam eva dassiṃ vivaṭaṃ carantaṃ, kenīdha lokasmi vikappayeyya. %\hfill\textcolor{gray}{\footnotesize 6}

\begin{enumerate}\item \textbf{他平定了一切法,或任何所见、所闻、所觉},即这宏慧的漏尽者,消除了任何所见、所闻或所觉,或彼等一切法中的魔军,以住立之相平定。\textbf{这具见},即这如是具清净之见者。\textbf{坦荡而行者},即以离于爱之蔽覆等而坦荡前行者。\textbf{在此世间,谁与同类},即在此世间,与何爱想或见想者同类?或者,由舍弃彼等故,与何先前所说的贪等同类?\end{enumerate}

\subsection\*{\textbf{801} {\footnotesize 〔PTS 794〕}}

\textbf{他们不设想,不预设,他们不说「极度清净」,\\}
\textbf{舍离了系缚于取著的系缚,不对世间任何存有希望。}

Na kappayanti na purekkharonti, “accantasuddhī” ti na te vadanti;\\
ādānaganthaṃ gathitaṃ visajja, āsaṃ na kubbanti kuhiñci loke. %\hfill\textcolor{gray}{\footnotesize 7}

\begin{enumerate}\item 此颂的连结和语义为:更有什么?因为他们既然如此,不以二种设想和预设中的任一来\textbf{设想、预设},由证得第一义极度清净故,\textbf{他们不说}非极度清净的无作见和常见为\textbf{极度清净}。\textbf{舍离了系缚于取著的系缚},即便由取著色等的四种,仍以圣道之剑舍离、切断了系缚、捆缚于自身心相续中取著的系缚。其余自明。\end{enumerate}

\subsection\*{\textbf{802} {\footnotesize 〔PTS 795〕}}

\textbf{婆罗门越过界限,已知、已见,他无所摄取,\\}
\textbf{不染于贪,不染于离贪,他于此不执取更多。}

Sīmātigo brāhmaṇo tassa natthi, ñatvā va disvā va samuggahītaṃ;\\
na rāgarāgī na virāgaratto, tassīdha natthi param uggahītan ti. %\hfill\textcolor{gray}{\footnotesize 8}

\begin{enumerate}\item 此颂是以基于一人的开示而说。而其连结仍与先前相同,当如是知此(连结)与其释义:更有什么?他这样的宏慧者由越过四种烦恼界限而\textbf{越过界限},由排除了恶而为\textbf{婆罗门},且以他心、宿住智\textbf{已知},以肉眼、天眼\textbf{已见},如\textbf{他}这样的对任何都\textbf{无所摄取},即是说无所执著。且他由无有欲贪\textbf{不染于贪},由无有色、无色贪\textbf{不染于离贪},由此,如此等者\textbf{于此不}以「此是\textbf{更多}」\textbf{执取}任何,即以阿罗汉为顶点完成了开示。\end{enumerate}

\begin{center}\vspace{1em}清净八颂经第四\\Suddhaṭṭhakasuttaṃ catutthaṃ.\end{center}