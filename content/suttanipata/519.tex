\section{诵彼岸道颂}

\begin{center}Pārāyanānugītigāthā\end{center}\vspace{1em}

\begin{enumerate}\item 当世尊开示「彼岸道」时,一万六千萦发者证得了阿罗汉,其余十四俱胝之数的天人得了法的现观。这便如古人所说:\begin{quoting}随后,在惬意的石(支提),在彼岸道的集会中,\\佛陀令十四俱胝的生命得至不死。\end{quoting}当开示终了,来集的人们由世尊的威力回到了各自的村镇。世尊也为随侍的十六(婆罗门)等数千比丘所围绕,回到了舍卫国。于此,褐者礼拜了世尊后说:「尊者!我前去告诉波婆利佛陀的出世,因为我已应允他。」在得到世尊的许可后,以智行去到乔陀婆利岸边,以步行朝草庵走去。
\item 婆罗门波婆利坐着观察道路,远远看到他除去了篮子、萦发等,以比丘的衣装而来,便得出结论「佛陀出世」。到达后便问他:「褐者!佛陀是否出世?」「唯!婆罗门!已经出世,坐于石支提,对我们开示了法,我将对你说。」随后,波婆利以极大的恭敬,与随众供养了他,备好坐具。褐者于此落坐后,便说了下文。\end{enumerate}

\subsection\*{\textbf{1138} {\footnotesize 〔PTS 1131〕}}

\textbf{「我将诵出彼岸道,」尊者褐者说,\\}
\textbf{「无垢的大慧者如其所见,即如是演说,\\}
\textbf{「无欲、消尽的龙象,为何会妄语?}

“Pārāyanam anugāyissaṃ, \textit{(icc āyasmā Piṅgiyo)}\\
yathāddakkhi tathākkhāsi, vimalo bhūrimedhaso;\\
nikkāmo nibbano nāgo, kissa hetu musā bhaṇe. %\hfill\textcolor{gray}{\footnotesize 1}

\begin{enumerate}\item \textbf{如其所见},即如自身以对(圣)谛的等正觉、以不共知之所见。\textbf{消尽},即无烦恼的丛林,或无渴爱。\textbf{为何会妄语},会以何烦恼而妄语,即显明其已舍弃了彼等。以此令婆罗门对教法产生热情。\end{enumerate}

\subsection\*{\textbf{1139} {\footnotesize 〔PTS 1132〕}}

\textbf{「舍弃了尘垢与愚痴者、舍弃了慢与覆藏者的\\}
\textbf{「具有德泽的话语,噫!我将宣扬!}

Pahīnamala-mohassa, māna-makkha-ppahāyino;\\
handāhaṃ kittayissāmi, giraṃ vaṇṇūpasañhitaṃ. %\hfill\textcolor{gray}{\footnotesize 2}

\begin{enumerate}\item \textbf{具有德泽},即具有功德。\end{enumerate}

\subsection\*{\textbf{1140} {\footnotesize 〔PTS 1133〕}}

\textbf{「除去暗冥的佛陀具一切眼,已到世间的边际,超越一切有,\\}
\textbf{「无漏,舍弃了一切苦,以真实得名者,梵天!为我所侍奉。}

Tamonudo Buddho samantacakkhu, lokantagū sabbabhavātivatto;\\
anāsavo sabbadukkhappahīno, saccavhayo Brahme upāsito me. %\hfill\textcolor{gray}{\footnotesize 3}

\begin{enumerate}\item \textbf{以真实得名者},即「佛陀」与以真实所得之名号相称。\textbf{梵天},即称呼此婆罗门(波婆利)。\end{enumerate}

\subsection\*{\textbf{1141} {\footnotesize 〔PTS 1134〕}}

\textbf{「好比鸟儿舍弃了灌木丛,飞入果实丰硕的森林,\\}
\textbf{「如是,我舍弃了短视者,如同天鹅到达了大湖。}

Dijo yathā kubbanakaṃ pahāya, bahupphalaṃ kānanam āvaseyya;\\
evam p’ahaṃ appadasse pahāya, mahodadhiṃ haṃso-r-iva ajjhapatto. %\hfill\textcolor{gray}{\footnotesize 4}

\begin{enumerate}\item \textbf{灌木丛},即小树林。\textbf{短视者},即波婆利以降的小慧者。\textbf{大湖},即阿耨达等大水聚。\end{enumerate}

\subsection\*{\textbf{1142} {\footnotesize 〔PTS 1135〕}}

\textbf{「在乔达摩的教法之前,他们先前所说的这些,\\}
\textbf{「『这曾是如此、这将是如此』,这一切都是传闻,\\}
\textbf{「这一切都是寻的增长。}

Ye ’me pubbe viyākaṃsu, huraṃ Gotamasāsanā;\\
icc āsi iti bhavissati, sabbaṃ taṃ itihītihaṃ;\\
sabbaṃ taṃ takkavaḍḍhanaṃ. %\hfill\textcolor{gray}{\footnotesize 5}

\begin{itemize}\item 案,此颂与黄金学童问第 1091 颂大致相同。\end{itemize}

\subsection\*{\textbf{1143} {\footnotesize 〔PTS 1136〕}}

\textbf{「除去暗冥者独自而坐,他放光,带来光芒,\\}
\textbf{「宏慧的乔达摩,大慧的乔达摩,}

Eko tamanud’āsino, jutimā so pabhaṅkaro;\\
Gotamo bhūripaññāṇo, Gotamo bhūrimedhaso. %\hfill\textcolor{gray}{\footnotesize 6}

\begin{enumerate}\item \textbf{宏慧},即智的幢幡 \textit{ñāṇadhajo}。\end{enumerate}

\begin{itemize}\item 案,\textbf{paññāṇo},二英译均作「智慧」解,但义注作「幢幡」解,考虑上下文以及该词在可教学童问第 1097~1098 颂中的意思,兹从英译。\end{itemize}

\subsection\*{\textbf{1144} {\footnotesize 〔PTS 1137〕}}

\textbf{「他对我开示的法,自见、无时、\\}
\textbf{「爱尽、无患,无处有其雷同者。」}

Yo me dhammam adesesi, sandiṭṭhikam akālikaṃ;\\
taṇhakkhayam anītikaṃ, yassa natthi upamā kvaci”. %\hfill\textcolor{gray}{\footnotesize 7}

\begin{enumerate}\item \textbf{自见、无时},即其果亲自可见,且无需间隔即可证。\textbf{无患},即无烦恼之患。\end{enumerate}

\subsection\*{\textbf{1145} {\footnotesize 〔PTS 1138〕}}

\textbf{「褐者!你是否须臾间离开过这\\}
\textbf{「宏慧的乔达摩,大慧的乔达摩,}

“Kiṃ nu tamhā vippavasasi, muhuttam api Piṅgiya;\\
Gotamā bhūripaññāṇā, Gotamā bhūrimedhasā. %\hfill\textcolor{gray}{\footnotesize 8}

\subsection\*{\textbf{1146} {\footnotesize 〔PTS 1139〕}}

\textbf{「他对你开示的法,自见、无时、\\}
\textbf{「爱尽、无患,无处有其雷同者。」}

Yo te dhammam adesesi, sandiṭṭhikam akālikaṃ;\\
taṇhakkhayam anītikaṃ, yassa natthi upamā kvaci”. %\hfill\textcolor{gray}{\footnotesize 9}

\subsection\*{\textbf{1147} {\footnotesize 〔PTS 1140〕}}

\textbf{「婆罗门!我未须臾间离开过这\\}
\textbf{「宏慧的乔达摩,大慧的乔达摩,}

“Nāhaṃ tamhā vippavasāmi, muhuttam api Brāhmaṇa;\\
Gotamā bhūripaññāṇā, Gotamā bhūrimedhasā. %\hfill\textcolor{gray}{\footnotesize 10}

\subsection\*{\textbf{1148} {\footnotesize 〔PTS 1141〕}}

\textbf{「他对我开示的法,自见、无时、\\}
\textbf{「爱尽、无患,无处有其雷同者。}

Yo me dhammam adesesi, sandiṭṭhikam akālikaṃ;\\
taṇhakkhayam anītikaṃ, yassa natthi upamā kvaci. %\hfill\textcolor{gray}{\footnotesize 11}

\subsection\*{\textbf{1149} {\footnotesize 〔PTS 1142〕}}

\textbf{「我用意看见他,如同用眼,日夜不放逸,婆罗门!\\}
\textbf{「我以礼敬度夜,因此,我认为未曾离开。}

Passāmi naṃ manasā cakkhunā va, rattindivaṃ Brāhmaṇa appamatto;\\
namassamāno vivasemi rattiṃ, ten’eva maññāmi avippavāsaṃ. %\hfill\textcolor{gray}{\footnotesize 12}

\subsection\*{\textbf{1150} {\footnotesize 〔PTS 1143〕}}

\textbf{「我的信、喜、意、念都不曾离开乔达摩的教法,\\}
\textbf{「无论宏慧者去向何方,我都向之倾身。}

Saddhā ca pīti ca mano sati ca, nāpenti me Gotamasāsanamhā;\\
yaṃ yaṃ disaṃ vajati bhūripañño, sa tena ten’eva nato’ham asmi. %\hfill\textcolor{gray}{\footnotesize 13}

\subsection\*{\textbf{1151} {\footnotesize 〔PTS 1144〕}}

\textbf{「对于年迈、力弱的我,身体由此不能奔赴那里,\\}
\textbf{「我常常以思惟的游行到达,婆罗门!因为我的意与之相应。}

Jiṇṇassa me dubbala-thāmakassa, ten’eva kāyo na paleti tattha;\\
saṅkappayantāya vajāmi niccaṃ, mano hi me Brāhmaṇa tena yutto. %\hfill\textcolor{gray}{\footnotesize 14}

\begin{enumerate}\item \textbf{与之相应},即与佛陀相应。\end{enumerate}

\subsection\*{\textbf{1152} {\footnotesize 〔PTS 1145〕}}

\textbf{「在淤泥中躺着颤栗,从洲渚漂流到洲渚,\\}
\textbf{「然后,我看到了已度过暴流、无漏的等正觉。」}

Paṅke sayāno pariphandamāno, dīpā dīpaṃ upaplaviṃ;\\
ath’addasāsiṃ Sambuddhaṃ, oghatiṇṇam anāsavaṃ”. %\hfill\textcolor{gray}{\footnotesize 15}

\begin{enumerate}\item \textbf{淤泥},即爱欲的泥淖。\textbf{从洲渚漂流到洲渚},即从大师等至大师等。\end{enumerate}

\subsection\*{\textbf{1153} {\footnotesize 〔PTS 1146〕}}

\textbf{「正如婆迦利信解于信,以及善器与旷野乔达摩,\\}
\textbf{「如是,你也应信解于信!褐者!你将去到死亡境域的彼岸。」}

“Yathā ahū Vakkali muttasaddho, Bhadrāvudho Āḷavi-Gotamo ca;\\
evamevaṃ tvam pi pamuñcassu saddhaṃ, gamissasi tvaṃ Piṅgiya maccudheyyassa pāraṃ”. %\hfill\textcolor{gray}{\footnotesize 16}

\begin{enumerate}\item 在上颂的最后,世尊在得知褐者与波婆利的根已成熟后,仍立于舍卫国,放出金色的光。褐者正坐着对波婆利演说佛陀的功德,看到了这光,观察「这是什么」,看到世尊像是站在自己前面一样,便告诉婆罗门波婆利「佛陀来了」。婆罗门从坐起,合掌而立。世尊将光遍满,向婆罗门显明自己,在得知两人的适宜后,唯对褐者说了此颂。
\item 这里,\textbf{婆迦利}长老信解于信,以信之轭得证阿罗汉,如是,十六(婆罗门)中的\textbf{善器}及\textbf{旷野乔达摩}亦然。\textbf{如是,你也应信解于信},随后,从信解于信起,以「一切行无常」等方法开始修观,\textbf{你将去到死亡境域的彼岸},涅槃,即以阿罗汉为顶点完成了开示。当开示终了,褐者住于阿罗汉,波婆利住于阿那含果,而波婆利婆罗门的五百学生成了须陀洹。\end{enumerate}

\subsection\*{\textbf{1154} {\footnotesize 〔PTS 1147〕}}

\textbf{「听了牟尼的话,我更加净喜,这\\}
\textbf{「去蔽的等正觉、无荒秽者、具辩才者,}

“Esa bhiyyo pasīdāmi, sutvāna munino vaco;\\
vivaṭṭacchado Sambuddho, akhilo paṭibhānavā. %\hfill\textcolor{gray}{\footnotesize 17}

\begin{enumerate}\item \textbf{具辩才者},即具辩无碍解者。\end{enumerate}

\subsection\*{\textbf{1155} {\footnotesize 〔PTS 1148〕}}

\textbf{「证知了上天,了知了上下一切,\\}
\textbf{「大师彻底解决了自称有疑惑者的问题。}

Adhideve abhiññāya, sabbaṃ vedi varovaraṃ;\\
pañhān’antakaro Satthā, kaṅkhīnaṃ paṭijānataṃ. %\hfill\textcolor{gray}{\footnotesize 18}

\begin{itemize}\item 案,\textbf{varovaraṃ},PTS 及义注均作 \textit{parovaraṃ},义同。\end{itemize}

\subsection\*{\textbf{1156} {\footnotesize 〔PTS 1149〕}}

\textbf{「不可战胜、不动、无有其雷同之处,\\}
\textbf{「我定将到达,于此我没有疑惑,如是,请受持具信解之心的我!」}

Asaṃhīraṃ asaṅkuppaṃ, yassa natthi upamā kvaci;\\
addhā gamissāmi na m’ettha kaṅkhā, evaṃ maṃ dhārehi adhimuttacittan” ti. %\hfill\textcolor{gray}{\footnotesize 19}

\begin{enumerate}\item \textbf{不可战胜},即不为贪等战胜。\textbf{不动},即不变易法。以(上半颂的)两句说涅槃。\textbf{我定将到达},即我肯定将到达这无余涅槃界。\textbf{于此我没有疑惑},即我于此涅槃没有疑惑。\textbf{请受持具信解之心的我},褐者以世尊的教诫「如是,你也应信解于信」增长自己的信已,以信为轭而解脱,为显明其信解于信,而对世尊说「请受持具信解之心的我」,这里的意思是说「如同你对我说的,如是请受持信解」。\end{enumerate}

\begin{itemize}\item 案,经集正文于此结束,义注结云「十六婆罗门经注终」。\end{itemize}

\begin{center}\vspace{1em}彼岸道品第五\\Pārāyanavaggo pañcamo.\end{center}