\section{何戒经}

\begin{center}Kiṃsīla Sutta\end{center}\vspace{1em}

\subsection\*{\textbf{327} {\footnotesize 〔PTS 324〕}}

\textbf{何戒、何行事、增长何业的\\}
\textbf{人能完全住立,且圆满最高的义利?}

“Kiṃsīlo kiṃsamācāro, kāni kammāni brūhayaṃ;\\
naro sammā niviṭṭh’assa, uttamatthañ ca pāpuṇe”. %\hfill\textcolor{gray}{\footnotesize 1}

\begin{enumerate}\item 缘起为何?尊者舍利弗的俗家朋友,即长老父亲——Vaṅganta 婆罗门——的友人婆罗门的一个儿子,舍弃了五百六十俱胝的财产,在尊者舍利弗长老的跟前出家后,学习了所有佛语。长老对他反复教诫后,给予了业处,他却不能证得殊胜。随后,长老了知到「他应由佛引导」,便带了他到世尊跟前,就此比丘,不限于人,问了「何戒……」。然后,世尊便对他说了随后的几颂。
\item 这里,\textbf{何戒},即具足何等的止持之戒\footnote{止持之戒 \textit{vārittasīla}:PTS 本作「作持之戒 \textit{cārittasīla}」,若据下文「便未分别问题所及的作持之戒」,则 PTS 本似是。},或何等习性。\textbf{何行事},即与何等作持相应。\textbf{增长何业},即增长何等身业等。\textbf{人能完全住立},即欢喜的人能在教法内完全住立。\textbf{且圆满最高的义利},即是说能圆满一切义利中最高之阿罗汉。\end{enumerate}

\subsection\*{\textbf{328} {\footnotesize 〔PTS 325〕}}

\textbf{应尊敬长者、不妒忌,且应知时去见老师,\\}
\textbf{知道谈论法义的时机,应恭敬地听闻善说。}

“Vuḍḍhāpacāyī anusūyako siyā, kālaññū c’assa garūnaṃ dassanāya;\\
dhammiṃ kathaṃ erayitaṃ khaṇaññū, suṇeyya sakkacca subhāsitāni. %\hfill\textcolor{gray}{\footnotesize 2}

\begin{enumerate}\item 随后,世尊转向于「舍利弗受具足半月已达声闻波罗蜜,为什么问初业的凡夫之问」,了知到「是为门人」后,便未分别问题所及的作持之戒,为显明与其顺适的法,说了此颂等等。
\item 这里,有慧之长者、德之长者、出身之长者、年之长者等四种长者。因为即便生为少年,多闻的比丘在众少闻、高龄的比丘中间,由以博学之慧而为长者,即\textbf{慧之长者},即便众高龄的比丘也要在他跟前掌握佛语,期以教诫、裁断、解答问题。同样,即便是少年,具足证知的比丘名为\textbf{德之长者},基于他的教诫,众高龄者可得毗婆舍那之胎,圆满阿罗汉果。同样,即便是少年,经灌顶的刹帝利国王,或婆罗门,由值得其他人的礼拜,名为\textbf{出身之长者}。而一切最先出生者,名为\textbf{年之长者}。
\item 这里,因为以慧而言,除了世尊,没有与舍利弗长老相等者,同样,以德而言,由半月便通达一切声闻波罗蜜智,以出身而言,他投生于富裕婆罗门家族,所以,虽与此比丘年龄相同,他由此三因而为长者。而于此义,世尊唯就慧、德的长者之相而说「尊敬长者」。所以,此首句之义为:以对这样的长者行尊敬而\textbf{尊敬长者},以于这些长者的利养等除去妒忌而\textbf{不妒忌}。
\item \textbf{且应知时},即当贪欲生起,为去除它而前去见老师即为知时,当嗔恨……愚痴……懈怠生起,为去除它而前去见老师即为知时,由此,应如是知时\textbf{去见老师}。\textbf{法义},即与止观相应者。\textbf{谈论},即言说。\textbf{知道时机},即知晓此义的时机,或知道「听闻如此之义的时机为难得」。\textbf{应恭敬地听闻},即应恭敬地听闻此义。且不仅于此,还应恭敬地听闻其它与佛陀的功德等相关的\textbf{善说}之义。\end{enumerate}

\subsection\*{\textbf{329} {\footnotesize 〔PTS 326〕}}

\textbf{应适时去到老师近前,撇去顽固,举止谦卑,\\}
\textbf{对于语义、法、自制、梵行,应随念并实践。}

Kālena gacche garūnaṃ sakāsaṃ, thambhaṃ niraṅkatvā nivātavutti;\\
atthaṃ dhammaṃ saṃyamaṃ brahmacariyaṃ, anussare c’eva samācare ca. %\hfill\textcolor{gray}{\footnotesize 3}

\begin{enumerate}\item 即便知晓了「且应知时去见老师」中所说之理,以及为去除自身生起的贪等的时机,他前往老师跟前时,\textbf{应适时去到老师近前},想「我是行业处者及持头陀支者」,不应在礼拜支提、菩提树园、行乞之路、正中午时等的任何场合,见到老师站立,便为询问而前往,而应在自己的坐卧处,观察(老师)坐于坐处,平息了恼患,为询问业处等规定而前往之义。如是前往时,\textbf{撇去顽固,举止谦卑},消除造成顽固之相的慢而举止低卑,应如抹脚的布、断角的牛、拔牙的蛇般前往。
\item 然后,应随念并实践此老师所说的「语义、法……」。\textbf{语义},即所说之义。\textbf{法},即圣典之法。\textbf{自制},即戒。\textbf{梵行},即其余教法的梵行。\textbf{应随念并实践},即应在谈论的场合随念语义,应在谈论的场合随念法、自制、梵行,且不满足于仅作随念,而应对这一切实践、行事、受持、转起,即应对彼等谈论在自身的转起投入热情之义——因为如是行者,即为尽义务。\end{enumerate}

\subsection\*{\textbf{330} {\footnotesize 〔PTS 327〕}}

\textbf{乐法,喜法,住立于法,知抉择法,\\}
\textbf{不应从事污法的言论,应受如实的善说引领。}

Dhammārāmo dhammarato, dhamme ṭhito dhammavinicchayaññū;\\
n’evācare dhamma-sandosa-vādaṃ, tacchehi nīyetha subhāsitehi. %\hfill\textcolor{gray}{\footnotesize 4}

\begin{enumerate}\item 且此后,应「乐法,喜法,住立于法,知抉择法」。此中所有词中的法,即止观,而「乐、喜」义同。他应乐于法为\textbf{乐法},喜于法、不渴望其它为\textbf{喜法}。由转起法而\textbf{住立于法}。了知法的抉择「此为生智、此为灭智」为\textbf{知抉择法},他应如是。
\item 然后,对弱观者,王论等的旁论因于外在的色等生起喜乐而污染了止观之法,所以被称为「污法的言论」,\textbf{不应从事}此\textbf{污法的言论},反之,亲近住处、行处等之适宜者,\textbf{应受如实的善说引领}。此中,与止观相关即「如实」,其义为:应受如是的善说引领,应(如是)度时。\end{enumerate}

\subsection\*{\textbf{331} {\footnotesize 〔PTS 328〕}}

\textbf{戏笑、闲谈、悲、恼、行伪善、诡诈、贪求、慢、\\}
\textbf{愤激、粗俗、恶浊及沉迷,舍弃已,应离㤭慢而坚定。}

Hassaṃ jappaṃ paridevaṃ padosaṃ, māyākataṃ kuhanaṃ giddhi mānaṃ;\\
sārambhaṃ kakkasaṃ kasāvañ ca mucchaṃ, hitvā care vītamado ṭhitatto. %\hfill\textcolor{gray}{\footnotesize 5}

\begin{enumerate}\item 现在,为令「污法的言论」中极简略而说的、从事止观之比丘的随烦恼明了,与此外的随烦恼一起,说了此颂。
\item 文本也作 hāsaṃ。因为对修观的比丘,仅应于可笑之事抱以微笑,不应言说无义之论的\textbf{闲谈},不应于亲戚的不幸等生\textbf{悲},不应于树桩、荆棘等令意生\textbf{恼}。以\textbf{行伪善}所说的伪善、三种\textbf{诡诈}、于资具的\textbf{贪求}、由出身等的\textbf{慢}、被称为敌对于幸福的\textbf{愤激}、恶口之相的\textbf{粗俗}、贪等的\textbf{恶浊}、过度渴爱之相的\textbf{沉迷}等,这些过失,如欲求快乐者之于火坑、欲求洁净者之于粪坑、欲求活命者之于蛇毒等,当被舍弃。且\textbf{舍弃已},\textbf{应}以除去无病㤭等而\textbf{离㤭慢}、以无有心的散乱而\textbf{坚定}。因为如是行道者,以遍净一切随烦恼的修习,不久即证阿罗汉,因此世尊说「戏笑、闲谈……坚定」。\end{enumerate}

\subsection\*{\textbf{332} {\footnotesize 〔PTS 329〕}}

\textbf{所知是善说的精髓,所闻与所知的精髓是三摩地,\\}
\textbf{若人急躁、放逸,则其智慧与所闻不会增长。}

Viññāta-sārāni subhāsitāni, sutañ ca viññāta-samādhi-sāraṃ;\\
na tassa paññā ca sutañ ca vaḍḍhati, yo sāhaso hoti naro pamatto. %\hfill\textcolor{gray}{\footnotesize 6}

\begin{enumerate}\item 现在,若具足以「戏笑、闲谈」等方法所说的随烦恼的比丘,因为急躁而不行审视,贪者以贪而行,嗔者以嗔而行,且放逸而于善法的修习不常恒行,且对如此等人,以「应恭敬地听闻善说」等方法所说的教诫为徒然,所以,为显明此杂染与增长所闻等的敌对之相,以基于人的开示说了此颂。
\item 其义为:因为对这些与止观相关的\textbf{善说}的了知才是精髓。若能了知则善,否则只是取了声音,一无所成。以此闻所成智了知的\textbf{所闻}与此闻所成智\textbf{所知的精髓是三摩地},于此所知的诸法,其精髓是三摩地、心的不散乱、如实行道。因为仅以了知,无有义利可得成就。
\item 但是,\textbf{若}此\textbf{人}由为贪等所转而\textbf{急躁},由不常恒修习善法而\textbf{放逸},他便只是取声者。因此,由不能了知义利,由不能如实行道,\textbf{则其智慧与所闻不会增长}。\end{enumerate}

\subsection\*{\textbf{333} {\footnotesize 〔PTS 330〕}}

\textbf{喜于圣者宣说之法者,他们在语、意、行上都无比,\\}
\textbf{他们住立于寂静调柔、三摩地,得证所闻与智慧的精髓。}

Dhamme ca ye ariyapavedite ratā, anuttarā te vacasā manasā kammunā ca;\\
te santisoracca-samādhi-saṇṭhitā, sutassa paññāya ca sāram ajjhagū” ti. %\hfill\textcolor{gray}{\footnotesize 7}

\begin{enumerate}\item 如是显示了放逸有情的智慧、所闻的退失后,现在,为显示不放逸者证得此二者的精髓,说了此颂。
\item 这里,圣者宣说之法,即止观法。因为没有一佛未曾开示止观法便入灭的。所以,\textbf{喜于}、好于、不放逸于、常恒从事于\textbf{圣者宣说之法者,他们在语、意、行上都无比},他们由具足四种语善行、三种意善行、三种身善行,在语、意、行上都无比,较其余有情不等、最上、最胜。至此,以前分的戒显示了与圣道相应的戒。
\item 如是遍净戒者,「他们住立于寂静调柔、三摩地,得证所闻与智慧的精髓」,凡喜于圣者宣说之法者,他们不仅在语等上无比,还住立于寂静调柔与三摩地,得证所闻与智慧的精髓。表示期望时,用过去式 \textit{ajjhagū}。
\item 这里,寂静,即涅槃,调柔,即以喜于善妙之相而如实通达之慧,为寂静而调柔,即\textbf{寂静调柔},为以涅槃为所缘的道慧的同义语。\textbf{三摩地},即与之相应的道三摩地。\textbf{住立},即住立于此二者。\textbf{所闻与智慧的精髓},即阿罗汉果解脱。因为此梵行以解脱为精髓。
\item 如是,此中,世尊以「法」显明前分的行道,以「在语上无比」等显明戒蕴,以「寂静调柔、三摩地」显明慧蕴和定蕴,以此三蕴显明后分的行道已,以「所闻与智慧的精髓」显明不动解脱,以阿罗汉为顶点完成了开示。且在开示终了,此比丘即证须陀洹果,又经不久,便住于最上的阿罗汉果。\end{enumerate}

\begin{center}\vspace{1em}何戒经第九\\Kiṃsīlasuttaṃ navamaṃ.\end{center}