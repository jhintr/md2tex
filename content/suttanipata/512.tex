\section{胶耳学童问}

\begin{center}Jatukaṇṇi Māṇava Pucchā\end{center}\vspace{1em}

\subsection\*{\textbf{1103} {\footnotesize 〔PTS 1096〕}}

\textbf{「听说了不欲求爱欲的英雄,」尊者胶耳说,「我前来询问越过暴流的无爱欲者,\\}
\textbf{「请说说寂静的境地!俱生眼者!请如实地对我说!世尊!}

“Sutvān’ahaṃ vīram akāmakāmiṃ, \textit{(icc āyasmā Jatukaṇṇi)} oghātigaṃ puṭṭhum akāmam āgamaṃ;\\
santipadaṃ brūhi Sahajanetta, yathātacchaṃ Bhagavā brūhi me taṃ. %\hfill\textcolor{gray}{\footnotesize 1}

\begin{enumerate}\item \textbf{听说了不欲求爱欲的英雄},即我以「彼世尊亦即是」等方式听说了不欲求爱欲的英雄、佛陀。\textbf{俱生眼者},即与生俱来的一切知性之智眼者。\textbf{请对我说},即再次请求,因为应当请求千次,何况两次?\end{enumerate}

\subsection\*{\textbf{1104} {\footnotesize 〔PTS 1097〕}}

\textbf{「因为世尊征服了爱欲而行止,如同光辉的太阳以光芒(征服了)大地,\\}
\textbf{「宏慧者!请对小慧的我说法!我能了知\\}
\textbf{「于此舍弃生与老。」}

Bhagavā hi kāme abhibhuyya iriyati, ādicco va pathaviṃ tejī tejasā;\\
parittapaññassa me bhūripañña, ācikkha dhammaṃ yam ahaṃ vijaññaṃ;\\
jātijarāya idha vippahānaṃ”. %\hfill\textcolor{gray}{\footnotesize 2}

\subsection\*{\textbf{1105} {\footnotesize 〔PTS 1098〕}}

\textbf{「调伏对爱欲的贪求!胶耳!」世尊说,「视出离为安稳,\\}
\textbf{「你切莫有任何的执取或是丢弃。}

“Kāmesu vinaya gedhaṃ, \textit{(Jatukaṇṇī ti Bhagavā)} nekkhammaṃ daṭṭhu khemato;\\
uggahītaṃ nirattaṃ vā, mā te vijjittha kiñcanaṃ. %\hfill\textcolor{gray}{\footnotesize 3}

\begin{enumerate}\item \textbf{执取},即以爱、见等执取。\end{enumerate}

\subsection\*{\textbf{1106} {\footnotesize 〔PTS 1099〕}}

\textbf{「让先前的凋萎,你切莫有任何后来,\\}
\textbf{「如果你不执取中间,你将寂静而行。}

Yaṃ pubbe taṃ visosehi, pacchā te māhu kiñcanaṃ;\\
majjhe ce no gahessasi, upasanto carissasi. %\hfill\textcolor{gray}{\footnotesize 4}

\begin{enumerate}\item \textbf{先前的},即与过去诸行相关的已生起的烦恼。\end{enumerate}

\begin{itemize}\item 案,此颂全同执杖经第 956 颂。\end{itemize}

\subsection\*{\textbf{1107} {\footnotesize 〔PTS 1100〕}}

\textbf{「于一切名色离贪者,婆罗门!\\}
\textbf{「则无有能受制于死亡的诸漏。」}

Sabbaso nāmarūpasmiṃ, vītagedhassa Brāhmaṇa;\\
āsavāssa na vijjanti, yehi maccuvasaṃ vaje” ti. %\hfill\textcolor{gray}{\footnotesize 5}

\begin{enumerate}\item 如是,世尊同样以阿罗汉为顶点开示了此经。当开示终了,与先前一样,而有法的现观。\end{enumerate}

\begin{center}\vspace{1em}胶耳学童问第十一\\Jatukaṇṇimāṇavapucchā ekādasamā.\end{center}