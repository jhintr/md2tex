\section{洞窟八颂经}

\begin{center}Guhaṭṭhaka Sutta\end{center}\vspace{1em}

\begin{enumerate}\item 据说,世尊住于舍卫国时,宾头卢·婆罗豆婆遮尊者想坐在清凉之地昼住,去到位于㤭赏弥恒河边名为回转园的优填的庭园,而其它时候,他只是由过去的习行去到那里,像㤭梵波提长老去到三十三天的居处一样,如婆耆舍经释义中所说。他在那恒河边清凉的树下安止于等至后,坐而昼住。优填王也在这天入园游玩,白天以歌舞在园内嬉戏已,饮酒而醉,把头靠在一个女人的腰间,便睡着了。其他女眷想「国王睡了」,便起身在园内采摘花果,见到长老后,起了惭愧心,互相告诫「莫要作声」,轻声前往,礼拜了长老后,围绕而坐。长老从等至出起后,对她们说法,她们满意地说着「善哉,善哉」而听。
\item 坐着拿腰支着国王的头的女人想「她们丢下我去玩了」,嫉妒她们,便抖动大腿,唤醒了国王。国王醒后,不见了妻妾,便说「这些贱人们去哪了」。她说「她们不在意你,去找沙门玩了」。他便忿怒地朝长老走去。这些女眷看到国王后,有些起来,有些想「大王!我们在出家人跟前听法」而没有起来。他因此更加忿怒,也不礼拜长老,说「你为什么来」。「大王!为了远离」。他说「为远离而来的是这样被妻妾围绕而坐的吗?说说你的远离」。长老虽然自信,但想「他不是想知道远离而问的」,便默然。国王想「如果你不说,我就让红蚁咬你」,在某棵无忧树下取了一袋红蚁,撒在自己上面,抹干净身体后,又取了一袋,朝长老走去。长老怜悯他,「如果这国王冒犯我,会到恶趣的」,便以神通跃入空中而去。随后,女眷们说「大王!别的国王看到这样的出家人,会以花、香等供养,你却拿了红蚁袋攻击,预备去毁灭家族世系」。他了知了自己的过失而默然,问园林的守卫「长老别的日子也来这里吗」。「是的,大王」。「那么他来的时候,你报告给我」。一天,当长老来时,守卫便前去报告。国王前往长老处,问了问题后,便尽寿命皈依。然而,在被红蚁袋攻击的那天,长老在进入空中后又潜入地中,从世尊的香房里出来。世尊具念正知,正以右胁作狮子卧,见到长老后说「婆罗豆婆遮!为什么非时而来」。长老说「唯然,世尊」,告知了一切本末。世尊听后说「对贪求种种爱欲者,这远离之说又有何用」,仍以右胁而卧,为向长老开示法而说了此经。\end{enumerate}

\subsection\*{\textbf{779} {\footnotesize 〔PTS 772〕}}

\textbf{执著洞窟,为众多所覆蔽,持续沉溺于诱惑的人,\\}
\textbf{如此等者远于远离,因为在世间,爱欲不易舍弃。}

Satto guhāyaṃ bahunābhichanno, tiṭṭhaṃ naro mohanasmiṃ pagāḷho;\\
dūre vivekā hi tathāvidho so, kāmā hi loke na hi suppahāyā. %\hfill\textcolor{gray}{\footnotesize 1}

\begin{enumerate}\item \textbf{执著},即固执。\textbf{洞窟},即身体,因为身体是贪等猛兽的住处与场所,故称为「洞窟」。\textbf{为众多所覆蔽},为众多的贪等烦恼之结所覆蔽,以此而说内在的束缚。\textbf{持续},即以贪等而持续。\textbf{人},即有情。\textbf{沉溺于诱惑},种种爱欲被称为诱惑,因为人天迷惑于此,沉溺其中,以此而说外在的束缚。\textbf{如此等者远于远离},即这样的人远于、不近于身离等的三种远离。什么原因?\textbf{因为在世间,爱欲不易舍弃}。\end{enumerate}

\begin{itemize}\item 案,\textbf{三种远离},即身离、心离、镇伏离,见清净道论·说地遍品第 82 段及以下。\end{itemize}

\subsection\*{\textbf{780} {\footnotesize 〔PTS 773〕}}

\textbf{基于欲望,牵绊于有之悦乐,这些难以解脱,因为不能由他人得解脱,\\}
\textbf{关切着以后或以前,渴望着这些或先前的爱欲。}

Icchānidānā bhavasātabaddhā, te duppamuñcā na hi aññamokkhā;\\
pacchā pure vā pi apekkhamānā, ime va kāme purime va jappaṃ. %\hfill\textcolor{gray}{\footnotesize 2}

\begin{enumerate}\item \textbf{基于欲望},即因为渴爱。\textbf{牵绊于有之悦乐},即以乐受等,牵绊于有之悦乐。\textbf{这些难以解脱},即这些作为有之悦乐的物的法,或这些于此牵绊、基于欲望的有情,难以解脱。\textbf{因为不能由他人得解脱},即这些(物)不能由他人解脱,或这是说原因——这些有情难以解脱,为什么?因为不能被他人解脱。这意思是,若他们能解脱,则以自己的力量得解脱。\textbf{关切着以后或以前},即关切着未来或过去的爱欲。\textbf{渴望着这些或先前的爱欲},即以强烈的渴爱欲求着这些现在的爱欲,或先前过去、未来的两种。当知这两句与「这些难以解脱,因为不能由他人得解脱」相连属,否则无法得知「关切着、渴望着,他们做了什么,或被如何对待」。\end{enumerate}

\subsection\*{\textbf{781} {\footnotesize 〔PTS 774〕}}

\textbf{贪求爱欲者,执著的愚痴者,他们吝啬、住于不正、\\}
\textbf{陷入苦中,他们悲泣:「我们殁后会成为什么?」}

Kāmesu giddhā pasutā pamūḷhā, avadāniyā te visame niviṭṭhā;\\
dukkhūpanītā paridevayanti, “kiṃ sū bhavissāma ito cutāse”. %\hfill\textcolor{gray}{\footnotesize 3}

\begin{enumerate}\item 这些有情因渴爱于享用而\textbf{贪求爱欲},由从事于寻求等而\textbf{执著},由落入痴迷而\textbf{愚痴},由得到、悭吝及不能领受佛等的言语而\textbf{吝啬},\textbf{住于}身不正等的\textbf{不正},在临终时\textbf{陷入}死\textbf{苦中,他们悲泣「我们殁后会成为什么」}。\end{enumerate}

\subsection\*{\textbf{782} {\footnotesize 〔PTS 775〕}}

\textbf{所以,人应唯于此而学,应了知世上任何的「不正」,\\}
\textbf{不因此而行不正,因为智者们说「此生短暂」。}

Tasmā hi sikkhetha idh’eva jantu, yaṃ kiñci jaññā “visaman” ti loke;\\
na tassa hetū visamaṃ careyya, appañ h’idaṃ jīvitam āhu dhīrā. %\hfill\textcolor{gray}{\footnotesize 4}

\begin{enumerate}\item \textbf{应学},即应从事三学。\textbf{唯于此},即唯于此教法中。\end{enumerate}

\subsection\*{\textbf{783} {\footnotesize 〔PTS 776〕}}

\textbf{我看到世上这颤栗着的、陷于对有的渴爱的人类,\\}
\textbf{低贱的人们在死亡面前絮叨,不离对有与无有的渴爱。}

Passāmi loke pariphandamānaṃ, pajaṃ imaṃ taṇhagataṃ bhavesu;\\
hīnā narā maccumukhe lapanti, avītataṇhāse bhavābhavesu. %\hfill\textcolor{gray}{\footnotesize 5}

\begin{enumerate}\item \textbf{我看到},即我以肉眼等观察。\textbf{世上},即恶趣等中。\textbf{颤栗着},即处处颤栗着。\textbf{这人类},即这有情聚。\textbf{陷于渴爱},即落入渴爱,被征服、堕落的意思。\textbf{有},即欲有等。\textbf{低贱的人们},即低贱之业的人们。\textbf{在死亡面前絮叨},即至临终时,在死亡面前悲泣。\textbf{有与无有},即欲有等,或说「有与无有」即有有、再再而有。\end{enumerate}

\subsection\*{\textbf{784} {\footnotesize 〔PTS 777〕}}

\textbf{看!于我所颤栗着的人们,如少水的枯流中的鱼,\\}
\textbf{见到此后,无我所者应无执于诸有而行。}

Mamāyite passatha phandamāne, macche va appodake khīṇasote;\\
etam pi disvā amamo careyya, bhavesu āsattim akubbamāno. %\hfill\textcolor{gray}{\footnotesize 6}

\begin{enumerate}\item \textbf{我所},即由爱、见的我性而执为「我的」之物。\textbf{此},即此过患。\end{enumerate}

\subsection\*{\textbf{785} {\footnotesize 〔PTS 778〕}}

\textbf{对两端应调伏欲,遍知了触,无有贪求,\\}
\textbf{不做自责之事,智者不染于所见或所闻。}

Ubhosu antesu vineyya chandaṃ, phassaṃ pariññāya anānugiddho;\\
yad-attagarahī tad-akubbamāno, na lippatī diṭṭhasutesu dhīro. %\hfill\textcolor{gray}{\footnotesize 7}

\begin{enumerate}\item 如是,以第一颂显明了乐味,随后以四颂显明了过患,现在为显明有方法的出离及出离的功德,或以一切(偈颂)显明了爱欲的过患、卑下及烦恼,现在为显明出离的功德而说此二颂。这里,\textbf{两端},即触与触集等的两边。\textbf{应调伏欲},即已调伏欲贪。\textbf{遍知了触},即对眼触等的触,或对跟随触、与之相应的一切非色法及作为其所依、门、所缘等的色法等全体名色,以三遍知而遍知。\textbf{无有贪求},即对色等一切法无贪求。\textbf{智者不染于所见或所闻},他这样具有智慧的智者,于所见及所闻的诸法,不以二种粘著中的任一粘著而染,如虚空一般不染而极度明净。\end{enumerate}

\begin{itemize}\item 案,\textbf{二种粘著},即爱、见,见孙陀利迦婆罗豆婆遮经第 473 颂注。\end{itemize}

\subsection\*{\textbf{786} {\footnotesize 〔PTS 779〕}}

\textbf{遍知了想,他能度过暴流,牟尼不染于财产,\\}
\textbf{拔出了箭,不放逸而行,不希求此世与他世。}

Saññaṃ pariññā vitareyya oghaṃ, pariggahesu muni nopalitto;\\
abbūḷhasallo caram appamatto, nāsīsatī lokam imaṃ parañ cā ti. %\hfill\textcolor{gray}{\footnotesize 8}

\begin{enumerate}\item \textbf{遍知了想},不仅对触,还对欲想等的想,或对跟随想的先前所说的名色,以三遍知而遍知。以此行道,\textbf{他能度过}四种\textbf{暴流}。随后,度过暴流者、漏尽的\textbf{牟尼},以舍弃对爱、见的粘著,\textbf{不染于}爱、见的\textbf{财产}。由拔出贪等的箭而\textbf{拔出了箭},以所得的广大的念\textbf{不放逸而行}。或者先不放逸而行,再以此不放逸之行而拔出箭后,\textbf{不希求此世与他世}自、他的自性等。然后,「自最后心的灭,无取者如火一般灭去」,即以阿罗汉为顶点完成了开示。由于是对漏尽者开示,(世尊)仅作了法的指引,非以此开示令其生起更高的道、果。\end{enumerate}

\begin{center}\vspace{1em}洞窟八颂经第二\\Guhaṭṭhakasuttaṃ dutiyaṃ.\end{center}