\section{洞窟八颂经}

\begin{center}Guhaṭṭhaka Sutta\end{center}\vspace{1em}

\begin{enumerate}\item 缘起为何?据说,世尊住舍卫国时,宾头卢·婆罗豆婆遮尊者去到㤭赏弥恒河边名为回转园的优填(王)的庭园,想在清凉之地坐而昼住,而其它时候,他也仍以过去的习行去到那里,好比㤭梵波提长老去到三十三天的居处一样,这已如婆耆舍经注中所说。他在那恒河边清凉的树下安止于等至后,便坐而昼住。优填王也在这天入园游玩,白天的大部以歌舞在园内嬉戏已,饮酒而醉,把头靠在一个女人的腰间,便睡着了。其他女眷想「国王睡了」,便起身在园内采摘花果,见到长老,起了惭愧,互相劝诫「莫要作声」,轻声前往,礼拜了长老,便围绕而坐。长老从等至出起后,便对她们开示法,她们满意地说着「善哉、善哉」而听。
\item 坐着拿腰支着国王的头的女人想「她们丢下我去玩了」,嫉妒她们,便抖动大腿,唤醒了国王。国王醒后,不见了妻妾,便说:「这些贱人们在哪里?」她说:「她们不在意你,跑去找沙门玩了。」他便忿怒地朝长老走去。这些女眷看到国王后,有些起来,有些想「大王!我们在出家人跟前听法」而没有起来。他因此更加忿怒,也不礼拜长老,便说:「你为什么来?」「大王!为了远离。」他说:「为远离而来的是这样被妻妾围绕而坐的吗?说说你的远离。」长老虽然自信,但想「他不是想知道远离而问的」,便默然。国王想「要是你不说,我就让红蚁咬你」,在某棵无忧树下取了一袋红蚁,撒在自己上面,抹干净身体后,又取了一袋,朝长老走去。长老想「要是这国王冒犯我,会到苦处的」,为怜悯他,便以神变跃入空中而去。
\item 随后,女眷们便说:「大王!别的国王看到这样的出家人,会以花、香等供养,你却拿了红蚁袋发起攻击,预备去毁灭家族世系。」他了知了自己的过失而默然,问园林的守卫:「长老别的日子也来这里吗?」「唯!大王!」「那么,他来的时候,你就告知我。」一天,当长老来时,他便予告知。国王又前往长老处,问了问题后,便尽寿命皈依。
\item 而在被红蚁袋攻击的那天,长老在进入空中后又潜入地下,从世尊的香房里出来。世尊则具念正知,正以右胁作狮子卧,见到长老后便说:「婆罗豆婆遮!你为什么非时而来?」长老说「唯!世尊」,便告知了一切本末。世尊听后说「对这贪求种种爱欲者,远离之说又有何用」,仍以右胁而卧,为了向长老开示法,说了此经。\end{enumerate}

\subsection\*{\textbf{779} {\footnotesize 〔PTS 772〕}}

\textbf{执著洞窟,为众多所覆蔽,持续沉溺于诱惑的人,\\}
\textbf{如此等者远于远离,因为在世间,爱欲不易舍弃。}

Satto guhāyaṃ bahunābhichanno, tiṭṭhaṃ naro mohanasmiṃ pagāḷho;\\
dūre vivekā hi tathāvidho so, kāmā hi loke na hi suppahāyā. %\hfill\textcolor{gray}{\footnotesize 1}

\begin{enumerate}\item 这里,\textbf{执著},即固著。\textbf{洞窟},即身体。因为身体是贪等猛兽的住所而被称为「洞窟」。\textbf{为众多所覆蔽},即为众多的贪等烦恼之结所覆蔽,以此说内在的束缚。\textbf{持续},即因贪等持续。\textbf{人},即有情。\textbf{沉溺于诱惑},种种爱欲被称为诱惑,因为人天迷惑于此,沉溺其中,以此说外在的束缚。\textbf{如此等者远于远离},即这样的人远于、不近于身离等的三种远离\footnote{三种远离:即身离、心离、镇伏离,见\textbf{清净道论}·说地遍品第 82 段及以下。}。什么原因?\textbf{因为在世间,爱欲不易舍弃}。\end{enumerate}

\subsection\*{\textbf{780} {\footnotesize 〔PTS 773〕}}

\textbf{源于希望,系缚于有之悦乐,这些难以解脱,因为不能由他人解脱,\\}
\textbf{关切着以后或以前,渴望着这些或先前的爱欲。}

Icchānidānā bhavasātabaddhā, te duppamuñcā na hi aññamokkhā;\\
pacchā pure vā pi apekkhamānā, ime va kāme purime va jappaṃ. %\hfill\textcolor{gray}{\footnotesize 2}

\begin{enumerate}\item 如是,在第一颂中证明了「如此等者远于远离」,再为揭示如这般的有情的法性,说了此颂。这里,\textbf{源于希望},即因为渴爱。\textbf{系缚于有之悦乐},即系缚于乐受等的有之悦乐。\textbf{这些难以解脱},即这些作为有之悦乐的依处的法,或这些系缚于此、源于希望的有情,难以解脱。\textbf{因为不能由他人解脱},即不能由他人解脱。或者,这是在说原因:这些有情难以解脱,为什么?因为不能被他人解脱,其义即他们若能解脱,当以自己的力量解脱。
\item \textbf{关切着以后或以前},即关切着未来或过去的爱欲。\textbf{渴望着这些或先前的爱欲},即以强烈的渴爱愿求着这些现在的爱欲,或先前过去、未来之两种。当知这两句与「这些难以解脱,因为不能由他人解脱」相连属,否则无法得知「他们关切着、渴望着,正在做什么,或如何被对待」。\end{enumerate}

\subsection\*{\textbf{781} {\footnotesize 〔PTS 774〕}}

\textbf{贪求爱欲、执著、愚痴,他们吝啬、住于不正、\\}
\textbf{陷入苦中,他们悲泣「我们殁后会成为什么」?}

Kāmesu giddhā pasutā pamūḷhā, avadāniyā te visame niviṭṭhā;\\
dukkhūpanītā paridevayanti, “kiṃ sū bhavissāma ito cutāse”. %\hfill\textcolor{gray}{\footnotesize 3}

\begin{enumerate}\item 如是,在第一颂中证明了「如此等者远于远离」,且在第二颂揭示了如这般的有情的法性,现在,为揭示彼等恶业之行,说了此颂。其义为:这些有情因受用渴爱而\textbf{贪求爱欲},由从事于遍求等而\textbf{执著},由落入痴迷而\textbf{愚痴},由得到、悭吝及不能领受佛等之语而\textbf{吝啬},\textbf{住于}身不正等的\textbf{不正},在临终时\textbf{陷入}死\textbf{苦中,他们悲泣「我们殁后会成为什么」}?\end{enumerate}

\subsection\*{\textbf{782} {\footnotesize 〔PTS 775〕}}

\textbf{所以,人应唯于此修学,应了知世间任何的「不正」,\\}
\textbf{不应因此而行不正,因为智者们说「此生短暂」。}

Tasmā hi sikkhetha idh’eva jantu, yaṃ kiñci jaññā “visaman” ti loke;\\
na tassa hetū visamaṃ careyya, appañ h’idaṃ jīvitam āhu dhīrā. %\hfill\textcolor{gray}{\footnotesize 4}

\begin{enumerate}\item 正因为此,「所以……智者们说」。这里,\textbf{应修学},即应从事三学。\textbf{唯于此},即唯于此教法。其余自明。\end{enumerate}

\subsection\*{\textbf{783} {\footnotesize 〔PTS 776〕}}

\textbf{我看到世间这浑身颤栗着、陷于对诸有的渴爱的人类,\\}
\textbf{低贱的人们在死亡面前絮叨,不离对有与无有的渴爱。}

Passāmi loke pariphandamānaṃ, pajaṃ imaṃ taṇhagataṃ bhavesu;\\
hīnā narā maccumukhe lapanti, avītataṇhāse bhavābhavesu. %\hfill\textcolor{gray}{\footnotesize 5}

\begin{enumerate}\item 现在,为显示不如是而行者之落于危难,说了此颂。这里,\textbf{我看到},即我以肉眼等观察。\textbf{世间},即苦处等中。\textbf{浑身颤栗着},即处处颤栗着。\textbf{这人类},即这有情聚。\textbf{陷于渴爱},即落入渴爱,意即被征服、堕落。\textbf{对诸有},即对欲有等。\textbf{低贱的人们},即低贱之业的人们。\textbf{在死亡面前絮叨},即到临终时,在死亡面前悲泣。\textbf{有与无有},即欲有等,或「有与无有」即种种的有,即是说再再的有\footnote{有与无有,见\textbf{蛇经}第 6 颂的注。}。\end{enumerate}

\subsection\*{\textbf{784} {\footnotesize 〔PTS 777〕}}

\textbf{看!于执为我处颤栗者,如少水枯流中的鱼,\\}
\textbf{见到此后,无我所者应无执于诸有而行。}

Mamāyite passatha phandamāne, macche va appodake khīṇasote;\\
etam pi disvā amamo careyya, bhavesu āsattim akubbamāno. %\hfill\textcolor{gray}{\footnotesize 6}

\begin{enumerate}\item 现在,因为不离渴爱,且如是颤栗与絮叨,所以为激励对渴爱的调伏,说了此颂。这里,\textbf{执为我处},即由爱、见的我性而执为我所的依处。\textbf{看},即为招呼听众而说。\textbf{此},即此过患。其余自明。\end{enumerate}

\subsection\*{\textbf{785} {\footnotesize 〔PTS 778〕}}

\textbf{对两端应调伏欲,遍知了触,无有贪求,\\}
\textbf{不做自责之事,智者不染于所见、所闻。}

Ubhosu antesu vineyya chandaṃ, phassaṃ pariññāya anānugiddho;\\
yad attagarahī tad akubbamāno, na lippatī diṭṭhasutesu dhīro. %\hfill\textcolor{gray}{\footnotesize 7}

\begin{enumerate}\item 如是,此中以第一颂显明了乐味,随后以四颂显明了过患,现在为显明俱方法之出离及出离的功德,或者,以这一切(偈颂)显明了爱欲的过患、下劣与杂染,现在为显明出离的功德,说了以下二颂。
\item 这里,\textbf{两端},即触与触集等的两部分。\textbf{应调伏欲},即调伏了欲贪。\textbf{遍知了触},即对眼触等的触,或对与触相伴、与之相应的一切非色法,以及以其依处、门、所缘等的色法等全体名色,以三遍知而遍知。\textbf{无有贪求},即对色等一切法无贪求。\textbf{智者不染于所见、所闻},即他这样具有坚毅的智者,于所见及所闻之法,不以二种涂抹\footnote{二种涂抹:即爱、见,见\textbf{孙陀利迦婆罗豆婆遮经}第 473 颂注。}中的任一涂抹而染,如虚空一般离染而极度洁净。\end{enumerate}

\subsection\*{\textbf{786} {\footnotesize 〔PTS 779〕}}

\textbf{遍知了想,他能度过暴流,牟尼不染于资产,\\}
\textbf{拔出了箭,不放逸而行,不希求此世与他世。}

Saññaṃ pariññā vitareyya oghaṃ, pariggahesu muni nopalitto;\\
abbūḷhasallo caram appamatto, nāsīsatī lokam imaṃ parañ cā ti. %\hfill\textcolor{gray}{\footnotesize 8}

\begin{enumerate}\item 此颂的略义为:不仅对触,还对欲想等类的\textbf{想},或对与想相伴的先前所说的名色,以三遍知而遍知,以此行道,\textbf{他能度过}四种\textbf{暴流},随后,这度过暴流者、漏尽\textbf{牟尼},以舍弃对爱、见的涂抹,\textbf{不染于}爱、见之\textbf{资产},由拔出贪等的箭而\textbf{拔出了箭},以所得念的广大\textbf{不放逸而行},或者,于前分不放逸而行,以此不放逸行拔出箭后,\textbf{不希求}自、他之自体等类的\textbf{此世与他世},相反,由最后心的灭,无取者如火一般寂灭,即以阿罗汉为顶点完成了开示。(世尊)仅作了法的指引,而非以此开示令生起更高的道或果,由对漏尽者开示故。\end{enumerate}

\begin{center}\vspace{1em}洞窟八颂经第二\\Guhaṭṭhakasuttaṃ dutiyaṃ.\end{center}