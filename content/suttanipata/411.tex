\section{争辩争论经}

\begin{enumerate}\item 缘起为何?仍在此大集会中,(世尊)了知了有些天人生起「那么,争辩等八法\footnote{八法:即第 869 颂所说的「争辩、争论、悲、忧、悭吝、慢、傲慢、诽谤」等。}从何转起」之心后,为揭示彼法,以先前所述的方法,让相佛问自己问题后而说。这里,由问答次第安立故,所有颂的连结自明,而这些非自明之词的解释当知如下。\end{enumerate}

\subsection\*{\textbf{869} {\footnotesize 〔PTS 862〕}}

\textbf{「争辩、争论从何而生?还有悲、忧以及悭吝、\\}
\textbf{「慢、傲慢以及诽谤,请您快说它们从何而生?」}

\begin{enumerate}\item \textbf{争辩、争论从何而生},即争辩与其前分的争论,它们从何产生?\textbf{它们},即这一切八烦恼法。\textbf{请您快说},即请说这由我所问之义,我请求你。\textbf{快}是请求之义的不变词。\end{enumerate}

\subsection\*{\textbf{870} {\footnotesize 〔PTS 863〕}}

\textbf{「争辩、争论从喜爱而生,还有悲、忧以及悭吝、\\}
\textbf{「慢、傲慢以及诽谤,\\}
\textbf{「争辩、争论与悭吝相关,当争论生起便有诽谤。」}

\begin{enumerate}\item \textbf{从喜爱而生},即从喜爱之事物而生,而此中的应用则如「义释\footnote{即\textbf{大义释}第 98 段。}」中所说。\textbf{争辩、争论与悭吝相关},以此显示争辩、争论等不仅以喜爱之事物,亦以悭吝为缘。且此中,当知以争辩、争论为首来说彼等一切法。且正如悭吝之于彼等,争论亦于诽谤如此,因此说「\textbf{当争论生起便有诽谤}」。\end{enumerate}

\subsection\*{\textbf{871} {\footnotesize 〔PTS 864〕}}

\textbf{「那么世间的喜爱以何为因,还有行于世间的贪?\\}
\textbf{「人对未来的希望与达成以何为因?」}

\begin{enumerate}\item \textbf{那么世间的喜爱以何为因,还有行于世间的贪},即如「争辩从喜爱而生」所说,那么这世间的喜爱以何为因?且不仅是喜爱,还有那些贪婪的刹帝利等行(于世间),以贪为因、为贪征服而行,他们的这贪以何为因?即以一问而问二义。\textbf{希望与达成},即希望与此希望的成功。\textbf{人对未来},即是说目标。这亦唯一问。\end{enumerate}

\subsection\*{\textbf{872} {\footnotesize 〔PTS 865〕}}

\textbf{「世间的喜爱以欲为因,还有行于世间的贪,\\}
\textbf{「人对未来的希望与达成以此为因。」}

\begin{enumerate}\item \textbf{以欲为因},即以爱欲之欲等的欲为因。\textbf{还有行于世间的贪},即还有那些贪婪的刹帝利等行(于世间),他们的贪也以欲为因,亦以一而答二义。\textbf{以此为因},即是说仍以欲为因。且「以何为因」中的语法当知仍如针毛经所述\footnote{参见\textbf{针毛经}第 273 颂注。}。\end{enumerate}

\subsection\*{\textbf{873} {\footnotesize 〔PTS 866〕}}

\textbf{「世间的欲以何为因?还有抉择从何而生?\\}
\textbf{「以及忿怒、妄语、疑,或是沙门所说的诸法?」}

\begin{enumerate}\item \textbf{抉择},即以爱、见而抉择。\textbf{或是沙门所说的诸法},即佛沙门所说的其它与忿怒等相应,或如此类的不善法,它们从何而生?\end{enumerate}

\subsection\*{\textbf{874} {\footnotesize 〔PTS 867〕}}

\textbf{「凡在世间说『愉悦、不愉悦』,欲便依此而生,\\}
\textbf{「于诸色中见到了离有与有,人在世间做出抉择。}

\begin{enumerate}\item \textbf{欲便依此而生},此,即苦乐之受,依此被称为两处事物的愉悦、不愉悦,以愿求相合、相离便生起欲。至此解答了「世间的欲以何为因」之问。\textbf{于诸色中见到了离有与有},即于诸色中见到了生灭。\textbf{人在世间做出抉择},即在苦处等的世间,此人为了获得财富而做出爱之抉择,并以「自我在我中生起」等方法做出见之抉择。而此中的应用则如「义释\footnote{即\textbf{大义释}第 102 段。}」中所说。至此解答了「还有抉择从何而生」之问。\end{enumerate}

\subsection\*{\textbf{875} {\footnotesize 〔PTS 868〕}}

\textbf{「忿怒、妄语、疑,这些法当二者存在时也是,\\}
\textbf{「有疑者应在智路上修学,法由沙门在了知后宣说。」}

\begin{enumerate}\item \textbf{这些法当二者存在时也是},即这忿怒等法仅当愉悦、不愉悦二者存在时成立、生起。而彼等的发生则如「义释\footnote{即\textbf{大义释}第 103 段。}」中所说。至此解答了第三个问题。
\item 现在,若人对如是解答这些问题存有疑惑,为显示舍弃疑惑之法,而说「\textbf{有疑者应在智路上修学}」,即是说他应为了知见、为了证得智而修学三学。什么原因?\textbf{法由沙门在了知后宣说},因为佛沙门唯了知法而说,非于彼法不了知。然而,因自身的无智,未了知彼等者无能了知,而非因开示的过失,所以,有疑者应在智路上修学,法由沙门在了知后宣说。\end{enumerate}

\subsection\*{\textbf{876} {\footnotesize 〔PTS 869〕}}

\textbf{「愉悦、不愉悦以何为因?当什么不存在即无彼等?\\}
\textbf{「这『离有与有』之义,请对我说它以何为因?」}

\begin{enumerate}\item 在「\textbf{愉悦、不愉悦以何为因}」中的愉悦、不愉悦,即乐、苦受之意。\textbf{这「离有与有」之义,请对我说它以何为因},即愉悦、不愉悦的离有与有之义。此处作了性的转换\footnote{这里是说原文本应是阳性词,却用了中性格式。}。而这即是说:这愉悦、不愉悦的离有与有之义,请对我说它以何为因?且此中,从意义上当知离有与有只是愉悦、不愉悦的离有与有之事、离有与有之见。同样,「义释」在此问的解答部分说:有见以触为因,离有见亦以触为因。\footnote{即\textbf{大义释}第 105 段。}\end{enumerate}

\subsection\*{\textbf{877} {\footnotesize 〔PTS 870〕}}

\textbf{「愉悦、不愉悦以触为因,当触不存在即无彼等,\\}
\textbf{「这『离有与有』之义,我对你说,它以此为因。」}

\begin{enumerate}\item \textbf{以此为因},即以触为因。\end{enumerate}

\subsection\*{\textbf{878} {\footnotesize 〔PTS 871〕}}

\textbf{「那么世间的触以何为因?执取从何而生?\\}
\textbf{「当什么不存在即无我执?当什么灭去则诸触不触?」}

\begin{enumerate}\item \textbf{当什么灭去则诸触不触},即当超越了什么,眼触等五触不触。\end{enumerate}

\subsection\*{\textbf{879} {\footnotesize 〔PTS 872〕}}

\textbf{「缘名与色而有触,以希求为因而有执取,\\}
\textbf{「当希求不存在即无我执,当诸色灭去则诸触不触。」}

\begin{enumerate}\item \textbf{缘名与色},即缘相应之名与依处、所缘之色。\textbf{当诸色灭去则诸触不触},即当超越了诸色,眼触等五触不触。\end{enumerate}

\subsection\*{\textbf{880} {\footnotesize 〔PTS 873〕}}

\textbf{「体认何等者的色灭去?乐与苦又如何灭去?\\}
\textbf{「请对我说如何灭去!『我们应了知此』,这是我的心意。」}

\begin{enumerate}\item \textbf{体认何等者},即践行何等者。\textbf{色灭去},即色将不存在。\textbf{乐与苦},即是问可意、不可意之色。\end{enumerate}

\subsection\*{\textbf{881} {\footnotesize 〔PTS 874〕}}

\textbf{「非想之有想者,非异想之有想者,也非无想者,非灭想者,\\}
\textbf{「如是体认者的色即灭去,因为以想为因,即名为戏论。」}

\begin{enumerate}\item \textbf{非想之有想者},即如体认者的色灭去,他不是通过自然之想的有想者。\textbf{非异想之有想者},即也不是通过异想、异样之想的有想者,或失疯者、失心者。\textbf{也非无想者},即也不是入灭定者或无想有情。\textbf{非灭想者},即也不是以\begin{quoting}(超越)一切色想……(法集论第 265 段)\end{quoting}等方法超越想的得无色禅者。\textbf{如是体认者的色即灭去},即不住于「想之有想者」等的状态已,如\begin{quoting}当他如是等持心……,为获得空无边处的等至而引导其心。\end{quoting}所说,如是具无色道的体认者的色即灭去。\textbf{因为以想为因,即名为戏论},即显示以如是行道者的想为因,尤未能舍弃爱、见的戏论。\end{enumerate}

\subsection\*{\textbf{882} {\footnotesize 〔PTS 875〕}}

\textbf{「我们所问的,您都已向我们解说,我们另有所问,请您快说!\\}
\textbf{「于此,是否有些智者说,至此已是夜叉最上的清净,\\}
\textbf{「还是说除此之外,另有其它?」}

\begin{enumerate}\item \textbf{于此,是否有些智者说,至此已是夜叉最上的清净,还是说除此之外,另有其它},即问:于此,是否有智的沙门婆罗门说,至此已是有情最上的清净,还是说除此之外,另有其它较无色等至更高者?\end{enumerate}

\subsection\*{\textbf{883} {\footnotesize 〔PTS 876〕}}

\textbf{「于此,有些智者也说,至此已是夜叉最上的清净,\\}
\textbf{「然而其中有些,自称是善巧于无余依者,说尚有时日。}

\begin{enumerate}\item \textbf{有些也说,至此已是最上},即有些常论的沙门婆罗门,自认是智者,说至此已是最上的清净。\textbf{然而其中有些,说尚有时日},即其中有些断论者,说断为时日\footnote{时日 \textit{samaya}:据菩提比丘注 1937,也可译作「现观」,即 abhisamaya 的省略。}。\textbf{自称是善巧于无余依者},即于无余依为善巧论者。\end{enumerate}

\subsection\*{\textbf{884} {\footnotesize 〔PTS 877〕}}

\textbf{「了知了这些为『依止』,这牟尼了知了依止而省视,\\}
\textbf{「了知已即解脱,不落入争论,智者不入于有与无有。」}

\begin{enumerate}\item \textbf{了知了这些为依止},即了知了这些成见者「依止常、断见」。\textbf{这牟尼了知了依止而省视},即这佛牟尼之智者了知了依止而省视。\textbf{了知已即解脱},即从苦、无常等了知诸法已,即解脱。\textbf{不入于有与无有},即不聚集于再再的投生,即以阿罗汉为顶点完成了开示。当开示终了,仍与「前分离经」所说的一样,而有现观。\end{enumerate}

\begin{center}争辩争论经第十一\end{center}