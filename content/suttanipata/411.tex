\section{争辩争论经}

\begin{center}Kalahavivāda Sutta\end{center}\vspace{1em}

\begin{enumerate}\item 亦在此大集会中,了知了有些天人生起「那么,争辩等八法从何转起」之心后,为显明彼法,以如前所述的方法,使相佛问自己问题后而说。\end{enumerate}

\begin{itemize}\item 案,\textbf{八法},即 869 颂所说的「争辩、争论、悲、忧、悭吝、慢、傲慢、诽谤」等。\end{itemize}

\subsection\*{\textbf{869} {\footnotesize 〔PTS 862〕}}

\textbf{「争辩、争论从何而生?还有悲忧以及悭吝、\\}
\textbf{「慢、傲慢,以及诽谤,请您说说它们从何而生?」}

“Kuto pahūtā kalahā vivādā, paridevasokā sahamaccharā ca;\\
mānātimānā sahapesuṇā ca, kuto pahūtā te tad-iṅgha brūhi”. %\hfill\textcolor{gray}{\footnotesize 1}

\subsection\*{\textbf{870} {\footnotesize 〔PTS 863〕}}

\textbf{「争辩、争论从喜爱而生,还有悲忧以及悭吝、\\}
\textbf{「慢、傲慢,以及诽谤,\\}
\textbf{「争辩、争论与悭吝相关,当争论生起时则有诽谤。」}

“Piyappahūtā kalahā vivādā, paridevasokā sahamaccharā ca;\\
mānātimānā sahapesuṇā ca;\\
maccherayuttā kalahā vivādā, vivādajātesu ca pesuṇāni”. %\hfill\textcolor{gray}{\footnotesize 2}

\subsection\*{\textbf{871} {\footnotesize 〔PTS 864〕}}

\textbf{「那么在世间,喜爱,以及行于世间的贪,以何为因?\\}
\textbf{「人对未来的希望与达成,以何为因?」}

“Piyā su lokasmiṃ kutonidānā, ye cāpi lobhā vicaranti loke;\\
āsā ca niṭṭhā ca kutonidānā, ye samparāyāya narassa honti”. %\hfill\textcolor{gray}{\footnotesize 3}

\subsection\*{\textbf{872} {\footnotesize 〔PTS 865〕}}

\textbf{「在世间,喜爱,以及行于世间的贪,以欲为因,\\}
\textbf{「人对未来的希望与达成,以此为因。」}

“Chandānidānāni piyāni loke, ye cāpi lobhā vicaranti loke;\\
āsā ca niṭṭhā ca itonidānā, ye samparāyāya narassa honti”. %\hfill\textcolor{gray}{\footnotesize 4}

\subsection\*{\textbf{873} {\footnotesize 〔PTS 866〕}}

\textbf{「那么在世间,欲以何为因?抉择从何而生?\\}
\textbf{「以及忿怒、妄语、疑,或沙门所说的其它诸法?」}

“Chando nu lokasmiṃ kutonidāno, vinicchayā cāpi kuto pahūtā;\\
kodho mosavajjañ ca kathaṅkathā ca, ye vā pi dhammā samaṇena vuttā”. %\hfill\textcolor{gray}{\footnotesize 5}

\begin{enumerate}\item \textbf{抉择},即以爱、见而抉择。\end{enumerate}

\subsection\*{\textbf{874} {\footnotesize 〔PTS 867〕}}

\textbf{「他们在世间说『可意、不可意』,欲便依此而生起,\\}
\textbf{「于色等中见到了离有与有,人在世间做出抉择。}

“‘Sātaṃ asātan’ ti yam āhu loke, tam ūpanissāya pahoti chando;\\
rūpesu disvā vibhavaṃ bhavañ ca, vinicchayaṃ kubbati jantu loke. %\hfill\textcolor{gray}{\footnotesize 6}

\begin{enumerate}\item \textbf{于色等中见到了离有与有},即于色等中见到了生灭,\textbf{人在世间做出抉择},在恶趣等的世间,此人为获得财富而做爱的抉择,以及由「吾我已生起」等做见的抉择。\end{enumerate}

\subsection\*{\textbf{875} {\footnotesize 〔PTS 868〕}}

\textbf{「忿怒、妄语、疑,这些法也(生起),仅当二元存在时,\\}
\textbf{「有疑者应在智路上修学,沙门在了知后宣说法。」}

Kodho mosavajjañ ca kathaṅkathā ca, ete pi dhammā dvaya-m eva sante;\\
kathaṅkathī ñāṇapathāya sikkhe, ñatvā pavuttā samaṇena dhammā”. %\hfill\textcolor{gray}{\footnotesize 7}

\begin{enumerate}\item \textbf{仅当二元存在时},即这些忿怒等法仅当可意、不可意的二元存在时存在、生起。至此已解答了第三个问题。现在,若人对这些问题的解答生起疑惑,为显示驱除其疑惑的方法,而说\textbf{有疑者应在智路上修学},即他应为了智见、为了证得智而修学三学。什么原因?\textbf{沙门在了知后宣说法},因为佛沙门唯了知后而说,非于彼法不了知。然而,由自身的无智,而非由教法的过失,未了知者无法得知,所以,有疑者应在智路上修学,沙门在了知后宣说法。\end{enumerate}

\subsection\*{\textbf{876} {\footnotesize 〔PTS 869〕}}

\textbf{「可意、不可意以何为因?当什么不存在即无彼等?\\}
\textbf{「请告诉我『离有与有』以何为因?」}

“Sātaṃ asātañ ca kutonidānā, kismiṃ asante na bhavanti h’ete;\\
‘vibhavaṃ bhavañ cāpi’ yam etam atthaṃ, etaṃ me pabrūhi yatonidānaṃ”. %\hfill\textcolor{gray}{\footnotesize 8}

\begin{enumerate}\item \textbf{可意、不可意}是乐、苦受的意思。\end{enumerate}

\subsection\*{\textbf{877} {\footnotesize 〔PTS 870〕}}

\textbf{「可意、不可意以触为因,当触不存在即无彼等,\\}
\textbf{「我说『离有与有』以此为因。」}

“Phassanidānaṃ sātaṃ asātaṃ, phasse asante na bhavanti h’ete;\\
‘vibhavaṃ bhavañ cāpi’ yam etam atthaṃ, etaṃ te pabrūmi itonidānaṃ”. %\hfill\textcolor{gray}{\footnotesize 9}

\subsection\*{\textbf{878} {\footnotesize 〔PTS 871〕}}

\textbf{「那么在世间触以何为因?执取从何而生?\\}
\textbf{「当什么不存在即无我执?当什么灭去则诸触不触?」}

“Phasso nu lokasmi kutonidāno, pariggahā cāpi kuto pahūtā;\\
kismiṃ asante na mamattam atthi, kismiṃ vibhūte na phusanti phassā”. %\hfill\textcolor{gray}{\footnotesize 10}

\begin{enumerate}\item \textbf{当什么灭去则诸触不触},即当超越了什么,眼触等五触不触。\end{enumerate}

\subsection\*{\textbf{879} {\footnotesize 〔PTS 872〕}}

\textbf{「缘名与色而有触,以希求为因而有执取,\\}
\textbf{「当希求不存在即无我执,当诸色灭去则诸触不触。」}

“Nāmañ ca rūpañ ca paṭicca phasso, icchānidānāni pariggahāni;\\
icchāy’asantyā na mamattam atthi, rūpe vibhūte na phusanti phassā”. %\hfill\textcolor{gray}{\footnotesize 11}

\begin{enumerate}\item \textbf{缘名与色},即缘相应的名与所依、所缘的色。\end{enumerate}

\subsection\*{\textbf{880} {\footnotesize 〔PTS 873〕}}

\textbf{「对和合者,色如何灭去?乐与苦又如何灭去?\\}
\textbf{「请告诉我如何灭去!『我们应了知此』,这是我的心意。」}

“Kathaṃ sametassa vibhoti rūpaṃ, sukhaṃ dukhañ cāpi kathaṃ vibhoti;\\
etaṃ me pabrūhi yathā vibhoti, ‘taṃ jāniyāmā’ ti me mano ahu”. %\hfill\textcolor{gray}{\footnotesize 12}

\begin{enumerate}\item \textbf{对和合者},即对行道者。\end{enumerate}

\subsection\*{\textbf{881} {\footnotesize 〔PTS 874〕}}

\textbf{「非想之有想者,非异想之有想者,也非无想者,非灭想者,\\}
\textbf{「如是,对和合者,色即灭去,因为以想为因,即名为戏论。」}

“Na saññasaññī na visaññasaññī, no pi asaññī na vibhūtasaññī;\\
evaṃ sametassa vibhoti rūpaṃ, saññānidānā hi papañcasaṅkhā”. %\hfill\textcolor{gray}{\footnotesize 13}

\begin{enumerate}\item \textbf{非想之有想者},如对和合者,色即灭去,他不是通过自然之想的有想者。\textbf{非异想之有想者},也不是通过异想、异样之想的有想者,或疯子,或失心者。\textbf{也非无想者},也不是入灭定或无想有情的无想者。\textbf{非灭想者},也不是以(法集论)「一切色想」等方式超越想的得无色禅者。\textbf{如是,对和合者,色即灭去},不处于「想之有想者」等的状态已,如「他如是等持其心……,为获得空无边处的等至而引导其心」所说,如是,对具无色道的和合者,色即灭去。\textbf{因为以想为因,即名为戏论},即显示以如是行道者的想为因,尤未能舍弃爱、见的戏论。\end{enumerate}

\subsection\*{\textbf{882} {\footnotesize 〔PTS 875〕}}

\textbf{「我们所问的,您都已向我们解说,我们另有所问,请您说说!\\}
\textbf{「于此,有些智者说,至此已是夜叉最上的清净,\\}
\textbf{「还是说这里另有其它?」}

“Yaṃ taṃ apucchimha akittayī no, aññaṃ taṃ pucchāma tad-iṅgha brūhi;\\
ettāvat’aggaṃ nu vadanti h’eke, yakkhassa suddhiṃ idha paṇḍitāse;\\
udāhu aññam pi vadanti etto”. %\hfill\textcolor{gray}{\footnotesize 14}

\begin{enumerate}\item \textbf{另有其它},另有其它较之无色等至更高者。\end{enumerate}

\subsection\*{\textbf{883} {\footnotesize 〔PTS 876〕}}

\textbf{「于此,有些智者说,至此已是夜叉最上的清净,\\}
\textbf{「然而,其中有些,自称是善巧者,说于无余依尚有时日。}

“Ettāvat’aggam pi vadanti h’eke, yakkhassa suddhiṃ idha paṇḍitāse;\\
tesaṃ pan’eke samayaṃ vadanti, anupādisese kusalā vadānā. %\hfill\textcolor{gray}{\footnotesize 15}

\subsection\*{\textbf{884} {\footnotesize 〔PTS 877〕}}

\textbf{「了知了这些为『有所依』,牟尼了知了依止而省视,\\}
\textbf{「了知已便即解脱,不入争论,智者不入于有与无有。」}

Ete ca ñatvā ‘upanissitā’ ti, ñatvā munī nissaye so vimaṃsī;\\
ñatvā vimutto na vivādam eti, bhavābhavāya na sameti dhīro” ti. %\hfill\textcolor{gray}{\footnotesize 16}

\begin{itemize}\item 案,\textbf{有与无有},见蛇经第 6 颂注。\end{itemize}

\begin{center}\vspace{1em}争辩争论经第十一\\Kalahavivādasuttaṃ ekādasamaṃ.\end{center}