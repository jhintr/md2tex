\section{婆罗门法者经}

\begin{center}Brāhmaṇadhammika Sutta\end{center}\vspace{1em}

\textbf{如是我闻\footnote{此经旧译见中阿含经第 156 经「梵波罗延经」。}。一时世尊住舍卫国祇树给孤独园。尔时,众多㤭萨罗的富裕婆罗门衰老、年迈、高龄、迟暮、岁月已逝,往世尊处走去,走到后,问候了世尊,彼此寒暄已,坐在一边。}

Evaṃ me sutaṃ— ekaṃ samayaṃ Bhagavā Sāvatthiyaṃ viharati Jetavane Anāthapiṇḍikassa ārāme. Atha kho sambahulā Kosalakā brāhmaṇamahāsālā jiṇṇā vuḍḍhā mahallakā addhagatā vayo anuppattā yena Bhagavā ten’upasaṅkamiṃsu, upasaṅkamitvā Bhagavatā saddhiṃ sammodiṃsu, sammodanīyaṃ kathaṃ sāraṇīyaṃ vītisāretvā ekamantaṃ nisīdiṃsu.

\begin{enumerate}\item 缘起为何?这在其因缘中如「尔时,众多……」等方法所述。这里,\textbf{众多},即许多、非一。\textbf{㤭萨罗的},即㤭萨罗国的居民。\textbf{富裕婆罗门},即以出生为婆罗门,以富裕性为富裕。据说,那些经储蓄而存有八十俱胝之数的财产者,被称为富裕婆罗门,而他们便是如此,因此说是「富裕婆罗门」。\textbf{衰老},即老弱,因老而成齿落等状。\textbf{年迈},即肢体已至增长的极限。\textbf{高龄},即是说出生甚久。\textbf{迟暮},即旅途已达,意为经历了两三位国王的继位。\textbf{岁月已逝},即已到达最后的年龄。
\item 复次,当知此中亦可如是连接:\textbf{衰老},即古老,即是说长久保持家族传统。\textbf{年迈},即与戒正行等德的增长相应。\textbf{高龄},即具足财产,巨财、巨富。\textbf{迟暮},即践行于道,不违犯婆罗门的禁行等的禁忌而行。\textbf{岁月已逝},即已至生老之相的最终年龄。其余于此自明。
\item \textbf{问候了世尊},即询问可忍耐等,彼此转起相同的喜悦。且以此「乔达摩君尚可忍耐?尚可维持?少病少恼,有力轻起,安乐住否」等问候的谈论,由生起被称为喜悦的庆慰且值得问候为「应问候」,以义文之甜美,虽极长时亦值得忆念、值得常常令起,且由应忆念之相为「应忆念」,以及由听闻之乐为应问候,由随念之乐为应忆念,同样,由文句遍净为应问候,由语义遍净为应忆念,如是以多种理法\textbf{彼此寒暄已}\footnote{彼此寒暄已:这是意译,字面的直译即上文解释的「交换了应问候、应忆念的谈论」。},即令终结、完成已,欲问所为前来之义,便\textbf{坐在一边}。此以\begin{quoting}不在前,不在后,亦不近、不远,\\非于沟渠、逆风,亦非低处高处。\end{quoting}等方法,已在吉祥经注中说明。\end{enumerate}

\textbf{这些坐在一边的富裕婆罗门对世尊说:「乔达摩君!现今还有婆罗门在古昔婆罗门的婆罗门法上吗?」「众婆罗门!现今没有婆罗门在古昔婆罗门的婆罗门法上。」「善哉!若不麻烦乔达摩君的话,请乔达摩君对我们说说古昔婆罗门的婆罗门法!」「那么,众婆罗门!谛听!善加作意!我将说法。」「如是,先生!」这些富裕婆罗门答世尊。世尊说:}

Ekamantaṃ nisinnā kho te brāhmaṇamahāsālā Bhagavantaṃ etad avocuṃ: “sandissanti nu kho, bho Gotama, etarahi brāhmaṇā porāṇānaṃ brāhmaṇānaṃ brāhmaṇadhamme” ti? “Na kho, brāhmaṇā, sandissanti etarahi brāhmaṇā porāṇānaṃ brāhmaṇānaṃ brāhmaṇadhamme” ti. “Sādhu no bhavaṃ Gotamo porāṇānaṃ brāhmaṇānaṃ brāhmaṇadhammaṃ bhāsatu, sace bhoto Gotamassa agarū” ti. “Tena hi, brāhmaṇā, suṇātha, sādhukaṃ manasi karotha, bhāsissāmī” ti. “Evaṃ, bho” ti kho te brāhmaṇamahāsālā Bhagavato paccassosuṃ. Bhagavā etad avoca:

\begin{enumerate}\item 如是,\textbf{这些坐在一边的富裕婆罗门对世尊说}。说什么?即「\textbf{现今还有……}」等等。其义全都明了,仅此中的\textbf{在婆罗门的婆罗门法上},为在剔除了时空等法的婆罗门法上。
\item \textbf{那么,众婆罗门……},即因为你们向我请求,所以,众婆罗门!\textbf{谛听}!请倾耳!\textbf{善加作意}!如理作意!同样,以加行的清净而谛听,以意乐的清净而善加作意,以不散乱而谛听,以策励而善加作意,当以如是等的方法而知这些语句先前未说的旨趣\footnote{先前未说的旨趣:应是指\textbf{贱民经}中曾作的解释而言。}。
\item 于是,领受着世尊所说之语,\textbf{「如是,先生!」这些富裕婆罗门答世尊},当面听闻世尊之语。或者,他们便许诺,即是说他们于「谛听!善加作意」所说之义,以欲作便同意。于是,\textbf{世尊}对如是许诺的彼等\textbf{说}。说什么?即「从前的仙人们」等等。\end{enumerate}

\subsection\*{\textbf{287} {\footnotesize 〔PTS 284〕}}

\textbf{从前的仙人们是自制的苦行者,\\}
\textbf{舍弃了种种五欲,践行自己的义利。}

“Isayo pubbakā āsuṃ, saññatattā tapassino;\\
pañca kāmaguṇe hitvā, attadattham acārisuṃ. %\hfill\textcolor{gray}{\footnotesize 1}

\begin{enumerate}\item 这里,首先在初颂中,\textbf{自制},即以戒的制御而抑制心。\textbf{苦行者},即从事根律仪的苦行者。\textbf{践行自己的义利},即做研究颂诗、修习梵住等自己的义利。余皆自明。\end{enumerate}

\subsection\*{\textbf{288} {\footnotesize 〔PTS 285〕}}

\textbf{婆罗门没有牲畜,没有货币,没有谷物,\\}
\textbf{他们以诵经为财产和谷物,守护梵天的伏藏。}

Na pasū brāhmaṇān’āsuṃ, na hiraññaṃ na dhāniyaṃ;\\
sajjhāyadhanadhaññāsuṃ, brahmaṃ nidhim apālayuṃ. %\hfill\textcolor{gray}{\footnotesize 2}

\begin{enumerate}\item 在第二颂等中,其略注为:\textbf{婆罗门没有牲畜},即古昔的婆罗门没有牲畜,他们不拥有牲畜。\textbf{没有货币,没有谷物},即婆罗门没有货币,甚至连一个硬币也没有,同样,他们也没有米、稻、大麦、小麦等分为主粮、蔬菜的谷物。他们丢弃金银而成无贮藏者,唯\textbf{以诵经为财产和谷物},具足自身被称为研究颂诗的财产和谷物。由最胜及伴随故,慈等住被称为梵天的伏藏,他们总是以从事此修习来\textbf{守护梵天的伏藏}。\end{enumerate}

\subsection\*{\textbf{289} {\footnotesize 〔PTS 286〕}}

\textbf{那为他们所制的、放在门前的食物,\\}
\textbf{他们认为这应当布施给寻求信制者。}

Yaṃ nesaṃ pakataṃ āsi, dvārabhattaṃ upaṭṭhitaṃ;\\
saddhāpakatam esānaṃ, dātave tad amaññisuṃ. %\hfill\textcolor{gray}{\footnotesize 3}

\begin{enumerate}\item 对如是而住者,\textbf{那为他们所制的},即为这些婆罗门所做的。\textbf{放在门前的食物},即为彼彼施主以「我们将布施给婆罗门」而准备好,在各自的家门前放置的食物。\textbf{信制},即因信而制,即是说信施。\textbf{他们认为这},即施主等人认为,这准备好后放在门前的食物,应当布施给那些寻求信施的婆罗门,而非其他。因为他们不需要别的,而仅满足于衣食。\end{enumerate}

\subsection\*{\textbf{290} {\footnotesize 〔PTS 287〕}}

\textbf{以多彩的衣服,以及卧具、住所,\\}
\textbf{众繁荣的地方和王国礼敬这些婆罗门。}

Nānārattehi vatthehi, sayaneh’āvasathehi ca;\\
phītā janapadā raṭṭhā, te namassiṃsu brāhmaṇe. %\hfill\textcolor{gray}{\footnotesize 4}

\begin{enumerate}\item \textbf{以多彩的},即以多种染色所染的\textbf{衣服},以彩色铺盖所敷的\textbf{卧具},以一层、两层等楼阁的高贵\textbf{住所},即以如此的资助,\textbf{众繁荣的地方和王国},作为一一地区的地方和某某全国,以「礼敬婆罗门」来早晚礼敬婆罗门,如对天人般。\end{enumerate}

\subsection\*{\textbf{291} {\footnotesize 〔PTS 288〕}}

\textbf{婆罗门不可侵犯,不可战胜,受法守护,\\}
\textbf{没有人能以任何方式在家门前拒绝他们。}

Avajjhā brāhmaṇā āsuṃ, ajeyyā dhammarakkhitā;\\
na ne koci nivāresi, kuladvāresu sabbaso. %\hfill\textcolor{gray}{\footnotesize 5}

\begin{enumerate}\item 他们如是为世间所礼敬,\textbf{婆罗门}便\textbf{不可侵犯},且不仅不可侵犯,还\textbf{不可战胜},由不可征服以行伤害,便不可战胜。什么原因?\textbf{受法守护},因为受法守护。他们守护高贵的五戒之法,而\begin{quoting}法必守护法行者。(本生第 10:102 颂)\end{quoting}故受法守护而成不可侵犯、不可战胜之意。
\item \textbf{没有人能以任何方式在家}的内、外\textbf{门}等一切门\textbf{前拒绝他们},因为人们于这些共许为可喜、具足高贵的戒的婆罗门,如于父母般极度信赖,所以,没有人以「此处你不应进入」而拒绝。\end{enumerate}

\subsection\*{\textbf{292} {\footnotesize 〔PTS 289〕}}

\textbf{他们行持四十八年的童贞梵行,\\}
\textbf{在过去,婆罗门践行对明行的遍求。}

Aṭṭhacattālīsaṃ vassāni, komārabrahmacariyaṃ cariṃsu te;\\
vijjācaraṇapariyeṭṭhiṃ, acaruṃ brāhmaṇā pure. %\hfill\textcolor{gray}{\footnotesize 6}

\begin{enumerate}\item 如是受法守护、在家门前无拒而行者,从孩童之状开始,\textbf{他们行持四十八年的童贞梵行},当知其意为:连旃陀罗婆罗门\footnote{旃陀罗婆罗门、同梵婆罗门等五种婆罗门:见\textbf{增支部}第 5:192 经头那经。据菩提比丘注 1134,\textbf{同梵婆罗门}在完成学业后出家,修习四梵住,\textbf{同天婆罗门}娶婆罗门女子为妻,坚持行乞,在养育子嗣后出家,修习四禅,\textbf{界内婆罗门}娶婆罗门女子为妻,置办产业,生儿育女,而不出家,\textbf{破界婆罗门}与任一种姓女子交合,沉溺爱欲与繁衍,\textbf{旃陀罗婆罗门}与任一种姓女子交合,包括贱民,以任何——包括不适于婆罗门的——工作活命。}都如此,遑论同梵(婆罗门)等?如是行梵行的\textbf{婆罗门,在过去践行对明行的遍求},而非无梵行者。这里,遍求明,即研究颂诗,如说:\begin{quoting}他学习着颂诗,行持四十八年的童贞梵行。(增支部第 5:192 经)\end{quoting}遍求行,即守护戒。文本也作「去遍求明行 \textit{vijjācaraṇa-pariyeṭṭhum}」,即行遍求明行之义。\end{enumerate}

\subsection\*{\textbf{293} {\footnotesize 〔PTS 290〕}}

\textbf{婆罗门不去别处,他们也不买妻子,\\}
\textbf{唯以相爱结合,他们乐于同居。}

Na brāhmaṇā aññam agamuṃ, na pi bhariyaṃ kiṇiṃsu te;\\
sampiyen’eva saṃvāsaṃ, saṅgantvā samarocayuṃ. %\hfill\textcolor{gray}{\footnotesize 7}

\begin{enumerate}\item 且行了上述时间的梵行后,此后,营事家居的\textbf{婆罗门不去别处},不去刹帝利或吠舍等中的某处(娶妻),意指同天或界内(婆罗门)。同样,\textbf{他们也不}给予百千而\textbf{买妻子},如现今有些人在买,因为他们如法地寻求妻妾。
\item 如何?婆罗门在行持四十八年的梵行后,彷徨于乞求女子:「我已经历四十八年的梵行,如果有适龄的女孩,请给予我!」随后,若人有适龄的女孩,他便装扮她,带她出去,将水洒在站在门前的婆罗门手上,说了「婆罗门!我把她给你做妻子去养育」后即给予。
\item 那为什么他们在行了如此长久的梵行后还要寻求妻妾,而不成终生梵行者呢?以邪见。因为他们持如是之见:若不生孩子,便断绝了家族的世系,由此而堕地狱。据说,有四者害怕不应害怕之事:蚯蚓、松鸦、杓鹬、婆罗门。据说,蚯蚓由于害怕大地灭尽而饮食适量,不吃过多泥土,松鸦鸟由于害怕天空坠落而仰卧在蛋上,杓鹬鸟由于害怕地震而不敢舒适地用脚踩踏地面,婆罗门由于害怕家族世系的断绝而寻求妻妾。且说:\begin{quoting}蚯蚓与松鸦、杓鹬、婆罗门法者,\\这四者愚昧,害怕不可怕者。\end{quoting}
\item 如是如法地寻求妻妾已,\textbf{唯以相爱结合,他们乐于同居},唯以相爱、唯以彼此情爱,身心交融、接触、交际而乐于同居,即是说非由不爱、非由指责。\end{enumerate}

\subsection\*{\textbf{294} {\footnotesize 〔PTS 291〕}}

\textbf{除了这时期,直到月经结束,\\}
\textbf{在此期间,婆罗门不行交媾。}

Aññatra tamhā samayā, utuveramaṇiṃ pati;\\
antarā methunaṃ dhammaṃ, nāssu gacchanti brāhmaṇā. %\hfill\textcolor{gray}{\footnotesize 8}

\begin{enumerate}\item 且即便如是以相爱而结合,还应「除了……」。即在排卵期中,婆罗门尼当与婆罗门亲近,\textbf{除了这时期}、除去这时期,对已离排卵、\textbf{直到月经结束}的妻子,\textbf{在}直到这时期再来的\textbf{此期间}。\footnote{义注在此尚有:\textbf{nāssu gacchantī} ti n’eva gacchanti,这里未译出。Norman 说 assu < sma,能使现在时表达过去的意义,考虑到此经其它颂中的动词均为不定过去时,唯此处的 gacchanti 作现在时,当是。} \textbf{婆罗门},即指同天和界内(婆罗门)。\end{enumerate}

\subsection\*{\textbf{295} {\footnotesize 〔PTS 292〕}}

\textbf{梵行、戒、诚实、温柔、苦行、\\}
\textbf{调柔、不害及忍辱,为人称颂。}

Brahmacariyañ ca sīlañ ca, ajjavaṃ maddavaṃ tapaṃ;\\
soraccaṃ avihiṃsañ ca, khantiñ cāpi avaṇṇayuṃ. %\hfill\textcolor{gray}{\footnotesize 9}

\begin{enumerate}\item 而一般来说,所有人都称颂「梵行……」。这里,\textbf{梵行},即戒离淫欲。\textbf{戒},即其余四学处。\textbf{诚实},即正直状,意即不狡诈、不欺瞒。\textbf{温柔},即柔和状,意即不坚硬、不傲慢。\textbf{苦行},即根律仪。\textbf{调柔},即习性快乐、行事不违逆。\textbf{不害},即生性不以手等伤害,具悲悯相。\textbf{忍辱},即承受之忍辱。他们\textbf{称颂}上述这些品德,即便不能完全发起行道,也知见其价值,以言语称颂、赞赏。\end{enumerate}

\subsection\*{\textbf{296} {\footnotesize 〔PTS 293〕}}

\textbf{其中的最上者是努力勇猛的梵,\\}
\textbf{他即便在梦中也不行交媾。}

Yo nesaṃ paramo āsi, brahmā daḷhaparakkamo;\\
sa vāpi methunaṃ dhammaṃ, supinante pi nāgamā. %\hfill\textcolor{gray}{\footnotesize 10}

\begin{enumerate}\item 且在如是称颂者中,「其中的……交媾」。这些婆罗门中的\textbf{最上者是梵},即名为同梵的最上婆罗门,由具足努力、勇猛而\textbf{努力勇猛}。\textbf{vā} 字是阐明义,以此阐明他就是「这样的婆罗门」。\end{enumerate}

\subsection\*{\textbf{297} {\footnotesize 〔PTS 294〕}}

\textbf{效仿他的行仪,于此,一些生性有智者\\}
\textbf{也称颂梵行、戒及忍辱。}

Tassa vattam anusikkhantā, idh’eke viññujātikā;\\
brahmacariyañ ca sīlañ ca, khantiñ cāpi avaṇṇayuṃ. %\hfill\textcolor{gray}{\footnotesize 11}

\begin{enumerate}\item 随后,「效仿……忍辱」。此颂以初、后来说明第九颂所说的品德,阐明同天婆罗门。因为这些生性有智的智者效仿彼同梵婆罗门的行仪而出家、修习禅那,且他们唯以行道来称颂这些梵行等的品德。这一切婆罗门当以(增支部)五集·头那经\footnote{头那经 \textit{Doṇasutta}:旧译见\textbf{中阿含经}·梵志品·头那经。}中所说的方法而知。\end{enumerate}

\subsection\*{\textbf{298} {\footnotesize 〔PTS 295〕}}

\textbf{如法地乞求稻米、卧具、衣服,以及酥油,\\}
\textbf{收集已,然后举行祭祀。}

Taṇḍulaṃ sayanaṃ vatthaṃ, sappitelañ ca yāciya;\\
dhammena samodhānetvā, tato yaññam akappayuṃ. %\hfill\textcolor{gray}{\footnotesize 12}

\begin{enumerate}\item 现在为说界内婆罗门而说此颂。若这些界内婆罗门欲举行祭祀,因不能接受生谷而乞求种种\textbf{稻米},以及床椅等的\textbf{卧具},麻布等的\textbf{衣服},奶牛的\textbf{酥}和芝麻的\textbf{油}。\textbf{如法地乞求},即如「指定已,圣者们便站立着,这是圣者们的乞求」所说的称为「指定站立」的如法乞求。\end{enumerate}

\subsection\*{\textbf{299} {\footnotesize 〔PTS 296〕}}

\textbf{在备好的祭祀中,他们不杀牛,\\}
\textbf{好比母亲、父亲、兄弟,或任何其他亲戚,\\}
\textbf{牛是我们最好的朋友,用它们可以制药。}

Upaṭṭhitasmiṃ yaññasmiṃ, nāssu gāvo haniṃsu te;\\
yathā mātā pitā bhātā, aññe vā pi ca ñātakā;\\
gāvo no paramā mittā, yāsu jāyanti osadhā. %\hfill\textcolor{gray}{\footnotesize 13}

\begin{enumerate}\item \textbf{他们不杀牛},当知这里是以牛来说一切生物。为何缘由不杀?由梵行等的功德相应故。\textbf{用它们可以制药},用它们可以制针对胆汁等(病)的五种奶制品的药。\end{enumerate}

\begin{quoting}PTS 本将首行归入上颂。\end{quoting}

\subsection\*{\textbf{300} {\footnotesize 〔PTS 297〕}}

\textbf{他们给予食物,给予力量,还给予容貌,给予快乐,\\}
\textbf{了知了这些缘由,他们不杀牛。}

Annadā baladā c’etā, vaṇṇadā sukhadā tathā;\\
etam atthavasaṃ ñatvā, nāssu gāvo haniṃsu te. %\hfill\textcolor{gray}{\footnotesize 14}

\subsection\*{\textbf{301} {\footnotesize 〔PTS 298〕}}

\textbf{曼妙、硕大、貌美、享有名声的\\}
\textbf{婆罗门凭着那些法,热心于应作和不应作,\\}
\textbf{只要它们在世间存在,人类便快乐地增长。}

Sukhumālā mahākāyā, vaṇṇavanto yasassino;\\
brāhmaṇā sehi dhammehi, kiccākiccesu ussukā;\\
yāva loke avattiṃsu, sukham edhitth’ayaṃ pajā. %\hfill\textcolor{gray}{\footnotesize 15}

\begin{enumerate}\item \textbf{那些法},即自身的作持 \textit{sakehi cārittehi}。\textbf{人类},即有情世间。\end{enumerate}

\begin{quoting}案,\textbf{在世间存在}的当指婆罗门法,但义注似认为是这些婆罗门。\end{quoting}

\subsection\*{\textbf{302} {\footnotesize 〔PTS 299〕}}

\textbf{颠倒生于其间,自从渐渐见到了\\}
\textbf{国王的华丽,以及盛装的女人们,}

Tesaṃ āsi vipallāso, disvāna aṇuto aṇuṃ;\\
rājino ca viyākāraṃ, nāriyo samalaṅkatā. %\hfill\textcolor{gray}{\footnotesize 16}

\begin{enumerate}\item 义注说 302~303 是对破界婆罗门说的。
\item \textbf{颠倒},即颠倒想。\textbf{微少为微少} \textit{aṇuto aṇuṃ},即由低劣之义、有限之义、寡味之义,由微少的(五)欲功德之所生相较于禅那、沙门、涅槃之乐差距甚远 \textit{saṅkhyam pi anupagamanena} 故,得见欲乐为微少,或者,相较于出世间乐,微少的由自身所得的世间等持之乐为微少,更以其少量,得见欲乐为少量。\textbf{华丽},即成就。\end{enumerate}

\begin{quoting}案,\textbf{aṇuto aṇuṃ} 意为「渐渐地、点滴地」,Norman 英译作 “little by little”,但义注的意思似按字面解作「微少」,并以世间的欲乐为微少,详见下颂注。\end{quoting}

\subsection\*{\textbf{303} {\footnotesize 〔PTS 300〕}}

\textbf{套有纯种马匹、精致、彩色华盖的车辆,\\}
\textbf{以及规划好、按部分丈量的住处和居所。}

Rathe c’ājaññasaṃyutte, sukate cittasibbane;\\
nivesane nivese ca, vibhatte bhāgaso mite. %\hfill\textcolor{gray}{\footnotesize 17}

\begin{enumerate}\item \textbf{精致},即以木作、铜作而善加制作。\textbf{彩色华盖},即以狮皮等装饰的彩色华盖。\textbf{住处} \textit{nivesane},即房子的基地。\textbf{居所} \textit{nivese},即于此建造的房子。\textbf{规划好},即按长宽来规划好。\textbf{按部分丈量},即分成庭院、门、殿堂、重阁等各各部分来丈量。说的是什么?对于这些婆罗门,得见国王的华丽、盛装的女人们、如说的车辆、住处和居所等(过去)以为微少的欲乐,于这些苦处转起「乐」,而生起颠倒于过去转起之出离想的颠倒想。\end{enumerate}

\begin{quoting}案,\textbf{马匹} \textit{assa} 据义注补足。\end{quoting}

\subsection\*{\textbf{304} {\footnotesize 〔PTS 301〕}}

\textbf{牛群遍布,美女簇拥,\\}
\textbf{婆罗门贪求显赫的人间财富。}

Gomaṇḍala-paribyūḷhaṃ, nārīvaragaṇāyutaṃ;\\
uḷāraṃ mānusaṃ bhogaṃ, abhijjhāyiṃsu brāhmaṇā. %\hfill\textcolor{gray}{\footnotesize 18}

\begin{quoting}案,\textbf{varaṅganā} PED 解释作「美女」,义注说 \textbf{nārīvaragaṇāyutan} \textit{ti varanārīgaṇasaṃyuttaṃ}。\end{quoting}

\subsection\*{\textbf{305} {\footnotesize 〔PTS 302〕}}

\textbf{于是,他们编集了颂诗,然后去往甘蔗王处,\\}
\textbf{「你有丰厚的财产和谷物,\\}
\textbf{「祭祀吧!你的财宝众多,祭祀吧!你的财产众多。」}

Te tattha mante ganthetvā, Okkākaṃ tad’upāgamuṃ;\\
“pahūtadhanadhañño si,\\
yajassu bahu te vittaṃ, yajassu bahu te dhanaṃ”. %\hfill\textcolor{gray}{\footnotesize 19}

\begin{enumerate}\item \textbf{于是},即说因为他们贪求财产。\textbf{tadupāgamun} \textit{ti tadā upāgamuṃ}. \textbf{你有丰厚的财产和谷物},即你来世将有丰厚的财产和谷物的意思。\end{enumerate}

\begin{quoting}案,\textbf{甘蔗王} \textit{Okkāka/Ikṣvāku}。义注说 \textbf{yajassū} \textit{ti yajāhi},这里用的也是中间语态,见耕田婆罗豆婆遮经注。\end{quoting}

\subsection\*{\textbf{306} {\footnotesize 〔PTS 303〕}}

\textbf{随后,被婆罗门说服的国王,车乘之主,\\}
\textbf{马牲、人牲、掷棒祭、娑摩祭、无遮祭,\\}
\textbf{举行完这些祭祀,他赐予婆罗门以财产:}

Tato ca rājā saññatto, brāhmaṇehi rathesabho;\\
assamedhaṃ purisamedhaṃ, sammāpāsaṃ vājapeyyaṃ niraggaḷaṃ;\\
ete yāge yajitvāna, brāhmaṇānam adā dhanaṃ. %\hfill\textcolor{gray}{\footnotesize 20}

\begin{quoting}案,\textbf{掷棒祭} \textit{sammāpāsa/śamyā-prāsa = śamyā-kṣepa}。义注说\textbf{无遮祭} \textit{niraggaḷa} 是马祭的另一种说法,但包括了地祭与人祭。\end{quoting}

\subsection\*{\textbf{307} {\footnotesize 〔PTS 304〕}}

\textbf{牛群、卧具、衣服,盛装的女人们,\\}
\textbf{套有纯种马匹、精致、彩色华盖的车辆,}

Gāvo sayanañ ca vatthañ ca, nāriyo samalaṅkatā;\\
rathe c’ājaññasaṃyutte, sukate cittasibbane. %\hfill\textcolor{gray}{\footnotesize 21}

\subsection\*{\textbf{308} {\footnotesize 〔PTS 305〕}}

\textbf{惬意的住处,按部分善加规划,\\}
\textbf{其中盈满了种种谷物,他赐予婆罗门以财产。}

Nivesanāni rammāni, suvibhattāni bhāgaso;\\
nānādhaññassa pūretvā, brāhmaṇānam adā dhanaṃ. %\hfill\textcolor{gray}{\footnotesize 22}

\subsection\*{\textbf{309} {\footnotesize 〔PTS 306〕}}

\textbf{于是他们获得了财产,乐于积贮,\\}
\textbf{他们陷入欲望,更多的渴爱增长,\\}
\textbf{于是,他们编集了颂诗,再次去往甘蔗王处,}

Te ca tattha dhanaṃ laddhā, sannidhiṃ samarocayuṃ;\\
tesaṃ icchāvatiṇṇānaṃ, bhiyyo taṇhā pavaḍḍhatha;\\
te tattha mante ganthetvā, Okkākaṃ punam upāgamuṃ. %\hfill\textcolor{gray}{\footnotesize 23}

\subsection\*{\textbf{310} {\footnotesize 〔PTS 307〕}}

\textbf{「正如水、土地、货币、财产和谷物,\\}
\textbf{「牛对于人也如是,因为这是生命的必需品,\\}
\textbf{「祭祀吧!你的财宝众多,祭祀吧!你的财产众多。」}

“Yathā āpo ca pathavī ca, hiraññaṃ dhanadhāniyaṃ;\\
evaṃ gāvo manussānaṃ, parikkhāro so hi pāṇinaṃ;\\
yajassu bahu te vittaṃ, yajassu bahu te dhanaṃ”. %\hfill\textcolor{gray}{\footnotesize 24}

\subsection\*{\textbf{311} {\footnotesize 〔PTS 308〕}}

\textbf{随后,被婆罗门说服的国王,车乘之主\\}
\textbf{在祭祀中杀了数百千头牛。}

Tato ca rājā saññatto, brāhmaṇehi rathesabho;\\
nekā satasahassiyo, gāvo yaññe aghātayi. %\hfill\textcolor{gray}{\footnotesize 25}

\subsection\*{\textbf{312} {\footnotesize 〔PTS 309〕}}

\textbf{它们没有用足、用角伤害任何人,\\}
\textbf{牛如同山羊,温顺,产成桶的奶,\\}
\textbf{国王捉住角,用刀杀了它们。}

Na pādā na visāṇena, nāssu hiṃsanti kenaci;\\
gāvo eḷakasamānā, soratā kumbhadūhanā;\\
tā visāṇe gahetvāna, rājā satthena ghātayi. %\hfill\textcolor{gray}{\footnotesize 26}

\subsection\*{\textbf{313} {\footnotesize 〔PTS 310〕}}

\textbf{随后,诸天、诸父、因陀罗、阿修罗、罗刹\\}
\textbf{呼喊着「非法」,因为刀落在牛上。}

Tato devā pitaro ca, Indo asura-rakkhasā;\\
“adhammo” iti pakkanduṃ, yaṃ satthaṃ nipatī gave. %\hfill\textcolor{gray}{\footnotesize 27}

\begin{enumerate}\item \textbf{诸父},即在婆罗门中得名的梵天。\end{enumerate}

\subsection\*{\textbf{314} {\footnotesize 〔PTS 311〕}}

\textbf{过去有三种疾病:欲望、饥饿、衰老,\\}
\textbf{从杀戮牲畜以后,乃有九十八种到来。}

Tayo rogā pure āsuṃ, icchā anasanaṃ jarā;\\
pasūnañ ca samārambhā, aṭṭhānavuti-m-āgamuṃ. %\hfill\textcolor{gray}{\footnotesize 28}

\subsection\*{\textbf{315} {\footnotesize 〔PTS 312〕}}

\textbf{这惩罚的非法成为古代的传统,\\}
\textbf{无辜者遭到杀害,祭司们违背了法。}

Eso adhammo daṇḍānaṃ, okkanto purāṇo ahu;\\
adūsikāyo haññanti, dhammā dhaṃsanti yājakā. %\hfill\textcolor{gray}{\footnotesize 29}

\subsection\*{\textbf{316} {\footnotesize 〔PTS 313〕}}

\textbf{如是,这古老卑贱的法受智者谴责,\\}
\textbf{当看到这样的事,人们便谴责祭司。}

Evam eso aṇudhammo, porāṇo viññugarahito;\\
yattha edisakaṃ passati, yājakaṃ garahatī jano. %\hfill\textcolor{gray}{\footnotesize 30}

\subsection\*{\textbf{317} {\footnotesize 〔PTS 314〕}}

\textbf{当法如是衰败时,首陀罗与吠舍分离,\\}
\textbf{刹帝利各各分离,妻子轻视丈夫。}

Evaṃ dhamme viyāpanne, vibhinnā sudda-vessikā;\\
puthū vibhinnā khattiyā, patiṃ bhariyā’vamaññatha. %\hfill\textcolor{gray}{\footnotesize 31}

\subsection\*{\textbf{318} {\footnotesize 〔PTS 315〕}}

\textbf{刹帝利、梵天的眷属,与其他受种姓守护者,\\}
\textbf{摈弃了出身论,沦于爱欲的控制。}

Khattiyā brahmabandhū ca, ye c’aññe gottarakkhitā;\\
jātivādaṃ nirākatvā, kāmānaṃ vasam anvagun” ti. %\hfill\textcolor{gray}{\footnotesize 32}

\textbf{如是说已,众富裕婆罗门对世尊说:「希有啊,乔达摩君!……愿乔达摩君能受持我们为优婆塞,从今起皈依,直至命终。」}

Evaṃ vutte, te brāhmaṇamahāsālā Bhagavantaṃ etad avocuṃ: “abhikkantaṃ, bho Gotama…pe… upāsake no bhavaṃ Gotamo dhāretu ajjatagge pāṇupete saraṇaṃ gate” ti.

\begin{center}\vspace{1em}婆罗门法者经第七\\Brāhmaṇadhammikasuttaṃ sattamaṃ.\end{center}