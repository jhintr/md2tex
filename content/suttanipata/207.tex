\section{婆罗门法经}

\begin{center}Brāhmaṇadhammika Sutta\end{center}\vspace{1em}

\textbf{如是我闻\footnote{此经旧译见中阿含经·梵志品·梵波罗延经。}。一时世尊住舍卫国祇树给孤独园。尔时,众多㤭萨罗的富裕婆罗门衰老、年迈、高龄、迟暮、岁月已逝,往世尊处走去,走到后,问候了世尊,彼此寒暄已,坐在一边。}

Evaṃ me sutaṃ— ekaṃ samayaṃ Bhagavā Sāvatthiyaṃ viharati Jetavane Anāthapiṇḍikassa ārāme. Atha kho sambahulā Kosalakā brāhmaṇamahāsālā jiṇṇā vuḍḍhā mahallakā addhagatā vayo anuppattā yena Bhagavā ten’upasaṅkamiṃsu, upasaṅkamitvā Bhagavatā saddhiṃ sammodiṃsu, sammodanīyaṃ kathaṃ sāraṇīyaṃ vītisāretvā ekamantaṃ nisīdiṃsu.

\begin{enumerate}\item 缘起为何?这在其因缘中如「尔时,众多……」等方法所述。这里,\textbf{众多},即许多、非一。\textbf{㤭萨罗的},即㤭萨罗国的居民。\textbf{富裕婆罗门},即以出生为婆罗门,以富裕性为富裕。据说,那些经储蓄而存有八十俱胝之数的财产者,被称为富裕婆罗门,而他们便是如此,因此说是「富裕婆罗门」。\textbf{衰老},即老弱,因老而成齿落等状。\textbf{年迈},即肢体已至增长的极限。\textbf{高龄},即是说出生甚久。\textbf{迟暮},即旅途已达,意为经历了两三位国王的继位。\textbf{岁月已逝},即已到达最后的年龄。
\item 复次,当知此中亦可如是连接:\textbf{衰老},即古老,即是说长久保持家族传统。\textbf{年迈},即与戒正行等德的增长相应。\textbf{高龄},即具足财产,巨财、巨富。\textbf{迟暮},即践行于道,不违犯婆罗门的禁行等的禁忌而行。\textbf{岁月已逝},即已至生老之相的最终年龄。其余于此自明。
\item \textbf{问候了世尊},即询问可忍耐等,彼此转起相同的喜悦。且以此「乔达摩君尚可忍耐?尚可维持?少病少恼,有力轻起,安乐住否」等问候的谈论,由生起被称为喜悦的庆慰且值得问候为「应问候」,以义文之甜美,虽极长时亦值得忆念、值得常常令起,且由应忆念之相为「应忆念」,以及由听闻之乐为应问候,由随念之乐为应忆念,同样,由文句遍净为应问候,由语义遍净为应忆念,如是以多种理法\textbf{彼此寒暄已}\footnote{彼此寒暄已:这是意译,字面的直译即上文解释的「交换了应问候、应忆念的谈论」。},即令终结、完成已,欲问所为前来之义,便\textbf{坐在一边}。此以\begin{quoting}不在前,不在后,亦不近、不远,\\非于沟渠、逆风,亦非低处高处。\end{quoting}等方法,已在吉祥经注中说明。\end{enumerate}

\textbf{这些坐在一边的富裕婆罗门对世尊说:「乔达摩君!现今还有婆罗门在古昔婆罗门的婆罗门法上吗?」「众婆罗门!现今没有婆罗门在古昔婆罗门的婆罗门法上。」「善哉!若不麻烦乔达摩君的话,请乔达摩君对我们说说古昔婆罗门的婆罗门法!」「那么,众婆罗门!谛听!善加作意!我将说法。」「如是,先生!」这些富裕婆罗门答世尊。世尊说:}

Ekamantaṃ nisinnā kho te brāhmaṇamahāsālā Bhagavantaṃ etad avocuṃ: “sandissanti nu kho, bho Gotama, etarahi brāhmaṇā porāṇānaṃ brāhmaṇānaṃ brāhmaṇadhamme” ti? “Na kho, brāhmaṇā, sandissanti etarahi brāhmaṇā porāṇānaṃ brāhmaṇānaṃ brāhmaṇadhamme” ti. “Sādhu no bhavaṃ Gotamo porāṇānaṃ brāhmaṇānaṃ brāhmaṇadhammaṃ bhāsatu, sace bhoto Gotamassa agarū” ti. “Tena hi, brāhmaṇā, suṇātha, sādhukaṃ manasi karotha, bhāsissāmī” ti. “Evaṃ, bho” ti kho te brāhmaṇamahāsālā Bhagavato paccassosuṃ. Bhagavā etad avoca:

\begin{enumerate}\item 如是,\textbf{这些坐在一边的富裕婆罗门对世尊说}。说什么?即「\textbf{现今还有……}」等等。其义全都明了,仅此中的\textbf{在婆罗门的婆罗门法上},为在剔除了时空等法的婆罗门法上。
\item \textbf{那么,众婆罗门……},即因为你们向我请求,所以,众婆罗门!\textbf{谛听}!请倾耳!\textbf{善加作意}!如理作意!同样,以加行的清净而谛听,以意乐的清净而善加作意,以不散乱而谛听,以策励而善加作意,当以如是等的方法而知这些语句先前未说的旨趣\footnote{先前未说的旨趣:应是指\textbf{贱民经}中曾作的解释而言。}。
\item 于是,领受着世尊所说之语,\textbf{「如是,先生!」这些富裕婆罗门答世尊},当面听闻世尊之语。或者,他们便许诺,即是说他们于「谛听!善加作意」所说之义,以欲作便同意。于是,\textbf{世尊}对如是许诺的彼等\textbf{说}。说什么?即「从前的仙人们」等等。\end{enumerate}

\subsection\*{\textbf{287} {\footnotesize 〔PTS 284〕}}

\textbf{从前的仙人们是自制的苦行者,\\}
\textbf{舍弃了种种五欲,践行自己的义利。}

“Isayo pubbakā āsuṃ, saññatattā tapassino;\\
pañca kāmaguṇe hitvā, attadattham acārisuṃ. %\hfill\textcolor{gray}{\footnotesize 1}

\begin{enumerate}\item 这里,首先在初颂中,\textbf{自制},即以戒的制御而抑制心。\textbf{苦行者},即从事根律仪的苦行者。\textbf{践行自己的义利},即做研究颂诗、修习梵住等自己的义利。余皆自明。\end{enumerate}

\subsection\*{\textbf{288} {\footnotesize 〔PTS 285〕}}

\textbf{婆罗门没有牲畜,没有货币,没有谷物,\\}
\textbf{他们以诵经为财产和谷物,守护梵天的伏藏。}

Na pasū brāhmaṇān’āsuṃ, na hiraññaṃ na dhāniyaṃ;\\
sajjhāyadhanadhaññāsuṃ, brahmaṃ nidhim apālayuṃ. %\hfill\textcolor{gray}{\footnotesize 2}

\begin{enumerate}\item 在第二颂等中,其略注为:\textbf{婆罗门没有牲畜},即古昔的婆罗门没有牲畜,他们不拥有牲畜。\textbf{没有货币,没有谷物},即婆罗门没有货币,甚至连一个硬币也没有,同样,他们也没有米、稻、大麦、小麦等分为主粮、蔬菜的谷物。他们丢弃金银而成无贮藏者,唯\textbf{以诵经为财产和谷物},具足自身被称为研究颂诗的财产和谷物。由最胜及伴随故,慈等住被称为梵天的伏藏,他们总是以从事此修习来\textbf{守护梵天的伏藏}。\end{enumerate}

\subsection\*{\textbf{289} {\footnotesize 〔PTS 286〕}}

\textbf{那为他们所制的、放在门前的食物,\\}
\textbf{他们认为这应当布施给寻求信制者。}

Yaṃ nesaṃ pakataṃ āsi, dvārabhattaṃ upaṭṭhitaṃ;\\
saddhāpakatam esānaṃ, dātave tad amaññisuṃ. %\hfill\textcolor{gray}{\footnotesize 3}

\begin{enumerate}\item 对如是而住者,\textbf{那为他们所制的},即为这些婆罗门所做的。\textbf{放在门前的食物},即为彼彼施主以「我们将布施给婆罗门」而准备好,在各自的家门前放置的食物。\textbf{信制},即因信而制,即是说信施。\textbf{他们认为这},即施主等人认为,这准备好后放在门前的食物,应当布施给那些寻求信施的婆罗门,而非其他。因为他们不需要别的,而仅满足于衣食。\end{enumerate}

\subsection\*{\textbf{290} {\footnotesize 〔PTS 287〕}}

\textbf{以多彩的衣服,以及卧具、住所,\\}
\textbf{众繁荣的地方和王国礼敬这些婆罗门。}

Nānārattehi vatthehi, sayaneh’āvasathehi ca;\\
phītā janapadā raṭṭhā, te namassiṃsu brāhmaṇe. %\hfill\textcolor{gray}{\footnotesize 4}

\begin{enumerate}\item \textbf{以多彩的},即以多种染色所染的\textbf{衣服},以彩色铺盖所敷的\textbf{卧具},以一层、两层等楼阁的高贵\textbf{住所},即以如此的资助,\textbf{众繁荣的地方和王国},作为一一地区的地方和某某全国,以「礼敬婆罗门」来早晚礼敬婆罗门,如对天人般。\end{enumerate}

\subsection\*{\textbf{291} {\footnotesize 〔PTS 288〕}}

\textbf{婆罗门不可侵犯,不可战胜,受法守护,\\}
\textbf{没有人能以任何方式在家门前拒绝他们。}

Avajjhā brāhmaṇā āsuṃ, ajeyyā dhammarakkhitā;\\
na ne koci nivāresi, kuladvāresu sabbaso. %\hfill\textcolor{gray}{\footnotesize 5}

\begin{enumerate}\item 他们如是为世间所礼敬,\textbf{婆罗门}便\textbf{不可侵犯},且不仅不可侵犯,还\textbf{不可战胜},由不可征服以行伤害,便不可战胜。什么原因?\textbf{受法守护},因为受法守护。他们守护高贵的五戒之法,而\begin{quoting}法必守护法行者。(本生第 10:102 颂)\end{quoting}故受法守护而成不可侵犯、不可战胜之意。
\item \textbf{没有人能以任何方式在家}的内、外\textbf{门}等一切门\textbf{前拒绝他们},因为人们于这些共许为可喜、具足高贵的戒的婆罗门,如于父母般极度信赖,所以,没有人以「此处你不应进入」而拒绝。\end{enumerate}

\subsection\*{\textbf{292} {\footnotesize 〔PTS 289〕}}

\textbf{他们行持四十八年的童贞梵行,\\}
\textbf{在过去,婆罗门践行对明行的遍求。}

Aṭṭhacattālīsaṃ vassāni, komārabrahmacariyaṃ cariṃsu te;\\
vijjācaraṇapariyeṭṭhiṃ, acaruṃ brāhmaṇā pure. %\hfill\textcolor{gray}{\footnotesize 6}

\begin{enumerate}\item 如是受法守护、在家门前无拒而行者,从孩童之状开始,\textbf{他们行持四十八年的童贞梵行},当知其意为:连旃陀罗婆罗门\footnote{旃陀罗婆罗门、同梵婆罗门等五种婆罗门:见\textbf{增支部}第 5:192 经头那经 \textit{Doṇasutta}。据菩提比丘注 1134,\textbf{同梵婆罗门}在完成学业后出家,修习四梵住,\textbf{同天婆罗门}娶婆罗门女子为妻,坚持行乞,在养育子嗣后出家,修习四禅,\textbf{界内婆罗门}娶婆罗门女子为妻,置办产业,生儿育女,而不出家,\textbf{破界婆罗门}与任一种姓女子交合,沉溺爱欲与繁衍,\textbf{旃陀罗婆罗门}与任一种姓女子交合,包括贱民,以任何——包括不适于婆罗门的——工作活命。}都如此,遑论同梵(婆罗门)等?如是行梵行的\textbf{婆罗门,在过去践行对明行的遍求},而非无梵行者。这里,遍求明,即研究颂诗,如说:\begin{quoting}他学习着颂诗,行持四十八年的童贞梵行。(增支部第 5:192 经)\end{quoting}遍求行,即守护戒。文本也作「去遍求明行 \textit{vijjācaraṇa-pariyeṭṭhum}」,即行遍求明行之义。\end{enumerate}

\subsection\*{\textbf{293} {\footnotesize 〔PTS 290〕}}

\textbf{婆罗门不去别处,他们也不买妻子,\\}
\textbf{唯以相爱结合,他们乐于同居。}

Na brāhmaṇā aññam agamuṃ, na pi bhariyaṃ kiṇiṃsu te;\\
sampiyen’eva saṃvāsaṃ, saṅgantvā samarocayuṃ. %\hfill\textcolor{gray}{\footnotesize 7}

\begin{enumerate}\item 且行了上述时间的梵行后,此后,营事家居的\textbf{婆罗门不去别处},不去刹帝利或吠舍等中的某处(娶妻),意指同天或界内(婆罗门)。同样,\textbf{他们也不}给予百千而\textbf{买妻子},如现今有些人在买,因为他们如法地寻求妻妾。
\item 如何?婆罗门在行持四十八年的梵行后,彷徨于乞求女子:「我已经历四十八年的梵行,如果有适龄的女孩,请给予我!」随后,若人有适龄的女孩,他便装扮她,带她出去,将水洒在站在门前的婆罗门手上,说了「婆罗门!我把她给你做妻子去养育」后即给予。
\item 那为什么他们在行了如此长久的梵行后还要寻求妻妾,而不成终生梵行者呢?以邪见。因为他们持如是之见:若不生孩子,便断绝了家族的世系,由此而堕地狱。据说,有四者害怕不应害怕之事:蚯蚓、松鸦、杓鹬、婆罗门。据说,蚯蚓由于害怕大地灭尽而饮食适量,不吃过多泥土,松鸦鸟由于害怕天空坠落而仰卧在蛋上,杓鹬鸟由于害怕地震而不敢舒适地用脚踩踏地面,婆罗门由于害怕家族世系的断绝而寻求妻妾。且说:\begin{quoting}蚯蚓与松鸦、杓鹬、婆罗门法者,\\这四者愚昧,害怕不可怕者。\end{quoting}
\item 如是如法地寻求妻妾已,\textbf{唯以相爱结合,他们乐于同居},唯以相爱、唯以彼此情爱,身心交融、接触、交际而乐于同居,即是说非由不爱、非由指责。\end{enumerate}

\subsection\*{\textbf{294} {\footnotesize 〔PTS 291〕}}

\textbf{除了这时期,直到月经结束,\\}
\textbf{在此期间,婆罗门不行交媾。}

Aññatra tamhā samayā, utuveramaṇiṃ pati;\\
antarā methunaṃ dhammaṃ, nāssu gacchanti brāhmaṇā. %\hfill\textcolor{gray}{\footnotesize 8}

\begin{enumerate}\item 且即便如是以相爱而结合,还应「除了……」。即在排卵期中,婆罗门尼当与婆罗门亲近,\textbf{除了这时期}、除去这时期,对已离排卵、\textbf{直到月经结束}的妻子,\textbf{在}直到这时期再来的\textbf{此期间}。\textbf{交媾},即为了交媾,此业格作为格用。\footnote{义注在此尚有:\textbf{nāssu gacchantī} ti n’eva gacchanti,这里未译。Norman 说 assu < sma,能使现在时表达过去的意义,考虑到此经其它颂中的动词多为不定过去时,当是。} \textbf{婆罗门},即指同天和界内(婆罗门)。\end{enumerate}

\subsection\*{\textbf{295} {\footnotesize 〔PTS 292〕}}

\textbf{梵行、戒、诚实、温柔、苦行、\\}
\textbf{调柔、不害及忍辱,为人称颂。}

Brahmacariyañ ca sīlañ ca, ajjavaṃ maddavaṃ tapaṃ;\\
soraccaṃ avihiṃsañ ca, khantiñ cāpi avaṇṇayuṃ. %\hfill\textcolor{gray}{\footnotesize 9}

\begin{enumerate}\item 而一般来说,所有人都称颂「梵行……」。这里,\textbf{梵行},即戒离淫欲。\textbf{戒},即其余四学处。\textbf{诚实},即正直状,意即不狡诈、不欺瞒。\textbf{温柔},即柔和状,意即不坚硬、不傲慢。\textbf{苦行},即根律仪。\textbf{调柔},即习性快乐、行事不违逆。\textbf{不害},即生性不以手等伤害,具悲悯相。\textbf{忍辱},即承受之忍辱。他们\textbf{称颂}上述这些品德,即便不能完全发起行道,也知见其价值,以言语称颂、赞赏。\end{enumerate}

\subsection\*{\textbf{296} {\footnotesize 〔PTS 293〕}}

\textbf{其中的最上者是努力勇猛的梵,\\}
\textbf{他即便在梦中也不行交媾。}

Yo nesaṃ paramo āsi, brahmā daḷhaparakkamo;\\
sa vāpi methunaṃ dhammaṃ, supinante pi nāgamā. %\hfill\textcolor{gray}{\footnotesize 10}

\begin{enumerate}\item 且在如是称颂者中,「其中的……交媾」。这些婆罗门中的\textbf{最上者是梵},即名为同梵的最上婆罗门,由具足努力、勇猛而\textbf{努力勇猛}。\textbf{vā} 字是阐明义,以此阐明他就是「这样的婆罗门」。\end{enumerate}

\subsection\*{\textbf{297} {\footnotesize 〔PTS 294〕}}

\textbf{效仿其行仪,于此,一些生性有智者\\}
\textbf{也称颂梵行、戒及忍辱。}

Tassa vattam anusikkhantā, idh’eke viññujātikā;\\
brahmacariyañ ca sīlañ ca, khantiñ cāpi avaṇṇayuṃ. %\hfill\textcolor{gray}{\footnotesize 11}

\begin{enumerate}\item 随后,「效仿……忍辱」。此颂以初、后来说明第九颂所说的品德,阐明同天婆罗门。因为这些生性有智的智者效仿彼同梵婆罗门的行仪而出家、修习禅那,且他们以行道来称颂这些梵行等的品德。这一切婆罗门当以(增支部)五集·头那经\footnote{头那经:旧译见中阿含经·梵志品·头那经。}中所说的方法而知。\end{enumerate}

\subsection\*{\textbf{298} {\footnotesize 〔PTS 295〕}}

\textbf{如法乞求米粒、卧具、衣服及酥油,\\}
\textbf{汇集已,随后举行献牲。}

Taṇḍulaṃ sayanaṃ vatthaṃ, sappitelañ ca yāciya;\\
dhammena samodhānetvā, tato yaññam akappayuṃ. %\hfill\textcolor{gray}{\footnotesize 12}

\begin{enumerate}\item 现在,为显明界内婆罗门而说此颂。其义为:其中的界内婆罗门,若欲举行祭祀,因拒绝接受生谷而\textbf{如法乞求}种种\textbf{米粒},以及床椅等的\textbf{卧具}、亚麻等的\textbf{衣服}、奶牛的\textbf{酥}和芝麻的\textbf{油},如\begin{quoting}指定已,圣者们便站立,这是圣者们的乞求。(本生第 7:59 颂)\end{quoting}所说的被称为「指定站立」之法乞求。若欲布施所需者,因此\textbf{汇集}、收集了——文本也作 samudānetvā,语义相同——所施的米粒等,\textbf{随后举行献牲},随后,获得已,他们便行布施。\end{enumerate}

\subsection\*{\textbf{299} {\footnotesize 〔PTS 296〕}}

\textbf{在备好的献牲中,他们并不杀牛,\footnote{PTS 本将此行纳入上一颂。}\\}
\textbf{好比母亲、父亲、兄弟,或其他亲戚,\\}
\textbf{牛是我们最好的朋友,从中产出药物,}

Upaṭṭhitasmiṃ yaññasmiṃ, nāssu gāvo haniṃsu te;\\
yathā mātā pitā bhātā, aññe vā pi ca ñātakā;\\
gāvo no paramā mittā, yāsu jāyanti osadhā. %\hfill\textcolor{gray}{\footnotesize 13}

\begin{enumerate}\item 且在行时,\textbf{在}这\textbf{备好的}被称为布施的\textbf{献牲中,他们并不杀牛},当知此中以牛为首来说一切生类。他们为何缘由不杀?由与梵行等的功德相应故。且特别地,「好比母亲……他们并不杀牛」。这里,\textbf{从中产出药物},即从中产出作为胆汁(病)等之药的五种奶制品\footnote{五种奶制品:见\textbf{有财者经}第 18 颂的注。}。\end{enumerate}

\subsection\*{\textbf{300} {\footnotesize 〔PTS 297〕}}

\textbf{且其给予食、给予力,还给予色、给予乐,\\}
\textbf{了知了这用意,他们并不杀牛。}

Annadā baladā c’etā, vaṇṇadā sukhadā tathā;\\
etam atthavasaṃ ñatvā, nāssu gāvo haniṃsu te. %\hfill\textcolor{gray}{\footnotesize 14}

\begin{enumerate}\item 在「给予食」等中,因为对受用五种奶制品者,饥饿平息、力量增长、肤色明净、身心的快乐生起,所以当知其给予食、给予力、给予色、给予乐。其余于此自明。\end{enumerate}

\subsection\*{\textbf{301} {\footnotesize 〔PTS 298〕}}

\textbf{曼妙、硕大、美貌、享有名声的\\}
\textbf{婆罗门凭自身的法,热心于应作、不应作,\\}
\textbf{只要他们在世间转起,这人类便增长快乐。}

Sukhumālā mahākāyā, vaṇṇavanto yasassino;\\
brāhmaṇā sehi dhammehi, kiccākiccesu ussukā;\\
yāva loke avattiṃsu, sukham edhitth’ayaṃ pajā. %\hfill\textcolor{gray}{\footnotesize 15}

\begin{enumerate}\item 如是,他们在献牲中不杀牛,身体为福德之威力所摄受,「曼妙……这人类便增长快乐」。这里,以柔嫩的手足等而\textbf{曼妙},以高大、宽阔而\textbf{硕大},以肤色金黄及身材匀称而\textbf{美貌},以利养、眷属的成就而\textbf{享有名声}。\textbf{自身的法},即自身的作持。\textbf{热心于应作、不应作},即于「此应作」之应作、「此不应作」之不应作投入热情之义。
\item 如是,这些古昔婆罗门既已如此可观、明净、为世间最上之应供,\textbf{只要他们}以此行道\textbf{在世间转起},便无有灾难、怖畏、祸害,得以\textbf{增长}种种品类的\textbf{快乐},或快乐地增长、快乐地得至增长。\textbf{这人类},即指有情世间。\end{enumerate}

\subsection\*{\textbf{302} {\footnotesize 〔PTS 299〕}}

\textbf{他们生起颠倒,自从渐渐见到了\\}
\textbf{国王的华丽,以及严饰的女人们,}

Tesaṃ āsi vipallāso, disvāna aṇuto aṇuṃ;\\
rājino ca viyākāraṃ, nāriyo samalaṅkatā. %\hfill\textcolor{gray}{\footnotesize 16}

\begin{enumerate}\item 然而,岁月迁延,欲犯破界之相的婆罗门,「他们生起颠倒……住处居所」。这里,\textbf{颠倒},即颠倒想。\textbf{由微末为微末}\footnote{由微末为微末 \textit{aṇuto aṇuṃ}:即颂中的「渐渐」,这里的译文未从义注,而径作副词,Norman 英译亦作 little by little。若按义注的解释,则作「由微末而被称为微末的欲乐」,将其与「国王的华丽、严饰的女人们」等并列,作为「见到」的宾语。},意即以低劣之义、有限之义、寡味之义,由作为微末的种种爱欲,\textbf{见到了}生起的欲乐相较于禅那、沙门性\footnote{沙门性 \textit{sāmañña}:PTS 本作「无量 \textit{appamaññā}」。}、涅槃之乐,以算数所不及而为微末,或者,相较于出世间乐,由作为微末的、以自身所得的世间定之乐而为微末,而欲乐亦由琐屑而为琐屑。\textbf{华丽},即成就。\end{enumerate}

\subsection\*{\textbf{303} {\footnotesize 〔PTS 300〕}}

\textbf{套有纯种马匹、精制、彩缀的车辆,\\}
\textbf{以及按部分规划、丈量的住处居所。}

Rathe c’ājaññasaṃyutte, sukate cittasibbane;\\
nivesane nivese ca, vibhatte bhāgaso mite. %\hfill\textcolor{gray}{\footnotesize 17}

\begin{enumerate}\item \textbf{精制},即以木作、铜作而善加制作。\textbf{彩缀},即以狮皮等装饰的彩缀。\textbf{住处},即房屋的基地。\textbf{居所},即于此建造的房屋。\textbf{规划},即按长宽来规划。\textbf{按部分丈量},即按庭院、门、楼阁、尖顶等,分成各各部分来丈量。
\item 这说的是什么?这些婆罗门,自从见到了由微末而被称为微末的欲乐、国王的华丽、严饰的女人们、所说品类的车辆、住处居所等,由对这些苦的依处转起乐,生起被称为「颠倒于过去转起之出离想」的颠倒想。\end{enumerate}

\subsection\*{\textbf{304} {\footnotesize 〔PTS 301〕}}

\textbf{牛群遍布,美女簇拥,\footnote{牛群遍布、美女簇拥:两者也是下行中「贪求」的宾语。}\\}
\textbf{婆罗门便贪求显赫的人间财富。}

Gomaṇḍalaparibyūḷhaṃ, nārīvaragaṇāyutaṃ;\\
uḷāraṃ mānusaṃ bhogaṃ, abhijjhāyiṃsu brāhmaṇā. %\hfill\textcolor{gray}{\footnotesize 18}

\begin{enumerate}\item 他们既如是作颠倒想,「牛群遍布……人间财富」。这里,\textbf{显赫},即广大。\textbf{人间财富},即众人的住处等财物。\textbf{贪求},即以「哎!这将是我们的」增长渴爱,陷入愿求。\end{enumerate}

\subsection\*{\textbf{305} {\footnotesize 〔PTS 302〕}}

\textbf{于此,他们编集了颂诗,然后去往甘蔗王处:\\}
\textbf{「你将有丰厚的财产和谷物,\\}
\textbf{「献祭吧!你的资产众多,献祭吧!你的财产众多。」}

Te tattha mante ganthetvā, Okkākaṃ tad’upāgamuṃ;\\
“pahūtadhanadhañño si,\\
yajassu bahu te vittaṃ, yajassu bahu te dhanaṃ”. %\hfill\textcolor{gray}{\footnotesize 19}

\begin{enumerate}\item 且当如是贪求时,想到「这些人善沐浴,善涂油,梳理须发,穿戴摩尼璎珞,沉溺种种五欲,我们即便受他们礼敬,却身污汗垢,长着腋毛指甲,无有财富,沦落到了最可悲悯的境地。且他们以象肩、马背、轿子、金车而游,我们则以双脚。他们住在两层等的楼阁露台,我们在林野树下。且他们睡在长毛毯等毯子所铺的高贵卧处,我们则在地上铺了草垫、皮片等。他们吃众味的食物,我们为搜罗而乞求。我们要如何和他们一样呢」,且确定说「财产是必需的,不能没有财产获此成就」,便破了吠陀,败坏了与法相应的古昔颂诗,编集了与非法相应的虚假颂诗,希求着财产,去往甘蔗王处,道过平安的话语等,说:「大王!我们有婆罗门世系中以谱系流传的古昔颂句,为守老师的秘传,我们未对任何人宣说,大王应当听听。」他们便称颂马牲等的献牲,且称颂后,为怂恿国王,便说:「献祭吧!大王!这样你将有丰厚的财产和谷物,你不欠缺献牲的资粮,这样,由你的献祭,家族七代会投生天界。」因此,世尊为显明彼等的本末,便说了此颂。
\item 这里,\textbf{于此},即是说因为他们贪求财富——此依格作理由义。\textbf{你将有丰厚的财产和谷物},即指将来而言——晓声者在表期望的未来也可用现在时。\textbf{资产、财产},金等宝物,由为幸福之缘故为资产,由为繁荣之缘故为财产\footnote{这是语源上的解释:幸福的原文为 vitti,资产为 vitta,繁荣为 samiddhi,财产为 dhanna。}。或者,资产,即作为幸福之缘的璎珞等器具,在\begin{quoting}丰厚的资产和器具。(长部·究罗檀头经第 331 段)\end{quoting}等处出现,财产,即货币、黄金等。
\item 这说的是什么?这些婆罗门编集了颂诗,便去往甘蔗王处。说什么?「大王!你的资产、财产众多,献祭吧!将来你会有丰厚的财产和谷物。」\end{enumerate}

\subsection\*{\textbf{306} {\footnotesize 〔PTS 303〕}}

\textbf{随后,被婆罗门说服的国王、车乘之主,\\}
\textbf{马牲、人牲、掷棒祭、娑摩祭、无遮祭——\\}
\textbf{举行完这些祭祀,他便赐予婆罗门财产。}

Tato ca rājā saññatto, brāhmaṇehi rathesabho;\\
assamedhaṃ purisamedhaṃ, sammāpāsaṃ vājapeyyaṃ niraggaḷaṃ;\\
ete yāge yajitvāna, brāhmaṇānam adā dhanaṃ. %\hfill\textcolor{gray}{\footnotesize 20}

\begin{enumerate}\item 如是说了缘由,被说服的「国王……赐予财产」。这里,\textbf{被说服},即被告知。\textbf{车乘之主},即以不可动摇之义,其在刹帝利中,如大车乘中的公牛一般。在「马牲」等中,在此歼害马即\textbf{马牲},意即伴有两个附祭\footnote{附祭 \textit{pariyañña}:DoP 的解释是 a secondary or accompanying rite。},为二十一类\footnote{二十一类:这里的类 \textit{yūpa},词典的解释是「系缚牺牲的柱子」。}应祭中除了地与人,其余所有丰盛供养之献牲。在此歼害人即\textbf{人牲},意即伴有四个附祭,为包括地的应祭,如马牲中所说的丰盛供养之献牲。在此掷棒即\textbf{掷棒祭},意即每天在掷棒后,于其落下处立起栏杆,以可动的柱子等,从娑罗室伐底河\footnote{娑罗室伐底河 \textit{Sarassatī/Sarasvatī}:印度梨俱吠陀时代的圣河,可能是印度河的支流,现已枯竭。}淹没之处开始,以逆向而行的旅行祭祀\footnote{旅行祭祀 \textit{satrayāga}:词典无 satra,这里从 PTS 本的 sātrā,被推测为 yātrā,待考。}之应祭。在此饮酒即\textbf{娑摩祭},意即伴有一个附祭,以十七头牲畜在橡木柱上的应祭,以十七份供养作献牲。在此无有遮拦即\textbf{无遮祭},意即伴有九个附祭,为包括地与人的应祭,如马牲中所说的丰盛供养,是一切牺牲法门之名、马牲的异名。其余于此自明。\end{enumerate}

\subsection\*{\textbf{307} {\footnotesize 〔PTS 304〕}}

\textbf{牛群、卧具与衣服,严饰的女人们,\\}
\textbf{套有纯种马匹、精致、彩缀的车辆,}

Gāvo sayanañ ca vatthañ ca, nāriyo samalaṅkatā;\\
rathe c’ājaññasaṃyutte, sukate cittasibbane. %\hfill\textcolor{gray}{\footnotesize 21}

\begin{enumerate}\item 现在,为显明所说的「他便赐予婆罗门财产」而说此二颂。因为那国王想「他们因长时的粗恶之食而疲累,让他们受用五种奶制品」,便赐予他们带有头牛的牛群,同样,想「他们因长时的露地之卧、粗布之衣、独睡、步行及树下等而疲累,让他们领受长毛毯等所铺的高贵卧处之乐」,便赐予他们贵重的卧具等。如是,他便赐予这种种品类的食物及货币黄金等财产。因此,世尊说了这二颂。\end{enumerate}

\subsection\*{\textbf{308} {\footnotesize 〔PTS 305〕}}

\textbf{惬意、按部分善加规划的住处\\}
\textbf{盈满了种种谷物,他便赐予婆罗门财产。}

Nivesanāni rammāni, suvibhattāni bhāgaso;\\
nānādhaññassa pūretvā, brāhmaṇānam adā dhanaṃ. %\hfill\textcolor{gray}{\footnotesize 22}

\subsection\*{\textbf{309} {\footnotesize 〔PTS 306〕}}

\textbf{他们于此获得财产后,便乐于贮藏,\\}
\textbf{陷入希求的他们,增长了更多渴爱,\\}
\textbf{于此,他们编集了颂诗,再次去往甘蔗王处:}

Te ca tattha dhanaṃ laddhā, sannidhiṃ samarocayuṃ;\\
tesaṃ icchāvatiṇṇānaṃ, bhiyyo taṇhā pavaḍḍhatha;\\
te tattha mante ganthetvā, Okkākaṃ punam upāgamuṃ. %\hfill\textcolor{gray}{\footnotesize 23}

\begin{enumerate}\item 如是,从那国王跟前,「他们于此……甘蔗王处」。这说的是什么?\textbf{他们},即婆罗门,从那国王跟前,在祭祀中\textbf{获得财产后},长时每天如是寻求衣食,\textbf{便乐于贮藏}种种品类的物欲。随后,\textbf{陷入希求的他们},即因乳等五种奶制品之乐味,心陷入味爱:「牛的乳等自然美味,它们的肉肯定更美味!」如是,便对肉\textbf{增长了更多渴爱}。随后,他们便想:「如果我们杀了就吃,将受谴责,我们何不编集颂诗?」于是,便再次破了吠陀,\textbf{于此,他们编集了}随适于此\textbf{颂诗},即这些婆罗门编集了虚假颂诗作为其理由,再次去往甘蔗王处,说了此义——即下颂是。\end{enumerate}

\subsection\*{\textbf{310} {\footnotesize 〔PTS 307〕}}

\textbf{「好比水、地、货币、财产和谷物,\\}
\textbf{「牛对人也如是,因为这是生命的资助,\\}
\textbf{「献祭吧!你的资产众多,献祭吧!你的财产众多。」}

“Yathā āpo ca pathavī ca, hiraññaṃ dhanadhāniyaṃ;\\
evaṃ gāvo manussānaṃ, parikkhāro so hi pāṇinaṃ;\\
yajassu bahu te vittaṃ, yajassu bahu te dhanaṃ”. %\hfill\textcolor{gray}{\footnotesize 24}

\begin{enumerate}\item 这说的是什么?大王!我们的颂诗中提到:\textbf{好比水}在洗手等一切事务中为生命提供用途,对他们没有因此而来的恶。为什么?因为\textbf{这是生命的资助},意即为资助而生起。且好比这大\textbf{地}在行止等一切事务中,被称为钱币的\textbf{货币}、金银等类的\textbf{财产和}大麦小麦等类的\textbf{谷物}在买卖等一切事务中提供用途,\textbf{牛对人也如是},在一切事务中为行使用途而生起。所以,杀了它们,在种种品类的祭祀中,\textbf{献祭吧!你的资产众多,献祭吧!你的财产众多}。\end{enumerate}

\subsection\*{\textbf{311} {\footnotesize 〔PTS 308〕}}

\textbf{随后,被婆罗门说服的国王、车乘之主\\}
\textbf{便在献牲中杀了数百千头牛。}

Tato ca rājā saññatto, brāhmaṇehi rathesabho;\\
nekā satasahassiyo, gāvo yaññe aghātayi. %\hfill\textcolor{gray}{\footnotesize 25}

\begin{enumerate}\item 如是,仍以先前的方法,\textbf{随后,国王……杀了数百千头牛},在此之前,\textbf{它们并未用足、用角伤害任何人……杀了它们}。当时,据说,婆罗门用母牛填了牲穴,缚了吉祥的公牛,向国王介绍理由,说道:「大王!请献牛牲之祭!这样,你的梵界之路将会清净!」国王行了吉祥的仪式,捉住角,便杀了连头牛在内的数百千头牛。婆罗门在牺穴中切割了牛便吃,且披了黄、白、红色的毯子再杀。关于此,据说,牛见到披衣者会惊慌。因此,世尊说了下颂。\end{enumerate}

\subsection\*{\textbf{312} {\footnotesize 〔PTS 309〕}}

\textbf{它们并未用足、用角伤害任何人,\\}
\textbf{牛如同山羊,温顺,产成桶的奶,\\}
\textbf{国王捉住角,用刀杀了它们。}

Na pādā na visāṇena, nāssu hiṃsanti kenaci;\\
gāvo eḷakasamānā, soratā kumbhadūhanā;\\
tā visāṇe gahetvāna, rājā satthena ghātayi. %\hfill\textcolor{gray}{\footnotesize 26}

\subsection\*{\textbf{313} {\footnotesize 〔PTS 310〕}}

\textbf{随后,诸天、诸父、因陀、阿修罗、罗刹\\}
\textbf{便呼喊道「非法」,当刀落在牛上。}

Tato devā pitaro ca, Indo asura-rakkhasā;\\
“adhammo” iti pakkanduṃ, yaṃ satthaṃ nipatī gave. %\hfill\textcolor{gray}{\footnotesize 27}

\begin{enumerate}\item 如是,当那国王开始杀牛的无间,见到杀牛后,这些四大王等的\textbf{诸天},以及\textbf{诸父}——即在婆罗门中得名的诸梵,以及诸天帝释\textbf{因陀},以及住在山脚下、被称为魔鬼、夜叉的\textbf{阿修罗、罗刹}便发出如是之语「\textbf{非法}、非法」,且说道「呸!人类!呸!人类」而作\textbf{呼喊}。如是,这声音从地上开始,瞬间便到了梵界,遍满一方世间。什么原因?\textbf{当刀落在牛上},即是说因为刀落在牛上。\end{enumerate}

\subsection\*{\textbf{314} {\footnotesize 〔PTS 311〕}}

\textbf{过去有三种疾病:欲望、饥饿、衰老,\\}
\textbf{从杀戮牲畜以后,乃有九十八种到来。}

Tayo rogā pure āsuṃ, icchā anasanaṃ jarā;\\
pasūnañ ca samārambhā, aṭṭhānavuti-m-āgamuṃ. %\hfill\textcolor{gray}{\footnotesize 28}

\begin{enumerate}\item 且不仅诸天等呼喊,在世间还生起了另一种非义:\textbf{过去}凡\textbf{有三种疾病:欲望、饥饿、衰老},即是说任何希求之渴爱、饥饿和遍熟之衰老。\textbf{从杀戮牲畜以后,乃有九十八种到来},这些(疾病)因眼疾等类而至九十八种之义。\end{enumerate}

\subsection\*{\textbf{315} {\footnotesize 〔PTS 312〕}}

\textbf{这刑罚之非法便流传,成了古昔,\\}
\textbf{无辜者被杀害,祭司们从法跌落。}

Eso adhammo daṇḍānaṃ, okkanto purāṇo ahu;\\
adūsikāyo haññanti, dhammā dhaṃsanti yājakā. %\hfill\textcolor{gray}{\footnotesize 29}

\begin{enumerate}\item 现在,世尊为非难杀戮牲畜,说了此颂。其义为:\textbf{这},即被称为杀戮牲畜者,作为身刑罚等三种\textbf{刑罚}的某一刑罚,由背离法\textbf{便流传}、转起为\textbf{非法},且其由从此转起而为\textbf{古昔},从其流传开始,未以足等伤害任何人的\textbf{无辜}的牛\textbf{被杀害}。那行杀害的\textbf{祭司们},即行献牲的人们,\textbf{从法跌落}、灭没、减损。\end{enumerate}

\subsection\*{\textbf{316} {\footnotesize 〔PTS 313〕}}

\textbf{如是,这古昔微末的法受智者谴责,\\}
\textbf{当看到这样的事,人们便谴责祭司。}

Evam eso aṇudhammo, porāṇo viññugarahito;\\
yattha edisakaṃ passati, yājakaṃ garahatī jano. %\hfill\textcolor{gray}{\footnotesize 30}

\begin{enumerate}\item \textbf{如是,这微末的法},即如是,这低劣之法、卑劣之法,即是说非法。或者,因为此中的布施之法亦为琐屑,所以就此而说「微末之法」。\textbf{古昔},由从久远之时开始转起而为古昔。由受智者谴责,当知为\textbf{受智者谴责}。且因为受智者谴责,所以\textbf{当看到这样的事,人们便谴责祭司}。如何?即说「由婆罗门兴起了之前所不存者\footnote{不存者 \textit{abhūtaṃ}:原作 abbudaṃ,费解,兹从 PTS 本。},他们杀了牛吃肉」。这于此是传闻。\end{enumerate}

\subsection\*{\textbf{317} {\footnotesize 〔PTS 314〕}}

\textbf{当法如是差谬时,首陀罗与吠舍分离,\\}
\textbf{刹帝利各各分离,妻子轻视丈夫。}

Evaṃ dhamme viyāpanne, vibhinnā sudda-vessikā;\\
puthū vibhinnā khattiyā, patiṃ bhariyā’vamaññatha. %\hfill\textcolor{gray}{\footnotesize 31}

\begin{enumerate}\item \textbf{当法如是差谬时},即当古昔的婆罗门法亡失时。文本也作 viyāvatte,即倒错后成为其它之义。\textbf{首陀罗与吠舍分离},即先前和合而住的首陀罗与吠舍两者分离。\textbf{刹帝利各各分离},即众多刹帝利也彼此分离。\textbf{妻子轻视丈夫},即妻子为了居家而置身主宰之力,凭借孩子之力等\footnote{孩子之力等:\textbf{相应部}第 37:25 经略云:女人有五种力——色之力、财之力、亲属之力、孩子之力、戒之力。},轻视丈夫,即轻蔑、轻视、不恭敬给侍。\end{enumerate}

\subsection\*{\textbf{318} {\footnotesize 〔PTS 315〕}}

\textbf{刹帝利和梵天的眷属,以及其他受种姓守护者,\\}
\textbf{摈弃了出身论,沦于爱欲的控制。}

Khattiyā brahmabandhū ca, ye c’aññe gottarakkhitā;\\
jātivādaṃ nirākatvā, kāmānaṃ vasam anvagun” ti. %\hfill\textcolor{gray}{\footnotesize 32}

\begin{enumerate}\item 当如是彼此分离时,「刹帝利和梵天的眷属……沦于爱欲的控制」。刹帝利和婆罗门,以及其他吠舍、首陀罗,由守护各自的种姓,不致混血,而为\textbf{受种姓守护者}。他们全都\textbf{摈弃了}这\textbf{出身论},亡失了「我是刹帝利、我是婆罗门」等一切,\textbf{沦于}被称为种种五欲的\textbf{爱欲的控制},陷入执著,即是说因爱欲而无所不为。\end{enumerate}

\textbf{如是说已,这些富裕婆罗门对世尊说:「希有!乔达摩君!……从今起,尽寿命,请乔达摩君受持我们皈依为优婆塞!」}

Evaṃ vutte, te brāhmaṇamahāsālā Bhagavantaṃ etad avocuṃ: “abhikkantaṃ, bho Gotama…pe… upāsake no bhavaṃ Gotamo dhāretu ajjatagge pāṇupete saraṇaṃ gate” ti.

\begin{enumerate}\item 如是,此中,世尊以「从前的仙人们」等九颂赞美了古昔的婆罗门,以「其中的最上者」说同梵,以「效仿其行仪」说同天,以「如法乞求米粒」等四颂说界内,以「他们生起颠倒」等十七颂说破界,且在显示了诸天等因此邪行道而呼喊后,完成了开示。
\item 但旃陀罗婆罗门却未在此说及。为什么?因为无有欠缺的出现。以婆罗门法的成就,而有同梵、同天、界内的出现,以欠缺而有破界,但这在「头那经」中所说品类的旃陀罗婆罗门,以婆罗门法的欠缺也未出现。为什么?由已生起破损之法故。所以,未显示此便完成了开示。
\item 然而,现今连旃陀罗婆罗门也难得。如是,此婆罗门之法已消失。因此,头那婆罗门便说:「既然如此,乔达摩君!我们连旃陀罗婆罗门也够不上。」其余则如已述。\end{enumerate}

\begin{center}\vspace{1em}婆罗门法经第七\\Brāhmaṇadhammikasuttaṃ sattamaṃ.\end{center}