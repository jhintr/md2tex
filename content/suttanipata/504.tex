\section{富楼那学童问}

\subsection\*{\textbf{1050} {\footnotesize 〔PTS 1043〕}}

\textbf{「不动者、得见根本者,」尊者富楼那说,「我带着问题前来,\\}
\textbf{「依据什么,仙人、人类、刹帝利、婆罗门向诸天\\}
\textbf{「在此世间举行种种祭祀?我问你,世尊!请对我说说这个!」}

\begin{enumerate}\item (世尊)仍以先前的方法拒绝了空王而说此经。这里,\textbf{得见根本},即得见不善根等。\textbf{仙人},即名为仙人的萦发者。\textbf{祭祀},即所施。\textbf{举行},即寻求。\end{enumerate}

\subsection\*{\textbf{1051} {\footnotesize 〔PTS 1044〕}}

\textbf{「富楼那!凡仙人、人类、」世尊说,「刹帝利、婆罗门向诸天\\}
\textbf{「在此世间举行种种祭祀,\\}
\textbf{「富楼那!他们希求着这样的状态、束缚于老而举行祭祀。」}

\begin{enumerate}\item \textbf{希求着},即愿求着色等。\textbf{这样的状态},即是说人等之状。\textbf{束缚于老},即依于老。且这里以老为首来说一切流转之苦,以此显明依于流转之苦,仍未从此解脱者举行(祭祀)。\end{enumerate}

\subsection\*{\textbf{1052} {\footnotesize 〔PTS 1045〕}}

\textbf{「凡仙人、人类、」尊者富楼那说,「刹帝利、婆罗门向诸天\\}
\textbf{「在此世间举行种种祭祀,世尊!他们不放逸于祭祀之路,是否能\\}
\textbf{「度脱生与老?先生!我问你,世尊!请对我说说这个!」}

\begin{enumerate}\item 在「\textbf{他们不放逸于祭祀,是否能度脱生与老}」中,祭祀即祭祀之路。这是在说,是否他们不放逸于祭祀,当举行祭祀时,能度脱流转之苦?\end{enumerate}

\subsection\*{\textbf{1053} {\footnotesize 〔PTS 1046〕}}

\textbf{「他们希求、赞美、渴望、供奉,富楼那!」世尊说,「出于利养而渴望爱欲,\\}
\textbf{「他们从事祭祀,染著有贪,我说不能度脱生老。」}

\begin{enumerate}\item \textbf{希求},即愿求获得色等。\textbf{赞美},即以「享祀丰絜,所施清净」等方法赞叹祭祀等。\textbf{渴望},即为获得色等而形之于言。\textbf{供奉},即布施。\textbf{出于利养而渴望爱欲},即出于获得色等而再再渴望爱欲,说「哎!愿这些是我们的」,即是说对此增长渴爱。\textbf{从事祭祀},即投入祭祀。\textbf{染著有贪},即如是,由这些希求等,仍为有贪所染,或由染著有贪故,在行这些希求等时,\textbf{不能度脱}、超过生等流转之苦。\end{enumerate}

\subsection\*{\textbf{1054} {\footnotesize 〔PTS 1047〕}}

\textbf{「如果他们从事祭祀,」尊者富楼那说,「不能以祭祀度脱生与老,先生!\\}
\textbf{「那么,谁能在天人的世间度脱生与老?先生!\\}
\textbf{「我问你,世尊!请对我说说这个!」}

\subsection\*{\textbf{1055} {\footnotesize 〔PTS 1048〕}}

\textbf{「省思了世间种种,富楼那!」世尊说,「他在世间没有任何动摇,\\}
\textbf{「寂静、无烟、无患、无待,我说他度脱了生老。」}

\begin{enumerate}\item \textbf{省思},即以智审视。\textbf{种种},即彼此,即是说他人的自体、自己的自体等的彼此。\textbf{无烟},即无有身恶行等之烟。\textbf{无患},即无有贪等患。\textbf{他度脱了},即他这样的阿罗汉度脱了生老。此中其余皆自明。
\item 如是,世尊仍以阿罗汉为顶点开示了此经。当开示终了,这婆罗门与一千弟子即住于阿罗汉,而其他数千人生起了法眼。余皆同前。\end{enumerate}

\begin{center}富楼那学童问第三\end{center}