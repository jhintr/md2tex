\section{富楼那学童问}

\begin{center}Puṇṇaka Māṇava Pucchā\end{center}\vspace{1em}

\begin{enumerate}\item 同样,以先前的方法拒绝了空王而说此(颂)。\end{enumerate}

\subsection\*{\textbf{1050} {\footnotesize 〔PTS 1043〕}}

\textbf{「不动者、得见根本者,」尊者富楼那说,「我带着问题前来,\\}
\textbf{「依据什么,仙人、人类、刹帝利、婆罗门于此世间\\}
\textbf{「向诸天举行种种祭祀?我问您,世尊!请对我说说这个!」}

“Anejaṃ mūladassāviṃ, \textit{(icc āyasmā Puṇṇako)} atthi pañhena āgamaṃ;\\
kiṃ nissitā isayo manujā, khattiyā brāhmaṇā devatānaṃ;\\
yaññam akappayiṃsu puthū’dha loke, pucchāmi taṃ Bhagavā brūhi me taṃ”. %\hfill\textcolor{gray}{\footnotesize 1}

\begin{enumerate}\item \textbf{得见根本者},即得见不善的根本等者。\textbf{仙人},即名为仙人的萦发者。\textbf{祭祀},即供品。\textbf{举行},即寻求。\end{enumerate}

\begin{itemize}\item 案,译文调整了 d-e 两行的部分语序,以下二颂亦然。\end{itemize}

\subsection\*{\textbf{1051} {\footnotesize 〔PTS 1044〕}}

\textbf{「富楼那!凡仙人、人类、」世尊说,「刹帝利、婆罗门于此世间\\}
\textbf{「向诸天举行种种祭祀,\\}
\textbf{「富楼那!他们希求着这样的状态、系缚于老而举行祭祀。」}

“Ye kec’ime isayo manujā, \textit{(Puṇṇakā ti Bhagavā)} khattiyā brāhmaṇā devatānaṃ;\\
yaññam akappayiṃsu puthū’dha loke;\\
āsīsamānā Puṇṇaka itthattaṃ, jaraṃ sitā yaññam akappayiṃsu”. %\hfill\textcolor{gray}{\footnotesize 2}

\begin{enumerate}\item \textbf{这样的状态},即人类等的状态。\textbf{系缚于老},即依于老,这里以老为首而说一切轮回之苦,以此显明依于轮回之苦而未能解脱者举行(祭祀)。\end{enumerate}

\subsection\*{\textbf{1052} {\footnotesize 〔PTS 1045〕}}

\textbf{「凡仙人、人类、」尊者富楼那说,「刹帝利、婆罗门于此世间\\}
\textbf{「向诸天举行种种祭祀,世尊!他们不放逸于祭祀,是否能\\}
\textbf{「度脱生与老?先生!我问您,世尊!请对我说说这个!」}

“Ye kec’ime isayo manujā, \textit{(icc āyasmā Puṇṇako)} khattiyā brāhmaṇā devatānaṃ;\\
yaññam akappayiṃsu puthū’dha loke, kacci’ssu te Bhagavā yaññapathe appamattā;\\
atāruṃ jātiñ ca jarañ ca Mārisa, pucchāmi taṃ Bhagavā brūhi me taṃ”. %\hfill\textcolor{gray}{\footnotesize 3}

\subsection\*{\textbf{1053} {\footnotesize 〔PTS 1046〕}}

\textbf{「他们希求、赞美、渴望、献供,富楼那!」世尊说,「出于利养而渴望爱欲,\\}
\textbf{「他们从事祭祀,染著于有贪,我说不能度脱生与老。」}

“Āsīsanti thomayanti abhijappanti juhanti, \textit{(Puṇṇakā ti Bhagavā)} kāmābhijappanti paṭicca lābhaṃ;\\
te yājayogā bhavarāgarattā, nātariṃsu jātijaran ti brūmi”. %\hfill\textcolor{gray}{\footnotesize 4}

\begin{enumerate}\item \textbf{希求},即希求获得色等。\textbf{赞美},即以「享祀丰絜,所施清净」等方式赞叹祭祀等。\textbf{渴望},即为获得色等而形诸于言。\textbf{献供},即布施。\textbf{出于利养而渴望爱欲},即出于获得色等而再再渴望爱欲,说「哎呀!愿这些是我们的」,即指于此增长渴爱。\end{enumerate}

\subsection\*{\textbf{1054} {\footnotesize 〔PTS 1047〕}}

\textbf{「如果他们从事祭祀,」尊者富楼那说,「不能以祭祀度脱生与老,先生!\\}
\textbf{「那么,在天人的世间,谁能度脱生与老?先生!\\}
\textbf{「我问您,世尊!请对我说说这个!」}

“Te ce nātariṃsu yājayogā, \textit{(icc āyasmā Puṇṇako)} yaññehi jātiñ ca jarañ ca Mārisa;\\
atha ko carahi devamanussaloke, atāri jātiñ ca jarañ ca Mārisa;\\
pucchāmi taṃ Bhagavā brūhi me taṃ”. %\hfill\textcolor{gray}{\footnotesize 5}

\subsection\*{\textbf{1055} {\footnotesize 〔PTS 1048〕}}

\textbf{「省思了世间的种种,富楼那!」世尊说,「他对世间任何都无动摇,\\}
\textbf{「寂静、无烟、无患、无待,我说他能度脱生与老。」}

“Saṅkhāya lokasmi paroparāni, \textit{(Puṇṇakā ti Bhagavā)} yass’iñjitaṃ natthi kuhiñci loke;\\
santo vidhūmo anīgho nirāso, atāri so jātijaran ti brūmī” ti. %\hfill\textcolor{gray}{\footnotesize 6}

\begin{enumerate}\item \textbf{种种},即彼此,即是说他人的自性、自己的自性等彼此的意思。\textbf{无烟},即离于身恶行等之烟。\textbf{无患},即离于贪等恼。
\item 如是,世尊同样以阿罗汉为顶点开示了此经。当开示终了,这婆罗门与一千弟子即住于阿罗汉性,而其余数千人生起了法眼。余如前说。\end{enumerate}

\begin{itemize}\item 案,\textbf{寂静、无烟、无患、无待},可参见孙陀利迦婆罗豆婆遮经第 465 颂注。\end{itemize}

\begin{center}\vspace{1em}富楼那学童问第三\\Puṇṇakamāṇavapucchā tatiyā.\end{center}