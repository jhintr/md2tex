\chapter{檜羔裘詁訓傳第十三}

\begin{quoting}\textbf{釋文}檜者,高辛氏之火正祝融之後,妘姓之國也,其封域在古豫州外方之北、滎波之南,居溱洧之間,祝融之故墟,是子男之國,後為鄭武所并焉,王云「周武王封之於濟洛河潁之間,為檜子」。\end{quoting}

\section{羔裘}

%{\footnotesize 三章、章四句}

\textbf{羔裘,大夫以道去其君也。國小而迫,君不用道,好潔其衣服、逍遙遊燕而不能自強於政治,故作是詩也。}{\footnotesize 以道去其君者,三諫不從,待放於郊,得玦乃去。}

\textbf{羔裘逍遙,狐裘以朝。}{\footnotesize 羔裘以遊燕,狐裘以適朝。箋云諸侯之朝服緇衣羔裘,大腊而息民則有黃衣狐裘,今以朝服燕、祭服朝,是其好絜衣服也。先言燕,後言朝,見君之志不能自強於政治。}\textbf{豈不爾思,勞心忉忉。}{\footnotesize 國無政令,使我心勞。箋云爾,女也。三諫不從,待放而去,思君如是,心忉忉然。}

\textbf{羔裘翱翔,狐裘在堂。}{\footnotesize 堂,公堂也。箋云翱翔,猶逍遙也。}\textbf{豈不爾思,我心憂傷。}

\textbf{羔裘如膏,日出有曜。}{\footnotesize 日出照曜,然後見其如膏。}\textbf{豈不爾思,中心是悼。}{\footnotesize 悼,動也。箋云悼,猶哀傷也。}

\begin{quoting}\textbf{陳奐}動,古慟字。\end{quoting}

\section{素冠}

%{\footnotesize 三章、章三句}

\textbf{素冠,刺不能三年也。}{\footnotesize 喪禮「子為父、父卒為母皆三年」,時人恩薄禮廢,不能行也。}

\textbf{庶見素冠兮,棘人欒欒兮,}{\footnotesize 庶,幸也。素冠,練冠也。棘,急也。欒欒,瘠貌。箋云喪禮既祥祭而縞冠素紕,時人皆解緩,無三年之恩於其父母而廢其喪禮,故覬幸一見素冠急於哀戚之人,形貌欒欒然瘦瘠也。}\textbf{勞心慱慱兮。}{\footnotesize 慱慱,憂勞也。箋云勞心者,憂不得見。}

\begin{quoting}棘,古瘠字。欒 \texttt{luán},魯詩作臠,說文「臠,臞也」。慱 \texttt{tuán}。\end{quoting}

\textbf{庶見素衣兮,}{\footnotesize 素冠,故素衣也。箋云除成喪者,其祭也朝服縞冠,朝服緇衣素裳,然則此言素衣者,謂素裳也。}\textbf{我心傷悲兮,聊與子同歸兮。}{\footnotesize 願見有禮之人,與之同歸。箋云聊,猶且也。且與子同歸,欲之其家,觀其居處。}

\textbf{庶見素韠兮,}{\footnotesize 箋云祥祭朝服素韠者,韠從裳色。}\textbf{我心蘊結兮,聊與子如一兮。}{\footnotesize 子夏三年之喪畢,見於夫子,援琴而弦,衎衎而樂,作而曰「先王制禮,不敢不及」,夫子曰「君子也」,閔子騫三年之喪畢,見於夫子,援琴而弦,切切而哀,作而曰「先王制禮,不敢過也」,夫子曰「君子也」,子路曰「敢問何謂也」,夫子曰「子夏哀已盡,能引而致之於禮,故曰君子也,閔子騫哀未盡,能自割以禮,故曰君子也,夫三年之喪,賢者之所輕,不肖者之所勉」。箋云聊與子如一,且欲與之居處,觀其行也。}

\begin{quoting}\textbf{朱熹}與子如一,甚於同歸也。\end{quoting}

\section{隰有萇楚}

%{\footnotesize 三章、章四句}

\textbf{隰有萇楚,疾恣也。國人疾其君之淫恣而思無情慾者也。}{\footnotesize 恣,謂狡㹟淫戲,不以禮也。}

\textbf{隰有萇楚,猗儺其枝。}{\footnotesize 興也。萇楚,銚弋也。猗儺,柔順也。箋云銚弋之性,始生正直,及其長大則其枝猗儺而柔順,不妄尋蔓草木,興者,喻人少而端愨則長大無情慾。}\textbf{夭之沃沃,樂子之無知。}{\footnotesize 夭,少也。沃沃,壯佼也。箋云知,匹也。疾君之恣,故於人年少沃沃之時樂其無妃匹之意。}

\begin{quoting}猗儺,魯詩作旖旎。\end{quoting}

\textbf{隰有萇楚,猗儺其華。夭之沃沃,樂子之無家。}{\footnotesize 箋云無家,謂無夫婦室家之道。}

\textbf{隰有萇楚,猗儺其實。夭之沃沃,樂子之無室。}

\begin{quoting}\textbf{錢鍾書}室家之累,於身最切,舉示以概憂生之嗟耳。\end{quoting}

\section{匪風}

%{\footnotesize 三章、章四句}

\textbf{匪風,思周道也。國小政亂,憂及禍難而思周道焉。}

\textbf{匪風發兮,匪車偈兮。}{\footnotesize 發發飄風,非有道之風,偈偈疾驅,非有道之車。}\textbf{顧瞻周道,中心怛兮。}{\footnotesize 怛,傷也。下國之亂,周道滅也。箋云周道,周之政令也。迴首曰顧。}

\begin{quoting}\textbf{王念孫}匪,當為彼。偈 \texttt{jié}。\textbf{馬瑞辰}周道又為通道,亦大道也,凡詩周道,皆謂大路。\end{quoting}

\textbf{匪風飄兮,匪車嘌兮。}{\footnotesize 迴風為飄。嘌嘌,無節度也。}\textbf{顧瞻周道,中心弔兮。}{\footnotesize 弔,傷也。}

\begin{quoting}說文「嘌 \texttt{piāo},疾也」。\end{quoting}

\textbf{誰能亨魚,溉之釜鬵。}{\footnotesize 溉,滌也。鬵,釜屬。亨魚煩則碎,治民煩則散,知亨魚則知治民矣。箋云誰能者,言人偶能割亨者。}\textbf{誰將西歸,懷之好音。}{\footnotesize 周道在乎西。懷,歸也。箋云誰將者,亦言人偶能輔周道治民者也。檜在周之東,故言西歸。有能西仕於周者,我則懷之以好音,謂周之舊政令。}

\begin{quoting}鬵 \texttt{xín}。\end{quoting}

%\begin{flushright}檜國四篇、十二章、四十五句\end{flushright}