\chapter{甫田之什詁訓傳第二十一}

\section{甫田}

%{\footnotesize 四章、章十句}

\textbf{甫田,刺幽王也。君子傷今而思古焉。}{\footnotesize 刺者,刺其倉廪空虛,政煩賦重,農人失職。}

\textbf{倬彼甫田,歲取十千。}{\footnotesize 倬,明貌。甫田,天下田也。十千,言多也。箋云甫之言丈夫也。明乎彼大古之時,以丈夫稅田也。歲取十千,於井田之法則一成之數也,九夫為井,井稅一夫,其田百畝,井十為通,通稅十夫,其田千畝,通十為成,成方十里,成稅百夫,其田萬畝,欲見其數,從井通起,故言十千。上地穀畝一鍾。}\textbf{我取其陳,食我農人,自古有年。}{\footnotesize 尊者食新,農夫食陳。箋云倉廪有餘,民得賒貰取食之,所以紓官之蓄滯,亦使民愛存新穀,自古者豐年之法如此。}\textbf{今適南畝,或耘或耔,黍稷薿薿。}{\footnotesize 耘,除草也。耔,雝本也。箋云今者,今成王之法也,使農人之南畝,治其禾稼,功至力盡,則薿薿然而茂盛。於古言稅法,今言治田,互辭。}\textbf{攸介攸止,烝我髦士。}{\footnotesize 烝進、髦俊也。治田得穀,俊士以進。箋云介,舍也。禮,使民鋤作耘耔,閑暇則於廬舍及所止息之處以道藝相講肄,以進其為俊士之行。}

\begin{quoting}倬 \texttt{zhuō},大也。薿 \texttt{nǐ}。\textbf{林義光}介讀為愒 \texttt{kài},說文「愒,息也」,介古作匄,愒从匄得聲,則介、愒古同音,書酒誥云「爾乃自介用逸」,又云「不惟自息乃逸」,自介即自息,介亦愒之假借也。\end{quoting}

\textbf{以我齊明,與我犧羊,以社以方。}{\footnotesize 器實曰齊,在器曰盛。社,后土也。方,迎四方氣於郊也。箋云以絜齊豐盛與我純色之羊,秋祭社與四方,為五穀成熟,報其功也。}\textbf{我田既臧,農夫之慶。}{\footnotesize 箋云臧,善也。我田事已善則慶賜農夫,謂大蜡之時,勞農以休息之也,年不順成則八蜡不通。}\textbf{琴瑟擊鼓,以御田祖,以祈甘雨,以介我稷黍,以穀我士女。}{\footnotesize 田祖,先嗇也。穀,善也。箋云御迎、介助、穀養也。設樂以迎祭先嗇,謂郊後始耕也,以求甘雨,佑助我禾稼,我當以養士女也。周禮曰「凡國祈年于田祖,吹豳雅,擊土鼓,以樂田畯」。}

\begin{quoting}齊明,即齍 \texttt{zī} 盛,釋文「齊,本又作齍」,\textbf{馬瑞辰}爾雅釋詁「明,成也」,釋名「成,盛也」,明為成,即為盛,傳箋皆以齊盛釋齊明,正以明為盛之假借。御 \texttt{yà},迎祭。\end{quoting}

\textbf{曾孫來止,以其婦子,饁彼南畝。田畯至喜,攘其左右,嘗其旨否。}{\footnotesize 箋云曾孫,謂成王也。攘,讀當為饟,饁、饟,饋也。田畯,司嗇,今之嗇夫也。喜,讀為饎,饎,酒食也。成王來止,謂出觀農事也,親與后、世子行,使知稼穡之艱難也。為農人之在南畝者設饋以勸之,司嗇至,則又加之以酒食,饟其左右從行者,成王親為嘗其饋之美否,示親之也。}\textbf{禾易長畝,終善且有。}{\footnotesize 易,治也。長畝,竟畝也。}\textbf{曾孫不怒,農夫克敏。}{\footnotesize 敏,疾也。箋云禾治而竟畝,成王則無所責怒,謂此農夫能且敏也。}

\begin{quoting}\textbf{馬瑞辰}攘,古讓字,此詩攘即揖讓字,謂田畯將嘗其酒食而先讓左右從行之人,示有禮也。易與移一聲之轉,說文「移,禾相倚移也」,倚移讀若阿那,為禾盛之貌。\textbf{陳奐}有讀「歲其有」之有。\end{quoting}

\textbf{曾孫之稼,如茨如梁。曾孫之庾,如坻如京。}{\footnotesize 茨,積也。梁,車梁也。京,高丘也。箋云稼,禾也,謂有藁者也。茨,屋蓋也。上古之稅法,近者納稯,遠者納粟米。庾,露積穀也。坻,水中之高地也。}\textbf{乃求千斯倉,乃求萬斯箱。}{\footnotesize 箋云成王見禾穀之稅委積之多,於是求千倉以處之,萬車以載之,是言年豐,收入踰前也。}\textbf{黍稷稻粱,農夫之慶。報以介福,萬壽無疆。}{\footnotesize 箋云慶,賜也。年豐則勞賜,農夫益厚,既有黍稷,加以稻粱。報者為之求福,助於八蜡之神,萬壽無疆竟也。}

\begin{quoting}\textbf{于省吾}如梁之梁本應作荊,荊與梁、粱並從刅聲,字本相通,茨本蒺藜,係蔓生密集之草,荊為叢生之木,詩人詠曾孫之稼,以茨之密集與荊之叢生為比,係形容禾稼之多,其言曾孫之庾如坻如京,係形容庾囤之高。坻,通阺,說文「秦謂陵阪為阺」。\end{quoting}

\section{大田}

%{\footnotesize 四章、二章章八句、二章章九句}

\textbf{大田,刺幽王也。言矜寡不能自存焉。}{\footnotesize 幽王之時,政煩賦重,而不務農事,蟲災害穀,風雨不時,萬民飢饉,矜寡無所取活,故時臣思古以刺之。}

\textbf{大田多稼,既種既戒,既備乃事。}{\footnotesize 箋云大田,謂地肥美可墾耕,多為稼,可以授民者也。將稼者,必先相地之宜而擇其種,季冬,命民出五種,計耦耕事,修耒耜,具田器,此之謂戒,是既備矣,至孟春土長冒橛,陳根可拔而事之。}\textbf{以我覃耜,俶載南畝。}{\footnotesize 覃,利也。箋云俶,讀為熾。載,讀為菑栗之菑。時至,民以其利耜,熾菑發所受之地,趨農急也。田一歲曰菑。}\textbf{播厥百穀,既庭且碩,曾孫是若。}{\footnotesize 庭,直也。箋云碩大、若順也。民既熾菑,則種其眾穀,眾穀生,盡條直茂大,成王於是則止力役以順民事,不奪其時。}

\begin{quoting}戒,同械。覃 \texttt{yǎn},同剡。釋文「俶,始也,載,事也」。\textbf{俞樾}庭,讀為挺。\end{quoting}

\textbf{既方既皁,既堅既好,不稂不莠。}{\footnotesize 實未堅熟曰皁。稂,童梁也。莠,似苗也。箋云方,房也,謂孚甲始生而未合時也。盡生房矣,盡成實矣,盡堅熟矣,盡齊好矣,而無稂莠,擇種之善、民力之專、時氣之和所致之。}\textbf{去其螟螣,及其蟊賊,無害我田稺。}{\footnotesize 食心曰螟,食葉曰螣,食根曰蟊,食節曰賊。箋云此四蟲者,嘗害我田中之稺禾,故明君以正己而去之。}\textbf{田祖有神,秉畀炎火。}{\footnotesize 炎火,盛陽也。箋云螟螣之屬,盛陽氣嬴則生之,今明君為政,田祖之神不受此害,持之付與炎火,使自消亡。}

\begin{quoting}稂莠 \texttt{láng yǒu}。螣 \texttt{tè}。\end{quoting}

\textbf{有渰萋萋,興雨祁祁,雨我公田,遂及我私。}{\footnotesize 渰,雲興貌。萋萋,雲行貌。祁祁,徐也。箋云古者陰陽和,風雨時,其來祁祁然而不暴疾,其民之心先公後私,令天正雨於公田,因及私田爾。此言民怙君德,蒙其餘惠。}\textbf{彼有不穫稺,此有不斂穧,彼有遺秉,此有滯穗,伊寡婦之利。}{\footnotesize 秉,把也。箋云成王之時,百穀既多,種同齊熟,收刈促遽,力皆不足,而有不穫不斂遺秉滯穗,故聽矜寡取之以為利。}

\begin{quoting}渰 \texttt{yǎn}。興雨,釋文「本或作興雲」。穧 \texttt{jì},禾捆。\end{quoting}

\textbf{曾孫來止,以其婦子。饁彼南畝,田畯至喜。}{\footnotesize 箋云喜,讀為饎,饎,酒食也。成王出觀農事,饋食耕者以勸之也,司嗇至,則又加之以酒食,勞倦之爾。}\textbf{來方禋祀,以其騂黑,與其黍稷。以享以祀,以介景福。}{\footnotesize 騂,牛也。黑,羊豕也。箋云成王之來,則又禋祀四方之神祈報焉。陽祀用騂牲,陰祀用黝牲。}

\begin{quoting}說文「禋,絜祀也」,左傳隱十一年杜注「絜齊以享,謂之禋祀」。\end{quoting}

\section{瞻彼洛矣}

%{\footnotesize 三章、章六句}

\textbf{瞻彼洛矣,刺幽王也。思古明王能爵命諸侯、賞善罰惡焉。}

\textbf{瞻彼洛矣,維水泱泱。}{\footnotesize 興也。洛,宗周溉浸水也。泱泱,深廣貌。箋云瞻,視也。我視彼洛水灌溉以時,其澤浸潤,以成嘉穀,興者,喻古明王恩澤加於天下,爵命賞賜,以成賢者也。}\textbf{君子至止,福祿如茨。}{\footnotesize 箋云君子至止者,謂來受爵命者也。爵命為福,賞賜為祿。茨,屋蓋也,如屋蓋,喻多也。}\textbf{韎韐有奭,以作六師。}{\footnotesize 韎韐者,茅蒐染韋也,一入曰韎韐,所以代韠也。天子六軍。箋云此諸侯世子也,除三年之喪,服士服而來,未遇爵命之時,時有征伐之事,天子以其賢,任為軍將,使代卿士將六軍而出。韎者,茅蒐染也,茅蒐,韎聲也。韐,祭服之韠,合韋為之。其服爵弁服,䊷衣纁裳者也。}

\begin{quoting}韎韐 \texttt{mèi gé},周禮司服「凡兵事,韋弁服」。奭,通赩 \texttt{xì},赤色。\end{quoting}

\textbf{瞻彼洛矣,維水泱泱。君子至止,鞸琫有珌。}{\footnotesize 鞸,容刀鞸也。琫上飾,珌下飾也。天子玉琫而珧珌,諸侯璗琫而璆珌,大夫鐐琫而鏐珌,士珕琫而珕珌。箋云此人世子之賢者也,既受爵命賞賜,而加賜容刀有飾,顯其能制斷。}\textbf{君子萬年,保其家室。}{\footnotesize 箋云德如是則能長安,其家室親,家室親,安之尤難,安則無篡殺之禍也。}

\begin{quoting}鞸,說文「刀室也」。\textbf{戴震}鞸琫 \texttt{bǐng běng} 有珌 \texttt{bì},猶上章韎韐有奭,奭,赤貌,珌,文貌。\end{quoting}

\textbf{瞻彼洛矣,維水泱泱。君子至止,福祿既同。}{\footnotesize 箋云此人世子之能繼世位者也,其爵命賞賜盡與其先君受命者同而已,無所加也。}\textbf{君子萬年,保其家邦。}

\begin{quoting}同,說文「合會也」。\end{quoting}

\section{裳裳者華}

%{\footnotesize 四章、章六句}

\textbf{裳裳者華,刺幽王也。古之仕者世祿,小人在位則讒諂並進,棄賢者之類、絕功臣之世焉。}{\footnotesize 古者,古昔明王時也。小人,斥今幽王也。}

\begin{quoting}\textbf{魏源}詩古微曰裳裳者華,亦諸侯嗣位初朝見之詩,故與瞻洛相次,孔子曰「于裳裳者華,見賢者世保其祿也」,次瞻洛後,蓋朝於東都所作。\end{quoting}

\textbf{裳裳者華,其葉湑兮。}{\footnotesize 興也。裳裳,猶堂堂也。湑,盛貌。箋云興者,華堂堂於上,喻君也,葉湑然於下,喻臣也,明王賢臣以德相承而治道興,則讒諂遠矣。}\textbf{我覯之子,我心寫兮。我心寫兮,是以有譽處兮。}{\footnotesize 箋云覯,見也。之子,是子也,謂古之明王也。言我得見古之明王,則我心所憂寫而去矣,我心所憂既寫,是則君臣相與,聲譽常處也。憂者,憂讒諂並進。}

\begin{quoting}裳,魯詩、韓詩作常。譽,通豫,安樂也。\end{quoting}

\textbf{裳裳者華,芸其黃矣。}{\footnotesize 芸,黃盛也。箋云華芸然而黃,興明王德之盛也,不言葉,微見無賢臣也。}\textbf{我覯之子,維其有章矣。維其有章矣,是以有慶矣。}{\footnotesize 箋云章,禮文也。言我得見古之明王,雖無賢臣,猶能使其政有禮文法度,政有禮文法度,是則我有慶賜之榮也。}

\textbf{裳裳者華,或黃或白。}{\footnotesize 箋云華或有黃者,或有白者,興明王之德時有駁而不純。}\textbf{我覯之子,乘其四駱。乘其四駱,六轡沃若。}{\footnotesize 言世祿也。箋云我得見明王德之駁者,雖無慶譽,猶能免於讒諂之害,守我先人之祿位,乘其四駱之馬,六轡沃若然。}

\textbf{左之左之,君子宜之。右之右之,君子有之。}{\footnotesize 左,陽道,朝祀之事。右,陰道,喪戎之事。箋云君子,斥其先人也,多才多藝,有禮於朝,有功於國。}\textbf{維其有之,是以似之。}{\footnotesize 似,嗣也。箋云維我先人有是二德,故先王使之世祿,子孫嗣之,今遇讒諂並進而見棄絕。}

\begin{quoting}\textbf{馬瑞辰}左之右之,謂左輔右弼。宜,說文「所安也」。有,廣雅「取也」。似,同嗣。末句謂維取用賢人,方能承其世祿。\end{quoting}

\section{桑扈}

%{\footnotesize 四章、章四句}

\textbf{桑扈,刺幽王也。君臣上下動無禮文焉。}{\footnotesize 動無禮文,舉事而不用先王禮法威儀也。}

\begin{quoting}\textbf{王質}詩總聞曰當是諸侯來朝而歸國餞送之際,美戒兼同。\end{quoting}

\textbf{交交桑扈,有鶯其羽。}{\footnotesize 興也。鶯然有文章。箋云交交,猶佼佼,飛往來貌。桑扈,竊脂也。興者,竊脂飛而往來有文章,人觀視而愛之,喻君臣以禮法威儀升降於朝廷,則天下亦觀視而仰樂之。}\textbf{君子樂胥,受天之祜。}{\footnotesize 胥,皆也。箋云胥,有才知之名也。祜,福也。王者樂臣下有才知文章,則賢人在位,庶官不曠,政和而民安,天予之以福祿。}

\begin{quoting}\textbf{朱熹}胥,語詞。\textbf{馬瑞辰}皆、嘉一聲之轉,廣雅釋言「皆,嘉也」,樂胥猶言樂嘉,嘉亦樂也。\end{quoting}

\textbf{交交桑扈,有鶯其領。}{\footnotesize 領,頸也。}\textbf{君子樂胥,萬邦之屏。}{\footnotesize 屏,蔽也。箋云王者之德樂賢知在位,則能為天下蔽捍四表患難也,蔽捍之者,謂蠻夷率服不侵畔。}

\begin{quoting}爾雅釋宮「屏謂之樹」,注「小牆當門中」。\end{quoting}

\textbf{之屏之翰,百辟為憲。}{\footnotesize 翰幹、憲法也。箋云辟,君也。王者之德,外能蔽捍四表之患難,內能立功立事,為之楨幹,則百辟卿士莫不脩職而法象之。}\textbf{不戢不難,受福不那。}{\footnotesize 戢,聚也。不戢,戢也,不難,難也。那,多也。不多,多也。箋云王者位至尊,天所子也,然而不自斂以先王之法,不自難以亡國之戒,則其受福祿亦不多也。}

\begin{quoting}爾雅釋詁「楨,幹也」,郭注「幹,所以當牆兩邊障土者也」。不,皆語詞也。爾雅釋詁「戢,和也」。難,同儺 \texttt{nuó},說文「儺,行有節也」。\textbf{陳奐}說文「齊謂多為㚌」,方言「大物盛多,齊宋之郊、楚魏之際曰夥」,史記陳勝世家「楚人謂多為夥」,那與夥同。\end{quoting}

\textbf{兕觥其觩,旨酒思柔。}{\footnotesize 箋云兕觥,罰爵也。古之王者與群臣燕飲,上下無失禮者,其罰爵徒觩然陳設而已,其飲美酒,思得柔順中和與共其樂,言不幠敖自淫恣也。}\textbf{彼交匪敖,萬福來求。}{\footnotesize 箋云彼,彼賢者也。賢者居處恭,執事敬,與人交必以禮,則萬福之祿就而求之,謂登用爵命,加以慶賜。}

\begin{quoting}觩,彎曲貌,釋文「本或作觓」。\textbf{馬瑞辰}柔之義為嘉,抑之詩曰「無不柔嘉」,柔亦嘉也。漢書五行志引作「匪交匪傲」,應劭注「言在位者不儌訐、不倨傲也」。\textbf{王引之}求,讀與逑同,逑,聚也,謂福祿來聚。\end{quoting}

\section{鴛鴦}

%{\footnotesize 四章、章四句}

\textbf{鴛鴦,刺幽王也。思古明王交於萬物有道、自奉養有節焉。}{\footnotesize 交於萬物有道,謂順其性,取之以時,不暴夭也。}

\textbf{鴛鴦于飛,畢之羅之。}{\footnotesize 興也。鴛鴦,匹鳥。太平之時,交於萬物有道,取之以時,於其飛乃畢掩而羅之。箋云匹鳥,言其止則相耦,飛則為雙,性馴耦也,此交萬物之實也,而言興者,廣其義也,獺祭魚而後漁,豺祭獸而後田,此亦皆其將縱散時也。}\textbf{君子萬年,福祿宜之。}{\footnotesize 箋云君子,謂明王也。交於萬物,其德如是,則宜壽考,受福祿也。}

\begin{quoting}\textbf{馬瑞辰}宜、綏皆安也。\end{quoting}

\textbf{鴛鴦在梁,戢其左翼。}{\footnotesize 言休息。箋云梁,石絕水之梁。戢,斂也。鴛鴦休息於梁,明王之時,人不驚駭,斂其左翼,以右翼掩之,自若無恐懼。}\textbf{君子萬年,宜其遐福。}{\footnotesize 箋云遐,遠也,遠,猶久也。}

\begin{quoting}釋文引韓詩「戢,捷也,捷其噣于左也」。\end{quoting}

\textbf{乘馬在廐,摧之秣之。}{\footnotesize 摧,莝也。秣,粟也。箋云摧,今莝字也。古者明王所乘之馬繫於廐,無事則委之以莝,有事乃予之穀,言愛國用也。以興於其身亦猶然,齊而後三舉設盛饌,恆日則減焉,此之謂有節也。}\textbf{君子萬年,福祿艾之。}{\footnotesize 艾,養也。箋云明王愛國用、自奉養之節如此,故宜久為福祿所養也。}

\begin{quoting}說文「莝,斬芻」。爾雅釋詁「艾,相也,相,輔也」。\end{quoting}

\textbf{乘馬在廐,秣之摧之。君子萬年,福祿綏之。}{\footnotesize 箋云綏,安也。}

\section{頍弁}

%{\footnotesize 三章、章十二句}

\textbf{頍弁,諸公刺幽王也。暴戾無親,不能宴樂同姓、親睦九族,孤危將亡,故作是詩也。}{\footnotesize 戾,虐也,暴虐,謂其政教如雨雪也。}

\textbf{有頍者弁,實維伊何。}{\footnotesize 興也。頍,弁貌。弁,皮弁也。箋云實,猶是也。言幽王服是皮弁之冠,是維何為乎,言其宜以宴而弗為也。禮,天子諸侯朝服以宴天子之朝,皮弁以日視朝。}\textbf{爾酒既旨,爾殽既嘉。}{\footnotesize 箋云旨、嘉皆美也。女酒已美矣,女殽已美矣,何以不用與族人宴也,言其知具其禮而弗為也。}\textbf{豈伊異人,兄弟匪他。}{\footnotesize 箋云此言王當所與宴者,豈有異人疏遠者乎,皆兄弟與王。無他,言至親,又刺其弗為也。}\textbf{蔦與女蘿,施于松柏。}{\footnotesize 蔦,寄生也。女蘿,菟絲、松蘿也。喻諸公非自有尊,託王之尊。箋云託王之尊者,王明則榮,王衰則微。刺王不親九族,孤特自恃,不知己之將危亡也。}\textbf{未見君子,憂心弈弈。既見君子,庶幾說懌。}{\footnotesize 弈弈然無所薄也。箋云君子,斥幽王也。幽王久不與諸公宴,諸公未得見幽王之時,懼其將危亡,己無所依怙,故憂而心弈弈然,故言我若已得見幽王諫正之,則庶幾其變改,意解懌也。}

\begin{quoting}頍 \texttt{kuǐ},\textbf{林義光}按毛大東傳云「跂,隅貌」,頍猶跂也,謂弁頂尖銳,其上有隅也。\textbf{陳奐}僖八年穀梁傳曰「弁冕雖舊,必加於首,周室雖衰,必先諸侯」,然則王者之在上位,猶皮弁之在人首,故以為喻也。爾雅釋訓「弈弈,憂也」。\end{quoting}

\textbf{有頍者弁,實維何期。}{\footnotesize 箋云何期,猶伊何也。期,辭也。}\textbf{爾酒既旨,爾肴既時。}{\footnotesize 時,善也。}\textbf{豈伊異人,兄弟具來。}{\footnotesize 箋云具,猶皆也。}\textbf{蔦與女蘿,施于松上。未見君子,憂心怲怲。既見君子,庶幾有臧。}{\footnotesize 怲怲,憂盛滿也。臧,善也。}

\textbf{有頍者弁,實維在首。爾酒既旨,爾肴既阜。豈伊異人,兄弟甥舅。}{\footnotesize 箋云阜,猶多也。謂吾舅者,吾謂之甥也。}\textbf{如彼雨雪,先集維霰。}{\footnotesize 霰,暴雪也。箋云將大雨雪,始必微溫,雪自上下,遇溫氣而摶,謂之霰,久而寒勝,則大雪矣。喻幽王之不親九族,亦有漸自微至甚,如先霰後大雪。}\textbf{死喪無日,無幾相見。樂酒今夕,君子維宴。}{\footnotesize 箋云王政既衰,我無所依怙,死亡無有日數,能復幾何與王相見也,且今夕喜樂此酒,此乃王之宴禮也。刺幽王將喪亡,哀之也。}

\section{車舝}

%{\footnotesize 五章、章六句}

\textbf{車舝,大夫刺幽王也。褒姒嫉妬,無道並進,讒巧敗國,德澤不加於民,周人思得賢女以配君子,故作是詩也。}

\begin{quoting}左傳昭二十五年「叔孫婼如宋迎女,賦車舝」。\end{quoting}

\textbf{間關車之舝兮,思孌季女逝兮。}{\footnotesize 興也。間關,設舝也。孌,美貌。季女,謂有齊季女也。箋云逝,往也。大夫嫉褒姒之為惡,故嚴車設其舝,思得孌然美好之少女有齊莊之德者,往迎之以配幽王,代褒姒也,既幼而美,又齊莊,庶其當王意。}\textbf{匪飢匪渴,德音來括。}{\footnotesize 括,會也。箋云時讒巧敗國,下民離散,故大夫汲汲欲迎季女,行道雖飢不飢,雖渴不渴,覬得之而來,使我王更脩德教,合會離散之人。}\textbf{雖無好友,式燕且喜。}{\footnotesize 箋云式,用也。我得德音而來,雖無同好之賢友,我猶用是燕飲,相慶且喜。}

\begin{quoting}間關,擬聲詞。舝,同轄。孌,說文「慕也」,段注「在小篆為今之戀,慕也,孌、戀為古今字」。燕,通宴。\end{quoting}

\textbf{依彼平林,有集維鷮。辰彼碩女,令德來敎。}{\footnotesize 依,茂木貌。平林,林木之在平地者也。鷮,雉也。辰,時也。箋云平林之木茂,則耿介之鳥往集焉,喻王若有茂美之德,則其時賢女來配之,與相訓告,改脩德教。}\textbf{式燕且譽,好爾無射。}{\footnotesize 箋云爾,女,女,王也。射,厭也。我於碩女來教,則用是燕飲酒,且稱王之聲譽,我愛好王無有厭也。}

\begin{quoting}鷮 \texttt{jiāo},說文「長尾雉,走且鳴」,\textbf{陳奐}平林之有鷮,以喻賢女之在父母家也。辰,善也。譽,通豫。射,通斁。\end{quoting}

\textbf{雖無旨酒,式飲庶幾。雖無嘉殽,式食庶幾。雖無德與女,式歌且舞。}{\footnotesize 箋云諸大夫覬得賢女以配王,於是酒雖不美猶用之燕飲,殽雖不美猶食之,人皆庶幾於王之變改,得輔佐之,雖無其德,我與女用是歌舞,相樂喜之至也。}

\begin{quoting}庶幾,\textbf{林義光}願望之詞。\end{quoting}

\textbf{陟彼高岡,析其柞薪。析其柞薪,其葉湑兮。}{\footnotesize 箋云陟,登也。登高岡者,必析其木以為薪,析其木以為薪者,為其葉茂盛,蔽岡之高也,此喻賢女得在王后之位,則必辟除嫉妬之女,亦為其蔽君之明。}\textbf{鮮我覯爾,我心寫兮。}{\footnotesize 箋云鮮善、覯見也。善乎,我得見女如是,則我心中之憂除去也。}

\begin{quoting}\textbf{馬瑞辰}按漢廣有刈薪之言,南山有析薪之句,豳風之伐柯與娶妻同喻,詩中以析薪喻昏姻者不一而足。\end{quoting}

\textbf{高山仰止,景行行止。四牡騑騑,六轡如琴。}{\footnotesize 景,大也。箋云景,明也。諸大夫以為賢女既進,則王亦庶幾古人,有高德者則慕仰之,有明行者則而行之。其御群臣使之有禮,如御四馬騑騑然,持其教令,使之調均,亦如六轡緩急有和也。}\textbf{覯爾新昬,以慰我心。}{\footnotesize 慰,安也。箋云我得見女之新昏如是,則以慰除我心之憂也。新昏,謂季女也。}

\begin{quoting}釋文「仰止,本或作仰之」,\textbf{于省吾}謂此二止字皆之字之訛。\end{quoting}

\section{靑蠅}

%{\footnotesize 三章、章四句}

\textbf{靑蠅,大夫刺幽王也。}

\textbf{營營靑蠅,止于樊。}{\footnotesize 興也。營營,往來貌。樊,藩也。箋云興者,蠅之為蟲,汙白使黑,汙黑使白,喻佞人變亂善惡也。言止于藩,欲外之,令遠物也。}\textbf{豈弟君子,無信讒言。}{\footnotesize 箋云豈弟,樂易也。}

\textbf{營營靑蠅,止于棘。讒人罔極,交亂四國。}{\footnotesize 箋云極,猶已也。}

\textbf{營營靑蠅,止于榛。}{\footnotesize 榛,所以為藩也。}\textbf{讒人罔極,構我二人。}{\footnotesize 箋云構,合也,合,猶交亂也。}

\begin{quoting}構,釋文引韓詩「亂也」。\end{quoting}

\section{賓之初筵}

%{\footnotesize 五章、章十四句}

\textbf{賓之初筵,衛武公刺時也。幽王荒廢,媟近小人,飲酒無度,天下化之,君臣上下沈湎淫液,武公既入而作是詩也。}{\footnotesize 淫液者,飲酒時情態也。武公入者,入為王卿士。}

\begin{quoting}後漢書孔融傳李注引韓詩「衛武公飲酒悔過也」。\end{quoting}

\textbf{賓之初筵,左右秩秩。}{\footnotesize 秩秩然肅敬也。箋云筵,席也。左右,謂折旋揖讓也。秩秩,知也。先王將祭,必射以擇士,大射之禮,賓初入門,登堂即席,其趨翔威儀甚審知,言不失禮也。射禮有三,有大射、有賓射、有燕射。}\textbf{籩豆有楚,殽核維旅。}{\footnotesize 楚,列貌。殽,豆實也。核,加籩也。旅,陳也。箋云豆實,菹醢也,籩實有桃梅之屬。凡非穀而食之曰殽。}\textbf{酒既和旨,飲酒孔偕。}{\footnotesize 箋云和旨,猶調美也。孔,甚也。王之酒已調美,眾賓之飲酒又威儀齊一,言主人敬其事而眾賓肅慎也。}\textbf{鍾鼓既設,舉醻逸逸。}{\footnotesize 逸逸,往來次序也。箋云鍾鼓於是言既設者,將射故縣也。}\textbf{大侯既抗,弓矢斯張。}{\footnotesize 大侯,君侯也。抗,舉也。有燕射之禮。箋云舉者,舉鵠而棲之於侯也,周禮梓人「張皮侯而棲鵠」,天子諸侯之射皆張三侯,故君侯謂之大侯,大侯張而弓矢亦張節也。將祭而射,謂之大射,下章言「烝衎烈祖」,其非祭與。}\textbf{射夫既同,獻爾發功。}{\footnotesize 箋云射夫,眾射者也。獻,猶奏也。既比眾耦,乃誘射,射者乃登射,各奏其發矢中的之功。}\textbf{發彼有的,以祈爾爵。}{\footnotesize 的,質也。祈,求也。箋云發,發矢也。射者與其耦拾發,發矢之時,各心競云「我以此求爵女」。爵,射爵也。射之禮,勝者飲不勝,所以養病也,故論語曰「下而飲,其爭也君子」。}

\textbf{籥舞笙鼓,樂既和奏。烝衎烈祖,以洽百禮。}{\footnotesize 秉籥而舞,與笙鼓相應。箋云籥,管也。殷人先求諸陽,故祭祀先奏樂,滌蕩其聲也。烝進、衎樂、烈美、洽合也。奏樂和,必進樂其先祖,於是又合見天下諸侯所獻之禮。}\textbf{百禮既至,有壬有林。}{\footnotesize 壬大、林君也。箋云壬,任也,謂卿大夫也。諸侯所獻之禮既陳於庭,有卿大夫,又有國君,言天下徧至,得萬國之歡心。}\textbf{錫爾純嘏,子孫其湛。}{\footnotesize 嘏,大也。箋云純,大也。嘏,謂尸與主人以福也。湛,樂也。王受神之福於尸,則王之子孫皆喜樂也。}\textbf{其湛曰樂,各奏爾能。賓載手仇,室人入又。}{\footnotesize 手,取也。室人,主人也。主人請射於賓,賓許諾,自取其匹而射,主人亦入于次,又射以耦賓也。箋云子孫各奏爾能者,謂既湛之後,各酌獻尸,尸酢而卒爵也。士之祭禮,上嗣舉奠,因而酌尸,天子則有子孫獻尸之禮,文王世子曰「其登餕獻受爵則以上嗣」是也。仇,讀曰㪺。室人,有室中之事者,謂佐食也。又,復也。賓手挹酒,室人復酌為加爵。}\textbf{酌彼康爵,以奏爾時。}{\footnotesize 酒所以安體也。時,中者也。箋云康,虛也。時,謂心所尊者也。加爵之間,賓與兄弟交錯相醻,卒爵者,酌之以其所尊,亦交錯而已,又無次也。}

\begin{quoting}有壬有林,\textbf{戴震}此以形容百禮既至,壬壬然盛大,林林然多而不亂。入又,又入也。\textbf{陳奐}賓與室人對稱,故傳以室人為主人。康爵,賈誼弔屈原賦「斡棄周鼎而寶康瓠」,史記集解「康瓠,大瓠」。\textbf{馬瑞辰}詩何以云「以奏爾時」,蓋飲不中者以致罰,正所以進中者以致慶耳。\end{quoting}

\textbf{賓之初筵,溫溫其恭。}{\footnotesize 箋云此復言初筵者,既祭,王與族人燕之筵也,王與族人燕,以異姓為賓。溫溫,柔和也。}\textbf{其未醉止,威儀反反。曰既醉止,威儀幡幡。舍其坐遷,屢舞僊僊。}{\footnotesize 反反,言重慎也。幡幡,失威儀也。遷徙、屢數也。僊僊然。箋云此言賓初即筵之時,能自敕戒以禮,至於旅酬而小人之態出,言王既不得君子以為賓,又不得有恆之人,所以敗亂天下,率如此也。}\textbf{其未醉止,威儀抑抑。曰既醉止,威儀怭怭。是曰既醉,不知其秩。}{\footnotesize 抑抑,慎密也。怭怭,媟嫚也。秩,常也。}

\begin{quoting}反反,釋文引韓詩作昄昄。\textbf{馬瑞辰}古者飲酒之禮取觶、奠觶皆坐,又凡禮盛者坐卒爵,其餘則皆立飲,又有升降、興拜、復席、復位諸禮,皆可以「遷」統之,舍其坐遷,蓋謂舍其當坐當遷之禮耳。\end{quoting}

\textbf{賓既醉止,載號載呶。亂我籩豆,屢舞僛僛。是曰既醉,不知其郵。側弁之俄,屢舞傞傞。}{\footnotesize 號呶,號呼讙呶也。僛僛,舞不能自正也。傞傞,不止也。箋云郵過、側傾也。俄,傾貌。此更言賓既醉而異章者,著為無筭爵以後也。}\textbf{既醉而出,並受其福。醉而不出,是謂伐德。飲酒孔嘉,維其令儀。}{\footnotesize 箋云出,猶去也。孔甚、令善也。賓醉則出,與主人俱有美譽,醉至若此,是誅伐其德也。飲酒而誠得嘉賓,則於禮有善威儀,武公見王之失禮,故以此言箴之。}

\begin{quoting}呶 \texttt{náo}。郵,段注以為同尤。傞傞 \texttt{suō},三家詩作姕姕。\end{quoting}

\textbf{凡此飲酒,或醉或否。既立之監,或佐之史。彼醉不臧,不醉反耻。}{\footnotesize 立酒之監,佐酒之史。箋云凡此者,凡此時天下之人也。飲酒於有醉者,有不醉者,則立監使視之,又助以史使督酒,欲令皆醉也,彼醉則已不善,人所非惡,反復取未醉者耻罰之,言此者,疾之也。}\textbf{式勿從謂,無俾大怠。匪言勿言,匪由勿語。}{\footnotesize 箋云式,讀曰慝。勿,猶無也。俾使、由從也。武公見時人多說醉者之狀,或以取怨致讐,故為設禁,醉者有惡過,女無就而謂之也,當防護之,無使顛仆,至於怠慢也,其所陳說,非所當說,無為人說之也,亦無從而行之也,亦無以語人也,皆為其聞之將恚怒也。}\textbf{由醉之言,俾出童羖。}{\footnotesize 羖,羊不童也。箋云女從行醉者之言,使女出無角之羖羊,脅以無然之物,使戒深也。羖羊之性,牝牡有角。}\textbf{三爵不識,矧敢多又。}{\footnotesize 箋云矧況、又復也。當言我於此醉者,飲三爵之不知,況能知其多復飲乎。三爵者,獻也、酬也、酢也。}

\begin{quoting}\textbf{馬瑞辰}古者飲酒皆立之監,以防失禮,惟老者有乞言之典,更佐以史,少者則否,故云「或佐之史」,監以察儀,史以記言,下文「式勿從謂,無俾大怠」,察儀之事也,「匪言勿言,匪由勿語」,乞言於老者而勉以慎言之詞也。又曰爾雅釋詁「謂,勤也」,勤為勤勞之勤,亦為相勸勉之勤,勿從謂者,勿從而勸勤之使更飲也,故即繼之以「無俾大怠」耳。又曰爾雅釋言「訊,言也」,廣雅「言,問也」,匪言勿言,上言字當讀為訊言之言,猶曾子事父母篇「弗訊不言」也,方言、廣雅並曰「由,式也」,式猶法也,匪由勿語,猶孝經「非法不道」也。大雅抑傳「童,羊之無角者也」。又,通侑,\textbf{姚際恆}謂三爵之禮亦不識,況敢又多飲乎。\end{quoting}

%\begin{flushright}甫田之什十篇、三十九章、二百九十六句\end{flushright}