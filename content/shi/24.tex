\chapter{生民之什詁訓傳第二十四}

\begin{quoting}\textbf{釋文}自生民至卷阿八篇,成王周公之正大雅。\end{quoting}

\section{生民}

%{\footnotesize 八章、四章章十句、四章章八句}

\textbf{生民,尊祖也。后稷生於姜嫄,文武之功起于后稷,故推以配天焉。}

\textbf{厥初生民,時維姜嫄。}{\footnotesize 生民,本后稷也。姜,姓也。后稷之母配高辛氏帝焉。箋云厥其、初始、時是也。言周之始祖,其生之者是姜嫄也。姜姓者,炎帝之後,有女名嫄,當堯之時,為高辛氏之世妃,本后稷之初生,故謂之生民。}\textbf{生民如何,克禋克祀,以弗無子。}{\footnotesize 禋敬、弗去也。去無子,求有子,古者必立郊禖焉,玄鳥至之日,以大牢祠于郊禖,天子親往,后妃率九嬪御,乃禮天子所御,帶以弓韣,授以弓矢,于郊禖之前。箋云克,能也。弗之言祓也。姜嫄之生后稷如何乎,乃禋祀上帝於郊禖,以祓除其無子之疾而得其福也。能者,言齊肅當神明意也。二王之後得用天子之禮。}\textbf{履帝武敏歆,攸介攸止。載震載夙,載生載育,時維后稷。}{\footnotesize 履,踐也。帝,高辛氏之帝也。武迹、敏疾也。從於帝而見于天,將事齊敏也。歆饗、介大也。攸止,福祿所止也。震動、夙早、育長也。后稷播百穀以利民。箋云帝,上帝也。敏,拇也。介,左右也。夙之言肅也。祀郊禖之時,時則有大神之迹,姜嫄履之,足不能滿,履其拇指之處,心體歆歆然,其左右所止住,如有人道感己者也,於是遂有身而肅戒不復御,後則生子而養長之,名曰棄,舜臣堯而舉之,是為后稷。}

\begin{quoting}\textbf{陳奐}緜「民之初生」傳「民,周民也」,此民亦為周民。周禮鄭玄注「禋 \texttt{yīn} 之言煙,周人尚臭,煙氣之臭聞者,積柴實牲體焉,或有玉帛,燔燎而升煙,所以報陽也」。弗,三家詩作祓,除也。介,同愒 \texttt{qì},說文「愒,息也」。震,同娠。夙,同肅。\end{quoting}

\textbf{誕彌厥月,先生如達。}{\footnotesize 誕大、彌終、達生也。姜嫄之子先生者也。箋云達,羊子也。大矣后稷之在其母,終人道十月而生,生如達之生,言易也。}\textbf{不坼不副,無菑無害。}{\footnotesize 言易也。凡人在母,母則病,生則坼副菑害其母,橫逆人道。}\textbf{以赫厥靈,上帝不寧。不康禋祀,居然生子。}{\footnotesize 赫,顯也。不寧,寧也。不康,康也。箋云康、寧皆安也。姜嫄以赫然顯著之徵,其有神靈審矣,此乃天帝之氣也,心猶不安之,又不安徒以禋祀而無人道,居默然自生子,懼時人不信也。}

\begin{quoting}誕,發語詞。坼 \texttt{chè}、副 \texttt{pì} 皆裂也。\end{quoting}

\textbf{誕寘之隘巷,牛羊腓字之。}{\footnotesize 誕大、寘置、腓辟、字愛也。天生后稷,異之於人,欲以顯其靈也。帝不順天,是不明也,故承天意而異之于天下。箋云天異之,故姜嫄置后稷於牛羊之徑,亦所以異之。}\textbf{誕寘之平林,會伐平林。}{\footnotesize 牛羊而辟人者,理也,置之平林,又為人所收取之。}\textbf{誕寘之寒冰,鳥覆翼之。}{\footnotesize 大鳥來,一翼覆之,一翼藉之,人而收取之,又其理也,故置之於寒冰。}\textbf{鳥乃去矣,后稷呱矣。}{\footnotesize 於是知有天異,往取之矣,后稷呱呱然而泣。}\textbf{實覃實訏,厥聲載路。}{\footnotesize 覃長、訏大、路大也。箋云實之言適也。覃謂始能坐也,訏謂張口嗚呼也,是時聲音則已大矣。}

\begin{quoting}腓,通庇。說文「字,乳也」。覃訏 \texttt{tán xū},言其哭聲長且亮也。\end{quoting}

\textbf{誕實匍匐,克岐克嶷,以就口食。}{\footnotesize 岐,知意也。嶷,識也。箋云能匍匐,則岐岐然意有所知也,其貌嶷嶷然,有所識別也,以此至于能就眾人口自食,謂六七歲時。}\textbf{蓺之荏菽,荏菽旆旆。禾役穟穟,麻麥幪幪,瓜瓞唪唪。}{\footnotesize 荏菽,戎菽也。旆旆然,長也。役,列也。穟穟,苗好美也。幪幪然,茂盛也。唪唪然,多實也。箋云蓺,樹也。戎菽,大豆也。就口食之時則有種殖之志,言天性也。}

\begin{quoting}役,說文引詩作穎,禾穎,即禾穗。唪 \texttt{běng}。\end{quoting}

\textbf{誕后稷之穡,有相之道。}{\footnotesize 相,助也。箋云大矣后稷之掌稼穡,有見助之道,謂若神助之力。}\textbf{茀厥豐草,種之黃茂。實方實苞,實種實褎,實發實秀,實堅實好,實穎實栗。即有邰家室。}{\footnotesize 茀,治也。黃,嘉穀也。茂,美也。方,極畝也。苞,本也。種,雜種也。褎,長也。發,盡發也。不榮而實曰秀。穎,垂穎也。栗,其實栗栗然。邰,姜嫄之國也。堯見天因邰而生后稷,故國后稷於邰,命使事天,以顯神順天命耳。箋云豐、苞亦茂也。方,齊等也。種,生不雜也。褎,枝葉長也。發,發管時也。栗,成就也。后稷教民除治茂草,使種黍稷,黍稷生則茂好,孰則大成,以此成功,堯改封於邰,就其成國之家室,無變更也。}

\begin{quoting}茀,同拂,廣雅「拂,除也、拔也」。\textbf{馬瑞辰}墨子明鬼篇「擇五穀之芳黃以為酒醴粢盛」,是五穀通可謂之黃。方,通放,萌芽出土也。苞,物叢生也。\textbf{馬瑞辰}種當讀如左傳「余髮如此種種」之種,程氏瑤田曰「種,出地短」是也。褎 \texttt{yòu}。即,往也。\end{quoting}

\textbf{誕降嘉種,維秬維秠,維穈維芑。}{\footnotesize 天降嘉種。秬,黑黍也。秠,一稃二米也。穈,赤苗也。芑,白苗也。箋云天應堯之顯后稷,為之下嘉種。}\textbf{恆之秬秠,是穫是畝。恆之穈芑,是任是負。以歸肇祀。}{\footnotesize 恆徧、肇始也。始歸郊祀也。箋云任,猶抱也。肇,郊之神位也。后稷以天為己下此四穀之故,則徧種之,成孰則穫而畝計之,抱負以歸於郊祀天。得祀天者,二王之後也。}

\begin{quoting}秠 \texttt{pī}。穈 \texttt{mén}。恆,同亘,徧也。\end{quoting}

\textbf{誕我祀如何,或舂或揄,或簸或蹂。釋之叟叟,烝之浮浮。}{\footnotesize 揄,抒臼也。或簸糠者,或蹂黍者。釋,淅米也。叟叟,聲也。浮浮,氣也。箋云蹂之言潤也。大矣,我后稷之祀天如何乎,美而將說其事也。舂而抒出之,簸之又潤濕之,將復舂之,趨於鑿也。釋之烝之,以為酒及簠簋之實。}\textbf{載謀載惟,取蕭祭脂。取羝以軷,載燔載烈。}{\footnotesize 嘗之日涖卜來歲之芟,獮之日涖卜來歲之戒,社之日涖卜來歲之稼,所以興來而繼往也,穀熟而謀,陳祭而卜矣。取蕭合黍稷,臭達牆屋,既奠而後爇蕭,合馨香也。羝羊,牡羊也。軷,道祭也。傅火曰燔,貫之加於火曰烈。箋云惟,思也。烈之言爛也。后稷既為郊祀之酒及其米,則諏謀其日,思念其禮,至其時,取蕭草與祭牲之脂,爇之於行神之位,馨香既聞,取羝羊之體以祭神,又燔烈其肉為尸羞焉。自此而往郊。}\textbf{以興嗣歲。}{\footnotesize 興來歲、繼往歲也。箋云嗣歲,今新歲也。以先歲之物齊敬犯軷而祀天者,將求新歲之豐年也,孟春之月令曰「乃擇元日,祈穀于上帝」。}

\begin{quoting}揄 \texttt{yóu},同舀,從臼中取米。烝,同蒸。蕭,香蒿,今名艾。軷 \texttt{bó},于省吾謂剝羊皮。\end{quoting}

\textbf{卬盛于豆,于豆于登,其香始升。上帝居歆,胡臭亶時。}{\footnotesize 卬,我也。木曰豆,瓦曰登。豆,薦菹醢也,登,大羹也。箋云胡之言何也。亶,誠也。我后稷盛菹醢之屬當於豆者於登者,其馨香始上行,上帝則安而歆享之,何芳臭之誠得其時乎,美之也。祀天用瓦豆,陶器質也。}\textbf{后稷肇祀,庶無罪悔,以迄于今。}{\footnotesize 迄,至也。箋云庶,眾也。后稷肇祀上帝於郊,而天下眾民咸得其所,無有罪過也,子孫蒙其福以至於今,故推以配天焉。}

\begin{quoting}卬,仰古字,\textbf{于省吾}古人祭祀時,設豆于俎几之上,祭者跪拜於神主之前,執燔烈之肉以上盛于豆,故曰仰盛于豆。居,語詞。\textbf{馬瑞辰}胡臭,謂芳臭之大。\end{quoting}

\section{行葦}

%{\footnotesize 八章、章四句,故言七章、二章章六句、五章章四句}

\textbf{行葦,忠厚也。周家忠厚,仁及草木,故能內睦九族,外尊事黃耇,養老乞言,以成其福祿焉。}{\footnotesize 九族,自己上至高祖、下至玄孫之親也。黃,黃髮也。耇,凍梨也。乞言,從求善言可以為政者,敦史受之。}

\textbf{敦彼行葦,牛羊勿踐履。方苞方體,維葉泥泥。}{\footnotesize 敦,聚貌。行,道也。葉初生泥泥。箋云苞,茂也。體,成形也。敦敦然道旁之葦,牧牛羊者毋使躐履折傷之,草物方茂盛,以其終將為人用,故周之先王為此愛之,況於人乎。}

\begin{quoting}敦 \texttt{tuán} 彼,即敦敦,重言也。泥,韓詩作苨。\end{quoting}

\textbf{戚戚兄弟,莫遠具爾。或肆之筵,或授之几。}{\footnotesize 戚戚,內相親也。肆,陳也。或陳設筵者,或授几者。箋云莫,無也。具,猶俱也。爾,謂進之也。王與族人燕,兄弟之親,無遠無近,俱揖而進之,年稚者為設筵而已,老者加之以几。}

\begin{quoting}漢書文三王傳引詩,顏注「爾,近也,言王之族親無疏遠,皆昵近也」。\end{quoting}

\textbf{肆筵設席,授几有緝御。}{\footnotesize 設席,重席也。緝御,踧踖之容也。箋云緝,猶續也。御,侍也。兄弟之老者,既為設重席授几,又有相續代而侍者,謂敦史也。}\textbf{或獻或酢,洗爵奠斝。}{\footnotesize 斝,爵也,夏曰醆,殷曰斝,周曰爵。箋云進酒於客曰獻,客答之曰酢。主人又洗爵醻客,客受而奠之,不舉也。用殷爵者,尊兄弟也。}

\textbf{醓醢以薦,或燔或炙。嘉殽脾臄,或歌或咢。}{\footnotesize 以肉曰醓醢。臄,函也。歌者,比於琴瑟也。徒擊鼓曰咢。箋云薦之禮,韭菹則醓醢也。燔用肉,炙用肝,以脾函為加,故謂之嘉。}

\begin{quoting}醓 \texttt{tǎn}。脾,同膍,即牛百葉。臄 \texttt{jué},牛舌。咢 \texttt{è}。\end{quoting}

\textbf{敦弓既堅,四鍭既鈞。舍矢既均,}{\footnotesize 敦弓,畫弓也,天子敦弓。鍭,矢參亭。已均,中蓺。箋云舍之言釋也。蓺,質也。周之先王將養老,先與群臣行射禮,以擇其可與者以為賓。}\textbf{序賓以賢。}{\footnotesize 言賓客次序皆賢。孔子射於矍相之圃,觀者如堵牆,射至於司馬,使子路執弓矢出,延射曰「奔軍之將、亡國之大夫與為人後者不入,其餘皆入」,蓋去者半,入者半,又使公罔之裘、序點揚觶而語,公罔之裘揚觶而語曰「幼壯孝弟、耆耋好禮、不從流俗、修身以俟死者不在此位」,蓋去者半,處者半,序點又揚觶而語曰「好學不倦、好禮不變、耄勤稱道不亂者不在此位也」,蓋僅有存焉。箋云序賓以賢,謂以射中多少為次第。}

\textbf{敦弓既句,既挾四鍭。}{\footnotesize 天子之弓合九而成規。箋云射禮,搢三挾一箇,言已挾四鍭,則已徧釋之。}\textbf{四鍭如樹,}{\footnotesize 言皆中也。}\textbf{序賓以不侮。}{\footnotesize 言其皆有賢才也。箋云不侮者,敬也,其人敬於禮則射多中。}

\begin{quoting}句,同彀,弓引滿也。\end{quoting}

\textbf{曾孫維主,酒醴維醹。酌以大斗,以祈黃耇。}{\footnotesize 曾孫,成王也。醹,厚也。大斗,長三尺也。祈,報也。箋云祈,告也。今我成王承先王之法度,為主人,亦既序賓矣,有醇厚之酒醴,以大斗酌而嘗之而美,故以告黃耇之人,徵而養之也。飲酒之禮曰「告於先生君子可也」。}

\begin{quoting}斗,同枓,說文「勺也」,\textbf{陳奐}酌以大斗者,言挹取酒醴用大枓以注尊中。\end{quoting}

\textbf{黃耇台背,以引以翼。}{\footnotesize 台背,大老也。引長、翼敬也。箋云台之言鮐也,大老則背有鮐文。既告老人,及其來也,以禮引之,以禮翼之。在前曰引,在旁曰翼。}\textbf{壽考維祺,以介景福。}{\footnotesize 祺,吉也。箋云介,助也。養老人而得吉,所以助大福也。}

\section{既醉}

%{\footnotesize 八章、章四句}

\textbf{既醉,太平也。醉酒飽德,人有士君子之行焉。}{\footnotesize 成王祭宗廟,旅醻下徧群臣,至于無筭爵,故云醉焉。乃見十倫之義,志意充滿,是謂之飽德。}

\begin{quoting}\textbf{林義光}此詩為工祝奉尸命以致嘏於主人之辭。禮運注「嘏,祝為尸致福于主人之辭也」。\end{quoting}

\textbf{既醉以酒,既飽以德。}{\footnotesize 既者,盡其禮,終其事。箋云禮謂旅醻之屬,事謂惠施先後及歸俎之類。}\textbf{君子萬年,介爾景福。}{\footnotesize 箋云君子,斥成王也。介助、景大也。成王,女有萬年之壽,天又助女以大福,謂五福也。}

\begin{quoting}以,其也,二字古通用。\end{quoting}

\textbf{既醉以酒,爾殽既將。}{\footnotesize 將,行也。箋云爾,女也。殽,謂牲體也。成王之為群臣俎實,以尊卑差次行之。}\textbf{君子萬年,介爾昭明。}{\footnotesize 箋云昭,光也。}

\begin{quoting}將,通臧,美也。\end{quoting}

\textbf{昭明有融,高朗令終。}{\footnotesize 融長、朗明也。始於饗燕,終於享祀。箋云有又、令善也。天既助女以光明之道,又使之長,有高明之譽而以善名終,是其長也。}\textbf{令終有俶,公尸嘉告。}{\footnotesize 俶,始也。公尸,天子以卿,言諸侯也。箋云俶,猶厚也。既始有善令,終又厚之,公尸以善言告之,謂嘏辭也。諸侯有功德者,入為天子卿大夫,故云公尸,公,君也。}

\textbf{其告維何,籩豆靜嘉。}{\footnotesize 恆豆之菹,水草之和也,其醢,陸產之物也。加豆,陸產也,其醢,水物也,籩豆之薦,水土之品也。不敢用常褻味而貴多品,所以交於神明者,言道之徧至也。箋云公尸所以善言告之,是何故乎,乃用籩豆之物,絜清而美,政平氣和所致故也。}\textbf{朋友攸攝,攝以威儀。}{\footnotesize 言相攝佐者以威儀也。箋云朋友謂群臣同志好者也。言成王之臣皆有仁孝士君子之行,其所以相攝佐威儀之事。}

\begin{quoting}自此以下皆嘏辭也。靜,同靖,善也。\end{quoting}

\textbf{威儀孔時,君子有孝子。}{\footnotesize 箋云孔,甚也。言成王之臣威儀甚得其宜,皆君子之人,有孝子之行。}\textbf{孝子不匱,永錫爾類。}{\footnotesize 匱竭、類善也。箋云永,長也。孝子之行非有竭極之時,長以與女之族類,謂廣之以教道天下也。春秋傳曰「穎考叔,純孝也,施及莊公」。}

\begin{quoting}匱,同墜,據金文,「不墜」為周人常用語。周語「叔向曰『類也者,不忝前哲之謂也』。」\end{quoting}

\textbf{其類維何,室家之壺。}{\footnotesize 壺,廣也。箋云壺之言梱也。其與女之族類云何乎,室家先以相梱致,已乃及於天下。}\textbf{君子萬年,永錫祚胤。}{\footnotesize 胤,嗣也。箋云永,長也。成王,女有萬年之壽,天又長予女福祚至于子孫。}

\begin{quoting}壺 \texttt{kǔn},本義為宮中道,引申為齊,即齊家之意。\end{quoting}

\textbf{其胤維何,天被爾祿。}{\footnotesize 祿,福也。箋云天予女福祚至于子孫云何乎,天覆被女以祿位,使錄臨天下。}\textbf{君子萬年,景命有僕。}{\footnotesize 僕,附也。箋云成王,女既有萬年之壽,天之大命又附著於女,謂使為政教。}

\begin{quoting}僕,奴僕。\end{quoting}

\textbf{其僕維何,釐爾女士。}{\footnotesize 釐,予也。箋云天之大命附著於女云何乎,予女以女而有士行者,謂生淑媛,使為之妃。}\textbf{釐爾女士,從以孫子。}{\footnotesize 箋云從,隨也。天既予女以女而有士行者,又使生賢知之子孫以隨之,謂傳世也。}

\begin{quoting}釐 \texttt{lí},同賚,賜予。女士,魯詩作士女,士、女相對為言,皆奴僕也。\end{quoting}

\section{鳧鷖}

%{\footnotesize 五章、章六句}

\textbf{鳧鷖,守成也。太平之君子能持盈守成,神祇祖考安樂之也。}{\footnotesize 君子,斥成王也。言君子者,太平之時則皆然,非獨成王也。}

\begin{quoting}上既醉言正祭,而此詩言繹祭也,穀梁傳宣八年「繹者,祭之旦日之享賓也」,即賓尸也。\end{quoting}

\textbf{鳧鷖在涇,公尸來燕來寧。}{\footnotesize 鳧,水鳥也。鷖,鳧屬。太平則萬物眾多。箋云涇,水中也,水鳥而居水中,猶人為公尸之在宗廟也,故以喻焉。祭祀既畢,明日又設禮而與尸燕。成王之時,尸來燕也,其心安,不以己實臣之故自嫌。言此者,美成王事尸之禮備。}\textbf{爾酒既清,爾殽既馨。公尸燕飲,福祿來成。}{\footnotesize 馨,香之遠聞也。箋云爾者,女成王也。女酒殽清美,以與公尸燕樂飲酒之故,祖考以福祿來成女。}

\begin{quoting}爾雅「水直波曰俓」,即涇字。\end{quoting}

\textbf{鳧鷖在沙,公尸來燕來宜。}{\footnotesize 沙,水旁也。宜,宜其事也。箋云水鳥以居水中為常,今出在水旁,喻祭四方百物之尸也。其來燕也,心自以為宜,亦不以己實臣自嫌也。}\textbf{爾酒既多,爾殽既嘉。}{\footnotesize 言酒品齊多而殽備美。}\textbf{公尸燕飲,福祿來為。}{\footnotesize 厚為孝子也。箋云為,猶助也,助成王也。}

\begin{quoting}\textbf{孔疏}為,謂助為也,論語「夫子為衛君乎?夫子不為也」,並以「為」為助。\end{quoting}

\textbf{鳧鷖在渚,公尸來燕來處。}{\footnotesize 渚,沚也。處,止也。箋云水中之有渚,猶平地之有丘也,喻祭天地之尸也。以配至尊之故,其來燕似若止得其處。}\textbf{爾酒既湑,爾殽伊脯。公尸燕飲,福祿來下。}{\footnotesize 箋云湑,酒之泲者也。天地之尸尊,事尊不以褻味,泲酒脯而已。}

\begin{quoting}\textbf{林義光}處字本義為依几而立,其引申義為安樂,故謂安樂為處,古人多以譽處連言,譽為豫之借字,爾雅「豫,樂也」,又云「豫,安也」,處即譽也。\end{quoting}

\textbf{鳧鷖在潨,公尸來燕來宗。}{\footnotesize 潨,水會也。宗,尊也。箋云潨,水外之高者也,有瘞埋之象,喻祭社稷山川之尸。其來燕也,有尊主人之意。}\textbf{既燕于宗,福祿攸降。公尸燕飲,福祿來崇。}{\footnotesize 崇,重也。箋云既,盡也。宗,社宗也。群臣下及民,盡有祭社之禮而燕飲焉,為福祿所下也。今王祭社,又以尸燕,福祿之來乃重厚也。天子以下,其社神同,故云然。}

\begin{quoting}潨 \texttt{zhōng},說文「小水入大水曰潨」。\end{quoting}

\textbf{鳧鷖在亹,公尸來止熏熏。}{\footnotesize 亹,山絕水也。熏熏,和說也。箋云亹之言門也。燕七祀之尸於門戶之外,故以喻焉。其來也,不敢當王之燕禮,故變言來止熏熏,坐不安之意。}\textbf{旨酒欣欣,燔炙芬芬。公尸燕飲,無有後艱。}{\footnotesize 欣欣然樂也。芬芬,香也。無有後艱,言不敢多祈也。箋云艱,難也。小神之尸卑,用美酒,有燔炙,可用褻味也,又不能致福祿,但令王自今無有後艱而已。}

\begin{quoting}漢書地理志顏注「亹 \texttt{mén} 者,水流夾山岸深若門也」,即峽口。來止,說文引詩亦作來燕。\textbf{俞樾}熏熏、欣欣字當互易。\end{quoting}

\section{假樂}

%{\footnotesize 四章、章六句}

\textbf{假樂,嘉成王也。}

\textbf{假樂君子,顯顯令德。宜民宜人,受祿于天。}{\footnotesize 假,嘉也。宜民宜人,宜安民、宜官人也。箋云顯,光也。天嘉樂成王,有光光之善德,安民官人皆得其宜,以受福祿於天。}\textbf{保右命之,自天申之。}{\footnotesize 申,重也。箋云成王之官人也,群臣保右而舉之,乃後命用之,又用天意申敕之,如舜之敕伯禹、伯夷之屬。}

\begin{quoting}假,同嘉,左傳、禮記引詩均作嘉。\end{quoting}

\textbf{干祿百福,子孫千億。穆穆皇皇,宜君宜王。}{\footnotesize 宜君王天下也。箋云干,求也。十萬曰億。天子穆穆,諸侯皇皇。成王行顯顯之令德,求祿得百福,其子孫亦勤行而求之,得祿千億,故或為諸侯,或為天子,言皆相勖以道。}\textbf{不愆不忘,率由舊章。}{\footnotesize 箋云愆過、率循也。成王之令德不過誤、不遺失,循用舊典之文章,謂周公之禮法。}

\begin{quoting}孟子離婁引詩云「遵先王之法而過者,未之有也」。\end{quoting}

\textbf{威儀抑抑,德音秩秩。無怨無惡,率由群匹。}{\footnotesize 抑抑,美也。秩秩,有常也。箋云抑抑,密也。秩秩,清也。成王立朝之威儀致密無所失,教令又清明,天下皆樂仰之,無有怨惡,循用群臣之賢者,其行能匹耦己之心。}\textbf{受福無疆,四方之綱。}

\begin{quoting}抑抑,同懿懿。\textbf{馬瑞辰}上章「率由舊章」為法祖,此章「率由群匹」為從眾。\end{quoting}

\textbf{之綱之紀,燕及朋友。}{\footnotesize 朋友,群臣也。箋云成王能為天下之綱紀,謂立法度以理治之也,其燕飲常與群臣,非徒樂族人而已。}\textbf{百辟卿士,媚于天子。不解于位,民之攸墍。}{\footnotesize 墍,息也。箋云百辟,畿內諸侯也。卿士,卿之有事也。媚,愛也。成王以恩意及群臣,群臣故皆愛之,不解於其職位,民之所以休息,由此也。}

\begin{quoting}卿士,左傳隱三年杜注「王卿之執政者」。墍 \texttt{xì},同愒,休息。\end{quoting}

\section{公劉}

%{\footnotesize 六章、章十句}

\textbf{公劉,召康公戒成王也。成王將涖政,戒以民事,美公劉之厚於民而獻是詩也。}{\footnotesize 公劉者,后稷之曾孫也,夏之始衰,見迫逐,遷於豳而有居民之道。成王始幼少,周公居攝政,反歸之,成王將蒞政,召公與周公相成王為左右,召公懼成王尚幼稚,不留意於治民之事,故作是詩美公劉,以深戒之。}

\textbf{篤公劉,匪居匪康。迺埸迺疆,迺積迺倉。迺裹餱糧,于橐于囊,思輯用光。}{\footnotesize 篤,厚也。公劉居於邰而遭夏人亂,迫逐公劉,公劉乃辟中國之難,遂平西戎而遷其民,邑於豳焉。迺埸迺疆,言修其疆埸也。迺積迺倉,言民事時和,國有積倉也。小曰橐,大曰囊。思輯用光,言民相與和睦,以顯於時也。箋云厚乎公劉之為君也,不以所居為居,不以所安為安,邰國乃有疆埸也,乃有積委及倉也,安安而能遷,積而能散,為夏人迫逐己之故,不忍鬪其民,乃裹糧食於囊橐之中,棄其餘而去,思在和其民人,用光大其道,為今子孫之基。}\textbf{弓矢斯張,干戈戚揚,爰方啟行。}{\footnotesize 戚,斧也。揚,鉞也。張其弓矢,秉其干戈戚揚,以方開道路去之豳,蓋諸侯之從者十有八國焉。箋云干,盾也。戈,句孑戟也。爰,曰也。公劉之去邰,整其師旅,設其兵器,告其士卒曰「為女方開道而行」,明己之遷非為迫逐之故,乃欲全民也。}

\begin{quoting}埸 \texttt{yì},田界。積,庾也,露天存糧處。有底曰囊,無底曰槖。\end{quoting}

\textbf{篤公劉,于胥斯原。既庶既繁,既順迺宣,而無永歎。}{\footnotesize 胥相、宣徧也。民無長歎,猶文王之無悔也。箋云于,於也。廣平曰原。厚乎公劉之於相此原地以居民,民既眾矣,既多矣,既順其事矣,又乃使之時耕,民皆安今之居而無長歎,思其舊時也。}\textbf{陟則在巘,復降在原。何以舟之,維玉及瑤,鞞琫容刀。}{\footnotesize 巘,小山,別於大山也。舟,帶也。瑤,言有美德也。下曰鞞,上曰琫,言德有度數也。容刀,言有武事也。箋云陟升、降下也。公劉之相此原地也,由原而升巘,復下在原,言反覆之,重居民也,民亦愛公劉之如是,故進玉瑤容刀之佩。}

\begin{quoting}\textbf{馬瑞辰}言民心既順其情,乃宣暢也,故下即言「而無永歎」矣。巘 \texttt{yǎn}。舟,\textbf{馬瑞辰}說文「帀徧也」,字通作周,帶周於身,故舟得訓帶。鞞琫,見瞻彼洛矣注。\textbf{陳奐}佩刀以為容飾,故曰容刀。\end{quoting}

\textbf{篤公劉,逝彼百泉,瞻彼溥原。迺陟南岡,乃覯于京。}{\footnotesize 溥大、覯見也。箋云逝往、瞻視、溥廣也。山脊曰岡,絕高為之京。厚乎公劉之相此原地也,往之彼百泉之間,視其廣原可居之處,乃升其南山之脊,乃見其可居者於京,謂可營立都邑之處。}\textbf{京師之野,于時處處,于時廬旅,于時言言,于時語語。}{\footnotesize 是京乃大眾所宜居之也。廬,寄也。直言曰言,論難曰語。箋云于於、時是也。京地乃眾民所宜居之野也,於是處其所當處者,廬舍其賓旅,言其所當言,語其所當語,謂安民館客,施教令也。}

\begin{quoting}\textbf{陳奐}山脊曰岡,岡即豳山之岡也,豳山在百泉之南,故曰南岡。\textbf{馬瑞辰}吳斗南曰「京者,地名,師者,都邑之稱,如洛邑亦稱洛師之類」,其說是也,京師連稱始此,後遂以名天子居焉。\end{quoting}

\textbf{篤公劉,于京斯依。蹌蹌濟濟,俾筵俾几。}{\footnotesize 箋云蹌蹌濟濟,士大夫之威儀也。俾,使也。厚乎公劉之居於此京,依而築宮室,其既成也,與群臣士大夫飲酒以落之,群臣則相使為公劉設几筵,使之升坐。}\textbf{既登乃依,乃造其曹,執豕于牢,酌之用匏。}{\footnotesize 賓已登席坐矣,乃依几矣。曹,群也。執豕于牢,新國則殺禮也。酌之用匏,儉以質也。箋云公劉既登堂負扆而立,群臣乃適其牧群,搏豕於牢中,以為飲酒之殽。酌酒以匏為爵,言忠敬也。}\textbf{食之飲之,君之宗之。}{\footnotesize 為之君,為之大宗也。箋云宗,尊也。公劉雖去邰國來遷,群臣從而君之尊之,猶在邰也。}

\begin{quoting}禮記曲禮「凡行容,大夫濟濟,士蹌蹌 \texttt{qiāng}」。造,三家詩作告,同祰,說文「吿祭也」。曹,同䄚,祭猪神。\textbf{馬瑞辰}據下云「執豕于牢」,知詩「乃造其曹」謂將用豕而告祭於豕先。\end{quoting}

\textbf{篤公劉,既溥既長,既景迺岡。相其陰陽,觀其流泉。}{\footnotesize 既景乃岡,考於日景,參之高岡。箋云厚乎公劉之居豳也,既廣其地之東西,又長其南北,既以日景定其經界於山之脊,觀相其陰陽寒煖所宜、流泉浸潤所及,皆為利民富國。}\textbf{其軍三單,度其隰原,徹田為糧。}{\footnotesize 三單,相襲也。徹,治也。箋云邰,后稷上公之封,大國之制三軍,以其餘卒為羨,今公劉遷於豳,民始從之,丁夫適滿三軍之數,單者,無羨卒也。度其隰與原田之多少,徹之使出稅以為國用。什一而稅謂之徹,魯哀公曰「二,吾猶不足,如之何其徹也」。}\textbf{度其夕陽,豳居允荒。}{\footnotesize 山西曰夕陽。荒,大也。箋云允,信也。夕陽者,豳之所處也。度其廣輪,豳之所處,信寬大也。}

\begin{quoting}單,同禪,代也。徹田,開墾田地。\end{quoting}

\textbf{篤公劉,于豳斯館。涉渭為亂,取厲取鍜。}{\footnotesize 館,舍也。正絕流曰亂。鍛,石也。箋云鍛,石所以為鍛質也。厚乎公劉,於豳地作此宮室,乃使人渡渭水,為舟絕流,而南取鍛厲斧斤之石,可以利器,用伐取材木,給築事也。}\textbf{止基迺理,爰眾爰有。夾其皇澗,遡其過澗。}{\footnotesize 皇,澗名也。溯,鄉也。過,澗名也。箋云爰,曰也。止基,作宮室之功止,而後疆理其田野,校其夫家人數日益多矣,器物有足矣,皆布居澗水之旁。}\textbf{止旅乃密,芮鞫之即。}{\footnotesize 密,安也。芮,水厓也。鞫,究也。箋云芮之言內也,水之內曰隩,水之外曰鞫。公劉居豳既安,軍旅之役止,士卒乃安,亦就澗水之內外而居,修田事也。}

\begin{quoting}止,「之」字之訛,此也。遡,面對。\textbf{胡承珙}凡水相入之處皆曰汭,其會合襟帶必有隈曲,內曲即芮,外曲即鞫。孔疏「此則來者愈眾,並水之內曲外曲而皆居之」。\end{quoting}

\section{泂酌}

%{\footnotesize 三章、章五句}

\textbf{泂酌,召康公戒成王也。言皇天親有德、饗有道也。}

\textbf{泂酌彼行潦,挹彼注茲,可以餴饎。}{\footnotesize 泂,遠也。行潦,流潦也。餴,餾也。饎,酒食也。箋云流潦,水之薄者也,遠酌取之,投大器之中,又挹之注之於此小器,而可以沃酒食之餴者,以有忠信之德、齊絜之誠以薦之故也,春秋傳曰「人不易物,惟德繄物」。}\textbf{豈弟君子,民之父母。}{\footnotesize 樂以彊教之,易以說安之,民皆有父之尊,有母之親。}

\begin{quoting}餴 \texttt{fēn},蒸也。饎 \texttt{chì}。呂氏春秋曰「愷悌君子,民之父母」,愷者大也,悌者長也,君子之德長且大者,則為民父母。\end{quoting}

\textbf{泂酌彼行潦,挹彼注茲,可以濯罍。}{\footnotesize 濯,滌也。罍,祭器。}\textbf{豈弟君子,民之攸歸。}

\textbf{泂酌彼行潦,挹彼注茲,可以濯溉。}{\footnotesize 溉,清也。}\textbf{豈弟君子,民之攸墍。}{\footnotesize 箋云墍,息也。}

\begin{quoting}溉,同概,古漆器酒尊。墍,見假樂卒章注。\end{quoting}

\section{卷阿}

%{\footnotesize 十章、六章章五句、四章章六句}

\textbf{卷阿,召康公戒成王也。言求賢用吉士也。}{\footnotesize 吉,猶善也。}

\textbf{有卷者阿,飄風自南。}{\footnotesize 興也。卷,曲也。飄風,回風也。惡人被德化而消,猶飄風之入曲阿也。箋云大陵曰阿。有大陵卷然而曲,回風從長養之方來入之。興者,喻王當屈體以待賢者,賢者則猥來就之,如飄風之入曲阿然。其來也,為長養民。}\textbf{豈弟君子,來游來歌,以矢其音。}{\footnotesize 矢,陳也。箋云王能待賢者如是,則樂易之君子來就王游,而歌以陳出其聲音,言其將以樂王也,感王之善心也。}

\begin{quoting}卷 \texttt{quán}。\end{quoting}

\textbf{伴奐爾游矣,優游爾休矣。}{\footnotesize 伴奐,廣大有文章也。箋云伴奐,自縱㢮之意也。賢者既來,王以才官秩之,各任其職,女則得伴奐而優游自休息也。孔子曰「無為而治者,其舜也與,恭己正南面而已」,言任賢故逸也。}\textbf{豈弟君子,俾爾彌爾性,似先公酋矣。}{\footnotesize 彌,終也。似,嗣也。酋,終也。箋云俾,使也。樂易之君子來在位,乃使女終女之性命,無困病之憂,嗣先君之功而終成之。}

\textbf{爾土宇昄章,亦孔之厚矣。}{\footnotesize 昄,大也。箋云土宇,謂居民以土地屋宅也。孔,甚也。女得賢者,與之為治,使居宅民大得其法則,王恩惠亦甚厚矣。勸之使然。}\textbf{豈弟君子,俾爾彌爾性,百神爾主矣。}{\footnotesize 箋云使女為百神主,謂群受饗而佐之。}

\begin{quoting}昄,音義同版,昄章,猶版圖也。孟子萬章「使之主祭而百神享之,是天受之」。\end{quoting}

\textbf{爾受命長矣,茀祿爾康矣。}{\footnotesize 茀,小也。箋云茀福、康安也。女得賢者,與之承順天地,則受久長之命,福祿又安女。}\textbf{豈弟君子,俾爾彌爾性,純嘏爾常矣。}{\footnotesize 嘏,大也。箋云純,大也。予福曰嘏。使女大受神之福以為常。}

\textbf{有馮有翼,有孝有德,以引以翼。}{\footnotesize 有馮有翼,道可馮依,以為輔翼也。引長、翼敬也。箋云馮,馮几也。翼,助也。有孝,斥成王也。有德,謂群臣也。王之祭祀,擇賢者以為尸,尊之,豫撰几擇佐食,廟中有孝子,有群臣,尸之入也,使祝贊道之,扶翼之,尸至,設几,佐食助之。尸者神象,故事之如祖考。}\textbf{豈弟君子,四方為則。}{\footnotesize 箋云則,法也。王之臣有是樂易之君子,則天下莫不放效以為法。}

\begin{quoting}\textbf{馬瑞辰}王尚書曰「爾雅,善父母為孝,推而言之,則為善德之通稱」,此詩有孝有德亦泛言有善有德,不必專指孝親言。行葦箋「在前曰引,在旁曰翼」。\end{quoting}

\textbf{顒顒卬卬,如圭如璋,令聞令望。}{\footnotesize 顒顒,溫貌。卬卬,盛貌。箋云令,善也。王有賢臣,與之以禮義相切瑳,體貌則顒顒然敬順,志氣則卬卬然高朗,如玉之圭璋也,人聞之則有善聲譽,人望之則有善威儀,德行相副。}\textbf{豈弟君子,四方為綱。}{\footnotesize 箋云綱者,能張眾目。}

\textbf{鳳皇于飛,翽翽其羽,亦集爰止。}{\footnotesize 鳳皇,靈鳥仁瑞也,雄曰鳳,雌曰皇。翽翽,眾多也。箋云翽翽,羽聲也。亦,亦眾鳥也。爰,于也。鳳皇往飛翽翽然,亦與眾鳥集於所止,眾鳥慕鳳皇而來,喻賢者所在,群士皆慕而往仕也。因時鳳皇至,故以喻焉。}\textbf{藹藹王多吉士,維君子使,媚于天子。}{\footnotesize 藹藹,猶濟濟也。箋云媚,愛也。王之朝多善士藹藹然,君子在上位者率化之,使之親愛天子,奉職盡力。}

\begin{quoting}維,同惟。\textbf{陳啟源}毛詩稽古編曰詩十章,凡十言君子,而其六則言豈弟,箋疏皆目大臣,即敘所謂賢也,敘所謂吉士,則經文之「藹藹吉士、藹藹吉人」也,能信任大賢,處之尊位,則眾賢滿朝矣。\end{quoting}

\textbf{鳳皇于飛,翽翽其羽,亦傅于天。}{\footnotesize 箋云傅,猶戾也。}\textbf{藹藹王多吉人,維君子命,媚于庶人。}{\footnotesize 箋云命,猶使也。善士親愛庶人,謂撫擾之,令不失職。}

\textbf{鳳皇鳴矣,于彼高岡。梧桐生矣,于彼朝陽。}{\footnotesize 梧桐,柔木也。山東曰朝陽。梧桐不生山岡,大平而後生朝陽。箋云鳳皇鳴于山脊之上者,居高視下,觀可集止,喻賢者待禮乃行,翔而後集。梧桐生者,猶明君出也,生於朝陽者,被溫仁之氣,亦君德也。鳳皇之性,非梧桐不棲,非竹實不食。}\textbf{菶菶萋萋,雝雝喈喈。}{\footnotesize 梧桐盛也,鳳皇鳴也,臣竭其力,則地極其化,天下和洽,則鳳皇樂德。箋云菶菶萋萋,喻君德盛也,雝雝喈喈,喻民臣和協。}

\begin{quoting}\textbf{姚際恆}詩意本是高岡朝陽,梧桐生其上,而鳳凰棲於梧桐之上鳴焉,今鳳凰言高岡,梧桐言朝陽,互見也。菶 \texttt{běng}。\end{quoting}

\textbf{君子之車,既庶且多。君子之馬,既閑且馳。}{\footnotesize 上能錫以車馬,行中節,馳中法也。箋云庶眾、閑習也。今賢者在位,王錫其車眾多矣,其馬又閑習於威儀能馳矣。大夫有乘馬,有貳車。}\textbf{矢詩不多,維以遂歌。}{\footnotesize 不多,多也。明王使公卿獻詩以陳其志,遂為工師之歌焉。箋云矢,陳也。我陳作此詩,不復多也,欲令遂為樂歌,王日聽之,則不損今之成功也。}

\section{民勞}

%{\footnotesize 五章、章十句}

\textbf{民勞,召穆公刺厲王也。}{\footnotesize 厲王,成王七世孫也。時賦斂重數,徭役煩多,人民勞苦,輕為姧宄,彊陵弱,眾暴寡,作寇害,故穆公以刺之。}

\begin{quoting}\textbf{釋文}從此至桑柔五篇,是厲王變大雅。\end{quoting}

\textbf{民亦勞止,汔可小康。惠此中國,以綏四方。}{\footnotesize 汔,危也。中國,京師也。四方,諸夏也。箋云汔,幾也。康、綏皆安也。惠,愛也。今周民罷勞矣,王幾可以小安之乎,愛京師之人以安天下,京師者,諸夏之根本。}\textbf{無縱詭隨,以謹無良。式遏寇虐,憯不畏明。}{\footnotesize 詭隨,詭人之善、隨人之惡者。以謹無良,慎小以懲大也。憯,曾也。箋云謹,猶慎也。良善、式用、遏止也。王為政無聽於詭人之善不肯行而隨人之惡者,以此敕慎無善之人,又用此止為寇虐、曾不畏敬明白之刑罪者,疾時有之。}\textbf{柔遠能邇,以定我王。}{\footnotesize 柔,安也。箋云能,猶侞也。邇,近也。安遠方之國,順侞其近者,當以此定我周家為王之功。言我者,同姓親也。}

\begin{quoting}汔,乞求。\textbf{于省吾}求可以小安,非有希於郅治之隆也,其意婉而諷矣。\textbf{陳奐}縱,當依左傳作從,箋以「聽」釋從,其字不誤也。\textbf{王引之}詭隨,疊韻字,不得分訓。\textbf{馬瑞辰}此詩每章皆言「詭隨」,而但曰無縱,可知其為小惡,下文云以謹,曰式遏,明其惡漸大矣。憯 \texttt{cǎn}。\textbf{王引之}古者謂相善為相能。\end{quoting}

\textbf{民亦勞止,汔可小休。惠此中國,以為民逑。}{\footnotesize 休,定也。逑,合也。箋云休,止息也。合,聚也。}\textbf{無縱詭隨,以謹惽怓。式遏寇虐,無俾民憂。}{\footnotesize 惽怓,大亂也。箋云惽怓,猶讙譁也,謂好爭訟者也。俾,使也。}\textbf{無棄爾勞,以為王休。}{\footnotesize 休,美也。箋云勞,猶功也。無廢女始時勤政事之功,以為女王之美。述其始時者,誘掖之也。}

\begin{quoting}惽怓 \texttt{mèn náo}。\end{quoting}

\textbf{民亦勞止,汔可小息。惠此京師,以綏四國。}{\footnotesize 息,止也。}\textbf{無縱詭隨,以謹罔極。式遏寇虐,無俾作慝。}{\footnotesize 慝,惡也。箋云罔無、極中也。無中,所行不得中正。}\textbf{敬慎威儀,以近有德。}{\footnotesize 求近德也。}

\begin{quoting}\textbf{姚際恆}末二句教之以近君子也。\end{quoting}

\textbf{民亦勞止,汔可小愒。惠此中國,俾民憂泄。}{\footnotesize 愒息、泄去也。箋云泄,猶出也、發也。}\textbf{無縱詭隨,以謹醜厲。式遏寇虐,無俾正敗。}{\footnotesize 醜眾、厲危也。箋云厲,惡也,春秋傳曰「其父為厲」。敗,壞也,無使先王之正道壞。}\textbf{戎雖小子,而式弘大。}{\footnotesize 戎,大也。箋云戎,猶女也。式,用也。弘,猶廣也。今王女雖小子自遇,而女用事於天下甚廣大也。易曰「君子出其言善則千里之外應之,況其邇者乎,出其言不善則千里之外違之,況其邇者乎」,是以此戒之。}

\begin{quoting}正,同政,下章同。\end{quoting}

\textbf{民亦勞止,汔可小安。惠此中國,國無有殘。}{\footnotesize 賊義曰殘。箋云王愛此京師之人,則天下邦國之君不為殘酷矣。}\textbf{無縱詭隨,以謹繾綣。式遏寇虐,無俾正反。}{\footnotesize 繾綣,反覆也。}\textbf{王欲玉女,是用大諫。}{\footnotesize 箋云玉者,君子比德焉,王乎,我欲令女如玉然,故作是詩,用大諫正女。此穆公至忠之言。}

\begin{quoting}\textbf{林義光}玉女,謂財貨與女色也。\end{quoting}

\section{板}

%{\footnotesize 八章、章八句}

\textbf{板,凡伯刺厲王也。}{\footnotesize 凡伯,周同姓,周公之胤也,入為王卿士。}

\begin{quoting}古本詩經皆作版,荀子楊倞注「大雅版之詩」可證。\end{quoting}

\textbf{上帝板板,下民卒癉。出話不然,為猶不遠。}{\footnotesize 板板,反也。上帝,以稱王者也。癉,病也。話,善言也。猶,道也。箋云猶,謀也。王為政反先王與天之道,天下之民盡病,其出善言而不行之也,此為謀不能遠圖,不知禍之將至。}\textbf{靡聖管管,不實於亶。}{\footnotesize 管管,無所依也。亶,誠也。箋云王無聖人之法度,管管然以心自恣,不能用實於誠信之言,言行相違也。}\textbf{猶之未遠,是用大諫。}{\footnotesize 猶,圖也。箋云王之謀不能圖遠,用是故我大諫王也。}

\begin{quoting}板板,魯詩作版版。卒癉 \texttt{cuì dǎn}。\end{quoting}

\textbf{天之方難,無然憲憲。天之方蹶,無然泄泄。}{\footnotesize 憲憲,猶欣欣也。蹶,動也。泄泄,猶沓沓也。箋云天,斥王也。王方欲艱難天下之民,又方變更先王之道,臣乎,女無憲憲然、無沓沓然為之制法度達其意,以成其惡。}\textbf{辭之輯矣,民之洽矣。辭之懌矣,民之莫矣。}{\footnotesize 輯和、洽合、懌說、莫定也。箋云辭,辭氣,謂政教也。王者政教和說順於民,則民心合定。此戒語時之大臣。}

\begin{quoting}泄 \texttt{yì},亦作呭,說文「呭,多言也」。辭,同辝,金文辝訓我。莫,同慔,勉力。\end{quoting}

\textbf{我雖異事,及爾同寮。我即爾謀,聽我囂囂。}{\footnotesize 寮,官也。囂囂,猶謷謷也。箋云及與、即就也。我雖與爾職事異者,乃與女同官,俱為卿士,我就女而謀,欲忠告以善道,女反聽我言,謷謷然不肯受。}\textbf{我言維服,勿以為笑。先民有言,詢于芻蕘。}{\footnotesize 芻蕘,薪采者。箋云服,事也。我所言乃今之急事,女無笑之,古之賢者有言,有疑事當與薪采者謀之,匹夫匹婦或知及之,況於我乎。}

\textbf{天之方虐,無然謔謔。老夫灌灌,小子蹻蹻。}{\footnotesize 謔謔然喜樂。灌灌,猶欵欵也。蹻蹻,驕貌。箋云今王方為酷虐之政,女無謔謔然以讒慝助之,老夫諫女欵欵然,自謂也,女反蹻蹻然如小子,不聽我言。}\textbf{匪我言耄,爾用憂謔。多將熇熇,不可救藥。}{\footnotesize 八十曰耄。熇熇然,熾盛也。箋云將,行也。今我言非老耄有失誤,乃告女用可憂之事,而女反如戲謔,多行熇熇慘毒之惡,誰能止其禍。}

\begin{quoting}憂,同優,優謔,調笑也。熇 \texttt{hè}。\end{quoting}

\textbf{天之方懠,無為夸毗。威儀卒迷,善人載尸。}{\footnotesize 懠,怒也。夸毗,體柔人也。箋云王方行酷虐之威怒,女無夸毗以形體順從之,君臣之威儀盡迷亂,賢人君子則如尸矣,不復言語。時厲王虐而弭謗。}\textbf{民之方殿屎,則莫我敢葵。喪亂蔑資,曾莫惠我師。}{\footnotesize 殿屎,呻吟也。蔑無、資財也。箋云葵,揆也。民方愁苦而呻吟,則忽然無有揆度知其然者,其遭喪禍,又素以賦斂空虛,無財貨以共其事,窮困如此,又曾不肯惠施以賙贍眾民,言無恩也。}

\begin{quoting}孔疏引孫炎曰「夸毗,屈己卑身以柔順人也」。載,則也。殿屎 \texttt{xī},同唸吚,呻吟也。資,同濟,止息。師,眾也。\end{quoting}

\textbf{天之牖民,如壎如篪,如璋如圭,如取如攜。}{\footnotesize 牖,道也。如壎如篪,言相和也。如璋如圭,言相合也。如取如攜,言必從也。箋云王之道民以禮義,則民和合而從之如此。}\textbf{攜無曰益,牖民孔易。民之多辟,無自立辟。}{\footnotesize 辟,法也。箋云易,易也。女攜掣民東與西與,民皆從女所為,無曰是何益為,道民在己,甚易也,民之行多為邪辟者,乃女君臣之過,無自謂所建為法也。}

\begin{quoting}牖,孔疏「牖與誘古字通用」。曰,語詞。益,同隘,阻礙。\textbf{馬瑞辰}謂邪僻之世,不可執法以繩人。\end{quoting}

\textbf{价人維藩,大師維垣,大邦維屏,大宗維翰。}{\footnotesize 价,善也。藩,屏也。垣,牆也。王者天下之大宗。翰,幹也。箋云价,甲也,被甲之人,謂卿士掌軍事者。大師,三公也。大邦,成國諸侯也。大宗,王之同姓之適子也。王當用公卿諸侯及宗室之貴者為藩屏垣幹,為輔弼,無疏遠之。}\textbf{懷德維寧,宗子維城。無俾城壞,無獨斯畏。}{\footnotesize 懷,和也。箋云斯,離也。和女德,無行酷虐之政,以安女國,以是為宗子之城,使免於難,遂行酷虐則禍及宗子,是謂城壞,城壞則乖離,而女獨居而畏矣。宗子謂王之適子。}

\begin{quoting}价,魯詩作介,善也。\end{quoting}

\textbf{敬天之怒,無敢戲豫。敬天之渝,無敢馳驅。}{\footnotesize 戲豫,逸豫也。馳驅,自恣也。箋云渝,變也。}\textbf{昊天曰明,及爾出王。昊天曰旦,及爾游衍。}{\footnotesize 王往、旦明、游行、衍溢也。箋云及,與也。昊天在上,人仰之皆謂之明,常與女出入往來,游溢相從,視女所行善惡,可不慎乎。}

\begin{quoting}敬,魯詩作畏。\end{quoting}

%\begin{flushright}生民之什十篇、六十五章、四百三十三句\end{flushright}