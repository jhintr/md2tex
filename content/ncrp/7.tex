\chapter{Lesson 7}

\section*{Reading 1}

“Etha tumhe, Kālāmā, mā anussavena, mā paramparāya, mā itikirāya, mā piṭakasampadānena, … mā samaṇo no garū ti. Yadā tumhe, Kālāmā, attanā va jāneyyātha ‘ime dhammā akusalā, ime dhammā sāvajjā, ime dhammā viññugarahitā, ime dhammā samattā samādinnā ahitāya dukkhāya saṃvattantī’ ti, atha tumhe, Kālāmā, pajaheyyātha. Taṃ kiṃ maññatha, Kālāmā, lobho purisassa ajjhattaṃ uppajjamāno uppajjati hitāya vā ahitāya vā” ti?

“Ahitāya, bhante.”

“Luddho panāyaṃ, Kālāmā, purisapuggalo lobhena abhibhūto pariyādinnacitto, pāṇaṃ pi hanati, adinnaṃ pi ādiyati, paradāraṃ pi gacchati, musā pi bhaṇati, paraṃ pi tathattāya samādapeti, yaṃ’sa hoti dīgharattaṃ ahitāya dukkhāyā” ti.

“Evaṃ, bhante.”

“Taṃ kiṃ maññatha, Kālāmā, doso purisassa ajjhattaṃ uppajjamāno uppajjati hitāya vā ahitāya vā” ti?

“Ahitāya, bhante.”

“Duṭṭho panāyaṃ, Kālāmā, purisapuggalo dosena abhibhūto pariyādinnacitto, pāṇaṃ pi hanati, adinnaṃ pi ādiyati, paradāraṃ pi gacchati, musā pi bhaṇati, paraṃ pi tathattāya samādapeti, yaṃ’sa hoti dīgharattaṃ ahitāya dukkhāyā” ti.

“Evaṃ, bhante.”

“Taṃ kiṃ maññatha, Kālāmā, moho purisassa ajjhattaṃ uppajjamāno uppajjati hitāya vā ahitāya vā” ti?

“Ahitāya, bhante.”

“Mūḷho panāyaṃ, Kālāmā, purisapuggalo mohena abhibhūto pariyādinnacitto, pāṇaṃ pi hanati, adinnaṃ pi ādiyati, paradāraṃ pi gacchati, musā pi bhaṇati, paraṃ pi tathattāya samādapeti, yaṃ’sa hoti dīgharattaṃ ahitāya dukkhāyā” ti.

“Evaṃ, bhante.”

“Taṃ kiṃ maññatha, Kālāmā, ime dhammā kusalā vā akusalā vā” ti?

“Akusalā, bhante.”

“Sāvajjā vā anavajjā vā” ti?

“Sāvajjā, bhante.”

“Viññugarahitā vā viññuppasatthā vā” ti?

“Viññugarahitā, bhante.”

“Samattā samādinnā ahitāya dukkhāya saṃvattanti, no vā? Kathaṃ vā ettha hotī” ti?

“Samattā, bhante, samādinnā ahitāya dukkhāya saṃvattantī ti. Evaṃ no ettha hotī” ti. (A 3.7.5)

\section*{Reading 2}

Nāhaṃ, brāhmaṇa, sabbaṃ diṭṭhaṃ bhāsitabbaṃ ti vadāmi, na panāhaṃ, brāhmaṇa, sabbaṃ diṭṭhaṃ na bhāsitabbaṃ ti vadāmi, nāhaṃ, brāhmaṇa, sabbaṃ sutaṃ bhāsitabbaṃ ti vadāmi, na panāhaṃ, brāhmaṇa, sabbaṃ sutaṃ na bhāsitabbaṃ ti vadāmi, nāhaṃ, brāhmaṇa, sabbaṃ mutaṃ bhāsitabbaṃ ti vadāmi, na panāhaṃ, brāhmaṇa, sabbaṃ mutaṃ na bhāsitabbaṃ ti vadāmi, nāhaṃ, brāhmaṇa, sabbaṃ viññātaṃ bhāsitabbaṃ ti vadāmi, na panāhaṃ, brāhmaṇa, sabbaṃ viññātaṃ na bhāsitabbaṃ ti vadāmi.

Yaṃ hi, brāhmaṇa, diṭṭhaṃ bhāsato akusalā dhammā abhivaḍḍhanti, kusalā dhammā parihāyanti, evarūpaṃ diṭṭhaṃ na bhāsitabbaṃ ti vadāmi. Yaṃ ca khv’assa, brāhmaṇa, diṭṭhaṃ abhāsato kusalā dhammā parihāyanti, akusalā dhammā abhivaḍḍhanti, evarūpaṃ diṭṭhaṃ bhāsitabbaṃ ti vadāmi.

Yaṃ hi, brāhmaṇa, sutaṃ bhāsato akusalā dhammā abhivaḍḍhanti, kusalā dhammā parihāyanti, evarūpaṃ sutaṃ na bhāsitabbaṃ ti vadāmi. Yaṃ ca khv’assa, brāhmaṇa, sutaṃ abhāsato kusalā dhammā parihāyanti, akusalā dhammā abhivaḍḍhanti, evarūpaṃ sutaṃ bhāsitabbaṃ ti vadāmi.

Yaṃ hi, brāhmaṇa, mutaṃ bhāsato akusalā dhammā abhivaḍḍhanti, kusalā dhammā parihāyanti, evarūpaṃ mutaṃ na bhāsitabbaṃ ti vadāmi. Yaṃ ca khv’assa, brāhmaṇa, mutaṃ abhāsato kusalā dhammā parihāyanti, akusalā dhammā abhivaḍḍhanti, evarūpaṃ mutaṃ bhāsitabbaṃ ti vadāmi.

Yaṃ hi, brāhmaṇa, viññātaṃ bhāsato akusalā dhammā abhivaḍḍhanti, kusalā dhammā parihāyanti, evarūpaṃ viññātaṃ na bhāsitabbaṃ ti vadāmi. Yaṃ ca khv’assa, brāhmaṇa, viññātaṃ abhāsato kusalā dhammā parihāyanti, akusalā dhammā abhivaḍḍhanti, evarūpaṃ viññātaṃ bhāsitabbaṃ ti vadāmī ti. (A 4.19.3)

\section*{Reading 3}

Saccaṃ bhaṇe na kujjheyya,\\
dajjā’ppasmiṃ pi yācito,\\
etehi tīhi ṭhānehi,\\
gacche devāna santike.

Kāyappakopaṃ rakkheyya,\\
kāyena saṃvuto siyā,\\
kāyaduccaritaṃ hitvā,\\
kāyena sucaritaṃ care.

Vacīpakopaṃ rakkheyya,\\
vācāya saṃvuto siyā,\\
vacīduccaritaṃ hitvā,\\
vācāya sucaritaṃ care.

Manopakopaṃ rakkheyya,\\
manasā saṃvuto siyā,\\
manoduccaritaṃ hitvā,\\
manasā sucaritaṃ care. (Dhp 17)

Yo pāṇam atipāteti,\\
musāvādaṃ ca bhāsati,\\
loke adinnaṃ ādiyati,\\
paradāraṃ ca gacchati.

Surāmerayapānaṃ ca,\\
yo naro anuyuñjati,\\
idh’evam eso lokasmiṃ,\\
mūlaṃ khaṇati attano. (Dhp 18)

\section*{Reading 4}

Sace labhetha nipakaṃ sahāyaṃ,\\
saddhiṃ caraṃ sādhuvihāridhīraṃ,\\
Abhibhuyya sabbāni parissayāni,\\
careyya ten’attamano satīmā.

No ce labhetha nipakaṃ sahāyaṃ,\\
saddhiṃ caraṃ sādhuvihāridhīraṃ,\\
Rājā va raṭṭhaṃ vijitaṃ pahāya,\\
eko care mātaṅg’araññe va nāgo. (Dhp 23)

\section*{Further Reading 1}

Tayo’me, brāhmaṇa, aggī pahātabbā parivajjetabbā, na sevitabbā. Katame tayo? Rāgaggi, dosaggi, mohaggi.

Kasmā cāyaṃ, brāhmaṇa, rāgaggi pahātabbo parivajjetabbo, na sevitabbo? Ratto kho, brāhmaṇa, rāgena abhibhūto pariyādinnacitto kāyena duccaritaṃ carati, vācāya duccaritaṃ carati, manasā duccaritaṃ carati. So kāyena duccaritaṃ caritvā, vācāya duccaritaṃ caritvā, manasā duccaritaṃ caritvā kāyassa bhedā paraṃ maraṇā apāyaṃ duggatiṃ vinipātaṃ nirayaṃ upapajjati. Tasmāyaṃ rāgaggi pahātabbo parivajjetabbo, na sevitabbo.

Kasmā cāyaṃ, brāhmaṇa, dosaggi pahātabbo parivajjetabbo, na sevitabbo? Duṭṭho kho, brāhmaṇa, dosena abhibhūto pariyādinnacitto kāyena duccaritaṃ carati, vācāya duccaritaṃ carati, manasā duccaritaṃ carati. So kāyena duccaritaṃ caritvā, vācāya duccaritaṃ caritvā, manasā duccaritaṃ caritvā kāyassa bhedā paraṃ maraṇā apāyaṃ duggatiṃ vinipātaṃ nirayaṃ upapajjati. Tasmāyaṃ dosaggi pahātabbo parivajjetabbo, na sevitabbo.

Kasmā cāyaṃ, brāhmaṇa, mohaggi pahātabbo parivajjetabbo, na sevitabbo? Mūḷho kho, brāhmaṇa, mohena abhibhūto pariyādinnacitto kāyena duccaritaṃ carati, vācāya duccaritaṃ carati, manasā duccaritaṃ carati. So kāyena duccaritaṃ caritvā, vācāya duccaritaṃ caritvā, manasā duccaritaṃ caritvā kāyassa bhedā paraṃ maraṇā apāyaṃ duggatiṃ vinipātaṃ nirayaṃ upapajjati. Tasmāyaṃ mohaggi pahātabbo parivajjetabbo, na sevitabbo.

Ime kho tayo, brāhmaṇa, aggī pahātabbā parivajjetabbā, na sevitabbā. (A 7.5.4 Dutiyaaggisuttaṃ)

\section*{Further Reading 2}

Rājā āha “Bhante Nāgasena, kiṃlakkhaṇā paññā” ti?

“Pubbeva kho, mahārāja, mayā vuttaṃ ‘chedanalakkhaṇā paññā’ ti, api ca obhāsanalakkhaṇā paññā” ti.

“Kathaṃ, bhante, obhāsanalakkhaṇā paññā” ti?

“Paññā, mahārāja, uppajjamānā avijjandhakāraṃ vidhameti, vijjobhāsaṃ janeti, ñāṇālokaṃ vidaṃseti, ariyasaccāni pākaṭāni karoti, tato yogāvacaro ‘aniccan’ ti vā ‘dukkhan’ ti vā ‘anattā’ ti vā sammappaññāya passatī” ti.

“Opammaṃ karohī” ti.

“Yathā, mahārāja, puriso andhakāre gehe padīpaṃ paveseyya, paviṭṭho padīpo andhakāraṃ vidhameti, obhāsaṃ janeti, ālokaṃ vidaṃseti, rūpāni pākaṭāni karoti, evameva kho, mahārāja, paññā uppajjamānā avijjandhakāraṃ vidhameti, vijjobhāsaṃ janeti, ñāṇālokaṃ vidaṃseti, ariyasaccāni pākaṭāni karoti, tato yogāvacaro ‘aniccan’ ti vā ‘dukkhan’ ti vā ‘anattā’ ti vā sammappaññāya passati. Evaṃ kho, mahārāja, obhāsanalakkhaṇā paññā” ti.

“Kallo’si, bhante Nāgasenā” ti.

(Miln III.1.14 Paññālakkhaṇapañho)

\section*{Further Reading 3}

“Bhante Nāgasena, nav’ime puggalā mantitaṃ guyhaṃ vivaranti na dhārenti. Katame nava? Rāgacarito, dosacarito, mohacarito, bhīruko, āmisagaruko, itthī, soṇḍo, paṇḍako, dārako” ti.

Thero āha “Tesaṃ ko doso” ti?

“Rāgacarito, bhante Nāgasena, rāgavasena mantitaṃ guyhaṃ vivarati na dhāreti, dosacarito, bhante, dosavasena mantitaṃ guyhaṃ vivarati na dhāreti, mūḷho mohavasena mantitaṃ guyhaṃ vivarati na dhāreti, bhīruko bhayavasena mantitaṃ guyhaṃ vivarati na dhāreti, āmisagaruko āmisahetu mantitaṃ guyhaṃ vivarati na dhāreti, itthī ittaratāya mantitaṃ guyhaṃ vivarati na dhāreti, soṇḍiko surālolatāya mantitaṃ guyhaṃ vivarati na dhāreti, paṇḍako anekaṃsikatāya mantitaṃ guyhaṃ vivarati na dhāreti, dārako capalatāya mantitaṃ guyhaṃ vivarati na dhāreti.” - Bhavatīha

Ratto duṭṭho ca mūḷho ca,\\
bhīru āmisagaruko,\\
itthī soṇḍo paṇḍako ca,\\
navamo bhavati dārako.

Nav’ete puggalā loke,\\
ittarā calitā calā,\\
etehi mantitaṃ guyhaṃ,\\
khippaṃ bhavati pākaṭan ti.

(Miln IV)

\section*{Further Reading 4}

Middhī yadā hoti mahagghaso ca,\\
niddāyitā samparivattasāyī,\\
Mahāvarāho va nivāpapuṭṭho,\\
punappunaṃ gabbham upeti mando.

Appamādaratā hotha,\\
sacittam anurakkhatha,\\
duggā uddharath’attānaṃ,\\
paṅke sanno va kuñjaro. (Dhp 23)