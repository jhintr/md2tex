\chapter{Lesson 2}

\section*{Reading 1}

Kiccho manussapaṭilābho,\\
kicchaṃ maccānaṃ jīvitaṃ,\\
kicchaṃ saddhammassavanaṃ,\\
kiccho buddhānamuppādo.

Sabbapāpassa akaraṇaṃ,\\
kusalassa upasampadā,\\
sacittapariyodapanaṃ,\\
etaṃ buddhāna sāsanaṃ. \hfill(Dhp 14)

Na hi verena verāni,\\
sammantīdha kudācanaṃ,\\
averena ca sammanti,\\
esa dhammo sanantano. \hfill(Dhp 1)

\section*{Reading 2}

Tīhi, bhikkhave, aṅgehi samannāgato pāpaṇiko abhabbo anadhigataṃ vā bhogaṃ adhigantuṃ, adhigataṃ vā bhogaṃ phātiṃ kātuṃ. Katamehi tīhi? Idha, bhikkhave, pāpaṇiko pubbaṇhasamayaṃ na sakkaccaṃ kammantaṃ adhiṭṭhāti, majjhaṇhikasamayaṃ na sakkaccaṃ kammantaṃ adhiṭṭhāti, sāyaṇhasamayaṃ na sakkaccaṃ kammantaṃ adhiṭṭhāti. Imehi kho, bhikkhave, tīhi aṅgehi samannāgato pāpaṇiko abhabbo anadhigataṃ vā bhogaṃ adhigantuṃ, adhigataṃ vā bhogaṃ phātiṃ kātuṃ.

Evameva kho, bhikkhave, tīhi dhammehi samannāgato bhikkhu abhabbo anadhigataṃ vā kusalaṃ dhammaṃ adhigantuṃ, adhigataṃ vā kusalaṃ dhammaṃ phātiṃ kātuṃ. Katamehi tīhi? Idha, bhikkhave, bhikkhu pubbaṇhasamayaṃ na sakkaccaṃ samādhinimittaṃ adhiṭṭhāti, majjhaṇhikasamayaṃ na sakkaccaṃ samādhinimittaṃ adhiṭṭhāti, sāyaṇhasamayaṃ na sakkaccaṃ samādhinimittaṃ adhiṭṭhāti. Imehi kho, bhikkhave, tīhi dhammehi samannāgato bhikkhu abhabbo anadhigataṃ vā kusalaṃ dhammaṃ adhigantuṃ, adhigataṃ vā kusalaṃ dhammaṃ phātiṃ kātuṃ.

Tīhi, bhikkhave, aṅgehi samannāgato pāpaṇiko bhabbo anadhigataṃ vā bhogaṃ adhigantuṃ, adhigataṃ vā bhogaṃ phātiṃ kātuṃ. Katamehi tīhi? Idha, bhikkhave, pāpaṇiko pubbaṇhasamayaṃ sakkaccaṃ kammantaṃ adhiṭṭhāti, majjhaṇhikasamayaṃ… pe… sāyaṇhasamayaṃ sakkaccaṃ kammantaṃ adhiṭṭhāti. Imehi kho, bhikkhave, tīhi aṅgehi samannāgato pāpaṇiko bhabbo anadhigataṃ vā bhogaṃ adhigantuṃ, adhigataṃ vā bhogaṃ phātiṃ kātuṃ.

Evameva kho, bhikkhave, tīhi dhammehi samannāgato bhikkhu bhabbo anadhigataṃ vā kusalaṃ dhammaṃ adhigantuṃ, adhigataṃ vā kusalaṃ dhammaṃ phātiṃ kātuṃ. Katamehi tīhi? Idha, bhikkhave, bhikkhu pubbaṇhasamayaṃ sakkaccaṃ samādhinimittaṃ adhiṭṭhāti, majjhaṇhikasamayaṃ… pe… sāyaṇhasamayaṃ sakkaccaṃ samādhinimittaṃ adhiṭṭhāti. Imehi kho, bhikkhave, tīhi dhammehi samannāgato bhikkhu bhabbo anadhigataṃ vā kusalaṃ dhammaṃ adhigantuṃ, adhigataṃ vā kusalaṃ dhammaṃ phātiṃ kātun ti. \hfill(A 3.19)

\section*{Reading 3}

Evameva kho, bhikkhave, appakā te sattā ye manussesu paccājāyanti; atha kho ete’va sattā bahutarā ye aññatra manussehi paccājāyanti.

Evameva kho, bhikkhave, appakā te sattā ye majjhimesu janapadesu paccājāyanti; atha kho ete’va sattā bahutarā ye paccantimesu janapadesu paccājāyanti.

Evameva kho, bhikkhave, appakā te sattā ye paññavanto, ajaḷā, aneḷamūgā paṭibalā subhāsitadubbhāsitassa atthamaññātuṃ; atha kho ete’va sattā bahutarā ye duppaññā jaḷā eḷamūgā na paṭibalā subhāsitadubbhāsitassa atthamaññātuṃ.

Evameva kho, bhikkhave, appakā te sattā ye ariyena paññācakkhunā samannāgatā; atha kho ete’va sattā bahutarā ye avijjāgatā sammūḷhā.

Evameva kho, bhikkhave, appakā te sattā ye labhanti tathāgataṃ dassanāya; atha kho ete’va sattā bahutarā ye na labhanti tathāgataṃ dassanāya.

Evameva kho, bhikkhave, appakā te sattā ye labhanti tathāgatappaveditaṃ dhammavinayaṃ savaṇāya; atha kho ete’va sattā bahutarā, ye na labhanti tathāgatappaveditaṃ dhammavinayaṃ savaṇāya. \hfill(A 1.16.4)

\section*{Further Reading 1}

Tīṇi’māni, bhikkhave, nidānāni kammānaṃ samudayāya.

Katamāni tīṇi? Lobho nidānaṃ kammānaṃ samudayāya, doso nidānaṃ kammānaṃ samudayāya, moho nidānaṃ kammānaṃ samudayāya.

Yaṃ, bhikkhave, lobhapakataṃ kammaṃ lobhajaṃ lobhanidānaṃ lobhasamudayaṃ, taṃ kammaṃ akusalaṃ, taṃ kammaṃ sāvajjaṃ, taṃ kammaṃ dukkhavipākaṃ, taṃ kammaṃ kammasamudayāya saṃvattati; na taṃ kammaṃ kammanirodhāya saṃvattati.

Yaṃ, bhikkhave, dosapakataṃ kammaṃ dosajaṃ dosanidānaṃ dosasamudayaṃ, taṃ kammaṃ akusalaṃ, taṃ kammaṃ sāvajjaṃ, taṃ kammaṃ dukkhavipākaṃ, taṃ kammaṃ kammasamudayāya saṃvattati; na taṃ kammaṃ kammanirodhāya saṃvattati.

Yaṃ, bhikkhave, mohapakataṃ kammaṃ mohajaṃ mohanidānaṃ mohasamudayaṃ, taṃ kammaṃ akusalaṃ, taṃ kammaṃ sāvajjaṃ, taṃ kammaṃ dukkhavipākaṃ, taṃ kammaṃ kammasamudayāya saṃvattati. Na taṃ kammaṃ kammanirodhāya saṃvattati.

Imāni kho, bhikkhave, tīṇi nidānāni kammānaṃ samudayāya.

Tīṇi’māni bhikkhave nidānāni kammānaṃ samudayāya.

Katamāni tīṇi? Alobho nidānaṃ kammānaṃ samudayāya, adoso nidānaṃ kammānaṃ samudayāya, amoho nidānaṃ kammānaṃ samudayāya.

Yaṃ, bhikkhave, alobhapakataṃ kammaṃ alobhajaṃ alobhanidānaṃ alobhasamudayaṃ, taṃ kammaṃ kusalaṃ, taṃ kammaṃ anavajjaṃ, taṃ kammaṃ sukhavipākaṃ, taṃ kammaṃ kammanirodhāya saṃvattati; na taṃ kammaṃ kammasamudayāya saṃvattati.

Yaṃ, bhikkhave, adosapakataṃ kammaṃ adosajaṃ adosanidānaṃ adosasamudayaṃ, taṃ kammaṃ kusalaṃ, taṃ kammaṃ anavajjaṃ, taṃ kammaṃ sukhavipākaṃ, taṃ kammaṃ kammanirodhāya saṃvattati; na taṃ kammaṃ kammasamudayāya saṃvattati.

Yaṃ, bhikkhave, amohapakataṃ kammaṃ amohajaṃ amohanidānaṃ amohasamudayaṃ, taṃ kammaṃ kusalaṃ, taṃ kammaṃ anavajjaṃ, taṃ kammaṃ sukhavipākaṃ, taṃ kammaṃ kammanirodhāya saṃvattati; na taṃ kammaṃ kammasamudayāya saṃvattati.

Imāni kho, bhikkhave, tīṇi nidānāni kammānaṃ samudayāyā ti. \hfill(A 3.11.9)

\section*{Further Reading 2}

Pañcahi, bhikkhave, dhammehi samannāgato bhikkhu cavati, nappatiṭṭhāti saddhamme.

Katamehi pañcahi?

Assaddho, bhikkhave, bhikkhu cavati, nappatiṭṭhāti saddhamme. Ahiriko, bhikkhave, bhikkhu cavati, nappatiṭṭhāti saddhamme. Anottappī, bhikkhave, bhikkhu cavati, nappatiṭṭhāti saddhamme. Kusīto, bhikkhave, bhikkhu cavati, nappatiṭṭhāti saddhamme. Duppañño, bhikkhave, bhikkhu cavati, nappatiṭṭhāti saddhamme.

Imehi kho, bhikkhave, pañcahi dhammehi samannāgato bhikkhu cavati, nappatiṭṭhāti saddhamme.

Pañcahi, bhikkhave, dhammehi samannāgato bhikkhu na cavati, patiṭṭhāti saddhamme.

Katamehi pañcahi?

Saddho, bhikkhave, bhikkhu na cavati, patiṭṭhāti saddhamme. Hirimā, bhikkhave, bhikkhu na cavati, patiṭṭhāti saddhamme. Ottappī, bhikkhave, bhikkhu na cavati, patiṭṭhāti saddhamme. Āraddhaviriyo, bhikkhave, bhikkhu na cavati, patiṭṭhāti saddhamme. Paññavā, bhikkhave, bhikkhu na cavati, patiṭṭhāti saddhamme.

Imehi kho, bhikkhave, pañcahi dhammehi samannāgato bhikkhū na cavati, patiṭṭhāti saddhamme.