\chapter{Lesson 3}

\section*{Reading 1}

“Bhante Nāgasena, atthi koci satto, yo imamhā kāyā aññaṃ kāyaṃ saṅkamatī?” ti.

“Na hi, mahārājā” ti.

“Yadi, bhante Nāgasena, imamhā kāyā aññaṃ kāyaṃ saṅkamanto natthi, nanu mutto bhavissati pāpakehi kammehī?” ti.

“Āma, mahārāja. Yadi na paṭisandaheyya, mutto bhavissati pāpakehi kammehi. Yasmā ca kho, mahārāja, paṭisandahati, tasmā na parimutto pāpakehi kammehī” ti. \hfill(Miln III.5.7 Buddhavaggo, Aññakāyasaṅkamanapañho)

\begin{center}
    * * * * * * *
\end{center}

“Bhante Nāgasena, na ca saṅkamati, paṭisandahati cā?” ti.

“Āma, mahārāja, na ca saṅkamati paṭisandahati cā” ti.

“Katham, bhante Nāgasena, na ca saṅkamati paṭisandahati ca? Opammaṃ karohī” ti.

“Yathā, mahārāja, kocideva puriso padīpato padīpaṃ padīpeyya, kinnu kho so, mahārāja, padīpo padīpamhā saṅkamanto?” ti.

“Na hi bhante” ti.

“Evameva kho, mahārāja, na ca saṅkamati paṭisandahati cā” ti. \hfill(Miln III.5.5 Buddhavaggo, Asaṅkamanapaṭisandahanapañho)

\section*{Reading 2}

“Taṃ kiṃ maññatha, Sāḷhā, atthi lobho” ti?

“Evaṃ, bhante.”

“Abhijjhā ti kho ahaṃ, Sāḷhā, etamatthaṃ vadāmi. Luddho kho ayaṃ, Sāḷhā, abhijjhālū pāṇam pi hanati, adinnam pi ādiyati, paradāram pi gacchati, musā pi bhaṇati… yaṃ’sa hoti dīgharattaṃ ahitāya dukkhāyā” ti.

“Evaṃ, bhante.”

“Taṃ kim maññatha, Sāḷhā, atthi doso” ti?

“Evaṃ, bhante.”

“Byāpādo ti kho ahaṃ, Sāḷhā, etamatthaṃ vadāmi. Duṭṭho kho ayaṃ, Sāḷhā, byāpannacitto pāṇam pi hanati, adinnam pi ādiyati, paradāram pi gacchati, musā pi bhaṇati… yaṃ’sa hoti dīgharattaṃ ahitāya dukkhāyā” ti.

“Evaṃ, bhante.”

“Taṃ kim maññatha, Sāḷhā, atthi moho” ti?

“Evaṃ, bhante.”

“Avijjā ti kho ahaṃ, Sāḷhā, etamatthaṃ vadāmi. Mūḷho kho ayaṃ, Sāḷhā, avijjāgato pāṇam pi hanati, adinnam pi ādiyati, paradāram pi gacchati, musā pi bhaṇati… yaṃ’sa hoti dīgharattaṃ ahitāya dukkhāyā” ti.

“Evaṃ, bhante.”

“Taṃ kiṃ maññatha, Sāḷhā, ime dhammā kusalā vā akusalā vā” ti?

“Akusalā, bhante.”

“Sāvajjā vā anavajjā vā” ti?

“Sāvajjā, bhante.”

“Viññugarahitā vā viññuppasatthā vā” ti?

“Viññugarahitā, bhante.” \hfill(A 3.7.6)

\section*{Reading 3}

Yasmā ca kho, bhikkhave, sakkā akusalaṃ pajahituṃ, tasmāhaṃ evaṃ vadāmi “akusalaṃ, bhikkhave, pajahathā” ti. Akusalaṃ ca h’idaṃ, bhikkhave, pahīnaṃ ahitāya, dukkhāya saṃvatteyya, nāhaṃ evaṃ vadeyyaṃ “akusalaṃ, bhikkhave, pajahathā” ti. Yasmā ca kho, bhikkhave, akusalaṃ pahīnaṃ hitāya sukhāya saṃvattati, tasmāhaṃ evaṃ vadāmi “akusalaṃ, bhikkhave, pajahathā” ti.

Kusalaṃ, bhikkhave, bhāvetha. Sakkā, bhikkhave, kusalaṃ bhāvetuṃ. … Yasmā ca kho, bhikkhave, sakkā kusalaṃ bhāvetuṃ, tasmāhaṃ evaṃ vadāmi “kusalaṃ, bhikkhave, bhāvethā” ti. Kusalaṃ ca h’idaṃ, bhikkhave, bhāvitaṃ ahitāya, dukkhāya saṃvatteyya, nāhaṃ evaṃ vadeyyaṃ “kusalaṃ, bhikkhave, bhāvethā” ti. Yasmā ca kho, bhikkhave, kusalaṃ bhāvitaṃ hitāya, sukhāya saṃvattati, tasmāhaṃ evaṃ vadāmi “kusalaṃ, bhikkhave, bhāvethā” ti. \hfill(A 2.2.19)

\section*{Further Reading 1}

“Taṃ kiṃ maññatha, Sāḷhā, atthi alobho” ti?

“Evam, bhante.”

“Anabhijjhā ti kho ahaṃ, Sāḷhā, etamatthaṃ vadāmi. Aluddho kho ayaṃ, Sāḷhā, anabhijjhālū n’eva pāṇaṃ hanati, na adinnaṃ ādiyati, na paradāraṃ gacchati, na musā bhaṇati, param pi na tathattāya samādapeti, yaṃ’sa hoti dīgharattaṃ hitāya sukhāyā” ti.

“Evam, bhante.”

“Taṃ kiṃ maññatha, Sāḷhā, atthi adoso” ti?

“Evam, bhante.”

“Abyāpādo ti kho ahaṃ, Sāḷhā, etamatthaṃ vadāmi. Aduṭṭho kho ayaṃ, Sāḷhā, abyāpannacitto n’eva pāṇaṃ hanati, na adinnaṃ ādiyati, na paradāraṃ gacchati, na musā bhaṇati, param pi na tathattāya samādapeti, yaṃ’sa hoti dīgharattaṃ hitāya sukhāyā” ti.

“Evam, bhante.”

“Taṃ kim maññatha, Sāḷhā, atthi amoho” ti?

“Evam, bhante.”

“Vijjā ti kho ahaṃ, Sāḷhā, etamatthaṃ vadāmi. Amūḷho kho ayaṃ, Sāḷhā, vijjāgato n’eva pāṇaṃ hanati, na adinnaṃ ādiyati, na paradāraṃ gacchati, na musā bhaṇati, param pi na tathattāya samādapeti, yaṃ’sa hoti dīgharattaṃ hitāya sukhāyā” ti.

“Evam, bhante.”

“Taṃ kiṃ maññatha, Sāḷhā, ime dhammā kusalā vā akusalā vā” ti?

“Kusalā, bhante.”

“Sāvajjā vā anavajjā vā” ti?

“Anavajjā, bhante.”

“Viññugarahitā vā viññuppasatthā vā” ti?

“Viññuppasatthā, bhante.”

“Samattā samādinnā hitāya sukhāya saṃvattanti, no vā… ?”

“Samattā, bhante, samādinnā hitāya sukhāya saṃvattantī” ti.

“… Yadā tumhe, Sāḷhā, attanā va jāneyyātha ‘ime dhammā kusalā, ime dhammā anavajjā, ime dhammā viññuppasatthā, ime dhammā samattā samādinnā hitāya sukhāya saṃvattantī’ ti, atha tumhe, Sāḷhā, upasampajja vihareyyāthā” ti. \hfill(A 3.7.6)

\section*{Further Reading 2}

“Nāhaṃ, bhikkhave, aññaṃ ekadhammam pi samanupassāmi, yaṃ evaṃ abhāvitaṃ akammaniyaṃ hoti, yathayidaṃ, bhikkhave, cittaṃ. Cittaṃ, bhikkhave, abhāvitaṃ akammaniyaṃ hotī” ti.

“Nāhaṃ, bhikkhave, aññaṃ ekadhammam pi samanupassāmi, yaṃ evaṃ bhāvitaṃ kammaniyaṃ hoti, yathayidaṃ, bhikkhave, cittaṃ. Cittaṃ, bhikkhave, bhāvitaṃ kammaniyaṃ hotī” ti.

“Nāhaṃ, bhikkhave, aññaṃ ekadhammam pi samanupassāmi, yaṃ evaṃ abhāvitaṃ mahato anatthāya saṃvattati, yathayidaṃ, bhikkhave, cittaṃ. Cittaṃ, bhikkhave, abhāvitaṃ mahato anatthāya saṃvattatī” ti.

“Nāhaṃ, bhikkhave, aññaṃ ekadhammam pi samanupassāmi, yaṃ evaṃ bhāvitaṃ mahato atthāya saṃvattati, yathayidaṃ, bhikkhave, cittaṃ. Cittaṃ, bhikkhave, bhāvitaṃ mahato atthāya saṃvattatī” ti.

“Nāhaṃ, bhikkhave, aññaṃ ekadhammam pi samanupassāmi, yaṃ evaṃ abhāvitaṃ apātubhūtaṃ mahato anatthāya saṃvattati, yathayidaṃ, bhikkhave, cittaṃ. Cittaṃ, bhikkhave, abhāvitaṃ apātubhūtaṃ mahato anatthāya saṃvattatī” ti.

“Nāhaṃ, bhikkhave, aññaṃ ekadhammam pi samanupassāmi, yaṃ evaṃ bhāvitaṃ pātubhūtaṃ mahato atthāya saṃvattati, yathayidaṃ, bhikkhave, cittaṃ. Cittaṃ, bhikkhave, bhāvitaṃ pātubhūtaṃ mahato atthāya saṃvattatī” ti.

“Nāhaṃ, bhikkhave, aññaṃ ekadhammam pi samanupassāmi, yaṃ evaṃ abhāvitaṃ abahulīkataṃ mahato anatthāya saṃvattati, yathayidaṃ, bhikkhave, cittaṃ. Cittaṃ, bhikkhave, abhāvitaṃ abahulīkataṃ mahato anatthāya saṃvattatī” ti.

“Nāhaṃ, bhikkhave, aññaṃ ekadhammam pi samanupassāmi, yaṃ evaṃ bhāvitaṃ bahulīkataṃ mahato atthāya saṃvattati, yathayidaṃ, bhikkhave, cittaṃ. Cittaṃ, bhikkhave, bhāvitaṃ bahulīkataṃ mahato atthāya saṃvattatī” ti.

“Nāhaṃ, bhikkhave, aññaṃ ekadhammam pi samanupassāmi, yaṃ evaṃ abhāvitaṃ abahulīkataṃ dukkhādhivāhaṃ hoti, yathayidaṃ, bhikkhave, cittaṃ. Cittaṃ, bhikkhave, abhāvitaṃ abahulīkataṃ dukkhādhivāhaṃ hotī” ti. \hfill(A 3.7.6)

\section*{Further Reading 3}

Idam kho pana bhikkhave, dukkhaṃ ariyasaccaṃ

“Jāti pi dukkhā, jarā pi dukkhā, maraṇam pi dukkhaṃ, appiyehi sampayogo pi dukkho, piyehi vippayogo pi dukkho‚ yaṃ p’icchaṃ na labhati tam pi dukkhaṃ, saṅkhittena pañc’upādānakkhandhā pi dukkhā.” \hfill(D 22 Mahāsatipaṭṭhānasuttaṃ)

\section*{Further Reading 4}

“Bhante Nāgasena, kiṃ lakkhaṇaṃ viññāṇan” ti?

“Vijānanalakkhaṇaṃ, mahārāja, viññāṇan” ti.

“Opammaṃ karohī” ti.

“Yathā, mahārāja, nagaraguttiko majjhe nagare siṅghāṭake nisinno passeyya puratthimadisato purisaṃ āgacchantaṃ, passeyya dakkhiṇadisato purisaṃ āgacchantaṃ, passeyya pacchimadisato purisaṃ āgacchantaṃ, passeyya uttaradisato purisaṃ āgacchantaṃ. Evameva kho, mahārāja, yañca puriso cakkhunā rūpaṃ passati, taṃ viññāṇena vijānāti, yañca sotena saddaṃ suṇāti, taṃ viññāṇena vijānāti, yañca ghānena gandhaṃ ghāyati, taṃ viññāṇena vijānāti, yañca jivhāya rasaṃ sāyati, taṃ viññāṇena vijānāti, yañca kāyena phoṭṭhabbaṃ phusati, taṃ viññāṇena vijānāti, yañca manasā dhammaṃ vijānāti, taṃ viññāṇena vijānāti. ”

“Evaṃ kho, mahārāja, vijānanalakkhaṇaṃ viññāṇan” ti.

“Kallo’si, bhante Nāgasenā” ti. \hfill(Miln III.3.12 Vicāravaggo, Viññāṇalakkhaṇapañho)