\chapter{文學第四}

\subsection*{1}

\textbf{鄭玄在馬融門下,}{\footnotesize \textbf{融自敘}曰融,字季長,右扶風茂陵人,少而好問,學無常師,大將軍鄧騭召為舍人,棄,遊武都,會羌虜起,自關以西道斷,融以謂古人有言「左手據天下之圖,而右手刎其喉,愚夫不為,何則?生貴於天下也,豈以曲俗咫尺為羞,滅無限之身哉」,因往應之,為校書郎,出為南郡太守。}\textbf{三年不得相見,高足弟子傳授而已,嘗算渾天不合,諸弟子莫能解,或言玄能者,融召令算,一轉便決,眾咸駭服,及玄業成辭歸,既而融有「禮樂皆東」之歎,}{\footnotesize \textbf{高士傳}曰玄,字康成,北海高密人,八世祖崇,漢尚書。\textbf{玄別傳}曰玄少好學書數,十三誦五經,好天文、占候、風角、隱術,年十七,見大風起,詣縣曰「某時當有火災」,至時果然,智者異之,年二十一,博極群書,精曆數圖緯之言,兼精算術,遂去吏,師故兗州刺史第五元先,就東郡張恭祖受周禮、禮記、春秋傳,周流博觀,毎經歷山川,及接顏一見,皆終身不忘,扶風馬季長以英儒著名,玄往從之,參考同異,季長后戚,嫚於待士,玄不得見,住左右,自起精廬,既因紹介得通,時涿郡盧子幹為門人冠首,季長又不解剖裂七事,玄思得五,子幹得二,季長謂子幹曰「吾與汝皆弗如也」,季長臨別,執玄手曰「大道東矣,子勉之」,後遇黨錮,隱居著述,凡百餘萬言,大將軍何進辟玄,乃縫掖相見,玄長八尺餘,鬚眉美秀,姿容甚偉,進待以賓禮,授以几杖,玄多所匡正,不用而退,袁紹辟玄,及去,餞之城東,欲玄必醉,會者三百餘人,皆離席奉觴,自旦及莫,度玄飲三百餘桮,而溫克之容,終日無怠,獻帝在許都,徵為大司農,行至元城,卒。}\textbf{恐玄擅名而心忌焉,玄亦疑有追,乃坐橋下,在水上據屐,融果轉式逐之,告左右曰:「玄在土下、水上而據木,此必死矣。」遂罷追,玄竟以得免。}{\footnotesize 馬融海內大儒,被服仁義,鄭玄名列門人,親傳其業,何猜忌而行鴆毒乎?委巷之言,賊夫人之子。}

\subsection*{2}

\textbf{鄭玄欲注春秋傳,尚未成時,行與服子慎遇宿客舍,先未相識,服在外車上與人說己注傳意,}{\footnotesize \textbf{漢南紀}曰服虔,字子慎,河南滎陽人,少行清苦,為諸生,尤明春秋左氏傳,為作訓解,舉孝廉,為尚書郎、九江太守。}\textbf{玄聽之良久,多與己同,玄就車與語曰:「吾久欲注,尚未了,聽君向言,多與吾同,今當盡以所注與君。」遂為服氏注。}

\subsection*{3}

\textbf{鄭玄家奴婢皆讀書,嘗使一婢,不稱旨,將撻之,方自陳說,玄怒,使人曳著泥中,須臾,復有一婢來,問曰:「胡為乎泥中?」}{\footnotesize 衛式微詩也。\textbf{毛公}曰泥中,衛邑名也。}\textbf{答曰:「薄言往愬,逢彼之怒。」}{\footnotesize 衛邶柏舟之詩。}

\subsection*{4}

\textbf{服虔既善春秋,將為注,欲參考同異,聞崔烈集門生講傳,}{\footnotesize \textbf{摯虞文章志}曰烈,字威考,高陽安平人,駰之孫、瑗之兄子也,靈帝時官至司徒、太尉,封陽平亭侯。}\textbf{遂匿姓名,為烈門人賃作食,毎當至講時,輒竊聽戶壁間,既知不能踰己,稍共諸生敘其短長,烈聞,不測何人,然素聞虔名,意疑之,明蚤往,及未寤,便呼:「子慎,子慎!」虔不覺驚應,遂相與友善。}

\subsection*{5}

\textbf{鍾會撰四本論始畢,甚欲使嵇公一見,置懷中,既定,畏其難,懷不敢出,於戶外遙擲,便面急走。}{\footnotesize \textbf{魏志}曰會論才性同異,傳於世,四本者,言才性同、才性異、才性合、才性離也,尚書傅嘏論同,中書令李豐論異,侍郎鍾會論合,屯騎校尉王廣論離。文多不載。}

\subsection*{6}

\textbf{何晏為吏部尚書,有位望,時談客盈坐,}{\footnotesize \textbf{文章敘錄}曰晏能清言,而當時權勢,天下談士,多宗尚之。\textbf{魏氏春秋}曰晏少有異才,善談易老。}\textbf{王弼未弱冠,往見之,晏聞弼名,}{\footnotesize \textbf{弼別傳}曰弼,字輔嗣,山陽高平人,少而察惠,十餘歲便好莊老,通辯能言,為傅嘏所知,吏部尚書何晏甚奇之,題之曰「後生可畏,若斯人者,可與言天人之際矣」,以弼補臺郎,弼事功雅非所長,益不留意,頗以所長笑人,故為時士所嫉,又為人淺而不識物情,初與王黎、荀融善,黎奪其黃門郎,於是恨黎,與融亦不終好,正始中以公事免,其秋遇癘疾亡,時年二十四,弼之卒也,晉景帝嗟歎之累日,曰「天喪予」,其為高識悼惜如此。}\textbf{因條向者勝理語弼曰:「此理僕以為極,可得復難不?」弼便作難,一坐人便以為屈,於是弼自為客主數番,皆一坐所不及。}

\subsection*{7}

\textbf{何平叔注老子,始成,詣王輔嗣,見王注精奇,迺神伏曰:「若斯人,可與論天人之際矣。」因以所注為道德二論。}{\footnotesize \textbf{魏氏春秋}曰弼論道約美不如晏,然自然出拔過之。}

\subsection*{8}

\textbf{王輔嗣弱冠詣裴徽,}{\footnotesize \textbf{永嘉流人名}曰徽,字文季,河東聞喜人,太常潛少弟也,仕至冀州刺史。}\textbf{徽問曰:「夫無者,誠萬物之所資,聖人莫肯致言,而老子申之無已,何邪?」}{\footnotesize \textbf{弼別傳}曰弼父為尚書郎,裴徽為吏部郎,徽見異之,故問。}\textbf{弼曰:「聖人體無,無又不可以訓,故言必及有,老莊未免於有,恆訓其所不足。」}

\subsection*{9}

\textbf{傅嘏善言虛勝,}{\footnotesize \textbf{魏志}曰嘏,字蘭碩,北地泥陽人,傅介子之後也,累遷河南尹、尚書,嘏嘗論才性同異,鍾會集而論之。\textbf{傅子}曰嘏既達治好正,而有清理識要,如論才性,原本精微,鮮能及之,司隸鍾會年甚少,嘏以明知交會。}\textbf{荀粲談尚玄遠,}{\footnotesize \textbf{粲別傳}曰粲,字奉倩,潁川潁陰人,太尉彧少子也,粲諸兄儒術論議各知名,粲能言玄遠,常以子貢稱「夫子之言性與天道,不可得而聞也」,然則六籍雖存,固聖人之糠粃,能言者不能屈。}\textbf{毎至共語,有爭而不相喻,裴冀州釋二家之義,通彼我之懷,常使兩情皆得,彼此俱暢。}{\footnotesize \textbf{粲別傳}曰粲太和初到京邑,與傅嘏談,嘏善名理,而粲尚玄遠,宗致雖同,倉卒時或格而不相得意,裴徽通彼我之懷,為二家釋,頃之,粲與嘏善。\textbf{管輅傳}曰裴使君有高才逸度,善言玄妙也。}

\subsection*{10}

\textbf{何晏注老子未畢,見王弼自說注老子旨,何意多所短,不復得作聲,但應諾諾,遂不復注,因作道德論。}{\footnotesize \textbf{文章敘錄}曰自儒者論以老子非聖人,絕禮棄學,晏說與聖人同,著論行於世也。}

\subsection*{11}

\textbf{中朝時,有懷道之流,有詣王夷甫咨疑者,值王昨已語多,小極,不復相酬答,乃謂客曰:「身今少惡,裴逸民亦近在此,君可往問。」}{\footnotesize \textbf{晉諸公贊}曰裴頠談理,與王夷甫不相推下。}

\subsection*{12}

\textbf{裴成公作崇有論,時人攻難之,莫能折,唯王夷甫來,如小屈,時人即以王理難裴,理還復申。}{\footnotesize \textbf{晉諸公贊}曰自魏太常夏侯玄、步兵校尉阮籍等,皆著道德論,于時侍中樂廣、吏部郎劉漢亦體道而言約,尚書令王夷甫講理而才虛,散騎常侍戴奧以學道為業,後進庾敳之徒皆希慕簡曠,頠疾世俗尚虛無之理,故著崇有二論以折之,才博喻廣,學者不能究,後樂廣與頠清閒欲說理,而頠辭喻豐博,廣自以體虛無,笑而不復言。\textbf{惠帝起居注}曰頠著二論以規虛誕之弊,文詞精富,為世名論。}

\subsection*{13}

\textbf{諸葛厷年少不肯學問,始與王夷甫談,便已超詣,王歎曰:「卿天才卓出,若復小加研尋,一無所愧。」厷後看莊老,更與王語,便足相抗衡。}{\footnotesize \textbf{王隱晉書}曰厷,字茂遠,琅邪人,魏雍州刺史緒之子,有逸才,仕至司空主簿。}

\subsection*{14}

\textbf{衛玠總角時問樂令夢,樂云是想,衛曰:「形神所不接而夢,豈是想邪?」樂云:「因也,未嘗夢乘車入鼠穴,擣䪢噉鐵杵,皆無想無因故也。」}{\footnotesize \textbf{周禮}有六夢,一曰正夢,謂無所感動,平安而夢也,二曰噩夢,謂驚愕而夢也,三曰思夢,謂覺時所思念也,四曰寤夢,謂覺時道之而夢也,五曰喜夢,謂喜說而夢也,六曰懼夢,謂恐懼而夢也。\textbf{按}樂所言想者,蓋思夢也,因者,蓋正夢也。}\textbf{衛思因,經日不得,遂成病,樂聞,故命駕為剖析之,衛即小差,樂歎曰:「此兒胸中當必無膏肓之疾。」}{\footnotesize \textbf{春秋傳}曰晉景公有疾,求醫於秦,秦伯使醫緩為之,未至,公夢疾為二豎子,曰「彼,良醫也,懼傷我焉」,其一曰「居肓之上、膏之下,若我何」,醫至,曰「疾不可為也,在肓之上、膏之下,攻之不可達,刺之不可及,藥不至焉」,公曰「良醫也」。\textbf{注}肓,鬲也,心下為膏。}

\subsection*{15}

\textbf{庾子嵩讀莊子,開卷一尺許便放去,曰:「了不異人意。」}{\footnotesize \textbf{晉陽秋}曰庾敳,字子嵩,潁川人,侍中峻第三子,恢廓有度量,自謂是老莊之徒,曰「昔未讀此書,意嘗謂至理如此,今見之,正與人意暗同」,仕至豫州長史。}

\subsection*{16}

\textbf{客問樂令「旨不至」者,樂亦不復剖析文句,直以麈尾柄确几曰:「至不?」客曰:「至。」樂因又舉麈尾曰:「若至者,那得去?」}{\footnotesize 夫藏舟潛往,交臂恆謝,一息不留,忽焉生滅,故飛鳥之影,莫見其移,馳車之輪,曾不掩地,是以去不去矣,庸有至乎?至不至矣,庸有去乎?然則前至不異後至,至名所以生,前去不異後去,去名所以立,今天下無去矣,而去者非假哉?既為假矣,而至者豈實哉?}\textbf{於是客乃悟服。樂辭約而旨達,皆此類。}

\subsection*{17}

\textbf{初,注莊子者數十家,莫能究其旨要,向秀於舊注外為解義,妙析奇致,大暢玄風,}{\footnotesize \textbf{秀別傳}曰秀與嵇康、呂安為友,趣舍不同,嵇康傲世不羈,安放逸邁俗,而秀雅好讀書,二子頗以此嗤之,後秀將注莊子,先以告康、安,康、安咸曰「書詎復須注,徒棄人作樂事耳」,及成,以示二子,康曰「爾故復勝不」,安乃驚曰「莊周不死矣」,後注周易,大義可觀,而與漢世諸儒互有彼此,未若隱莊之絕倫也。\textbf{秀本傳}或言,秀遊託數賢,蕭屑卒歲,都無注述,惟好莊子,聊應崔譔所注,以備遺忘云。\textbf{竹林七賢論}云秀為此義,讀之者無不超然,若已出塵埃而窺絕冥,始了視聽之表,有神德玄哲,能遺天下、外萬物,雖復使動競之人顧觀所徇,皆悵然自有振拔之情矣。}\textbf{惟秋水、至樂二篇未竟而秀卒,秀子幼,義遂零落,然猶有別本。郭象者,為人薄行,有儁才,}{\footnotesize \textbf{文士傳}曰象,字子玄,河南人,少有才理,慕道好學,託志老莊,時人咸以為王弼之亞,辟司空掾、太學博士。}\textbf{見秀義不傳於世,遂竊以為己注,乃自注秋水、至樂二篇,又易馬蹄一篇,其餘眾篇,或定點文句而已。}{\footnotesize \textbf{文士傳}曰象作莊子注,最有清辭遒旨。}\textbf{後秀義別本出,故今有向、郭二莊,其義一也。}

\subsection*{18}

\textbf{阮宣子有令聞,太尉王夷甫見而問曰:「老莊與聖教同異?」對曰:「將無同。」太尉善其言,辟之為掾,世謂三語掾。衛玠嘲之曰:「一言可辟,何假於三?」宣子曰:「苟是天下人望,亦可無言而辟,復何假一?」遂相與為友。}{\footnotesize \textbf{名士傳}曰阮脩,字宣子,陳留尉氏人,好老易,能言理,不喜見俗人,時誤相逢,即舍去,傲然無營,家無儋石之儲,晏如也,琅邪王處仲為鴻臚卿,謂曰「鴻臚丞差有祿,卿常無食,能作不」,脩曰「為復可耳」,遂為鴻臚丞、太子洗馬。}

\subsection*{19}

\textbf{裴散騎娶王太尉女,婚後三日,諸壻大會,}{\footnotesize \textbf{晉諸公贊}曰裴遐,字叔道,河東人,父綽,長水校尉,遐少有理稱,辟司空掾、散騎郎。\textbf{永嘉流人名}衍,字夷甫,第四女適遐也。}\textbf{當時名士,王、裴子弟悉集,郭子玄在坐,挑與裴談,子玄才甚豐瞻,始數交未快,郭陳張甚盛,裴徐理前語,理致甚微,四坐咨嗟稱快,}{\footnotesize \textbf{鄧粲晉紀}曰遐以辯論為業,善敘名理,辭氣清暢,泠然若琴瑟,聞其言者,知與不知,無不歎服。}\textbf{王亦以為奇,謂諸人曰:「君輩勿為爾,將受困寡人女壻。」}

\subsection*{20}

\textbf{衛玠始度江,見王大將軍,}{\footnotesize \textbf{敦別傳}曰敦,字處仲,琅邪臨沂人,少有名理,累遷青州刺史,避地江左,歷侍中、丞相、大將軍、揚州牧,以罪伏誅。}\textbf{因夜坐,大將軍命謝幼輿,}{\footnotesize \textbf{晉陽秋}曰謝鯤,字幼輿,陳郡人,父衡,晉碩儒,鯤性通簡,好老易,善音樂,以琴書為業,避亂江東,為豫章太守,王敦引為長史。\textbf{鯤別傳}曰鯤四十三卒,贈太常。}\textbf{玠見謝,甚說之,都不復顧王,遂達旦微言,王永夕不得豫,玠體素羸,恆為母所禁,爾夕忽極,於此病篤,遂不起。}{\footnotesize \textbf{玠別傳}曰玠少有名理,善易老,自抱羸疾,初不於外擅相酬對,時友歎曰「衛君不言,言必入冥」,武昌見大將軍王敦,敦與談論,咨嗟不能自已。}

\subsection*{21}

\textbf{舊云王丞相過江左,止道聲無哀樂、}{\footnotesize \textbf{嵇康聲無哀樂論}略曰夫殊方異俗,歌笑不同,使錯而用之,或聞哭而懽,或聽歌而戚,然哀樂之情均也,今用均同之情,發萬殊之聲,斯非音聲之無常乎。}\textbf{養生、}{\footnotesize \textbf{嵇叔夜養生論}曰夫蝨著頭而黑,麝食柏而香,頸處險而癭,齒居晉而黃,豈唯蒸之使重無使輊、芬之使香無使延哉?誠能蒸以靈芝,潤以醴泉,無為自得,體妙心玄,庶與羡門比壽,王喬爭年,何為不可養生哉。}\textbf{言盡意,}{\footnotesize \textbf{歐陽堅石言盡意論}略曰夫理得於心,非言不暢,物定於彼,非名不辨,名逐物而遷,言因理而變,不得相與為二矣,苟無其二,言無不盡矣。}\textbf{三理而已,然宛轉關生,無所不入。}

\subsection*{22}

\textbf{殷中軍為庾公長史,}{\footnotesize \textbf{按}庾亮僚屬名及中興書,浩為亮司馬,非為長史也。}\textbf{下都,王丞相為之集,桓公、王長史、王藍田、}{\footnotesize \textbf{王述別傳}曰述,字懷祖,太原晉陽人,祖湛、父承並有高名,述蚤孤,事親孝謹,簞瓢陋巷,宴安永日,由是為有識所知,襲爵藍田侯。}\textbf{謝鎮西並在,丞相自起解帳帶麈尾,語殷曰:「身今日當與君共談析理。」既共清言,遂達三更,丞相與殷共相往反,其餘諸賢略無所關,既彼我相盡,丞相乃歎曰:「向來語,乃竟未知理源所歸,至於辭喻不相負,正始之音,正當爾耳。」明旦,桓宣武語人曰:「昨夜聽殷、王清言甚佳,仁祖亦不寂寞,我亦時復造心,顧看兩王掾,}{\footnotesize 王濛、王述並為王導所辟。}\textbf{輒翣如生母狗馨。」}

\subsection*{23}

\textbf{殷中軍見佛經云:「理亦應阿堵上。」}{\footnotesize 佛經之行中國尚矣,莫詳其始。\textbf{牟子}曰漢明帝夜夢神人,身有日光,明日,博問群臣,通人傅毅對曰「臣聞天竺有道者號曰佛,輕舉能飛,身有日光,殆將其神也」,於是遣羽林將軍秦景、博士弟子王遵等十二人之大月氏國,寫取佛經四十二部,在蘭臺石室。\textbf{劉子政列仙傳}曰歷觀百家之中,以相檢驗,得仙者百四十六人,其七十四人已在佛經,故撰得七十,可以多聞博識者遐觀焉。如此,即漢成、哀之間已有經矣,與牟子傳記便為不同。\textbf{魏略西戎傳}曰天竺城中有臨兒國,浮屠經云,其國王生浮圖,浮圖者,太子也,父曰屑頭邪,母曰莫邪,浮屠者,身服色黃,髮如青絲,爪如銅,其母夢白象而孕,及生,從右脅出,而有髻,墜地能行七步。天竺又有神人曰沙津,昔漢哀帝元壽元年,博士弟子景慮受大月氏王使伊存口傳浮屠經,曰復豆者,其人也。\textbf{漢武故事}曰昆邪王殺休屠王,以其眾來降,得其金人之神,置之甘泉宮,金人皆長丈餘,其祭不用牛羊,惟燒香禮拜,上使依其國俗祀之。此神全類於佛,豈當漢武之時,其經未行於中土,而但神明事之邪?故驗劉向、魚豢之說,佛至自哀、成之世明矣,然則牟傳所言四十二者,其文今存非妄,蓋明帝遣使廣求異聞,非是時無經也。}

\subsection*{24}

\textbf{謝安年少時,請阮光祿道白馬論,}{\footnotesize \textbf{孔叢子}曰趙人公孫龍云「白馬非馬,馬者所以命形,白者所以命色,夫命色者非命形,故曰白馬非馬也」。}\textbf{為論以示謝,于時謝不即解阮語,重相咨盡,阮乃歎曰:「非但能言人不可得,正索解人亦不可得。」}{\footnotesize \textbf{中興書}曰裕甚精論難。}

\subsection*{25}

\textbf{褚季野語孫安國}{\footnotesize 褚裒、孫盛並已見。}\textbf{云:「北人學問,淵綜廣博。」孫答曰:「南人學問,清通簡要。」支道林聞之曰:「聖賢固所忘言,自中人以還,北人看書,如顯處視月,南人學問,如牖中窺日。」}{\footnotesize 支所言,但譬成孫、褚之理也,然則學廣則難周,難周則識闇,故如顯處視月,學寡則易覈,易覈則智明,故如牖中窺日也。}

\subsection*{26}

\textbf{劉真長與殷淵源談,劉理如小屈,殷曰:「惡卿不欲作將善雲梯仰攻。」}{\footnotesize \textbf{墨子}曰公輸般為高雲梯,欲以攻宋,墨子聞之,自魯往,裂裳裹足,日夜不休,十日十夜而至於郢,見楚王曰「聞大王將攻宋,有之乎」,王曰「然」,墨子曰「請令公輸般設攻宋之具,臣請試守之」,於是公輸般設攻宋之計,墨子縈帶守之,輸九攻之,而墨子九卻之,不能入,遂輟兵。}

\subsection*{27}

\textbf{殷中軍云:「康伯未得我牙後慧。」}{\footnotesize \textbf{浩別傳}曰浩善老易,能清言,康伯,浩甥也,甚愛之。}

\subsection*{28}

\textbf{謝鎮西少時,聞殷浩能清言,故往造之,殷未過有所通,為謝標榜諸義,作數百語,既有佳致,兼辭條豐蔚,甚足以動心駭聽,謝注神傾意,不覺流汗交面,殷徐語左右:「取手巾與謝郎拭面。」}{\footnotesize \textbf{按}殷浩大謝尚三歲,便是時流,或當貴其勝致,故為之揮汗。}

\subsection*{29}

\textbf{宣武集諸名勝講易,}{\footnotesize \textbf{易乾鑿度}曰孔子曰「易者,易也、變易也、不易也,三成德,為道包籥者,易也,其德也,光明四通,日月星辰布,八卦序、四時和也,變也者,天地不變,不能成朝,夫婦不變,不能成家,不易者,其位也,天在上、地在下,君南面、臣北面,父坐、子伏,此其不易也,故易者,天地人道也」。\textbf{鄭玄序易}曰易之為名也,一言而函三義,簡易一也,變易二也,不易三也,繫辭曰「乾坤,易之蘊也,易之門戶也」,又曰「乾確然示人易矣,坤隤然示人簡矣,易則易知,簡則易從」,此言其簡易法則也,又曰「其為道也屢遷,變動不居,周流六虛,上下無常,剛柔相易,不可以為典要,惟變所適」,此則言其從時出入移動也,又曰「天尊地卑,乾坤定矣,卑高以陳,貴賤位矣,動靜有常,剛柔斷矣」,此則言其張設布列不易也,據此三義而說易之道,廣矣大矣。}\textbf{日說一卦,簡文欲聽,聞此便還,曰:「義自當有難易,其以一卦為限邪?」}

\subsection*{30}

\textbf{有北來道人好才理,與林公相遇於瓦官寺,講小品,于時竺法深、孫興公悉共聽,此道人語,屢設疑難,林公辯答清析,辭氣俱爽,此道人毎輒摧屈,孫問深公:「上人當是逆風家,向來何以都不言?」}{\footnotesize \textbf{庾法暢人物論}曰法深學義淵博,名聲蚤著,弘道法師也。}\textbf{深公笑而不答,林公曰:「白旃檀非不馥,焉能逆風?」}{\footnotesize \textbf{成實論}曰波利質多天樹,其香則逆風而聞。}\textbf{深公得此義,夷然不屑。}

\subsection*{31}

\textbf{孫安國往殷中軍許共論,往反精苦,客主無間,左右進食,冷而復煗者數四,彼我奮擲麈尾,悉脫落,滿餐飯中,賓主遂至莫忘食,殷乃語孫曰:「卿莫作強口馬,我當穿卿鼻。」孫曰:「卿不見決鼻牛,人當穿卿頰。」}{\footnotesize \textbf{續晉陽秋}曰孫盛善理義,時中軍將軍殷浩擅名一時,能與劇談相抗者,惟盛而已。}

\subsection*{32}

\textbf{莊子逍遙篇,舊是難處,諸名賢所可鑽味,而不能拔理於郭、向之外,支道林在白馬寺中,將馮太常共語,}{\footnotesize \textbf{馮氏譜}曰馮懷,字祖思,長樂人,歷太常、護軍將軍。}\textbf{因及逍遙,支卓然標新理於二家之表,立異義於眾賢之外,皆是諸名賢尋味之所不得,後遂用支理。}{\footnotesize \textbf{向子期、郭子玄逍遙義}曰夫大鵬之上九萬,尺鷃之起榆枋,小大雖差,各任其性,苟當其分,逍遙一也,然物之芸芸,同資有待,得其所待,然後逍遙耳,惟聖人與物冥而循大變,為能無待而常通,豈獨自通而已,又從有待者不失其所待,不失,則同於大通矣。\textbf{支氏逍遙論}曰夫逍遙者,明至人之心也,莊生建言人道,而寄指鵬鷃,鵬以營生之路曠,故失適於體外,鷃以在近而笑遠,有矜伐於心內,至人乘天正而高興,遊無窮於放浪,物物而不物於物,則遙然不我得,玄感不為,不疾而速,則逍然靡不適,此所以為逍遙也,若夫有欲當其所足,足於所足,快然有似天真,猶飢者一飽,渴者一盈,豈忘烝嘗於糗糧,絕觴爵於醪醴哉?苟非至足,豈所以逍遙乎?此向、郭之注所未盡。}

\subsection*{33}

\textbf{殷中軍}{\footnotesize 浩也。}\textbf{嘗至劉尹所清言,良久,殷理小屈,遊辭不已,劉亦不復答,殷去後,乃云:「田舍兒,強學人作爾馨語。」}{\footnotesize 劉惔已見。}

\subsection*{34}

\textbf{殷中軍雖思慮通長,然於才性偏精,忽言及四本,便苦湯池鐵城,無可攻之勢。}{\footnotesize \textbf{神農書}曰夫有石城七仞,湯池百步,帶甲百萬而無粟者,不能自固也。}

\subsection*{35}

\textbf{支道林造即色論,}{\footnotesize \textbf{支道林集妙觀章}云夫色之性也,不自有色,色不自有,雖色而空,故曰色即為空、色復異空。}\textbf{論成,示王中郎,}{\footnotesize 王坦之已見。}\textbf{中郎都無言,支曰:「默而識之乎?」}{\footnotesize \textbf{論語}曰默而識之,誨人不倦,何有於我哉?}\textbf{王曰:「既無文殊,誰能見賞?」}{\footnotesize \textbf{維摩詰經}曰文殊師利問維摩詰云「何者是菩薩入不二法門」,時維摩詰默然無言,文殊師利歎曰「是真入不二法門也」。}

\subsection*{36}

\textbf{王逸少作會稽,初至,支道林在焉,孫興公謂王曰:「支道林拔新領異,胸懷所及乃自佳,卿欲見不?」王本自有一往雋氣,殊自輕之,後孫與支共載往王許,王都領域,不與交言,須臾支退,後正值王當行,車已在門,支語王曰:「君未可去,貧道與君小語。」因論莊子逍遙遊,支作數千言,才藻新奇,花爛映發,王遂披襟解帶,留連不能已。}{\footnotesize \textbf{支法師傳}曰法師研十地,則知頓悟於七住,尋莊周,則辯聖人之逍遙,當時名勝咸味其音旨。\textbf{道賢論}以七沙門比竹林七賢,遁比向秀,雅尚莊老,二子異時,風尚玄同也。}

\subsection*{37}

\textbf{三乘佛家滯義,支道林分判,使三乘炳然,諸人在下坐聽,皆云可通,支下坐,自共說,正當得兩,入三便亂,今義弟子雖傳,猶不盡得。}{\footnotesize \textbf{法華經}曰三乘者,一曰聲聞乘,二曰緣覺乘,三曰菩薩乘,聲聞者,悟四諦而得道也,緣覺者,悟因緣而得道也,菩薩者,行六度而得道也,然則羅漢得道,全由佛教,故以聲聞為名也,辟支佛得道,或聞因緣而解,或聽環珮而得悟,神能獨達,故以緣覺為名也,菩薩者,大道之人也,方便則止行六度,真教則通修萬善,功不為己,志存廣濟,故以大道為名也。}

\subsection*{38}

\textbf{許掾}{\footnotesize 詢也。}\textbf{年少時,人以比王苟子,}{\footnotesize 苟子,王脩小字也。\textbf{文字志}曰脩,字敬仁,太原晉陽人,父濛,司徒左長史,脩明秀有美稱,善隸行書,號曰流奕清舉,起家著作佐郎、琅邪王文學,轉中軍司馬,未拜而卒,時年二十四,昔王弼之沒,與脩同年,故脩弟熙乃歎曰「無愧於古人,而年與之齊也」。}\textbf{許大不平,時諸人士及林法師並在會稽西寺講,王亦在焉,許意甚忿,便往西寺與王論理,共決優劣,苦相折挫,王遂大屈,許復執王理,王執許理,更相覆疏,王復屈,許謂支法師曰:「弟子向語何似?」支從容曰:「君語佳則佳矣,何至相苦邪?豈是求理中之談哉?」}

\subsection*{39}

\textbf{林道人詣謝公,東陽時始總角,新病起,體未堪勞,與林公講論,遂至相苦,}{\footnotesize 東陽,謝朗也,已見。\textbf{中興書}曰朗博涉有逸才,善言玄理。}\textbf{母王夫人在壁後聽之,再遣信令還,而太傅留之,王夫人因自出云:「新婦少遭家難,一生所寄,惟在此兒。」因流涕抱兒以歸,謝公語同坐曰:「家嫂辭情忼慨,致可傳述,恨不使朝士見。」}{\footnotesize \textbf{謝氏譜}曰朗父據,取太康王韜女,名綏。}

\subsection*{40}

\textbf{支道林、許掾諸人共在會稽王齋頭,}{\footnotesize 簡文。}\textbf{支為法師,許為都講,}{\footnotesize \textbf{高逸沙門傳}曰道林時講維摩詰經。}\textbf{支通一義,四坐莫不厭心,許送一難,眾人莫不抃舞,但共嗟詠二家之美,不辯其理之所在。}

\subsection*{41}

\textbf{謝車騎在安西艱中,}{\footnotesize 安西,謝奕,已見。}\textbf{林道人往就語,將夕乃退,有人道上見者,問云:「公何處來?」答云:「今日與謝孝劇談一出來。」}{\footnotesize \textbf{玄別傳}曰玄能清言,善名理。}

\subsection*{42}

\textbf{支道林初從東出,住東安寺中,}{\footnotesize \textbf{高逸沙門傳}曰遁居會稽,晉哀帝欽其風味,遣中使至東迎之,遁遂辭丘壑,高步天邑。}\textbf{王長史宿構精理,並撰其才藻,往與支語,不大當對,王敘致作數百語,自謂是名理奇藻,支徐徐謂曰:「身與君別多年,君義言了不長進。」王大慚而退。}

\subsection*{43}

\textbf{殷中軍讀小品,}{\footnotesize 釋氏辨空經有詳者焉、有略者焉,詳者為大品,略者為小品。}\textbf{下二百籤,皆是精微,世之幽滯,嘗欲與支道林辯之,竟不得,今小品猶存。}{\footnotesize \textbf{高逸沙門傳}曰殷浩能言名理,自以有所不達,欲訪之於遁,遂邂逅不遇,深以為恨,其為名識賞重,如此之至焉。\textbf{語林}曰浩於佛經有所不了,故遣人迎林公,林乃虛懷欲往,王右軍駐之曰「淵源思致淵富,既未易為敵,且己所不解,上人未必能通,縱復服從,亦名不益高,若佻脫不合,便喪十年所保,可不須往」,林公亦以為然,遂止。}

\subsection*{44}

\textbf{佛經以為袪練神明,則聖人可致,}{\footnotesize \textbf{釋氏經}曰一切眾生皆有佛性,但能修智慧、斷煩惱,萬行具足,便成佛也。}\textbf{簡文云:「不知便可登峰造極不?然陶練之功,尚不可誣。」}

\subsection*{45}

\textbf{于法開始與支公爭名,後精漸歸支,意甚不忿,遂遁跡剡下,遣弟子出都,語使過會稽,于時支公正講小品,開戒弟子「道林講,比汝至,當在某品中」,因示語攻難數十番,云「舊此中不可復通」,弟子如言詣支公,正值講,因謹述開意,往反多時,林公遂屈,厲聲曰:「君何足復受人寄載來?」}{\footnotesize \textbf{名德沙門題目}曰于法開才辨從橫,以數術弘教。\textbf{高逸沙門傳}曰法開初以義學著名,後與支遁有競,故遁居剡縣,更學醫術。}

\subsection*{46}

\textbf{殷中軍問:「自然無心於禀受,何以正善人少、惡人多?」諸人莫有言者,劉尹答曰:「譬如寫水著地,正自縱橫流漫,略無正方圓者。」一時絕歎,以為名通。}{\footnotesize \textbf{莊子}曰天籟者,吹萬不同,而使其自己也。\textbf{郭子玄}注曰無既無矣,則不能生有,有之未生,又不能為生,然則生生者誰哉,塊然而自生耳,非我生也,我不生物,物不生我,則自己而然,謂之天然,天然非為也,故以天言之,所以明其自然故也。}

\subsection*{47}

\textbf{康僧淵初過江,未有知者,恆周旋市肆,乞索以自營,忽往殷淵源許,值盛有賓客,殷使坐,粗與寒溫,遂及義理,語言辭旨,曾無愧色,領略粗舉,一往參詣,由是知之。}{\footnotesize 僧淵氏族所出未詳,疑是胡人,尚書令沈約撰晉書,亦稱其有義學。}

\subsection*{48}

\textbf{殷、謝諸人共集,}{\footnotesize 殷浩、謝安。}\textbf{謝因問殷:「眼往屬萬形,萬形來入眼不?」}{\footnotesize \textbf{成實論}曰眼識不待到而知,虛塵假空與明,故得見色,若眼到色到,色間則無空明,如眼觸目,則不能見彼,當知眼識不到而知。依如此說,則眼不往,形不入,遙屬而見也。謝有問而殷無答,疑闕文。}

\subsection*{49}

\textbf{人有問殷中軍:「何以將得位而夢棺器,將得財而夢矢穢?」殷曰:「官本是臭腐,所以將得而夢棺屍,財本是糞土,所以將得而夢穢汙。」時人以為名通。}

\subsection*{50}

\textbf{殷中軍被廢東陽,}{\footnotesize 浩黜廢事別見。}\textbf{始看佛經,初視維摩詰,}{\footnotesize \textbf{僧肇注維摩經}曰維摩詰者,秦言淨名,蓋法身之大士,見居此土,以弘道也。}\textbf{疑般若波羅密太多,後見小品,恨此語少。}{\footnotesize 波羅密,此言到彼岸也。\textbf{經}云到者有六焉,一曰檀,檀者,施也,二曰毗黎,毗黎者,持戒也,三曰羼提,羼提者,忍辱也,四曰尸羅,尸羅者,精進也,五曰禪,禪者,定也,六曰般若,般若者,智慧也,然則五者為舟,般若為導,導則俱絕有相之流,升無相之彼岸也,故曰波羅密也。淵源未暢其致,少而疑其多,已而究其宗,多而患其少也。}

\subsection*{51}

\textbf{支道林、殷淵源俱在相王許,}{\footnotesize 簡文。}\textbf{相王謂二人:「可試一交言,而才性殆是淵源崤函之固,}{\footnotesize 崤,謂二陵之地,函,函谷關也,並秦之險塞,王者之居。\textbf{左思魏都賦}曰崤函帝王之宅。}\textbf{君其慎焉。」支初作,改轍遠之,數四交,不覺入其玄中,相王撫肩笑曰:「此自是其勝場,安可爭鋒。」}

\subsection*{52}

\textbf{謝公因子弟集聚,問毛詩何句最佳,遏稱曰:}{\footnotesize 謝玄小字,已見。}\textbf{「昔我往矣,楊柳依依,今我來思,雨雪霏霏。」公曰:「訏謨定命,遠猷辰告。」}{\footnotesize 大雅詩也。\textbf{毛萇}注曰訏,大也,謨,謀也,辰,時也。\textbf{鄭玄}注曰猷,圖也,大謀定命,謂正月始和,布政於邦國都鄙。}\textbf{謂此句偏有雅人深致。}

\subsection*{53}

\textbf{張憑舉孝廉出都,負其才氣,謂必參時彥,欲詣劉尹,鄉里及同舉者共笑之,張遂詣劉,劉洗濯料事,處之下坐,唯通寒暑,神意不接,張欲自發無端,頃之,長史諸賢來清言,客主有不通處,張乃遙於末坐判之,言約旨遠,足暢彼我之懷,一坐皆驚,真長延之上坐,清言彌日,因留宿至曉,張退,劉曰:「卿且去,正當取卿共詣撫軍。」張還船,同侶問何處宿,張笑而不答,須臾,真長遣傳教覓張孝廉船,同侶惋愕,即同載詣撫軍,至門,劉前進謂撫軍曰:「下官今日為公得一太常博士妙選。」既前,撫軍與之話言,咨嗟稱善曰:「張憑勃窣為理窟。」即用為太常博士。}{\footnotesize \textbf{宋明帝文章志}曰憑,字長宗,吳郡人,有意氣,為鄉閭所稱,學尚所得,敏而有文,太守以才選舉孝廉,試策高第,為惔所舉,補太常博士,累遷吏部郎、御史中丞。}

\subsection*{54}

\textbf{汰法師云:「六通、三明同歸,正異名耳。」}{\footnotesize \textbf{安法師傳}曰竺法汰者,體器弘簡,道情冥到,法師友而善焉。一說法汰即安公弟子也。\textbf{經}云六通者,三乘之功德也,一曰天眼通,見遠方之色,二曰天耳通,聞障外之聲,三曰身通,飛行隱顯,四曰它心通,水鏡萬慮,五曰宿命通,神知已往,六曰漏盡通,慧解累世,三明者,解脫在心,朗照三世者也。然則天眼、天耳、身通、它心、漏盡此五者,皆見在心之明也,宿命則過去心之明也,因天眼發未來之智,則未來心之明也,同歸異名,義在斯矣。}

\subsection*{55}

\textbf{支道林、許、謝盛德,共集王家,}{\footnotesize 許詢、謝安、王濛。}\textbf{謝顧謂諸人:「今日可謂彥會,時既不可留,此集固亦難常,當共言詠,以寫其懷。」許便問主人有莊子不,正得漁父一篇,}{\footnotesize \textbf{莊子}曰孔子遊乎緇帷之林,休坐乎杏壇之上,孔子弦歌鼓琴,奏曲未半,有漁者下船而來,鬚眉交白,被髮揄袂,行原以上,距陸而止,左手據膝,右手持頤以聽,曲終而招子貢、子路語曰「彼何為者也」,曰「孔氏」,曰「孔氏何治」,子貢曰「服忠信,行仁義,飾禮樂,選人倫,孔氏之所治也」,曰「有土之君歟」,曰「非也」,漁父曰「仁則仁矣,恐不免其身」,孔子聞而求問之,遂言八疵四病以誡孔子。}\textbf{謝看題,便各使四坐通,支道林先通,作七百許語,敘致精麗,才藻奇拔,眾咸稱善,於是四坐各言懷畢,謝問曰:「卿等盡不?」皆曰:「今日之言,少不自竭。」謝後粗難,因自敘其意,作萬餘語,才峰秀逸,}{\footnotesize \textbf{文字志}曰安神情秀悟,善談玄速。}\textbf{既自難干,加意氣擬託,蕭然自得,四坐莫不厭心,支謂謝曰:「君一往奔詣,故復自佳耳。」}

\subsection*{56}

\textbf{殷中軍、孫安國、王、謝能言諸賢,悉在會稽王許,殷與孫共論易象,妙於見形,}{\footnotesize 其\textbf{論}略曰聖人知觀器不足以達變,故表圓應於蓍龜,圓應不可為典要,故寄妙迹於六爻,六爻周流,唯化所適,故雖一畫而吉凶並彰,微一則失之矣,擬器託象而慶咎交著,繫器則失之矣,故設八卦者,蓋緣化之影迹也,天下者,寄見之一形也,圓影備未備之象,一形兼未形之形,故盡二儀之道,不與乾坤齊妙,風雨之變,不與巽坎同體矣。}\textbf{孫語道合,意氣干雲,一坐咸不安孫理,而辭不能屈,會稽王慨然歎曰:「使真長來,故應有以制彼。」既迎真長,孫意己不如,真長既至,先令孫自敘本理,孫粗說己語,亦覺殊不及向,劉便作二百許語,辭難簡切,孫理遂屈,一坐同時拊掌而笑,稱美良久。}

\subsection*{57}

\textbf{僧意在瓦官寺中,}{\footnotesize 未詳僧意氏族所出。}\textbf{王苟子來,}{\footnotesize 苟子,王脩小字。}\textbf{與共語,便使其唱理,意謂王曰:「聖人有情不?」王曰:「無。」重問曰:「聖人如柱邪?」王曰:「如籌算,雖無情,運之者有情。」僧意云:「誰運聖人邪?」苟子不得答而去。}{\footnotesize 諸本無僧意最後一句,意疑其闕,廣校眾本皆然,惟一書有之,故取以成其義。然王脩善言理,如此論,特不近人情,猶疑斯文為謬也。}

\subsection*{58}

\textbf{司馬太傅問謝車騎:「惠子其書五車,何以無一言入玄?」謝曰:「故當是其妙處不傳。」}{\footnotesize \textbf{莊子}曰惠施多方,其書五車,其道舛駮,其言不中,謂卵有毛、雞三足、馬有卵、犬可為羊、火不熱、目不見、龜長於蛇、丁子有尾、白狗黑、連環可解,能勝人之口,不能服人之心,蓋辯者之囿也。}

\subsection*{59}

\textbf{殷中軍被廢,徙東陽,大讀佛經,皆精解,惟至事數處不解,}{\footnotesize 事數,謂若五陰、十二入、四諦、十二因緣、五根、五力、七覺之屬。}\textbf{遇見一道人,問所籤,便釋然。}

\subsection*{60}

\textbf{殷仲堪精覈玄論,人謂莫不研究,殷乃歎曰:「使我解四本,談不翅爾。」}{\footnotesize \textbf{周祗隆安記}曰仲堪好學而有理思也。}

\subsection*{61}

\textbf{殷荊州曾問遠公:}{\footnotesize \textbf{張野遠法師銘}曰沙門釋惠遠,鴈門樓煩人,本姓賈氏,世為冠族,年十二,隨舅令狐氏遊學許洛,年二十一,欲南渡,就范宣子學,道阻不通,遇釋道安以為師,抽簪落髮,研求法藏,釋曇翼毎資以燈燭之費,識鑒淹遠,高悟冥賾,安常歎曰「道流東國,其在遠乎」,襄陽既沒,振錫南遊,結宇靈嶽,自年六十不復出山,名被流沙,彼國僧眾皆稱漢地有大乘沙門,毎至然香禮拜,輒東向致敬,年八十三而終。}\textbf{「易以何為體?」答曰:「易以感為體。」殷曰:「銅山西崩,靈鐘東應,便是易耶?」}{\footnotesize \textbf{東方朔傳}曰孝武皇帝時,未央宮前殿鐘無故自鳴,三日三夜不止,詔問太史待詔王朔,朔言恐有兵氣,更問東方朔,朔曰「臣聞銅者山之子,山者銅之母,以陰陽氣類言之,子母相感,山恐有崩弛者,故鐘先鳴,易曰『鳴鶴在陰,其子和之』,精之至也,其應在後五日內」,居三日,南郡太守上書言山崩,延袤二十餘里。\textbf{樊英別傳}曰漢順帝時,殿下鐘鳴,問英,對曰「蜀㟭山崩,山於銅為母,母崩子鳴,非聖朝災」,後蜀果上山崩,日月相應。二說微異,故並載之。}\textbf{遠公笑而不答。}

\subsection*{62}

\textbf{羊孚弟娶王永言女,}{\footnotesize 孚弟,輔也。\textbf{羊氏譜}曰輔,字幼仁,泰山人,祖楷,尚書郎,父綏,中書郎,輔仕至衛軍功曹,娶琅邪王訥之女,字僧首。}\textbf{及王家見壻,孚送弟俱往,時永言父東陽尚在,}{\footnotesize \textbf{王氏譜}曰訥之,字永言,琅邪人,祖彪之,光祿大夫,父臨之,東陽太守,訥之歷尚書左丞、御史中丞。}\textbf{殷仲堪是東陽女壻,亦在坐,}{\footnotesize \textbf{殷氏譜}曰仲堪娶琅邪王臨之女,字英彥。}\textbf{孚雅善理義,乃與仲堪道齊物,}{\footnotesize 莊子篇也。}\textbf{殷難之,羊云:「君四番後,當得見同。」殷笑曰:「乃可得盡,何必相同?」乃至四番後一通,殷咨嗟曰:「僕便無以相異。」歎為新拔者久之。}

\subsection*{63}

\textbf{殷仲堪云:「三日不讀道德經,便覺舌本間強。」}{\footnotesize \textbf{晉安帝紀}曰仲堪有思理,能清言。}

\subsection*{64}

\textbf{提婆初至,為東亭第講阿毗曇,}{\footnotesize \textbf{出經敘}曰僧伽提婆,罽賓人,姓瞿曇氏,儁朗有深鑒,苻堅至長安,出諸經,後渡江,遠法師請譯阿毗曇。\textbf{遠法師阿毗曇敘}曰阿毗曇心者,三藏之要領,詠歌之微言,源流廣大,管綜眾經,領其宗會,故作者以心為名焉,有出家開士字法勝,以阿毗曇源流廣大,卒難尋究,別撰斯部,凡二百五十偈,以為要解,號之曰心,罽賓沙門僧伽提婆,少玩斯文,因請令譯焉。阿毗曇者,晉言大法也。\textbf{道標法師}曰阿毗曇者,秦言無比法也。}\textbf{始發講,坐裁半,僧彌便云:「都已曉。」即於坐分數四有意道人,更就餘屋自講,提婆講竟,東亭問法岡道人曰:}{\footnotesize 法岡,未詳氏族。}\textbf{「弟子都未解,阿彌那得已解?所得云何?」曰:「大略全是,故當小未精覈耳。」}{\footnotesize \textbf{出經敘}曰提婆以隆安初遊京師,東亭侯王珣迎至舍講阿毗曇,提婆宗致既明,振發義奧,王僧彌一聽便自講,其明義易啓人心如此,未詳年卒。}

\subsection*{65}

\textbf{桓南郡與殷荊州共談,毎相攻難,年餘後,但一兩番,桓自歎才思轉退,殷云:「此乃是君轉解。」}{\footnotesize \textbf{周祗隆安記}曰玄善言理,棄郡還國,常與殷荊州仲堪終日談論不輟。}

\subsection*{66}

\textbf{文帝嘗令東阿王七步中作詩,不成者行大法,應聲便為詩曰:「煮豆持作羹,漉豉以為汁,萁在釜下然,豆在釜中泣,本自同根生,相煎何太急?」帝深有慚色。}{\footnotesize \textbf{魏志}曰陳思王植,字子建,文帝同母弟也,年十餘歲,誦詩論及辭賦數萬言,善屬文,太祖嘗視其文曰「汝倩人邪」,植跪曰「出言為論,下筆成章,顧當面試,奈何倩人」,時鄴銅雀臺新成,太祖悉將諸子登之,使各為賦,植援筆立成,可觀,性簡易,不治威儀,輿馬服飾,不尚華麗,毎見難問,應聲而答,太祖寵愛之,幾為太子者數矣,文帝即位,封鄄城侯,後徙雍丘,復封東阿,植毎求試不得,而國亟遷易,汲汲無懽,年四十一薨。}

\subsection*{67}

\textbf{魏朝封晉文王為公,備禮九錫,文王固讓不受,公卿將校當詣府敦喻,司空鄭沖}{\footnotesize 沖已見。}\textbf{馳遣信就阮籍求文,籍時在袁孝尼家,}{\footnotesize \textbf{袁氏世紀}曰準,字孝尼,陳郡陽夏人,父渙,魏郎中令,準忠信居正,不恥下問,唯恐人不勝己也,世事多險,故恬退不敢求進,著書十萬餘言。\textbf{荀綽兗州記}曰準有雋才,泰始中位給事中。}\textbf{宿醉扶起,書札為之,無所點定,乃寫付使,時人以為神筆。}{\footnotesize \textbf{顧愷之晉文章記}曰阮籍勸進,落落有宏致,至轉說徐而攝之也。一本注\textbf{阮籍勸進文}略曰竊聞明公固讓,沖等眷眷,實懷愚心,以為聖王作制,百代同風,褒德賞功,其來久矣,周公藉已成之業,據既安之勢,光宅曲阜,奄有龜蒙,明公宜奉聖旨,受茲介福也。}

\subsection*{68}

\textbf{左太沖作三都賦初成,}{\footnotesize \textbf{思別傳}曰思,字太沖,齊國臨淄人,父雍,起於筆札,多所掌練,為殿中御史,思蚤喪母,雍憐之,不甚教其書學,及長,博覽名文,遍閲百家,司空張華辟為祭酒,賈謐舉為祕書郎,謐誅,歸鄉里,專思著述,齊王冏請為記室參軍,不起,時為三都賦未成也,後數年疾終,其三都賦改定,至終乃上,初,作蜀都賦云「金馬電發於高岡,碧雞振翼而雲披,鬼彈飛丸以礌礉,火井騰光以赫曦」,今無鬼彈,故其賦往往不同,思為人无吏幹而有文才,又頗以椒房自矜,故齊人不重也。}\textbf{時人互有譏訾,思意不愜,後示張公,}{\footnotesize 張華已見。}\textbf{張曰:「此二京可三,然君文未重於世,宜以經高名之士。」思乃詢求於皇甫謐,}{\footnotesize \textbf{王隱晉書}曰謐,字士安,安定朝那人,漢太尉嵩曾孫也,祖叔獻,灞陵令,父叔侯,舉孝廉,謐族從皆累世富貴,獨守寒素,所養叔母歎曰「昔孟母以三徙成子,曾父以烹豕存教,豈我居不卜鄰,何爾魯之甚乎?修身篤學,自汝得之,於我何有」,因對之流涕,謐乃感激,年二十餘,就鄉里席坦受書,遭人而問,少有寧日,武帝借其書二車,遂博覽,太子中庶子、議郎徵,並不就,終於家。}\textbf{謐見之嗟歎,遂為作敘,於是先相非貳者,莫不斂袵讚述焉。}{\footnotesize \textbf{思別傳}曰思造張載,問㟭、蜀事,交接亦疎,皇甫謐西州高士,摯仲治宿儒知名,非思倫匹,劉淵林、衛伯輿並蚤終,皆不為思賦序注也,凡諸注解,皆思自為,欲重其文,故假時人名姓也。}

\subsection*{69}

\textbf{劉伶著酒德頌,意氣所寄。}{\footnotesize \textbf{名士傳}曰伶,字伯倫,沛郡人,肆意放蕩,以宇宙為狹,常乘鹿車,攜一壺酒,使人荷鍤隨之,云「死便掘地以埋」,土木形骸,遨游一世。\textbf{竹林七賢論}曰伶處天地間,悠悠蕩蕩,无所用心,嘗與俗士相牾,其人攘袂而起,欲必築之,伶和其色曰「雞肋豈足以當尊拳」,其人不覺廢然而返,未嘗措意文章,終其世,凡著酒德頌一篇而已,其辭曰「有大人先生者,以天地為一朝,萬朞為須臾,日月為扃牖,八荒為庭衢,行无轍迹,居无室廬,幕天席地,縱意所如,行則操巵執觚,動則挈榼提壺,唯酒是務,焉知其餘?有貴介公子,縉紳處士,聞吾風聲,議其所以,乃奮袂攘襟,怒目切齒,陳說禮法,是非鋒起,先生於是方捧罌承糟,銜杯漱醪,奮髯箕踞,枕麴藉糟,無思無慮,其樂陶陶,兀然而醉,慌爾而醒,靜聽不聞雷霆之聲,熟視不見太山之形,不覺寒暑之切肌,利欲之感情,俯觀萬物之擾擾,如江漢之載浮萍,二豪侍側焉,如蜾蠃之與螟蛉」。}

\subsection*{70}

\textbf{樂令善於清言,而不長於手筆,將讓河南尹,請潘岳為表,}{\footnotesize \textbf{晉陽秋}曰岳,字安仁,滎陽人,夙以才穎發名,善屬文,清綺絕世,蔡邕未能過也,仕至黃門侍郎,為孫秀所害。}\textbf{潘云:「可作耳,要當得君意。」樂為述己所以為讓,標位二百許語,潘直取錯綜,便成名筆,時人咸云:「若樂不假潘之文,潘不取樂之旨,則無以成斯矣。」}

\subsection*{71}

\textbf{夏侯湛作周詩成,}{\footnotesize \textbf{文士傳}曰湛,字孝若,譙國人,魏征西將軍夏侯淵曾孫也,有盛才,文章巧思,善補雅詞,名亞潘岳,歷中書侍郎。\textbf{湛集}載其敘曰周詩者,南陔、白華、華黍、由庚、崇丘、由儀六篇,有其義而亡其辭,湛續其亡,故云周詩也。}\textbf{示潘安仁,安仁曰:「此非徒溫雅,乃別見孝悌之性。」}{\footnotesize 其詩曰「既殷斯虔,仰說洪恩,夕定辰省,奉朝侍昏,宵中告退,雞鳴在門,孳孳恭誨,夙夜是敦」。}\textbf{潘因此遂作家風詩。}{\footnotesize 岳家風詩載其宗祖之德及自戒也。}

\subsection*{72}

\textbf{孫子荊除婦服,作詩以示王武子,}{\footnotesize \textbf{孫楚集}云婦,胡毋氏也。其詩曰「時邁不停,日月電流,神爽登遐,忽已一周,禮制有敘,告除靈丘,臨祠感痛,中心若抽」。}\textbf{王曰:「未知文生於情,情生於文,}{\footnotesize 一作「文於情生,情於文生」。}\textbf{覽之悽然,增伉儷之重。」}

\subsection*{73}

\textbf{太叔廣甚辯給,而摯仲治長於翰墨,俱為列卿,毎至公坐,廣談,仲治不能對,退著筆難廣,廣又不能答。}{\footnotesize \textbf{王隱晉書}曰廣,字季思,東平人,拜成都王為太弟,欲使詣洛,廣子孫多在洛,慮害,乃自殺。摯虞,字仲治,京兆長安人,祖茂,秀才,父模,太僕卿,虞少好學,師事皇甫謐,善校練文義,多所著述,歷祕書監、太常卿,從惠帝至長安,遂流離鄠杜間,性好博古,而文籍蕩盡,永嘉五年,洛中大饑,遂餓而死。虞與廣名位略同,廣長口才,虞長筆才,俱少政事,眾坐廣談,虞不能對,虞退筆難廣,廣不能答,於是更相嗤笑,紛然於世,廣無可記,虞多所錄,於斯為勝也。}

\subsection*{74}

\textbf{江左殷太常父子並能言理,亦有辯訥之異,揚州口談至劇,太常輒云:「汝更思吾論。」}{\footnotesize \textbf{中興書}曰殷融,字洪遠,陳郡人,桓彝有人倫鑒,見融甚歎美之,著象不盡意、大賢須易論,理義精微,談者稱焉,兄子浩亦能清言,毎與浩談,有時而屈,退而著論,融更居長,為司徒左西屬,飲酒善舞,終日嘯詠,未嘗以世務自嬰,累遷吏部尚書、太常卿,卒。}

\subsection*{75}

\textbf{庾子嵩作意賦成,}{\footnotesize \textbf{晉陽秋}曰敳,永嘉中為石勒所害,先是,敳見王室多難,知終嬰其禍,乃作意賦以寄懷。}\textbf{從子文康見,問曰:「若有意邪,非賦之所盡,若無意邪,復何所賦?」答曰:「正在有意無意之間。」}

\subsection*{76}

\textbf{郭景純詩云:「林無靜樹,川無停流。」}{\footnotesize \textbf{王隱晉書}曰郭璞,字景純,河東聞喜人,父瑗,建平太守。\textbf{璞別傳}曰璞奇博多通,文藻粲麗,才學賞豫,足參上流,其詩賦誄頌並傳於世,而訥於言,造次詠語,常人無異,又不持儀檢,形質穨索,縱情嫚惰,時有醉飽之失,友人干令升戒之曰「此伐性之斧也」,璞曰「吾所受有分,恆恐用之不盡,豈酒色之能害」,王敦取為參軍,敦縱兵都輦,乃咨以大事,璞極言成敗,不為回屈,敦忌而害之。詩,璞幽思篇者。}\textbf{阮孚云:}{\footnotesize 阮孚別見。}\textbf{「泓崢蕭瑟,實不可言,毎讀此文,輒覺神超形越。」}

\subsection*{77}

\textbf{庾闡始作揚都賦,道溫、庾云:「溫挺義之標,庾作民之望,方響則金聲,比德則玉亮。」庾公聞賦成,求看,兼贈貺之,闡更改望為雋、以亮為潤云。}{\footnotesize \textbf{中興書}曰闡,字仲初,潁川人,太尉亮之族也,少孤,九歲便能屬文,遷散騎侍郎,領大著作,為揚都賦,邈絕當時,五十四卒。}

\subsection*{78}

\textbf{孫興公作庾公誄,袁羊曰:「見此張緩。」于時以為名賞。}{\footnotesize \textbf{袁氏家傳}曰喬有文才。}

\subsection*{79}

\textbf{庾仲初作揚都賦成,以呈庾亮,亮以親族之懷,大為其名價云:「可三二京、四三都。」於此人人競寫,都下紙為之貴,謝太傅云:「不得爾,此是屋下架屋耳,事事擬學,而不免儉狹。」}{\footnotesize \textbf{王隱論揚雄太玄經}曰玄經雖妙,非益也,是以古人謂其屋下架屋。}

\subsection*{80}

\textbf{習鑿齒史才不常,宣武甚器之,未三十,便用為荊州治中,鑿齒謝牋亦云:「不遇明公,荊州老從事耳。」後至都見簡文,返命,宣武問:「見相王何如?」答云:「一生不曾見此人。」從此忤旨,出為衡陽郡,性理遂錯,於病中猶作漢晉春秋,品評卓逸。}{\footnotesize \textbf{續晉陽秋}曰鑿齒少而博學,才情秀逸,溫甚奇之,自州從事歲中三轉至治中,後以忤旨,左遷戶曹參軍、衡陽太守,在郡著漢晉春秋,斥溫覬覦之心也。\textbf{鑿齒集}載其論略曰靜漢末累世之交爭、廓九域之蒙晦、大定千載之盛功者,皆司馬氏也,若以魏有代王之德,則不足,有靜亂之功,則孫劉鼎立,共王、秦政猶不見敘於帝王,況暫制數州之眾哉?且漢有係周之業,則晉無所承魏之迹矣,春秋之時,吳楚稱王,若推有德,彼必自係於周,不推吳楚者也,況長轡廟堂,吳蜀兩定,天下之功也。}

\subsection*{81}

\textbf{孫興公云:「三都、二京,五經鼓吹。」}{\footnotesize 言此五賦是經典之羽翼。}

\subsection*{82}

\textbf{謝太傅問主簿陸退:}{\footnotesize \textbf{陸氏譜}曰退,字黎民,吳郡人,高祖凱,吳丞相,祖仰,吏部郎,父伊,州主簿,退仕至光祿大夫。}\textbf{「張憑何以作母誄,而不作父誄?」退答曰:「故當是丈夫之德,表於事行,婦人之美,非誄不顯。」}{\footnotesize \textbf{陸氏譜}曰退,憑壻也。}

\subsection*{83}

\textbf{王敬仁年十三,作賢人論,長史送示真長,真長答云:「見敬仁所作論,便足參微言。」}{\footnotesize \textbf{脩集}載其論曰或問「易稱賢人,黃裳元吉,苟未能闇與理會,何得不求通?求通則有損,有損則元吉之稱將虛設乎」,答曰「賢人誠未能闇與理會,當居然體從,比之理盡,猶一豪之領一梁,一豪之領一梁,雖於理有損,不足以撓梁,賢有情之至寡,豪有形之至小,豪不至撓梁,於賢人何有損之者哉」。}

\subsection*{84}

\textbf{孫興公云:「潘文爛若披錦,無處不善,}{\footnotesize \textbf{續文章志}曰岳為文,選言簡章,清綺絕倫。}\textbf{陸文若排沙簡金,往往見寶。」}{\footnotesize \textbf{文章傳}曰機善屬文,司空張華見其文章,篇篇稱善,猶譏其作文大治,謂曰「人之作文,患於不才,至子為文,乃患太多也」。}

\subsection*{85}

\textbf{簡文稱許掾云:「玄度五言詩,可謂妙絕時人。」}{\footnotesize \textbf{續晉陽秋}曰詢有才藻,善屬文,自司馬相如、王褒、揚雄諸賢,世尚賦頌,皆體則詩騷,傍綜百家之言,及至建安而詩章大盛,逮乎西朝之末,潘陸之徒雖時有質文,而宗歸不異也,正始中,王弼、何晏好莊老玄勝之談,而世遂貴焉,至江左李充尤盛,故郭璞五言始會合道家之言而韻之,詢及太原孫綽轉相祖尚,又加以三世之辭,而詩騷之體盡矣,詢、綽並為一時文宗,自此作者悉體之,至義熙中,謝混始改。}

\subsection*{86}

\textbf{孫興公作天台賦成,以示范榮期,}{\footnotesize \textbf{中興書}曰范啓,字榮期,慎陽人,父堅,護軍,啓以才義顯於世,仕至黃門郎。}\textbf{云:「卿試擲地,要作金石聲。」范曰:「恐子之金石,非宮商中聲。」然毎至佳句,}{\footnotesize 「赤城霞起而建標,瀑布飛流而界道」,此賦之佳處。}\textbf{輒云:「應是我輩語。」}

\subsection*{87}

\textbf{桓公見謝安石作簡文諡議,看竟,擲與坐上諸客曰:「此是安石碎金。」}{\footnotesize \textbf{劉謙之晉紀}載安議曰謹按諡法「一德不懈曰簡,道德博聞曰文」,易簡而天下之理得,觀乎人文,化成天下,儀之景行,猶有彷彿,宜尊號曰太宗,諡曰簡文。}

\subsection*{88}

\textbf{袁虎少貧,}{\footnotesize 虎,袁宏小字也。}\textbf{嘗為人傭載運租,謝鎮西經船行,其夜清風朗月,聞江渚間估客船上有詠詩聲,甚有情致,所誦五言,又其所未嘗聞,歎美不能已,即遣委曲訊問,乃是袁自詠其所作詠史詩,因此相要,大相賞得。}{\footnotesize \textbf{續晉陽秋}曰虎少有逸才,文章絕麗,曾為詠史詩,是其風情所寄,少孤而貧,以運租為業,鎮西謝尚時鎮牛渚,乘秋佳風月,率爾與左右微服泛江,會虎在運租船中諷詠,聲既清會,辭又藻拔,非尚所曾聞,遂住聽之,乃遣問訊,答曰「是袁臨汝郎誦詩,即其詠史之作也」,尚佳其率有勝致,即遣要迎,談話申旦,自此名譽日茂。}

\subsection*{89}

\textbf{孫興公云:「潘文淺而淨,陸文深而蕪。」}

\subsection*{90}

\textbf{裴郎作語林,始出,大為遠近所傳,時流年少無不傳寫,各有一通,載王東亭作經王公酒壚下賦,甚有才情。}{\footnotesize \textbf{裴氏家傳}曰裴榮,字榮期,河東人,父穉,豐城令,榮期少有風姿才氣,好論古今人物,撰語林數卷,號曰裴子。檀道鸞謂裴松之,以為啓作語林,榮儻別名啓乎?}

\subsection*{91}

\textbf{謝萬作八賢論,與孫興公往反,小有利鈍,}{\footnotesize \textbf{中興書}曰萬善屬文,能談論。\textbf{萬集}載其敘四隱四顯,為八賢之論,謂漁父、屈原、季主、賈誼、楚老、龔勝、孫登、嵇康也,其旨以處者為優,出者為劣,孫綽難之,以謂體玄識遠者,出處同歸。文多不載。}\textbf{謝後出以示顧君齊,}{\footnotesize \textbf{顧氏譜}曰夷,字君齊,吳郡人,祖廞,孝廉,父霸,少府卿,夷辟州主簿,不就。}\textbf{顧曰:「我亦作,知卿當無所名。」}

\subsection*{92}

\textbf{桓宣武命袁彥伯作北征賦,}{\footnotesize \textbf{續晉陽秋}曰宏從溫征鮮卑,故作北征賦,宏文之高者。}\textbf{既成,公與時賢共看,咸嗟歎之,時王珣在坐云:「恨少一句,得寫字足韻,當佳。」袁即於坐攬筆益云:「感不絕於余心,泝流風而獨寫。」公謂王曰:「當今不得不以此事推袁。」}{\footnotesize \textbf{宏集}載其賦云聞所聞於相傳,云獲麟於此野,誕靈物以瑞德,奚授體於虞者,悲尼父之慟泣,似實慟而非假,豈一物之足傷,實致傷於天下,感不絕於余心,遡流風而獨寫。\textbf{晉陽秋}曰宏嘗與王珣、伏滔同侍溫坐,溫令滔讀其賦,至「致傷於天下」,於此改韻,云「此韻所詠,慨深千載,今於天下之後便移韻,於寫送之致,如為未盡」,滔乃云「得益寫一句,或當小勝」,桓公語宏「卿試思益之」,宏應聲而益,王、伏稱善。}

\subsection*{93}

\textbf{孫興公道曹輔佐才如白地明光錦,}{\footnotesize \textbf{中興書}曰曹毗,字輔佐,譙國人,魏大司馬休曾孫也,好文籍,能屬詞,累遷太學博士、尚書郎、光祿勳。}\textbf{裁為負版絝,}{\footnotesize \textbf{論語}曰孔子式負版者。\textbf{鄭氏注}曰版,謂邦國籍也,負之者,賤隸人也。}\textbf{非無文采,酷無裁製。}

\subsection*{94}

\textbf{袁伯彥作名士傳成,}{\footnotesize 宏以夏侯太初、何平叔、王輔嗣為正始名士,阮嗣宗、嵇叔夜、山巨源、向子期、劉伯倫、阮仲容、王濬仲為竹林名士,裴叔則、樂彥輔、王夷甫、庾子嵩、王安期、阮千里、衛叔寶、謝幼輿為中朝名士。}\textbf{見謝公,公笑曰:「我嘗與諸人道江北事,特作狡獪耳,彥伯遂以著書。」}

\subsection*{95}

\textbf{王東亭到桓公吏,既伏閣下,桓公令人竊取其白事,東亭即於閣下更作,無復向一字。}{\footnotesize \textbf{續晉陽秋}曰珣學涉通敏,文高當世。}

\subsection*{96}

\textbf{桓宣武北征,}{\footnotesize \textbf{溫別傳}曰溫以太和四年上疏自征鮮卑。}\textbf{袁虎時從,被責免官,會須露布文,喚袁倚馬前令作,手不輟筆,俄得七紙,殊可觀,東亭在側,極歎其才,袁虎云:「當令齒舌間得利。」}

\subsection*{97}

\textbf{袁宏始作東征賦,都不道陶公,胡奴誘之狹室中,臨以白刃,}{\footnotesize 胡奴,陶範,別見。}\textbf{曰:「先公勳業如是,君作東征賦,云何相忽略?」宏窘蹙無計,便答:「我大道公,何以云無?」因誦曰:「精金百鍊,在割能斷,功則治人,職思靖亂,長沙之勳,為史所讚。」}{\footnotesize \textbf{續晉陽秋}曰宏為大司馬記室參軍,後為東征賦,悉稱過江諸名望,時桓溫在南州,宏語眾云「我決不及桓宣城」,時伏滔在溫府,與宏善,苦諫之,宏笑而不答,滔密以啓溫,溫甚忿,以宏一時文宗,又聞此賦有聲,不欲令人顯聞之,後遊青山飲酌,既歸,公命宏同載,眾為危懼,行數里,問宏曰「聞君作東征賦,多稱先賢,何故不及家君」,宏答曰「尊公稱謂,自非下官所敢專,故未呈啓,不敢顯之耳」,溫乃云「君欲為何辭」,宏即答云「風鑒散朗,或搜或引,身雖可亡,道不可隕,則宣城之節,信為允也」,溫泫然而止。二說不同,故詳載焉。}

\subsection*{98}

\textbf{或問顧長康:「君箏賦何如嵇康琴賦?」顧曰:「不賞者,作後出相遺,深識者,亦以高奇見貴。」}{\footnotesize \textbf{中興書}曰愷之博學有才氣,為人遲鈍而自矜尚,為時所笑。\textbf{宋明帝文章志}曰桓溫云「顧長康體中癡黠各半,合而論之,正平平耳」,世云有三絕,畫絕、文絕、癡絕。\textbf{續晉陽秋}曰愷之矜伐過實,諸年少因相稱譽,以為戲弄,為散騎常侍,與謝瞻連省,夜於月下長詠,自云得先賢風制,瞻毎遙贊之,愷之得此,彌自力忘倦,瞻將眠,語搥腳人令代,愷之不覺有異,遂幾申旦而後止。}

\subsection*{99}

\textbf{殷仲文天才宏贍,}{\footnotesize \textbf{續晉陽秋}曰仲文雅有才藻,著文數十篇。}\textbf{而讀書不甚廣,傅亮歎曰:}{\footnotesize 亮別見。}\textbf{「若使殷仲文讀書半袁豹,}{\footnotesize \textbf{丘淵之文章敘}曰豹,字士蔚,陳郡人,祖耽,歷陽太守,父質,琅邪內史,豹隆安中著作佐郎,累遷太尉長史、丹陽尹,義熙九年卒。}\textbf{才不減班固。」}{\footnotesize \textbf{續漢書}曰固,字孟堅,右扶風人,幼有儁才,學無常師,善屬文,經傳無不究覽。}

\subsection*{100}

\textbf{羊孚作雪贊云:「資清以化,乘氣以霏,遇象能鮮,即潔成輝。」桓胤遂以書扇。}{\footnotesize \textbf{中興書}曰胤,字茂祖,譙國人,祖沖,太尉,父嗣,江州刺史,胤少有清操,以恬退見稱,仕至中書令,玄敗,徙安成郡,後見誅。}

\subsection*{101}

\textbf{王孝伯在京行散,至其弟王睹戶前,}{\footnotesize 睹,王爽小字也。\textbf{中興書}曰爽,字季明,恭第四弟也,仕至侍中,恭事敗,贈太常。}\textbf{問:「古詩中何句為最?」睹思未答,孝伯詠:「所遇無故物、焉得不速老,此句為佳。」}

\subsection*{102}

\textbf{桓玄嘗登江陵城南樓云:「我今欲為王孝伯作誄。」因吟嘯良久,隨而下筆,一坐之間,誄以之成。}{\footnotesize \textbf{晉安帝紀}曰玄文翰之美,高於一世。\textbf{玄集}載其誄敘曰隆安二年九月十七日,前將軍、青兗二州刺史、太原王孝伯薨,川岳降神,哲人是育,既爽其靈,不貽其福,天道茫昧,孰測倚伏,犬馬反噬,豺狼翹陸,嶺摧高梧,林殘故竹,人之云亡,邦國喪牧,于以誄之,爰旌芳郁。文多,不盡載。}

\subsection*{103}

\textbf{桓玄初并西夏,領荊江二州,二府一國,}{\footnotesize \textbf{玄別傳}曰玄既克殷仲堪,後楊佺期,遣使諷朝廷,朝廷以玄都督八州,領江州、荊州二刺史。}\textbf{于時始雪,五處俱賀,五版並入,玄在聽事上,版至即答版後,皆粲然成章,不相揉雜。}

\subsection*{104}

\textbf{桓玄下都,羊孚時為兗州別駕,從京來詣門,牋云:「自頃世故睽離,心事淪薀,明公啓晨光於積晦,澄百流以一源。」桓見牋,馳喚前,云:「子道,子道,來何遲?」即用為記室參軍,孟昶}{\footnotesize 別見。}\textbf{為劉牢之主簿,}{\footnotesize \textbf{續晉陽秋}曰牢之,字道堅,彭城人,世以將顯,父遁,征虜將軍,牢之沈毅多計數,為謝玄參軍,苻堅之役,以驍猛成功,及平王恭,轉徐州刺史,桓玄下都,以牢之為前鋒,行征西將軍,玄至歸降,用為會稽內史,欲解其兵,奔而縊死。}\textbf{詣門謝,見云:「羊侯,羊侯,百口賴卿。」}