\chapter{賞譽第八}

\subsection*{1}

\textbf{陳仲舉嘗歎曰:「若周子居者,真治國之器,}{\footnotesize \textbf{汝南先賢傳}曰周乘,字子居,汝南安城人,天姿聰朗,高峙嶽立,非陳仲舉、黃叔度之儔則不交也,仲舉嘗歎曰「周子居者,真治國之器也」,為太山太守,甚有惠政。}\textbf{譬諸寶劍,則世之干將。」}{\footnotesize \textbf{吳越春秋}曰吳王闔閭請干將作劍,干將者,吳人,其妻曰莫邪,干將採五山之精、六金之英,候天地,伺陰陽,百神臨視,而金鐵之精未流,夫妻乃翦髮及爪而投之鑪中,金鐵乃濡,遂成二劍,陽曰干將,而作龜文,陰曰莫邪,而作漫理,干將匿其陽,出其陰以獻闔閭,闔閭甚寶重之。}

\subsection*{2}

\textbf{世目李元禮:「謖謖如勁松下風。」}{\footnotesize \textbf{李氏家傳}曰膺嶽峙淵清,峻貌貴重,華夏稱曰「潁川李府君,頵頵如玉山,汝南陳仲舉,軒軒若千里馬,南陽朱公叔,飂飂如行松柏之下」。}

\subsection*{3}

\textbf{謝子微見許子將兄弟,曰:「平輿之淵,有二龍焉。」見許子政弱冠之時,歎曰:「若許子政者,有幹國之器,正色忠謇,則陳仲舉之匹,}{\footnotesize \textbf{汝南先賢傳}曰謝甄,字子微,汝南邵陵人,明識人倫,雖郭林宗不及甄之鑒也,見許子將兄弟弱冠時,則曰「平輿之淵有二龍」,仕為豫章從事。許虔,字子政,平輿人,體尚高潔,雅正寬亮,謝子微見虔兄弟,歎曰「若許子政者,幹國之器也」。虔弟劭,聲未發時,時人以謂不如虔,虔恆撫髀稱劭,自以為不及也,釋褐為郡功曹,黜姦廢惡,一郡肅然,年三十五卒。\textbf{海內先賢傳}曰許劭,字子將,虔弟也,山峙淵停,行應規表,邵陵謝子微高才遠識,見劭十歲時,歎曰「此乃希世之偉人也」,初,劭拔樊子昭於市肆,出虞承賢於客舍,召李叔才於無聞,擢郭子瑜於小吏,廣陵徐孟本來臨汝南,聞劭高名,召功曹,時袁紹以公族為濮陽長,棄官還,副車從騎將入郡界,乃歎曰「許子將秉持清格,豈可以吾輿服見之邪」,遂單馬而歸,辟公府掾,敦辟皆不就,避地江南,卒於豫章也。}\textbf{伐惡退不肖,范孟博之風。」}{\footnotesize \textbf{張璠漢紀}曰范滂,字孟博,汝南伊陽人,為功曹,辟公府掾,升車攬轡,有澄清天下之志,百城聞滂高名,皆解印綬去,為黨事見誅。}

\subsection*{4}

\textbf{公孫度目邴原:「所謂雲中白鶴,非燕雀之網所能羅也。」}{\footnotesize \textbf{魏書}曰度,字叔濟,襄平人,累遷冀州刺史、遼東太守。\textbf{邴原別傳}曰原,字根矩,東管朱虛人,少孤,數歲時過書舍而泣,師問曰「童子何泣也」,原曰「凡得學者,有親也,一則願其不孤,二則羡其得學,中心感傷,故泣耳」,師惻然曰「苟欲學,不須資也」,於是就業,長則博覽洽聞,金玉其行,知世將亂,避地遼東,公孫度厚禮之,中國既寧,欲還鄉里,為度禁絕,原密自治嚴,謂部落曰「移比近郡」,以觀其意,皆曰「樂移」,原舊有捕魚大船,請村落,皆令熟醉,因夜去之,數日,度乃覺,吏欲追之,度曰「邴君所謂雲中白鶴,非鶉鷃之網所能羅也」,魏王辟祭酒,累遷五官中郎長史。}

\subsection*{5}

\textbf{鍾士季目王安豐「阿戎了了解人意」,}{\footnotesize \textbf{王隱晉書}曰戎少清明曉悟。}\textbf{謂「裴公之談,經日不竭」,}{\footnotesize 裴頠已見。}\textbf{吏部郎闕,文帝問其人於鍾會,會曰:「裴楷清通,王戎簡要,皆其選也。」於是用裴。}{\footnotesize \textbf{按}諸書皆云鍾會薦裴楷、王戎於晉文王,文王辟以為掾,不聞為吏部郎。}

\subsection*{6}

\textbf{王濬沖、裴叔則二人總角詣鍾士季,須臾去後,客問鍾曰:「向二童何如?」鍾曰:「裴楷清通,王戎簡要,後二十年,此二賢當為吏部尚書,冀爾時天下無滯才。」}{\footnotesize \textbf{晉陽秋}曰戎為兒童,鍾會異之。}

\subsection*{7}

\textbf{諺曰:「後來領袖有裴秀。」}{\footnotesize \textbf{虞預晉書}曰秀,字季彥,河東聞喜人,父潛,魏太常,秀有風操,八歲能著文,叔父徽,有聲名,秀年十餘歲,有賓客詣徽,出則過秀,時人為之語曰「後進領袖有裴秀」,大將軍辟為掾,父終,推財與兄,年二十五,遷黃門侍郎,晉受禪,封鉅鹿公,後累遷左光祿、司空,四十八薨,諡元公,配食宗廟。}

\subsection*{8}

\textbf{裴令公目夏侯太初:「肅肅如入廊廟中,不修敬而人自敬。」}{\footnotesize \textbf{禮記}曰周豐謂魯哀公曰「宗廟社稷之中,未施敬而民自敬」。}\textbf{一曰:「如入宗廟,琅琅但見禮樂器,見鍾士季,如觀武庫,但覩矛戟,見傅蘭碩,汪廧靡所不有,見山巨源,如登山臨下,幽然深遠。」}{\footnotesize 玄、會、嘏、濤並已見上。}

\subsection*{9}

\textbf{羊公還洛,郭奕為野王令,}{\footnotesize \textbf{晉諸公贊}曰奕,字泰業,太原陽曲人,累世舊族,奕有才望,歷雍州刺史、尚書。}\textbf{羊至界,遣人要之,郭便自往,既見,歎曰:「羊叔子何必減郭太業。」復往羊許,小悉還,又歎曰:「羊叔子去人遠矣。」羊既去,郭送之彌日,一舉數百里,遂以出境免官,復歎曰:「羊叔子何必減顏子。」}

\subsection*{10}

\textbf{王戎目山巨源:「如璞玉渾金,人皆欽其寶,莫知名其器。」}{\footnotesize \textbf{顧愷之畫贊}曰濤無所標名,淳深淵默,人莫見其際,而囂然亦入道,故見者莫能稱謂,而服其偉量。}

\subsection*{11}

\textbf{羊長和父繇,與太傅祜同堂相善,仕至車騎掾,蚤卒,長和兄弟五人,幼孤,}{\footnotesize \textbf{羊氏譜}曰繇,字堪甫,太山人,祖續,漢太尉,不拜,父祕,京兆太守,繇歷車騎掾,娶樂國禎女,生五子,秉、洽、式、亮、忱也。}\textbf{祜來哭,見長和哀容舉止,宛若成人,乃歎曰:「從兄不亡矣。」}

\subsection*{12}

\textbf{山公舉阮咸為吏部郎,目曰:「清真寡欲,萬物不能移也。」}{\footnotesize \textbf{名士傳}曰咸,字仲容,陳留人,籍兄子也,任達不拘,當世皆怪其所為,及與之處,少嗜欲,哀樂至到,過絕於人,然後皆忘其向議,為散騎侍郎,山濤舉為吏部,武帝不用,太原郭奕見之心醉,不覺歎服,解音,好酒以卒。\textbf{山濤啓事}曰吏部郎史曜出,處缺,當選,濤薦咸曰「真素寡欲,深識清濁,萬物不能移也,若在官人之職,必妙絕於時」,詔用陸亮。\textbf{晉陽秋}曰咸行己多違禮度,濤舉以為吏部郎,世祖不許。\textbf{竹林七賢論}曰山濤之舉阮咸,固知上不能用,蓋惜曠世之儁,莫識其真故耳,夫以咸之所犯,方外之意,稱其清真寡欲,則迹外之意自見耳。}

\subsection*{13}

\textbf{王戎目阮文業:「清倫有鑒識,漢元以來,未有此人。」}{\footnotesize \textbf{杜篤新書}曰阮武,字文業,陳留尉氏人,父諶,侍中,武闊達博通,淵雅之士。\textbf{陳留志}曰武,魏末清河太守,族子籍,年總角未知名,武見而偉之,以為勝己,知人多此類,著書十八篇,謂之阮子,終於家。郭泰友人宋子俊稱泰「自漢元以來,未有林宗之匹」。}

\subsection*{14}

\textbf{武元夏目裴、王曰:「戎尚約,楷清通。」}{\footnotesize \textbf{虞預晉書}曰武陔,字元夏,沛國竹邑人,父周,魏光祿大夫,陔及二弟歆、茂皆總角見稱,並有品望,鄉人諸父未能覺其多少,時同郡劉公榮名知人,嘗造周,周見其三子,公榮曰「君三子皆國士,元夏器量最優,有輔佐之風,力仕宦,可為亞公,叔夏、季夏不減常伯、納言也」,陔至左僕射。}

\subsection*{15}

\textbf{庾子嵩目和嶠:「森森如千丈松,雖磊砢有節目,施之大廈,有棟梁之用。」}{\footnotesize \textbf{晉諸公贊}曰嶠常慕其舅夏侯玄為人,故於朝士中峨然不群,時類憚其風節。}

\subsection*{16}

\textbf{王戎云:「太尉神姿高徹,如瑤林瓊樹,自然是風塵外物。」}{\footnotesize \textbf{名士傳}曰夷甫天形奇特,明秀若神。\textbf{八王故事}曰石勒見夷甫,謂長史孔萇曰「吾行天下多矣,未嘗見如此人,當可活不」,萇曰「彼晉三公,不為我用」,勒曰「雖然,要不可加以鋒刃也」,夜使推牆殺之。}

\subsection*{17}

\textbf{王汝南既除所生服,遂停墓所,兄子濟每來拜墓,略不過叔,叔亦不候,濟脫時過,止寒溫而已,後聊試問近事,答對甚有音辭,出濟意外,濟極惋愕,仍與語,轉造清微,濟先略無子姪之敬,既聞其言,不覺懍然,心形俱肅,遂留共語,彌日累夜,濟雖儁爽,自視缺然,乃喟然歎曰:「家有名士,三十年而不知。」濟去,叔送至門,濟從騎有一馬絕難乘,少能騎者,濟聊問叔:「好騎乘不?」曰:「亦好爾。」濟又使騎難乘馬,叔姿形既妙,回策如縈,名騎無以過之,濟益歎其難測非復一事,}{\footnotesize \textbf{鄧粲晉紀}曰王湛,字處沖,太原人,隱德,人莫之知,雖兄弟宗族亦以為癡,惟父昶異焉,昶喪,居墓次,兄子濟往省湛,見牀頭有周易,謂湛曰「叔父用此何為,頗曾看不」,湛笑曰「體中佳時,脫復看耳,今日當與汝言」,因共談易,剖析入微,妙言奇趣,濟所未聞,歎不能測,濟性好馬,而所乘馬駿駛,意甚愛之,湛曰「此雖小駛,然力薄不堪苦,近見督郵馬當勝此,但養不至耳」,濟取督郵馬,穀食十數日,與湛試之,湛未嘗乘馬,卒然便馳騁,步驟不異於濟,而馬不相勝,湛曰「今直行車路,何以別馬勝不,惟當就蟻封耳」,於是就蟻封盤馬,果倒踣,其儁識天才乃爾。}\textbf{既還,渾問濟:「何以暫行累日?」濟曰:「始得一叔。」渾問其故,濟具歎述如此,渾曰:「何如我?」濟曰:「濟以上人。」武帝每見濟,輒以湛調之曰:「卿家癡叔死未?」濟常無以答,既而得叔,後武帝又問如前,濟曰:「臣叔不癡。」稱其實美,帝曰:「誰比?」濟曰:「山濤以下,魏舒以上。」}{\footnotesize \textbf{晉陽秋}曰濟有人倫鑒識,其雅俗是非,少有優調,見湛,歎服其德宇,時人謂湛「上方山濤不足,下比魏舒有餘」,湛聞之曰「欲以我處季孟之間乎」。\textbf{王隱晉書}曰魏舒,字陽元,任城人,幼孤,為外氏甯家所養,甯氏起宅,相者曰「當出貴甥」,外祖母意以盛氏甥小而惠,謂應相也,舒曰「當為外氏成此宅相」,少名遲鈍,叔父衡使守水碓,每言「舒堪八百戶長,我願畢矣」,舒不以介意,身長八尺二寸,不修常人近事,少工射,著韋衣入山澤,每獵大獲,為後將軍鍾毓長史,毓與參佐射戲,舒常為坐畫籌,後值朋人少,以舒充數,於是發無不中,加博措閑雅,殆盡其妙,毓歎謝之曰「吾之不足盡卿,如此射矣」,轉相國參軍,晉王每朝罷,目送之曰「魏舒堂堂,人之領袖」,累遷侍中、司徒。}\textbf{於是顯名,年二十八始宦。}

\subsection*{18}

\textbf{裴僕射,時人謂為「言談之林藪」。}{\footnotesize \textbf{惠帝起居注}曰頠理甚淵博,贍於論難。}

\subsection*{19}

\textbf{張華見褚陶,語陸平原曰:「君兄弟龍躍雲津,顧彥先鳳鳴朝陽,謂東南之寶已盡,不意復見褚生。」陸曰:「公未覩不鳴不躍者耳。」}{\footnotesize \textbf{褚氏家傳}曰陶,字季雅,吳郡錢塘人,褚先生後也,陶聰惠絕倫,年十三,作鷗鳥、水碓二賦,宛陵嚴仲弼見而奇之曰「褚先生復出矣」,弱不好弄,清淡閑默,以墳典自娛,語所親曰「聖賢備在黃卷中,舍此何求」,州郡辟不就,吳歸命世祖,補臺郎、建忠校尉,司空張華與陶書曰「二陸龍躍於江漢,彥先鳳鳴於朝陽,自此以來,常恐南金已盡,而復得之於吾子,故知延州之德不孤,淵岱之寶不匱」,仕至中尉。}

\subsection*{20}

\textbf{有問秀才:「吳舊姓何如?」答曰:「吳府君聖王之老成,明時之俊乂,朱永長理物之至德,清選之高望,嚴仲弼九皋之鳴鶴,空谷之白駒,顧彥先八音之琴瑟,五色之龍章,張威伯歲寒之茂松,幽夜之逸光,陸士衡、士龍鴻鵠之裴回,懸鼓之待槌,}{\footnotesize 秀才,蔡洪也。集載\textbf{洪與刺史周俊書}曰一日侍坐,言及吳士,詢於芻蕘,遂見下問,造次承顏,載辭不舉,敕令條列名狀,退輒思之,今稱疏所知。吳展,字士季,下邳人,忠足矯非,清足厲俗,信可結神,才堪幹世,仕吳為廣州刺史、吳郡太守,吳平,還下邳,閉門自守,不交賓客,誠聖王之老成,明時之儁乂也。朱誕,字永長,吳郡人,體履清和,黃中通理,吳朝舉賢良,累遷議郎,今歸在家,誠理物之至德,清選之高望也。嚴隱,字仲弼,吳郡人,稟氣清純,思度淵偉,吳朝舉賢良,宛陵令,吳平,去職,九皋之鳴鶴,空谷之白駒也。張暢,字威伯,吳郡人,稟性堅明,志行清朗,居磨涅之中,無淄磷之損,歲寒之松柏,幽夜之逸光也。\textbf{陸雲別傳}曰雲,字士龍,吳大司馬抗之第五子,機同母之弟也,儒雅有俊才,容貌瓌偉,口敏能談,博聞彊記,善著述,六歲便能賦詩,時人以為項託、揚烏之儔也,年十八,刺史周俊命為主簿,俊常歎曰「陸士龍,當今之顏淵也」,累遷太子舍人、清河內史,為成都王所害。}\textbf{凡此諸君,以洪筆為鉏耒,以紙札為良田,以玄默為稼穡,以義理為豐年,以談論為英華,以忠恕為珍寶,著文章為錦繡,蘊五經為繒帛,坐謙虛為席薦,張義讓為帷幕,行仁義為室宇,修道德為廣宅。」}{\footnotesize \textbf{按}蔡所論士十六人,無陸機兄弟,又無「凡此諸君」以下,疑益之。}

\subsection*{21}

\textbf{人問王夷甫:「山巨源義理何如,是誰輩?」王曰:「此人初不肯以談自居,然不讀老莊,時聞其詠,往往與其旨合。」}{\footnotesize \textbf{顧愷之畫贊}曰濤有而不恃,皆此類也。}

\subsection*{22}

\textbf{洛中雅雅有三嘏,劉粹字純嘏,宏字終嘏,漠字沖嘏,是親兄弟,王安豐甥,並是王安豐女壻,宏,真長祖也。}{\footnotesize \textbf{晉諸公贊}曰粹,沛國人,歷侍中、南中郎將。宏,歷祕書監、光祿大夫。\textbf{晉後略}曰漠少以清識為名,與王夷甫友善,並好以人倫為意,故世人許以才智之名,自相國右長史出為襄州刺史,以貴簡稱。\textbf{按}劉氏譜「劉邠妻,武周女,生粹、宏、漠」,非王氏甥。}\textbf{洛中錚錚馮惠卿,名蓀,是播子,}{\footnotesize \textbf{晉後略}曰播,字友聲,長樂人,位至大宗正,生蓀。\textbf{八王故事}曰蓀少以才悟,識當世之宜,蚤歷清職,仕至侍中,為長沙王所害。}\textbf{蓀與邢喬俱司徒李胤外孫,及胤子順並知名,時稱「馮才清,李才明,純粹邢」。}{\footnotesize \textbf{晉諸公贊}曰喬,字曾伯,河間人,有才學,仕至司隸校尉。順,字曼長,仕至太僕卿。}

\subsection*{23}

\textbf{衛伯玉為尚書令,見樂廣與中朝名士談議,奇之曰:「自昔諸人沒已來,常恐微言將絕,今乃復聞斯言於君矣。」命子弟造之曰:「此人,人之水鏡也,見之若披雲霧覩青天。」}{\footnotesize \textbf{晉陽秋}曰尚書令衛瓘見廣曰「昔何平叔諸人沒,常謂清言盡矣,今復聞之於君」。\textbf{王隱晉書}曰衛瓘有名理,及與何晏、鄧颺等數共談講,見廣奇之曰「每見此人,則瑩然猶廓雲霧而覩青天」。}

\subsection*{24}

\textbf{王太尉曰:「見裴令公精明朗然,籠蓋人上,非凡識也,若死而可作,當與之同歸。」或云王戎語。}{\footnotesize \textbf{禮記}曰趙文子與叔譽觀於九原,文子曰「死者如可作也,吾誰與歸」。\textbf{鄭玄}曰作,起也。}

\subsection*{25}

\textbf{王夷甫自歎:「我與樂令談,未嘗不覺我言為煩。」}{\footnotesize \textbf{晉陽秋}曰樂廣善以約言厭人心,其所不知,默如也,太尉王夷甫、光祿大夫裴叔則能清言,常曰「與樂君言,覺其簡至,吾等皆煩」。}

\subsection*{26}

\textbf{郭子玄有俊才,能言老莊,庾敳嘗稱之,每曰:「郭子玄何必減庾子嵩。」}{\footnotesize \textbf{名士傳}曰郭象,字子玄,自黃門郎為太傅主簿,任事用勢,傾動一府,敳謂象曰「卿自是當世大才,我疇昔之意,都已盡矣」,其伏理推心,皆此類也。}

\subsection*{27}

\textbf{王平子目太尉:「阿兄形似道,而神鋒太儁。」太尉答曰:「誠不如卿落落穆穆。」}{\footnotesize \textbf{王隱晉書}曰澄通朗好人倫,情無所繫。}

\subsection*{28}

\textbf{太傅府有三才,劉慶孫長才,}{\footnotesize \textbf{晉陽秋}曰太傅將召劉輿,或曰「輿猶膩也,近將汙人」,太傅疑而禦之,輿乃密視天下兵簿諸屯戍及倉庫處所,人穀多少、牛馬器械、水陸地形皆默識之,是時軍國多事,每會議事,自潘滔以下皆不知所對,輿便屈指籌計,所發兵仗處所,糧廩運轉,事無凝滯,於是太傅遂委仗之。}\textbf{潘陽仲大才,裴景聲清才。}{\footnotesize \textbf{八王故事}曰劉輿才長綜覈,潘滔以博學為名,裴邈彊力方正,皆為東海王所暱,俱顯一府,故時人稱曰「輿長才,滔大才,邈清才」也。}

\subsection*{29}

\textbf{林下諸賢各有儁才子,籍子渾器量弘曠,}{\footnotesize \textbf{世語}曰渾,字長成,清虛寡欲,位至太子中庶子。}\textbf{康子紹清遠雅正,}{\footnotesize 已見。}\textbf{濤子簡疏通高素,}{\footnotesize \textbf{虞預晉書}曰簡,字季倫,平雅有父風,與嵇紹、劉漠等齊名,遷尚書,出為征南將軍。}\textbf{咸子瞻虛夷有遠志,瞻弟孚爽朗多所遺,}{\footnotesize \textbf{名士傳}曰瞻,字千里,夷任而少嗜欲,不修名行,自得於懷,讀書不甚研求而識其要,仕至太子舍人,年三十卒。\textbf{中興書}曰孚風韻疎誕,少有門風,初為安東參軍,蓬髮飲酒,不以王務嬰心。}\textbf{秀子純、悌並令淑有清流,}{\footnotesize \textbf{竹林七賢論}曰純,字長悌,位至侍中。悌,字叔遜,位至御史中丞。\textbf{晉諸公贊}曰洛陽敗,純、悌出奔,為賊所害。}\textbf{戎子萬子有大成之風,苗而不秀,}{\footnotesize \textbf{晉諸公贊}曰王綏,字萬子,辟太尉掾,不就,年十九卒。\textbf{晉書}曰戎子萬,有美號而太肥,戎令食穅,而肥愈甚也。}\textbf{惟伶子無聞,凡此諸子,惟瞻為冠,紹、簡亦見重當世。}

\subsection*{30}

\textbf{庾子躬有廢疾,甚知名,家在城西,號曰城西公府。}{\footnotesize \textbf{虞預晉書}曰琮,字子躬,潁川人,太常峻第二子,仕至太尉掾。}

\subsection*{31}

\textbf{王夷甫語樂令:「名士無多人,故當容平子知。」}{\footnotesize \textbf{王澄別傳}曰澄風韻邁達,志氣不群,從兄戎,兄夷甫,名冠當年,四海人士一為澄所題目,則二兄不復措意,云「已經平子」,其見重如此,是以名聞益盛,天下知與不知莫不傾注,澄後事迹不逮,朝野失望,及舊遊識見者,猶曰「當今名士也」。}

\subsection*{32}

\textbf{王太尉云:「郭子玄語議如懸河寫水,注而不竭。」}{\footnotesize \textbf{名士傳}曰子玄有儁才,能言莊老。}

\subsection*{33}

\textbf{司馬太傅府多名士,一時儁異,庾文康云:「見子嵩在其中,常自神王。」}{\footnotesize \textbf{晉陽秋}曰敳為太傅從事中郎。}

\subsection*{34}

\textbf{太傅東海王鎮許昌,以王安期為記室參軍,雅相知重,敕世子毗曰:「夫學之所益者淺,體之所安者深,閑習禮度,不如式瞻儀形,諷味遺言,不如親承音旨,王參軍人倫之表,汝其師之。」或曰「王、趙、鄧三參軍,人倫之表,汝其師之」,謂安期、鄧伯道、趙穆也,}{\footnotesize \textbf{趙吳郡行狀}曰穆,字季子,汲郡人,貞淑平粹,才識清通,歷尚書郎、太傅參軍,後太傅越與穆及王承、阮瞻、鄧攸書曰「禮,八歲出就外傅,十年曰幼,學,明可以漸先王之教也,然學之所受者淺,體之所安者深,是以閑習禮度,不如式瞻軌儀,諷味遺言,不如親承辭旨,小兒毗既無令淑之資,未聞道德之風,欲屈諸君,時以閑豫,周旋燕誨也」,穆歷晉明帝師、冠軍將軍、吳郡太守,封南鄉侯。}\textbf{袁宏作名士傳,直云「王參軍」,或云趙家先猶有此本。}

\subsection*{35}

\textbf{庾太尉少為王眉子所知,庾過江,歎王曰:「庇其宇下,使人忘寒暑。」}{\footnotesize \textbf{晉諸公贊}曰玄少希慕簡曠。\textbf{八王故事}曰玄為陳留太守,或勸玄過江投琅邪王,玄曰「王處仲得志於彼,家叔猶不免害,豈能容我」,謂其器宇不容於敦也。}

\subsection*{36}

\textbf{謝幼輿曰:「友人王眉子清通簡暢,嵇延祖弘雅劭長,董仲道卓犖有致度。」}{\footnotesize \textbf{王隱晉書}曰董養,字仲道,太始初到洛,不干祿求榮,永嘉中,洛城東北角步廣里中地陷,中有二鵝,蒼者飛去,白者不能飛,問之博識者,不能知,養聞,歎曰「昔周時所盟會狄泉,此地也,卒有二鵝,蒼者胡象,後胡當入洛,白者不能飛,此國諱也」。\textbf{謝鯤元化論序}曰陳留董仲道於元康中見惠帝廢楊悼后,升太學堂,歎曰「建此堂也,將何為乎?每見國家赦書,謀反逆皆赦,孫殺王父母、子殺父母不赦,以為王法所不容也,奈何公卿處議,文飾禮典以至此乎?天人之理既滅,大亂斯起」,顧謂謝鯤、阮孚曰「易稱知幾其神乎,君等可深藏矣」,乃與妻荷擔入蜀,莫知其所終。}

\subsection*{37}

\textbf{王公目太尉:「巖巖清峙,壁立千仞。」}{\footnotesize \textbf{顧愷之夷甫畫贊}曰夷甫天形瓌特,識者以為巖巖秀峙,壁立千仞。}

\subsection*{38}

\textbf{庾太尉在洛下,問訊中郎,}{\footnotesize 庾敳。}\textbf{中郎留之云:「諸人當來。」尋溫元甫、}{\footnotesize \textbf{晉諸公贊}曰溫幾,字元甫,太原人,才性清婉,歷司徒右長史、湘州刺史,卒官。}\textbf{劉王喬、}{\footnotesize \textbf{曹嘉之晉紀}曰劉疇,字王喬,彭城人,父訥,司隸校尉,疇善談名理,曾避亂塢壁,有胡數百欲害之,疇無懼色,援笳而吹之,為出塞入塞之聲,以動其遊客之思,於是群胡皆泣而去之,位至司徒左長史。}\textbf{裴叔則俱至,酬酢終日,庾公猶憶劉、裴之才儁、元甫之清中。}{\footnotesize 中,一作平。}

\subsection*{39}

\textbf{蔡司徒在洛,見陸機兄弟住參佐廨中,三間瓦屋,士龍住東頭,士衡住西頭,士龍為人文弱可愛,士衡長七尺餘,聲作鍾聲,言多忼慨。}{\footnotesize \textbf{文士傳}曰雲性弘靜,怡怡然為士友所宗,機清厲有風格,為鄉黨所憚。}

\subsection*{40}

\textbf{王長史是庾子躬外孫,}{\footnotesize \textbf{王氏譜}曰濛父訥,娶潁州庾琮之女,字三壽也。}\textbf{丞相目子躬云:「入理泓然,我已上人。」}{\footnotesize 子躬,子嵩兄也。}

\subsection*{41}

\textbf{庾太尉目庾中郎:「家從談談之許。」}{\footnotesize \textbf{名士傳}曰敳不為辨析之談,而舉其旨要,太尉王夷甫雅重之也。一作家從談之祖。從,一作誦。許,一作辭。}

\subsection*{42}

\textbf{庾公目中郎:「神氣融散,差如得上。」}{\footnotesize \textbf{晉陽秋}曰敳頹然淵放,莫有動其聽者。}

\subsection*{43}

\textbf{劉琨稱祖車騎為朗詣,曰:「少為王敦所歎。」}{\footnotesize \textbf{虞預晉書}曰逖,字士稚,范陽遒人,豁蕩不修儀檢,輕財好施。\textbf{晉陽秋}曰逖與司空劉琨俱以雄豪著名,年二十四,與琨同辟司州主簿,情好綢繆,共被而寢,中夜聞雞鳴,俱起曰「此非惡聲也」,每語世事,或中宵起坐,相謂曰「若四海鼎沸,豪傑共起,吾與足下相避中原耳」,為汝南太守,值京師傾覆,率流民數百家南度,行達泗口,安東板為徐州刺史,逖既有豪才,常忼慨以中原為己任,乃說中宗雪復神州之計,拜為豫州刺史,使自招募,逖遂率部曲百餘家,北度江,誓曰「祖逖若不清中原而復濟此者,有如大江」,攻城略地,招懷義士,屢摧石虎,虎不敢復闚河南,石勒為逖母墓置守吏,劉琨與親舊書曰「吾枕戈待旦,志梟逆虜,常恐祖生先吾著鞭耳」,會其病卒,先有妖星見豫州分,逖曰「此必為我也,天未欲滅寇故耳」,贈車騎將軍。}

\subsection*{44}

\textbf{時人目庾中郎:「善於託大,長於自藏。」}{\footnotesize \textbf{名士傳}曰敳雖居職任,未嘗以事自嬰,從容博暢,寄通而已,是時天下多故,機事屢起,有為者拔奇吐異,而禍福繼之,敳常默然,故憂喜不至也。}

\subsection*{45}

\textbf{王平子邁世有儁才,少所推服,每聞衛玠言,輒歎息絕倒。}{\footnotesize \textbf{玠別傳}曰玠少有名理,善通莊老,琅邪王平子高氣不群,邁世獨傲,每聞玠之語議,至於理會之間、要妙之際,輒絕倒於坐,前後三聞,為之三倒,時人遂曰「衛君談道,平子三倒」。}

\subsection*{46}

\textbf{王大將軍與元皇表云:「舒風概簡正,允作雅人,自多於邃,}{\footnotesize 王舒已見。\textbf{王邃別傳}曰邃,字處重,琅邪人,舒弟也,意局剛清,以政事稱,累遷中領軍、尚書左僕射。舒、邃並敦從弟。}\textbf{最是臣少所知拔,中間夷甫、澄見語『卿知處明、茂弘,茂弘已有令名,真副卿清論,處明親疏無知之者,吾常以卿言為意,殊未有得,恐已悔之』,臣慨然曰『君以此試』,頃來始乃有稱之者,言常人正自患知之使過,不知使負實。」}{\footnotesize 使,一作便。}

\subsection*{47}

\textbf{周侯於荊州敗績還,未得用,王丞相與人書曰:「雅流弘器,何可得遺?」}{\footnotesize \textbf{鄧粲晉紀}曰顗為荊州,始至,而建平民傅密等叛迎蜀賊,顗狼狽失據,陶侃救之,得免,顗至武昌投王敦,敦更選侃代顗,顗還建康,未即得用也。}

\subsection*{48}

\textbf{時人欲題目高坐而未能,桓廷尉以問周侯,周侯曰:「可謂卓朗。」桓公曰:「精神淵著。」}{\footnotesize \textbf{高坐傳}曰庾亮、周顗、桓彝一代名士,一見和尚,披衿致契,曾為和尚作目,久之未得,有云「尸利密可稱卓朗」,於是桓始咨嗟,以為標之極似,宣武嘗云「少見和尚,稱其精神淵著,當年出倫」,其為名士所歎如此。}

\subsection*{49}

\textbf{王大將軍稱其兒云:「其神候似欲可。」}{\footnotesize 王應也。}

\subsection*{50}

\textbf{卞令目叔向:「朗朗如百間屋。」}{\footnotesize \textbf{春秋左氏傳}曰叔向,羊舌肸也,晉大夫。}

\subsection*{51}

\textbf{王敦為大將軍,鎮豫章,衛玠避亂,從洛投敦,相見欣然,談話彌日,于時謝鯤為長史,敦謂鯤曰:「不意永嘉之中,復聞正始之音,阿平若在,當復絕倒。」}{\footnotesize \textbf{玠別傳}曰玠至武昌見王敦,敦與之談論,彌日信宿,敦顧謂僚屬曰「昔王輔嗣吐金聲於中朝,此子今復玉振於江表,微言之緒,絕而復續,不悟永嘉之中,復聞正始之音,阿平若在,當復絕倒矣」。}

\subsection*{52}

\textbf{王平子與人書,稱其兒「風氣日上,足散人懷」。}{\footnotesize \textbf{永嘉流人名}曰澄第四子徽。\textbf{澄別傳}曰徽邁上有父風。}

\subsection*{53}

\textbf{胡毋彥國吐佳言如屑,後進領袖。}{\footnotesize 言談之流,靡靡如解木出屑也。}

\subsection*{54}

\textbf{王丞相云:「刁玄亮之察察,戴若思之巖巖,}{\footnotesize \textbf{虞預書}曰戴儼,字若思,廣陵人,才義辯濟,有風標鋒穎,累遷征西將軍,為王敦所害,贈左光祿大夫,儀同三司。}\textbf{卞望之之峰距。」}{\footnotesize \textbf{卞壼別傳}曰壼,字望之,濟陰冤句人,父粹,太常卿,壼少以貴正見稱,累遷御史中丞,權門屏迹,轉領軍、尚書令,蘇峻作亂,率眾拒戰,父子二人俱死王難。\textbf{鄧粲晉紀}曰初,咸和中,貴遊子弟能談嘲者,慕王平子、謝幼輿等為達,壼厲色於朝曰「悖禮傷教,罪莫斯甚,中朝傾覆,實由於此」,欲奏治之,王導、庾亮不從,乃止,其後皆折節為名士。\textbf{語林}曰孔坦為侍中,密啓成帝,不宜拜曹夫人,丞相聞之曰「王茂弘駑痾耳,若卞望之之巖巖,刁玄亮之察察,戴若思之峰距,當敢爾不」。此言殊有由緒,故聊載之耳。}

\subsection*{55}

\textbf{大將軍語右軍:「汝是我佳子弟,}{\footnotesize \textbf{按}王氏譜,羲之是敦從父兄子。}\textbf{當不減阮主簿。」}{\footnotesize \textbf{中興書}曰阮裕少有德行,王敦聞其名,召為主簿,知敦有不臣之心,縱酒昏酣,不綜其事。}

\subsection*{56}

\textbf{世目周侯:「嶷如斷山。」}{\footnotesize \textbf{晉陽秋}曰顗正情嶷然,雖一時儕類,皆無敢媟近。}

\subsection*{57}

\textbf{王丞相招祖約夜語,至曉不眠,明旦有客,公頭鬢未理,亦小倦,客曰:「公昨如是,似失眠。」公曰:「昨與士少語,遂使人忘疲。」}

\subsection*{58}

\textbf{王大將軍與丞相書,稱楊朗曰:「世彥識器理致,才隱明斷,既為國器,且是楊侯準之子,}{\footnotesize \textbf{世語}曰準,字始立,弘農華陰人,曾祖彪、祖脩有名前世,父囂,典軍校尉,準元康末為冀州刺史。\textbf{荀綽冀州記}曰準見王綱不振,遂縱酒不以官事規意,消搖卒歲而已,成都王知準不治,猶以其名士,惜而不遣,召為軍咨議祭酒,府散停家,關東諸侯欲以準補三事,以示懷賢尚德之事,未施行而卒,時年二十有七。}\textbf{位望殊為陵遲,卿亦足與之處。」}

\subsection*{59}

\textbf{何次道往丞相許,丞相以麈尾指坐,呼何共坐曰:「來,來!此是君坐。」}{\footnotesize 何充已見。}

\subsection*{60}

\textbf{丞相治楊州廨舍,按行而言曰:「我正為次道治此爾。」何少為王公所重,故屢發此歎。}{\footnotesize \textbf{晉陽秋}曰充,導妻姊之子,明穆皇后之妹夫也,思韻淹濟,有文義才情,導深器之,由是少有美譽,遂歷顯位,導有副貳己使繼相意,故屢顯此指於上下。}

\subsection*{61}

\textbf{王丞相拜司徒而歎曰:「劉王喬若過江,我不獨拜公。」}{\footnotesize \textbf{曹嘉之晉紀}曰疇有重名,永嘉中為閻鼎所害,司徒蔡謨每歎曰「若使劉王喬得南渡,司徒公之美選也」。}

\subsection*{62}

\textbf{王藍田為人晚成,時人乃謂之癡,}{\footnotesize \textbf{晉陽秋}曰述體道清粹,簡貴靜正,怡然自足,不交非類,雖群英紛紛,俊乂交馳,述獨蔑然,曾不慕羨,由是名譽久蘊。}\textbf{王丞相以其東海子,辟為掾,常集聚,王公每發言,眾人競贊之,述於末坐曰:「主非堯舜,何得事事皆是?」丞相甚相歎賞。}{\footnotesize 言非聖人,不能無過,意譏讚述之徒。}

\subsection*{63}

\textbf{世目楊朗「沈審經斷」,蔡司徒云:「若使中朝不亂,楊氏作公方未已。」謝公云:「朗是大才。」}{\footnotesize \textbf{八王故事}曰楊準有六子,曰喬、髦、朗、琳、俊、伸,皆得美名,論者以謂悉有台輔之望,文康庾公每追歎曰「中朝不亂,諸楊作公未已也」。}

\subsection*{64}

\textbf{劉萬安即道真從子,庾公}{\footnotesize 琮,字子躬。}\textbf{所謂「灼然玉舉」,又云:「千人亦見,百人亦見」。}{\footnotesize \textbf{劉氏譜}曰綏,字萬安,高平人,祖奧,太祝令,父斌,著作郎,綏歷驃騎長史。}

\subsection*{65}

\textbf{庾公為護軍,屬桓廷尉覓一佳吏,乃經年,桓後遇見徐寧而知之,遂致於庾公曰:「人所應有,其不必有,人所應無,己不必無,真海岱清士。」}{\footnotesize \textbf{徐江州本事}曰徐寧,字安期,東海郯人,通朗有德素,少知名,初為輿縣令,譙國桓彝有人倫鑒識,嘗去職無事,至廣陵尋親舊,遇風,停浦中累日,在船憂邑,上岸消搖,見一空宇,有似廨舍,彝訪之,云「輿縣廨也,令姓徐名寧」,彝既獨行,思逢悟賞,聊造之,寧清惠博涉,相遇怡然,遂停宿,因留數夕,與寧結交而別,至都,謂庾亮曰「吾為卿得一佳吏部郎」,亮問所在,彝即敘之,累遷吏部郎、左將軍、江州刺史。}

\subsection*{66}

\textbf{桓茂倫云「褚季野皮裏陽秋」,謂其裁中也。}{\footnotesize \textbf{晉陽秋}曰裒簡穆有器識,故為彝所目也。}

\subsection*{67}

\textbf{何次道嘗送東人,瞻望見賈寧在後輪中,曰:「此人不死,終為諸侯上客。」}{\footnotesize \textbf{晉陽秋}曰寧,字建寧,長樂人,賈氏孽子也,初自結於王應、諸葛瑤,應敗,浮遊吳會,吳人咸侮辱之,聞京師亂,馳出投蘇峻,峻甚暱之,以為謀主,及峻聞義軍起,自姑孰屯於石頭,是寧之計,峻敗,先降,仕至新安太守。}

\subsection*{68}

\textbf{杜弘治墓崩,哀容不稱,庾公顧謂諸客曰:「弘治至羸,不可以致哀。」}{\footnotesize \textbf{晉陽秋}曰杜乂,字弘治,京兆人,祖預、父錫有譽前朝,乂少有令名,仕丹陽丞,蚤卒,成帝納乂女為后。}\textbf{又曰:「弘治哭不可哀。」}

\subsection*{69}

\textbf{世稱「庾文康為豐年玉,穉恭為荒年穀」,庾家論云是文康稱「恭為荒年穀,庾長仁為豐年玉」。}{\footnotesize 謂亮有廊廟之器,翼有匡世之才,各有用也。}

\subsection*{70}

\textbf{世目杜弘治標鮮,季野穆少。}{\footnotesize \textbf{江左名士傳}曰乂,清標令上也。}

\subsection*{71}

\textbf{有人目杜弘治:「標鮮清令,盛德之風,可樂詠也。」}{\footnotesize \textbf{語林}曰有人目杜弘治,標鮮甚清令,初若熙怡,容無韻,盛德之風,可樂詠也。}

\subsection*{72}

\textbf{庾公云「逸少國舉」,故庾倪為碑文云:「拔萃國舉。」}{\footnotesize 倪,庾倩小字也。\textbf{徐廣晉紀}曰倩,字少彥,司空冰子、皇后兄也,有才具,仕至太宰長史,桓溫以其宗彊,使下邳王晃誣與謀反而誅之。}

\subsection*{73}

\textbf{庾穉恭與桓溫書,稱「劉道生日夕在事,大小殊快,義懷通樂既佳,且足作友,正實良器,推此與君,同濟艱不者也」。}{\footnotesize \textbf{宋明帝文章志}曰劉恢,字道生,沛國人,識局明濟,有文武才,王濛每稱其思理淹通,蕃屏之高選,為車騎司馬,年三十六卒,贈前將軍。}

\subsection*{74}

\textbf{王藍田拜揚州,主簿請諱,教云:「亡祖、先君名播海內,遠近所知,內諱不出於外,}{\footnotesize \textbf{禮記}曰婦人之諱不出門。}\textbf{餘無所諱。」}

\subsection*{75}

\textbf{蕭中郎,孫丞公婦父,劉尹在撫軍坐,時擬為太常,劉尹云:「蕭祖周不知便可作三公不?自此以還,無所不堪。」}{\footnotesize \textbf{晉百官名}曰蕭輪,字祖周,樂安人。\textbf{劉謙之晉紀}曰輪有才學,善三禮,歷常侍、國子博士。}

\subsection*{76}

\textbf{謝太傅未冠,始出西,詣王長史,清言良久,去後,苟子問曰:}{\footnotesize 王濛、子脩並已見。}\textbf{「向客何如尊?」長史曰:「向客亹亹,為來逼人。」}

\subsection*{77}

\textbf{王右軍語劉尹:「故當共推安石。」劉尹曰:「若安石東山志立,當與天下共推之。」}{\footnotesize \textbf{續晉陽秋}曰初,安家於會稽上虞縣,優遊山林,六七年間,徵召不至,雖彈奏相屬,繼以禁錮,而晏然不屑也。}

\subsection*{78}

\textbf{謝公稱藍田:「掇皮皆真。」}{\footnotesize \textbf{徐廣晉紀}曰述貞審,真意不顯。}

\subsection*{79}

\textbf{桓溫行經王敦墓邊過,望之云:「可兒,可兒!」}{\footnotesize \textbf{孫綽與庾亮牋}曰王敦可人之目,數十年間也。}

\subsection*{80}

\textbf{殷中軍道王右軍云:「逸少清貴人,吾於之甚至,一時無所後。」}{\footnotesize \textbf{文章志}曰羲之高爽有風氣,不類常流也。}

\subsection*{81}

\textbf{王仲祖稱殷淵源:「非以長勝人,處長亦勝人。」}{\footnotesize \textbf{晉陽秋}曰浩善以通和接物也。}

\subsection*{82}

\textbf{王司州與殷中軍語,歎云:「己之府奧,蚤已傾寫而見,殷陳勢浩汗,眾源未可得測。」}{\footnotesize \textbf{徐廣晉紀}曰浩清言妙辯玄致,當時名流皆為其美譽。}

\subsection*{83}

\textbf{王長史謂林公:「真長可謂金玉滿堂。」林公曰:「金玉滿堂,復何為簡選?」王曰:「非為簡選,直致言處自寡耳。」}{\footnotesize 謂吉人之辭寡,非擇言而出也。}

\subsection*{84}

\textbf{王長史道江道群:「人可應有,乃不必有,人可應無,己必無。」}{\footnotesize \textbf{中興書}曰江灌,字道群,陳留人,僕射虨從弟也,有才器,與從兄逌名相亞,仕尚書、中護軍。}

\subsection*{85}

\textbf{會稽孔沈、魏顗、虞球、虞存、謝奉,並是四族之儁,于時之桀,}{\footnotesize 沈、存、顗、奉並別見。\textbf{虞氏譜}曰球,字和琳,會稽餘姚人,祖授,吳廣州刺史,父基,右軍司馬,球仕至黃門侍郎。}\textbf{孫興公目之曰:「沈為孔家金,顗為魏家玉,虞為長、琳宗,謝為弘道伏。」}{\footnotesize 長、琳,即存及球字也。弘道,謝奉字也。言虞氏宗長、琳之才,謝氏伏弘道之美也。}

\subsection*{86}

\textbf{王仲祖、劉真長造殷中軍談,談竟,俱載去,劉謂王曰:「淵源真可。」王曰:「卿故墮其雲霧中。」}{\footnotesize \textbf{中興書}曰浩能言理,談論精微,長於老易,故風流者皆宗歸之。}

\subsection*{87}

\textbf{劉尹每稱王長史云:「性至通,而自然有節。」}{\footnotesize \textbf{濛別傳}曰濛之交物,虛己納善,恕而後行,希見其喜慍之色,凡與一面,莫不敬而愛之,然少孤,事諸母甚謹,篤義穆親,不修小潔,以清貧見稱。}

\subsection*{88}

\textbf{王右軍道謝萬石「在林澤中,為自遒上」,歎林公「器朗神儁」,}{\footnotesize \textbf{支遁別傳}曰遁任心獨往,風期高亮。}\textbf{道祖士少「風領毛骨,恐沒世不復見如此人」,道劉真長「標雲柯而不扶疎」。}{\footnotesize \textbf{劉尹別傳}曰惔既令望,姻婭帝室,故屢居達官,然性不偶俗,心淡榮利,雖身登顯列而每挹降,閑靜自守而已。}

\subsection*{89}

\textbf{簡文目庾赤玉「省率治除」,謝仁祖云:「庾赤玉胷中無宿物。」}{\footnotesize 赤玉,庾統小字。\textbf{中興書}曰統,字長仁,潁川人,衛將軍擇子也,少有令名,仕至尋陽太守。}

\subsection*{90}

\textbf{殷中軍道韓太常曰:「康伯少自標置,居然是出群器,及其發言遣辭,往往有情致。」}{\footnotesize \textbf{續晉陽秋}曰康伯清和有思理,幼為舅殷浩所稱。}

\subsection*{91}

\textbf{簡文道王懷祖:「才既不長,於榮利又不淡,直以真率少許,便足對人多多許。」}{\footnotesize \textbf{晉陽秋}曰述少貧約,簞瓢陋巷,不求聞達,由是為有識所重。}

\subsection*{92}

\textbf{林公謂王右軍:「長史作數百語,無非德音,如恨不苦。」}{\footnotesize 苦謂窮人以辭。}\textbf{王曰:「長史自不欲苦物。」}

\subsection*{93}

\textbf{殷中軍與人書,道謝萬「文理轉遒,成殊不易」。}{\footnotesize \textbf{中興書}曰萬才器儁秀,善自衒曜,故致有時譽,兼善屬文,能談論,時人稱之。}

\subsection*{94}

\textbf{王長史云:「江思悛思懷所通,不翅儒域。」}{\footnotesize \textbf{徐廣晉紀}曰江惇,字思悛,陳留人,僕射虨弟也,性篤學,手不釋書,博覽墳典,儒道兼綜,徵聘無所就,年四十九而卒。}

\subsection*{95}

\textbf{許玄度送母,始出都,人問劉尹:「玄度定稱所聞不?」劉曰:「才情過於所聞。」}{\footnotesize \textbf{許氏譜}曰玄度母,華軼女也。\textbf{按}詢集,詢出都迎姊,於路賦詩,續晉陽秋亦然,而此言送母,疑繆矣。}

\subsection*{96}

\textbf{阮光祿云:「王家有三年少,右軍、安期、長豫。」}{\footnotesize 阮裕、王悅、安期王應並已見。}

\subsection*{97}

\textbf{謝公道豫章:「若遇七賢,必自把臂入林。」}{\footnotesize \textbf{江左名士傳}曰鯤通簡有識,不修威儀,好迹逸而心整,形濁而言清,居身若穢,動不累高,鄰家有女,嘗往挑之,女方織,以梭投折其兩齒,既歸,傲然長嘯曰「猶不廢我嘯歌」,其不事形骸如此。}

\subsection*{98}

\textbf{王長史歎林公:「尋微之功,不減輔嗣。」}{\footnotesize \textbf{支遁別傳}曰遁神心警悟,清識玄遠,嘗至京師,王仲祖稱其造微之功,不異王弼。}

\subsection*{99}

\textbf{殷淵源在墓所幾十年,于時朝野以擬管、葛,起不起,以卜江左興亡。}{\footnotesize \textbf{續晉陽秋}曰時穆帝幼沖,母后臨朝,簡文親賢民望,任登宰輔,桓溫有平蜀洛之勳,擅彊西陝,帝自料文弱,無以抗之,陳郡殷浩,素有盛名,時論比之管、葛,故徵浩為揚州,溫知意在抗己,甚忿焉。}

\subsection*{100}

\textbf{殷中軍道右軍:「清鑒貴要。」}{\footnotesize \textbf{晉安帝紀}曰羲之風骨清舉也。}

\subsection*{101}

\textbf{謝太傅為桓公司馬,}{\footnotesize \textbf{續晉陽秋}曰初,安優游山水,以敷文析理自娛,桓溫在西蕃,欽其盛名,諷朝廷請為司馬,以世道未夷,志存匡濟,年四十,起家應務也。}\textbf{桓詣謝,值謝梳頭,遽取衣幘,桓公云:「何煩此。」因下共語至暝,既去,謂左右曰:「頗曾見如此人不?」}

\subsection*{102}

\textbf{謝公作宣武司馬,屬門生數十人於田曹中郎趙悅子,}{\footnotesize \textbf{伏滔大司馬寮屬名}曰悅,字悅子,下邳人,歷大司馬參軍、左衛將軍。}\textbf{悅子以告宣武,宣武云:「且為用半。」趙俄而悉用之,曰:「昔安石在東山,縉紳敦逼,恐不豫人事,況今自鄉選,反違之邪?」}

\subsection*{103}

\textbf{桓宣武表云:「謝尚神懷挺率,少致民譽。」}{\footnotesize 溫集載其\textbf{平洛表}曰今中州既平,宜時綏定,鎮西將軍豫州刺史尚,神懷挺率,少致人譽,是以入贊百揆,出蕃方司,宜進據洛陽,撫寧黎庶,謂可本官都督司州諸軍事。}

\subsection*{104}

\textbf{世目謝尚為令達,阮遙集云:「清暢似達。」或云:「尚自然令上。」}{\footnotesize \textbf{晉陽秋}曰尚率易挺達,超悟令上也。}

\subsection*{105}

\textbf{桓大司馬病,謝公往省病,從東門入,}{\footnotesize 溫時在姑孰。}\textbf{桓公遙望,歎曰:「吾門中久不見如此人。」}

\subsection*{106}

\textbf{簡文目敬豫為「朗豫」。}{\footnotesize 王恬已見。\textbf{文字志}曰恬識理明貴,為後進冠冕也。}

\subsection*{107}

\textbf{孫興公為庾公參軍,共游白石山,衛君長在坐,}{\footnotesize \textbf{衛氏譜}曰永,字君長,成陽人,位至左軍長史。}\textbf{孫曰:「此子神情都不關山水,而能作文。」庾公曰:「衛風韻雖不及卿諸人,傾倒處亦不近。」孫遂沐浴此言。}

\subsection*{108}

\textbf{王右軍目陳玄伯:「壘塊有正骨。」}{\footnotesize 陳泰已見。}

\subsection*{109}

\textbf{王長史云:「劉尹知我,勝我自知。」}{\footnotesize \textbf{濛別傳}曰濛與沛國劉惔齊名,時人以濛比袁曜卿,惔比荀奉倩,而共交友,甚相知賞也。}

\subsection*{110}

\textbf{王、劉聽林公講,王語劉曰:「向高坐者,故是凶物。」復更聽,王又曰:「自是鉢釪後王、何人也。」}{\footnotesize \textbf{高逸沙門傳}王濛恆尋遁,遇祗洹寺中講,正在高坐上,每舉麈尾,常領數百言,而情理俱暢,預坐百餘人,皆結舌注耳,濛云聽講眾僧「向高坐者,是鉢釪後王、何人也」。}

\subsection*{111}

\textbf{許玄度言:「琴賦所謂『非至精者,不能與之析理』,劉尹其人,『非淵靜者,不能與之閑止』,簡文其人。」}{\footnotesize 嵇叔夜琴賦也。劉惔真長,丹陽尹。}

\subsection*{112}

\textbf{魏隱兄弟少有學義,}{\footnotesize \textbf{魏氏譜}曰隱,字安時,會稽上虞人,歷義興太守、御史中丞,弟逷,黃門郎。}\textbf{總角詣謝奉,奉與語,大說之,曰:「大宗雖衰,魏氏已復有人。」}

\subsection*{113}

\textbf{簡文云:「淵源語不超詣簡至,然經綸思尋處,故有局陳。」}

\subsection*{114}

\textbf{初,法汰北來未知名,}{\footnotesize \textbf{車頻秦書}曰釋道安為慕容俊所掠,欲投襄陽,行至新野,集眾議曰「今遭凶年,不依國主,則法事難舉」,乃分僧眾,使竺法汰詣揚州,曰「彼多君子,上勝可投」,法汰遂渡江至揚土焉。}\textbf{王領軍供養之,}{\footnotesize \textbf{中興書}曰王洽,字敬和,丞相導第三子,累遷吳郡內史,為士民所懷,徵拜中領軍,尋加中書令,不拜,年二十六而卒。}\textbf{每與周旋,行來往名勝許,輒與俱,不得汰,便停車不行,因此名遂重。}{\footnotesize \textbf{名德沙門題目}曰法汰高亮開達。孫綽為\textbf{汰贊}曰淒風拂林,明泉映壑,爽爽法汰,校德無怍,事外瀟灑,神內恢廓,實從前起,名隨後躍。\textbf{泰元起居注}曰法汰以十五年卒,烈宗詔曰「法汰師喪逝,哀痛傷懷,可贈錢十萬」。}

\subsection*{115}

\textbf{王長史與大司馬書,道淵源「識致安處,足副時談」。}

\subsection*{116}

\textbf{謝公云:「劉尹語審細。」}{\footnotesize 孫綽為\textbf{惔誄敘}曰神猶淵鏡,言必珠玉。}

\subsection*{117}

\textbf{桓公語嘉賓:「阿源有德有言,向使作令僕,足以儀刑百揆,朝廷用違其才耳。」}{\footnotesize 嘉賓,郗超小字也。阿源,殷浩也。}

\subsection*{118}

\textbf{簡文語嘉賓:「劉尹語末後亦小異,回復其言,亦乃無過。」}

\subsection*{119}

\textbf{孫興公、許玄度共在白樓亭,}{\footnotesize \textbf{會稽記}曰亭在山陰,臨流映壑也。}\textbf{共商略先往名達,林公既非所關,聽訖云:「二賢故自有才情。」}

\subsection*{120}

\textbf{王右軍道東陽「我家阿林,章清太出」。}{\footnotesize 林,應為臨。\textbf{王氏譜}曰臨之,字仲產,琅邪人,僕射彪之子,仕至東陽太守。}

\subsection*{121}

\textbf{王長史與劉尹書,道淵源「觸事長易」。}

\subsection*{122}

\textbf{謝中郎云:「王修載樂託之性,出自門風。」}{\footnotesize \textbf{王氏譜}曰耆之,字修載,琅邪人,荊州刺史廙第三子,歷中書郎、鄱陽太守、給事中。}

\subsection*{123}

\textbf{林公云:「王敬仁是超悟人。」}{\footnotesize \textbf{文字志}曰脩之少有秀令之稱。}

\subsection*{124}

\textbf{劉尹先推謝鎮西,謝後雅重劉,曰:「昔嘗北面。」}{\footnotesize \textbf{按}謝尚年長於惔,神穎夙彰,而曰北面於劉,非可信。}

\subsection*{125}

\textbf{謝太傅稱王修齡曰:「司州可與林澤遊。」}{\footnotesize \textbf{王胡之別傳}曰胡之常遺世務,以高尚為情,與謝安相善也。}

\subsection*{126}

\textbf{諺曰:「楊州獨步王文度,後來出人郗嘉賓。」}{\footnotesize \textbf{續晉陽秋}曰超少有才氣,越世負俗,不循常檢,時人為一代盛譽者,語曰「大才槃槃謝家安,江東獨步王文度,盛德日新郗嘉賓」。其語小異,故詳錄焉。}

\subsection*{127}

\textbf{人問王長史江虨兄弟群從,王答曰:「諸江皆復足自生活。」}{\footnotesize 虨及弟淳、從灌並有德行,知名於世。}

\subsection*{128}

\textbf{謝太傅道安北:「見之乃不使人厭,然出戶去,不復使人思。」}{\footnotesize 安北,王坦之也。\textbf{續晉陽秋}曰謝安初攜幼穉同好,養志海濱,襟情超暢,尤好聲律,然抑之以禮,在哀能至,弟萬之喪,不聽絲竹者將十年,及輔政,而修室第園館,麗車服,雖朞功之慘,不廢妓樂,王坦之因苦諫焉。\textbf{按}謝公蓋以王坦之好直言,故不思爾。}

\subsection*{129}

\textbf{謝公云:「司州造勝遍決。」}{\footnotesize \textbf{宋明帝文章志}曰胡之性簡,好達玄言也。}

\subsection*{130}

\textbf{劉尹云:「見何次道飲酒,使人欲傾家釀。」}{\footnotesize 充飲酒能溫克。}

\subsection*{131}

\textbf{謝太傅語真長:「阿齡於此事,故欲太厲。」}{\footnotesize 修齡,王胡之小字也。}\textbf{劉曰:「亦名士之高操者。」}{\footnotesize \textbf{胡之別傳}曰胡之治身清約,以風操自居。}

\subsection*{132}

\textbf{王子猷說:「世目士少為朗,我家亦以為徹朗。」}{\footnotesize \textbf{晉諸公贊}曰祖約少有清稱。}

\subsection*{133}

\textbf{謝公云:「長史語甚不多,可謂有令音。」}{\footnotesize \textbf{王濛別傳}曰濛性和暢,能清言,談道貴理中,簡而有會,商略古賢,顯默之際,辭旨劭令,往往有高致。}

\subsection*{134}

\textbf{謝鎮西道敬仁:「文學鏃鏃,無能不新。」}{\footnotesize \textbf{語林}曰敬仁有異才,時賢皆重之,王右軍在郡迎敬仁,叔仁輒同車,常惡其遲,後以馬迎敬仁,雖復風雨,亦不以車也。}

\subsection*{135}

\textbf{劉尹道江道群:「不能言而能不言。」}{\footnotesize 江灌已見。}

\subsection*{136}

\textbf{林公云:「見司州警悟交至,使人不得住,亦終日忘疲。」}{\footnotesize \textbf{王胡之別傳}曰胡之少有風尚,才器率舉,有秀悟之稱。}

\subsection*{137}

\textbf{世稱「苟子秀出,阿興清和」。}{\footnotesize 苟子已見。阿興,王蘊小字。}

\subsection*{138}

\textbf{簡文云:「劉尹茗柯有實理。」}{\footnotesize 柯,一作朾,又作仃,又作打。}

\subsection*{139}

\textbf{謝胡兒作著作郎,嘗作王堪傳,}{\footnotesize \textbf{晉諸公贊}曰堪,字世冑,東平壽張人,少以高亮義正稱,為尚書左丞,有準繩操,為石勒所害,贈太尉。}\textbf{不諳堪是何似人,咨謝公,謝公答曰:「世冑亦被遇,堪,烈之子,}{\footnotesize \textbf{晉諸公贊}曰烈,字陽秀,蚤知名魏朝,為治書御史。}\textbf{阮千里姨兄弟,潘安仁中外,安仁詩所謂『子親伊姑,我父惟舅』,是許允壻。」}{\footnotesize \textbf{岳集}曰堪為成都王軍司馬,岳送至北邙別,作詩曰「微微髮膚,受之父母,峩峩王侯,中外之首,子親伊姑,我父惟舅」。}

\subsection*{140}

\textbf{謝太傅重鄧僕射,常言:「天地無知,使伯道無兒。」}{\footnotesize \textbf{晉陽秋}曰鄧攸既棄子,遂無復繼嗣,為有識傷惜。}

\subsection*{141}

\textbf{謝公與王右軍書曰:「敬和棲託好佳。」}{\footnotesize \textbf{中興書}曰洽於公子中最知名,與潁川荀羡俱有美稱。}

\subsection*{142}

\textbf{吳四姓舊目云:「張文、朱武、陸忠、顧厚。」}{\footnotesize \textbf{吳錄士林}曰吳郡有顧、陸、朱、張,為四姓,三國之間,四姓盛焉。}

\subsection*{143}

\textbf{謝公語王孝伯:「君家藍田,舉體無常人事。」}{\footnotesize \textbf{按}述雖簡,而性不寬裕,投火怒蠅,方之未甚,若非太傅虛相褒飾,則世說謬設斯語也。}

\subsection*{144}

\textbf{許掾嘗詣簡文,爾夜風恬月朗,乃共作曲室中語,襟懷之詠,偏是許之所長,辭寄清婉,有逾平日,簡文雖契素,此遇尤相咨嗟,不覺造厀,共叉手語,達於將旦,既而曰:「玄度才情,故未易多有許。」}{\footnotesize \textbf{續晉陽秋}曰詢能言理,曾出都迎姊,簡文皇帝、劉真長說其情旨及襟懷之詠,每造厀賞對,夜以繫日。}

\subsection*{145}

\textbf{殷允出西,郗超與袁虎書云:「子思求良朋,託好足下,勿以開美求之。」}{\footnotesize \textbf{中興書}曰允,字子思,陳郡人,太常康第六子,恭素謙退,有儒者之風,歷吏部尚書。}\textbf{世目袁為開美,故子敬詩曰「袁生開美度」。}

\subsection*{146}

\textbf{謝車騎問謝公:「真長性至峭,何足乃重?」答曰:「是不見耳,阿見子敬,尚使人不能已。」}{\footnotesize \textbf{語林}曰羊驎因酒醉,撫謝左軍謂太傅曰「此家詎復後鎮西」,太傅曰「汝阿見子敬,便沐浴為論兄輩」。推此言意,則安以玄不見真長,故不重耳,見子敬尚重之,況真長乎?}

\subsection*{147}

\textbf{謝公領中書監,王東亭有事應同上省,王後至,坐促,王、謝雖不通,太傅猶斂厀容之,}{\footnotesize 王、謝不通事別見。}\textbf{王神意閑暢,謝公傾目,還謂劉夫人曰:「向見阿瓜,故自未易有,}{\footnotesize \textbf{按}王珣小字法護,而此言阿瓜,未為可解,儻小名有兩耳。}\textbf{雖不相關,正是使人不能已已。」}

\subsection*{148}

\textbf{王子敬語謝公:「公故蕭灑。」謝曰:「身不蕭灑,君道身最得,身正自調暢。」}{\footnotesize \textbf{續晉陽秋}曰安弘雅有器,風神調暢也。}

\subsection*{149}

\textbf{謝車騎初見王文度曰:「見文度雖蕭灑相遇,其復愔愔竟夕。」}

\subsection*{150}

\textbf{范豫章謂王荊州:}{\footnotesize 范甯、王忱並已見。}\textbf{「卿風流儁望,真後來之秀。」王曰:「不有此舅,焉有此甥?」}

\subsection*{151}

\textbf{子敬與子猷書,道:「兄伯蕭索寡會,遇酒則酣暢忘反,乃自可矜。」}

\subsection*{152}

\textbf{張天錫世雄涼州,以力弱詣京師,雖遠方殊類,亦邊人之桀也,}{\footnotesize 天錫已見。}\textbf{聞皇京多才,欽羡彌至,猶在渚住,司馬著作往詣之,}{\footnotesize 未詳。}\textbf{言容鄙陋,無可觀聽,天錫心甚悔來,以遐外可以自固,王彌有儁才,美譽當時,聞而造焉,}{\footnotesize \textbf{續晉陽秋}曰珉風情秀發,才辭富贍。}\textbf{既至,天錫見其風神清令,言話如流,陳說古今,無不貫悉,又諳人物氏族中來,皆有證據,天錫訝服。}

\subsection*{153}

\textbf{王恭始與王建武甚有情,後遇袁悅之間,遂致疑隟,}{\footnotesize \textbf{晉安帝紀}曰初,忱與族子恭少相善,齊聲見稱,及並登朝,俱為主相所待,內外始有不咸之論,恭獨深憂之,乃告忱曰「悠悠之論,頗有異同,當由驃騎簡於朝覲故也,將無從容切言之邪?若主相諧睦,吾徒得戮力明時,復何憂哉」,忱以為然,而慮弗見用,乃令袁悅具言之,悅每欲間恭,乃於王坐責讓恭曰「卿何妄生同異,疑誤朝野」,其言切厲,恭雖惋悵,謂忱為搆己也,忱雖心不負恭,而無以自亮,於是情好大離,而怨隟成矣。}\textbf{然每至興會,故有相思時,恭嘗行散至京口射堂,于時清露晨流,新桐初引,恭目之曰:「王大故自濯濯。」}

\subsection*{154}

\textbf{司馬太傅為二王目曰:「孝伯亭亭直上,阿大羅羅清疎。」}{\footnotesize 恭,正亮亢烈,忱,通朗誕放。}

\subsection*{155}

\textbf{王恭有清辭簡旨,能敘說,而讀書少,頗有重出,}{\footnotesize \textbf{中興書}曰恭雖才不多,而清辯過人。}\textbf{有人道孝伯「常有新意,不覺為煩」。}

\subsection*{156}

\textbf{殷仲堪喪後,桓玄問仲文:「卿家仲堪,定是何似人?」仲文曰:「雖不能休明一世,足以映徹九泉。」}{\footnotesize \textbf{續晉陽秋}曰仲堪,仲文之從兄也,少有美譽。}