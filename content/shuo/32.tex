\chapter{讒險第三十二}

\subsection*{1}

\textbf{王平子形甚散朗,內實勁俠。}{\footnotesize \textbf{鄧粲晉紀}云劉琨嘗謂澄曰「卿形雖散朗,而內勁狹,以此處世,難得其死」,澄默然無以答,後果為王敦所害,劉琨聞之曰「自取死耳」。}

\subsection*{2}

\textbf{袁悅有口才,能短長說,亦有精理,始作謝玄參軍,頗被禮遇,後丁艱,服除還都,唯齎戰國策而已,語人曰:「少年時讀論語、老子,又看莊、易,此皆是病痛事,當何所益邪?天下要物,正有戰國策。」既下,說司馬孝文王,大見親待,幾亂機軸,俄而見誅。}{\footnotesize \textbf{袁氏譜}曰悅,字元禮,陳郡陽夏人,父朗,給事中,仕至驃騎咨議,太元中,悅有寵於會稽王,每勸專覽朝權,王頗納其言,王恭聞其說,言於孝武,乃託以它罪,殺悅於市中,既而朋黨同異之聲,播於朝野矣。}

\subsection*{3}

\textbf{孝武甚親敬王國寶、王雅,}{\footnotesize \textbf{雅別傳}曰雅,字茂建,東海沂人,少知名。\textbf{晉安帝紀}曰雅之為侍中,孝武甚信而重之,王珣、王恭特以地望見禮,至於親幸,莫及雅者,上每置酒燕集,或召雅未至,上不先舉觴,時議謂珣、恭宜傅東宮,而雅以寵幸,超授太傅、尚書左僕射。}\textbf{雅薦王珣於帝,帝欲見之,嘗夜與國寶、雅相對,帝微有酒色,令喚珣,垂至,已聞卒傳聲,國寶自知才出珣下,恐傾奪其寵,因曰:「王珣當今名流,陛下不宜有酒色見之,自可別詔召也。」帝然其言,心以為忠,遂不見珣。}

\subsection*{4}

\textbf{王緒數讒殷荊州於王國寶,殷甚患之,求術於王東亭,曰:「卿但數詣王緒,往輒屏人,因論它事,如此則二王之好離矣。」殷從之,國寶見王緒問曰:「比與仲堪屏人何所道?」緒云:「故是常往來,無它所論。」國寶謂緒於己有隱,果情好日疎,讒言以息。}{\footnotesize \textbf{按}國寶得寵於會稽王,由緒獲進,同惡相求,有如市賈,終至誅夷,曾不攜貳,豈有仲堪微間而成離隟。}