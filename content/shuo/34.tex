\chapter{紕漏第三十四}

\subsection*{1}

\textbf{王敦初尚主,}{\footnotesize 敦尚武帝女舞陽公主,字脩褘。}\textbf{如廁,見漆箱盛乾棗,本以塞鼻,王謂廁上亦下果,食遂至盡,既還,婢擎金澡盤盛水,瑠璃盌盛澡豆,因倒著水中而飲之,謂是乾飯,群婢莫不掩口而笑之。}

\subsection*{2}

\textbf{元皇初見賀司空,言及吳時事,問:「孫皓燒鋸截一賀頭,是誰?」司空未得言,元皇自憶曰:「是賀劭。」}{\footnotesize 劭即循父也。皓凶暴驕矜,劭上書切諫,皓深恨之,親近憚劭貞正,譖云謗毀國事,被詰責,後還復職,劭中惡風,口不能言語,皓疑劭託疾,收付酒藏,考掠千數,卒無一言,鋸殺之。}\textbf{司空流涕曰:「臣父遭遇無道,創巨痛深,無以仰答明詔。」}{\footnotesize \textbf{禮記}創巨者其日久,痛深者其愈遲。}\textbf{元皇愧慙,三日不出。}

\subsection*{3}

\textbf{蔡司徒渡江,見彭蜞,大喜曰:「蟹有八足,加以二螯。」令烹之,既食,吐下委頓,方知非蟹,後向謝仁祖說此事,謝曰:「卿讀爾雅不熟,幾為勸學死。」}{\footnotesize \textbf{大戴禮勸學篇}曰蟹二螯八足,非蛇蟺之穴無所寄託者,用心躁也。故蔡邕為勸學章取義焉。\textbf{爾雅}曰螖蠌小者勞。即彭蜞也,似蟹而小,今彭蜞小於蟹,而大於彭螖,即爾雅所謂螖蠌也,然此三物皆八足二螯,而狀甚相類,蔡謨不精其小大,食而致弊,故謂讀爾雅不熟也。}

\subsection*{4}

\textbf{任育長年少時甚有令名,武帝崩,選百二十挽郎,一時之秀彥,育長亦在其中,王安豐選女壻,從挽郎搜其勝者,且擇取四人,任猶在其中,童少時神明可愛,時人謂育長影亦好,自過江,便失志,王丞相請先度時賢共至石頭迎之,猶作疇日相待,一見便覺有異,坐席竟,下飲,便問人云:「此為茶、為茗?」覺有異色,乃自申明云:「向問飲為熱、為冷耳。」嘗行從棺邸下度,流涕悲哀,王丞相聞之曰:「此是有情癡。」}{\footnotesize \textbf{晉百官名}曰任瞻,字育長,樂安人,父琨,少府卿,瞻歷謁者僕射、都尉、天門太守。}

\subsection*{5}

\textbf{謝虎子嘗上屋熏鼠,}{\footnotesize 虎子,據小字。據,字玄道,尚書裒第二子,年三十三亡。}\textbf{胡兒既無由知父為此事,聞人道「癡人有作此者」,戲笑之,時道此非復一過,太傅既了己之不知,因其言次,語胡兒曰:「世人以此謗中郎,亦言我共作此。」}{\footnotesize 中郎,據也,章仲反。\textbf{按}世有兄弟三人,則謂第二者為中,今謝昆弟有六,而以據為中郎,未可解,當由有三時,以中為稱,因仍不改也。}\textbf{胡兒懊熱,一月日閉齋不出,太傅虛託引己之過,以相開悟,可謂德教。}

\subsection*{6}

\textbf{殷仲堪父病虛悸,聞牀下蟻動,謂是牛鬬,}{\footnotesize \textbf{殷氏譜}曰殷師,字師子,祖識、父融並有名,師至驃騎咨議,生仲堪。\textbf{續晉陽秋}曰仲堪父曾有失心病,仲堪腰不解帶,彌年父卒。}\textbf{孝武不知是殷公,問仲堪:「有一殷,病如此不?」仲堪流涕而起曰:「臣進退唯谷。」}{\footnotesize 大雅詩也。\textbf{毛公}注曰谷,窮也。}

\subsection*{7}

\textbf{虞嘯父為孝武侍中,帝從容問曰:「卿在門下,初不聞有所獻替。」虞家富春,近海,謂帝望其意氣,對曰:「天時尚煗,䱥魚蝦鮓未可致,尋當有所上獻。」帝撫掌大笑。}{\footnotesize \textbf{中興書}曰嘯父,會稽人,光祿潭之孫,右將軍純之子,少歷顯位,與王廞同廢為庶人,義熙初,為會稽內史。}

\subsection*{8}

\textbf{王大喪後,朝論或云「國寶應作荊州」,}{\footnotesize \textbf{晉安帝紀}曰王忱死,會稽王欲以國寶代之,孝武中詔用仲堪,乃止。}\textbf{國寶主簿夜函白事,云「荊州事已行」,國寶大喜,其夜開閤,喚綱紀,話勢雖不及作荊州,而意色甚恬,曉遣參問,都無此事,即喚主簿數之曰:「卿何以誤人事邪?」}