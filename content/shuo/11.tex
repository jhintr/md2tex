\chapter{捷悟第十一}

\subsection*{1}

\textbf{楊德祖為魏武主簿,時作相國門,始搆榱桷,魏武自出看,使人題門作「活」字便去,楊見,即令壞之,既竟,曰:「門中活,闊字,王正嫌門大也。」}{\footnotesize \textbf{文士傳}曰楊脩,字德祖,弘農人,太尉彪子,少有才學思幹,魏武為丞相,辟為主簿,脩常白事,知必有反覆教,豫為答對數紙,以次牒之而行,敕守者曰「向白事,必教出相反覆,若按此次第連答之」,已而風吹紙次亂,守者不別,而遂錯誤,公怒推問,脩慚懼,然以所白甚有理,終亦是脩,後為武帝所誅。}

\subsection*{2}

\textbf{人餉魏武一桮酪,魏武噉少許,蓋頭上題「合」字以示眾,眾莫能解,次至楊脩,脩便噉,曰:「公教人噉一口也,復何疑?」}

\subsection*{3}

\textbf{魏武嘗過曹娥碑下,楊脩從,碑背上見題作「黃絹幼婦,外孫䪢臼」八字,魏武謂脩曰:「解不?」答曰:「解。」魏武曰:「卿未可言,待我思之。」行三十里,魏武乃曰:「吾已得。」令脩別記所知,脩曰:「黃絹,色絲也,於字為絕,幼婦,少女也,於字為妙,外孫,女子也,於字為好,䪢臼,受辛也,於字為辤,所謂絕妙好辤也。」魏武亦記之,與脩同,乃歎曰:「我才不及卿,乃覺三十里。」}{\footnotesize \textbf{會稽典錄}曰孝女曹娥者,上虞人,父盱,能撫節按歌,婆娑樂神,漢安二年,迎伍君神,泝濤而上,為水所淹,不得其尸,娥年十四,號慕思盱,乃投瓜于江,存其父尸曰「父在此,瓜當沈」,旬有七日,瓜偶沈,遂自投於江而死,縣長度尚悲憐其義,為之改葬,命其弟子邯鄲子禮為之作碑。\textbf{按}曹娥碑在會稽中,而魏武、楊脩未嘗過江也。\textbf{異苑}曰陳留蔡邕避難過吳,讀碑文,以為詩人之作,無詭妄也,因刻石旁作八字,魏武見而不能了,以問群寮,莫有解者,有婦人浣於汾渚,曰「第四車解」,既而,禰正平也,衡即以離合義解之,或謂此婦人即娥靈也。}

\subsection*{4}

\textbf{魏武征袁本初,治裝,餘有數十斛竹片,咸長數寸,眾並謂不堪用,正令燒除,太祖思所以用之,謂可為竹椑楯,而未顯其言,馳使問主簿楊德祖,應聲答之,與帝正同,眾伏其辯悟。}

\subsection*{5}

\textbf{王敦引軍垂至大桁,明帝自出中堂,溫嶠為丹陽尹,帝令斷大桁,故未斷,帝大怒,瞋目,左右莫不悚懼,}{\footnotesize \textbf{按}晉陽秋、鄧紀皆云「敦將至,嶠燒朱雀橋以阻其兵」,而云「未斷大桁,致帝怒」,大為譌謬,一本云「帝自勸嶠入」,一本作「噉飲帝怒」,此則近也。}\textbf{召諸公來,嶠至不謝,但求酒炙,王導須臾至,徒跣下地,謝曰:「天威在顏,遂使溫嶠不容得謝。」嶠於是下謝,帝乃釋然,諸公共歎王機悟名言。}

\subsection*{6}

\textbf{郗司空在北府,桓宣武惡其居兵權,}{\footnotesize \textbf{南徐州記}曰徐州人多勁悍,號精兵,故桓溫常曰「京口酒可飲,箕可用,兵可使」。}\textbf{郗於事機素暗,遣牋詣桓:「方欲共獎王室,脩復園陵。」世子嘉賓出行,於道上聞信至,急取牋,視竟,寸寸毀裂,便回,還更作牋,自陳老病,不堪人間,欲乞閑地自養,宣武得牋大喜,即詔轉公督五郡、會稽太守。}{\footnotesize \textbf{晉陽秋}曰大司馬將討慕容暐,表求申勸平北愔及袁真等嚴辦,愔以羸疾求退,詔大司馬領愔所任。\textbf{按}中興書,愔辭此行,溫責其不從,轉授會稽,世說為謬。}

\subsection*{7}

\textbf{王東亭作宣武主簿,嘗春月與石頭兄弟乘馬出郊,時彥同遊者連鑣俱進,}{\footnotesize 石頭,桓熙小字。\textbf{中興書}曰熙,字伯道,溫長子也,仕至豫州刺史。}\textbf{唯東亭一人常在前,覺數十步,諸人莫之解,石頭等既疲倦,俄而乘輿回,諸人皆似從官,唯東亭奕奕在前,其悟捷如此。}