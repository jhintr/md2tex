\chapter{忿狷第三十一}

\subsection*{1}

\textbf{魏武有一妓,聲最清高,而情性酷惡,欲殺則愛才,欲置則不堪,於是選百人一時俱教,少時,果有一人聲及之,便殺惡性者。}

\subsection*{2}

\textbf{王藍田性急,嘗食雞子,以筯刺之,不得,便大怒,舉以擲地,雞子於地圓轉未止,仍下地以屐齒蹍之,又不得,瞋甚,復於地取內口中,齧破即吐之,王右軍聞而大笑曰:「使安期有此性,猶當無一豪可論,況藍田邪?」}{\footnotesize \textbf{中興書}曰述清貴簡正,少所推屈,唯以性急為累。安期,述父也,有名德,已見。}

\subsection*{3}

\textbf{王司州嘗乘雪往王螭許,}{\footnotesize 王胡之、王恬並已見。恬,小字螭虎。}\textbf{司州言氣少有牾逆於螭,便作色不夷,司州覺惡,便輿牀就之,持其臂曰:「汝詎復足與老兄計?」}{\footnotesize \textbf{按}王氏譜,胡之是恬從祖兄。}\textbf{螭撥其手曰:「冷如鬼手馨,彊來捉人臂。」}

\subsection*{4}

\textbf{桓宣武與袁彥道樗蒱,袁彥道齒不合,遂厲色擲去五木,溫太真云:「見袁生遷怒,知顏子為貴。」}{\footnotesize \textbf{論語}曰哀公問弟子孰為好學,孔子曰「有顏回者,好學,不遷怒,不貳過,不幸短命死矣」。}

\subsection*{5}

\textbf{謝無奕性粗彊,以事不相得,自往數王藍田,肆言極罵,王正色面壁不敢動,半日,謝去良久,轉頭問左右小吏曰:「去未?」答云:「已去。」然後復坐,時人歎其性急而能有所容。}

\subsection*{6}

\textbf{王令詣謝公,值習鑿齒已在坐,當與併榻,王徙倚不坐,公引之與對榻,去後,語胡兒曰:「子敬實自清立,但人為爾多矜咳,殊足損其自然。」}{\footnotesize \textbf{劉謙之晉紀}曰王獻之性甚整峻,不交非類。}

\subsection*{7}

\textbf{王大、王恭嘗俱在何僕射坐,}{\footnotesize \textbf{中興書}曰何澄,字子玄,清正有器望,歷尚書左僕射。}\textbf{恭時為丹陽尹,大始拜荊州,}{\footnotesize \textbf{靈鬼志謠徵}曰初,桓石民為荊州,鎮上明,民忽歌黃曇曲曰「黃曇英,揚州大佛來上明」,少時,石民死,王忱為荊州。佛大,忱小字也。}\textbf{訖將乖之際,大勸恭酒,恭不為飲,大逼彊之,轉苦,便各以帬帶繞手,恭府近千人,悉呼入齋,大左右雖少,亦命前,意便欲相殺,何僕射無計,因起排坐二人之間,方得分散,所謂勢利之交,古人羞之。}

\subsection*{8}

\textbf{桓南郡小兒時,與諸從兄弟各養鵝共鬬,南郡鵝每不如,甚以為忿,迺夜往鵝欄間,取諸兄弟鵝悉殺之,既曉,家人咸以驚駭,云是變怪,以白車騎,車騎曰:「無所致怪,當是南郡戲耳。」問,果如之。}