\chapter{排調第二十五}

\subsection*{1}

\textbf{諸葛瑾為豫州,遣別駕到臺,}{\footnotesize 瑾已見。}\textbf{語云:「小兒知談,卿可與語。」連往詣恪,}{\footnotesize \textbf{江表傳}曰恪,字元遜,瑾長子也,少有才名,發藻岐嶷,辯論應機,莫與為對,孫權見而奇之,謂瑾曰「藍田生玉,真不虛也」,仕吳至太傅,為孫峻所害。}\textbf{恪不與相見,後於張輔吳坐中相遇,}{\footnotesize \textbf{環濟吳紀}曰張昭,字子布,忠正有才義,仕吳,為輔吳將軍。}\textbf{別駕喚恪:「咄咄郎君。」恪因嘲之曰:「豫州亂矣,何咄咄之有?」答曰:「君明臣賢,未聞其亂。」恪曰:「昔唐堯在上,四凶在下。」答曰:「非唯四凶,亦有丹朱。」於是一坐大笑。}

\subsection*{2}

\textbf{晉文帝與二陳共車,過喚鍾會同載,即駛車委去,比出,已遠,既至,因嘲之曰:「與人期行,何以遲遲?望卿遙遙不至。」會答曰:「矯然懿實,何必同群?」帝復問會:「皋繇何如人?」答曰:「上不及堯舜,下不逮周孔,亦一時之懿士。」}{\footnotesize 二陳,騫與泰也。會父名繇,故以遙遙戲之。騫父矯,宣帝諱懿,泰父群,祖父寔,故以此酬之。}

\subsection*{3}

\textbf{鍾毓為黃門郎,有機警,在景王坐燕飲,時陳群子玄伯、武周子元夏同在坐,}{\footnotesize \textbf{魏志}曰武周,字伯南,沛國竹邑人,仕至光祿大夫。}\textbf{共嘲毓,景王曰:「皋繇何如人?」對曰:「古之懿士。」顧謂玄伯、元夏曰:「君子周而不比,群而不黨。」}{\footnotesize \textbf{孔安國注論語}曰忠信為周,阿黨為比,黨,助也,君子雖眾,不相私助。}

\subsection*{4}

\textbf{嵇、阮、山、劉在竹林酣飲,王戎後往,步兵曰:「俗物已復來敗人意。」}{\footnotesize \textbf{魏氏春秋}曰時謂王戎未能超俗也。}\textbf{王笑曰:「卿輩意,亦復可敗邪?」}

\subsection*{5}

\textbf{晉武帝問孫皓:}{\footnotesize \textbf{吳錄}曰皓,字元宗,一名彭祖,大皇帝孫也,景帝崩,皓嗣位,為晉所滅,封歸命侯。}\textbf{「聞南人好作爾汝歌,頗能為不?」皓正飲酒,因舉觴勸帝而言曰:「昔與汝為鄰,今與汝為臣,上汝一桮酒,令汝壽萬春。」帝悔之。}

\subsection*{6}

\textbf{孫子荊年少時欲隱,語王武子「當枕石漱流」,誤曰「漱石枕流」,王曰:「流可枕,石可漱乎?」孫曰:「所以枕流,欲洗其耳,}{\footnotesize \textbf{逸士傳}曰許由為堯所讓,其友巢父責之,由乃過清泠水洗耳拭目,曰「向聞貪言,負吾之友」。}\textbf{所以漱石,欲礪其齒。」}

\subsection*{7}

\textbf{頭責秦子羽云:}{\footnotesize 子羽未詳。}\textbf{「子曾不如太原溫顒、潁川荀㝢、}{\footnotesize 溫顒已見。\textbf{荀氏譜}曰㝢,字景伯,祖彧,太尉,父俁,御史中丞。\textbf{世語}曰㝢少與裴楷、王戎、杜默俱有名,仕晉,至尚書。}\textbf{范陽張華、士卿劉許、}{\footnotesize \textbf{晉百官名}曰劉許,字文生,涿鹿郡人,父放,魏驃騎將軍,許,惠帝時為宗正卿。\textbf{按}許與張華同范陽人,故曰士卿,互其辭也,宗正卿,或曰士卿。}\textbf{義陽鄒湛、河南鄭詡,}{\footnotesize \textbf{晉諸公贊}曰湛,字潤甫,新野人,以文義達,仕至侍中。詡,字思淵,滎陽開封人,為衛尉卿,祖泰,揚州刺史,父褒,司空。}\textbf{此數子者,或謇喫無宮商,或尪陋希言語,或淹伊多姿態,或讙譁少智諝,或口如含膠飴,或頭如巾齏杵,}{\footnotesize \textbf{文士傳}曰華為人少威儀,多姿態。推意此語,則此六句還以目上六人,而「口如含膠飴」則指鄒湛,湛辯麗英博,而有此稱,未詳。}\textbf{而猶以文采可觀,意思詳序,攀龍附鳳,並登天府。」}{\footnotesize \textbf{張敏集}載頭責子羽文曰余友有秦生者,雖有姊夫之尊,少而狎焉,同時好暱,有太原溫長仁顒、潁川荀景伯㝢、范陽張茂先華、士卿劉文生許、南陽鄒潤甫湛、河南鄭思淵詡,數年之中,繼踵登朝,而此賢身處陋巷,屢沽而無善價,亢志自若,終不衰墮,為之慨然,又怪諸賢既已在位,曾無伐木嚶鳴之聲,甚違王貢彈冠之義,故因秦生容貌之盛,為頭責之文以戲之,并以嘲六子焉,雖似諧謔,實有興也,其文曰,維泰始元年,頭責子羽曰「吾託子為頭,萬有餘日矣,大塊稟我以精,造我以形,我為子植髮膚、置鼻耳、安眉須、插牙齒、眸子摛光、雙顴隆起,每至出入之間,遨遊市里,行者辟易,坐者竦跽,或稱君侯,或言將軍,捧手傾側,佇立崎嶇,如此者,故我形之足偉也,子冠冕不戴,金銀不佩,釵以當笄,帢以代幗,旨味弗嘗,食粟茹菜,隈摧園間,糞壤汙黑,歲莫年過,曾不自悔,子厭我於形容,我賤子乎意態,若此者乎,必子行己之累也,子遇我如讐,我視子如仇,居常不樂,兩者俱憂,何其鄙哉!子欲為人寶也,則當如皋陶、后稷、巫咸、伊陟,保乂王家,永見封殖,子欲為名高也,則當如許由、子臧、卞隨、務光,洗耳逃祿,千歲流芳,子欲為遊說也,則當如陳軫、蒯通、陸生、鄧公,轉禍為福,令辭從容,子欲為進趣也,則當如賈生之求試、終軍之請使,砥礪鋒穎,以幹王事,子欲為恬淡也,則當如老聃之守一、莊周之自逸,廓然離俗,志陵雲日,子欲為隱遁也,則當如榮期之帶索、漁父之瀺灂,棲遲神丘,垂餌巨壑,此一介之所以顯身成名者也,今子上不希道德,中不效儒墨,塊然窮賤,守此愚惑,察子之情,觀子之志,退不為於處士,進無望於三事,而徒翫日勞形,習為常人之所喜,不亦過乎」,於是子羽愀然深念而對曰「凡所教敕,謹聞命矣,以受性拘係,不閑禮義,設以天幸,為子所寄,今欲使吾為忠也,即當如伍胥屈平,欲使吾為信也,則當殺身以成名,欲使吾為介節邪,則當赴水火以全貞,此四者,人之所忌,故吾不敢造意」,頭曰「子所謂天刑地網,剛德之尤,不登山抱木,則褰裳赴流,吾欲告爾以養性,誨爾以優游,而與蟣蝨同情,不聽我謀,悲哉!俱寓人體,而獨為子頭,且擬人其倫,喻子儕偶,子不如太原溫顒、潁川荀㝢、范陽張華、士卿劉許、南陽鄒湛、河南鄭詡,此數子者,或謇喫無宮商,或尪陋希言語,或淹伊多姿態,或讙譁少智諝,或口如含膠飴,或頭如巾齏杵,而猶以文采可觀,意思詳序,攀龍附鳳,並登天府,夫舐痔得車,沈淵得珠,豈若夫子徒令脣舌腐爛、手足沾濡哉?居有事之世而恥為權圖,譬猶鑿池抱甕,難以求富,嗟乎子羽!何異檻中之熊,深穽之虎,石間饑蟹,竇中之鼠,事力雖勤,見功甚苦,宜其拳局煎蹙,至老無所希也,支離其形,猶能不困,非命也夫,豈與夫子同處也」。}

\subsection*{8}

\textbf{王渾與婦鍾氏共坐,見武子從庭過,渾欣然謂婦曰:「生兒如此,足慰人意。」婦笑曰:「若使新婦得配參軍,生兒故可不啻如此。」}{\footnotesize \textbf{王氏家譜}曰淪,字太沖,司空穆侯中子,司徒渾弟也,醇粹簡遠,貴老莊之學,用心淡如也,為老子例略、周紀,年二十餘,舉孝廉,不行,歷大將軍參軍,年二十五卒,大將軍為之流涕。}

\subsection*{9}

\textbf{荀鳴鶴、陸士龍二人未相識,俱會張茂先坐,張令共語,以其並有大才,可勿作常語,陸舉手曰:「雲間陸士龍。」荀答曰:「日下荀鳴鶴。」陸曰:「既開青雲覩白雉,何不張爾弓、布爾矢?」荀答曰:「本謂雲龍騤騤,定是山鹿野麋,獸弱弩彊,是以發遲。」張乃撫掌大笑。}{\footnotesize \textbf{晉百官名}曰荀隱,字鳴鶴,潁川人。\textbf{荀氏家傳}曰隱祖昕,樂安太守,父岳,中書郎,隱與陸雲在張華坐語,互相反覆,陸連受屈,隱辭皆美麗,張公稱善云,世有此書,尋之未得,歷太子舍人、廷尉平,蚤卒。}

\subsection*{10}

\textbf{陸太尉詣王丞相,}{\footnotesize 陸玩已見。}\textbf{王公食以酪,陸還遂病,明日與王牋云:「昨食酪小過,通夜委頓,民雖吳人,幾為傖鬼。」}

\subsection*{11}

\textbf{元帝皇子生,普賜群臣,殷洪喬謝曰:}{\footnotesize 殷羡已見。}\textbf{「皇子誕育,普天同慶,臣無勳焉,而猥頒厚賚。」中宗笑曰:「此事豈可使卿有勳邪?」}

\subsection*{12}

\textbf{諸葛令、王丞相共爭姓族先後,王曰:「何不言葛、王而云王、葛?」令曰:「譬言驢馬,不言馬驢,驢寧勝馬邪?」}{\footnotesize 諸葛恢已見。}

\subsection*{13}

\textbf{劉真長始見王丞相,時盛暑之月,丞相以腹熨彈棊局,曰:「何乃渹?」}{\footnotesize 吳人以冷為渹。}\textbf{劉既出,人問見王公云何,劉曰:「未見他異,唯聞作吳語耳。」}{\footnotesize \textbf{語林}曰真長云「丞相何奇,止能作吳語及細唾也」。}

\subsection*{14}

\textbf{王公與朝士共飲酒,舉瑠璃盌謂伯仁曰:「此盌腹殊空,謂之寶器,何邪?」}{\footnotesize 以戲周之無能。}\textbf{答曰:「此盌英英,誠為清徹,所以為寶耳。」}

\subsection*{15}

\textbf{謝幼輿謂周侯曰:「卿類社樹,遠望之,峨峨拂青天,就而視之,其根則群狐所託,下聚溷而已。」}{\footnotesize 謂顗好媟瀆故。}\textbf{答曰:「枝條拂青天,不以為高,群狐亂其下,不以為濁,聚溷之穢,卿之所保,何足自稱?」}

\subsection*{16}

\textbf{王長豫幼便和令,丞相愛恣甚篤,每共圍棊,丞相欲舉行,長豫按指不聽,丞相笑曰:「詎得爾?相與似有瓜葛。」}{\footnotesize \textbf{蔡邕}曰瓜葛,疎親也。}

\subsection*{17}

\textbf{明帝問周伯仁:「真長何如人?」答曰:「故是千斤犗特。」王公笑其言,伯仁曰:「不如卷角牸,有盤辟之好。」}{\footnotesize 以戲王也。}

\subsection*{18}

\textbf{王丞相枕周伯仁厀,指其腹曰:「卿此中何所有?」答曰:「此中空洞無物,然容卿輩數百人。」}

\subsection*{19}

\textbf{干寶向劉真長}{\footnotesize \textbf{中興書}曰寶,字令升,新蔡人,祖正,吳奮武將軍,父瑩,丹陽丞,寶少以博學才器著稱,歷散騎常侍。}\textbf{敘其搜神記,}{\footnotesize \textbf{孔氏志怪}曰寶父有嬖人,寶母至妬,葬寶父時,因推著藏中,經十年而母喪,開墓,其婢伏棺上,就視猶煖,漸有氣息,輿還家,終日而蘇,說寶父常致飲食,與之接寢,恩情如生,家中吉凶,輒語之,校之悉驗,平復數年後方卒,寶因作搜神記,中云「有所感起」是也。}\textbf{劉曰:「卿可謂鬼之董狐。」}{\footnotesize \textbf{春秋傳}曰趙穿攻晉靈公於桃園,趙宣子未出境而復,太史書「趙盾弒其君」,宣子曰「不然」,對曰「子為正卿,亡不越境,反不討賊,非子而誰」,孔子曰「董狐,古之良史也,書法不隱,趙盾,古之賢大夫也,為法受惡」。}

\subsection*{20}

\textbf{許思文往顧和許,顧先在帳中眠,許至,便徑就牀角枕共語,}{\footnotesize 許璪已見。}\textbf{既而喚顧共行,顧乃命左右取枕上新衣,易己體上所著,許笑曰:「卿乃復有行來衣乎?」}

\subsection*{21}

\textbf{康僧淵目深而鼻高,王丞相每調之,僧淵曰:「鼻者面之山,}{\footnotesize \textbf{管輅別傳}曰鼻者天中之山。\textbf{相書}曰鼻之所在為天中,鼻有山象,故曰山。}\textbf{目者面之淵,山不高則不靈,淵不深則不清。」}

\subsection*{22}

\textbf{何次道往瓦官寺禮拜甚勤,}{\footnotesize 充崇釋氏,甚加敬也。}\textbf{阮思曠語之曰:「卿志大宇宙,}{\footnotesize \textbf{尹子}曰天地四方曰宇,往古來今曰宙。}\textbf{勇邁終古。」}{\footnotesize 終古,往古也。\textbf{楚辭}曰吾不能忍此終古也。}\textbf{何曰:「卿今日何故忽見推?」阮曰:「我圖數千戶郡,尚不能得,卿迺圖作佛,不亦大乎?」}{\footnotesize 思曠,裕也。}

\subsection*{23}

\textbf{庾征西大舉征胡,既成行,止鎮襄陽,}{\footnotesize \textbf{晉陽秋}曰翼率眾入沔,將謀伐狄,既至襄陽,狄尚彊,未可決戰,會康帝崩,兄冰薨,留長子方之守襄陽,自馳還夏口。}\textbf{殷豫章與書,送一折角如意以調之,}{\footnotesize 豫章,殷羡。}\textbf{庾答書曰:「得所致,雖是敗物,猶欲理而用之。」}

\subsection*{24}

\textbf{桓大司馬乘雪欲獵,先過王、劉諸人許,真長見其裝束單急,問:「老賊欲持此何作?」桓曰:「我若不為此,卿輩亦那得坐談?」}{\footnotesize \textbf{語林}曰宣武征還,劉尹數十里迎之,桓都不語,直云「垂長衣,談清言,竟是誰功」,劉答曰「晉德靈長,功豈在爾」。二人說小異,故詳載之。}

\subsection*{25}

\textbf{褚季野問孫盛:「卿國史何當成?」孫云:「久應竟,在公無暇,故至今日。」褚曰:「古人述而不作,何必在蠶室中?」}{\footnotesize \textbf{漢書}曰李陵降匈奴,武帝甚怒,太史令司馬遷盛明陵之忠,帝以遷為陵遊說,下遷腐刑,乃述唐虞以來,至于獲麟,為史記,遷與任安書曰「李陵既生降,僕又茸之以蠶室」。\textbf{蘇林}注曰腐刑者,作密室蓄火,時如蠶室。舊時平陰有蠶室獄。}

\subsection*{26}

\textbf{謝公在東山,朝命屢降而不動,後出為桓宣武司馬,將發新亭,朝士咸出瞻送,高靈時為中丞,亦往相祖,先時,多少飲酒,因倚如醉,戲曰:「卿屢違朝旨,高臥東山,諸人每相與言『安石不肯出,將如蒼生何』,今亦蒼生將如卿何?」謝笑而不答。}{\footnotesize 高靈已見。\textbf{婦人集}載桓玄問王凝之妻謝氏曰「太傅東山二十餘年,遂復不終,其理云何」,謝答曰「亡叔太傅先正,以無用為心,顯隱為優劣,始末正當動靜之異耳」。}

\subsection*{27}

\textbf{初,謝安在東山居,布衣,時兄弟已有富貴者,翕集家門,傾動人物,劉夫人戲謂安曰:「大丈夫不當如此乎?」謝乃捉鼻曰:「但恐不免耳。」}

\subsection*{28}

\textbf{支道林因人就深公買印山,深公答曰:「未聞巢、由買山而隱。」}{\footnotesize \textbf{逸士傳}曰巢父者,堯時隱人,山居,不營世利,年老以樹為巢而寢其上,故號巢父。\textbf{高逸沙門傳}遁得深公之言,慙恧而已。}

\subsection*{29}

\textbf{王、劉每不重蔡公,二人嘗詣蔡,語良久,乃問蔡曰:「公自言何如夷甫?」答曰:「身不如夷甫。」王、劉相目而笑曰:「公何處不如?」答曰:「夷甫無君輩客。」}

\subsection*{30}

\textbf{張吳興年八歲,虧齒,}{\footnotesize 玄之已見。}\textbf{先達知其不常,故戲之曰:「君口中何為開狗竇?」張應聲答曰:「正使君輩從此中出入。」}

\subsection*{31}

\textbf{郝隆七月七日出日中仰臥,人問其故,答曰:「我曬書。」}{\footnotesize \textbf{征西寮屬名}曰隆,字佐治,汲郡人,仕吳至征西參軍。}

\subsection*{32}

\textbf{謝公始有東山之志,後嚴命屢臻,勢不獲已,始就桓公司馬,于時人有餉桓公藥草,中有遠志,公取以問謝:「此藥又名小草,何一物而有二稱?」}{\footnotesize \textbf{本草}曰遠志,一名棘宛,其葉名小草。}\textbf{謝未即答,時郝隆在坐,應聲答曰:「此甚易解,處則為遠志,出則為小草。」謝甚有愧色,桓公目謝而笑曰:「郝參軍此過乃不惡,亦極有會。」}

\subsection*{33}

\textbf{庾園客詣孫監,值行,見齊莊在外,尚幼,而有神意,庾試之曰:「孫安國何在?」即答曰:「庾穉恭家。」庾大笑曰:「諸孫大盛,有兒如此。」又答曰:「未若諸庾之翼翼。」還,語人曰:「我故勝,得重喚奴父名。」}{\footnotesize \textbf{孫放別傳}曰放兄弟並秀異,與庾翼子園客同為學生,園客少有佳稱,因談笑嘲放曰「諸孫於今為盛」,盛,監君諱也,放即答曰「未若諸庾之翼翼」,放應機制勝,時人仰焉,司馬景王、陳、鍾諸賢相酬,無以踰也。}

\subsection*{34}

\textbf{范玄平在簡文坐,談欲屈,引王長史曰:「卿助我。」}{\footnotesize \textbf{范汪別傳}曰汪,字玄平,潁陽人,左將軍略之孫,少有不常之志,通敏多識,博涉經籍,致譽於時,歷吏部尚書、徐兗二州刺史。}\textbf{王曰:「此非拔山力所能助。」}{\footnotesize \textbf{史記}曰項羽為漢兵所圍,夜起歌曰「力拔山兮氣蓋世,時不利兮騅不逝」。}

\subsection*{35}

\textbf{郝隆為桓公南蠻參軍,三月三日會,作詩,不能者罰酒三升,隆初以不能受罰,既飲,攬筆便作一句云「娵隅躍清池」,桓問:「娵隅是何物?」答曰:「蠻名魚為娵隅。」桓公曰:「作詩何以作蠻語?」隆曰:「千里投公,始得蠻府參軍,那得不作蠻語也?」}

\subsection*{36}

\textbf{袁羊嘗詣劉恢,恢在內眠未起,袁因作詩調之曰:「角枕粲文茵,錦衾爛長筵。」}{\footnotesize \textbf{唐詩}曰晉獻公好攻戰,國人多喪,其詩曰「角枕粲兮,錦衾爛兮,予美亡此,誰與獨旦」。袁故嘲之。}\textbf{劉尚晉明帝女,}{\footnotesize \textbf{晉陽秋}曰恢尚廬陵長公主,名南弟。}\textbf{主見詩,不平曰:「袁羊,古之遺狂。」}

\subsection*{37}

\textbf{殷洪遠答孫興公詩云:「聊復放一曲。」劉真長笑其語拙,問曰:「君欲云那放?」殷曰:「榻臘亦放,何必其鎗鈴邪?」}{\footnotesize 殷融已見。}

\subsection*{38}

\textbf{桓公既廢海西,立簡文,}{\footnotesize \textbf{晉陽秋}曰海西公諱奕,字延齡,成帝子也,興寧中即位,少同閹人之疾,使宮人與左右淫通生子,大司馬溫自廣陵還姑孰,過京都,以皇太后令,廢帝為海西公。}\textbf{侍中謝公見桓公拜,桓驚笑曰:「安石,卿何事至爾?」謝曰:「未有君拜於前,臣立於後。」}

\subsection*{39}

\textbf{郗重熙與謝公書,道:「王敬仁聞一年少懷問鼎,}{\footnotesize 郗曇、王脩已見。\textbf{史記}曰楚莊王觀兵於周郊,周定王使王孫滿迎勞楚王,王問鼎大小輕重,對曰「在德不在鼎」,莊王曰「子無阻九鼎,楚國折鉤之喙,足以為九鼎也」。}\textbf{不知桓公德衰,為復後生可畏?」}{\footnotesize \textbf{春秋傳}曰齊桓公伐楚,責苞茅之不貢。\textbf{論語}曰後生可畏,焉知來者之不如今?\textbf{孔安國}曰後生,少年。}

\subsection*{40}

\textbf{張蒼梧是張憑之祖,嘗語憑父曰:「我不如汝。」憑父未解所以,蒼梧曰:「汝有佳兒。」}{\footnotesize \textbf{張蒼梧碑}曰君諱鎮,字義遠,吳國吳人,忠恕寬明,簡正貞粹,泰安中,除蒼梧太守,討王含有功,封興道縣侯。}\textbf{憑時年數歲,歛手曰:「阿翁,詎宜以子戲父?」}

\subsection*{41}

\textbf{習鑿齒、孫興公未相識,同在桓公坐,桓語孫:「可與習參軍共語。」孫云:「蠢爾蠻荊,敢與大邦為讐?」習云:「薄伐獫狁,至于太原。」}{\footnotesize 小雅詩也。\textbf{毛詩注}曰蠢,動也,荊蠻,荊之蠻也,獫狁,北夷也。習鑿齒,襄陽人,孫興公,太原人,故因詩以相戲也。}

\subsection*{42}

\textbf{桓豹奴是王丹陽外生,形似其舅,桓甚諱之,}{\footnotesize 豹奴,桓嗣小字。\textbf{中興書}曰嗣,字恭祖,車騎將軍沖子也,少有清譽,仕至江州刺史。\textbf{王氏譜}曰混,字奉正,中軍將軍恬子,仕至丹陽尹。}\textbf{宣武云:「不恆相似,時似耳,恆似是形,時似是神。」桓逾不說。}

\subsection*{43}

\textbf{王子猷詣謝萬,林公先在坐,瞻矚甚高,王曰:「若林公鬚髮並全,神情當復勝此不?」謝曰:「脣齒相須,不可以偏亡,}{\footnotesize \textbf{春秋傳}曰脣亡齒寒。}\textbf{鬚髮何關於神明?」林公意甚惡,曰:「七尺之軀,今日委君二賢。」}

\subsection*{44}

\textbf{郗司空拜北府,}{\footnotesize \textbf{南徐州記}曰舊徐州都督以東為稱,晉氏南遷,徐州刺史王舒加北中郎將,北府之號,自此起也。}\textbf{王黃門詣郗門拜,云:「應變將略,非其所長。」驟詠之不已,郗倉謂嘉賓曰:「公今日拜,子猷言語殊不遜,深不可容。」}{\footnotesize 倉,郗融小字也。\textbf{郗氏譜}曰融,字景山,愔第二子,辟琅邪王文學,不拜而蚤終。}\textbf{嘉賓曰:「此是陳壽作諸葛評,}{\footnotesize \textbf{蜀志陳壽}評曰亮連年動眾,而無成功,蓋應變將略,非其所長也。\textbf{王隱晉書}曰壽,字承祚,巴西安漢人,好學,善著述,仕至中庶子,初,壽父為馬謖參軍,諸葛亮誅謖,髡其父頭,亮子瞻又輕壽,故壽撰蜀志,以愛憎為評也。}\textbf{人以汝家比武侯,復何所言?」}

\subsection*{45}

\textbf{王子猷詣謝公,謝曰:「云何七言詩?」}{\footnotesize \textbf{東方朔傳}曰漢武帝在栢梁臺上,使群臣作七言詩。七言詩自此始也。}\textbf{子猷承問,答曰:「昂昂若千里之駒,汎汎若水中之鳧。」}{\footnotesize 出離騷。}

\subsection*{46}

\textbf{王文度、范榮期俱為簡文所要,范年大而位小,王年小而位大,將前,更相推在前,既移久,王遂在范後,王因謂曰:「簸之揚之,穅秕在前。」范曰:「洮之汰之,沙礫在後。」}{\footnotesize 王坦之、范啓已見。一說是孫綽、習鑿齒言。}

\subsection*{47}

\textbf{劉遵祖少為殷中軍所知,稱之於庾公,庾公甚忻然,便取為佐,既見,坐之獨榻上與語,劉爾日殊不稱,庾小失望,遂名之為「羊公鶴」,昔羊叔子有鶴善舞,嘗向客稱之,客試使驅來,氃氋而不肯舞,故稱比之。}{\footnotesize \textbf{徐廣晉紀}曰劉爰之,字遵祖,沛郡人,少有才學,能言理,歷中書郎、宣城太守。}

\subsection*{48}

\textbf{魏長齊雅有體量,而才學非所經,初宦當出,虞存嘲之曰:「與卿約法三章,談者死,文筆者刑,商略抵罪。」魏怡然而笑,無忤於色。}{\footnotesize \textbf{魏氏譜}曰顗,字長齊,會稽人,祖胤,處士,父說,大鴻臚卿,顗仕至山陰令。\textbf{漢書}曰沛公入咸陽,召諸父老曰「天下苦秦苛法久矣,今與父老約法三章耳,殺人者死,傷人及盜抵罪」。\textbf{應劭}注曰抵,至也,但至於罪。}

\subsection*{49}

\textbf{郗嘉賓書與袁虎,道戴安道、謝居士云:「恆任之風,當有所弘耳。」以袁無恆,故以此激之。}{\footnotesize 袁、戴、謝並已見。}

\subsection*{50}

\textbf{范啓與郗嘉賓書曰:「子敬舉體無饒縱,掇皮無餘潤。」郗答曰:「舉體無餘潤,何如舉體非真者?」范性矜假多煩,故嘲之。}

\subsection*{51}

\textbf{二郗奉道,二何奉佛,皆以財賄,謝中郎云:「二郗諂於道,二何佞於佛。」}{\footnotesize \textbf{中興書}曰郗愔及弟曇奉天師道。\textbf{晉陽秋}曰何充性好佛道,崇修佛寺,供給沙門以百數,久在揚州,徵役吏民,功賞萬計,是以為遐邇所譏,充弟準,亦精勤,唯讀佛經、營治寺廟而已矣。}

\subsection*{52}

\textbf{王文度在西州,與林法師講,韓、孫諸人並在坐,林公理每欲小屈,孫興公曰:「法師今日如著弊絮在荊棘中,觸地挂閡。」}

\subsection*{53}

\textbf{范榮期見郗超俗情不淡,戲之曰:「夷、齊、巢、許,一詣垂名,何必勞神苦形,支策據梧邪?」郗未答,韓康伯曰:「何不使遊刃皆虛?」}{\footnotesize \textbf{莊子}曰昭文之鼓琴,師曠之支策,惠子之據梧,三子之智幾矣,皆其盛也,故載之末年。庖丁為文惠君解牛,三年之後,未嘗見全牛也,用刀十九年矣,所解數千牛,而刀刃若新發於硎,文惠君問之,庖丁曰「彼節者有閒,而刀刃無厚,以無厚入有閒,恢恢乎其於遊刃必有餘地」。}

\subsection*{54}

\textbf{簡文在殿上行,右軍與孫興公在後,右軍指簡文語孫曰:「此噉名客。」簡文顧曰:「天下自有利齒兒。」後王光祿作會稽,謝車騎出曲阿祖之,}{\footnotesize 王蘊、謝玄已見。}\textbf{王孝伯罷祕書丞在坐,謝言及此事,因視孝伯曰:「王丞齒似不鈍。」王曰:「不鈍,頗亦驗。」}

\subsection*{55}

\textbf{謝遏夏月嘗仰臥,謝公清晨卒來,不暇著衣,跣出屋外,方躡履問訊,公曰:「汝可謂前倨而後恭。」}{\footnotesize \textbf{戰國策}曰蘇秦說惠王而不見用,黑貂之裘弊,黃金百斤盡,大困而歸,父母不與言,妻不為下機,嫂不為炊,後為從長,行過洛陽,車騎輜重甚眾,秦之昆弟妻嫂側目不敢視,秦笑謂其嫂曰「何先倨而後恭」,嫂謝曰「見季子位高而金多」,秦歎曰「一人之身,富貴則親戚畏懼,貧賤則輕易之,而況於他人哉」。}

\subsection*{56}

\textbf{顧長康作殷荊州佐,請假還東,爾時例不給布颿,顧苦求之乃得,發至破冢,遭風大敗,}{\footnotesize \textbf{周祗隆安記}曰破冢,洲名,在華容縣。}\textbf{作牋與殷云:「地名破冢,真破冢而出,行人安穩,布颿無恙。」}

\subsection*{57}

\textbf{苻朗初過江,}{\footnotesize \textbf{裴景仁秦書}曰朗,字元達,苻堅從兄子,性宏放,神氣爽悟,堅常曰「吾家千里駒也」,堅為慕容沖所圍,朗降謝玄,用為員外散騎侍郎,吏部郎王忱與兄國寶命駕詣之,沙門法汰問朗曰「見王吏部兄弟未」,朗曰「非一狗面人心,又一人面狗心者是邪」,忱醜而才,國寶美而狠故也,朗常與朝士宴,時賢並用唾壺,朗欲夸之,使小兒跪而張口,唾而含出,又善識味,會稽王道子為設精饌,訖,問「關中之食,孰若於此」,朗曰「皆好,唯鹽味小生」,即問宰夫,如其言,或人殺雞以食之,朗曰「此雞棲,恆半露」,問之亦驗,又食鵝炙,知白黑之處,咸試而記之,無豪釐之差,著苻子數十篇,蓋老莊之流也,朗矜高忤物,不容於世,後眾讒而殺之。}\textbf{王咨議大好事,問中國人物及風土所生,終無極已,}{\footnotesize \textbf{王氏譜}曰肅之,字幼恭,右將軍羲之第四子,歷中書郎、驃騎咨議。}\textbf{朗大患之,次復問奴婢貴賤,朗云:「謹厚有識中者,乃至十萬,無意為奴婢問者,止數千耳。」}

\subsection*{58}

\textbf{東府客館是版屋,謝景重詣太傅,時賓客滿中,初不交言,直仰視云:「王乃復西戎其屋。」}{\footnotesize \textbf{秦詩敘}曰襄公備其兵甲,以討西戎,婦人閔其君子,故作詩曰「在其版屋,亂我心曲」。\textbf{毛公}注曰西戎之版屋也。}

\subsection*{59}

\textbf{顧長康噉甘蔗,先食尾,問所以,云:「漸至佳境。」}

\subsection*{60}

\textbf{孝武屬王珣求女壻,曰:「王敦、桓溫,磊砢之流,既不可復得,且小如意,亦好豫人家事,酷非所須,正如真長、子敬比,最佳。」珣舉謝混,後袁山松欲擬謝婚,}{\footnotesize \textbf{續晉陽秋}曰山松,陳郡人,祖喬,益州刺史,父方平,義興太守,山松歷祕書監、吳國內史,孫恩作亂,見害,初,帝為晉陵公主訪壻於王珣,珣舉謝混云「人才不及真長、不減子敬」,帝曰「如此便已足矣」。}\textbf{王曰:「卿莫近禁臠。」}

\subsection*{61}

\textbf{桓南郡與殷荊州語次,因共作了語,顧愷之曰:「火燒平原無遺燎。」桓曰:「白布纏棺豎旒旐。」殷曰:「投魚深淵放飛鳥。」次復作危語,桓曰:「矛頭淅米劍頭炊。」殷曰:「百歲老翁攀枯枝。」顧曰:「井上轆轤臥嬰兒。」殷有一參軍在坐,云:「盲人騎瞎馬,夜半臨深池。」殷曰:「咄咄逼人。」仲堪眇目故也。}{\footnotesize \textbf{中興書}曰仲堪父嘗疾患經時,仲堪衣不解帶數年,自分劑湯藥,誤以藥手拭淚,遂眇一目。}

\subsection*{62}

\textbf{桓玄出射,有一劉參軍與周參軍朋賭,垂成,唯少一破,劉謂周曰:「卿此起不破,我當撻卿。」周曰:「何至受卿撻?」劉曰:「伯禽之貴,尚不免撻,而況於卿?」}{\footnotesize \textbf{尚書大傳}曰伯禽與康叔見周公,三見而三笞,康叔有駭色,謂伯禽曰「有商子者,賢人也,與子見之」,乃見商子而問焉,商子曰「南山之陽有木焉,名喬」,二三子往觀之,見喬實高高然而上,反以告商子,商子曰「喬者,父道也,南山之陰有木焉,名曰梓」,二三子復往觀焉,見梓實晉晉然而俯,反以告商子,商子曰「梓者,子道也」,二三子明日見周公,入門而趨,登堂而跪,周公拂其首,勞而食之曰「爾安見君子乎」。\textbf{禮記}曰成王有罪,周公則撻伯禽。亦其義也。}\textbf{周殊無忤色,桓語庾伯鸞曰:}{\footnotesize \textbf{晉東宮百官名}曰庾鴻,字伯鸞,潁川人。\textbf{庾氏譜}曰鴻祖義,吳國內史,父楷,左衛將軍,鴻仕至輔國內史。}\textbf{「劉參軍宜停讀書,周參軍且勤學問。」}

\subsection*{63}

\textbf{桓南郡與道曜講老子,王侍中為主簿在坐,桓曰:「王主簿,可顧名思義。」王未答,且大笑,桓曰:「王思道能作大家兒笑。」}{\footnotesize 道曜,未詳。思道,王禎之小字也,老子明道,禎之字思道,故曰「顧名思義」。}

\subsection*{64}

\textbf{祖廣行恆縮頭,詣桓南郡,始下車,桓曰:「天甚晴朗,祖參軍如從屋漏中來。」}{\footnotesize \textbf{祖氏譜}曰廣,字淵度,范陽人,父台之,光祿大夫,廣仕至護軍長史。}

\subsection*{65}

\textbf{桓玄素輕桓崖,崖在京下有好桃,玄連就求之,遂不得佳者,}{\footnotesize 崖,桓脩小字。\textbf{續晉陽秋}曰脩少為玄所侮,於言端常嗤鄙之。}\textbf{玄與殷仲文書,以為嗤笑曰:「德之休明,肅慎貢其楛矢,如其不爾,籬壁間物亦不可得也。」}{\footnotesize \textbf{國語}曰仲尼在陳,有隼集陳侯之庭而死,楛矢貫之,石砮尺有咫,問於仲尼,對曰「隼之來遠矣,此肅慎之矢也,昔武王克商,通道于九夷百蠻,使各以方賄貢,於是肅慎氏貢楛矢,古者分異姓之職,使不忘服也,故分陳以肅慎之貢,若求之故府,其可得」,使求,得之金櫝,如初。}