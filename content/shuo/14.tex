\chapter{容止第十四}

\subsection*{1}

\textbf{魏武將見匈奴使,自以形陋,不足雄遠國,}{\footnotesize \textbf{魏氏春秋}曰武王姿貌短小,而神明英發。}\textbf{使崔季珪代,帝自捉刀立牀頭,既畢,令間諜問曰:「魏王何如?」匈奴使答曰:「魏王雅望非常,}{\footnotesize \textbf{魏志}曰崔琰,字季珪,清河東武城人,聲姿高暢,眉目疏朗,鬚長四尺,甚有威重。}\textbf{然牀頭捉刀人,此乃英雄也。」魏武聞之,追殺此使。}

\subsection*{2}

\textbf{何平叔美姿儀,面至白,魏明帝疑其傅粉,正夏月,與熱湯餅,既噉,大汗出,以朱衣自拭,色轉皎然。}{\footnotesize \textbf{魏略}曰晏性自喜,動靜粉帛不去手,行步顧影。\textbf{按}此言則晏之妖麗,本資外飾,且晏養自宮中,與帝相長,豈復疑其形姿,待騐而明也。}

\subsection*{3}

\textbf{魏明帝使后弟毛曾與夏侯玄共坐,時人謂「蒹葭倚玉樹」。}{\footnotesize \textbf{魏志}曰玄為黃門侍郎,與毛曾並坐,玄甚恥之,曾說形於色,明帝恨之,左遷玄為羽林監。}

\subsection*{4}

\textbf{時人目「夏侯太初朗朗如日月之入懷,李安國頹唐如玉山之將崩」。}{\footnotesize \textbf{魏略}曰李豐,字安國,衛尉李義子也,識別人物,海內注意,明帝得吳降人,問江東聞中國名士為誰,以安國對之,是時豐為黃門郎,改名宣,上問安國所在,左右公卿即具以豐對,上曰「豐名乃被於吳越邪」,仕至中書令,為晉王所誅。}

\subsection*{5}

\textbf{嵇康身長七尺八寸,風姿特秀,}{\footnotesize \textbf{康別傳}曰康長七尺八寸,偉容色,土木形骸,不加飾厲,而龍章鳳姿,天質自然,正爾在群形之中,便自知非常之器。}\textbf{見者歎曰:「蕭蕭肅肅,爽朗清舉。」或云:「肅肅如松下風,高而徐引。」山公曰:「嵇叔夜之為人也,巖巖若孤松之獨立,其醉也,傀俄若玉山之將崩。」}

\subsection*{6}

\textbf{裴令公目王安豐:「眼爛爛如巖下電。」}{\footnotesize 王戎形狀短小,而目甚清炤,視日不眩。}

\subsection*{7}

\textbf{潘岳妙有姿容,好神情,}{\footnotesize \textbf{岳別傳}曰岳姿容甚美,風儀閒暢。}\textbf{少時挾彈出洛陽道,婦人遇者,莫不連手共縈之,左太沖絕醜,}{\footnotesize \textbf{續文章志}曰思貌醜顇,不持儀飾。}\textbf{亦復效岳遊遨,於是群嫗齊共亂唾之,委頓而返。}{\footnotesize \textbf{語林}曰安仁至美,每行,老嫗以果擲之,滿車,張孟陽至醜,每行,小兒以瓦石投之,亦滿車。二說不同。}

\subsection*{8}

\textbf{王夷甫容貌整麗,妙於談玄,恆捉白玉柄麈尾,與手都無分別。}

\subsection*{9}

\textbf{潘安仁、夏侯湛並有美容,喜同行,時人謂之「連璧」。}{\footnotesize \textbf{八王故事}曰岳與湛著契,故好同遊。}

\subsection*{10}

\textbf{裴令公有儁容姿,一旦有疾至困,惠帝使王夷甫往看,裴方向壁臥,聞王使至,強回視之,王出語人曰:「雙目閃閃,若巖下電,精神挺動,體中故小惡。」}{\footnotesize \textbf{名士傳}曰楷病困,詔遣黃門郎王夷甫省之,楷回眸屬夷甫云「竟未相識」,夷甫還,亦歎其神儁。}

\subsection*{11}

\textbf{有人語王戎曰:「嵇延祖卓卓如野鶴之在雞群。」答曰:「君未見其父耳。」}{\footnotesize 康已見上。}

\subsection*{12}

\textbf{裴令公有儁容儀,脫冠冕,粗服亂頭皆好,時人以為「玉人」,見者曰:「見裴叔則如玉山上行,光映照人。」}

\subsection*{13}

\textbf{劉伶身長六尺,貌甚醜顇,而悠悠忽忽,土木形骸。}{\footnotesize \textbf{梁祚魏國統}曰劉伶,字伯倫,形貌醜陋,身長六尺,然肆意放蕩,悠焉獨暢,自得一時,常以宇宙為狹。}

\subsection*{14}

\textbf{驃騎王武子是衛玠之舅,儁爽有風姿,見玠輒歎曰:「珠玉在側,覺我形穢。」}{\footnotesize \textbf{玠別傳}曰驃騎王濟,玠之舅也,嘗與同遊,語人曰「昨日吾與外生共坐,若明珠之在側,朗然來照人」。}

\subsection*{15}

\textbf{有人詣王太尉,遇安豐、大將軍、丞相在坐,往別屋,見季胤、平子,}{\footnotesize \textbf{石崇金谷詩敘}曰王詡,字季胤,琅邪人。\textbf{王氏譜}曰詡,夷甫弟也,仕至脩武令。}\textbf{還,語人曰:「今日之行,觸目見琳琅珠玉。」}

\subsection*{16}

\textbf{王丞相見衛洗馬曰:「居然有羸形,雖復終日調暢,若不堪羅綺。」}{\footnotesize \textbf{玠別傳}曰玠素抱羸疾。\textbf{西京賦}曰始徐進而羸形,似不勝乎羅綺。}

\subsection*{17}

\textbf{王大將軍稱太尉:「處眾人中,似珠玉在瓦石間。」}

\subsection*{18}

\textbf{庾子嵩長不滿七尺,腰帶十圍,頹然自放。}

\subsection*{19}

\textbf{衛玠從豫章至下都,人久聞其名,觀者如堵牆,玠先有羸疾,體不堪勞,遂成病而死,時人謂「看殺衛玠」。}{\footnotesize \textbf{玠別傳}曰玠在群伍之中,寔有異人之望,齠齔時,乘白羊車於洛陽市上,咸曰「誰家璧人」,於是家門州黨號為「璧人」。\textbf{按}永嘉流人名曰「玠以永嘉六年五月六日至豫章,其年六月二十日卒」,此則玠之南度豫章四十五日,豈暇至下都而亡乎?且諸書皆云玠亡在豫章,而不云在下都也。}

\subsection*{20}

\textbf{周伯仁道桓茂倫:「嶔崎歷落,可笑人。」或云謝幼輿言。}

\subsection*{21}

\textbf{周侯說王長史父:}{\footnotesize \textbf{王氏譜}曰訥,字文淵,太原人,祖默,尚書,父祐,散騎常侍,訥始過江,仕至新淦令。}\textbf{「形貌既偉,雅懷有槩,保而用之,可作諸許物也。」}

\subsection*{22}

\textbf{祖士少見衛君長云:「此人有旄杖下形。」}

\subsection*{23}

\textbf{石頭事故,朝廷傾覆,}{\footnotesize \textbf{晉陽秋}曰蘇峻自姑孰至于石頭,逼遷天子,峻以倉屋為宮,使人守衛。\textbf{靈鬼志謠徵}曰明帝末有謠歌「側側力,放馬出山側,大馬死,小馬餓」,後峻遷帝於石頭,御膳不具。}\textbf{溫忠武與庾文康投陶公求救,陶公云:「肅祖顧命不見及,且蘇峻作亂,釁由諸庾,誅其兄弟,不足以謝天下。」}{\footnotesize \textbf{徐廣晉紀}曰肅祖遺詔,庾亮、王導輔幼主而進大臣官,陶侃、祖約不在其例,侃、約疑亮寢遺詔也。\textbf{中興書}曰初,庾亮欲徵蘇峻,卞壼不許,溫嶠及三吳欲起兵衛帝室,亮不聽,下制曰「妄起兵者誅」,故峻得作亂京邑也。}\textbf{于時庾在溫船後聞之,憂怖無計,別日,溫勸庾見陶,庾猶豫未能往,溫曰:「溪狗我所悉,卿但見之,必無憂也。」庾風姿神貌,陶一見便改觀,談宴竟日,愛重頓至。}

\subsection*{24}

\textbf{庾太尉在武昌,秋夜氣佳景清,佐吏殷浩、王胡之之徒登南樓理詠,音調始遒,聞函道中有屐聲甚厲,定是庾公,俄而率左右十許人步來,諸賢欲起避之,公徐云:「諸君少住,老子於此處興復不淺。」因便據胡牀,與諸人詠謔,竟坐甚得任樂,後王逸少下,與丞相言及此事,丞相曰:「元規爾時風範,不得不小穨。」右軍答曰:「唯丘壑獨存。」}{\footnotesize \textbf{孫綽庾亮碑文}曰公雅好所託,常在塵垢之外,雖柔心應世,蠖屈其迹,而方寸湛然,固以玄對山水。}

\subsection*{25}

\textbf{王敬豫有美形,問訊王公,王公撫其肩曰:「阿奴,恨才不稱!」又云:「敬豫事事似王公。」}{\footnotesize \textbf{語林}曰謝公云「小時在殿廷會見丞相,便覺清風來拂人」。}

\subsection*{26}

\textbf{王右軍見杜弘治,歎曰:「面如凝脂,眼如點漆,此神仙中人。」}{\footnotesize \textbf{江左名士傳}曰永和中,劉真長、謝仁祖共商略中朝人士,或曰「杜弘治清標令上,為後來之美,又面如凝脂,眼如點漆,粗可得方諸衛玠」。}\textbf{時人有稱王長史形者,蔡公曰:「恨諸人不見杜弘治耳。」}

\subsection*{27}

\textbf{劉尹道桓公:「鬢如反猬皮,眉如紫石稜,自是孫仲謀、司馬宣王一流人。」}{\footnotesize \textbf{宋明帝文章志}曰溫為溫嶠所賞,故名溫。\textbf{吳志}曰孫權,字仲謀,策弟也,漢使者劉琬語人曰「吾觀孫氏兄弟,雖並有才秀明達,皆祿祚不終,唯中弟孝廉,形貌魁偉,骨體不恆,有大貴之表」。\textbf{晉陽秋}曰宣王天姿傑邁,有英雄之略。}

\subsection*{28}

\textbf{王敬倫風姿似父,作侍中,加授桓公公服,從大門入,桓公望之曰:「大奴固自有鳳毛。」}{\footnotesize 大奴,王劭也,已見。\textbf{中興書}曰劭美姿容,持儀也。}

\subsection*{29}

\textbf{林公道王長史:「斂衿作一來,何其軒軒韶舉。」}{\footnotesize \textbf{語林}曰王仲祖有好儀形,每覽鏡自照,曰「王文開那生如馨兒」,時人謂之達也。}

\subsection*{30}

\textbf{時人目王右軍:「飄如遊雲,矯若驚龍。」}

\subsection*{31}

\textbf{王長史嘗病,親疏不通,林公來,守門人遽啓之曰:「一異人在門,不敢不啓。」王笑曰:「此必林公。」}{\footnotesize \textbf{按}語林曰「諸人嘗要阮光祿共詣林公,阮曰『欲聞其言,惡見其面』」,此則林公之形,信當醜異。}

\subsection*{32}

\textbf{或以方謝仁祖不乃重者,桓大司馬曰:「諸君莫輕道,仁祖企腳北窻下彈琵琶,故自有天際真人想。」}{\footnotesize \textbf{晉陽秋}曰尚善音樂。\textbf{裴子}云丞相嘗曰「堅石挈腳枕琵琶,有天際想」。堅石,尚小名。}

\subsection*{33}

\textbf{王長史為中書郎,往敬和許,}{\footnotesize 敬和,王洽,已見。}\textbf{爾時積雪,長史從門外下車,步入尚書省,敬和遙望,歎曰:「此不復似世中人。」}

\subsection*{34}

\textbf{簡文作相王時,與謝公共詣桓宣武,王珣先在內,桓語王:「卿嘗欲見相王,可住帳裏。」二客既去,桓謂王曰:「定何如?」王曰:「相王作輔,自然湛若神君,}{\footnotesize \textbf{續晉陽秋}曰帝美風姿,舉止端詳。}\textbf{公亦萬夫之望,不然,僕射何得自沒?」}{\footnotesize 僕射,謝安。}

\subsection*{35}

\textbf{海西時,諸公每朝,朝堂猶暗,唯會稽王來,軒軒如朝霞舉。}

\subsection*{36}

\textbf{謝車騎道謝公:「遊肆復無乃高唱,但恭坐捻鼻顧睞,便自有寢處山澤間儀。」}

\subsection*{37}

\textbf{謝公云:「見林公雙眼,黯黯明黑。」孫興公:「見林公稜稜露其爽。」}

\subsection*{38}

\textbf{庾長仁與諸弟入吳,欲住亭中宿,諸弟先上,見群小滿屋,都無相避意,長仁曰:「我試觀之。」乃策杖將一小兒,始入門,諸客望其神姿,一時退匿。}{\footnotesize 長仁已見。一說是庾亮。}

\subsection*{39}

\textbf{有人歎王恭形茂者,云:「濯濯如春月柳。」}