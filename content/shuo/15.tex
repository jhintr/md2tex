\chapter{自新第十五}

\subsection*{1}

\textbf{周處年少時,兇彊俠氣,為鄉里所患,}{\footnotesize \textbf{處別傳}曰處,字子隱,吳郡陽羡人,父魴,吳鄱陽太守,處少孤,不治細行。\textbf{晉陽秋}曰處輕果薄行,州郡所棄。}\textbf{又義興水中有蛟,山中有邅跡}{\footnotesize 一作白額。}\textbf{虎,並皆暴犯百姓,義興人謂為三橫,而處尤劇,或說處殺虎斬蛟,實冀三橫唯餘其一,處即刺殺虎,又入水擊蛟,蛟或浮或沒,行數十里,處與之俱,經三日三夜,鄉里皆謂已死,更相慶,竟殺蛟而出,聞里人相慶,始知為人情所患,有自改意,}{\footnotesize \textbf{孔氏志怪}曰義興有邪足虎,溪渚長橋有蒼蛟,並大噉人,郭西周,時謂郡中三害。周即處也。}\textbf{乃入吳尋二陸,平原不在,正見清河,具以情告,並云:「欲自修改,而年已蹉跎,終無所成。」清河曰:「古人貴朝聞夕死,況君前途尚可,且人患志之不立,亦何憂令名不彰邪?」處遂改勵,終為忠臣孝子。}{\footnotesize \textbf{晉陽秋}曰處仕晉為御史中丞,多所彈糺,氐人齊萬年反,乃令處距萬年,伏波孫秀欲表處母老,處曰「忠孝之道,何當得兩全」,乃進戰,斬首萬計,弦絕矢盡,左右勸退,處曰「此是吾授命之日」,遂戰而沒。}

\subsection*{2}

\textbf{戴淵少時,遊俠不治行檢,嘗在江淮間攻掠商旅,陸機赴假還洛,輜重甚盛,淵使少年掠劫,淵在岸上,據胡床,指麾左右,皆得其宜,淵既神姿峰穎,雖處鄙事,神氣猶異,機於船屋上遙謂之曰:「卿才如此,亦復作劫邪?」淵便泣涕,投劍歸機,辭厲非常,機彌重之,定交,作筆荐焉,}{\footnotesize \textbf{虞預晉書}曰機薦淵於趙王倫曰「蓋聞繁弱登御,然後高墉之功顯,孤竹在肆,然後降神之曲成,伏見處士戴淵,砥節立行,有井渫之潔,安窮樂志,無風塵之慕,誠東南之遺寶,朝廷之貴璞也,若得寄跡康衢,必能結軌驥騄,耀質廊廟,必能垂光瑜璠,夫枯岸之民,果於輸珠,潤山之客,烈於貢玉,蓋明暗呈形,則庸識所甄也」,倫即辟淵。}\textbf{過江,仕至征西將軍。}