\chapter{政事第三}

\subsection*{1}

\textbf{陳仲弓為太丘長,時吏有詐稱母病求假,事覺收之,令吏殺焉,主簿請付獄,考眾姦,仲弓曰:「欺君不忠,病母不孝,不忠不孝,其罪莫大,考求眾姦,豈復過此?」}{\footnotesize 陳寔已別見。}

\subsection*{2}

\textbf{陳仲弓為太丘長,有劫賊殺財主,主者捕之,未至發所,道聞民有在草不起子者,回車往治之,主簿曰:「賊大,宜先按討。」仲弓曰:「盜殺財主,何如骨肉相殘?」}{\footnotesize \textbf{按}後漢時賈彪有此事,不聞寔也。}

\subsection*{3}

\textbf{陳元方年十一時,}{\footnotesize 陳紀已見。}\textbf{候袁公,袁公問曰:「賢家君在太丘,遠近稱之,何所履行?」元方曰:「老父在太丘,彊者綏之以德,弱者撫之以仁,恣其所安,久而益敬。」}{\footnotesize \textbf{袁宏漢紀}曰寔為太丘,其政不嚴而治,百姓敬之。}\textbf{袁公曰:「孤往者嘗為鄴令,正行此事,不知卿家君法孤,孤法卿父?」}{\footnotesize 檢眾漢書,袁氏諸公未知誰為鄴令,故闕其文以待通識者。}\textbf{元方曰:「周公、孔子異世而出,周旋動靜,萬里如一,周公不師孔子,孔子亦不師周公。」}

\subsection*{4}

\textbf{賀太傅作吳郡,初不出門,吳中諸強族輕之,乃題府門云:「會稽雞,不能啼。」}{\footnotesize \textbf{環濟吳紀}曰賀邵,字興伯,會稽山陰人,祖齊、父景並歷吳官,邵歷散騎常侍,出為吳郡太守,後遷太子太傅。}\textbf{賀聞故出行,至門反顧,索筆足之曰:「不可啼,殺吳兒。」於是至諸屯邸,檢校諸顧、陸役使官兵及藏逋亡,悉以事言上,罪者甚眾,陸抗時為江陵都督,}{\footnotesize \textbf{吳錄}曰抗,字幼節,吳郡人,丞相遜子、孫策外孫也,為江陵都督,累遷大司馬、荊州牧。}\textbf{故下請孫皓,然後得釋。}

\subsection*{5}

\textbf{山公以器重朝望,年踰七十,猶知管時任,}{\footnotesize \textbf{虞預晉書}曰山濤,字巨源,河內懷人,祖本,郡孝廉,父曜,冤句令,濤蚤孤而貧,少有器量,宿士猶不慢之,年十七,宗人謂宣帝曰「濤當與景、文共綱紀天下者也」,帝戲曰「卿小族,那得此快人邪」,好莊老,與嵇康善,為河內從事,與石鑒共傳宿,濤夜起蹋鑒曰「今何等時而眠也,知太傅臥何意」,鑒曰「宰相三日不朝,與尺一令歸第,君何慮焉」,濤曰「咄,石生!無事馬蹄間也」,投傳而去,果有曹爽事,遂隱身不交世務,累遷吏部尚書、僕射、太子少傅、司徒,年七十九薨,諡康侯。}\textbf{貴勝年少,若和、裴、王之徒並共言詠,有署閣柱曰:「閣東有大牛,和嶠鞅,裴楷鞦,王濟剔嬲不得休。」}{\footnotesize \textbf{王隱晉書}曰初,濤領吏部,潘岳內非之,密為作謠曰「閣東有大牛,王濟鞅,裴楷鞦,和嶠刺促不得休」。\textbf{竹林七賢論}曰濤之處選,非望路絕,故貽是言。}\textbf{或云潘尼作之。}{\footnotesize \textbf{文士傳}曰尼,字正叔,滎陽人,祖最,尚書左丞,父滿,平原太守,並以文學稱,尼少有清才,文詞溫雅,初應州辟,終太常卿。}

\subsection*{6}

\textbf{賈充初定律令,}{\footnotesize \textbf{晉諸公贊}曰充,字公閭,襄陵人,父逵,魏豫州刺史,充早知名,起家為尚書郎,遷廷尉,聽訟稱平,晉受禪,封魯郡公,充有才識,明達治體,加善刑法,由此與散騎常侍裴楷共定科令,蠲除密網,以為晉律,薨,贈太宰。}\textbf{與羊祜共咨太傅鄭沖,}{\footnotesize \textbf{王隱晉書}曰沖,字文和,滎陽開封人,有核練才,清虛寡欲,喜論經史,草衣縕袍,不以為憂,累遷司徒、太保,晉受禪,進太傅。}\textbf{沖曰:「皋陶嚴明之旨,非僕闇懦所探。」羊曰:「上意欲令小加弘潤。」沖乃粗下意。}{\footnotesize \textbf{續晉陽秋}曰初,文帝命荀勖、賈充、裴秀等分定禮儀律令,皆先咨鄭沖,然後施行也。}

\subsection*{7}

\textbf{山司徒前後選,殆周遍百官,舉無失才,凡所題目,皆如其言,惟用陸亮,是詔所用,與公意異,爭之不從,亮亦尋為賄敗。}{\footnotesize \textbf{晉諸公贊}曰亮,字長興,河內野王人,太常陸乂兄也,性高明而率至,為賈充所親待,山濤為左僕射,領選,濤行業即與充異,自以為世祖所敬,選用之事,與充咨論,充毎不得其所欲,好事者說充「宜授心腹人為吏部尚書,參同選舉,若意不齊,事不得諧,可不召公與選,而實得敘所懷」,充以為然,乃啓亮公忠無私,濤以亮將與己異,又恐其協情不允,累啓亮可為左丞,初非選官才,世祖不許,濤乃辭疾還家,亮在職果不能允,坐事免官。}

\subsection*{8}

\textbf{嵇康被誅後,山公舉康子紹為祕書丞,}{\footnotesize \textbf{山公啓事}曰詔選祕書丞,濤薦曰「紹平簡溫敏,有文思,又曉音,當成濟也,猶宜先作祕書郎」,詔曰「紹如此,便可為丞,不足復為郎也」。\textbf{晉諸公贊}曰康遇事後二十年,紹乃為濤所拔。\textbf{王隱晉書}曰時以紹父康被法,選官不敢舉,年二十八,山濤啓用之,世祖發詔,以為祕書丞。}\textbf{紹咨公出處,}{\footnotesize \textbf{竹林七賢論}曰紹懼不自容,將解褐,故咨之於濤。}\textbf{公曰:「為君思之久矣!天地四時,猶有消息,而況人乎?」}{\footnotesize \textbf{王隱晉書}曰紹,字延祖,雅有文才,山濤啓武帝云云。}

\subsection*{9}

\textbf{王安期為東海郡,}{\footnotesize \textbf{名士傳}曰王承,字安期,太原晉陽人,父湛,汝南太守,承沖淡寡欲,無所循尚,累遷東海內史,為政清靜,吏民懷之,避亂渡江,是時道路寇盜,人懷憂懼,承毎遇艱險,處之怡然,元皇為鎮東,引為從事中郎。}\textbf{小吏盜池中魚,綱紀推之,王曰:「文王之囿,與眾共之,}{\footnotesize \textbf{孟子}曰齊宣王問「文王之囿方七十里,有諸?若是其大乎」,對曰「民猶以為小也」,王曰「寡人之囿方四十里,民猶以為大,何邪」,孟子曰「文王之囿,芻蕘者往焉,與民同之,民以為小,不亦宜乎?今王之囿,殺麋鹿者如殺人罪,是以四十里為穽於國中也,民以為大,不亦宜乎」。}\textbf{池魚復何足惜?」}

\subsection*{10}

\textbf{王安期作東海郡,吏錄一犯夜人來,王問:「何處來?」云:「從師家受書還,不覺日晩。」王曰:「鞭撻甯越以立威名,恐非致理之本。」}{\footnotesize \textbf{呂氏春秋}曰甯越者,中牟鄙人也,苦耕稼之勞,謂其友曰「何為可以免此苦也」,其友曰「莫如學也,學三十歲則可以達矣」,甯越曰「請以十五歲,人將休,吾不敢休,人將臥,吾不敢臥」,學十五歲而為周威公之師也。}\textbf{使吏送令歸家。}

\subsection*{11}

\textbf{成帝在石頭,}{\footnotesize \textbf{晉世譜}曰帝諱衍,字世根,明帝太子,年二十二崩。}\textbf{任讓在帝前戮侍中鍾雅、}{\footnotesize \textbf{晉陽秋}曰讓,樂安人,諸任之後,隨蘇峻作亂。\textbf{雅別傳}曰雅,字彥胄,潁川長社人,魏太傅鍾繇弟仲常曾孫也,少有才志,累遷至侍中。}\textbf{右衛將軍劉超,}{\footnotesize \textbf{晉陽秋}曰超,字世踰,琅邪人,漢成陽景王六世孫,封臨沂慈鄉侯,遂家焉,父徵,為琅邪國上將軍,超為縣小吏,稍遷記室掾、安東舍人,忠清慎密,為中宗所拔,自以職在中書,絕不與人交關書疏,閉門不通賓客,家無儋石之儲,討王敦有功,封零陽伯,為義興太守,而受拜及往還朝,莫有知者,其慎默如此,遷右衛大將軍。}\textbf{帝泣曰:「還我侍中。」讓不奉詔,遂斬超、雅。}{\footnotesize \textbf{雅別傳}曰蘇峻逼主上幸石頭,雅與劉超並侍帝側匡衛,與石頭中人密期拔至尊出,事覺被害。}\textbf{事平之後,陶公與讓有舊,欲宥之,許柳}{\footnotesize \textbf{許氏譜}曰柳,字季祖,高陽人,祖允,魏中領軍,父猛,吏部郎。\textbf{劉謙之晉紀}曰柳妻,祖逖子渙女,蘇峻招祖約為逆,約遣柳以眾會,峻既克京師,拜丹陽尹,後以罪誅。}\textbf{兒思妣者至佳,諸公欲全之,}{\footnotesize \textbf{許氏譜}曰永,字思妣。}\textbf{若全思妣,則不得不為陶全讓,於是欲並宥之,事奏,帝曰:「讓是殺我侍中者,不可宥。」諸公以少主不可違,並斬二人。}

\subsection*{12}

\textbf{王丞相拜揚州,賓客數百人並加霑接,人人有說色,惟有臨海一客姓任}{\footnotesize \textbf{語林}曰任,名顒,時宦在都,預王公坐。}\textbf{及數胡人為未洽,公因便還到,過任邊,云:「君出,臨海便無復人。」任大喜說,因過胡人前彈指云:「蘭闍,蘭闍!」群胡同笑,四坐並懽。}{\footnotesize \textbf{晉陽秋}曰王導接誘應會,少有牾者,雖疎交常賓,一見多輸寫款誠,自謂為導所遇,同之舊暱。}

\subsection*{13}

\textbf{陸太尉詣王丞相咨事,過後輒翻異,王公怪其如此,後以問陸,}{\footnotesize \textbf{陸玩別傳}曰玩,字士瑤,吳郡吳人,祖瑁、父英仕郡有譽,玩器量淹雅,累遷侍中、尚書左僕射、尚書令,贈太尉。}\textbf{陸曰:「公長民短,臨時不知所言,既後覺其不可耳。」}

\subsection*{14}

\textbf{丞相嘗夏月至石頭看庾公,庾公正料事,丞相云:「暑可小簡之。」庾公曰:「公之遺事,天下亦未以為允。」}{\footnotesize \textbf{殷羡言行}曰王公薨後,庾冰代相,網密刑峻,羡時行,遇收捕者於途,慨然歎曰「丙吉問牛喘,似不爾」,嘗從容謂冰曰「卿輩自是網目不失,皆是小道小善耳,至如王公,故能行無理事」,謝安石毎歎詠此唱,庾赤玉曾問羡「王公治何似?詎是所長」,羡曰「其餘令績,不復稱論,然三捉三治,三休三敗」。}

\subsection*{15}

\textbf{丞相末年,略不復省事,正封籙諾之,自歎曰:「人言我憒憒,後人當思此憒憒。」}{\footnotesize \textbf{徐廣曆紀}曰導阿衡三世,經綸夷險,政務寬恕,事從簡易,故垂遺愛之譽也。}

\subsection*{16}

\textbf{陶公性檢厲,勤於事,}{\footnotesize \textbf{晉陽秋}曰侃練核庶事,勤務稼穡,雖戎陳武士皆勸厲之,有奉饋者,皆問其所由,若力役所致,懽喜慰賜,若他所得,則呵辱還之,是以軍民勤於農稼,家給人足,性纖密好問,頗類趙廣漢,嘗課營種柳,都尉夏施盜拔武昌郡西門所種,侃後自出,駐車施門,問「此是武昌西門柳,何以盜之」,施惶怖首伏,三軍稱其明察,侃勤而整,自強不息,又好督勸他人,常云「民生在勤,大禹聖人,猶惜寸陰,至於凡俗,當惜分陰,豈可遊逸,生無益於時,死無聞於後,是自棄也」,又「老莊浮華,非先王之法言而不敢行,君子當正其衣冠,攝以威儀,何有亂頭養望,自謂宏達邪」。\textbf{中興書}曰侃嘗檢校佐吏,若得樗蒲博弈之具,投之曰「樗蒲,老子入胡所作,外國戲耳,圍棊,堯舜以教愚子,博弈,紂所造,諸君國器,何以為此?若王事之暇,患邑邑者,文士何不讀書,武士何不射弓」,談者無以易也。}\textbf{作荊州時,敕船官悉錄鋸木屑,不限多少,咸不解此意,後正會,值積雪始晴,聽事前除雪後猶濕,於是悉用木屑覆之,都無所妨。官用竹皆令錄厚頭,積之如山,後桓宣武伐蜀,裝船,悉以作釘。又云嘗發所在竹篙,有一官長連根取之,仍當足,乃超兩階用之。}

\subsection*{17}

\textbf{何驃騎作會稽,}{\footnotesize \textbf{晉陽秋}曰何充,字次道,廬江人,思韻淹通,有文義才情,累遷會稽內史、侍中、驃騎將軍、揚州刺史,贈司徒。}\textbf{虞存弟謇作郡主簿,}{\footnotesize \textbf{孫統存誄敘}曰存,字道長,會稽山陰人也,祖陽,散騎常侍,父偉,州西曹,存幼而卓拔,風情高逸,歷衛軍長史、尚書吏部郎。\textbf{范汪棊品}曰謇,字道直,仕至郡功曹。}\textbf{以何見客勞損,欲斷常客,使家人節量,擇可通者,作白事成以見存,存時為何上佐,正與謇共食,語云:「白事甚好,待我食畢作教。」食竟,取筆題白事後云:「若得門庭長如郭林宗者,當如所白,}{\footnotesize \textbf{泰別傳}曰泰,字林宗,有人倫鑒識,題品海內之士,或在幼童,或在里肆,後皆成英彥六十餘人,自著書一卷,論取士之本,未行,遭亂亡失。}\textbf{汝何處得此人?」謇於是止。}

\subsection*{18}

\textbf{王、劉與林公共看何驃騎,驃騎看文書不顧之,}{\footnotesize \textbf{晉陽秋}曰何充與王濛、劉惔好尚不同,由此見譏於當世。}\textbf{王謂何曰:「我今故與林公來相看,望卿擺撥常務,應對共言,那得方低頭看此邪?」何曰:「我不看此,卿等何以得存?」諸人以為佳。}

\subsection*{19}

\textbf{桓公在荊州,全欲以德被江漢,恥以威刑肅物,}{\footnotesize \textbf{溫別傳}曰溫以永和元年自徐州遷荊州刺史,在州寬和,百姓安之。}\textbf{令史受杖,正從朱衣上過,桓式年少,從外來,}{\footnotesize 式,桓歆小字也。\textbf{桓氏譜}曰歆,字叔道,溫第三子,仕至尚書。}\textbf{云:「向從閣下過,見令史受杖,上捎雲根,下拂地足。」意譏不著,桓公云:「我猶患其重。」}

\subsection*{20}

\textbf{簡文為相,事動經年,然後得過,桓公甚患其遲,常加勸免,太宗曰:「一日萬機,那得速。」}{\footnotesize \textbf{尚書皋陶謨}一日萬機。\textbf{孔安國}曰幾,微也,言當戒懼萬事之微。}

\subsection*{21}

\textbf{山遐去東陽,王長史就簡文索東陽云:「承藉猛政,故可以和靜致治。」}{\footnotesize \textbf{東陽記}云遐,字彥林,河內人,祖濤,司徒,父簡,儀同三司,遐歷武陵王友、東陽太守。\textbf{江惇傳}曰山遐為東陽,風政嚴苛,多任刑殺,郡內苦之,惇隱東陽,以仁恕懷物,遐感其德,為微損威猛。}

\subsection*{22}

\textbf{殷浩始作揚州,}{\footnotesize \textbf{浩別傳}曰浩,字淵源,陳郡長平人,祖識,濮陽相,父羡,光祿勳,浩少有重名,仕至揚州刺史、中軍將軍。\textbf{中興書}曰建元初,庾亮兄弟、何充等相尋薨,太宗以撫軍輔政,徵浩為揚州,從民譽也。}\textbf{劉尹行,日小欲晩,便使左右取襆,人問其故,答曰:「刺史嚴,不敢夜行。」}

\subsection*{23}

\textbf{謝公時,兵厮逋亡,多近竄南塘下諸舫中,或欲求一時搜索,謝公不許,云:「若不容置此輩,何以為京都?」}{\footnotesize \textbf{續晉陽秋}曰自中原喪亂,民離本域,江左造創,豪族并兼,或客寓流離,名籍不立,太元中,外禦強氐,蒐簡民實,三吳頗加澄檢,正其里伍,其中時有山湖遁逸往來都邑者,後將軍安方接客,時人有於坐言宜糺舍藏之失者,安毎以厚德化物,去其煩細,又以強寇入境,不宜加動人情,乃答之云「卿所憂,在於客耳,然不爾,何以為京都」,言者有慚色。}

\subsection*{24}

\textbf{王大為吏部郎,}{\footnotesize 王忱已見。}\textbf{嘗作選草,臨當奏,王僧彌來,聊出示之,}{\footnotesize 僧彌,王珉小字也。\textbf{珉別傳}曰珉,字季琰,琅邪人,丞相導孫,中領軍洽少子,有才蓺,善行書,名出兄珣右,累遷侍中、中書令,贈太常。}\textbf{僧彌得便以己意改易所選者近半,王大甚以為佳,更寫即奏。}

\subsection*{25}

\textbf{王東亭與張冠軍善,}{\footnotesize 張玄已見。}\textbf{王既作吳郡,人問小令曰:}{\footnotesize \textbf{續晉陽秋}曰王獻之為中書令,王珉代之,時人曰大、小王令。}\textbf{「東亭作郡,風政何似?」答曰:「不知治化何如,惟與張祖希情好日隆耳。」}

\subsection*{26}

\textbf{殷仲堪當之荊州,王東亭問曰:「德以居全為稱,仁以不害物為名,方今宰牧華夏,處殺戮之職,與本操將不乖乎?」殷答曰:「皋陶造刑辟之制,不為不賢,}{\footnotesize \textbf{古史考}曰庭堅,號曰皋陶,舜謀臣也,舜舉之於堯,堯令作士,主刑。}\textbf{孔丘居司寇之任,未為不仁。」}{\footnotesize \textbf{家語}曰孔子自魯司空為大司寇,三日而誅亂法大夫少正卯。}