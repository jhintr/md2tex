\chapter{規箴第十}

\subsection*{1}

\textbf{漢武帝乳母嘗於外犯事,帝欲申憲,乳母求救東方朔,}{\footnotesize \textbf{漢書}曰朔,字曼倩,平原厭次人。\textbf{朔別傳}曰朔,南陽步廣里人。\textbf{列仙傳}云朔是楚人,武帝時上書說便宜,拜郎中,宣帝初,棄官而去,共謂歲星也。}\textbf{朔曰:「此非脣舌所爭,爾必望濟者,將去時但當屢顧帝,慎勿言,此或可萬一冀耳。」乳母既至,朔亦侍側,因謂曰:「汝癡耳!帝豈復憶汝乳哺時恩邪?」帝雖才雄心忍,亦深有情戀,乃悽然愍之,即敕免罪。}{\footnotesize \textbf{史記滑稽傳}曰漢武帝少時,東武侯母嘗養帝,後號大乳母,其子孫從奴橫暴長安中,當道奪人衣物,有司請徙乳母於邊,奏可,乳母入辭,帝所幸倡郭舍人發言陳辭,雖不合大道,然令人主和說,乳母乃先見,為下泣,舍人曰「即入辭,勿去,數還顧」,乳母如其言,舍人疾言罵之曰「咄,老女子!何不疾行,陛下已壯矣,寧尚須乳母活邪?尚何還顧邪」,於是人主憐之,詔止毋徙,罰請者。}

\subsection*{2}

\textbf{京房與漢元帝共論,因問帝:「幽厲之君何以亡,所任何人?」答曰:「其任人不忠。」房曰:「知不忠而任之,何邪?」曰:「亡國之君,各賢其臣,豈知不忠而任之?」房稽首曰:「將恐今之視古,亦猶後之視今也。」}{\footnotesize \textbf{漢書}曰京房,字君明,東郡頓丘人,尤好鍾律,知音聲,以孝廉為郎,是時中書令石顯專權,及友人五鹿充宗為尚書令,與房同經,論議相是非,而此二人用事,房嘗宴見,問上曰「幽厲之君何以亡,所任何人」,上曰「君亦不明,而臣巧佞」,房曰「知其巧佞而任之邪,將以為賢邪」,上曰「賢之」,房曰「然則今何以知其不賢」,上曰「以其時亂而君危知之」,房曰「是任賢而理,任不肖而亂,自然之道也,幽厲何不覺悟而蚤納賢,何為卒任不肖以至亡」,於是上曰「亂亡之君,各賢其臣,令皆覺悟,安得亂亡之君」,房曰「齊桓、二世何不以幽厲疑之,而任豎刁、趙高,政治日亂邪」,上曰「唯有道者能以往知來耳」,房曰「自陛下即位,盜賊不禁,刑人滿市」云云,問上曰「今治也、亂也」,上曰「然愈於彼」,房曰「前二君皆然,臣恐後之視今,猶今之視前也」,上曰「今為亂者誰」,房曰「上所親與圖事帷幄中者」,房指謂石顯及充宗,顯等乃建言,宜試房以郡守,遂以房為東郡,顯發其私事,坐棄市。}

\subsection*{3}

\textbf{陳元方遭父喪,哭泣哀慟,軀體骨立,其母愍之,竊以錦被蒙上,郭林宗弔而見之,謂曰:「卿海內之儁才,四方是則,如何當喪,錦被蒙上?孔子曰『衣夫錦也,食夫稻也,於汝安乎』,}{\footnotesize \textbf{論語}曰宰我問「三年之喪,朞已久矣」,子曰「食夫稻,衣夫錦,於汝安乎?夫君子居喪,食旨不甘,聞樂不樂,居處不安,故不為也,今汝安,則為之」。}\textbf{吾不取也。」奮衣而去,自後賓客絕百所日。}{\footnotesize 所,一作許。}

\subsection*{4}

\textbf{孫休好射雉,至其時則晨去夕反,群臣莫不止諫:「此為小物,何足甚躭?」休曰:「雖為小物,耿介過人,朕所以好之。」}{\footnotesize \textbf{環濟吳紀}曰休,字子烈,吳大帝第六子,初封琅邪王,夢乘龍上天,顧不見尾,孫琳廢少主,迎休立之,銳意典籍,欲畢覽百家之事,頗好射雉,至春,晨出莫反,唯此時舍書,崩,諡景皇帝。\textbf{條列吳事}曰休在位烝烝,無有遺事,唯射雉可譏。}

\subsection*{5}

\textbf{孫皓問丞相陸凱曰:「卿一宗在朝有幾人?」陸答曰:「二相、五侯、將軍十餘人。」皓曰:「盛哉!」陸曰:「君賢臣忠,國之盛也,父慈子孝,家之盛也,今政荒民弊,覆亡是懼,臣何敢言盛?」}{\footnotesize \textbf{吳錄}曰凱,字敬風,吳人,丞相遜族子,忠鯁有大節,篤志好學,初為建忠校尉,雖有軍事,手不釋卷,累遷左丞相,時後主暴虐,凱正直彊諫,以其宗族彊盛,不敢加誅也。}

\subsection*{6}

\textbf{何晏、鄧颺令管輅作卦,云:「不知位至三公不?」卦成,輅稱引古義,深以戒之,颺曰:「此老生之常談。」}{\footnotesize \textbf{輅別傳}曰輅,字公明,平原人也,明周易,聲發徐州,冀州刺史裴徽舉秀才,謂曰「何、鄧二尚書有經國才略,於物理無不精也,何尚書神明清徹,殆破秋豪,君當慎之,自言不解易中九事,必當相問,比至洛,宜善精其理」,輅曰「若九事皆至義,不足勞思,若陰陽者,精之久矣」,輅至洛陽,果為何尚書問,九事皆明,何曰「君論陰陽,此世無雙也」,時鄧尚書在,曰「此君善易,而語初不論易中辭義,何邪」,輅答曰「夫善易者,不論易也」,何尚書含笑贊之曰「可謂要言不煩也」,因謂輅曰「聞君非徒善論易,至於分蓍思爻,亦為神妙,試為作一卦,知位當至三公不?又頃夢青蠅數十來鼻頭上,驅之不去,有何意故」,輅曰「鴟鴞,天下賤鳥也,及其在林食桑椹,則懷我好音,況輅心過草木,注情葵藿,敢不盡忠?唯察之爾,昔元、凱之相重華,宣慈惠和,仁義之至也,周公之翼成王,坐以待旦,敬慎之至也,故能流光六合,萬國咸寧,然後據鼎足而登金鉉,調陰陽而濟兆民,此履道之休應,非卜筮之所明也,今君侯位重山岳,勢若雷霆,望雲赴景,萬里馳風,而懷德者少,畏威者眾,殆非小心翼翼、多福之士,又鼻者,艮也,此天中之山,高而不危,所以長守貴也,今青蠅臭惡之物而集之焉,位峻者顛,輕豪者亡,必至之分也,夫變化雖相生,極則有害,虛滿雖相受,溢則有竭,聖人見陰陽之性,明存亡之理,損益以為衰,抑進以為退,是故山在地中曰謙,雷在天上曰大壯,謙則裒多益寡,大壯則非禮不履,伏願君侯上尋文王六爻之旨,下思尼父彖象之義,則三公可決,青蠅可驅」,鄧曰「此老生之常談」,輅曰「夫老生者見不生,常談者見不談也」。}\textbf{晏曰:「知幾其神乎,古人以為難,交疎吐誠,今人以為難,今君一面盡二難之道,可謂『明德惟馨』,詩不云乎,『中心藏之,何日忘之』。」}{\footnotesize \textbf{名士傳}曰是時曹爽輔政,識者慮有危機,晏有重名,與魏姻戚,內雖懷憂,而無復退也,著五言詩以言志,曰「鴻鵠比翼遊,群飛戲太清,常畏大網羅,憂禍一旦并,豈若集五湖,從流唼浮萍,永寧曠中懷,何為怵惕驚」,蓋因輅言,懼而賦詩。}

\subsection*{7}

\textbf{晉武帝既不悟太子之愚,必有傳後意,諸名臣亦多獻直言,帝嘗在陵雲臺上坐,衛瓘在側,欲申其懷,因如醉跪帝前,以手撫牀曰:「此坐可惜。」帝雖悟,因笑曰:「公醉邪?」}{\footnotesize \textbf{晉陽秋}曰初,惠帝之為太子,咸謂不能親政事,衛瓘每欲陳啓廢之而未敢也,後因會醉,遂跪牀前曰「臣欲有所啓」,帝曰「公所欲言者,何邪」,瓘欲言而復止者三,因以手撫牀曰「此坐可惜」,帝意乃悟,因謬曰「公真大醉也」,帝後悉召東宮官屬大會,令左右齎尚書處事以示太子,令處決,太子不知所對,賈妃以問外人,代太子對,多引古詞義,給使張弘曰「太子不學,陛下所知,宜以見事斷,不宜引書也」,妃從之,弘具草奏,令太子書呈,帝大說,以示瓘,於是賈充語妃曰「衛瓘老奴,幾敗汝家」,妃由是怨瓘,後遂誅之。}

\subsection*{8}

\textbf{王夷甫婦,郭泰寧女,}{\footnotesize \textbf{晉諸公贊}曰郭豫,字太寧,太原人,仕至相國參軍,知名,早卒。}\textbf{才拙而性剛,聚斂無厭,干豫人事,夷甫患之而不能禁,時其鄉人幽州刺史李陽,京都大俠,}{\footnotesize \textbf{晉百官名}曰陽,字景祖,高平人,武帝時為幽州刺史。\textbf{語林}曰陽性遊俠,盛暑,一日詣數百家別,賓客與別,常填門,遂死於几下,故懼之。}\textbf{猶漢之樓護,}{\footnotesize \textbf{漢書遊俠傳}曰護,字君卿,齊人,學經傳,甚得名譽,母死,送葬車三千兩,仕至天水太守。}\textbf{郭氏憚之,夷甫驟諫之,乃曰:「非但我言卿不可,李陽亦謂卿不可。」郭氏小為之損。}

\subsection*{9}

\textbf{王夷甫雅尚玄遠,常嫉其婦貪濁,口未嘗言錢字,}{\footnotesize \textbf{晉陽秋}曰夷甫善施舍,父時有假貸者,皆與焚券,未嘗謀貨利之事。\textbf{王隱晉書}曰夷甫求富貴得富貴,資財山積,用不能消,安須問錢乎?而世以不問為高,不亦惑乎。}\textbf{婦欲試之,令婢以錢遶牀,不得行,夷甫晨起,見錢閡行,謂婢曰:「舉卻阿堵物。」}

\subsection*{10}

\textbf{王平子年十四五,見王夷甫妻郭氏貪欲,令婢路上儋糞,平子諫之,並言不可,郭大怒,謂平子曰:「昔夫人臨終,以小郎囑新婦,不以新婦囑小郎。」}{\footnotesize \textbf{永嘉流人名}曰澄父乂,第三娶樂安任氏女,生澄。}\textbf{急捉衣裾,將與杖,平子饒力,爭得脫,踰窗而走。}

\subsection*{11}

\textbf{元帝過江猶好酒,王茂弘與帝有舊,常流涕諫,帝許之,命酌酒一酣,從是遂斷。}{\footnotesize \textbf{鄧粲晉紀}曰上身服儉約,以先時務,性素好酒,將渡江,王導深以諫,帝乃令左右進觴,飲而覆之,自是遂不復飲,克己復禮,官修其方,而中興之業隆焉。}

\subsection*{12}

\textbf{謝鯤為豫章太守,從大將軍下至石頭,敦謂鯤曰:「余不得復為盛德之事矣。」鯤曰:「何為其然?但使自今已後,日亡日去耳。」}{\footnotesize \textbf{鯤別傳}曰鯤之諷切雅正,皆此類也。}\textbf{敦又稱疾不朝,鯤諭敦曰:「近者明公之舉,雖欲大存社稷,然四海之內,實懷未達,若能朝天子,使群臣釋然,萬物之心於是乃服,仗民望以從眾懷,盡沖退以奉主上,如斯,則勳侔一匡,名垂千載。」時人以為名言。}{\footnotesize \textbf{晉陽秋}曰鯤為豫章太守,王敦將肆逆,以鯤有時望,逼與俱行,既克京邑,將旋武昌,鯤曰「不就朝覲,鯤懼天下私議也」,敦曰「君能保無變乎」,對曰「鯤近日入覲,主上側席,遲得見公,宮省穆然,必無不虞之慮,公若入朝、鯤請侍從」,敦曰「正復殺君等數百,何損於時」,遂不朝而去。}

\subsection*{13}

\textbf{元皇帝時,廷尉張闓}{\footnotesize \textbf{葛洪富民塘頌}曰闓,字敬緒,丹陽人,張昭孫也。\textbf{中興書}曰闓,晉陵內史,甚有威德,轉至廷尉卿。}\textbf{在小市居,私作都門,蚤閉晚開,群小患之,詣州府訴,不得理,遂至檛登聞鼓,猶不被判,聞賀司空出至破岡,連名詣賀訴,}{\footnotesize \textbf{賀循別傳}曰循,字彥先,會稽山陰人,本姓慶,高祖純,避漢帝諱,改為賀氏,父劭,吳中書令,以忠正見害,循少嬰家禍,流放荒裔,吳平乃還,秉節高舉,元帝為安東,上循為吳國內史。}\textbf{賀曰:「身被徵作禮官,不關此事。」群小叩頭曰:「若府君復不見治,便無所訴。」賀未語,令且去,見張廷尉當為及之,張聞,即毀門,自至方山迎賀,賀出見辭之曰:「此不必見關,但與君門情,相為惜之。」張愧謝曰:「小人有如此,始不即知,蚤已毀壞。」}

\subsection*{14}

\textbf{郗太尉晚節好談,既雅非所經,而甚矜之,}{\footnotesize \textbf{中興書}曰鑒少好學博覽,雖不及章句,而多所通綜。}\textbf{後朝覲,以王丞相末年多可恨,每見,必欲苦相規誡,王公知其意,每引作它言,臨還鎮,故命駕詣丞相,丞相翹鬚厲色,上坐便言:「方當乖別,必欲言其所見。」意滿口重,辭殊不流,王公攝其次曰:「後面未期,亦欲盡所懷,願公勿復談。」郗遂大瞋,冰衿而出,不得一言。}

\subsection*{15}

\textbf{王丞相為揚州,遣八部從事之職,顧和時為下傳還,同時俱見,諸從事各奏二千石官長得失,至和獨無言,王問顧曰:「卿何所聞?」答曰:「明公作輔,寧使網漏吞舟,何緣采聽風聞,以為察察之政?」丞相咨嗟稱佳,諸從事自視缺然也。}

\subsection*{16}

\textbf{蘇峻東征沈充,}{\footnotesize \textbf{晉陽秋}曰充,字士居,吳興人,少好兵,諂事王敦,敦克京邑,以充為車騎將軍,領吳國內史,明帝伐王敦,充率眾就王含,謂其妻曰「男兒不建豹尾,不復歸矣」,敦死,充將吳儒斬首送京都。}\textbf{請吏部郎陸邁與俱,}{\footnotesize \textbf{陸碑}曰邁,字公高,吳郡人,器識清敏,風檢澄峻,累遷振威太守、尚書吏部郎。}\textbf{將至吳,密勑左右,令入閶門放火以示威,陸知其意,謂峻曰:「吳治平未久,必將有亂,若為亂階,請從我家始。」峻遂止。}

\subsection*{17}

\textbf{陸玩拜司空,}{\footnotesize \textbf{玩別傳}曰是時王導、郗鑒、庾亮相繼薨殂,朝野憂懼,以玩德望,乃拜司空,玩辭讓不獲,乃歎息謂朋友曰「以我為三公,是天下無人矣」,時人以為知言。}\textbf{有人詣之,索美酒,得,便自起,瀉著梁柱間地,祝曰:「當今乏才,以爾為柱石之用,莫傾人棟梁。」玩笑曰:「戢卿良箴。」}

\subsection*{18}

\textbf{小庾在荊州,公朝大會,問諸僚佐曰:「我欲為漢高、魏武何如?」}{\footnotesize 翼別見。\textbf{宋明帝文章志}曰庾翼名輩,豈應狂狷如此哉?時若有斯言,亦傳聞者之謬矣。}\textbf{一坐莫答,長史江虨曰:「願明公為桓文之事,不願作漢高、魏武也。」}

\subsection*{19}

\textbf{羅君章為桓宣武從事,}{\footnotesize \textbf{含別傳}曰刺史庾亮初命含為部從事,桓溫臨州,轉參軍。}\textbf{謝鎮西作江夏,往檢校之,}{\footnotesize \textbf{中興書}曰尚為建武將軍、江夏相。}\textbf{羅既至,初不問郡事,徑就謝數日,飲酒而還,桓公問有何事,君章云:「不審公謂謝尚何似人?」桓公曰:「仁祖是勝我許人。」君章云:「豈有勝公人而行非者,故一無所問。」桓公奇其意而不責也。}

\subsection*{20}

\textbf{王右軍與王敬仁、許玄度並善,二人亡後,右軍為論議更克,孔巖誡之曰:「明府昔與王、許周旋有情,及逝沒之後,無慎終之好,民所不取。」右軍甚愧。}

\subsection*{21}

\textbf{謝中郎在壽春敗,臨奔走,猶求玉帖鐙,太傅在軍,前後初無損益之言,爾日猶云:「當今豈須煩此?」}{\footnotesize 按萬未死之前,安猶未仕,高臥東山,又何肯輕入軍旅邪?世說此言,迂謬已甚。}

\subsection*{22}

\textbf{王大語東亭:「卿乃復論成不惡,那得與僧彌戲?」}{\footnotesize \textbf{續晉陽秋}曰珉有儁才,與兄珣並有名,而聲出珣右,故時人為之語曰「法護非不佳,僧彌難為兄」。}

\subsection*{23}

\textbf{殷覬病困,看人政見半面,殷荊州興晉陽之甲,}{\footnotesize \textbf{春秋公羊傳}曰晉趙鞅取晉陽之甲,以逐荀寅、士吉射,寅、吉射者,君側之惡人。}\textbf{往與覬別,涕零,屬以消息所患,覬答曰:「我病自當差,正憂汝患耳。」}{\footnotesize \textbf{晉安帝紀}曰殷仲堪舉兵,覬弗與同,且以己居小任,唯當守局而已,晉陽之事,非所宜豫也,仲堪每邀之,覬輒曰「吾進不敢同,退不敢異」,遂以憂卒。}

\subsection*{24}

\textbf{遠公在廬山中,}{\footnotesize \textbf{豫章舊志}曰廬俗,字君孝,本姓匡,夏禹苗裔,東野王之子,秦末,百越君長與吳芮助漢定天下,野王亡軍中,漢八年,封俗鄢陽男,食邑茲部,印曰廬君,俗兄弟七人皆好道術,遂寓于洞庭之山,故世謂廬山,孝武元封五年,南巡狩,浮江,親覩神靈,乃封俗為大明公,四時秩祭焉。\textbf{遠法師廬山記}曰山在江州尋陽郡,左挾彭澤,右傍通川,有匡俗先生,出自殷周之際,遁世隱時,潛居其下,或云匡俗受道於仙人,而共遊其嶺,遂託室崖岫,即巖成館,故時人謂為神仙之廬而命焉。\textbf{法師遊山記}曰自託此山二十三載,再踐石門,四遊南嶺,東望香鑪峰,北眺九江,傳聞有石井方湖,中有赤鱗踊出,野人不能敘,直歎其奇而已矣。}\textbf{雖老,講論不輟,弟子中或有惰者,遠公曰:「桑榆之光,理無遠照,但願朝陽之暉,與時並明耳。」執經登坐,諷誦朗暢,詞色甚苦,高足之徒,皆肅然增敬。}

\subsection*{25}

\textbf{桓南郡好獵,每田狩,車騎甚盛,五六十里中,旌旗蔽隰,騁良馬,馳擊若飛,雙甄所指,不避陵壑,或行陳不整,麏兔騰逸,參佐無不被繫束,桓道恭,玄之族也,}{\footnotesize \textbf{桓氏譜}曰道恭,字祖猷,彝同堂弟也,父赤之,太學博士,道恭歷淮南太守、偽楚江夏相,義熙初伏誅。}\textbf{時為賊曹參軍,頗敢直言,常自帶絳綿繩著腰中,玄問:「此何為?」答曰:「公獵,好縛人士,會當被縛,手不能堪芒也。」玄自此小差。}

\subsection*{26}

\textbf{王緒、王國寶相為脣齒,並弄權要,}{\footnotesize \textbf{王氏譜}曰緒,字仲業,太原人,祖延,父乂,撫軍。\textbf{晉安帝紀}曰緒為會稽王從事中郎,以佞邪親幸,王珣、王恭惡國寶與緒亂政,與殷仲堪克期同舉,內匡朝廷,及恭表至,乃斬緒以說諸侯。國寶,平北將軍坦之第三子,太傅謝安,國寶婦父也,惡而抑之不用,安薨,相王輔政,遷中書令,有妾數百,從弟緒有寵於王,深為其說,國寶權動內外,王珣、王恭、殷仲堪為孝武所待,不為相王所眄,恭抗表討之,車胤又爭之,會稽王既不能拒諸侯兵,遂委罪國寶,付廷尉賜死。}\textbf{王大不平其如此,乃謂緒曰:「汝為此歘歘,曾不慮獄吏之為貴乎?」}{\footnotesize \textbf{史記}曰有上書告漢丞相欲反,文帝下之廷尉,勃既出,歎曰「吾嘗將百萬之軍,安知獄吏之為貴也」。}

\subsection*{27}

\textbf{桓玄欲以謝太傅宅為營,謝混曰:「召伯之仁,猶惠及甘棠,}{\footnotesize \textbf{韓詩外傳}曰昔周道之隆,召伯在朝,有司請召民,召伯曰「以一身勞百姓,非吾先君文王之志也」,乃暴處於棠下而聽訟焉,詩人見召伯休息之棠,美而歌之曰「蔽芾甘棠,勿翦勿伐,召伯所苃」。}\textbf{文靖之德,更不保五畝之宅。」玄慙而止。}