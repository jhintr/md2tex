\chapter{豳七月詁訓傳第十五}

\begin{quoting}\textbf{釋文}豳者,戎狄之地名也,夏道衰,后稷之曾孫公劉自邰而出居焉,其封域在雍州岐山之北、原隰之野,於漢屬右扶風郇邑,周公遭流言之難,居東都,思公劉、大王為豳公憂勞民事,以此敘己志而作七月、鴟鴞之詩,成王悟而迎之,以致大平,故大師述其詩,為豳國之風焉。\end{quoting}

\section{七月}

%{\footnotesize 八章、章十一句}

\textbf{七月,陳王業也。周公遭變,故陳后稷先公風化之所由、致王業之艱難也。}{\footnotesize 周公遭變者,管蔡流言,辟居東都。}

\textbf{七月流火,九月授衣。}{\footnotesize 火,大火也。流,下也。九月霜始降,婦功成,可以授冬衣矣。箋云大火者,寒暑之候也,火星中而寒暑退,故將言寒,先著火所在。}\textbf{一之日觱發,二之日栗烈,無衣無褐,何以卒歲。}{\footnotesize 一之日,十之餘也,一之日,周正月也。觱發,風寒也。二之日,殷正月也。栗烈,寒氣也。箋云褐,毛布也。卒,終也。此二正之月,人之貴者無衣,賤者無褐,將何以終歲乎,是故八月則當績也。}\textbf{三之日于耜,四之日舉趾,同我婦子,饁彼南畝,田畯至喜。}{\footnotesize 三之日,夏正月也,豳土晚寒。于耜,始修耒耜也。四之日,周四月也,民無不舉足而耕矣。饁,饋也。田畯,田大夫也。箋云同,猶俱也。喜,讀為饎,饎,酒食也。耕者之婦子俱以饟來至於南畝之中,其見田大夫又為設酒食焉,言勸其事,又愛其吏也。此章陳人以衣食為急,餘章廣而成之。}

\begin{quoting}火,即心宿二,夏曆五月現於南方,位置最高,六月後偏西下行。\textbf{馬瑞辰}凡言授衣者,皆授使為之也,蓋九月婦功成,絲麻之事已畢,始可為衣,非謂九月冬衣已成,遂以授人也。一之日,夏曆十一月也。觱 \texttt{bì}。饁 \texttt{yè}。\end{quoting}

\textbf{七月流火,九月授衣。}{\footnotesize 箋云將言女功之始,故又本於此。}\textbf{春日載陽,有鳴倉庚。女執懿筐,遵彼微行,爰求柔桑。}{\footnotesize 倉庚,離黃也。懿筐,深筐也。微行,牆下徑也。五畝之宅,樹之以桑。箋云載之言則也。陽,溫也。溫而倉庚又鳴,可蠶之候也。柔桑,稺桑也,蠶始生宜稺桑也。}\textbf{春日遲遲,采蘩祁祁。女心傷悲,殆及公子同歸。}{\footnotesize 遲遲,舒緩也。蘩,皤蒿也,所以生蠶。祁祁,眾多也。傷悲,感事苦也。春女悲,秋士悲,感其物化也。殆始、及與也。豳公子躬率其民,同時出、同時歸也。箋云春女感陽氣而思男,秋士感陰氣而思女,是其物化,所以悲也,悲則始有與公子同歸之志,欲嫁焉,女感事苦而生此志,是謂豳風。}

\begin{quoting}春日並下章蠶月,皆夏曆三月也。\end{quoting}

\textbf{七月流火,八月萑葦。}{\footnotesize 薍為萑,葭為葦,豫畜萑葦,可以為曲也。箋云將言女功自始至成,故亦又本於此。}\textbf{蠶月條桑,取彼斧斨,以伐遠揚,猗彼女桑。}{\footnotesize 斨,方銎也。遠,枝遠也。揚,條揚也。角而束之曰猗。女桑,荑桑也。箋云條桑,枝落之采其葉也。女桑,少枝,長條不枝落者,束而采之。}\textbf{七月鳴鵙,八月載績,載玄載黃,我朱孔陽,為公子裳。}{\footnotesize 鵙,伯勞也。載績,絲事畢而麻事起矣。玄,黑而有赤也。朱,深纁也。陽,明也。祭服玄衣纁裳。箋云伯勞鳴,將寒之候也,五月則鳴,豳地晚寒,鳥物之候從其氣焉。凡染者,春暴練,夏纁玄,秋染夏。為公子裳,厚於其所貴者說也。}

\begin{quoting}萑 \texttt{huán}。條,韓詩作挑,修剪。猗 \texttt{yī},同掎,\textbf{胡承珙}蓋女枝柔弱,不伐其條,但牽引使曲而采之。\end{quoting}

\textbf{四月秀葽,五月鳴蜩。八月其穫,十月隕蘀。}{\footnotesize 不榮而實曰秀。葽,葽草也。蜩,螗也。穫,禾可穫也。隕墜、蘀落也。箋云夏小正「四月王萯秀」,葽其是乎。秀葽也,鳴蜩也,穫禾也,隕蘀也,四者皆物成而將寒之候,物成自秀葽始。}\textbf{一之日于貉,取彼狐貍,為公子裘。}{\footnotesize 于貉,謂取狐貍皮也。狐貉之厚以居孟冬,天子始裘。箋云于貉,往搏貉以自為裘也,狐貍以共尊者,言此者,時寒宜助女功。}\textbf{二之日其同,載纘武功,言私其豵,獻豜于公。}{\footnotesize 纘繼、功事也。豕一歲曰豵,三歲曰豜,大獸公之,小獸私之。箋云其同者,君臣及民因習兵俱出田也,不用仲冬,亦豳地晚寒也。豕生三曰豵。}

\begin{quoting}葽 \texttt{yāo},今名遠志。說文「艸木凡皮葉落陊地為蘀 \texttt{tuò}」。貉 \texttt{hé}。\textbf{馬瑞辰}同之言會合也,謂冬田大合眾也。豜 \texttt{jiān}。\end{quoting}

\textbf{五月斯螽動股,六月莎雞振羽。七月在野,八月在宇,九月在戶,十月蟋蟀,入我牀下。}{\footnotesize 斯螽,蚣蝑也。莎雞羽成而振訊之。箋云自七月在野至十月入我牀下,皆謂蟋蟀也,言此三物之如此,著將寒有漸,非卒來也。}\textbf{穹窒熏鼠,塞向墐戶。}{\footnotesize 穹窮、窒塞也。向,北出牖也。墐,塗也。庶人蓽戶。箋云為此四者以備寒。}\textbf{嗟我婦子,曰為改歲,入此室處。}{\footnotesize 箋云曰為改歲者,歲終而「一之日觱發,二之日栗烈」,當避寒氣而入所穹窒墐戶之室而居之,至此而女功止。}

\begin{quoting}墐 \texttt{jìn}。曰,韓詩作聿,發語詞。\end{quoting}

\textbf{六月食鬱及薁,七月亨葵及菽。八月剝棗,十月穫稻,為此春酒,以介眉壽。}{\footnotesize 鬱,棣屬。薁,蘡薁也。剝,擊也。春酒,凍醪也。眉壽,豪眉也。箋云介,助也。既以鬱下及棗助男功,又穫稻而釀酒以助其養老之具,是謂豳雅。}\textbf{七月食瓜,八月斷壺,九月叔苴,采荼薪樗,食我農夫。}{\footnotesize 壺,瓠也。叔,拾也。苴,麻子也。樗,惡木也。箋云瓜瓠之畜、麻實之糝、乾荼之菜、惡木之薪,亦所以助男養農夫之具。}

\begin{quoting}薁 \texttt{yù}。剝,同撲。棗、稻皆以釀酒。\end{quoting}

\textbf{九月築場圃,}{\footnotesize 春夏為圃,秋冬為場。箋云場圃同地耳,物生之時,耕治之以種菜茹,至物盡成熟,築堅以為場。}\textbf{十月納禾稼,黍稷重穋,禾麻菽麥。}{\footnotesize 後熟曰重,先熟曰穋。箋云納,內也。治於場而內之囷倉也。}\textbf{嗟我農夫,我稼既同,上入執宮功。}{\footnotesize 入為上,出為下。箋云既同,言已聚也。可以上入都邑之宅,治宮中之事矣。於是時,男之野功畢也。}\textbf{晝爾于茅,宵爾索綯。}{\footnotesize 宵夜、綯絞也。箋云爾,女也,女當晝日往取茅歸,夜作絞索,以待時用。}\textbf{亟其乘屋,其始播百穀。}{\footnotesize 乘,升也。箋云亟急、乘治也。十月定星將中,急當治野廬之屋。其始播百穀,謂祈來年百穀于公社。}

\begin{quoting}重 \texttt{tóng},三家詩作種,即穜字。穋 \texttt{lù}。上,同尚。\textbf{王引之}索綯,猶言糾繩。\end{quoting}

\textbf{二之日鑿冰沖沖,三之日納于凌陰,四之日其蚤,獻羔祭韭。}{\footnotesize 冰盛水腹,則命取冰於山林。沖沖,鑿冰之意。凌陰,冰室也。箋云古者日在北陸而藏冰,西陸朝覿而出之,祭司寒而藏之,獻羔而啟之,其出之也,朝之祿位,賓食喪祭於是乎用之,月令「仲春,天子乃獻羔開冰,先薦寢廟」,周禮凌人之職「夏頒冰掌事,秋刷」。上章備寒,故此章備暑,后稷先公禮教備也。}\textbf{九月肅霜,十月滌場。朋酒斯饗,曰殺羔羊。}{\footnotesize 肅,縮也,霜降而收縮萬物。滌,埽也,場功畢入也。兩樽曰朋。饗者,鄉人飲酒也,鄉人以狗,大夫加以羔羊。箋云十月民事男女俱畢,無飢寒之憂,國君閒於政事而饗群臣。}\textbf{躋彼公堂,稱彼兕觥,萬壽無疆。}{\footnotesize 公堂,學校也。觥,所以誓眾也。疆,竟也。箋云於饗而正齒位,故因時而誓焉。飲酒既樂,欲大壽無竟,是謂豳頌。}

\begin{quoting}肅霜,即肅爽。滌場,即滌蕩。稱,同偁,爾雅釋言「偁,舉也」。\textbf{陳啟源}蓋七月詩歷言豳民農桑之事,於其畢也,終歲勤勤,乃得斗酒相勞。\end{quoting}

\section{鴟鴞}

%{\footnotesize 四章、章五句}

\textbf{鴟鴞,周公救亂也。成王未知周公之志,公乃為詩以遺王,名之曰鴟鴞焉。}{\footnotesize 未知周公之志者,未知其欲攝政之意。}

\textbf{鴟鴞鴟鴞,既取我子,無毀我室。}{\footnotesize 興也。鴟鴞,鸋鴂也。無能毀我室者,攻堅之故也,寧亡二子,不可以毀我周室。箋云重言鴟鴞者,將述其意之所欲言,丁寧之也。室,猶巢也。鴟鴞言已取我子者,幸無毀我巢,我巢積日累功,作之甚苦,故愛惜之也。時周公竟武王之喪,欲攝政,成周道,致太平之功,管叔蔡叔等流言云「公將不利於孺子」,成王不知其意而多罪其屬黨。興者,喻此諸臣乃世臣之子孫,其父祖以勤勞有此官位土地,今若誅殺之,無絕其位,奪其土地,王意欲誚公,此之由然。}\textbf{恩斯勤斯,鬻子之閔斯。}{\footnotesize 恩愛、鬻稚、閔病也。稚子,成王也。箋云鴟鴞之意殷勤於此,稚子當哀閔之。此取鴟鴞子者,指稚子也,以喻諸臣之先臣亦殷勤於此,成王亦宜哀閔之。}

\begin{quoting}恩,魯詩作殷,恩情即殷勤。鬻,通鞠,爾雅釋言「鞠,稚也」。\end{quoting}

\textbf{迨天之未陰雨,徹彼桑土,綢繆牖戶。}{\footnotesize 迨及、徹剝也。桑土,桑根也。箋云綢繆,猶纏綿也。此鴟鴞自說作巢至苦如是,以喻諸臣之先臣亦及文武未定天下,積日累功,以固定此官位與土地。}\textbf{今女下民,或敢侮予。}{\footnotesize 箋云我至苦矣,今女我巢下之民,寧有敢侮慢欲毀之者乎,意欲恚怒之,以喻諸臣之先臣固定此官位土地,亦不欲見其絕奪。}

\begin{quoting}徹,同撤。土,韓詩作杜。\end{quoting}

\textbf{予手拮据,予所捋荼,予所蓄租,予口卒瘏。}{\footnotesize 拮据,撠挶也。荼,萑苕也。租為、瘏病也。手病口病,故能免乎大鳥之難。箋云此言作之至苦,故能攻堅,人不得取其子。}\textbf{曰予未有室家。}{\footnotesize 謂我未有室家。箋云我作之至苦如是者,曰我未有室家之故。}

\begin{quoting}玉篇「拮据,手病也」。租,同蒩,茅藉也。捋荼與蓄租對言。\textbf{馬瑞辰}卒瘏與拮据相對成文,卒當讀為顇,字通作悴,卒瘏皆為病。\end{quoting}

\textbf{予羽譙譙,予尾翛翛。}{\footnotesize 譙譙,殺也。翛翛,敝也。箋云手口既病,羽尾又殺敝,言己勞苦甚。}\textbf{予室翹翹,風雨所漂搖,予維音嘵嘵。}{\footnotesize 翹翹,危也。嘵嘵,懼也。箋云巢之翹翹而危,以其所託枝條弱也,以喻今我子孫不肖,故使我家道危也。風雨,喻成王也。音嘵嘵然恐懼,告愬之意。}

\begin{quoting}\textbf{馬瑞辰}人面之焦枯曰䩌顇,鳥羽之焦殺曰譙譙 \texttt{qiáo},其義一也。翛 \texttt{xiāo},唐石經作脩。嘵 \texttt{xiāo}。\end{quoting}

\section{東山}

%{\footnotesize 四章、章十二句}

\textbf{東山,周公東征也。周公東征三年而歸,勞歸士,大夫美之,故作是詩也。一章言其完也,二章言其思也,三章言其室家之望女也,四章樂男女之得及時也。君子之於人,序其情而閔其勞,所以說也,說以使民,民忘其死,其唯東山乎。}{\footnotesize 成王既得金縢之書,親迎周公,周公歸攝政,三監及淮夷叛,周公乃東伐之,三年而後歸耳。分別章意者,周公於是志伸,美而詳之。}

\textbf{我徂東山,慆慆不歸。我來自東,零雨其濛。}{\footnotesize 慆慆,言久也。濛,雨貌。箋云此四句者,序歸士之情也,我往之東山既久勞矣,歸又道遇雨濛濛然,是尤苦也。}\textbf{我東曰歸,我心西悲。}{\footnotesize 公族有辟,公親素服,不舉樂,為之變,如其倫之喪。箋云我在東山,常曰歸也,我心則念西而悲。}\textbf{制彼裳衣,勿士行枚。}{\footnotesize 士事、枚微也。箋云勿,猶無也。女制彼裳衣而來,謂兵服也,亦初無行陳銜枚之事,言前定也,春秋傳曰「善用兵者不陳」。}\textbf{蜎蜎者蠋,烝在桑野。}{\footnotesize 蜎蜎,蠋貌。蠋,桑蟲也。烝,寘也。箋云蠋蜎蜎然特行,久處桑野,有似勞苦者。古者聲寘、填、塵同也。}\textbf{敦彼獨宿,亦在車下。}{\footnotesize 箋云敦敦然獨宿於車下,此誠有勞苦之心。}

\begin{quoting}慆 \texttt{tāo},三家詩作滔。零,雨落也。\textbf{馬瑞辰}蓋制其歸途所服之衣,非謂兵服。蜎 \texttt{yuān}。蠋,三家詩作蜀。敦 \texttt{duī}。\end{quoting}

\textbf{我徂東山,慆慆不歸。我來自東,零雨其濛。果臝之實,亦施于宇。伊威在室,蠨蛸在戶。町畽鹿場,熠燿宵行。}{\footnotesize 果臝,括樓也。伊威,委黍也。蠨蛸,長踦也。町畽,鹿跡也。熠燿,燐也,燐,螢火也。箋云此五物者,家無人則然,令人感思。}\textbf{不可畏也,伊可懷也。}{\footnotesize 箋云伊,當作繄,繄,猶是也。懷,思也。室中久無人,故有此五物,是不足可畏,乃可為憂思。}

\begin{quoting}臝 \texttt{luǒ}。蠨蛸 \texttt{xiāo shāo},一名喜蛛。町畽 \texttt{tǐng tuǎn}。\textbf{馬瑞辰}熠燿為螢光,與町畽為鹿跡相對成文,宵行與鹿場對文。\end{quoting}

\textbf{我徂東山,慆慆不歸。我來自東,零雨其濛。鸛鳴于垤,婦歎于室。洒埽穹窒,我征聿至。}{\footnotesize 垤,螘塚也,將陰雨則穴處,先知之矣。鸛好水,長鳴而喜也。箋云鸛,水鳥也,將陰雨則鳴。行者於陰雨尤苦,婦念之則歎於室也。穹窮、窒塞、洒灑、埽拚也。穹窒,鼠穴也。而我君子行役,述其日月,今且至矣,言婦望也。}\textbf{有敦瓜苦,烝在栗薪。}{\footnotesize 敦,猶專專也。烝,眾也。言我心苦,事又苦也。箋云此又言婦人思其君子之居處,專專如瓜之繫綴焉。瓜之瓣有苦者,以喻其心苦也。烝塵、栗析也。言君子又久見使析薪,於事尤苦也。古者聲栗、裂同。}\textbf{自我不見,于今三年。}

\begin{quoting}垤 \texttt{dié}。栗,即蓼字,苦菜。\end{quoting}

\textbf{我徂東山,慆慆不歸。我來自東,零雨其濛。}{\footnotesize 箋云凡先著此四句者,皆為序歸士之情。}\textbf{倉庚于飛,熠燿其羽。}{\footnotesize 箋云倉庚仲春而鳴,嫁取之候也。熠燿其羽,羽鮮明也。歸士始行之時,新合昏禮,今還,故極序其情以樂之。}\textbf{之子于歸,皇駁其馬。}{\footnotesize 黃白曰皇,駵白曰駁。箋云之子于歸,謂始嫁時也,皇駁其馬,車服盛也。}\textbf{親結其縭,九十其儀。}{\footnotesize 縭,婦人之褘也。母戒女,施衿結帨。九十其儀,言多儀也。箋云女歸,父母既戒之,庶母又申之。九十其儀,喻丁寧之多。}\textbf{其新孔嘉,其舊如之何。}{\footnotesize 言久長之道也。箋云嘉,善也。其新來時甚善,至今則久矣,不知其如何也,又極序其情,樂而戲之。}

\begin{quoting}皇,魯詩作騜。縭 \texttt{lí}。\end{quoting}

\section{破斧}

%{\footnotesize 三章、章六句}

\textbf{破斧,美周公也。周大夫以惡四國焉。}{\footnotesize 惡四國者,惡其流言毀周公也。}

\textbf{既破我斧,又缺我斨。}{\footnotesize 隋銎曰斧。斧斨,民之用也,禮義,國家之用也。箋云四國流言,既破毀我周公,又損傷我成王,以此二者為大罪。}\textbf{周公東征,四國是皇。}{\footnotesize 四國,管蔡商奄也。皇,匡也。箋云周公既反攝政,東伐此四國,誅其君罪,正其民人而已。}\textbf{哀我人斯,亦孔之將。}{\footnotesize 將,大也。箋云此言周公之哀我民人,其德亦甚大也。}

\textbf{既破我斧,又缺我錡。}{\footnotesize 鑿屬曰錡。}\textbf{周公東征,四國是吪。}{\footnotesize 吪,化也。}\textbf{哀我人斯,亦孔之嘉。}{\footnotesize 箋云嘉,善也。}

\begin{quoting}\textbf{陳喬樅}釜之有足者名錡,鏵之有齒者亦名錡。吪,魯詩作訛。\end{quoting}

\textbf{既破我斧,又缺我銶。}{\footnotesize 木屬曰銶。}\textbf{周公東征,四國是遒。}{\footnotesize 遒,固也。箋云遒,斂也。}\textbf{哀我人斯,亦孔之休。}{\footnotesize 休,美也。}

\begin{quoting}\textbf{胡承珙}銶 \texttt{qiú} 亦臿類,蓋起土之物,臿鍬不殊。遒,同揫,說文「揫,束也」。\end{quoting}

\section{伐柯}

%{\footnotesize 二章、章四句}

\textbf{伐柯,美周公也。周大夫刺朝廷之不知也。}{\footnotesize 成王既得雷雨大風之變,欲迎周公,而朝廷群臣猶惑於管蔡之言,不知周公之聖德,疑於王迎之禮,是以刺之。}

\begin{quoting}後世遂稱做媒為「伐柯」或「作伐」。\end{quoting}

\textbf{伐柯如何,匪斧不克。}{\footnotesize 柯,斧柄也,禮義者,亦治國之柄。箋云克,能也。伐柯之道,唯斧乃能之,此以類求其類也,以喻成王欲迎周公,當使賢者先往。}\textbf{取妻如何,匪媒不得。}{\footnotesize 媒,所以用禮也,治國不能用禮則不安。箋云媒者,能通二姓之言,定人室家之道,以喻王欲迎周公,當先使曉王與周公之意者又先往。}

\textbf{伐柯伐柯,其則不遠。}{\footnotesize 以其所願乎上交乎下,以其所願乎下事乎上,不遠求也。箋云則,法也。伐柯者必用柯,其大小長短近取法於柯,所謂不遠求也。王欲迎周公使還,其道亦不遠,人心足以知之。}\textbf{我覯之子,籩豆有踐。}{\footnotesize 踐,行列貌。箋云覯,見也。之子,是子也,斥周公也。王欲迎周公,當以饗燕之饌行,至則歡樂以說之。}

\section{九罭}

%{\footnotesize 四章、一章四句、三章章三句}

\textbf{九罭,美周公也。周大夫刺朝廷之不知也。}

\textbf{九罭之魚,鱒魴。}{\footnotesize 興也。九罭,緵罟,小魚之網也。鱒魴,大魚也。箋云設九罭之罟,乃後得鱒魴之魚,言取物各有器也,興者,喻王欲迎周公之來,當有其禮。}\textbf{我覯之子,袞衣繡裳。}{\footnotesize 所以見周公也。袞衣,卷龍也。箋云王迎周公,當以上公之服往見之。}

\begin{quoting}罭 \texttt{yù}。\end{quoting}

\textbf{鴻飛遵渚。}{\footnotesize 鴻不宜循渚也。箋云鴻,大鳥也。不宜與鳧鷖之屬飛而循渚,以喻周公今與凡人處東都之邑,失其所也。}\textbf{公歸無所,於女信處。}{\footnotesize 周公未得禮也。再宿曰信。箋云信,誠也。時東都之人欲周公留不去,故曉之云「公西歸而無所居,則可就女誠處是東都也」,今公當歸復其位,不得留也。}

\begin{quoting}\textbf{陳奐}女,猶爾也,爾,此也。\end{quoting}

\textbf{鴻飛遵陸。}{\footnotesize 陸非鴻所宜止。}\textbf{公歸不復,於女信宿。}{\footnotesize 宿,猶處也。}

\textbf{是以有袞衣兮,無以我公歸兮,}{\footnotesize 無與公歸之道也。箋云是,是東都也。東都之人欲周公留為之君,故云「是以有袞衣」,謂成王所賫來袞衣,願其封周公於此,以袞衣命留之,無以公西歸。}\textbf{無使我心悲兮。}{\footnotesize 箋云周公西歸而東都之人心悲,恩德之愛至深也。}

\begin{quoting}\textbf{聞一多}有,藏之也。以,使也。\end{quoting}

\section{狼跋}

%{\footnotesize 二章、章四句}

\textbf{狼跋,美周公也。周公攝政,遠則四國流言,近則王不知,周大夫美其不失其聖也。}{\footnotesize 不失其聖者,聞流言不惑,王不知不怨,終立其志,成周之王功,致太平,復成王之位,又為之大師,終始無愆,聖德著焉。}

\textbf{狼跋其胡,載疐其尾。}{\footnotesize 興也。跋躐、疐跲也。老狼有胡,進則躐其胡,退則跲其尾,進退有難,然而不失其猛。箋云興者,喻周公進則躐其胡,猶始欲攝政,四國流言,辟之而居東都也,退則跲其尾,謂後復成王之位而老,成王又留之,其如是,聖德無玷缺。}\textbf{公孫碩膚,赤舄几几。}{\footnotesize 公孫,成王也,豳公之孫也。碩大、膚美也。赤舄,人君之盛屨也。几几,絇貌。箋云公,周公也。孫,讀當如「公孫于齊」之孫,孫之言孫遁也。周公攝政七年,致太平,復成王之位,孫遁辟此成功之大美,欲老,成王又留之以為大師,履赤舄几几然。}

\begin{quoting}疐 \texttt{zhì},韓詩作躓。\textbf{馬瑞辰}碩膚者,心寬體胖之象。舄 \texttt{xì}。\end{quoting}

\textbf{狼疐其尾,載跋其胡。公孫碩膚,德音不瑕。}{\footnotesize 瑕,過也。箋云不瑕,言不可疵瑕也。}

%\begin{flushright}豳國七篇、二十七章、二百三句\end{flushright}