\chapter{周南關雎詁訓傳第一}

\begin{quoting}\textbf{釋文}周者代名,其地在禹貢雍州之域、岐山之陽,於漢屬扶風美陽縣,南者言周之德化自岐陽而先披南方,故序云「化自北而南也」,漢廣序又云「文王之道被於南國」是也,從關雎至騶虞二十五篇謂之正風。案二南者,左傳所謂漢陽諸姬也。\end{quoting}

\section{關雎}

%{\footnotesize 五章、章四句,故言三章、一章四句、二章章八句}

\textbf{關雎,后妃之德也,風之始也,所以風天下而正夫婦也,故用之鄉人焉,用之邦國焉。風,風也,教也,風以動之,教以化之。詩者,志之所之也,在心為志,發言為詩,情動於中而形於言,言之不足故嗟歎之,嗟嘆之不足故永歌之,永歌之不足,不知手之舞之足之蹈之也。情發於聲,聲成文,謂之音。}{\footnotesize 發,猶見也。聲,謂宮商角徵羽也。聲成文者,宮商上下相應。}\textbf{治世之音安以樂,其政和,亂世之音怨以怒,其政乖,亡國之音哀以思,其民困,故正得失、動天地、感鬼神,莫近於詩,先王以是經夫婦、成孝敬、厚人倫、美教化、移風俗。故詩有六義焉,一曰風,二曰賦,三曰比,四曰興,五曰雅,六曰頌。上以風化下,下以風刺上,主文而譎諫,言之者無罪,聞之者足以戒,故曰風。}{\footnotesize 風化、風刺皆謂譬喻,不斥言也。主文,主與樂之宮商相應也。譎諫,詠歌依違不直諫。}\textbf{至于王道衰、禮義廢、政教失、國異政、家殊俗,而變風變雅作矣,國史明乎得失之迹,傷人倫之廢,哀刑政之苛,吟詠情性,以風其上,達於事變而懷其舊俗者也,故變風發乎情,止乎禮義,發乎情,民之性也,止乎禮義,先王之澤也。是以一國之事,繫一人之本,謂之風,言天下之事,形四方之風,謂之雅。雅者,正也,言王政之所由廢興也。政有小大,故有小雅焉,有大雅焉。頌者,美盛德之形容,以其成功告於神明者也。是謂四始,詩之至也。}{\footnotesize 始者,王道興衰之所由。}\textbf{然則關雎麟趾之化,王者之風,故繫之周公,南言化自北而南也,鵲巢騶虞之德,諸侯之風也,先王之所以教,故繫之召公。}{\footnotesize 自,從也,從北而南,謂其化從岐周被江漢之域也。先王,斥大王王季。}\textbf{周南召南,正始之道,王化之基,是以關雎樂得淑女,以配君子,憂在進賢,不淫其色,哀窈窕、思賢才而無傷善之心焉,是關雎之義也。}{\footnotesize 哀,蓋字之誤也,當為衷,衷謂中心恕之。無傷善之心,謂好逑也。}

\textbf{關關雎鳩,在河之洲。}{\footnotesize 興也。關關,和聲也。雎鳩,王雎也,鳥摯而有別。水中可居者曰洲。后妃說樂君子之德,無不和諧,又不淫其色,慎固幽深,若雎鳩之有別焉,然後可以風化天下,夫婦有別則父子親,父子親則君臣敬,君臣敬則朝廷正,朝廷正則王化成。箋云摯之言至也,謂王雎之鳥雄雌情意至然而有別。}\textbf{窈窕淑女,君子好逑。}{\footnotesize 窈窕,幽閒也。淑善、逑匹也。言后妃有關雎之德,是幽閒貞專之善女,宜為君子之好匹。箋云怨耦曰仇。言后妃之德和諧,則幽閒處深宮貞專之善女能為君子和好眾妾之怨者,言皆化后妃之德,不嫉妬,謂三夫人以下。}

\textbf{參差荇菜,左右流之。}{\footnotesize 荇,接余也。流,求也。后妃有關雎之德,乃能共荇菜、備庶物,以事宗廟也。箋云左右,助也。言后妃將共荇菜之葅,必有助而求之者,言三夫人九嬪以下皆樂后妃之事。}\textbf{窈窕淑女,寤寐求之。}{\footnotesize 寤覺、寐寢也。箋云言后妃覺寐則常求此賢女,欲與之共己職也。}

\begin{quoting}流,同摎,廣雅「摎,捋也」。\end{quoting}

\textbf{求之不得,寤寐思服。}{\footnotesize 服,思之也。箋云服,事也。求賢女而不得,覺寐則思己職事當誰與共之乎。}\textbf{悠哉悠哉,輾轉反側。}{\footnotesize 悠,思也。箋云思之哉思之哉,言己誠思之。臥而不周曰輾。}

\begin{quoting}思,語詞。案後二章皆為此反側之思也。\end{quoting}

\textbf{參差荇菜,左右采之。}{\footnotesize 箋云言后妃既得荇菜,必有助而采之者。}\textbf{窈窕淑女,琴瑟友之。}{\footnotesize 宜以琴瑟友樂之。箋云同志為友。言賢女之助后妃共荇菜,其情意乃與琴瑟之志同,共荇菜之時樂必作。}

\textbf{參差荇菜,左右芼之。}{\footnotesize 芼,擇也。箋云后妃既得荇菜,必有助而擇之者。}\textbf{窈窕淑女,鍾鼓樂之。}{\footnotesize 德盛者宜有鍾鼓之樂。箋云琴瑟在堂,鍾鼓在庭,言共荇菜之時上下之樂皆作,盛其禮也。}

\section{葛覃}

%{\footnotesize 三章、章六句}

\textbf{葛覃,后妃之本也。后妃在父母家則志在於女功之事,躬儉節用,服澣濯之衣,尊敬師傅,則可以歸安父母,化天下以婦道也。}{\footnotesize 躬儉節用由於師傅之教,而後言尊敬師傅者,欲見其性亦自然。可以歸安父母,言嫁而得意,猶不忘孝。}

\textbf{葛之覃兮,施于中谷,維葉萋萋。}{\footnotesize 興也。覃,延也。葛所以為絺綌,女功之事煩辱者。施,移也。中谷,谷中也。萋萋,茂盛貌。箋云葛者,婦人之所有事也,此因葛之性以興焉。興者,葛延蔓于谷中,喻女在父母之家,形體浸浸日長大也,葉萋萋然,喻其容色美盛。}\textbf{黃鳥于飛,集于灌木,其鳴喈喈。}{\footnotesize 黃鳥,摶黍也。灌木,叢木也。喈喈,和聲之遠聞也。箋云葛延蔓之時則摶黍飛鳴,亦因以興焉,飛集叢木,興女有嫁于君子之道,和聲之遠聞,興女有才美之稱達於遠方。}

\begin{quoting}\textbf{馬瑞辰}詩以葛之生此而延彼,興女之自母家而適夫家。\end{quoting}

\textbf{葛之覃兮,施于中谷,維葉莫莫。}{\footnotesize 莫莫,成就之貌。箋云成就者,其可采用之時。}\textbf{是刈是濩,為絺為綌,服之無斁。}{\footnotesize 濩,煮之也。精曰絺,麄曰綌。斁,厭也。古者王后織玄紞,公侯夫人紘綖,卿之內子大帶,大夫命婦成祭服,士妻朝服,庶士以下各衣其夫。箋云服,整也。女在父母之家,未知將所適,故習之以絺綌煩辱之事,乃能整治之無厭倦,是其性貞專。}

\textbf{言告師氏,言告言歸。}{\footnotesize 言,我也。師,女師也。古者女師教以婦德、婦言、婦容、婦功,祖廟未毀,教于公宮三月,祖廟既毀,教于宗室。婦人謂嫁曰歸。箋云我告師氏者,我見教告于女師也,教告我以適人之道。重言我者,尊重師教也。公宮、宗室於族人皆為貴。}\textbf{薄汙我私,薄澣我衣。}{\footnotesize 汙,煩也。私,燕服也。婦人有副褘盛飾,以朝事舅姑,接見于宗廟,進見于君子,其餘則私也。箋云煩,煩撋之,用功深。澣,謂濯之耳。衣,謂褘衣以下至褖衣。}\textbf{害澣害否,歸寧父母。}{\footnotesize 害,何也。私服宜澣,公服宜否。寧,安也,父母在則有時歸寧耳。箋云我之衣服,今者何所當見澣乎,何所當否乎,言常自潔清以事君子。}

\begin{quoting}\textbf{陳奐}言、曰、云三字同義,有在句首者,為發聲,若漢廣之「言刈其楚」之類是也,有在句中者,為語助,若柏舟「靜言思之」之類是也。漢書翟義傳顏注「害,讀曰曷」。\end{quoting}

\section{卷耳}

%{\footnotesize 四章、章四句}

\textbf{卷耳,后妃之志也。又當輔佐君子,求賢審官,知臣下之勤勞,內有進賢之志而無險詖私謁之心,朝夕思念,至於憂勤也。}{\footnotesize 謁,請也。}

\textbf{采采卷耳,不盈頃筐。}{\footnotesize 憂者之興也。采采,事采之也。卷耳,苓耳也。頃筐,畚屬,易盈之器也。箋云器之易盈而不盈者,志在輔佐君子,憂思深也。}\textbf{嗟我懷人,寘彼周行。}{\footnotesize 懷思、寘置、行列也。思君子官賢人,置周之列位。箋云周之列位,謂朝廷臣也。}

\begin{quoting}\textbf{馬瑞辰}嗟為語詞,嗟我懷人,猶言我懷人也。\end{quoting}

\textbf{陟彼崔嵬,我馬虺隤。}{\footnotesize 陟,升也。崔嵬,土山之戴石者。虺隤,病也。箋云我,我使臣也。臣以兵役之事行出,離其列位,身勤勞於山險而馬又病,君子宜知其然。}\textbf{我姑酌彼金罍,維以不永懷。}{\footnotesize 姑,且也。人君黃金罍。永,長也。箋云我,我君也。臣出使,功成而反,君且當設饗燕之禮,與之飲酒以勞之,我則以是不復長憂思也。言且者,君賞功臣或多於此。}

\textbf{陟彼高岡,我馬玄黃。我姑酌彼兕觥,維以不永傷。}{\footnotesize 山脊曰岡。玄馬病則黃。兕觥,角爵也。傷,思也。箋云此章為意不盡,申殷勤也。觥,罰爵也,饗燕所以有之者,禮自立司正之後,旅酬必有醉而失禮者,罰之亦所以為樂。}

\textbf{陟彼砠矣,我馬瘏矣,我僕痡矣,云何吁矣。}{\footnotesize 石山戴土曰砠。瘏,病也。痡,亦病也。吁,憂也。箋云此章言臣既勤勞於外,僕馬皆病,而今云何乎其亦憂矣,深閔之辭。}

\section{樛木}

%{\footnotesize 三章、章四句}

\textbf{樛木,后妃逮下也。言能逮下而無嫉妬之心焉。}{\footnotesize 后妃能和諧眾妾,不嫉妬其容貌,恆以善言逮下而安之。}

\textbf{南有樛木,葛藟纍之。}{\footnotesize 興也。南,南土也。木下曲曰樛。南土之葛藟茂盛。箋云木枝以下垂之故,故葛也藟也得纍而蔓之而上下俱盛,興者,喻后妃能以惠下逮眾妾,使得其次序,則眾妾上附事之而禮義亦俱盛。南土,謂荊楊之域。}\textbf{樂只君子,福履綏之。}{\footnotesize 履祿、綏安也。箋云妃妾以禮義相與和,又能以禮樂樂其君子,使為福祿所安。}

\textbf{南有樛木,葛藟荒之。樂只君子,福履將之。}{\footnotesize 荒奄、將大也。箋云此章申殷勤之意。將,猶扶助也。}

\textbf{南有樛木,葛藟縈之。樂只君子,福履成之。}{\footnotesize 縈,旋也。成,就也。}

\section{螽斯}

%{\footnotesize 三章、章四句}

\textbf{螽斯,后妃子孫眾多也。言若螽斯不妬忌,則子孫眾多也。}{\footnotesize 忌,有所諱惡於人。}

\begin{quoting}案其初振作,及長戒慎,老而靜蟄,其詩之意歟。\end{quoting}

\textbf{螽斯羽,詵詵兮。}{\footnotesize 螽斯,蚣蝑也。詵詵,眾多也。箋云凡物有陰陽情欲者無不妬忌,維蚣蝑不耳,各得受氣而生子,故能詵詵然眾多,后妃之德能如是則宜然。}\textbf{宜爾子孫,振振兮。}{\footnotesize 振振,仁厚也。箋云后妃之德寬容不嫉妬,則宜女之子孫,使其無不仁厚。}

\begin{quoting}詵 \texttt{shēn}。\textbf{馬瑞辰}古文宜作㝖,竊謂宜从多聲,即有多義,宜爾子孫,猶云多爾子孫也。\end{quoting}

\textbf{螽斯羽,薨薨兮。宜爾子孫,繩繩兮。}{\footnotesize 薨薨,眾多也。繩繩,戒慎也。}

\begin{quoting}繩、慎雙聲通用,大雅下武「繩其祖武」,三家詩作慎字。\end{quoting}

\textbf{螽斯羽,揖揖兮。宜爾子孫,蟄蟄兮。}{\footnotesize 揖揖,會聚也。蟄蟄,和集也。}

\begin{quoting}爾雅釋詁「蟄,靜也」。\end{quoting}

\section{桃夭}

%{\footnotesize 三章、章四句}

\textbf{桃夭,后妃之所致也。不妬忌,則男女以正,昬姻以時,國無鰥民也。}{\footnotesize 老無妻曰鰥。}

\textbf{桃之夭夭,灼灼其華。}{\footnotesize 興也。桃,有華之盛者。夭夭,其少壯也。灼灼,華之盛也。箋云興者,喻時婦人皆得以年盛時行也。}\textbf{之子于歸,宜其室家。}{\footnotesize 之子,嫁子也。于,往也。宜以有室家無踰時者。箋云宜者,謂男女年時俱當。}

\begin{quoting}\textbf{馬瑞辰}宜與儀通,爾雅「儀,善也」。\end{quoting}

\textbf{桃之夭夭,有蕡其實。}{\footnotesize 蕡,實貌。非但有華色,又有婦德。}\textbf{之子于歸,宜其家室。}{\footnotesize 家室,猶室家也。}

\begin{quoting}\textbf{于省吾}有蕡其實,即有斑其實,桃實將熟,紅白相間,其實斑然。\end{quoting}

\textbf{桃之夭夭,其葉蓁蓁。}{\footnotesize 蓁蓁,至盛貌。有色有德,形體至盛也。}\textbf{之子于歸,宜其家人。}{\footnotesize 一家之人盡以為宜。箋云家人,猶室家也。}

\section{兔罝}

%{\footnotesize 三章、章四句}

\textbf{兔罝,后妃之化也。關雎之化行,則莫不好德,賢人眾多也。}

\textbf{肅肅兔罝,椓之丁丁。}{\footnotesize 肅肅,敬也。兔罝,兔罟也。丁丁,椓杙聲也。箋云罝兔之人,鄙賤之事,猶能恭敬,則是賢者眾多也。}\textbf{赳赳武夫,公侯干城。}{\footnotesize 赳赳,武貌。干,扞也。箋云干也城也,皆以禦難也,此罝兔之人賢者也,有武力,可任為將帥之德,諸侯可任以國守,扞城其民,折衝禦難於未然。}

\begin{quoting}肅,同縮。干,盾也。\end{quoting}

\textbf{肅肅兔罝,施于中逵。}{\footnotesize 逵,九逵之道。}\textbf{赳赳武夫,公侯好仇。}{\footnotesize 箋云怨耦曰仇。此罝兔之人,敵國有來侵伐者,可使和好之,亦言賢也。}

\textbf{肅肅兔罝,施于中林。}{\footnotesize 中林,林中。}\textbf{赳赳武夫,公侯腹心。}{\footnotesize 可以制斷,公侯之腹心。箋云此罝兔之人於行攻伐可用為策謀之臣,使之慮無,亦言賢也。}

\section{芣苢}

%{\footnotesize 三章、章四句}

\textbf{芣苢,后妃之美也。和平,則婦人樂有子矣。}{\footnotesize 天下和,政教平也。}

\textbf{采采芣苢,薄言采之。}{\footnotesize 采采,非一辭也。芣苢,馬舄,馬舄,車前也,宜懷妊焉。薄,辭也。采,取也。箋云薄言,我薄也。}\textbf{采采芣苢,薄言有之。}{\footnotesize 有,藏之也。}

\begin{quoting}\textbf{馬瑞辰}廣雅釋詁「有,取也」,孔子弟子冉求字有,正取名字相因,求與有皆取也,大雅瞻卬篇「人有土田,女反有之」,有之猶取之也。\end{quoting}

\textbf{采采芣苢,薄言掇之。}{\footnotesize 掇,拾也。}\textbf{采采芣苢,薄言捋之。}{\footnotesize 捋,取也。}

\begin{quoting}\textbf{胡承珙}掇是拾其子之既落者,捋是捋其子之未落者。\end{quoting}

\textbf{采采芣苢,薄言袺之。}{\footnotesize 袺,執衽也。}\textbf{采采芣苢,薄言襭之。}{\footnotesize 扱衽曰襭。}

\section{漢廣}

%{\footnotesize 三章、章八句}

\textbf{漢廣,德廣所及也。文王之道被于南國,美化行乎江漢之域,無思犯禮,求而不可得也。}{\footnotesize 紂時淫風徧于天下,維江漢之域先受文王之教化。}

\textbf{南有喬木,不可休息。漢有游女,不可求思。}{\footnotesize 興也。南方之木美喬上竦也。思,辭也。漢上游女,無求思者。箋云不可者,本有可道也,木以高其枝葉之故,故人不得就而止息也。興者,喻賢女雖出游流水之上,人無欲求犯禮者,亦由貞潔使之然。}\textbf{漢之廣矣,不可泳思。江之永矣,不可方思。}{\footnotesize 潛行為泳。永長、方泭也。箋云漢也江也,其欲渡之者,必有潛行乘泭之道,今以廣長之故,故不可也。又喻女之貞潔,犯禮而往,將不至也。}

\begin{quoting}息,韓詩作思。方言「泭謂之篺 \texttt{pái},篺謂之筏,筏,秦晉之通語也」。\end{quoting}

\textbf{翹翹錯薪,言刈其楚。}{\footnotesize 翹翹,薪貌。錯,雜也。箋云楚,雜薪之中尤翹翹者。我欲刈取之,以喻眾女皆貞潔,我又欲取其尤高潔者。}\textbf{之子于歸,言秣其馬。}{\footnotesize 秣,養也。六尺以上曰馬。箋云之子,是子也。謙不敢斥其適己,於是子之嫁我,願秣其馬,致禮餼,示有意焉。}\textbf{漢之廣矣,不可泳思。江之永矣,不可方思。}

\begin{quoting}\textbf{魏源}詩古微曰三百篇言取妻者,皆以析薪取興,蓋古者嫁娶必以燎炬為燭,故南山之析薪、綢繆之束薪、豳風之伐柯,皆與此錯薪、刈楚同興,秣馬、秣駒,即婚禮親迎御輪之禮。\end{quoting}

\textbf{翹翹錯薪,言刈其蔞。}{\footnotesize 蔞,草中之翹翹然。}\textbf{之子于歸,言秣其駒。}{\footnotesize 五尺以上曰駒。}\textbf{漢之廣矣,不可泳思。江之永矣,不可方思。}

\section{汝墳}

%{\footnotesize 三章、章四句}

\textbf{汝墳,道化行也。文王之化行乎汝墳之國,婦人能閔其君子,猶勉之以正也。}{\footnotesize 言此婦人被文王之化,厚事其君子。}

\textbf{遵彼汝墳,伐其條枚。}{\footnotesize 遵,循也。汝,水名也。墳,大防也。枝曰條,幹曰枚。箋云伐薪於汝水之側,非婦人之事,以言己之君子賢者而處勤勞之職,亦非其事。}\textbf{未見君子,惄如調饑。}{\footnotesize 惄,饑意也。調,朝也。箋云惄,思也。未見君子之時,如朝饑之思食。}

\begin{quoting}\textbf{馬瑞辰}墳,通作濆,方言「墳,地大也,青幽之間凡土而高且大者謂之墳」,李巡爾雅注「濆謂崖岸,狀如墳墓,名大防也」,是知水崖之濆與大防之墳為一。惄,韓詩作愵,方言「愵,憂也,秦晉之間凡志而不得、欲而不獲、高而有墜、得而中亡謂之溼,或謂之惄」。調,魯詩作朝。\end{quoting}

\textbf{遵彼汝墳,伐其條肄。}{\footnotesize 肄,餘也,斬而復生曰肄。}\textbf{既見君子,不我遐棄。}{\footnotesize 既已、遐遠也。箋云已見君子,君子反也,于己反得見之,知其不遠棄我而死亡,於思則愈,故下章而勉之。}

\textbf{魴魚頳尾,王室如燬。}{\footnotesize 赬,赤也,魚勞則尾赤。燬,火也。箋云君子仕於亂世,其顏色瘦病,如魚勞則尾赤,所以然者,畏王室之酷烈,是時紂存。}\textbf{雖則如燬,父母孔邇。}{\footnotesize 孔甚、邇近也。箋云辟此勤勞之處,或時得罪,父母甚近,當念之以免於害,不能為踈遠者計也。}

\section{麟之趾}

%{\footnotesize 三章、章三句}

\textbf{麟之趾,關雎之應也。關雎之化行,則天下無犯非禮,雖衰世之公子皆信厚如麟趾之時也。}{\footnotesize 關雎之時,以麟為應,後世雖衰,猶存關雎之化者,君之宗族猶尚振振然,有似麟應之時,無以過也。}

\textbf{麟之趾,振振公子,}{\footnotesize 興也。趾,足也。麟信而應禮,以足至者也。振振,信厚也。箋云興者,喻今公子亦信厚,與禮相應,有似於麟。}\textbf{于嗟麟兮。}{\footnotesize 于嗟,歎辭。}

\textbf{麟之定,振振公姓,}{\footnotesize 定,題也。公姓,公同姓。}\textbf{于嗟麟兮。}

\begin{quoting}定,魯詩作顁,同頂。儀禮特牲饋食禮「子姓兄弟」,鄭注「所祭者之子孫,言子姓者,子之所生」,賈疏「姓之言生也,云子之所生,則孫是也」。\end{quoting}

\textbf{麟之角,振振公族,}{\footnotesize 麟角,所以表其德也。公族,公同祖也。箋云麟角之末有肉,示有武而不用。}\textbf{于嗟麟兮。}

\begin{quoting}汾沮洳傳「公族,公屬」。\end{quoting}

%\begin{flushright}周南之國十一篇、三十六章、百五十九句\end{flushright}