\chapter{文王之什詁訓傳第二十三}

\begin{quoting}\textbf{釋文}自此以下至卷阿十八篇是文王武王成王周公之正大雅,據盛隆之時而推序天命,上述祖考之美,皆國之大事,故為正大雅焉。文王至靈臺八篇是文王之大雅,下武、文王有聲二篇是武王之大雅。\end{quoting}

\section{文王}

%{\footnotesize 七章、章八句}

\textbf{文王,文王受命作周也。}{\footnotesize 受命,受天命而王天下,制立周邦。}

\begin{quoting}案生民、公劉、緜、皇矣、文王、大明六首,乃周人之史詩也。\end{quoting}

\textbf{文王在上,於昭于天。}{\footnotesize 在上,在民上也。於,歎辭。昭,見也。箋云文王初為西伯,有功於民,其德著見於天,故天命之以為王,使君天下也,崩,謚曰文。}\textbf{周雖舊邦,其命維新。}{\footnotesize 乃新在文王也。箋云大王聿來胥宇而國於周,王迹起矣,而未有天命,至文王而受命,言新者,美之也。}\textbf{有周不顯,帝命不時。}{\footnotesize 有周,周也。不顯,顯也,顯,光也。不時,時也,時,是也。箋云周之德不光明乎,光明矣,天命之不是乎,又是矣。}\textbf{文王陟降,在帝左右。}{\footnotesize 言文王升接天,下接人也。箋云在,察也。文王能觀知天意,順其所為,從而行之。}

\begin{quoting}於 \texttt{wū}。\textbf{朱熹}是以周邦雖自后稷始封,千有餘年,而其受天命則自今始也。不,通丕。\textbf{馬瑞辰}時,當讀為承,時、承一聲之轉,承者,美大之詞,當讀「文王烝哉」之烝,釋文引韓詩曰「烝,美也」。\end{quoting}

\textbf{亹亹文王,令聞不已。陳錫哉周,侯文王孫子。文王孫子,本支百世。}{\footnotesize 亹亹,勉也。哉載、侯維也。本,本宗也。支,支子也。箋云令善、哉始、侯君也。勉勉乎不倦,文王之勤用明德也,其善聲聞日見稱歌無止時也,乃由能敷恩惠之施以受命造始周國,故天下君之,其子孫,適為天子,庶為諸侯,皆百世。}\textbf{凡周之士,不顯亦世。}{\footnotesize 不世顯德乎,士者世祿也。箋云凡周之士,謂其臣有光明之德者,亦得世世在位,重其功也。}

\begin{quoting}陳,同申,重複、一再。\textbf{朱熹}令聞不已,是以上帝敷錫于周。\textbf{王引之}不顯亦世,言其世之顯也,不與亦皆語詞耳。\end{quoting}

\textbf{世之不顯,厥猶翼翼。思皇多士,生此王國。王國克生,維周之楨。}{\footnotesize 翼翼,恭敬。思,辭也。皇天、楨幹也。箋云猶謀、思願也。周之臣既世世光明,其為君之謀事忠敬翼翼然,又願天多生賢人於此邦,此邦能生之,則是我周之幹事之臣。}\textbf{濟濟多士,文王以寧。}{\footnotesize 濟濟,多威儀也。}

\begin{quoting}猶,通猷,謀略。皇,美也。\end{quoting}

\textbf{穆穆文王,於緝熙敬止。假哉天命,有商孫子。}{\footnotesize 穆穆,美也。緝熙,光明也。假,固也。箋云穆穆乎文王,有天子之容,於美乎,又能敬其光明之德,堅固哉天為此命之,使臣有殷之子孫。}\textbf{商之孫子,其麗不億。上帝既命,侯于周服。}{\footnotesize 麗,數也。盛德不可為眾也。箋云于,於也。商之孫子,其數不徒億,多言之也,至天已命文王之後,乃為君於周之九服之中,言眾之不如德也。}

\begin{quoting}假,大也,\textbf{王先謙}漢書劉向傳引孔子讀此詩而釋之曰「大哉天命」,則假宜從爾雅訓大。侯,乃也,于周服,即服于周。\end{quoting}

\textbf{侯服于周,天命靡常。}{\footnotesize 則見天命之無常也。箋云無常者,善則就之,惡則去之。}\textbf{殷士膚敏,祼將于京。厥作祼將,常服黼冔。}{\footnotesize 殷士,殷侯也。膚美、敏疾也。祼,灌鬯也,周人尚臭。將行、京大也。黼,白與黑也。冔,殷冠也,夏后氏曰收,周曰冕。箋云殷之臣壯美而敏,來助周祭,其助祭自服殷之服,明文王以德不以彊。}\textbf{王之藎臣,無念爾祖。}{\footnotesize 藎,進也。無念,念也。箋云今王之進用臣,當念女祖為之法。王,斥成王。}

\begin{quoting}膚敏,黽勉。祼 \texttt{guàn}。常,通尚。冔 \texttt{xǔ}。\end{quoting}

\textbf{無念爾祖,聿脩厥德。永言配命,自求多福。}{\footnotesize 聿述、永長、言我也。我長配天命而行,爾庶國亦當自求多福。箋云長,猶常也。王既述脩祖德,常言當配天命而行,則福祿自來。}\textbf{殷之未喪師,克配上帝。}{\footnotesize 帝乙已上也。箋云師,眾也。殷自紂父之前未喪天下之時,皆能配天而行,故不亡也。}\textbf{宜鑒于殷,駿命不易。}{\footnotesize 駿,大也。箋云宜以殷王賢愚為鏡,天之大命,不可改易。}

\begin{quoting}聿,述、遵行。言,語詞。\end{quoting}

\textbf{命之不易,無遏爾躬。宣昭義問,有虞殷自天。}{\footnotesize 遏止、義善、虞度也。箋云宣徧、有又也。天之大命已不可改易矣,當使子孫長行之,無終女身則止,徧明以禮義問老成人,又度殷所以順天之事而施行之。}\textbf{上天之載,無聲無臭。儀刑文王,萬邦作孚。}{\footnotesize 載事、刑法、孚信也。箋云天之道難知也,耳不聞聲音,鼻不聞香臭,儀法文王之事,則天下咸信而順之也。}

\begin{quoting}義問,即令聞。\textbf{于省吾}有虞殷自天,應讀作又虞依自天,這是說,應宣昭義問,而揆度之以依於天,言事事以天為準。\textbf{馬瑞辰}載、事古音近通用,堯典「有能奮庸熙帝之載」,史記五帝本紀載作事。儀,象、法式,刑,古型字,儀刑二字同義,引伸為效法。作,則。\end{quoting}

\section{大明}

%{\footnotesize 八章、四章章六句、四章章八句}

\textbf{大明,文王有明德,故天復命武王也。}{\footnotesize 二聖相承,其明德日以廣大,故曰大明。}

\begin{quoting}\textbf{馬瑞辰}大明蓋對小雅有小明篇而言,逸周書世俘解「籥人奏武,王入進萬,獻明明三終」,孔晁注「明明,詩篇名」,當即此詩,是此詩又以明明名篇,蓋即取首句為篇名耳。\end{quoting}

\textbf{明明在下,赫赫在上。}{\footnotesize 明明,察也。文王之德,明明於下,故赫赫然著見於天。箋云明明者,文王武王施明德于天下,其徵應炤晢見於天,謂三辰效驗。}\textbf{天難忱斯,不易維王。天位殷適,使不挾四方。}{\footnotesize 忱,信也。紂居天位而殷之正適也。挾,達也。箋云天之意難信矣,不可改易者,天子也,今紂居天位而又殷之正適,以其為惡,乃棄絕之,使教令不行於四方,四方共叛之,是天命無常,維德是予耳,言此者,厚美周也。}

\begin{quoting}忱,三家詩或作諶,或作訦,信也。位,即立,古位、立同字。\end{quoting}

\textbf{摯仲氏任,自彼殷商。來嫁于周,曰嬪于京。乃及王季,維德之行。}{\footnotesize 摯國任姓之中女也。嬪婦、京大也。王季,大王之子,文王之父也。箋云京,周國之地,小別名也。及,與也。摯國中女曰大任,從殷商之畿內嫁為婦於周之京,配王季,而與之共行仁義之德,同志意也。}\textbf{大任有身,生此文王。}{\footnotesize 大任,仲任也。身,重也。箋云重,謂懷孕也。}

\begin{quoting}摯國,在今河南汝寧。\textbf{朱彬}經傳考證曰行,列也,維德之行,猶言德與之齊等。身,三家詩作娠。\end{quoting}

\textbf{維此文王,小心翼翼。昭事上帝,聿懷多福。厥德不回,以受方國。}{\footnotesize 回,違也。箋云小心翼翼,恭慎貌。昭明、聿述、懷思也。方國,四方來附者。此言文王之有德,亦由父母也。}

\begin{quoting}聿,發語詞。懷,招也。\end{quoting}

\textbf{天監在下,有命既集。文王初載,天作之合。在洽之陽,在渭之涘。}{\footnotesize 集就、載識、合配也。洽,水也。渭,水也。涘,厓也。箋云天監視善惡於下,其命將有所依就,則豫福助之於文王,生適有所識,則為之生配於氣勢之處,使必有賢才,謂生大姒。}\textbf{文王嘉止,大邦有子。}{\footnotesize 嘉,美也。箋云文王聞大姒之賢,則美之曰「大邦有子女可以為妃」,乃求昏。}

\begin{quoting}有,詞頭。初載,即位初年也。爾雅「妃,合也」,妃與配通。洽 \texttt{hé},出陝西郃陽,古莘國地。相鼠釋文引韓詩「止,節」。大邦,即莘國。\end{quoting}

\textbf{大邦有子,俔天之妹。}{\footnotesize 俔,磬也。箋云既使問名,還則卜之,又知大姒之賢,尊之如天之有女弟。}\textbf{文定厥祥,}{\footnotesize 言大姒之有文德也。祥,善也。箋云問名之後,卜而得吉,則文王以禮定其吉祥,謂使納幣也。}\textbf{親迎于渭。}{\footnotesize 言賢聖之配也。箋云賢女配聖人,得其宜,故備禮也。}\textbf{造舟為梁,不顯其光。}{\footnotesize 言受命之宜,王基乃始於是也。天子造舟,諸侯維舟,大夫方舟,士特舟,造舟然後可以顯其光輝。箋云迎大姒而更為梁者,欲其昭著,示後世敬昏禮也,不明乎其禮之有光輝,美之也。天子造舟,周制也,殷時未有等制。}

\begin{quoting}俔 \texttt{qiàn},說文「譬諭也」。不,發語詞。\end{quoting}

\textbf{有命自天,命此文王,于周于京。纘女維莘,長子維行。}{\footnotesize 纘,繼也。莘,大姒國也。長子,長女也,能行大任之德焉。箋云天為將命文王,君天下於周京之地,故亦為作合,使繼大任之女事於莘國,莘國之長女大姒則配文王,維德之行。}\textbf{篤生武王,保右命爾,燮伐大商。}{\footnotesize 篤厚、右助、燮和也。箋云天降氣于大姒,厚生聖子武王,安而助之,又遂命之爾,使協和伐殷之事,協和伐殷之事,謂合位三五也。}

\begin{quoting}纘 \texttt{zuǎn},同㜺 \texttt{zàn},廣韻「㜺,好容貌」。\textbf{馬瑞辰}㜺女謂好女,猶言淑女、碩女、靜女,皆美德之稱,詩言莘國有好女,倒其文則曰纘女維莘。又曰上言維德之行者,言太任德配王季,此言長子維行,言太姒德等文王也。篤,語詞,\textbf{馬瑞辰}尚書凡言大者皆語辭,丕、誕、洪、宏皆大也,亦皆語詞,詩生民「誕彌厥月」,誕字八見,皆詞也,按墨子經篇「厚有所大也」,是厚與大同義,故篤訓厚,亦為語詞。燮,同襲,左傳「有鍾鼓曰伐,無曰襲」。\end{quoting}

\textbf{殷商之旅,其會如林。矢于牧野,維予侯興。}{\footnotesize 旅,眾也。如林,言眾而不為用也。矢陳、興起也。言天下之望周也。箋云殷盛合其兵眾,陳於商郊之牧野,而天乃予諸侯有德者當起為天子,言天去紂,周師勝也。}\textbf{上帝臨女,無貳爾心。}{\footnotesize 言無敢懷貳心也。箋云臨,視也。女,女武王也。天護視女,伐紂必克,無有疑心。}

\begin{quoting}會,三家詩作旝 \texttt{kuài},旗也。矢,誓也。\end{quoting}

\textbf{牧野洋洋,檀車煌煌,駟騵彭彭。}{\footnotesize 洋洋,廣也。煌煌,明也。駠馬白腹曰騵。言上周下殷也。箋云言其戰地寬廣,明不用權詐也,兵車鮮明,馬又彊,則暇且整。}\textbf{維師尚父,時維鷹揚,涼彼武王。}{\footnotesize 師,大師也。尚父,可尚可父。鷹揚,如鷹之飛揚也。涼,佐也。箋云尚父,呂望也,尊稱焉。鷹,鷙鳥也。佐武王者,為之上將。}\textbf{肆伐大商,會朝清明。}{\footnotesize 肆,疾也。會,甲也。不崇朝而天下清明。箋云肆,故今也。會,合也。以天期已至,兵甲之彊,師率之武,故今伐殷,合兵以清明,書牧誓曰「時甲子昧爽,武王朝至于商郊牧野,乃誓」。}

\begin{quoting}時,是也。涼,魯詩韓詩作亮,爾雅「左右,亮也」。肆,魯詩作襲。清,韓詩作瀞,即淨字。\textbf{林義光}會朝清明,言適會早晨清明之時也,牧誓云「時甲子昧爽,王朝至于商郊牧野乃誓」,周語冷州鳩言「武王伐殷,以二月癸亥夜陳未畢而雨」,然則夜陳而朝誓師者,必以遇雨未獲畢陳,至朝而清明,乃復陳之也。\end{quoting}

\section{緜}

%{\footnotesize 九章、章六句}

\textbf{緜,文王之興,本由大王也。}

\textbf{緜緜瓜瓞,民之初生,自土沮漆。}{\footnotesize 興也。緜緜,不絕貌。瓜,紹也。瓞,瓝也。民,周民也。自用、土居也。沮水、漆水也。箋云瓜之本實,繼先歲之瓜必小,狀似瓝,故謂之瓞,緜緜然若將無長大時,興者,喻后稷乃帝嚳之胄,封於邰,其後公劉失職,遷于豳,居沮漆之地,歷世亦緜緜然,至大王而德益盛,得其民心而生王業,故本周之興,云于沮漆也。}\textbf{古公亶父,陶復陶穴,未有家室。}{\footnotesize 古公,豳公也,古,言久也,亶父,字,或殷以名言,質也。古公處豳,狄人侵之,事之以皮幣,不得免焉,事之以犬馬,不得免焉,事之以珠玉,不得免焉,乃屬其耆老而告之曰「狄人之所欲者,吾土地也,吾聞之君子,不以其所養人者害人,二三子何患乎無君」,去之,踰梁山,邑于岐山之下,豳人曰「仁人之君,不可失也」,從之如歸市,陶其土而復之,陶其壤而穴之。室內曰家,未有寢廟,亦未敢有家室。箋云古公據文王本其祖也,諸侯之臣稱其君曰公。復者,復於土上,鑿地曰穴,皆如陶然,本其在豳時也。傳自古公處豳而下,為二章發。}

\begin{quoting}瓞 \texttt{dié},小瓜也。土,齊詩作杜,水名,與漆均在豳地。沮,同徂。復,說文引詩从穴,地室也。\end{quoting}

\textbf{古公亶父,來朝走馬。率西水滸,至于岐下。爰及姜女,聿來胥宇。}{\footnotesize 率,循也。滸,水厓也。姜女,大姜也。胥相、宇居也。箋云來朝走馬,言其辟惡早且疾也。循西水涯,沮漆水側也。爰于、及與、聿自也。於是與其妃大姜自來相可居者,著大姜之賢知也。}

\begin{quoting}走,玉篇引詩作趣,疾也。爰、聿皆語詞也。\end{quoting}

\textbf{周原膴膴,堇荼如飴。爰始爰謀,爰契我龜。}{\footnotesize 周原,沮漆之間也。膴膴,美也。堇,菜也。荼,苦菜也。契,開也。箋云廣平曰原,周之原地在岐山之南,膴膴然肥美,其所生菜雖有性苦者,皆甘如飴也,此地將可居,故於是始與豳人之從己者謀,謀從,又於是契灼其龜而卜之,卜之則又從矣。}\textbf{曰止曰時,築室于茲。}{\footnotesize 箋云時是、茲此也。卜從則曰可止居於是,可作室家於此,定民心也。}

\begin{quoting}膴,韓詩作腜,\textbf{馬瑞辰}腜與飴、謀、龜、時、茲為韻,毛詩字雖作膴,其音亦當如腜字,音梅。又曰始亦謀也,爾雅基、肇皆訓為始,又皆訓為謀,則始與謀義正相成耳。時,善也。\end{quoting}

\textbf{廼慰廼止,廼左廼右。廼疆廼理,廼宣廼畝。自西徂東,周爰執事。}{\footnotesize 慰安、爰於也。箋云時耕曰宣。徂,往也。民心定,乃安隱其居,乃左右而處之,乃疆理其經界,乃時耕其田畝,於是從西方而往東之人皆於周執事,競出力也。豳與周原不能為西東,據至時從水滸言也。}

\begin{quoting}慰,方言「居也」。孔疏「宣訓為徧也、發也,天時已至,令民徧發土地,故謂之宣」。\textbf{馬瑞辰}梓材又曰「為厥疆畝」,傳曰「為其疆畔畝壟,然後功成」,即此詩廼畝也,上言疆理者,定其大界,此又別其畝壟。周,徧也,\textbf{朱熹}言靡事不為也。\end{quoting}

\textbf{乃召司空,乃召司徒,俾立室家。}{\footnotesize 箋云俾,使也。司空、司徒,卿官也,司空掌營國邑,司徒掌徒役之事,故召之使立室家之位處。}\textbf{其繩則直,縮版以載,作廟翼翼。}{\footnotesize 言不失繩直也。乘謂之縮。君子將營宮室,宗廟為先,廐庫為次,居室為後。箋云繩者,營其廣輪方制之正也,既正,則以索縮其築版,上下相承而起,廟成則嚴顯翼翼然。乘,聲之誤,當為繩也。}

\begin{quoting}載,栽也,\textbf{馬瑞辰}謂樹立其築牆長版也。\end{quoting}

\textbf{捄之陾陾,度之薨薨。築之登登,削屢馮馮。}{\footnotesize 捄,虆也。陾陾,眾也。度,居也,言百姓之勸勉也。登登,用力也。削牆鍛屢之聲馮馮然。箋云捄,捊也。度,猶投也。築牆者捊聚壤土,盛之以虆,而投諸版中。}\textbf{百堵皆興,鼛鼓弗勝。}{\footnotesize 皆,俱也。鼛,大鼓也,長一丈二尺。或鼛或鼓,言勸事樂功也。箋云五版為堵。興,起也。百堵同時起,鼛鼓不能止之使休息也。凡大鼓之側有小鼓,謂之應鼙、朔鼙,周禮曰「以鼛鼓鼓役事」。}

\begin{quoting}捄 \texttt{jiū}。陾陾 \texttt{réng},與下三疊字皆聲也。屢,應作婁,隆高也。\textbf{陳奐}「度,居也」下有薨薨二字。\textbf{俞樾}百堵皆興,則眾聲並作,鼛鼓之聲轉不足以勝之矣。\end{quoting}

\textbf{廼立臯門,臯門有伉。廼立應門,應門將將。}{\footnotesize 王之郭門曰臯門。伉,高貌。王之正門曰應門。將將,嚴正也。美大王作郭門以致臯門,作正門以致應門焉。箋云諸侯之宮,外門曰臯門,朝門曰應門,內有路門,天子之宮,加以庫雉。}\textbf{廼立冢土,戎醜攸行。}{\footnotesize 冢大、戎大、醜眾也。冢土,大社也,起大事,動大眾,必先有事乎社而後出,謂之宜,美大王之社,遂為大社也。箋云大社者,出大眾,將所告而行也,春秋傳曰「蜃,宜社之肉」。}

\begin{quoting}土,通社。前言作廟,今乃立社也。\end{quoting}

\textbf{肆不殄厥慍,亦不隕厥問。柞棫拔矣,行道兌矣,}{\footnotesize 肆,故今也。慍恚、隕墜也。兌,成蹊也。箋云小聘曰問。柞,櫟也。棫,白桵也。文王見太王立冢土,有用大眾之義,故不絕去其恚惡惡人之心,亦不廢其聘問鄰國之禮,今以柞棫生柯葉之時,使大夫將師旅出聘問,其行道士眾兌然,不有征伐之意。}\textbf{混夷駾矣,維其喙矣。}{\footnotesize 駾突、喙困也。箋云混夷,夷狄國也,見文王之使者將士眾過己國,則惶怖驚走,奔突入此柞棫之中而逃,甚困劇也。是之謂一年伐混夷,太王辟狄,文王伐混夷,成道興國,其志一也。}

\begin{quoting}\textbf{馬瑞辰}爾雅釋詁「肆,故也」,又曰「肆,故、今也」,字各為義,非以故今二字連讀。問,聲問、名譽。混夷,亦作昆夷。\end{quoting}

\textbf{虞芮質厥成,文王蹶厥生。}{\footnotesize 質,成也。成,平也。蹶,動也。虞芮之君相與爭田,久而不平,乃相謂曰「西伯,仁人也,盍往質焉」,乃相與朝周,入其竟,則耕者讓畔,行者讓路,入其邑,男女異路,班白不提挈,入其朝,士讓為大夫,大夫讓為卿,二國之君感而相謂曰「我等小人,不可以履君子之庭」,乃相讓,以其所爭田為間田而退,天下聞之而歸者四十餘國。箋云虞芮之質平,而文王動其緜緜民初生之道,謂廣其德而王業大。}\textbf{予曰有疏附,予曰有先後,予曰有奔奏,予曰有禦侮。}{\footnotesize 率下親上曰疏附,相道前後曰先後,喻德宣譽曰奔奏,武臣折衝曰禦侮。箋云予,我也,詩人自我也。文王之德所以至然者,我念之曰「此亦由有疏附、先後、奔奏、禦侮之臣力也」。疏附,使疏者親也,奔奏,使人歸趨之。}

\begin{quoting}蹶 \texttt{guì}。生,性也。\end{quoting}

\section{棫樸}

%{\footnotesize 五章、章四句}

\textbf{棫樸,文王能官人也。}

\begin{quoting}\textbf{汪質}毛詩異義曰國之大事在祀與戎,舉此二者以明賢才之用。\end{quoting}

\textbf{芃芃棫樸,薪之槱之。}{\footnotesize 興也。芃芃,木盛貌。棫,白桵也。樸,枹木也。槱,積也。山木茂盛,萬民得而薪之,賢人眾多,國家得用蕃興。箋云白桵相樸屬而生者,枝條芃芃然,豫斫以為薪,至祭皇天上帝及三辰,則聚積以燎之。}\textbf{濟濟辟王,左右趣之。}{\footnotesize 趣,趍也。箋云辟,君也,君王,謂文王也。文王臨祭祀,其容濟濟然敬,左右之諸臣皆促疾於事,謂相助積薪。}

\begin{quoting}樸,說文作㯷,棗也。說文「槱 \texttt{yóu},積火燎之也」。辟 \texttt{bì}。\end{quoting}

\textbf{濟濟辟王,左右奉璋。}{\footnotesize 半圭曰璋。箋云璋,璋瓚也。祭祀之禮,王祼,以圭瓚,諸臣助之,亞祼,以璋瓚。}\textbf{奉璋峨峨,髦士攸宜。}{\footnotesize 峨峨,盛壯也。髦,俊也。箋云士,卿士也。奉璋之儀峨峨然,故今俊士之所宜。}

\textbf{淠彼涇舟,烝徒楫之。}{\footnotesize 淠,舟行貌。楫,櫂也。箋云烝,眾也。淠淠然涇水中之舟順流而行者,乃眾徒船人以楫櫂之故也,興眾臣之賢者行君政令。}\textbf{周王于邁,六師及之。}{\footnotesize 天子六軍。箋云于往、邁行、及與也。周王往行,謂出兵征伐也。二千五百人為師。今王興師行者,殷末之制,未有周禮,周禮五師為軍,軍萬二千五百人。}

\begin{quoting}淠 \texttt{pì} 彼,即淠淠。\textbf{王先謙}軍舟浮涇而行,眾徒鼓楫,水聲淠淠然也。春秋繁露「周王于邁,六師及之,此文王之伐崇也」。\end{quoting}

\textbf{倬彼雲漢,為章于天。}{\footnotesize 倬,大也。雲漢,天河也。箋云雲漢之在天,其為文章譬猶天子為法度于天下。}\textbf{周王壽考,遐不作人。}{\footnotesize 遐,遠也,遠不作人也。箋云周王,文王也。文王是時九十餘矣,故云壽考。遠不作人者,其政變化紂之惡俗,近如新作人也。}

\begin{quoting}左傳成八年引詩「愷悌君子,遐不作人」,杜注「言文王能遠用善人,不,語助」,言文王德教也。\end{quoting}

\textbf{追琢其章,金玉其相。}{\footnotesize 追,雕也,金曰雕,玉曰琢。相,質也。箋云周禮追師掌追衡筓,則追亦治玉也。相,視也,猶觀視也。追琢玉使成文章,喻文王為政先以心研精,合於禮義,然後施之,萬民視而觀之,其好而樂之,如覩金玉然,言其政可樂也。}\textbf{勉勉我王,綱紀四方。}{\footnotesize 箋云我王,謂文王也。以罔罟喻為政,張之為綱,理之為紀。}

\begin{quoting}追 \texttt{duī},魯詩作雕。\textbf{陳啟源}毛詩稽古編曰章,周王之文也,相,周王之質也,追琢者其文,比其修飾也,金玉者其質,比其精純也。勉勉,三家詩作亹亹,文王毛傳「亹亹,勉也」。\end{quoting}

\section{旱麓}

%{\footnotesize 六章、章四句}

\textbf{旱麓,受祖也。周之先祖世脩后稷公劉之業,大王王季申以百福干祿焉。}

\begin{quoting}受祖,謂祭祀而獲福也。\end{quoting}

\textbf{瞻彼旱麓,榛楛濟濟。}{\footnotesize 旱,山名也。麓,山足也。濟濟,眾多也。箋云旱山之足林木茂盛者,得山雲雨之潤澤也,喻周邦之民獨豐樂者,被其君德教。}\textbf{豈弟君子,干祿豈弟。}{\footnotesize 干,求也。言陰陽和,山藪殖,故君子得以干祿樂易。箋云君子,謂大王王季。以有樂易之德施於民,故其求祿亦得樂易。}

\begin{quoting}\textbf{王應麟}詩地理考引漢書地理志曰漢中郡南鄭縣旱山,沱水所出,東北入漢。楛 \texttt{hù}。荀子注「樂易,歡樂平易也,所謂愷悌也」。\end{quoting}

\textbf{瑟彼玉瓚,黃流在中。}{\footnotesize 玉瓚,圭瓚也。黃金,所以飾。流,鬯也。九命,然後錫以秬鬯圭瓚。箋云瑟,絜鮮貌。黃流,秬鬯也。圭瓚之狀,以圭為柄,黃金為勺,青金為外,朱中央矣。殷王帝乙之時,王季為西伯,以功德受此賜。}\textbf{豈弟君子,福祿攸降。}{\footnotesize 箋云攸所、降下也。}

\begin{quoting}圭瓚,祭祀時酒器,以圭為柄,區別於棫樸之璋瓚。\textbf{陳奐}黃即勺,流即酒,黃流在中,言秬鬯之酒自勺中流出也。案秬鬯,黑黍與鬱金香草所釀之酒,用於祭祀降神。\end{quoting}

\textbf{鳶飛戾天,魚躍于淵。}{\footnotesize 言上下察也。箋云鳶,鴟之類,鳥之貪惡者也,飛而至天,喻惡人遠去,不為民害也。魚跳躍于淵中,喻民喜得所。}\textbf{豈弟君子,遐不作人。}{\footnotesize 箋云遐,遠也。言大王王季之德近於變化,使如新作人。}

\textbf{清酒既載,騂牡既備。}{\footnotesize 言年豐畜碩也。箋云既載,謂已在尊中也。祭祀之事,先為清酒,其次擇牲,故舉二者。}\textbf{以享以祀,以介景福。}{\footnotesize 言祀所以得福也。箋云介助、景大也。}

\begin{quoting}載,設飪也。騂牡,周人尚赤,故祭祀用之。享,孝敬。介,求也。\end{quoting}

\textbf{瑟彼柞棫,民所燎矣。}{\footnotesize 瑟,眾貌。箋云柞棫之所以茂盛者,乃人熂燎除其旁草,養治之,使無害也。}\textbf{豈弟君子,神所勞矣。}{\footnotesize 箋云勞,勞來,猶言佑助。}

\begin{quoting}燎,同尞,說文「尞,柴祭天也」。\end{quoting}

\textbf{莫莫葛藟,施于條枚。}{\footnotesize 莫莫,施貌。箋云葛也藟也,延蔓於木之枝本而茂盛,喻子孫依緣先人之功而起。}\textbf{豈弟君子,求福不回。}{\footnotesize 箋云不回者,不違先祖之道。}

\begin{quoting}條,枝也,枚,幹也。\end{quoting}

\section{思齊}

%{\footnotesize 四章、章六句,故言五章、二章章六句、三章章四句}

\textbf{思齊,文王所以聖也。}{\footnotesize 言非但天性,德有所由成。}

\textbf{思齊大任,文王之母。思媚周姜,京室之婦。}{\footnotesize 齊莊、媚愛也。周姜,大姜也。京室,王室也。箋云京,周地名也。常思莊敬者,大任也,乃為文王之母,又常思愛大姜之配大王之禮,故能為京室之婦,言其德行純備,故生聖子也。大姜言周,大任言京,見其謙恭,自卑小也。}\textbf{大姒嗣徽音,則百斯男。}{\footnotesize 大姒,文王之妃也,大姒十子,眾妾則宜百子也。箋云徽,美也。嗣大任之美音,謂續行其善教令。}

\textbf{惠于宗公,神罔時怨,神罔時恫。}{\footnotesize 宗公,宗神也。恫,痛也。箋云惠,順也。宗公,大臣也。文王為政,咨於大臣,順而行之,故能當於神明,神明無是怨恚其所行者,無是痛傷其所為者,其將無有凶禍。}\textbf{刑于寡妻,至于兄弟,以御于家邦。}{\footnotesize 刑,法也。寡妻,適妻也。御,迎也。箋云寡妻,寡有之妻,言賢也。御,治也。文王以禮法接待其妻,至于宗族,以此又能為政治于家邦也,書曰「乃寡兄勗」,又曰「越乃御事」。}

\begin{quoting}\textbf{馬瑞辰}宗、尊雙聲,宗公即先公也,言其久則曰古公,言其尊則曰宗公,又宗、崇古通用,崇,高也,則宗公猶云高祖,與尊義正相近。\textbf{經義述聞}時,所也。恫 \texttt{tōng},說文「痛也,一曰呻吟也」。\textbf{胡承珙}適與庶對,庶為眾則適為寡矣。\end{quoting}

\textbf{雝雝在宮,肅肅在廟。}{\footnotesize 雝雝,和也。肅肅,敬也。箋云宮,謂辟廱宮也。群臣助文王養老則尚和,助祭於廟則尚敬,言得禮之宜。}\textbf{不顯亦臨,無射亦保。}{\footnotesize 以顯臨之,保安無猒也。箋云臨,視也。保,猶居也。文王之在辟雝也,有賢才之質而不明者,亦得觀於禮,於六藝無射才者,亦得居於位,言養善使之積小致高大。}\textbf{肆戎疾不殄,烈假不瑕。}{\footnotesize 肆,故今也。戎,大也。故今大疾害人者,不絕之而自絕也。烈業、假大也。箋云厲、假皆病也。瑕,已也。文王於辟雝德如此,故大疾害人者不絕之而自絕,為厲假之行者不已之而自已,言化之深也。}

\begin{quoting}\textbf{朱熹}言文王在閨門之內則極其和,在宗廟之中則極其敬。不、亦皆語詞,下同。射,通斁,厭足。戎疾,西戎之禍患也。不,語詞。烈,同厲,說文作癘。假,同瘕,即蠱字。瑕,通遐,\textbf{王先謙}言凡如惡病害人者以遐遠矣。\end{quoting}

\textbf{不聞亦式,不諫亦入。}{\footnotesize 言性與天合也。箋云式,用也。文王之祀於宗廟,有仁義之行而不聞達者,亦用之助祭,有孝悌之行而不能諫爭者,亦得入,言其使人器之,不求備也。}\textbf{肆成人有德,小子有造。}{\footnotesize 造,為也。箋云成人,謂大夫士也。小子,其弟子也。文王在於宗廟德如此,故大夫士皆有德,子弟皆有所造成。}\textbf{古之人無斁,譽髦斯士。}{\footnotesize 古之人無厭於有名譽之俊士。箋云古之人,謂聖王明君也。口無擇言,身無擇行,以身化其臣下,故令此士皆有名譽於天下,成其俊乂之美也。}

\begin{quoting}\textbf{王先謙}言古之人教士無厭斁,故能使斯士皆成為譽髦也。\end{quoting}

\section{皇矣}

%{\footnotesize 八章、章十二句}

\textbf{皇矣,美周也。天監代殷,莫若周,周世世脩德,莫若文王。}{\footnotesize 監,視也。天視四方可以代殷王天下者,維有周爾,世世脩行道德,維有文王盛爾。}

\textbf{皇矣上帝,臨下有赫。監觀四方,求民之莫。}{\footnotesize 皇大、莫定也。箋云臨,視也。大矣,天之視天下赫然甚明,以殷紂之暴亂,乃監察天下之眾國,求民之定,謂所歸就也。}\textbf{維此二國,其政不獲。維彼四國,爰究爰度。}{\footnotesize 二國,殷夏也。彼,彼有道也。四國,四方也。究謀、度居也。箋云二國,謂今殷紂及崇侯也。正長、獲得也。四國,謂密也、阮也、徂也、共也。度亦謀也。殷崇之君其行暴亂,不得於天心,密阮徂共之君於是又助之謀,言同於惡也。}\textbf{上帝耆之,憎其式廓。乃眷西顧,此維與宅。}{\footnotesize 耆,惡也。廓,大也。憎其用大位、行大政。顧,顧西土也。宅,居也。箋云耆,老也。天須假此二國,養之至老猶不變改,憎其所用為惡者浸大也,乃眷然運視西顧,見文王之德而與之居,言天意常在文王所。}

\begin{quoting}莫,魯詩、齊詩作瘼,說文「瘼,病也」。耆,通恉 \texttt{zhǐ},\textbf{林義光}恉之言指,謂意之所向也,言上帝究度四國之後,意向于周,以為可作民主。憎,同增。式廓,規模。\textbf{陳奐}眷,顧貌。與,魯詩作予。漢書郊祀志曰「乃眷西顧,此維與宅」,言天以文王之都為居也。\end{quoting}

\textbf{作之屏之,其菑其翳。脩之平之,其灌其栵。啟之辟之,其檉其椐。攘之剔之,其檿其柘。}{\footnotesize 木立死曰菑,自斃為翳。灌,叢生也。栵,栭也。檉,河柳也。椐,樻也。檿,山桑也。箋云天既顧文王,四方之民則大歸往之,岐周之地險隘多樹木,乃競刊除而自居處,言樂就有德之甚。}\textbf{帝遷明德,串夷載路。}{\footnotesize 徙就文王之德也。串習、夷常、路大也。箋云串夷即混夷,西戎國名也。路,應也。天意去殷之惡,就周之德,文王則侵伐混夷以應之。}\textbf{天立厥配,受命既固。}{\footnotesize 配,媲也。箋云天既顧文王,又為之生賢妃,謂大姒也,其受命之道已堅固也。}

\begin{quoting}作,同槎,說文「槎,衺斫也」。屏,同摒。菑 \texttt{zì}。栵,同烈,方言「烈,枿 \texttt{niè} 餘也」,爾雅釋詁「枿,餘也」,郭注「陳鄭之間曰枿,晉衛之間曰烈,皆伐木餘也」。\textbf{陳喬樅}三家詩遺說考曰檉 \texttt{chēng} 椐易生之木,故其地則啟之闢之,檿柘 \texttt{yǎn zhè} 有用之材,故其樹則攘而剔之,如是者,土地既廣,樹木亦茂。路,同露,\textbf{馬瑞辰}方言、廣雅並云「露,敗也」,詩謂帝遷明德,串夷則瘠敗罷憊而去,故曰載路。厥配,文王也。\end{quoting}

\textbf{帝省其山,柞棫斯拔,松柏斯兌。}{\footnotesize 兌,易直也。箋云省,善也。天既顧文王,乃和其國之風雨,使其山樹木茂盛,言非徒養其民人而已。}\textbf{帝作邦作對,自大伯王季。}{\footnotesize 對,配也。從大伯之見王季也。箋云作,為也。天為邦,謂興周國也,作配,謂為生明君也,是乃自大伯王季時則然矣,大伯讓於王季而文王起。}\textbf{維此王季,因心則友。則友其兄,則篤其慶,載錫之光。}{\footnotesize 因,親也。善兄弟曰友。慶善、光大也。箋云篤厚、載始也。王季之心親親而又善於宗族,又尤善於兄大伯,乃厚明其功美,始使之顯著也,大伯以讓為功美,王季乃能厚明之,使傳世稱之,亦其德也。}\textbf{受祿無喪,奄有四方。}{\footnotesize 喪亡、奄大也。箋云王季以有因心則友之德,故世世受福祿,至於覆有天下。}

\begin{quoting}\textbf{朱熹}言帝省其山,而見其木拔道通,則知民之歸之者益眾矣。因,古姻字,親愛。\end{quoting}

\textbf{維此王季,帝度其心,貊其德音。其德克明,克明克類,克長克君。}{\footnotesize 心能制義曰度。貊,靜也。箋云德正應和曰貊,照臨四方曰明。類,善也。勤施無私曰類,教誨不倦曰長,賞慶刑威曰君。}\textbf{王此大邦,克順克比。}{\footnotesize 慈和徧服曰順,擇善而從曰比。箋云王,君也。王季稱王,追王也。}\textbf{比于文王,其德靡悔。}{\footnotesize 經緯天地曰文。箋云靡,無也。王季之德比于文王無有所悔也,必比于文王者,德以聖人為匹。}\textbf{既受帝祉,施于孫子。}{\footnotesize 箋云帝,天也。祉,福也。施,猶易也、延也。}

\begin{quoting}貊,韓詩作莫,文選西征賦引韓詩章句「寞,靜也」,爾雅釋言「漠,清也」。\textbf{于省吾}此詩本應作「王此大邦,克順克從」,屬詞與韻讀無有不符。比,及也。\end{quoting}

\textbf{帝謂文王,無然畔援,無然歆羨,誕先登于岸。}{\footnotesize 無是畔道,無是援取,無是貪羨。岸,高位也。箋云畔援,猶跋扈也。誕大、登成、岸訟也。天語文王曰「女無如是跋扈者妄出兵也,無如是貪羨者侵人土地也,欲廣大德美者,當先平獄訟、正曲直也」。}\textbf{密人不恭,敢距大邦,侵阮徂共。}{\footnotesize 國有密須氏,侵阮,遂往侵共。箋云阮也、徂也、共也三國犯周,而文王伐之,密須之人乃敢距其義兵,違正道,是不直也。}\textbf{王赫斯怒,爰整其旅,以按徂旅,以篤于周祜,以對于天下。}{\footnotesize 旅師、按止也。旅,地名也。對,遂也。箋云赫,怒意。斯,盡也。五百人為旅。對,答也。文王赫然與其群臣盡怒,曰整其軍旅而出,以却止徂國之兵眾,以厚周當王之福,以答天下鄉周之望。}

\begin{quoting}誕,發語詞。按,孟子引詩作遏,二字雙聲。旅,同莒,古國名。對,\textbf{陳奐}遂又為安,孟子云「文王一怒而安天下之民」,即其義也。\end{quoting}

\textbf{依其在京,侵自阮疆。陟我高岡,無矢我陵,我陵我阿,無飲我泉,我泉我池。}{\footnotesize 京,大阜也。矢,陳也。箋云京,周地名。陟,登也。矢,猶當也。大陵曰阿。文王但發其依居京地之眾,以往侵阮國之疆,登其山脊而望阮之兵,兵無敢當其陵及阿者,又無敢飲食於其泉及池水者。小出兵而令驚怖如此,此以德攻,不以眾也。陵泉重言者,美之也。每言我者,據後得而有之而言。}\textbf{度其鮮原,居岐之陽,在渭之將。萬邦之方,下民之王。}{\footnotesize 小山別大山曰鮮。將,側也。方,則也。箋云度謀、鮮善也。方,猶鄉也。文王見侵阮而兵不見敵,知己德盛而威行,可以遷居,以定天下之人心也,于是乃始謀居善原廣平之地,亦在岐山之南隅也,而居渭水之側,為萬國之所鄉,作下民之君,後竟徙都於豐。}

\begin{quoting}依,殷也,二字雙聲通用。侵,同寑,息也。\textbf{馬瑞辰}依其在京是已還兵於周京,則寑自阮疆是追述其息兵於阮疆之始。陟我高岡以下五句,謂敵不敢來犯也。鮮,通巘 \texttt{yǎn}。\textbf{陳奐}將之為言牆也,爾雅「畢,堂牆」,堂牆為山厓邊側之名,其水厓邊側亦如是也。王,歸往。\end{quoting}

\textbf{帝謂文王,予懷明德,不大聲以色,不長夏以革,不識不知,順帝之則。}{\footnotesize 懷,歸也。不大聲見於色。革,更也,不以長大有所更。箋云夏,諸夏也。天之言云「我歸人君有光明之德,而不虛廣言語,以外作容貌」。不長諸夏以變更王法者,其為人不識古、不知今,順天之法而行之者。此言天之道尚誠實,貴性自然。}\textbf{帝謂文王,詢爾仇方,同爾兄弟,以爾鉤援,與爾臨衝,以伐崇墉。}{\footnotesize 仇,匹也。鉤,鉤梯也,所以鉤引上城者。臨,臨車也。衝,衝車也。墉,城也。箋云詢,謀也。怨耦曰仇,仇方謂旁國。諸侯為暴亂大惡者,女當謀征討之,以和協女兄弟之國,率與之往,親親則多志齊心壹也。當此之時,崇侯虎倡紂為無道,罪尤大也。}

\begin{quoting}以,與也,下句同。\textbf{馬瑞辰}不長夏以革者,不齊之以刑也,夏謂夏楚,扑作教刑也,革謂鞭革,鞭作官刑也。\end{quoting}

\textbf{臨衝閑閑,崇墉言言。執訊連連,攸馘安安。是類是禡,是致是附,四方以無侮。}{\footnotesize 閑閑,動搖也。言言,高大也。連連,徐也。攸,所也。馘,獲也,不服者,殺而獻其左耳曰馘。於內曰類,於野曰禡。致,致其社稷群神。附,附其先祖,為之立後,尊其尊而親其親。箋云言言,猶孽孽,將壞貌。訊,言也。執所生得者而言問之,及獻所馘,皆徐徐以禮為之,不尚促速也。類也禡也,師祭也。無侮者,文王伐崇而無復敢侮慢周者。}\textbf{臨衝茀茀,崇墉仡仡。是伐是肆,是絕是忽,四方以無拂。}{\footnotesize 茀茀,彊盛也。仡仡,猶言言也。肆,疾也。忽,滅也。箋云伐謂擊刺之。肆,犯突也,春秋傳曰「使勇而無剛者肆之」。拂,猶佹也,言無復佹戾文王者。}

\begin{quoting}類、禡 \texttt{mà} 皆祭天也,\textbf{陳奐}野是征國之野,先類後禡,依行師之次序也。\textbf{馬瑞辰}致者,致人民土地,說文「致,送詣也」,送而付之曰致,已克而不取之謂也。附,通拊,安撫。以,因也,下同。仡仡,同屹屹。拂,釋文引王肅曰「違也」。\end{quoting}

\section{靈臺}

%{\footnotesize 五章、章四句}

\textbf{靈臺,民始附也。文王受命而民樂其有靈德,以及鳥獸昆蟲焉。}{\footnotesize 民者,冥也,其見仁道遲,故於是乃附也。天子有靈臺者,所以觀祲象,察氣之妖祥也,文王受命而作邑于豐、立靈臺,春秋傳曰「公既視朔,遂登觀臺以望而書雲物,為備故也」。}

\textbf{經始靈臺,經之營之。庶民攻之,不日成之。}{\footnotesize 神之精明者稱靈,四方而高曰臺。經,度之也。攻,作也。不日有成也。箋云文王應天命,度始靈臺之基止,營表其位,眾民則築作,不設期日而成之,言說文王之德,勸其事,忘己勞也。觀臺而曰靈者,文王化行似神之精明,故以名焉。}

\begin{quoting}經、始二字同義。\textbf{馬瑞辰}積仁為靈,蓋亦訓靈為善,因有善德而名其臺為靈臺。\end{quoting}

\textbf{經始勿亟,庶民子來。}{\footnotesize 箋云亟,急也。度始靈臺之基止,非有急成之意,眾民各以子成父事而來攻之。}\textbf{王在靈囿,麀鹿攸伏。}{\footnotesize 囿,所以域養禽獸也,天子百里,諸侯四十里。靈囿,言靈道行於囿也。麀,牝也。箋云攸,所也。文王親至靈囿,視牝鹿所遊伏之處,言愛物也。}

\begin{quoting}麀 \texttt{yōu}。\end{quoting}

\textbf{麀鹿濯濯,白鳥翯翯。}{\footnotesize 濯濯,娛遊也。翯翯,肥澤也。箋云鳥獸肥盛喜樂,言得其所。}\textbf{王在靈沼,於牣魚躍。}{\footnotesize 沼,池也。靈沼,言靈道行於沼也。牣,滿也。箋云靈沼之水,魚盈滿其中,皆跳躍,亦言其得所。}

\begin{quoting}翯 \texttt{hè},魯詩作皜。於 \texttt{wū}。\end{quoting}

\textbf{虡業維樅,賁鼓維鏞。於論鼓鍾,於樂辟廱。}{\footnotesize 植者曰虡,橫者曰栒。業,大版也。樅,崇牙也。賁,大鼓也。鏞,大鍾也。論,思也。水旋丘如璧曰辟廱,以節觀者。箋云論之言倫也。虡也栒也,所以縣鍾鼓也,設大版於上,刻畫以為飾。文王立靈臺而知民之歸附,作靈囿靈沼而知鳥獸之得其所,以為音聲之道與政通,故合樂以詳之,於得其倫理乎,鼓與鍾也,於喜樂乎,諸在辟廱中者,言感於中和之至。}

\begin{quoting}虡 \texttt{jù}。孔疏曰懸鍾磬之處,又以彩色為大牙,其狀隆然,謂之崇牙。賁 \texttt{fén}。辟廱,文王離宮名,辟即璧,廱即水澤池沼,離宮中有池沼如璧,故以名焉,非漢儒之辟廱也。\end{quoting}

\textbf{於論鼓鍾,於樂辟廱。鼉鼓逢逢,矇瞍奏公。}{\footnotesize 鼉,魚屬。逢逢,和也。有眸子而無見曰矇,無眸子曰瞍。公,事也。箋云凡聲,使瞽矇為之。}

\begin{quoting}鼉 \texttt{tuó}。逢逢,聲也。公,通功,謂靈臺落成也。\end{quoting}

\section{下武}

%{\footnotesize 六章、章四句}

\textbf{下武,繼文也。武王有聖德,復受天命,能昭先人之功焉。}{\footnotesize 繼文者,繼文王之王業而成之。昭,明也。}

\begin{quoting}\textbf{馬瑞辰}按此詩序言「繼文」,與文王有聲言「繼伐」相對成文,詩中「世德作求、應侯順德」皆尚文德之事。\end{quoting}

\textbf{下武維周,世有哲王。}{\footnotesize 武,繼也。箋云下,猶後也。哲,知也。後人能繼先祖者,維有周家最大,世世益有明知之王,謂大王王季文王稍就盛也。}\textbf{三后在天,王配于京。}{\footnotesize 三后,大王王季文王也。王,武王也。箋云此三后既沒登遐,精氣在天矣,武王又能配行其道於京,謂鎬京也。}

\textbf{王配于京,世德作求。}{\footnotesize 箋云作為、求終也。武王配行三后之道於鎬京者,以其世世積德,庶為終成其大功。}\textbf{永言配命,成王之孚。}{\footnotesize 箋云永長、言我也。命,猶教令也。孚,信也。此為武王言也。今長我之配行三后之教令者,欲成我周家王道之信也。王德之道成於信,論語曰「民無信不立」。}

\begin{quoting}\textbf{馬瑞辰}求,當讀為逑,逑,匹也、配也,作求即作配耳,此言作配於周三王也,言王所以配於京者,由其可與世德作配耳。\end{quoting}

\textbf{成王之孚,下土之式。}{\footnotesize 式,法也。箋云王道尚信,則天下以為法,勤行之。}\textbf{永言孝思,孝思維則。}{\footnotesize 則其先人也。箋云長我孝心之所思,所思者,其維則三后之所行。子孫以順祖考為孝。}

\begin{quoting}言、思皆語詞也。\end{quoting}

\textbf{媚茲一人,應侯順德。}{\footnotesize 一人,天子也。應當、侯維也。箋云媚愛、茲此也。可愛乎武王,能當此順德,謂能成其祖考之功也,易曰「君子以順德,積小以高大」。}\textbf{永言孝思,昭哉嗣服。}{\footnotesize 箋云服,事也。明哉,武王之嗣行祖考之事,謂伐紂定天下。}

\begin{quoting}順,魯詩作慎,二字古通用。\end{quoting}

\textbf{昭茲來許,繩其祖武。}{\footnotesize 許進、繩戒、武迹也。箋云茲此、來勤也。武王能明此勤行,進於善道,戒慎其祖考所踐履之迹,美其終成之。}\textbf{於萬斯年,受天之祜。}{\footnotesize 箋云祜,福也。天下樂仰武王之德,欲其壽考之言也。}

\begin{quoting}茲,三家詩作哉。來許,與上章嗣服同義,即指武王,三家詩許作御,\textbf{馬瑞辰}廣雅許、御並訓進,詩五章皆首尾相承,此特易字以協下韻,哉與茲聲同,來猶後也,後猶嗣也,來許猶云後進。繩,三家詩作慎,繼承。萬斯,即萬萬。\end{quoting}

\textbf{受天之祜,四方來賀。於萬斯年,不遐有佐。}{\footnotesize 遠夷來佐也。箋云武王受此萬年之壽,不遠有佐,言其輔佐之臣亦宜蒙其餘福也,書曰「公其以予萬億年」,亦君臣同福祿也。}

\section{文王有聲}

%{\footnotesize 八章、章五句}

\textbf{文王有聲,繼伐也。武王能廣文王之聲,卒其伐功也。}{\footnotesize 繼伐者,文王伐崇而武王伐紂。}

\begin{quoting}\textbf{方玉潤}此詩專以遷都定鼎為言。\end{quoting}

\textbf{文王有聲,遹駿有聲。遹求厥寧,遹觀厥成。}{\footnotesize 箋云遹述、駿大、求終、觀多也。文王有令聞之聲者,乃述行有令聞之聲之道所致也,所述者,謂大王王季也,又述行終其安民之道,又述行多其成民之德,言周德之世益盛。}\textbf{文王烝哉。}{\footnotesize 烝,君也。箋云君哉者,言其誠得人君之道。}

\begin{quoting}遹 \texttt{yù},同聿、曰,發語詞。\textbf{陳奐}釋文引韓詩云「烝,美也」,烝哉,即君哉,美歎詞。\end{quoting}

\textbf{文王受命,有此武功。既伐于崇,作邑于豐。}{\footnotesize 箋云武功,謂伐四國及崇之功也。作邑者,徙都于豐,以應天命。}\textbf{文王烝哉。}

\textbf{築城伊淢,作豐伊匹。匪棘其欲,遹追來孝。}{\footnotesize 淢,成溝也。匹,配也。箋云方十里曰成,淢,其溝也,廣深各八尺。棘急、來勤也。文王受命而猶不自足,築豐邑之城,大小適與成偶,大於諸侯,小於天子之制,此非以急成從己之欲,欲廣都邑,乃述追王季勤孝之行,進其業也。}\textbf{王后烝哉。}{\footnotesize 后,君也。箋云變謚言王后者,非其盛事,不以義謚。}

\begin{quoting}淢 \texttt{xù},魯詩、韓詩作洫。遹、來皆語詞。\end{quoting}

\textbf{王公伊濯,維豐之垣。四方攸同,王后維翰。}{\footnotesize 濯大、翰幹也。箋云公,事也。文王述行大王王季之王業,其事益大,作邑於豐,城之既成,又垣之,立宮室,乃為天下所同心而歸之。王后為之幹者,正其政教,定其法度。}\textbf{王后烝哉。}

\textbf{豐水東注,維禹之績。四方攸同,皇王維辟。}{\footnotesize 績業、皇大也。箋云績功、辟君也。昔堯時洪水而豐水亦氾濫為害,禹治之,使入渭,東注于河,禹之功也,文王武王今得作邑於其旁地,為天下所同心而歸。大王為之君乃由禹之功,故引美之。豐邑在豐水之西,鎬京在豐水之東。}\textbf{皇王烝哉。}{\footnotesize 箋云變王后言大王者,武王之事又益大。}

\begin{quoting}辟,法則。\textbf{方玉潤}以豐水為兩京樞紐,豐水之東即鎬,遞下鎬京無迹。\end{quoting}

\textbf{鎬京辟廱,自西自東,自南自北,無思不服。}{\footnotesize 武王作邑於鎬京。箋云自,由也。武王於鎬京行辟廱之禮,自四方來觀者皆感化其德,心無不歸服者。}\textbf{皇王烝哉。}

\begin{quoting}思,語詞。\end{quoting}

\textbf{考卜維王,宅是鎬京。維龜正之,武王成之。}{\footnotesize 箋云考,猶稽也。宅,居也。稽疑之法,必契灼龜而卜之,武王卜居是鎬京之地,龜則正之,謂得吉兆,武王遂居之,脩三后之德,以伐紂定天下,成龜兆之占,功莫大於此。}\textbf{武王烝哉。}

\begin{quoting}宅,齊詩作度,亦居也。正,同貞,卜問。\end{quoting}

\textbf{豐水有芑,武王豈不仕,詒厥孫謀,以燕翼子。}{\footnotesize 芑,草也。仕事、燕安、翼敬也。箋云詒,猶傳也。孫,順也。豐水猶以其潤澤生草,武王豈不以其功業為事乎,以之為事,故傳其所以順天下之謀,以安其敬事之子孫,謂使行之也,書曰「厥考翼,其肯曰『我有後,弗棄基』」。}\textbf{武王烝哉。}{\footnotesize 箋云上言皇王而變言武王者,皇,大也,始大其業,至武王伐紂成之,故言武王也。}

\begin{quoting}芑 \texttt{qǐ}。孫,同訓。\end{quoting}

%\begin{flushright}文王之什十篇、六十六章、四百一十四句\end{flushright}