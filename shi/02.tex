\chapter{召南鵲巢詁訓傳第二}

\begin{quoting}\textbf{釋文}召亦地名也,在岐山之陽扶風雍縣南有召亭,案周召皆周之舊土,文王受命後以賜二公為菜地,二南之風皆文王未受命之詩也,周南十一篇是先王之所以教聖人之深迹,故繫之公旦,召南十四篇是先王之教化文王所行之淺迹,故繫之君奭。\end{quoting}

\section{鵲巢}

%{\footnotesize 三章、章四句}

\textbf{鵲巢,夫人之德也。國君積行累功,以致爵位,夫人起家而居有之,德如鳲鳩,乃可以配焉。}{\footnotesize 起家而居有之,謂嫁於諸侯也。夫人有均壹之德,如鳲鳩然,而後可配國君。}

\textbf{維鵲有巢,維鳩居之。}{\footnotesize 興也。鳩,鳲鳩,秸鞠也,鳲鳩不自為巢,居鵲之成巢。箋云鵲之作巢,冬至架之,至春乃成,猶國君積行累功,故以興焉。興者,鳲鳩因鵲成巢而居有之,而有均壹之德,猶國君夫人來嫁,居君子之室,德亦然。室,燕寢也。}\textbf{之子于歸,百兩御之。}{\footnotesize 百兩,百乘也。諸侯之子嫁於諸侯,送御皆百乘。箋云之子,是子也。御,迎也。是如鳲鳩之子,其往嫁也,家人送之,良人迎之,車皆百乘,象有百官之盛。}

\begin{quoting}兩,今作輛,孔疏「謂之兩者,風俗通以為車有兩輪,馬有四匹,車稱兩,馬稱駟」。御,同訝、迓。\end{quoting}

\textbf{維鵲有巢,維鳩方之。}{\footnotesize 方,有之也。}\textbf{之子于歸,百兩將之。}{\footnotesize 將,送也。}

\begin{quoting}\textbf{馬瑞辰}詩百兩皆指迎者而言,將者,奉也衛也,首章往迎則曰御之,二章在途則曰將之,三章既至則曰成之,此詩之次也。\end{quoting}

\textbf{維鵲有巢,維鳩盈之。}{\footnotesize 盈,滿也。箋云滿者,言眾媵姪娣之多。}\textbf{之子于歸,百兩成之。}{\footnotesize 能成百兩之禮也。箋云是子有鳲鳩之德,宜配國君,故以百兩之禮送迎成之。}

\section{采蘩}

%{\footnotesize 三章、章四句}

\textbf{采蘩,夫人不失職也。夫人可以奉祭祀,則不失職矣。}{\footnotesize 奉祭祀者,采蘩之事也,不失職者,夙夜在公也。}

\textbf{于以采蘩,于沼于沚。}{\footnotesize 蘩,皤蒿也。于於、沼池、沚渚也。公侯夫人執蘩菜以助祭,神饗德與信,不求備焉,沼沚谿澗之草猶可以薦,王后則荇菜也。箋云于以,猶言往以也。執蘩菜者,以豆薦蘩葅。}\textbf{于以用之,公侯之事。}{\footnotesize 之事,祭事也。箋云言夫人於君祭祀而薦此豆也。}

\begin{quoting}七月傳「蘩,皤蒿也,所以生蠶」。\end{quoting}

\textbf{于以采蘩,于澗之中。}{\footnotesize 山夾水曰澗。}\textbf{于以用之,公侯之宮。}{\footnotesize 宮,廟也。}

\textbf{被之僮僮,夙夜在公。}{\footnotesize 被,首飾也。僮僮,竦敬也。夙,早也。箋云公,事也。早夜在事,謂視濯溉饎爨之事。禮記「主婦髲髢」。}\textbf{被之祁祁,薄言還歸。}{\footnotesize 祁祁,舒遲也,去事有儀也。箋云言,我也。祭事畢,夫人釋祭服而去髲髢,其威儀祁祁然而安舒,無罷倦之失。我還歸者,自廟反其燕寢。}

\begin{quoting}被,同髲。\textbf{王先謙}三家僮僮作童童,魯韓說曰「童童,盛也」。七月傳「祁祁,眾多也」。\end{quoting}

\section{草蟲}

%{\footnotesize 三章、章七句}

\textbf{草蟲,大夫妻能以禮自防也。}

\begin{quoting}\textbf{王照圓}兩年事爾,君子行役當春夏間,涉秋未歸,故感蟲鳴而思,至來年春夏猶未歸,故復有後二章。\end{quoting}

\textbf{喓喓草蟲,趯趯阜螽。}{\footnotesize 興也。喓喓,聲也。草蟲,常羊也。趯趯,躍也。阜螽,蠜也。卿大夫之妻待禮而行,隨從君子。箋云草蟲鳴,阜螽躍而從之,異種同類,猶男女嘉時以禮相求呼。}\textbf{未見君子,憂心忡忡。}{\footnotesize 忡忡,猶衝衝也。婦人雖適人,有歸宗之義。箋云未見君子者,謂在塗時也,在塗而憂,憂不當君子,無以寧父母,故心衝衝然,是其不自絕於其族之情。}\textbf{亦既見止,亦既覯止,我心則降。}{\footnotesize 止,辭也。覯遇、降下也。箋云既見,謂已同牢而食也,既覯,謂已昏也。始者憂於不當,今君子待己以禮,庶自此可以寧父母,故心下也。易曰「男女覯精,萬物化生」。}

\begin{quoting}喓 \texttt{yāo}。趯 \texttt{tì}。案時已秋矣。\end{quoting}

\textbf{陟彼南山,言采其蕨。}{\footnotesize 南山,周南山也。蕨,鼈也。箋云言,我也。我采者,在塗而見采鼈,采者得其所欲得,猶己今之行者欲得禮以自喻也。}\textbf{未見君子,憂心惙惙。}{\footnotesize 惙惙,憂也。}\textbf{亦既見止,亦既覯止,我心則說。}{\footnotesize 說,服也。}

\begin{quoting}惙,短氣貌也。案時已來年春矣。\end{quoting}

\textbf{陟彼南山,言采其薇。}{\footnotesize 薇,菜也。}\textbf{未見君子,我心傷悲。}{\footnotesize 嫁女之家,不息火三日,思相離也。箋云維父母思己,故己亦傷悲。}\textbf{亦既見止,亦既覯止,我心則夷。}{\footnotesize 夷,平也。}

\section{采蘋}

%{\footnotesize 三章、章四句}

\textbf{采蘋,大夫妻能循法度也。能循法度,則可以承先祖、共祭祀矣。}{\footnotesize 女子十年不出,姆教婉娩聽從,執麻枲,治絲繭,織紝組紃,學女事以共衣服,觀於祭祀,納酒漿籩豆葅醢,禮相助奠,十有五而筓,二十而嫁。此言能循法度者,今既嫁為大夫妻,能循其為女之時所學所觀之事以為法度。}

\textbf{于以采蘋,南澗之濵。于以采藻,于彼行潦。}{\footnotesize 蘋,大蓱也。濵,涯也。藻,聚藻也。行潦,流潦也。箋云古者婦人先嫁三月,祖廟未毀,教于公宮,祖廟既毀,教于宗室,教以婦德、婦言、婦容、婦功,教成之祭,牲用魚,芼用蘋藻,所以成婦順也,此祭祭女所出祖也,法度莫大於四教,是又祭以成之,故舉以言焉。蘋之言賓也,藻之言澡也,婦人之行尚柔順,自潔清,故取名以為戒。}

\begin{quoting}\textbf{馬瑞辰}今按行潦對潢汙言,溝水之流曰衍,雨水之大曰潦 \texttt{lǎo},行與潦為二,猶潢與汙為二。\end{quoting}

\textbf{于以盛之,維筐及筥。于以湘之,維錡及釜。}{\footnotesize 方曰筐,圓曰筥。湘,亨也。錡,釜屬,有足曰錡,無足曰釜。箋云亨蘋藻者於魚湆之中,是鉶羹之芼。}

\textbf{于以奠之,宗室牖下。}{\footnotesize 奠,置也。宗室,大宗之廟也。大夫士祭於宗廟,奠於牖下。箋云牖下,戶牖間之前,祭不於室中者,凡昏事,於女禮設几筵於戶外,此其義也與。宗子主此祭,維君使有司為之。}\textbf{誰其尸之,有齊季女。}{\footnotesize 尸主、齊敬、季少也。蘋藻,薄物也,澗潦,至質也,筐筥錡釜,陋器也,少女,微主也。古之將嫁女者,必先禮之於宗室,牲用魚,芼之以蘋藻。箋云主設羹者季女則非禮也。女將行,父禮之而俟迎者,蓋母薦之,無祭事也。祭禮,主婦設羹,教成之祭更使季女者,成其婦禮也。季女不主魚,魚俎實男子設之,其粢盛蓋以黍稷。}

\section{甘棠}

%{\footnotesize 三章、章三句}

\textbf{甘棠,美召伯也。召伯之教明於南國。}{\footnotesize 召伯,姬姓,名奭,食采于召,作上公,為二伯,後封於燕,此美其為伯之功,故言伯云。}

\begin{quoting}此召伯謂召虎也,見小雅黍苗、大雅崧高諸篇。\end{quoting}

\textbf{蔽芾甘棠,勿翦勿伐,召伯所茇。}{\footnotesize 蔽芾,小貌。甘棠,杜也。翦去、伐擊也。箋云茇,草舍也。召伯聽男女之訟,不重煩勞百姓,止舍小棠之下而聽斷焉,國人被其德,說其化,思其人,敬其樹。}

\textbf{蔽芾甘棠,勿翦勿敗,召伯所憩。}{\footnotesize 憩,息也。}

\textbf{蔽芾甘棠,勿翦勿拜,召伯所說。}{\footnotesize 說,舍也。箋云拜之言拔也。}

\begin{quoting}\textbf{王質}說,或為税,止,詩税意多通用說字。\end{quoting}

\section{行露}

%{\footnotesize 三章、一章三句、二章章六句}

\textbf{行露,召伯聽訟也。衰亂之俗微,貞信之教興,彊暴之男不能侵陵貞女也。}{\footnotesize 衰亂之俗微、貞信之教興者,此殷之末世,周之盛德,當文王與紂之時。}

\textbf{厭浥行露,豈不夙夜,謂行多露。}{\footnotesize 興也。厭浥,濕意也。行,道也。豈不,言有是也。箋云夙,早也。厭浥然濕,道中始有露,謂二月中嫁取時也。言我豈不知當早夜成昏禮與,謂道中之露大多,故不行耳。今彊暴之男以此多露之時,禮不足而強來,不度時之可否,故云然。周禮仲春之月,令會男女之無夫家者,行事必以昏昕。}

\begin{quoting}厭,魯詩、韓詩作湆 \texttt{qì},說文「幽濕也」。\textbf{馬瑞辰}謂疑畏之假借,凡詩上言豈不、豈敢者,下句多言畏,僖二十年左傳引此詩,杜注「言豈不欲早暮而行,懼多露之濡己」,以懼釋畏,似亦訓謂為畏。\end{quoting}

\textbf{誰謂雀無角,何以穿我屋。誰謂女無家,何以速我獄。}{\footnotesize 不思物變而推其類,雀之穿屋,似有角者。速召、獄埆也。箋云女,女彊暴之男變異也。人皆謂雀之穿屋似有角,彊暴之男召我而獄,似有室家之道於我也,物有似而不同,雀之穿屋不以角乃以咮,今彊暴之男召我而獄,不以室家之道於我乃以侵陵,物與事有似而非者,士師所當審也。}\textbf{雖速我獄,室家不足。}{\footnotesize 昏禮,純帛不過五兩。箋云幣可備也,室家不足,謂媒妁之言不和,六禮之來強委之。}

\begin{quoting}左傳襄二十五年「言以足志,文以足言」,杜注「足,猶成也」。\end{quoting}

\textbf{誰謂鼠無牙,何以穿我墉。誰謂女無家,何以速我訟。}{\footnotesize 墉,牆也。視牆之穿,推其類可謂鼠有牙。}\textbf{雖速我訟,亦不女從。}{\footnotesize 不從,終不棄禮而隨此彊暴之男。}

\begin{quoting}陸佃埤雅「鼠有齒而無牙」。\end{quoting}

\section{羔羊}

%{\footnotesize 三章、章四句}

\textbf{羔羊,鵲巢之功致也。召南之國化文王之政,在位皆節儉正直,德如羔羊也。}{\footnotesize 鵲巢之君積行累功,以致此羔羊之化,在位卿大夫競相切化,皆如此羔羊之人。}

\textbf{羔羊之皮,素絲五紽。}{\footnotesize 小曰羔,大曰羊。素,白也。紽,數也。古者素絲以英裘,不失其制,大夫羔裘以居。}\textbf{退食自公,委蛇委蛇。}{\footnotesize 公,公門也。委蛇,行可從迹也。箋云退食,謂減膳也。自,從也,從於公,謂正直順于事也。委蛇,委曲自得之貌,節儉而順,心志定,故可自得也。}

\begin{quoting}五,象交叉之形,\textbf{陳奐}當讀為交午之午,周禮壺涿氏「午貫象齒」,故書午為五,此五、午相通之例。釋文「它,本作佗,或作紽」,小弁傳「佗,加也」。五佗,即交加也。委蛇,韓詩作逶迤,疊韻詞。\end{quoting}

\textbf{羔羊之革,素絲五緎。}{\footnotesize 革,猶皮也。緎,縫也。}\textbf{委蛇委蛇,自公退食。}{\footnotesize 箋云自公退食,猶退食自公。}

\begin{quoting}革,同䙐,玉篇「䙐,裘裏也」,\textbf{馬瑞辰}古者裘皆表其毛,而為之裏以附于革,謂之䙐,詩「羔羊之皮,素絲五紽」,皮言其表也,「羔羊之革,素絲五緎」,革言其裏也,「羔羊之縫,素絲五緫」合言其表與裏也。\end{quoting}

\textbf{羔羊之縫,素絲五緫。}{\footnotesize 縫,言縫殺之大小得其制。緫,數也。}\textbf{委蛇委蛇,退食自公。}

\begin{quoting}緫 \texttt{zōng},細密也,\textbf{陳奐}此傳數字當讀數罟之數 \texttt{cù}。\end{quoting}

\section{殷其靁}

%{\footnotesize 三章、章六句}

\textbf{殷其靁,勸以義也。召南之大夫遠行從政,不遑寧處,其室家能閔其勤勞,勸以義也。}{\footnotesize 召南大夫,召伯之屬。遠行,謂使出邦畿。}

\textbf{殷其靁,在南山之陽。}{\footnotesize 殷,雷聲也。山南曰陽。靁出地奮,震驚百里,山出雲雨,以潤天下。箋云靁以喻號令於南山之陽,又喻其在外也,召南大夫以王命施號令於四方,猶靁殷殷然發聲於山之陽也。}\textbf{何斯違斯,莫敢或遑。}{\footnotesize 何,此君子也。斯此、違去、遑暇也。箋云何乎此君子適居此,復去此,轉行遠,從事於王所命之方,無敢或閒暇時,閔其勤勞。}\textbf{振振君子,歸哉歸哉。}{\footnotesize 振振,信厚也。箋云大夫信厚之君子,為君使,功未成,歸哉歸哉,勸以為臣之義,未得歸也。}

\begin{quoting}廣雅釋詁「或,有也」。\end{quoting}

\textbf{殷其靁,在南山之側。}{\footnotesize 亦在其陰與左右也。}\textbf{何斯違斯,莫敢遑息。}{\footnotesize 息,止也。}\textbf{振振君子,歸哉歸哉。}

\begin{quoting}說文「息,喘也」。\end{quoting}

\textbf{殷其靁,在南山之下。}{\footnotesize 或在其下。箋云下,謂山足。}\textbf{何斯違斯,莫或遑處。}{\footnotesize 處,居也。}\textbf{振振君子,歸哉歸哉。}

\section{摽有梅}

%{\footnotesize 三章、章四句}

\textbf{摽有梅,男女及時也。召南之國被文王之化,男女得以及時也。}

\begin{quoting}\textbf{陳奐}梅、媒聲同,故詩人見梅而起興。\end{quoting}

\textbf{摽有梅,其實七兮。}{\footnotesize 興也。摽,落也。盛極則隋落者梅也,尚在樹者七。箋云興者,梅實尚餘七未落,喻始衰也,謂女二十,春盛而不嫁,至夏則衰。}\textbf{求我庶士,迨其吉兮。}{\footnotesize 吉,善也。箋云我,我當嫁者。庶眾、迨及也。求女之當嫁者之眾士宜及其善時,善時謂年二十,雖夏未大衰。}

\begin{quoting}摽 \texttt{biào}。荀子楊注「士者,未娶妻之稱」。\end{quoting}

\textbf{摽有梅,其實三兮。}{\footnotesize 在者三也。箋云此夏鄉晚,梅之隋落差多,在者餘三耳。}\textbf{求我庶士,迨其今兮。}{\footnotesize 今,急辭也。}

\textbf{摽有梅,頃筐塈之。}{\footnotesize 塈,取也。箋云頃筐取之,謂夏已晚,頃筐取之於地。}\textbf{求我庶士,迨其謂之。}{\footnotesize 不待備禮也,三十之男,二十之女,禮未備則不待禮會而行之者,所以蕃育民人也。箋云謂,勤也,女年二十而無嫁端則有勤望之憂。不待禮會而行之者,謂明年仲春不待以禮會之也。時禮雖不備,相奔不禁。}

\begin{quoting}塈 \texttt{qì}。謂,同會。案頃筐可取,其實無多矣。\end{quoting}

\section{小星}

%{\footnotesize 二章、章五句}

\textbf{小星,惠及下也。夫人無妬忌之行,惠及賤妾,進御於君,知其命有貴賤,能盡其心矣。}{\footnotesize 以色曰妒,以行曰忌。命,謂禮命貴賤。}

\begin{quoting}\textbf{程大昌}此為使臣行役之詩。\end{quoting}

\textbf{嘒彼小星,三五在東。}{\footnotesize 嘒,微貌。小星,眾無名者。三心、五噣,四時更見。箋云眾無名之星隨心噣在天,猶諸妾隨夫人以次序進御於君也。心在東方,三月時也,噣在東方,正月時也,如是終歲列宿更見。}\textbf{肅肅宵征,夙夜在公,寔命不同。}{\footnotesize 肅肅,疾貌。宵夜、征行、寔是也。命不得同於列位也。箋云夙,早也。謂諸妾肅肅然夜行,或早或夜,在於君所。以次序進御者,是其禮命之數不同也。凡妾御於君不當夕。}

\textbf{嘒彼小星,維參與昴。}{\footnotesize 參,伐也。昴,留也。箋云此言眾無名之星亦隨伐留在天。}\textbf{肅肅宵征,抱衾與裯,寔命不猶。}{\footnotesize 衾,被也。裯,禪被也。猶,若也。箋云裯,牀帳也。諸妾夜行,抱衾與牀帳,待進御之次序不若,亦言尊卑異也。}

\section{江有汜}

%{\footnotesize 三章、章五句}

\textbf{江有汜,美媵也,勤而無怨,嫡能悔過也。文王之時,江沱之間有嫡不以其媵備數,媵遇勞而無怨,嫡亦自悔也。}{\footnotesize 勤者,以己宜媵而不得,心望之。}

\textbf{江有汜。}{\footnotesize 興也。決復入為汜。箋云興者,喻江水大,汜水小,然得並流,似嫡媵宜俱行。}\textbf{之子歸,不我以。不我以,其後也悔。}{\footnotesize 嫡能自悔也。箋云之子,是子也,是子,謂嫡也。婦人謂嫁曰歸。以,猶與也。}

\begin{quoting}說文「㠯,用也」,㠯、以古今字。\end{quoting}

\textbf{江有渚。}{\footnotesize 渚,小洲也,水枝成渚。箋云江水流而渚留,是嫡與己異心,使己獨留不行。}\textbf{之子歸,不我與。不我與,其後也處。}{\footnotesize 處,止也。箋云嫡悔過自止。}

\begin{quoting}\textbf{馬瑞辰}蓋江遇渚則分,過渚復合也。\end{quoting}

\textbf{江有沱。}{\footnotesize 沱,江之別者。箋云岷山道江東別為沱。}\textbf{之子歸,不我過。不我過,其嘯也歌。}{\footnotesize 箋云嘯,蹙口而出聲。嫡有所思而為之,既覺自悔而歌。歌者,言其悔過,以自解說也。}

\section{野有死麕}

%{\footnotesize 三章、二章章四句、一章三句}

\textbf{野有死麕,惡無禮也。天下大亂,彊暴相陵,遂成淫風,被文王之化,雖當亂世,猶惡無禮也。}{\footnotesize 無禮者,為不由媒妁,鴈幣不至,劫脅以成昏,謂紂之世。}

\textbf{野有死麕,白茅包之。}{\footnotesize 郊外曰野。包,裹也。凶荒則殺禮,猶有以將之,野有死麕,群田之獲而分其肉。白茅,取潔清也。箋云亂世之民貧,而彊暴之男多行無禮,故貞女之情欲令人以白茅裹束野中田者所分麕肉為禮而來。}\textbf{有女懷春,吉士誘之。}{\footnotesize 懷,思也。春,不暇待秋也。誘,道也。箋云有貞女思仲春以禮與男會,吉士使媒人道成之,疾時無禮而言然。}

\begin{quoting}文選李注「今江東人呼鹿為麕」。儀禮士昏禮「納徵,玄纁、束帛、儷皮」,鄭注「皮,鹿皮也」。\end{quoting}

\textbf{林有樸樕,野有死鹿,白茅純束。}{\footnotesize 樸樕,小木也。野有死鹿,廣物也。純束,猶包之也。箋云樸樕之中及野有死鹿,皆可以白茅包裹束以為禮,廣可用之物,非獨麕也。純,讀如屯。}\textbf{有女如玉。}{\footnotesize 德如玉也。箋云如玉者,取其堅而潔白。}

\begin{quoting}樸樕 \texttt{sù},疊韻詞。\textbf{胡承珙}詩於昏禮每言析薪,古者昏禮或本有薪芻之饋耳。純,同稇 \texttt{kǔn},說文「稇,絭束也」,段注「絭束,謂以繩束之」。\end{quoting}

\textbf{舒而脫脫兮,}{\footnotesize 舒,徐也。脫脫,舒遲也。箋云貞女欲吉士以禮來,脫脫然舒也。又疾時無禮,彊暴之男相劫脅。}\textbf{無感我帨兮,}{\footnotesize 感,動也。帨,佩巾也。箋云奔走失節,動其佩巾。}\textbf{無使尨也吠。}{\footnotesize 尨,狗也。非禮相陵則狗吠。}

\begin{quoting}古而、如、然三字通用。脫 \texttt{tuì}。感,三家詩作撼。說文「尨 \texttt{máng},犬之多毛者」。\end{quoting}

\section{何彼襛矣}

%{\footnotesize 三章、章四句}

\textbf{何彼襛矣,美王姬也。雖則王姬,亦下嫁於諸侯,車服不繫其夫,下王后一等,猶執婦道以成肅雝之德也。}{\footnotesize 下王后一等,謂車乘厭翟,勒面繢緫,服則褕翟。}

\textbf{何彼襛矣,唐棣之華。}{\footnotesize 興也。襛,猶戎戎也。唐棣,栘也。箋云何乎彼戎戎者,乃栘之華,興者,喻王姬顏色之美盛。}\textbf{曷不肅雝,王姬之車。}{\footnotesize 肅敬、雝和。箋云曷何、之往也。何不敬和乎,王姬往乘車也,言其嫁時始乘車則已敬和。}

\textbf{何彼襛矣,華如桃李。平王之孫,齊侯之子。}{\footnotesize 平,正也。武王女,文王孫,適齊侯之子。箋云華如桃李者,興王姬與齊侯之子顏色俱盛。正王者,德能正天下之王。}

\begin{quoting}\textbf{馬瑞辰}詩所云平王之孫,乃平王之外孫,齊侯之子,謂齊侯之女子。\end{quoting}

\textbf{其釣維何,維絲伊緡。齊侯之子,平王之孫。}{\footnotesize 伊維、緡綸也。箋云釣者以此有求於彼,何以為之乎,以絲之為綸,則是善釣也,以言王姬與齊侯之子以善道相求。}

\begin{quoting}\textbf{朱熹}絲之合而為綸,猶男女之合而為昏也。\end{quoting}

\section{騶虞}

%{\footnotesize 二章、章三句}

\textbf{騶虞,鵲巢之應也。鵲巢之化行,人倫既正,朝廷既治,天下純被文王之化,則庶類蕃殖,蒐田以時,仁如騶虞則王道成也。}{\footnotesize 應者,應德自遠而至。}

\textbf{彼茁者葭,}{\footnotesize 茁,出也。葭,蘆也。箋云記蘆始出者,著春田之早晚。}\textbf{壹發五豝,}{\footnotesize 豕牝曰豝。虞人翼五豝,以待公之發。箋云君射一發而翼五豝者,戰禽獸之命,必戰之者,仁心之至。}\textbf{于嗟乎騶虞。}{\footnotesize 騶虞,義獸也,白虎黑文,不食生物,有至信之德則應之。箋云于嗟者,美之也。}

\begin{quoting}\textbf{馬瑞辰}穆天子傳「天子射鳥,有獸在葭中」,是葭亦藏獸之區,詩言葭蓬,皆謂豝豵所藏耳。壹,發語詞。廣雅「獸一歲為豵,二歲為豝,三歲為肩,四歲為特」。魯說、韓說「騶虞,天子掌馬獸官」。\end{quoting}

\textbf{彼茁者蓬,}{\footnotesize 蓬,草名也。}\textbf{壹發五豵,}{\footnotesize 一歲曰豵。箋云豕生三曰豵。}\textbf{于嗟乎騶虞。}

%\begin{flushright}召南之國十四篇、四十章、百七十七句\end{flushright}