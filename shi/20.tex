\chapter{谷風之什詁訓傳第二十}

\section{谷風}

%{\footnotesize 三章、章六句}

\textbf{谷風,刺幽王也。天下俗薄,朋友道絕焉。}

\begin{quoting}後漢書陰皇后紀光武詔書引此詩云「風人之戒,可不慎乎」。\end{quoting}

\textbf{習習谷風,維風及雨。}{\footnotesize 興也。風雨相感,朋友相須。箋云習習,和調之貌。興者,風而有雨則潤澤行,喻朋友同志則恩愛成。}\textbf{將恐將懼,維予與女。}{\footnotesize 箋云將,且也。恐懼,喻遭厄難勤苦之事也。當此之時,獨我與女爾,謂同其憂務。}\textbf{將安將樂,女轉棄予。}{\footnotesize 言朋友趨利,窮達相棄。箋云朋友無大故則不相遺棄,今女以志達而安樂,棄恩忘舊,薄之甚。}

\begin{quoting}\textbf{陳奐}將,猶方也。\textbf{馬瑞辰}與與棄對言,恐懼時獨我好汝,以見昔之厚,安樂時汝轉棄予,以見今之薄。\end{quoting}

\textbf{習習谷風,維風及頹。}{\footnotesize 頹,風之焚輪者也,風薄相扶而上,喻朋友相須而成。}\textbf{將恐將懼,寘予于懷。}{\footnotesize 箋云寘,置也,置我於懷,言至親己也。}\textbf{將安將樂,棄予如遺。}{\footnotesize 箋云如遺者,如人行道遺忘物,忽然不省存也。}

\begin{quoting}爾雅釋天「焚輪謂之頹」,孫炎注「迴風從上下曰頹」。\end{quoting}

\textbf{習習谷風,維山崔嵬。無草不死,無木不萎。}{\footnotesize 崔嵬,山巔也。雖盛夏萬物茂壯,草木無有不死而萎枝者。箋云此言東風生長之風也,山巔之上草木猶及之,然而盛夏養萬物之時,草木枝葉猶有萎槁者,以喻朋友雖以恩相養,亦安能不時有小訟乎。}\textbf{忘我大德,思我小怨。}{\footnotesize 箋云大德,切瑳以道,相成之謂也。}

\section{蓼莪}

%{\footnotesize 六章、四章章四句、二章章八句}

\textbf{蓼莪,刺幽王也。民人勞苦,孝子不得終養爾。}{\footnotesize 不得終養者,二親病亡之時,時在役所,不得見也。}

\textbf{蓼蓼者莪,匪莪伊蒿。}{\footnotesize 興也。蓼蓼,長大貌。箋云莪已蓼蓼長大,貌視之以為非莪,反謂之蒿,興者,喻憂思雖在役中,心不精識其事。}\textbf{哀哀父母,生我劬勞。}{\footnotesize 箋云哀哀者,恨不得終養父母,報其生長己之苦。}

\begin{quoting}蓼 \texttt{lù}。\textbf{馬瑞辰}莪蒿即茵陳蒿之類,常抱宿根而生,有子依母之象,故詩人借以取興,李時珍云「莪抱根叢生,俗謂之抱孃蒿」是也,蒿與蔚皆散生,故詩以喻不能終養。\end{quoting}

\textbf{蓼蓼者莪,匪莪伊蔚。}{\footnotesize 蔚,牡菣也。}\textbf{哀哀父母,生我勞瘁。}{\footnotesize 箋云瘁,病也。}

\textbf{缾之罄矣,維罍之耻。}{\footnotesize 缾小而罍大。罄,盡也。箋云缾小而盡,罍大而盈,言為罍耻者,刺王不使富分貧、眾恤寡。}\textbf{鮮民之生,不如死之久矣。}{\footnotesize 鮮,寡也。箋云此言供養日寡矣,而我尚不得養,恨之言也。}\textbf{無父何怙,無母何恃。出則銜恤,入則靡至。}{\footnotesize 箋云恤憂、靡無也。孝子之心,怙恃父母,依依然以為不可斯須無也,出門則思之而憂,旋入門又不見,如無所至。}

\begin{quoting}左傳昭二十四年引詩曰「王室之不寧,晉之耻也」。鮮民,寡民、孤子。至,親人,說文「親,至也」。\end{quoting}

\textbf{父兮生我,母兮鞠我。拊我畜我,長我育我。顧我復我,出入腹我。}{\footnotesize 鞠養、腹厚也。箋云父兮生我者,本其氣也。畜,起也。育,覆育也。顧,旋視也。復,反覆也。腹,懷抱也。}\textbf{欲報之德,昊天罔極。}{\footnotesize 箋云之,猶是也。我欲報父母是德,昊天乎我心無極。}

\begin{quoting}說文「拊,揗也」,段注「揗,摩也」。孟子「畜君者,好君也」。顧,說文「還視也」,引申為看護。復,同覆,庇護。\textbf{于省吾}古聲有重脣無輕脣,故古讀腹為抱,書召誥「夫知保抱攜持厥婦子」,抑「借曰未知,亦既抱子」,這是西周典籍對於子言保抱或抱之證。\end{quoting}

\textbf{南山烈烈,飄風發發。}{\footnotesize 烈烈然,至難也。發發,疾貌。箋云民人自苦見役,視南山則烈烈然,飄風發發然,寒且疾也。}\textbf{民莫不穀,我獨何害。}{\footnotesize 箋云穀,養也。言民皆得養其父母,我獨何故睹此寒苦之害。}

\textbf{南山律律,飄風弗弗。}{\footnotesize 律律,猶烈烈也。弗弗,猶發發也。}\textbf{民莫不穀,我獨不卒。}{\footnotesize 箋云卒,終也。我獨不得終養父母,重自哀傷也。}

\section{大東}

%{\footnotesize 七章、章八句}

\textbf{大東,刺亂也。東國困於役而傷於財,譚大夫作是詩以告病焉。}{\footnotesize 譚國在東,故其大夫尤苦征役之事也。魯莊公十年,齊師滅譚。}

\textbf{有饛簋飧,有捄棘匕。}{\footnotesize 興也。饛,滿簋貌。飧,熟食,謂黍稷也。捄,長貌。匕,所以載鼎實。棘,赤心也。箋云飧者,客始至主人所致之禮也,凡飧饔餼以其爵等為之牢禮之數陳。興者,喻古者天子施予之恩於天下厚。}\textbf{周道如砥,其直如矢。}{\footnotesize 如砥,貢賦平均也,如矢,賞罰不偏也。}\textbf{君子所履,小人所視。}{\footnotesize 箋云此言古者天子之恩厚也,君子皆法效而履行之,其如砥矢之平,小人又皆視之,共之無怨。}\textbf{睠言顧之,潸焉出涕。}{\footnotesize 睠,反顧也。潸,涕下貌。箋云言,我也。此二事者在乎前世,過而去矣,我從今顧視之,為之出涕,傷今不如古也。}

\begin{quoting}饛 \texttt{méng}。飧,說文「水澆飯也」。捄 \texttt{qiú}。\end{quoting}

\textbf{小東大東,杼柚其空。}{\footnotesize 空,盡也。箋云小也大也,謂賦斂之多少也,小亦於東,大亦於東,言其政偏,失砥矢之道也,譚無他貨,維絲麻爾,今盡杼柚不作也。}\textbf{糾糾葛屨,可以履霜。佻佻公子,行彼周行。}{\footnotesize 佻佻,獨行貌。公子,譚公子也。箋云葛屨,夏屨也。周行,周之列位也。言時財貨盡,雖公子衣屨不能順時,乃夏之葛屨也,今以履霜送轉餫,因見使行周之列位者而發幣焉,言雖困乏,猶不得止也。}\textbf{既往既來,使我心疚。}{\footnotesize 箋云既盡、疚病也。言譚人自虛竭餫送而往,周人則空盡受之,曾無反幣復禮之惠,是使我心傷病也。}

\begin{quoting}\textbf{惠周惕}詩說曰小東大東,言東國之遠近也,魯頌「遂荒大東」,箋「大東,極東」也。杼,說文「機持緯者」,柚,釋文「本又作軸」,說文「滕,機持經者」,段注「滕即軸也」。\textbf{馬瑞辰}既往既來,謂數數往來,疲於道路。\end{quoting}

\textbf{有洌氿泉,無浸穫薪。契契寤歎,哀我憚人。}{\footnotesize 洌,寒意也。側出曰氿泉。穫,艾也。契契,憂苦也。憚,勞也。箋云穫,落,木名也。既伐而析之以為薪,不欲使氿泉浸之,浸之則將濕腐不中用也,今譚大夫契契憂苦而寤歎,哀其民人之勞苦者,亦不欲使周之賦斂小東大東極盡之,極盡之則將困病亦猶是也。}\textbf{薪是穫薪,尚可載也。哀我憚人,亦可息也。}{\footnotesize 載,載乎意也。箋云薪是穫薪者,析是穫薪也。尚,庶幾也,庶幾析是穫薪,可載而歸,蓄之以為家用。哀我勞人,亦可休息,養之以待國事者也。}

\begin{quoting}氿 \texttt{guǐ}。憚,釋文「字亦作癉」。\end{quoting}

\textbf{東人之子,職勞不來。西人之子,粲粲衣服。}{\footnotesize 東人,譚人也。來,勤也。西人,京師人也。粲粲,鮮盛也。箋云職,主也。東人勞苦而不見謂勤,京師人衣服鮮絜而逸豫,言王政偏甚也。自此章以下言周道衰,其不言政偏,則言眾官廢職如是而已。}\textbf{舟人之子,熊羆是裘。}{\footnotesize 舟人,舟楫之人。熊羆是裘,言富也。箋云舟,當作周,裘,當作求,聲相近故也。周人之子,謂周世臣之子孫退在賤官,使搏熊羆,在冥氏、穴氏之職。}\textbf{私人之子,百僚是試。}{\footnotesize 私人,私家人也。是試,用於百官也。箋云此言周衰,群小得志。}

\begin{quoting}來,同徠 \texttt{lài},慰勞。\textbf{于省吾}周人之子,熊羆是求,係指田獵言之。\end{quoting}

\textbf{或以其酒,不以其漿。}{\footnotesize 或醉於酒,或不得漿。}\textbf{鞙鞙佩璲,不以其長。}{\footnotesize 鞙鞙,玉貌。璲,瑞也。箋云佩璲者,以瑞玉為佩,佩之鞙鞙然,居其官職,非其才之所長也,徒美其佩而無其德,刺其素餐。}\textbf{維天有漢,監亦有光。}{\footnotesize 漢,天河也,有光而無所明。箋云監,視也。喻王闓置官司而無督察之實。}\textbf{跂彼織女,終日七襄。}{\footnotesize 跂,隅貌。襄,反也。箋云襄,駕也,駕謂更其肆也,從旦至暮七辰,辰一移,因謂之七襄。}

\begin{quoting}監,鑑也,即鏡。跂 \texttt{qí},段注「頃,頭不正也,隅者,陬隅不正而角,織女三星成三角,言不正也」。\end{quoting}

\textbf{雖則七襄,不成報章。}{\footnotesize 不能反報成章也。箋云織女有織名爾,駕則有西無東,不如人織相反報成文章。}\textbf{睆彼牽牛,不以服箱。}{\footnotesize 睆,明星貌。何鼓謂之牽牛。服,牝服也。箱,大車之箱也。箋云以,用也,牽牛不可用於牝服之箱。}\textbf{東有啟明,西有長庚。}{\footnotesize 日旦出,謂明星為啟明,日既入,謂明星為長庚,庚,續也。箋云啟明、長庚皆有助日之名而無實光也。}\textbf{有捄天畢,載施之行。}{\footnotesize 捄,畢貌,畢,所以掩兔也,何嘗見其可用乎。箋云祭器有畢者,所以助載鼎實,今天畢則施於行列而已。}

\begin{quoting}\textbf{陳奐}報亦反也,反報猶反復。睆 \texttt{huǎn}。何鼓三星在天河北,織女三星在天河南。載,則。\end{quoting}

\textbf{維南有箕,不可以簸揚。維北有斗,不可以挹酒漿。}{\footnotesize 挹,㪺也。}\textbf{維南有箕,載翕其舌。維北有斗,西柄之揭。}{\footnotesize 翕,合也。箋云翕,猶引也,引舌者,謂上星相近。}

\begin{quoting}箕宿四星,形如簸箕。\textbf{馬瑞辰}翕、吸音同通用,故箋訓為引,玉篇引詩正作載吸其舌,箕四星,二為踵,二為舌,其形踵狹而舌廣,故曰載翕其舌,以見其主於收斂也。揭,舉也。\textbf{王先謙}下四句與上四句雖同言箕斗,自分兩義,上刺虛位,下刺斂民也。\end{quoting}

\section{四月}

%{\footnotesize 八章、章四句}

\textbf{四月,大夫刺幽王也。在位貪殘,下國構禍,怨亂並興焉。}

\textbf{四月維夏,六月徂暑。}{\footnotesize 徂,往也。六月,火星中,暑盛而往矣。箋云徂,猶始也。四月立夏矣,至六月乃始盛暑,興人為惡亦有漸,非一朝一夕。}\textbf{先祖匪人,胡寧忍予。}{\footnotesize 箋云匪,非也。寧,猶曾也。我先祖非人乎,人則當知患難,何為曾使我當此亂世乎。}

\begin{quoting}\textbf{王夫之}稗疏曰其云匪人者,猶非他人也,頍弁之詩曰「兄弟匪他」,義與此同,猶言「父母生我,胡俾我瘉」也。\end{quoting}

\textbf{秋日淒淒,百卉具腓。}{\footnotesize 淒淒,涼風也。卉,草也。腓,病也。箋云具,猶皆也。涼風用事而眾草皆病,興貪殘之政行而萬民困病。}\textbf{亂離瘼矣,爰其適歸。}{\footnotesize 離憂、瘼病、適之也。箋云爰,曰也。今政亂,國將有憂病者矣,曰此禍其所之歸乎,言憂病之禍必自之歸為亂。}

\begin{quoting}腓,同痱,病也。瘼 \texttt{mò}。\end{quoting}

\textbf{冬日烈烈,飄風發發。}{\footnotesize 箋云烈烈,猶栗烈也。發發,疾貌。言王為酷虐慘毒之政,如冬日之烈烈矣,其亟急行於天下,如飄風之疾也。}\textbf{民莫不穀,我獨何害。}{\footnotesize 箋云穀,養也。民莫不得養其父母者,我獨何故覩此寒苦之害。}

\textbf{山有嘉卉,侯栗侯梅。}{\footnotesize 箋云嘉善、侯維也。山有美善之草,生於梅栗之下,人取其實,蹂踐而害之,令不得蕃茂,喻上多賦斂,富人財盡而弱民與受困窮。}\textbf{廢為殘賊,莫知其尤。}{\footnotesize 廢,忕也。箋云尤,過也。言在位者貪殘,為民之害,無自知其行之過者,言忕於惡。}

\textbf{相彼泉水,載清載濁。}{\footnotesize 箋云相,視也。我視彼泉水之流,一則清,一則濁,刺諸侯並為惡,曾無一善。}\textbf{我日構禍,曷云能穀。}{\footnotesize 構成、曷逮也。箋云構,猶合集也。曷之言何也。穀,善也。言諸侯日作禍亂之行,何者可謂能善。}

\begin{quoting}\textbf{馬瑞辰}構者,遘之假借,構禍猶云遇禍也。\end{quoting}

\textbf{滔滔江漢,南國之紀。}{\footnotesize 滔滔,大水貌,其神足以綱紀一方。箋云江也漢也,南國之大水,紀理眾川,使不壅滯,喻吳楚之君能長理旁側小國,使得其所。}\textbf{盡瘁以仕,寧莫我有。}{\footnotesize 箋云瘁病、仕事也。今王盡病其封畿之內以兵役之事,使群臣有土地曾無自保有者,皆懼於危亡也,吳楚舊名貪殘,今周之政乃反不如。}

\begin{quoting}盡瘁,與憔悴同義。有,通友,親善也。\end{quoting}

\textbf{匪鶉匪鳶,翰飛戾天。匪鱣匪鮪,潛逃于淵。}{\footnotesize 鶉,鵰也,鵰鳶,貪殘之鳥也。大魚能逃處淵。箋云翰高、戾至、鱣鯉也。言鵰鳶之高飛,鯉鮪之處淵,性自然也,非鵰鳶能高飛,非鯉鮪能處淵,皆驚駭避害爾,喻民性安土重遷,今而逃走,亦畏亂政故。}

\begin{quoting}鶉 \texttt{tuán}。\end{quoting}

\textbf{山有蕨薇,隰有杞桋。}{\footnotesize 杞,枸檵也。桋,赤棘也。箋云此言草木生各得其所,人反不得其所,傷之也。}\textbf{君子作歌,維以告哀。}{\footnotesize 箋云告哀,言勞病而愬之。}

\section{北山}

%{\footnotesize 六章、三章章六句、三章章四句}

\textbf{北山,大夫刺幽王也。役使不均,己勞於從事而不得養其父母焉。}

\textbf{陟彼北山,言采其杞。}{\footnotesize 箋云言,我也。登山而采杞,非可食之物,喻己行役不得其事。}\textbf{偕偕士子,朝夕從事。}{\footnotesize 偕偕,強壯貌。士子,有王事者也。箋云朝夕從事,言不得休止。}\textbf{王事靡盬,憂我父母。}{\footnotesize 箋云靡,無也。盬,不堅固也。王事無不堅固,故我當盡力勤勞於役,久不得歸,父母思己而憂。}

\begin{quoting}說文「偕,彊也」。\end{quoting}

\textbf{溥天之下,莫非王土。率土之濱,莫非王臣。}{\footnotesize 溥大、率循、濱涯也。箋云此言王之土地廣矣,王之臣又眾矣,何求而不得,何使而不行。}\textbf{大夫不均,我從事獨賢。}{\footnotesize 賢,勞也。箋云王不均大夫之使,而專以我有賢才之故,獨使我從事於役,自苦之辭。}

\begin{quoting}溥,先秦引詩皆作普,孟子趙注「普,徧」。賢,本義多財,引申為多,\textbf{王夫之}小爾雅云「我從事獨賢,勞事獨多也」,賢之訓多,與射禮「某賢于某若干純」之賢同義。\end{quoting}

\textbf{四牡彭彭,王事傍傍。}{\footnotesize 彭彭然不得息,傍傍然不得已。}\textbf{嘉我未老,鮮我方將。}{\footnotesize 將,壯也。箋云嘉、鮮皆善也。王善我年未老乎,善我方壯乎,何獨久使我也。}\textbf{旅力方剛,經營四方。}{\footnotesize 旅,眾也。箋云王謂此事眾之氣力方盛乎,何乃勞苦使之經營四方。}

\begin{quoting}旅,同膂,筋力。\end{quoting}

\textbf{或燕燕居息,}{\footnotesize 燕燕,安息貌。}\textbf{或盡瘁事國。}{\footnotesize 盡力勞瘁,以從國事。}\textbf{或息偃在牀,或不已于行。}{\footnotesize 箋云不已,猶不止也。}

\begin{quoting}燕燕,魯詩作宴宴。行,道路。\end{quoting}

\textbf{或不知叫號,或慘慘劬勞。}{\footnotesize 叫呼、號召也。}\textbf{或棲遲偃仰,或王事鞅掌。}{\footnotesize 鞅掌,失容也。箋云鞅,猶何也。掌,謂捧之也。負何捧持以趨走,言促遽也。}

\begin{quoting}慘慘,釋文「字又作懆懆,憂慮貌」。棲遲,陳風衡門傳「遊息也」。\end{quoting}

\textbf{或湛樂飲酒,或慘慘畏咎。}{\footnotesize 箋云咎,猶罪過也。}\textbf{或出入風議,或靡事不為。}{\footnotesize 箋云風,猶放也。}

\begin{quoting}湛,同酖,說文「酖,樂酒也」。\textbf{馬瑞辰}放議,猶放言也。為,作也。\end{quoting}

\section{無將大車}

%{\footnotesize 三章、章四句}

\textbf{無將大車,大夫悔將小人也。}{\footnotesize 周大夫悔將小人。幽王之時,小人眾多,賢者與之從事,反見譖害,自悔與小人並。}

\textbf{無將大車,祇自塵兮。}{\footnotesize 大車,小人之所將也。箋云將,猶扶進也。祇,適也。鄙事者,賤者之所為也,君子為之,不堪其勞,以喻大夫而進舉小人,適自作憂累,故悔之。}\textbf{無思百憂,祇自疧兮。}{\footnotesize 疧,病也。箋云百憂者,眾小事之憂也。進舉小人,使得居位,不任其職,愆負及己,故以眾小事為憂,適自病也。}

\begin{quoting}\textbf{孔疏}冬官車人有車有大車,鄭云「大車,平地任載之車,其車駕牛」。疧 \texttt{zhěn}。\end{quoting}

\textbf{無將大車,維塵冥冥。}{\footnotesize 箋云冥冥者,蔽人目明,令無所見也,猶進舉小人,蔽傷己之功德。}\textbf{無思百憂,不出于熲。}{\footnotesize 熲,光也。箋云思眾小事以為憂,使人蔽闇不得出於光明之道。}

\begin{quoting}\textbf{馬瑞辰}熲音義與耿正同,邶柏舟耿耿不寐傳「耿耿,猶儆儆也」,禮少儀注「熲,警枕也」,儆、警說文並訓戒,不出于熲即謂不出于儆戒之中。\end{quoting}

\textbf{無將大車,維塵雝兮。}{\footnotesize 箋云雝,猶蔽也。}\textbf{無思百憂,祇自重兮。}{\footnotesize 箋云重,猶累也。}

\begin{quoting}雝,說文「字又作壅」。重,同腫。\end{quoting}

\section{小明}

%{\footnotesize 五章、三章章十二句、二章章六句}

\textbf{小明,大夫悔仕於亂世也。}{\footnotesize 名篇曰小明者,言幽王日小其明,損其政事,以至於亂。}

\textbf{明明上天,照臨下土。}{\footnotesize 箋云明明上天,喻王者當光明如日之中也,照臨下土,喻王者當察理天下之事,據時幽王不能然,故舉以刺之。}\textbf{我征徂西,至于艽野。二月初吉,載離寒暑。}{\footnotesize 艽野,遠荒之地。初吉,朔日也。箋云征行、徂往也。我行往之西方,至於遠荒之地,乃以二月朔日始行,至今則更夏暑冬寒矣,尚未得歸。詩人牧伯之大夫,使述其方之事,遭亂世勞苦而悔仕。}\textbf{心之憂矣,其毒大苦。}{\footnotesize 箋云憂之甚,心中如有毒藥也。}\textbf{念彼共人,涕零如雨。}{\footnotesize 箋云共人,靖共爾位以待賢者之君。}\textbf{豈不懷歸,畏此罪罟。}{\footnotesize 罟,網也。箋云懷,思也。我誠思歸,畏此刑罪羅網,我故不敢歸爾。}

\begin{quoting}說文「艽 \texttt{qiǔ},遠荒也」,段注「艽之言究也、窮也」。二月,周正也,夏曆十二月,詩云載離寒暑,即歲暮也。初吉,上旬吉日也。罪罟,即羅網,刑罰也,\textbf{馬瑞辰}說文「罪,捕魚竹網,罟,網也」,秦始以罪易辠,惟此詩罪罟二字平列,猶云網罟。\end{quoting}

\textbf{昔我往矣,日月方除。曷云其還,歲聿云莫。}{\footnotesize 除,除陳生新也。箋云四月為除,昔我往至于艽野以四月,自謂其時將即歸,何言其還,乃至歲晚尚不得歸。}\textbf{念我獨兮,我事孔庶。心之憂矣,憚我不暇。}{\footnotesize 憚,勞也。箋云孔甚、庶眾也。我事獨甚眾,勞我不暇,皆言王政不均,臣事不同也。}\textbf{念彼共人,睠睠懷顧。}{\footnotesize 箋云睠睠,有往仕之志也。}\textbf{豈不懷歸,畏此譴怒。}

\begin{quoting}\textbf{馬瑞辰}除即爾雅「十二月為涂之涂」,戴震曰「廣韻,涂,直魚切,與除同音通用」,方以智曰「謂歲將除也」。\end{quoting}

\textbf{昔我往矣,日月方奧。}{\footnotesize 奧,煖也。}\textbf{曷云其還,政事愈蹙。歲聿云莫,采蕭穫菽。}{\footnotesize 蹙,促也。箋云愈,猶益也。何言其還,乃至於政事更益蹙急,歲晚乃至采蕭獲菽尚不得歸。}\textbf{心之憂矣,自詒伊戚。}{\footnotesize 戚,憂也。箋云詒,遺也。我冒亂世而仕,自遺此憂,悔仕之辭。}\textbf{念彼共人,興言出宿。}{\footnotesize 箋云興,起也,夜臥起宿於外,憂不能宿於內也。}\textbf{豈不懷歸,畏此反覆。}{\footnotesize 箋云反覆,謂不以正罪見罪。}

\textbf{嗟爾君子,無恆安處。}{\footnotesize 箋云恆,常也。嗟女君子,謂其友未仕者也。人之居無常安之處,謂當安安而能遷,孔子曰「鳥則擇木」。}\textbf{靖共爾位,正直是與。神之聽之,式穀以女。}{\footnotesize 靖,謀也。正直為正,能正人之曲曰直。箋云共具、式用、穀善也。有明君謀具女之爵位,其志在於與正直之人為治,神明若祐而聽之,其用善人則必用女,是使聽天任命,不汲汲求仕之辭,言女位者,位無常主,賢人則是。}

\begin{quoting}共,同恭,禮記緇衣、韓詩外傳引詩皆作恭。以,與也。\end{quoting}

\textbf{嗟爾君子,無恆安息。}{\footnotesize 息,猶處也。}\textbf{靖共爾位,好是正直。神之聽之,介爾景福。}{\footnotesize 介、景皆大也。箋云好,猶與也。介,助也。神明聽之,則將助女以大福,謂遭是明君,道施行也。}

\begin{quoting}介,助也。\end{quoting}

\section{鼓鍾}

%{\footnotesize 四章、章五句}

\textbf{鼓鍾,刺幽王也。}

\textbf{鼓鍾將將,淮水湯湯,憂心且傷。}{\footnotesize 幽王用樂不與德比,會諸侯于淮上,鼓其淫樂以示諸侯,賢者為之憂傷。箋云為之憂傷者,嘉樂不野合,犧象不出門,今乃於淮水之上作先王之樂,失禮尤甚。}\textbf{淑人君子,懷允不忘。}{\footnotesize 箋云淑善、懷至也。古者善人君子,其用禮樂,各得其宜,至信不可忘。}

\begin{quoting}\textbf{王引之}允,語詞。\end{quoting}

\textbf{鼓鍾喈喈,淮水湝湝,憂心且悲。}{\footnotesize 喈喈,猶將將。湝湝,猶湯湯。悲,猶傷也。}\textbf{淑人君子,其德不回。}{\footnotesize 回,邪也。}

\textbf{鼓鍾伐鼛,淮有三洲,憂心且妯。}{\footnotesize 鼛,大鼓也。三洲,淮上地。妯,動也。箋云妯之言悼也。}\textbf{淑人君子,其德不猶。}{\footnotesize 猶,若也。箋云猶,當作瘉,瘉,病也。}

\begin{quoting}鼛 \texttt{gāo}。妯,同怞 \texttt{chōu},說文「怞,恨也,詩曰憂心且怞」。\end{quoting}

\textbf{鼓鍾欽欽,鼓瑟鼓琴,笙磬同音。}{\footnotesize 欽欽,言使人樂進也。笙磬,東方之樂也。同音,四縣皆同也。箋云同音者,謂堂上堂下八音克諧。}\textbf{以雅以南,以籥不僭。}{\footnotesize 為雅為南也,舞四夷之樂,大德廣所及也,東夷之樂曰昧,南夷之樂曰南,西夷之樂曰朱離,北夷之樂曰禁,以為籥舞,若是為和而不僭矣。箋云雅,萬舞也,萬也、南也、籥也,三舞不僭,言進退之旅也。周樂尚武,故謂萬舞為雅,雅,正也。籥舞,文樂也。}

\begin{quoting}雅、南皆樂器名,後孳乳為樂調名。籥,亦樂器名。\end{quoting}

\section{楚茨}

%{\footnotesize 六章、章十二句}

\textbf{楚茨,刺幽王也。政煩賦重,田萊多荒,饑饉降喪,民卒流亡,祭祀不饗,故君子思古焉。}{\footnotesize 田萊多荒,茨棘不除也,饑饉,倉庾不盈也,降喪,神不與福助也。}

\textbf{楚楚者茨,言抽其棘。自昔何為,我蓺黍稷。}{\footnotesize 楚楚,茨棘貌。抽,除也。箋云茨,蒺藜也,伐除蒺藜與棘,自古之人何乃勤苦為此事乎,我將樹黍稷焉,言古者先王之政以農為本。茨言楚楚,棘言抽,互辭也。}\textbf{我黍與與,我稷翼翼。我倉既盈,我庾維億。}{\footnotesize 露積曰庾。萬萬曰億。箋云黍與與,稷翼翼,蕃蕪貌,陰陽和、風雨時則萬物成,萬物成則倉庾充滿矣。倉言盈,庾言億,亦互辭,喻多也。十萬曰億也。}\textbf{以為酒食,以享以祀。以妥以侑,以介景福。}{\footnotesize 妥,安坐也。侑,勸也。箋云享獻、介助、景大也。以黍稷為酒食,獻之以祀先祖,既又迎尸,使處神坐而食之,為其嫌不飽,祝以主人之辭勸之,所以助孝子受大福也。}

\begin{quoting}\textbf{馬瑞辰}爾雅釋言「茦,刺」,方言「凡草木刺人,北燕、朝鮮之間謂之茦,自關而西謂之刺,江淮之間謂之棘」,棘為草名,又為凡草刺人之通稱,「楚楚者茨,言抽其棘」,棘即茨上之棘,猶「翹翹錯薪,言刈其楚」,楚即薪中之楚也。\end{quoting}

\textbf{濟濟蹌蹌,絜爾牛羊,以往烝嘗。或剝或亨,或肆或將。}{\footnotesize 濟濟蹌蹌,言有容也。亨,飪之也。肆陳、將齊也。或陳于牙,或齊其肉。箋云有容,言威儀敬慎也。冬祭曰烝,秋祭曰嘗。祭祀之禮,各有其事,有解剝其皮者,有煮熟之者,有肆其骨體於俎者,或奉持而進之者。}\textbf{祝祭于祊,祀事孔明。}{\footnotesize 祊,門內也。箋云孔,甚也。明,猶備也、絜也。孝子不知神之所在,故使祝博求之平生門內之旁,待賓客之處,祀禮於是甚明。}\textbf{先祖是皇,神保是饗。}{\footnotesize 皇大、保安也。箋云皇,暀也。先祖以孝子祀禮甚明之故,精氣歸暀之,其鬼神又安而饗其祭祀。}\textbf{孝孫有慶,報以介福,萬壽無疆。}{\footnotesize 箋云慶,賜也。疆,竟界也。}

\begin{quoting}蹌 \texttt{qiāng}。絜,或同挈。祊 \texttt{bēng},說文「門內祭,先祖所彷徨」。\textbf{朱熹}神保,蓋尸之嘉號,楚辭所謂靈保。說文「獻於神曰享,神食其所享曰饗」。\end{quoting}

\textbf{執爨踖踖,為俎孔碩,或燔或炙。}{\footnotesize 爨,饔爨、廪爨也。踖踖,言爨竈有容也。燔,取膟膋。炙,炙肉也。箋云燔,燔肉也,炙,肝炙也,皆從獻之俎也,其為之於爨,必取肉也肝也肥碩美者。}\textbf{君婦莫莫,為豆孔庶,為賓為客。}{\footnotesize 莫莫,言清靜而敬至也。豆謂肉羞、庶羞也,繹而賓尸及賓客。箋云君婦謂后也,凡適妻稱君婦,事舅姑之稱也。庶,䏧也,祭祀之禮,后夫人主共籩豆,必取肉物肥䏧美者。}\textbf{獻醻交錯,禮儀卒度,笑語卒獲。}{\footnotesize 東西為交,邪行為錯。度,法度也。獲,得時也。箋云始主人酌賓為獻,賓既酌主人,主人又自飲酌賓曰醻,至旅而爵交錯以徧。卒,盡也,古者於旅也語。}\textbf{神保是格,報以介福,萬壽攸酢。}{\footnotesize 格來、酢報也。}

\begin{quoting}踖 \texttt{jí}。莫莫,同慔慔。\end{quoting}

\textbf{我孔熯矣,式禮莫愆。工祝致告,徂賚孝孫。}{\footnotesize 熯,敬也。善其事曰工。賚,予也。箋云我,我孝孫也。式法、莫無、愆過、徂往也。孝孫甚敬矣,於禮法無過者,祝以此故致神意,吿主人使受嘏,既而以嘏之物往予主人。}\textbf{苾芬孝祀,神嗜飲食。卜爾百福,如幾如式。}{\footnotesize 幾期、式法也。箋云卜,予也。苾苾芬芬,有馨香矣,女之以孝敬享祀也,神乃歆嗜女之飲食,今予女之百福,其來如有期矣,多少如有法矣,此皆嘏辭之意。}\textbf{既齊既稷,既匡既勑。永錫爾極,時萬時億。}{\footnotesize 稷疾、勑固也。箋云齊,減取也。稷之言即也。永長、極中也。嘏之禮,祝徧取黍稷牢肉魚擩于醢以授尸,孝孫前就尸受之,天子使宰夫受之以筐,祝則釋嘏辭以勑之。又曰長賜女以中和之福,是萬是億,言多無數。}

\begin{quoting}熯,同謹,\textbf{于省吾}熯即謹之本字,金文覲不从見,勤不从力。工祝,官祝。\textbf{陳奐}齊、稷、匡、勑皆祭司肅敬之意。時,是,即福也。\end{quoting}

\textbf{禮儀既備,鍾鼓既戒。孝孫徂位,工祝致告。}{\footnotesize 致告,告利成也。箋云鍾鼓既戒,戒諸在廟中者以祭禮畢,孝孫往位,堂下西面位也,祝於是致孝孫之意,告尸以利成。}\textbf{神具醉止,皇尸載起。鼓鍾送尸,神保聿歸。}{\footnotesize 皇,大也。箋云具,皆也。皇,君也。載之言則也。尸,節神者也,神醉而尸謖,送尸而神歸,尸出入奏肆夏,尸稱君,尊之也。神安歸者,歸於天也。}\textbf{諸宰君婦,廢徹不遲。}{\footnotesize 箋云廢,去也。尸出而可徹,諸宰徹去諸饌,君婦籩豆而已。不遲,以疾為敬也。}\textbf{諸父兄弟,備言燕私。}{\footnotesize 燕而盡其私恩。箋云祭祀畢,歸賓客之俎,同姓則留與之燕,所以尊賓客、親骨肉也。}

\begin{quoting}戒,吿也。周禮膳夫「凡王祭祀,賓客食,則徹王之胙俎」,孔疏「言諸宰者,以膳夫是宰之屬官」。備,盡也。言,語詞。燕私,私宴。\end{quoting}

\textbf{樂具入奏,以綏後祿。爾肴既將,莫怨具慶。}{\footnotesize 綏,安也,安然後受福祿也。將,行也。箋云燕而祭時之樂,復皆入奏,以安後日之福祿,骨肉歡而君之福祿安,女之殽羞已行,同姓之臣無有怨者而皆慶君,是其歡也。}\textbf{既醉既飽,小大稽首。神嗜飲食,使君壽考。}{\footnotesize 箋云小大,猶長幼也。同姓之臣燕已醉飽,皆再拜稽首曰「神乃歆嗜君之飲食,使君壽且考」,此其慶辭。}\textbf{孔惠孔時,維其盡之。子子孫孫,勿替引之。}{\footnotesize 替廢、引長也。箋云惠,順也。甚順於禮,甚得其時,維君德能盡之,願子孫勿廢而長行之。}

\begin{quoting}孔疏「祭時在廟,燕當在寢,故言祭時之樂皆復來入於寢而奏之」。\textbf{馬瑞辰}廣雅釋詁「將,美也」,爾肴既將,猶頍弁詩「爾肴既嘉」。時,善也。\end{quoting}

\section{信南山}

%{\footnotesize 六章、章六句}

\textbf{信南山,刺幽王也。不能脩成王之業,疆理天下,以奉禹功,故君子思古焉。}

\textbf{信彼南山,維禹甸之。畇畇原隰,曾孫田之。}{\footnotesize 甸,治也。畇畇,墾辟貌。曾孫,成王也。箋云信乎彼南山之野,禹治而丘甸之,今原隰墾辟,則又成王之所佃,言成王乃遠脩禹之功,今王反不脩其業乎。六十四井為甸,甸方八里,居一成之中,成方十里,出兵車一乘,以為賦法。}\textbf{我疆我理,}{\footnotesize 疆,畫經界也。理,分地理也。}\textbf{南東其畝。}{\footnotesize 或南或東。}

\begin{quoting}\textbf{馬瑞辰}「信彼南山」與「節彼南山」、「倬彼甫田」句法相類,節、倬皆為貌,則信亦南山貌也,古伸字借作信。畇 \texttt{yún}。\textbf{于省吾}孫對先祖言,皆可稱曾孫。田,即上文甸也。\end{quoting}

\textbf{上天同雲,雨雪雰雰。}{\footnotesize 雰雰,雪貌,豐年之冬,必有積雪。}\textbf{益之以霡霂,既優既渥,}{\footnotesize 小雨曰霡霂。箋云成王之時,陰陽和,風雨時,冬有積雪,春而益之以小雨,潤澤則饒洽。}\textbf{既霑既足,生我百穀。}

\begin{quoting}爾雅「冬曰上天」。說文「同,合會也」。霡霂 \texttt{mài mù}。\textbf{馬瑞辰}足者浞 \texttt{zhuó} 之省借,說文「浞,小濡貌也」,詩言瀀、渥、霑、足,四者義皆相近,均以言雨澤之霑濡耳。\end{quoting}

\textbf{疆埸翼翼,黍稷彧彧。}{\footnotesize 埸,畔也。翼翼,讓畔也。彧彧,茂盛貌。}\textbf{曾孫之穡,以為酒食。畀我尸賓,壽考萬年。}{\footnotesize 箋云斂穫曰穡。畀,予也。成王以黍稷之稅為酒食,至祭祀齋戒則以賜尸與賓,尊尸與賓,所以敬神也,敬神則得壽考萬年。}

\begin{quoting}說文「大界曰疆,小界曰埸」。\end{quoting}

\textbf{中田有廬,疆埸有瓜,是剝是菹。}{\footnotesize 剝瓜為菹也。箋云中田,田中也,農人作廬焉,以便其田事。於畔上種瓜,瓜成又入其稅,天子剝削淹漬以為菹,貴四時之異物。}\textbf{獻之皇祖,曾孫壽考,受天之祜。}{\footnotesize 箋云皇君、祜福也。獻瓜菹於先祖者,順孝子之心也,孝子則獲福。}

\begin{quoting}菹 \texttt{zū},腌製。\end{quoting}

\textbf{祭以清酒,從以騂牡,享于祖考。}{\footnotesize 周尚赤也。箋云清謂玄酒也。酒,鬱鬯五齊三酒也。祭之禮,先以鬱鬯降神,然後迎牲,享于祖考,納亨時。}\textbf{執其鸞刀,以啟其毛,取其血膋。}{\footnotesize 鸞刀,刀有鸞者,言割中節也。箋云毛以告純也。膋,脂膏也,血以告殺,膋以升臭,合之黍稷,實之於蕭,合馨香也。}

\begin{quoting}騂,赤色,故傳云周尚赤也。膋 \texttt{liáo}。\end{quoting}

\textbf{是烝是享,苾苾芬芬,祀事孔明。}{\footnotesize 烝,進也。箋云既有牲物而進獻之,苾苾芬芬然香,祀禮於是則甚明也。}\textbf{先祖是皇,報以介福,萬壽無疆。}{\footnotesize 箋云皇之言暀也,先祖之靈歸暀是孝孫,而報之以福。}

%\begin{flushright}谷風之什十篇、五十四章、三百五十六句\end{flushright}