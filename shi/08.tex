\chapter{齊雞鳴詁訓傳第八}

\begin{quoting}\textbf{釋文}齊者,太師呂望所封之國也,其地少皞爽鳩氏之墟,在禹貢青州岱嶺之陰、濰淄之野,都營丘之側,禮記云「大公封於營丘」是也。\end{quoting}

\section{雞鳴}

%{\footnotesize 三章、章四句}

\textbf{雞鳴,思賢妃也。哀公荒淫怠慢,故陳賢妃貞女夙夜警戒相成之道焉。}

\textbf{雞既鳴矣,朝既盈矣。}{\footnotesize 雞鳴而夫人作,朝盈而君作。箋云雞鳴朝盈,夫人也君也可以起之常禮。}\textbf{匪雞則鳴,蒼蠅之聲。}{\footnotesize 蒼蠅之聲有似遠雞之鳴。箋云夫人以蠅聲為雞鳴,則起早於常禮,敬也。}

\begin{quoting}書大傳「雞鳴,然後應門擊柝,吿辟也,然後少師奏質明于階下」,鄭注「應門,朝門也,辟,啟也」。則,之也。\end{quoting}

\textbf{東方明矣,朝既昌矣。}{\footnotesize 東方明則夫人纚筓而朝,朝已昌盛則君聽朝。箋云東方明,朝既昌,亦夫人也君也可以朝之常禮。君日出而視朝。}\textbf{匪東方則明,月出之光。}{\footnotesize 見月出之光以為東方明。箋云夫人以月光為東方明,則朝亦敬也。}

\textbf{蟲飛薨薨,甘與子同夢。}{\footnotesize 古之夫人配其君子,亦不忘其敬。箋云蟲飛薨薨,東方且明之時,我猶樂與子臥而同夢,言親愛之無已。}\textbf{會且歸矣,無庶予子憎。}{\footnotesize 會,會於朝。卿大夫朝會於君朝聽政,夕歸治其家事。無庶予子憎,無見惡於夫人。箋云庶,眾也。蟲飛薨薨,所以當起者,卿大夫朝者且罷歸故也,無使眾臣以我故憎惡於子,戒之也。}

\begin{quoting}\textbf{馬瑞辰}庶,幸也,無庶,即庶無之倒文,予、與古今字,予子憎,正義引定本作「與子憎」,與,猶貽也,無庶予子憎,即庶無貽子憎,猶詩言「無父母貽罹」、左傳「無貽寡君羞」也。\end{quoting}

\section{還}

%{\footnotesize 三章、章四句}

\textbf{還,刺荒也。哀公好田獵,從禽獸而無厭,國人化之,遂成風俗,習於田獵謂之賢,閑於馳逐謂之好焉。}{\footnotesize 荒,謂政事廢亂。}

\textbf{子之還兮,遭我乎峱之閒兮。}{\footnotesize 還,便捷之貌。峱,山名。箋云子也我也皆士大夫也,俱出田獵而相遭也。}\textbf{並驅從兩肩兮,揖我謂我儇兮。}{\footnotesize 從,逐也。獸三歲曰肩。儇,利也。箋云並,併也。子也我也併驅而逐二獸,子則揖耦我謂我儇,譽之也,譽之者,以報前言還也。}

\begin{quoting}還,通旋。峱 \texttt{náo}。儇 \texttt{xuān},釋文引韓詩作婘。\end{quoting}

\textbf{子之茂兮,遭我乎峱之道兮。}{\footnotesize 茂,美也。}\textbf{並驅從兩牡兮,揖我謂我好兮。}{\footnotesize 箋云譽之言好者,以報前言茂也。}

\textbf{子之昌兮,遭我乎峱之陽兮。}{\footnotesize 昌,盛也。箋云昌,佼好貌。}\textbf{並驅從兩狼兮,揖我謂我臧兮。}{\footnotesize 狼,獸名。臧,善也。}

\begin{quoting}\textbf{胡承珙}首章舉其大者言之,秦風「奉時辰牡」,則田獵貴牡,故次章舉所貴者言之,陸疏云狼猛捷,自是難獲之獸。\end{quoting}

\section{著}

%{\footnotesize 三章、章三句}

\textbf{著,刺時也。時不親迎也。}{\footnotesize 時不親迎,故陳親迎之禮以刺之。}

\textbf{俟我於著乎而,充耳以素乎而,}{\footnotesize 俟,待也。門屏之間曰著。素,象瑱。箋云我,嫁者自謂也。待我於著,謂從君子而出至於著,君子揖之時也,我視君子則以素為充耳,謂所以縣瑱者,或名為紞,織之,人君五色,臣則三色而已,此言素者,目所先見而云。}\textbf{尚之以瓊華乎而。}{\footnotesize 瓊華,美石,士之服也。箋云尚,猶飾也。飾之以瓊華者,謂縣紞之末,所謂瑱也,人君以玉為之。瓊華,石色似瓊也。}

\begin{quoting}俟,同竢,說文「竢,待也」。著,同宁,爾雅釋宮「門屏之間謂之宁」。乎而,語詞。充耳,冠飾,由紞、纊、瑱組成。\textbf{姚際恆}瓊,赤玉,貴者用之,華、瑩、英取協韻,以贊其玉之色澤也。\end{quoting}

\textbf{俟我於庭乎而,充耳以靑乎而,}{\footnotesize 青,青玉。箋云待我於庭,謂揖我於庭時。靑,紞之靑。}\textbf{尚之以瓊瑩乎而。}{\footnotesize 瓊瑩,石似玉,卿大夫之服也。箋云石色似瓊、似瑩也。}

\textbf{俟我於堂乎而,充耳以黃乎而,}{\footnotesize 黃,黃玉。箋云黃,紞之黃。}\textbf{尚之以瓊英乎而。}{\footnotesize 瓊英,美石似玉者,人君之服也。箋云瓊英,猶瓊華也。}

\section{東方之日}

%{\footnotesize 二章、章五句}

\textbf{東方之日,刺衰也。君臣失道,男女淫奔,不能以禮化也。}

\textbf{東方之日兮,彼姝者子,在我室兮。}{\footnotesize 興也。日出東方,人君明盛,無不照察也。姝者,初昏之貌。箋云言東方之日者,愬之乎耳,有姝姝美好之子來在我室,欲與我為室家,我無如之何也。日在東方,其明未融,興者,喻君不明。}\textbf{在我室兮,履我即兮。}{\footnotesize 履,禮也。箋云即,就也。在我室者,以禮來我則就之,與之去也,言今者之子不以禮來也。}

\begin{quoting}\textbf{馬瑞辰}古人喻人顏色之美,多取譬於日月,宋玉神女賦「其始出也,耀乎若白日初出照屋梁,其少進也,皎若明月舒其光」,義本此詩。說文「履,足所依也」,段注「引伸之訓踐」。即,同膝,古人席地而坐,親近則躡彼膝足也。\end{quoting}

\textbf{東方之月兮,彼姝者子,在我闥兮。}{\footnotesize 月盛於東方。君明於上,若日也,臣察於下,若月也。闥,門內也。箋云月以興臣,月在東方,亦言不明。}\textbf{在我闥兮,履我發兮。}{\footnotesize 發,行也。箋云以禮來則我行而與之去。}

\begin{quoting}\textbf{王先謙}切言之,則闥為小門,渾言之,則門以內皆為闥,故毛傳但云「闥,門內也」。發,足也。\end{quoting}

\section{東方未明}

%{\footnotesize 三章、章四句}

\textbf{東方未明,刺無節也。朝廷興居無節,號令不時,挈壺氏不能掌其職焉。}{\footnotesize 號令,猶召呼也。挈壺氏,掌漏刻者。}

\textbf{東方未明,顛倒衣裳。}{\footnotesize 上曰衣,下曰裳。箋云挈壺氏失漏刻之節,東方未明而以為明,故群臣促遽,顛倒衣裳。群臣之朝,別色始入。}\textbf{顛之倒之,自公召之。}{\footnotesize 箋云自,從也。群臣顛倒衣裳而朝,人又從君所來而召之,漏刻失節,君又早興。}

\textbf{東方未晞,顛倒裳衣。}{\footnotesize 晞,明之始升。}\textbf{倒之顛之,自公令之。}{\footnotesize 令,告也。}

\begin{quoting}晞,同昕,說文「昕,且明,日將出也」。\end{quoting}

\textbf{折柳樊圃,狂夫瞿瞿。}{\footnotesize 柳,柔脆之木。樊,藩也。圃,菜園也。折柳以為藩園,無益於禁矣。瞿瞿,無守之貌。古者有挈壺氏以水火分日夜,以告時於朝。箋云柳木之不可以為藩,猶是狂夫不任挈壺氏之事。}\textbf{不能辰夜,不夙則莫。}{\footnotesize 辰時、夙早、莫晚也。箋云此言不任其事者恆失節數也。}

\begin{quoting}荀子楊注「瞿瞿,瞪視之貌」。\end{quoting}

\section{南山}

%{\footnotesize 四章、章六句}

\textbf{南山,刺襄公也。鳥獸之行,淫乎其妹,大夫遇是惡,作詩而去之。}{\footnotesize 襄公之妹,魯桓公夫人文姜也,襄公素與淫通,及嫁,公讁之,公與夫人如齊,夫人愬之襄公,襄公使公子彭生乘公而搤殺之。夫人久留於齊,莊公即位後乃來,猶復會齊侯于禚、于祝丘,又如齊師。齊大夫見襄公行惡如是,作詩以刺之,又非魯桓公不能禁制夫人而去之。}

\textbf{南山崔崔,雄狐綏綏。}{\footnotesize 興也。南山,齊南山也。崔崔,高大也。國君尊嚴如南山崔崔然,雄狐相隨綏綏然無別,失陰陽之匹。箋云雄狐行求匹耦於南山之上,形貌綏綏然,興者,喻襄公居人君之尊而為淫泆之行,其威儀可耻惡如狐。}\textbf{魯道有蕩,齊子由歸。}{\footnotesize 蕩,平易也。齊子,文姜也。箋云婦人謂嫁曰歸。言文姜既以禮從此道嫁于魯侯也。}\textbf{既曰歸止,曷又懷止。}{\footnotesize 懷,思也。箋云懷,來也。言文姜既曰嫁于魯侯矣,何復來為乎,非其來也。}

\begin{quoting}綏綏,韓詩作夊夊,玉篇「夊,行遲貌」,見有狐注。止,語詞。此章言襄公懷其妹也。\end{quoting}

\textbf{葛屨五兩,冠緌雙止。}{\footnotesize 葛屨,服之賤者。冠緌,服之尊者。箋云葛屨五兩,喻文姜與姪娣及傅姆同處。冠緌,喻襄公也,五人為奇,而襄公往,從而雙之,冠屨不宜同處,猶襄公文姜不宜為夫婦之道。}\textbf{魯道有蕩,齊子庸止。}{\footnotesize 庸,用也。}\textbf{既曰庸止,曷又從止。}{\footnotesize 箋云此言文姜既用此道嫁於魯侯,襄公何復送而從之為淫泆之行。}

\begin{quoting}五,通伍。兩,古緉字,說文「緉,履兩枚也」,段注「齊風『葛屨五兩』,履必兩而後成用也,是謂之緉」。緌 \texttt{ruí}。首二句言賤者貴者皆有匹偶也。此章言文姜往從其兄也。\end{quoting}

\textbf{蓺麻如之何,衡從其畝。}{\footnotesize 蓺,樹也。衡獵之,從獵之,種之然後得麻。箋云樹麻者必先耕治其田,然後樹之,以言人君取妻必先議於父母。}\textbf{取妻如之何,必告父母。}{\footnotesize 必告父母廟。箋云取妻之禮,議於生者,卜於死者,此之謂告。}\textbf{既曰告止,曷又鞠止。}{\footnotesize 鞠,窮也。箋云鞠,盈也。魯侯女既告父母而取,何復盈從令至于齊乎,非魯桓。}

\begin{quoting}齊民要術「凡種麻耕不厭其熟,縱橫七徧以上則麻無葉也」。此下二章皆非魯桓也。\end{quoting}

\textbf{析薪如之何,匪斧不克。}{\footnotesize 克,能也。箋云此言析薪必待斧乃能也。}\textbf{取妻如之何,匪媒不得。}{\footnotesize 箋云此言取妻必待媒乃得也。}\textbf{既曰得止,曷又極止。}{\footnotesize 極,至也。箋云女既以媒得之矣,何不禁制而恣極其邪意令至齊乎,又非魯桓。}

\section{甫田}

%{\footnotesize 三章、章四句}

\textbf{甫田,大夫刺襄公也。無禮義而求大功,不脩德而求諸侯,志大心勞,所以求者非其道也。}

\textbf{無田甫田,維莠驕驕。}{\footnotesize 興也。甫,大也。大田過度而無人功,終不能獲。箋云興者,喻人君欲立功致治,必勤身脩德,積小以成高大。}\textbf{無思遠人,勞心忉忉。}{\footnotesize 忉忉,憂勞也。箋云言無德而求諸侯,徒勞其心忉忉耳。}

\begin{quoting}田,同畋,說文「畋,平田也」。說文「甫,男子之美稱也」,段注「凡男子皆得稱之,以男子始冠之稱引申為始也,又引申為大也」。莠 \texttt{yǒu},即狗尾草。驕驕,韓詩作喬喬。\end{quoting}

\textbf{無田甫田,維莠桀桀。}{\footnotesize 桀桀,猶驕驕也。}\textbf{無思遠人,勞心怛怛。}{\footnotesize 怛怛,猶忉忉也。}

\begin{quoting}說文「怛,憯也,憯 \texttt{cǎn},痛也」。\end{quoting}

\textbf{婉兮孌兮,緫角丱兮。未幾見兮,突而弁兮。}{\footnotesize 婉孌,少好貌。緫角,聚兩髦也。丱,幼稚也。弁,冠也。箋云人君內善其身,外脩其德,居無幾何,可以立功,猶是婉孌之童子少自脩飾,丱然而稚,見之無幾何,突耳加冠為成人也。}

\section{盧令}

%{\footnotesize 三章、章二句}

\textbf{盧令,刺荒也。襄公好田獵畢弋而不脩民事,百姓苦之,故陳古以風焉。}{\footnotesize 畢,噣也。弋,繳射也。}

\textbf{盧令令,其人美且仁。}{\footnotesize 盧,田犬。令令,纓環聲。言人君能有美德,盡其仁愛,百姓欣而奉之,愛而樂之,順時遊田,與百姓共其樂,同其獲,故百姓聞而說之,其聲令令然。}

\begin{quoting}戰國策「韓國盧,天下之駿犬也」。\end{quoting}

\textbf{盧重環,}{\footnotesize 重環,子母環也。}\textbf{其人美且鬈。}{\footnotesize 鬈,好貌。箋云鬈,讀當為權,權,勇壯也。}

\textbf{盧重鋂,}{\footnotesize 鋂,一環貫二也。}\textbf{其人美且偲。}{\footnotesize 偲,才也。箋云才,多才也。}

\begin{quoting}鋂 \texttt{méi}。說文「偲 \texttt{cāi},彊力也」。\end{quoting}

\section{敝笱}

%{\footnotesize 三章、章四句}

\textbf{敝笱,刺文姜也。齊人惡魯桓公微弱,不能防閑文姜,使至淫亂,為二國患焉。}

\textbf{敝笱在梁,其魚魴鰥。}{\footnotesize 興也。鰥,大魚。箋云鰥,魚子也。魴也鰥也魚之易制者,然而敝敗之笱不能制,興者,喻魯桓微弱,不能防閑文姜,終其初時之婉順。}\textbf{齊子歸止,其從如雲。}{\footnotesize 如雲,言盛也。箋云其從,姪娣之屬。言文姜初嫁于魯桓之時,其從者之心意如雲然,雲之行順風耳,後知魯桓微弱,文姜遂淫恣,從者亦隨之為惡。}

\begin{quoting}鰥,三家詩作鯤,即草魚。\end{quoting}

\textbf{敝笱在梁,其魚魴鱮。}{\footnotesize 魴鱮,大魚。箋云鱮,似魴而弱鱗。}\textbf{齊子歸止,其從如雨。}{\footnotesize 如雨,言多也。箋云如雨,言無常,天下之則下,天不下則止,以言姪娣之善惡亦文姜所使止。}

\begin{quoting}鱮 \texttt{xù},即鰱魚。\end{quoting}

\textbf{敝笱在梁,其魚唯唯。}{\footnotesize 唯唯,出入不制。箋云唯唯,行相隨順之貌。}\textbf{齊子歸止,其從如水。}{\footnotesize 水,喻眾也。箋云水之性可停可行,亦言姪娣之善惡在文姜也。}

\section{載驅}

%{\footnotesize 四章、章四句}

\textbf{載驅,齊人刺襄公也。無禮義故盛其車服,疾驅於通道大都,與文姜淫,播其惡於萬民焉。}{\footnotesize 故,猶端也。}

\textbf{載驅薄薄,簟茀朱鞹。}{\footnotesize 薄薄,疾驅聲也。簟,方文蓆也。車之蔽曰茀。諸侯之路車,有朱革之質而羽飾。箋云此車襄公乃乘焉而來,與文姜會也。}\textbf{魯道有蕩,齊子發夕。}{\footnotesize 發夕,自夕發至旦。箋云襄公既無禮義,乃疾驅其乘車以入魯竟,魯之道路平易,文姜發夕由之往會焉,曾無慚耻之色。}

\begin{quoting}鞹 \texttt{kuò}。\textbf{王先謙}韓說曰「發,旦也」,齊子旦夕,猶言朝見暮見,即久處之義。\end{quoting}

\textbf{四驪濟濟,垂轡濔濔。}{\footnotesize 四驪,言物色盛也。濟濟,美貌。垂轡,轡之垂者。濔濔,眾也。箋云此又刺襄公乘是四驪而來,徒為淫亂之行。}\textbf{魯道有蕩,齊子豈弟。}{\footnotesize 言文姜於是樂易然。箋云此豈弟猶言發夕也,豈,讀當為闓,弟,古文尚書以弟為圛,圛,明也。}

\begin{quoting}濔 \texttt{nǐ},轡垂貌。爾雅釋言「闓圛,發也」,\textbf{王先謙}謂齊子留連久處之後,至開明乃發行耳。\end{quoting}

\textbf{汶水湯湯,行人彭彭。}{\footnotesize 湯湯,大貌。彭彭,多貌。箋云汶水之上蓋有都焉,襄公與文姜時所會。}\textbf{魯道有蕩,齊子翱翔。}{\footnotesize 翱翔,猶彷徉也。}

\textbf{汶水滔滔,行人儦儦。}{\footnotesize 滔滔,流貌。儦儦,眾貌。}\textbf{魯道有蕩,齊子遊敖。}

\section{猗嗟}

%{\footnotesize 三章、章六句}

\textbf{猗嗟,刺魯莊公也。齊人傷魯莊公有威儀技藝,然而不能以禮防閑其母,失子之道,人以為齊侯之子焉。}

\textbf{猗嗟昌兮,頎而長兮。}{\footnotesize 猗嗟,歎辭。昌,盛也。頎,長貌。箋云昌,佼好貌。}\textbf{抑若揚兮,}{\footnotesize 抑,美色。揚,廣揚。}\textbf{美目揚兮。}{\footnotesize 好目揚眉。}\textbf{巧趨蹌兮,射則臧兮。}{\footnotesize 蹌,巧趨貌。箋云臧,善也。}

\begin{quoting}\textbf{陳奐}猗嗟,疊韻。\textbf{馬瑞辰}按懿、抑古通用,抑詩外傳作懿是也,釋詁、詩烝民傳皆曰「懿,美也」。揚,韓詩作陽,曰「眉上曰陽」,\textbf{皮錫瑞}陽者,陽明之處也,今俗呼額角之側亦謂太陽,即同此義,然則自眉以及額角皆得為陽也。蹌 \texttt{qiàng}。射則,即射藝,下「舞則」同。\end{quoting}

\textbf{猗嗟名兮,美目清兮。}{\footnotesize 目上為名,目下為清。}\textbf{儀既成兮,終日射侯,不出正兮,展我甥兮。}{\footnotesize 二尺曰正。外孫曰甥。箋云成,猶備也。正,所以射於侯中者,天子五正,諸侯三正,大夫二正,士一正,外皆居其侯中參分之一焉。展,誠也。姊妹之子曰甥。容貌技藝如此,誠我齊之甥。言誠者,拒時人言齊侯之子。}

\begin{quoting}\textbf{馬瑞辰}名、明古通用,名,當讀明,明亦昌盛之義,三章首句皆贊美其容貌之盛大。\end{quoting}

\textbf{猗嗟孌兮,}{\footnotesize 孌,壯好貌。}\textbf{清揚婉兮。}{\footnotesize 婉,好眉目也。}\textbf{舞則選兮,射則貫兮。}{\footnotesize 選齊、貫中也。箋云選者,謂於倫等最上。貫,習也。}\textbf{四矢反兮,以禦亂兮。}{\footnotesize 四矢,乘矢。箋云反,復也,禮射三而止,每射四矢,皆得其故處,此之謂復。射必四矢者,象其能禦四方之亂。}

\begin{quoting}韓詩「言其舞則應雅樂也」。\textbf{陳奐}貫,訓中。\end{quoting}

%\begin{flushright}齊國十一篇、三十四章、百四十三句\end{flushright}