\chapter{陳宛丘詁訓傳第十二}

\begin{quoting}\textbf{釋文}陳者,胡公媯滿之所封也,其先虞舜之冑有虞遏父者,為周陶正,武王賴其器用與其神明之後,妻以元女,其子滿乃封於陳,以備三恪,其地宓羲之墟,在古豫州之界、宛丘之側。\end{quoting}

\section{宛丘}

%{\footnotesize 三章、章四句}

\textbf{宛丘,刺幽公也。淫荒昬亂,游蕩無度焉。}

\textbf{子之湯兮,宛丘之上兮。}{\footnotesize 子,大夫也。湯,蕩也。四方高、中央下曰宛丘。箋云子者,斥幽公,游蕩無所不為。}\textbf{洵有情兮,而無望兮。}{\footnotesize 洵,信也。箋云此君信有淫荒之情,其威儀無可觀望而則俲。}

\begin{quoting}\textbf{陳奐}陳有宛丘,猶之鄭有洧淵,皆是國人游觀之所。\end{quoting}

\textbf{坎其擊鼓,宛丘之下。}{\footnotesize 坎坎,擊鼓聲。}\textbf{無冬無夏,值其鷺羽。}{\footnotesize 值,持也。鷺鳥之羽可以為翳。箋云翳,舞者所持以指麾。}

\textbf{坎其擊缶,宛丘之道。}{\footnotesize 盎謂之缶。}\textbf{無冬無夏,值其鷺翿。}{\footnotesize 翿,翳也。}

\begin{quoting}翿 \texttt{dào}。\end{quoting}

\section{東門之枌}

%{\footnotesize 三章、章四句}

\textbf{東門之枌,疾亂也。幽公淫荒,風化之所行,男女棄其舊業,亟會於道路,歌舞於市井爾。}

\textbf{東門之枌,宛丘之栩。}{\footnotesize 枌,白榆也。栩,杼也。國之交會,男女之所聚。}\textbf{子仲之子,婆娑其下。}{\footnotesize 子仲,陳大夫氏。婆娑,舞也。箋云之子,男子也。}

\textbf{穀旦于差,南方之原。}{\footnotesize 穀,善也。原,大夫氏。箋云旦明、于曰、差擇也。朝日善明曰相擇矣,以南方原氏之女可以為上處。}\textbf{不績其麻,市也婆娑。}{\footnotesize 箋云績麻者,婦人之事也,疾其今不為。}

\begin{quoting}穀旦,吉日。差 \texttt{chāi}。原,高平之地。\textbf{朱熹}既差擇善旦以會于南方之原,於是棄其業以舞於市而往會也。\end{quoting}

\textbf{穀旦于逝,越以鬷邁。}{\footnotesize 逝往、鬷數、邁行也。箋云越於、鬷揔也。朝旦善明曰往矣,謂之所會處也,於是以揔行,欲男女合行。}\textbf{視爾如荍,貽我握椒。}{\footnotesize 荍,芘芣也。椒,芬香也。箋云男女交會而相說,曰「我視女之顏色美如芘芣之華然」,女乃遺我一握之椒,交情好也。此本淫亂之所由。}

\begin{quoting}\textbf{陳奐}越,讀同粵,爾雅「粵,于也」,采蘩、采蘋、擊鼓皆云「于以」,此云「越以」,皆合二字為發語之詞。鬷 \texttt{zōng}。荍 \texttt{qiáo},錦葵。\end{quoting}

\section{衡門}

%{\footnotesize 三章、章四句}

\textbf{衡門,誘僖公也。愿而無立志,故作是詩以誘掖其君也。}{\footnotesize 誘,進也。掖,扶持也。}

\textbf{衡門之下,可以棲遲。}{\footnotesize 衡門,橫木為門,言淺陋也。棲遲,遊息也。箋云賢者不以衡門之淺陋則不遊息於其下,以喻人君不可以國小則不興治致政化。}\textbf{泌之洋洋,可以樂飢。}{\footnotesize 泌,泉水也。洋洋,廣大也。樂飢,可以樂道忘飢。箋云飢者,不足於食也。泌水之流洋洋然,飢者見之,可飲以療飢,以喻人君愨愿,任用賢臣則政教成,亦猶是也。}

\begin{quoting}泌 \texttt{bì}。樂,魯詩韓詩作療。\end{quoting}

\textbf{豈其食魚,必河之魴。豈其取妻,必齊之姜。}{\footnotesize 箋云此言何必河之魴然後可食,取其口美而已,何必大國之女然後可妻,亦取貞順而已,以喻君任臣何必聖人,亦取忠孝而已。齊,姜姓。}

\textbf{豈其食魚,必河之鯉。豈其取妻,必宋之子。}{\footnotesize 箋云宋,子姓。}

\begin{quoting}案上二章,非魴鯉之為美,實以河為貴也。\end{quoting}

\section{東門之池}

%{\footnotesize 三章、章四句}

\textbf{東門之池,刺時也。疾其君子淫昬而思賢女以配君子也。}

\textbf{東門之池,可以漚麻。}{\footnotesize 興也。池,城池也。漚,柔也。箋云於池中柔麻,使可緝績作衣服,興者,喻賢女能柔順君子,成其德教。}\textbf{彼美淑姬,可與晤歌。}{\footnotesize 晤,遇也。箋云晤,猶對也。言淑姬賢女,君子宜與對歌相切化也。}

\begin{quoting}說文「漚,久漬也」。淑,當作叔,叔姬,猶孟姜也。孔疏「傳以晤為遇,亦為對偶之義」。\end{quoting}

\textbf{門之池,可以漚紵。彼美淑姬,可與晤語。}

\begin{quoting}紵 \texttt{zhù}。\end{quoting}

\textbf{東門之池,可以漚菅。彼美淑姬,可與晤言。}{\footnotesize 言,道也。}

\section{東門之楊}

%{\footnotesize 二章、章四句}

\textbf{東門之楊,刺時也。昬姻失時,男女多違,親迎,女猶有不至者也。}

\textbf{東門之楊,其葉牂牂。}{\footnotesize 興也。牂牂然盛貌。言男女失時,不逮秋冬。箋云楊葉牂牂,三月中也,興者,喻時晚也,失仲春之月。}\textbf{昬以為期,明星煌煌。}{\footnotesize 期而不至也。箋云親迎之禮以昏時,女留他色,不肯時行,乃至大星煌煌然。}

\begin{quoting}牂牂 \texttt{zāng},齊詩作將將,爾雅「將,大也」。明星,啓明星也。\end{quoting}

\textbf{東門之楊,其葉肺肺。}{\footnotesize 肺肺,猶牂牂也。}\textbf{昬以為期,明星晢晢。}{\footnotesize 晢晢,猶煌煌也。}

\begin{quoting}肺 \texttt{pèi},同巿,說文「巿,艸木盛巿巿然,讀若輩」。\end{quoting}

\section{墓門}

%{\footnotesize 二章、章六句}

\textbf{墓門,刺陳佗也。陳佗無良師傅,以至於不義,惡加於萬民焉。}{\footnotesize 不義者,謂弒君而自立。}

\textbf{墓門有棘,斧以斯之。}{\footnotesize 興也。墓門,墓道之門。斯,析也。幽間希行,用生此棘薪,維斧可以開析之。箋云興者,喻陳佗由不覩賢師良傅之訓道,至䧟於誅絕之罪。}\textbf{夫也不良,國人知之。}{\footnotesize 夫,傅相也。箋云良,善也。陳佗之師傅不善,群臣皆知之,言其罪惡著也。}\textbf{知而不已,誰昔然矣。}{\footnotesize 昔,久也。箋云已,猶去也。誰昔,昔也。國人皆知其有罪惡而不誅退,終致禍難,自古昔之時常然。}

\begin{quoting}\textbf{馬瑞辰}天問王逸注「晉大夫解居父聘吳,過陳之墓門」,墓門,蓋陳之城門。\textbf{陳奐}已,止也,國人皆知之,知之而不能救止也。誰昔,三家詩訓疇昔。\end{quoting}

\textbf{墓門有梅,有鴞萃止。}{\footnotesize 梅,柟也。鴞,惡聲之鳥也。萃,集也。箋云梅之樹善惡自耳,徒以鴞集其上而鳴,人則惡之,樹因惡矣,以喻陳佗之性本未必惡,師傅惡而陳佗從之而惡。}\textbf{夫也不良,歌以訊之。}{\footnotesize 訊,告也。箋云歌,謂作此詩也。既作,又使工歌之,是謂之告。}\textbf{訊予不顧,顛倒思予。}{\footnotesize 箋云予,我也。歌以告之,女不顧念我言,至於破滅顛倒之急,乃思我之言,言其晚也。}

\begin{quoting}梅,魯詩作棘,\textbf{馬瑞辰}棘、梅二木,美惡大小不相類,非詩取興之旨,古梅杏之梅,古文作槑,與棘形相近,蓋棘譌作槑。訊 \texttt{suì}。之,魯詩韓詩作止,語詞。\end{quoting}

\section{防有鵲巢}

%{\footnotesize 二章、章四句}

\textbf{防有鵲巢,憂讒賊也。宣公多信讒,君子憂懼焉。}

\textbf{防有鵲巢,邛有旨苕。}{\footnotesize 興也。防,邑也。邛,丘也。苕,草也。箋云防之有鵲巢,邛之有美苕,處勢自然,興者,喻宣公信多言之人,故致此讒人。}\textbf{誰侜予美,心焉忉忉。}{\footnotesize 侜,張誑也。箋云誰,誰讒人也。女眾讒人,誰侜張誑,欺我所美之人乎,使我心忉忉然。所美,謂宣公。}

\begin{quoting}防,堤也。\textbf{馬瑞辰}鵲巢宜於林木,今言防有,非其所應有也,不應有而以為有,所以為讒言也,苕生於下濕,今詩言邛有者,亦以喻讒言之不可信。侜 \texttt{zhōu},語有侜張。\end{quoting}

\textbf{中唐有甓,邛有旨鷊。}{\footnotesize 中,中庭也。唐,堂塗也。甓,令適也。鷊,綬草也。}\textbf{誰侜予美,心焉惕惕。}{\footnotesize 惕惕,猶忉忉也。}

\begin{quoting}甓 \texttt{pì},磚瓦。鷊 \texttt{yì},鋪地錦。\end{quoting}

\section{月出}

%{\footnotesize 三章、章四句}

\textbf{月出,刺好色也。在位不好德而說美色焉。}

\textbf{月出皎兮。}{\footnotesize 興也。皎,月光也。箋云興者,喻婦人有美色之白皙。}\textbf{佼人僚兮,舒窈糾兮。}{\footnotesize 僚,好貌。舒,遲也。窈糾,舒之姿也。}\textbf{勞心悄兮。}{\footnotesize 悄,憂也。箋云思而不見則憂。}

\begin{quoting}佼,同姣。窈糾 \texttt{yǎo jiǎo}。\end{quoting}

\textbf{出皓兮。佼人懰兮,舒懮受兮。勞心慅兮。}

\begin{quoting}玉篇「懮 \texttt{yōu} 受,舒遲之貌」。說文「慅 \texttt{cǎo},動也」,段注「月出『勞心慅兮』,常武『徐方繹騷』傳曰『騷,動也』,此謂騷即慅之假借字也」。\end{quoting}

\textbf{出照兮。佼人燎兮,舒夭紹兮。勞心慘兮。}

\begin{quoting}\textbf{胡承珙}文選西京賦「要紹修態」注「要紹,謂嬋娟作姿容也」,南都賦「要紹便娟」,要紹皆與夭紹同。慘,當作懆,即躁字。\end{quoting}

\section{株林}

%{\footnotesize 二章、章四句}

\textbf{株林,刺靈公也。淫乎夏姬,驅馳而往,朝夕不休息焉。}{\footnotesize 夏姬,陳大夫妻,夏徵舒之母,鄭女也。徵舒字子南,夫字御叔。}

\textbf{胡為乎株林,從夏南。}{\footnotesize 株林,夏氏邑也。夏南,夏徵舒也。箋云陳人責靈公「君何為之株林,從夏氏子南之母為淫泆之行」。}\textbf{匪適株林,從夏南。}{\footnotesize 箋云匪,非也。言我非之株林,從夏氏子南之母為淫泆之行,自之他耳,觝拒之辭。}

\begin{quoting}說文「郊外謂之野,野外謂之林」。\end{quoting}

\textbf{駕我乘馬,說于株野。乘我乘駒,朝食于株。}{\footnotesize 大夫乘駒。箋云我,國人。我,君也。君親乘君乘馬,乘君乘駒,變易車乘以至株林,或說舍焉,或朝食焉,又責之也。馬六尺以下曰駒。}

\begin{quoting}\textbf{王先謙}靈公初往夏氏,必託為遊株林,自株林至株野乃税其駕,然後微服入株邑,此詩乃實賦其事也。\end{quoting}

\section{澤陂}

%{\footnotesize 三章、章六句}

\textbf{澤陂,刺時也。言靈公君臣淫於其國,男女相說,憂思感傷焉。}{\footnotesize 君臣淫於國,謂與孔寧、儀行父也。感傷,謂涕泗滂沱。}

\textbf{彼澤之陂,有蒲與荷。}{\footnotesize 興也。陂,澤障也。荷,芙蕖也。箋云蒲,柔滑之物。芙蕖之莖曰荷,生而佼大。興者,蒲以喻所說男之性,荷以喻所說女之容體也,正以陂中二物興者,喻淫風由同姓生。}\textbf{有美一人,傷如之何。}{\footnotesize 傷無禮也。箋云傷,思也。我思此美人,當如之何而得見之。}\textbf{寤寐無為,涕泗滂沱。}{\footnotesize 自目曰涕,自鼻曰泗。箋云寤,覺也。}

\begin{quoting}陂 \texttt{bēi}。傷,魯詩韓詩作陽,爾雅「陽,予也」。\end{quoting}

\textbf{彼澤之陂,有蒲與蕳。}{\footnotesize 蕳,蘭也。箋云蕳,當作蓮,蓮,芙蕖實也。蓮以喻女之言信。}\textbf{有美一人,碩大且卷。}{\footnotesize 卷,好貌。}\textbf{寤寐無為,中心悁悁。}{\footnotesize 悁悁,猶悒悒也。}

\begin{quoting}釋文「卷,本又作婘」。悁 \texttt{yuān}。\end{quoting}

\textbf{彼澤之陂,有蒲菡萏。}{\footnotesize 菡萏,荷華也。箋云華以喻女之顏色。}\textbf{有美一人,碩大且儼。}{\footnotesize 儼,矜莊貌。}\textbf{寤寐無為,輾轉伏枕。}

\begin{quoting}菡萏 \texttt{hàn dàn}。儼,韓詩作㜝,重頤也。\end{quoting}

%\begin{flushright}陳國十篇、二十六章、百二十四句\end{flushright}