\chapter{卷之四\hspace{1ex}詩五言}

\section{擬古九首}

\begin{quoting}\textbf{其一}\end{quoting}

\textbf{榮榮窗下蘭,密密堂前柳。初與君別時,不謂行當久。出門萬里客,中道逢嘉友。未言心相醉,不在接杯酒。蘭枯柳亦衰,遂令此言負。多謝諸少年,相知不中厚。意氣傾人命,離隔復何有。}

\begin{quoting}\textbf{其二}\end{quoting}

\textbf{辭家夙嚴駕,當往志無終\footnote{無終:代指田疇,疇字子泰,漢北平無終人,為幽州牧劉虞至長安奔問行在。}。問君今何行,非商復非戎。聞有田子泰,節義為士雄。斯人久已死,鄉里習其風。生有高世名,既沒傳無窮。不學狂馳子,直在百年中。}

\begin{quoting}\textbf{其三}\end{quoting}

\textbf{仲春遘時雨,始雷發東隅\footnote{仲春始雷:言劉裕二月舉義兵。}。眾蟄各潛駭,草木從橫舒。翩翩新來燕,雙雙入我廬。先巢故尚在,相將還舊居\footnote{四句言安帝先被遷尋陽,後復辟返京師。}。自從分別來,門庭日荒蕪。我心固匪石,君情定何如。}

\begin{quoting}\textbf{其四}\end{quoting}

\textbf{迢迢百尺樓,分明望四荒。暮作歸雲宅,朝為飛鳥堂。山河滿目中,平原獨茫茫。古時功名士,慷慨爭此場。一旦百歲後,相與還北邙。松柏為人伐,高墳互低昂。頹基無遺主,遊魂在何方。榮華誠足貴,亦復可憐傷。}

\begin{quoting}\textbf{其五}\end{quoting}

\textbf{東方有一士,被服常不完。三旬九遇食\footnote{說苑立節:子思居衛,貧甚,三旬而九食。},十年著一冠。辛苦無此比,常有好容顏。我欲觀其人,晨去越河關。青松夾路生,白雲宿簷端。知我故來意,取琴為我彈。上絃驚別鶴,下絃操孤鸞。願留就君住,從今至歲寒。}

\begin{quoting}\textbf{其六}\end{quoting}

\textbf{蒼蒼谷中樹,冬夏常如茲。年年見霜雪,誰謂不知時\footnote{湯注:前四句興而比,以言吾有定見而不為談者所眩,似為白蓮社中人也。}。厭聞世上語,結友到臨淄。稷下多談士,指彼決吾疑。裝束既有日,已與家人辭。行行停出門,還坐更自思。不怨道里長,但畏人我欺。萬一不合意,永為世笑之。伊懷難具道,為君作此詩。}

\begin{quoting}\textbf{其七}\end{quoting}

\textbf{日暮天無雲,春風扇微和。佳人美清夜,達曙酣且歌。歌竟長歎息,持此感人多。皎皎雲間月,灼灼葉中華。豈無一時好,不久當如何。}

\begin{quoting}\textbf{其八}\end{quoting}

\textbf{少時壯且厲,撫劍獨行遊。誰言行遊近,張掖至幽州。飢食首陽薇,渴飲易水流。不見相知人,惟見古時丘。路邊兩高墳,伯牙與莊周。此士難再得,吾行欲何求。}

\begin{quoting}\textbf{其九}\end{quoting}

\textbf{種桑長江邊\footnote{桑為晉瑞,而江邊非種桑之地。},三年\footnote{劉裕立恭帝於戊午,逼禪於庚申,凡三年。}望當採。枝條始欲茂,忽值山河改。柯葉自摧折,根株浮滄海\footnote{柯葉枝條,蓋指司馬休之之事。}。春蠶既無食,寒衣欲誰待。本不植高原,今日復何悔。}

\section{雜詩十二首}

\begin{quoting}\textbf{其一}\end{quoting}

\textbf{人生無根蒂,飄如陌上塵。分散逐風轉,此已非常身。落地為兄弟,何必骨肉親。得歡當作樂,斗酒聚比鄰。盛年不重來,一日難再晨。及時當勉勵,歲月不待人。}

\begin{quoting}\textbf{其二}\end{quoting}

\textbf{白日淪西河,素月出東嶺。遙遙萬里輝,蕩蕩空中景。風來入房戶,夜中枕席冷。氣變悟時易,不眠知夕永。欲言無予和,揮杯勸孤影。日月擲人去,有志不獲騁。念此懷悲悽,終曉不能靜。}

\begin{quoting}\textbf{其三}\end{quoting}

\textbf{榮華難久居,盛衰不可量。昔為三春蕖,今作秋蓮房。嚴霜結野草,枯悴未遽央\footnote{黃文煥:半死半生之況,尤為慘戚,未遽央三字,添得味長。}。日月有環周,我去不再陽。眷眷往昔時,憶此斷人腸。}

\begin{quoting}\textbf{其四}\end{quoting}

\textbf{丈夫志四海,我願不知老。親戚共一處,子孫還相保。觴絃肆朝日\footnote{朝日:當作朝夕。},罇中酒不燥。緩帶盡歡娛,起晚眠常早。孰若當世士,冰炭滿懷抱。百年歸丘壟,用此空名道。}

\begin{quoting}\textbf{其五}\end{quoting}

\textbf{憶我少壯時,無樂自欣豫。猛志逸四海,騫翮思遠翥。荏苒歲月頹,此心稍已去。值歡無復娛,每每多憂慮。氣力漸衰損,轉覺日不如。壑舟\footnote{壑舟:喻時光飛逝。}無須臾,引我不得住。前塗當幾許,未知止泊處。古人惜寸陰,念此使人懼。}

\begin{quoting}\textbf{其六}\end{quoting}

\textbf{昔聞長老言,掩耳每不喜。奈何五十年,忽已親此事。求我盛年歡,一毫無復意。去去轉欲遠,此生豈再值。傾家持作樂,竟此歲月駛。有子不留金,何用身後置。}

\begin{quoting}\textbf{其七}\end{quoting}

\textbf{日月不肯遲,四時相催迫。寒風拂枯條,落葉掩長陌。弱質與運頹,玄鬢早已白。素標\footnote{素標:白髮。}插人頭,前塗漸就窄。家為逆旅舍\footnote{列子仲尼:處吾之家,如逆旅之舍。},我如當去客。去去欲何之,南山有舊宅。}

\begin{quoting}\textbf{其八}\end{quoting}

\textbf{代耕\footnote{孟子萬章:下士與庶人同祿,祿足以代其耕也。}本非望,所業在田桑。躬親未曾替,寒餒常糟糠。豈期過滿腹,但願飽粳糧。御冬足大布,粗絺以應陽。正爾不能得,哀哉亦可傷。人皆盡獲宜,拙生失其方。理也可奈何,且為陶一觴。}

\begin{quoting}\textbf{其九}\end{quoting}

\textbf{遙遙從羇役,一心處兩端\footnote{黃文煥:身在途而心在家也。}。掩淚泛東逝,順流追時遷。日沒星與昴\footnote{星與昴:代指宵征,見詩小星。},勢翳西山巔。蕭條隔天涯,惆悵念常飡。慷慨思南歸,路遐無由緣。關梁難虧替,絕音寄斯篇。}

\begin{quoting}\textbf{其十}\end{quoting}

\textbf{閑居執蕩志,時駛不可稽。驅役無停息,軒裳逝東崖。泛舟擬董司\footnote{擬:當作詣。董司:當指劉裕。},寒氣激我懷。歲月有常御,我來淹已彌。慷慨憶綢繆,此情久已離。荏苒經十載,暫為人所羇。庭宇翳餘木,倏忽日月虧。}

\begin{quoting}\textbf{其十一}\end{quoting}

\textbf{我行未云遠,回顧慘風涼。春燕應節起,高飛拂塵梁。邊雁悲無所,代謝歸北鄉。離鵾鳴清池,涉暑經秋霜。愁人難為辭,遙遙春夜長。}

\begin{quoting}\textbf{其十二}\end{quoting}

\textbf{嫋嫋松標雀,婉孌柔童子。年始三五間,喬柯何可倚。養色含津氣,粲然有心理。}

\section{詠貧士七首}

\begin{quoting}\textbf{其一}\end{quoting}

\textbf{萬族各有託,孤雲獨無依。曖曖空中滅,何時見餘暉。朝霞開宿霧,眾鳥相與飛。遲遲出林翮,未夕復來歸。量力守故轍,豈不寒與飢。知音茍不存,已矣何所悲。}

\begin{quoting}\textbf{其二}\end{quoting}

\textbf{凄厲歲云暮,擁褐曝前軒。南圃無遺秀\footnote{遺秀:餘穗。},枯條盈北園。傾壺絕餘瀝,闚竈不見煙。詩書塞座外,日昃不遑研。閑居非陳厄,竊有慍見言。何以慰吾懷,賴古多此賢。}

\begin{quoting}\textbf{其三}\end{quoting}

\textbf{榮叟老帶索,欣然方彈琴。原生納決履,清歌暢商音\footnote{韓詩外傳:原憲居魯,子貢往見之,原憲應門,振襟則肘見,納履則踵決,子貢曰「嘻!先生何病也」,憲曰「憲貧也,非病也,仁義之匿,車馬之節,憲不忍為也」,子貢慚,不辭而去,憲乃徐步曳杖,歌商頌而返,聲淪於天地,如出金石。}。重華去我久,貧士世相尋。弊襟不掩肘,藜羹常乏斟\footnote{墨子非儒:藜羹不糂 \texttt{sǎn}。呂覽引作斟。說文「糂,以米和羹也,古文糂從參」。晉書庾袞傳:歲大饑,藜羹不糝。}。豈忘襲輕裘,茍得非所欽。賜也徒能辯,乃不見吾心。}

\begin{quoting}\textbf{其四}\end{quoting}

\textbf{安貧守賤者,自古有黔婁。好爵吾不縈,厚饋吾不酬。一旦壽命盡,蔽覆仍不周。豈不知其極,非道故無憂。從來將千載,未復見斯儔。朝與仁義生,夕死復何求。}

\begin{quoting}\textbf{其五}\end{quoting}

\textbf{袁安困積雪,邈然不可干。阮公見錢入,即日棄其官。芻藁有常溫\footnote{古注:芻藁本供馬食,而貧者藉之以眠,故曰有常溫也。},採莒\footnote{何綽:莒,疑作稆,採莒二字史傳常見。}足朝飡。豈不實辛苦,所懼非飢寒。貧富常交戰\footnote{韓非子:子夏曰「吾入見先王之義,出見富貴,二者交戰於胸」。},道勝無戚顏。至德冠邦閭,清節映西關\footnote{邦閭、西關:分指袁安、阮公。}。}

\begin{quoting}\textbf{其六}\end{quoting}

\textbf{仲蔚\footnote{高士傳:張仲蔚者,平陵人也,隱身不仕,善屬文,好詩賦,常居窮素,所處蓬蒿沒人,時人莫識,唯劉龔知之。}愛窮居,遶宅生蒿蓬。翳然絕交游,賦詩頗能工。舉世無知者,止有一劉龔。此士胡獨然,實由罕所同。介焉安其業,所樂非窮通\footnote{莊子讓王:古之得道者,窮亦樂,通亦樂,所樂非窮通也。}。人事固以拙,聊得長相從。}

\begin{quoting}\textbf{其七}\end{quoting}

\textbf{昔在黃子廉,彈冠佐名州。一朝辭吏歸,清貧略難儔。年飢感仁妻,泣涕向我流。丈夫雖有志,固為兒女憂。惠孫一晤歎,腆贈竟莫酬。誰云固窮難,邈哉此前修。}

\section{詠二疏}

\textbf{大象轉四時,功成者自去。借問衰周來,幾人得其趣。游目漢廷中,二疏\footnote{二疏:疏廣為太子太傅,兄子受為太子少傅,宣帝時乞骸骨,送者車數百輛,歸鄉里,日令家供具設酒食,請族人與相娛樂。}復此舉。高嘯返舊居,長揖儲君傅。餞送傾皇朝,華軒盈道路。離別情所悲,餘榮何足顧。事勝感行人,賢哉豈常譽。厭厭閭里歡,所營非近務。促席延故老,揮觴道平素。問金終寄心\footnote{言因子孫託人問尚餘金有幾所,廣曰云云。},清言曉未悟。放意樂餘年,遑恤身後慮。誰云其人亡,久而道彌著。}

\section{詠三良}

\textbf{彈冠乘通津,但懼時我遺。服勤盡歲月,常恐功愈微。忠情謬獲露,遂為君所私。出則陪文輿,入必侍丹帷。箴規嚮已從,計議初無虧。一朝長逝後,願言同此歸。厚恩固難忘,君命安可違。臨穴罔惟疑,投義志攸希。荊棘籠高墳,黃鳥聲正悲。良人不可贖,泫然沾我衣。}

\section{詠荊軻}

\textbf{燕丹善養士,志在報強嬴。招集百夫良,歲暮得荊卿。君子死知己,提劍出燕京。素驥鳴廣陌,慷慨送我行。雄髮指危冠,猛氣衝長纓。飲餞易水上,四座列群英。漸離擊悲筑,宋意唱高聲。蕭蕭哀風逝,淡淡寒波生。商音更流涕,羽奏壯士驚。心去知不歸,且有後世名。登車何時顧,飛蓋入秦庭。凌厲越萬里,逶迤過千城。圖窮事自至,豪主正怔營。惜哉劍術疏,奇功遂不成。其人雖已沒,千載有餘情。}

\section{讀山海經十三首}

\begin{quoting}\textbf{其一}\end{quoting}

\textbf{孟夏草木長,遶屋樹扶疏。眾鳥欣有託,吾亦愛吾廬。既耕亦已種,時還讀我書。窮巷隔深轍,頗迴故人車。歡然酌春酒,摘我園中蔬。微雨從東來,好風與之俱。泛覽周王傳,流觀山海圖。俯仰終宇宙,不樂復何如。}

\begin{quoting}\textbf{其二}\end{quoting}

\textbf{玉臺凌霞秀,王母怡妙顏。天地共俱生,不知幾何年。靈化無窮已,館宇非一山\footnote{郭璞注:王母亦自有離宮別館,不專住一山也。}。高酣發新謠\footnote{穆天子傳載王母為穆王作謠云:白雲在天,丘陵自出,道里悠遠,山川間之,將子無死,尚復能來。},寧效俗中言。}

\begin{quoting}\textbf{其三}\end{quoting}

\textbf{迢遞槐江嶺,是謂玄圃丘。西南望崑墟,光氣難與儔。亭亭明玕照,落落清瑤流。恨不及周穆,託乘一來遊。}

\begin{quoting}\textbf{其四}\end{quoting}

\textbf{丹木生何許,迺在峚山陽。黃花復朱實,食之壽命長。白玉凝素液,瑾瑜發奇光。豈伊君子寶,見重我軒黃。}

\begin{quoting}\textbf{其五}\end{quoting}

\textbf{翩翩三青鳥,毛色奇可憐。朝為王母使,暮歸三危山。我欲因此鳥,具向王母言:在世無所須,惟酒與長年。}

\begin{quoting}\textbf{其六}\end{quoting}

\textbf{逍遙蕪皋上,杳然望扶木。洪柯百萬尋,森散覆暘谷。靈人侍丹池,朝朝為日浴。神景\footnote{神景:指日。}一登天,何幽不見燭。}

\begin{quoting}\textbf{其七}\end{quoting}

\textbf{粲粲三珠樹,寄生赤水陰。亭亭凌風桂,八幹共成林。靈鳳撫雲舞,神鸞調玉音。雖非世上寶,爰得王母心。}

\begin{quoting}\textbf{其八}\end{quoting}

\textbf{自古皆有沒,何人得靈長\footnote{世說新語黜免:簡文更答曰「若晉室靈長,明公便宜奉行此詔,如大運去矣,請避賢路」。}。不死復不老,萬歲如平常。赤泉給我飲,員丘足我糧。方與三辰\footnote{三辰:日月星。}遊,壽考豈渠央\footnote{渠央:遽央。}。}

\begin{quoting}\textbf{其九}\end{quoting}

\textbf{夸父誕宏志,乃與日競走。俱至虞淵\footnote{虞淵:禺谷,日所入也。}下,似若無勝負。神力既殊妙,傾河焉足有。餘迹寄鄧林,功竟在身後。}

\begin{quoting}\textbf{其十}\end{quoting}

\textbf{精衛銜微木,將以填滄海。刑天舞干戚,猛志故常在。同物既無慮,化去不復悔\footnote{言精衛、刑天皆有死,生既有為,死不復悔也。}。徒設在昔心,良晨詎可待\footnote{昔:通夕。陸機挽歌「大暮安可晨」可與此句互參。}。}

\begin{quoting}\textbf{其十一}\end{quoting}

\textbf{臣危\footnote{山海經海內西經:貳負之臣曰危,危與貳負殺窫窳 \texttt{yǎ yǔ},帝乃梏之疏屬之山,桎其右足,反縛兩手與髮,繫之山上木。}肆威暴,欽駓\footnote{山海經西山經:鍾山之子曰鼓,其狀人面而龍身,是與欽駓殺葆江於崑崙之陽,帝乃戮之鍾山之東,欽化為大鶚,鼓亦化為鵕。}違帝旨。窫窳強能變,祖江遂獨死。明明上天鑒,為惡不可履。長枯\footnote{枯:當作梏。}固已劇,鵕鶚豈足恃。}

\begin{quoting}\textbf{其十二}\end{quoting}

\textbf{鴟鴸\footnote{山海經南山經:柜山有鳥,其狀如鴟而人手,名曰鴸,見則其縣多放士。}見城邑,其國有放士。念彼懷王\footnote{懷王:楚懷王也。}世,當時數來止。青丘有奇鳥,自言獨見爾。本為迷者生,不以喻君子。}

\begin{quoting}\textbf{其十三}\end{quoting}

\textbf{巖巖\footnote{詩節南山:節彼南山,維石巖巖,赫赫師尹,民具爾瞻。}顯朝市,帝者慎用才。何以廢共鯀,重華為之來。仲父獻誠言,姜公乃見猜\footnote{仲父、姜公:指管仲、齊桓事。}。臨沒告飢渴,當復何及哉。}

\section{擬挽歌辭三首}

\textbf{有生必有死,早終非命促。昨暮同為人,今旦在鬼錄。魂氣散何之,枯形寄空木。嬌兒索父啼,良友撫我哭。得失不復知,是非安能覺。千秋萬歲後,誰知榮與辱。但恨在世時,飲酒不得足。}

\textbf{在昔無酒飲,今但湛空觴。春醪生浮蟻\footnote{浮蟻:酒熟,糟浮酒而似蟻。},何時更能嘗。肴案盈我前,親舊哭我傍。欲語口無音,欲視眼無光。昔在高堂寢,今宿荒草鄉。荒草無人眠,極視正茫茫。一朝出門去,歸來良未央。}

\textbf{荒草何茫茫,白楊亦蕭蕭。嚴霜九月中,送我出遠郊。四面無人居,高墳正嶕嶢。馬為仰天鳴,風為自蕭條。幽室一已閉,千年不復朝。千年不復朝,賢達無奈何。向來相送人,各已還其家。親戚或餘悲,他人亦已歌\footnote{論語述而:子於是日哭,則不歌。}。死去何所道,託體同山阿。}

\section{聯句}

\textbf{鳴雁乘風飛,去去當何極。念彼窮居士,如何不歎息。}{\footnotesize 淵明}

\textbf{雖欲騰九萬,扶搖竟無力。遠招王子喬,雲駕庶可飭。}{\footnotesize 愔之}

\textbf{顧侶正徘徊,離離翔天側。霜露豈不切,徒愛雙飛翼。}{\footnotesize 循之}

\textbf{高柯擢條幹,遠眺同天色。思絕慶未看,徒使生迷惑。}{\footnotesize 淵明}