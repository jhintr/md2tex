\chapter{卷之二\hspace{1ex}詩五言}

\section{形影神\hspace{1ex}{\footnotesize 并序}}

\begin{quoting}貴賤賢愚,莫不營營以惜生,斯甚惑焉,故極陳形影之苦,言神辨自然以釋之,好事君子,共取其心焉。\end{quoting}

\begin{quoting}\textbf{形贈影}\end{quoting}

\textbf{天地長不沒,山川無改時。草木得常理,霜露榮悴之。謂人最靈智,獨復不如茲。適見在世中,奄去靡歸期。奚覺無一人,親識豈相思。但餘平生物,舉目情悽洏。我無騰化術,必爾不復疑。願君取吾言,得酒莫茍辭。}

\begin{quoting}\textbf{影答形}\end{quoting}

\textbf{存生不可言,衛生每苦拙。誠願遊崑華,邈然茲道絕。與子相遇來,未嘗異悲悅\footnote{黃文煥:形笑影亦笑,形哭影亦哭,悲悦二字善狀。}。憩蔭若暫乖,止日終不別。此同既難常,黯爾俱時滅。身沒名亦盡,念之五情熱。立善有遺愛,胡可不自竭。酒云能消憂,方此詎不劣。}

\begin{quoting}\textbf{神釋}\end{quoting}

\textbf{大鈞\footnote{如淳:陶者作器於鈞上,此以造化為大鈞。}無私力,萬理自森著。人為三才中,豈不以我故。與君雖異物,生而相依附。結託善惡同,安得不相語。三皇大聖人,今復在何處。彭祖受永年,欲留不得住。老少同一死,賢愚無復數。日醉或能忘,將非促齡具。立善常所欣,誰當為汝譽。甚念傷吾生,正宜委運去。縱浪大化中,不喜亦不懼。應盡便須盡,無復獨多慮。}

\section{九日閑居\hspace{1ex}{\footnotesize 并序}}

\begin{quoting}余閑居,愛重九之名,秋菊盈園,而持醪靡由,空服九華\footnote{九華:重九之華,即菊花。},寄懷於言。\end{quoting}

\textbf{世短意恆多,斯人樂久生。日月依辰至,舉俗愛其名\footnote{魏文帝書云:九為陽數而日月並應,俗嘉其名,以為宜於長久。}。露淒暄風息,氣澈天象明。往燕無遺影,來雁有餘聲。酒能祛百慮,菊為制頹齡\footnote{晉傅統妻菊花頌:爰采爰拾,投之醇酒,服之延年,佩之黃耇。}。如何蓬廬士,空視時運傾。塵爵恥虛罍\footnote{詩蓼莪:缾之罄矣,維罍之耻。},寒華徒自榮。斂襟獨閑謠,緬焉起深情。棲遲固多娛,淹留豈無成。}

\section{歸園田居五首}

\textbf{少無適俗韻,性本愛丘山。誤落塵網中,一去三十年\footnote{三十年:乃十年之誇詞,十而言三十,古有其例。}。羈鳥戀舊林,池魚思故淵。開荒南野際,守拙歸園田。方宅十餘畝,草屋八九間。榆柳蔭後簷,桃李羅堂前。曖曖遠人村,依依墟里煙。狗吠深巷中,雞鳴桑樹巔。戶庭無塵雜,虛室有餘閑。久在樊籠裏,復得返自然。}

\textbf{野外罕人事,窮巷寡輪鞅。白日掩荊扉,對酒絕塵想。時復墟里人,披草共來往。相見無雜言,但道桑麻長。桑麻日已長,我土日已廣。常恐霜霰至,零落同草莽。}

\textbf{種豆南山下\footnote{種豆是即事,亦是用典,漢書楊惲傳:田彼南山,蕪穢不治,種一頃豆,落而為萁,人生行樂耳,須富貴何時。},草盛豆苗稀。晨興理荒穢,帶月荷鋤歸。道狹草木長,夕露沾我衣。衣沾不足惜,但使願無違。}

\textbf{久去山澤遊,浪莽林野娛。試攜子姪輩,披榛步荒墟。徘徊丘壠\footnote{丘壠:墓地。}間,依依昔人居。井竈有遺處,桑竹殘朽株。借問採薪者,此人皆焉如。薪者向我言,死沒無復餘。一世異朝市,此語真不虛。人生似幻化,終當歸空無。}

\textbf{悵恨獨策還,崎嶇歷榛曲。山澗清且淺,遇以濯吾足。漉我新熟酒,隻雞招近局\footnote{近局:近曲、近鄰。}。日入室中闇,荊薪代明燭。歡來苦夕短,已復至天旭。}

\section{遊斜川\hspace{1ex}{\footnotesize 并序}}

\begin{quoting}辛酉正月五日,天氣澄和,風物閑美,與二三鄰曲同遊斜川,臨長流,望曾城\footnote{曾城:層城,指廬山北之鄣山。},魴鯉躍鱗於將夕,水鷗乘和以飜飛,彼南阜\footnote{南阜:指廬山。}者,名實舊矣,不復乃為嗟歎,若夫曾城,傍無依接,獨秀中臯,遙想靈山,有愛嘉名,欣對不足,率共賦詩,悲日月之遂往,悼吾年之不留,各疏年紀鄉里,以記其時日。\end{quoting}

\textbf{開歲倏五十,吾生行歸休。念之動中懷,及辰為茲遊。氣和天惟澄,班坐依遠流。弱湍馳文魴,閑谷矯鳴鷗。迥澤散遊目,緬然睇曾丘。雖微九重秀,顧瞻無匹儔。提壺接賓侶,引滿更獻酬。未知從今去,當復如此不。中觴縱遙情,忘彼千載憂。且極今朝樂,明日非所求。}

\section{示周續之祖企謝景夷三郎時三人共在城北講禮校書}

\textbf{負痾頹簷下,終日無一欣。藥石有時閑,念我意中人。相去不尋常,道路邈何因。周生述孔業,祖謝響然臻。道喪向千載,今朝復斯聞。馬隊\footnote{蕭統陶淵明傳:三人共在城北講禮,加以讎校,所住公廨,近於馬肆。}非講肆,校書亦已勤。老夫有所愛,思與爾為鄰。願言謝諸子,從我潁水濱\footnote{潁水濱:許由洗耳處。}。}

\section{乞食}

\textbf{飢來驅我去,不知竟何之。行行至斯里,叩門拙言辭。主人解余意,遺贈豈虛來。談諧終日夕,觴至輒傾杯。情欣新知勸,言詠遂賦詩。感子漂母惠,愧我非韓才。銜戢\footnote{銜戢:銜之於口,戢之於心。}知何謝,冥報以相貽。}

\section{諸人共遊周家墓柏下}

\textbf{今日天氣佳,清吹與鳴彈。感彼柏下人,安得不為歡。清歌散新聲,綠酒開芳顏。未知明日事,余襟良已殫。}

\section{怨詩楚調示龐主簿鄧治中}

\textbf{天道幽且遠,鬼神茫昧然。結髮念善事,僶俛六九年。弱冠逢世阻,始室\footnote{禮記曲禮:三十而有室,始理男事。}喪其偏。炎火屢焚如,螟蜮恣中田。風雨縱橫至,收斂不盈廛。夏日抱長飢,寒夜無被眠。造夕思雞鳴,及晨願烏遷。在己何怨天,離憂悽目前。吁嗟身後名,於我若浮煙。慷慨獨悲歌,鍾期信為賢。}

\section{答龐參軍\hspace{1ex}{\footnotesize 并序}}

\begin{quoting}三復來貺,欲罷不能,自爾鄰曲,冬春再交,欵然良對,忽成舊遊,俗諺云「數面成親舊」,況情過此者乎?人事好乖,便當語離,楊公所歎,豈惟常悲,吾抱疾多年,不復為文,本既不豐,復老病繼之,輒依周禮往復之義,且為別後相思之資。\end{quoting}

\textbf{相知何必舊,傾蓋定前言。有客賞我趣,每每顧林園。談諧無俗調,所說聖人篇。或有數㪷酒,閑飲自歡然。我實幽居士,無復東西緣。物新人惟舊,弱毫多所宣。情通萬里外,形跡滯江山。君其愛體素,來會在何年。}

\section{五月旦作和戴主簿}

\textbf{虛舟縱逸棹,回復遂無窮。發歲始俯仰,星紀\footnote{星紀:於辰在丑,此指癸丑。}奄將中。南窗罕悴物,北林榮且豐。神萍\footnote{神萍:雨師,見楚辭天問。}寫時雨,晨色奏景風\footnote{史記律書:景風者,居南方,景者,言陽道竟,故曰景風。}。既來孰不去,人理固有終。居常待其盡,曲肱豈傷沖。遷化或夷險,肆志無窊隆。即事如已高,何必升華嵩。}

\section{連雨獨飲}

\textbf{運生會歸盡,終古謂之然。世間有松喬,於今定何間。故老贈余酒,乃言飲得仙。試酌百情遠,重觴忽忘天。天豈去此哉,任真無所先。雲鶴有奇翼,八表須臾還。自我抱茲獨,僶俛四十年。形骸久已化,心在復何言。}

\section{移居二首}

\textbf{昔欲居南村,非為卜其宅。聞多素心人,樂與數晨夕。懷此頗有年,今日從茲役。弊廬何必廣,取足蔽牀席。鄰曲時時來,抗言談在昔。奇文共欣賞,疑義相與析。}

\textbf{春秋多佳日,登高賦新詩。過門更相呼,有酒斟酌之。農務各自歸,閑暇輒相思。相思則披衣,言笑無厭時。此理將不勝\footnote{理勝蓋晉人常語。},無為忽去茲。衣食當須紀\footnote{紀:理也。},力耕不吾欺。}

\section{和劉柴桑}

\textbf{山澤久見招,胡事乃躊躇。直為親舊故,未忍言索居。良辰入奇懷,挈杖還西廬。荒塗無歸人,時時見廢墟。茅茨已就治,新疇復應畬。谷風轉淒薄,春醪解飢劬。弱女雖非男,慰情良勝無\footnote{案劉程之,字仲思,自號遺民,隱居廬山西林,與周續之、陶淵明等稱「尋陽三隱」,不以妻子為意,故上句因言谷風,又以有女無男為恨,故詩以此解之。}。栖栖世中事,歲月共相疏\footnote{共相疏:我棄世,世亦棄我也。}。耕織稱其用,過此奚所須。去去百年外,身名同翳如。}

\section{酬劉柴桑}

\textbf{窮居寡人用,時忘四運周。櫚庭多落葉,慨然知已秋。新葵鬱北墉,嘉穟養南疇。今我不為樂,知有來歲不。命室攜童弱,良日登遠遊。}

\section{和郭主簿二首}

\textbf{藹藹堂前林,中夏貯清陰。凱風因時來,回飇開我襟。息交遊閑業,臥起弄書琴。園蔬有餘滋,舊榖猶儲今。營己良有極,過足非所欽。舂秫作美酒,酒熟吾自斟。弱子戲我側,學語未成音。此事真復樂,聊用忘華簪。遙遙望白雲,懷古一何深。}

\textbf{和澤同三春,華華\footnote{華華:當作垂垂,漸漸也。}涼秋節。露凝無游氛,天高風景\footnote{風景:當作夙景。}澈。陵岑聳逸峰,遙瞻皆奇絕。芳菊開林耀\footnote{開林耀:當作耀林開,與冠巖列對文。},青松冠巖列。懷此貞秀姿,卓為霜下傑。銜觴念幽人,千載撫爾訣。檢素不獲展,厭厭竟良月。}

\section{於王撫軍座送客}

\textbf{秋日淒且厲,百卉具已腓\footnote{腓:病也、變也。}。爰以履霜節,登高餞將歸。寒氣冒山澤,遊雲倏無依。洲渚四緬邈,風水互乖違。瞻夕欣良讌,離言聿云悲。晨鳥暮來還,懸車\footnote{淮南子:日至悲泉,是謂懸車。}斂餘輝。逝止判殊路,旋駕悵遲遲。目送回舟遠,情隨萬化遺。}

\section{與殷晉安別\hspace{1ex}{\footnotesize 并序}}

\begin{quoting}殷先作晉安南府長史掾,因居潯陽,後作太尉參軍,移家東下,作此以贈。\end{quoting}

\textbf{遊好非少長\footnote{少長:少久。},一遇盡殷勤。信宿酬清話,益復知為親。去歲家南里,薄作少時鄰。負杖肆遊從,淹留忘宵晨。語默自殊勢,亦知當乖分。未謂事已及,興言在茲春。飄飄西來風,悠悠東去雲。山川千里外,言笑難為因。良才不隱世,江湖多賤貧。脫有經過便,念來存故人。}

\section{贈羊長史\hspace{1ex}{\footnotesize 并序}}

\begin{quoting}左軍羊長史銜使秦川,作此與之。\end{quoting}

\textbf{愚生三季後,慨然念黃虞。得知千載外,正賴古人書。賢聖留餘跡,事事在中都。豈忘遊心目,關河不可踰。九域甫已一,逝將理舟輿。聞君當先邁,負痾不獲俱。路若經商山\footnote{商山:秦末四皓隱居所在。},為我少躊躇。多謝綺與甪,精爽\footnote{沃儀仲曰:四皓不肯輕出,元亮不肯終仕,後人即前人精爽也。}今何如。紫芝誰復採\footnote{古今樂錄:四皓隱居,高祖聘之不出,作歌曰「漠漠高山,深谷逶迤,曄曄紫芝,可以療飢,唐虞世遠,吾將安歸,駟馬高蓋,其憂甚大,富貴而畏人兮,不若貧賤之肆志」。},深谷久應蕪。駟馬無貰\footnote{貰 \texttt{shì}:寬縱、免除。}患,貧賤有交娛。清謠結心曲,人乖運見疏。擁懷累代下,言盡意不舒。}

\section{歲暮和張常侍}

\textbf{市朝悽舊人,驟驥\footnote{驟驥:言白駒之過隙也。}感悲泉\footnote{悲泉:日落處。}。明旦非今日,歲暮余何言。素顏斂光潤,白髮一已繁。闊哉秦穆談,旅力豈未愆\footnote{書秦誓:番番良士,旅力既愆,我尚有之。}。向夕長風起,寒雲沒西山。厲厲氣遂嚴,紛紛飛鳥還。民生鮮常在,矧伊愁苦纏。屢闕清酤至,無以樂當年。窮通靡攸慮,顦顇由化遷。撫己有深懷,履運\footnote{履運:言歲暮迎新。}增慨然。}

\section{和胡西曹示顧賊曹}

\textbf{蕤賓\footnote{月令:仲夏之月,律中蕤賓。}五月中,清朝起南颸。不駛亦不遲,飄飄吹我衣。重雲蔽白日,閑雨紛微微。流目視西園,曄曄榮紫葵。於今甚可愛,奈何當復衰。感物願及時,每恨靡所揮\footnote{靡所揮:言無酒可飲。}。悠悠待秋稼,寥落將賒遲\footnote{賒遲:延緩。}。逸想不可淹,猖狂獨長悲。}

\section{悲從弟仲德}

\textbf{銜哀過舊宅,悲淚應心零。借問為誰悲,懷人在九冥。禮服名群從\footnote{群從 \texttt{zòng}:同宗堂親。},恩愛若同生。門前執手時,何意爾先傾。在數竟不免,為山不及成。慈母沉哀疚,二胤纔數齡。雙位\footnote{雙位:夫妻靈位。}委空館,朝夕無哭聲。流塵集虛坐,宿草旅前庭。階除曠遊跡,園林獨餘情。翳然乘化去,終天不復形。遲遲將回步,惻惻悲襟盈。}

