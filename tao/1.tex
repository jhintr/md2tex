\chapter{卷之一\hspace{1ex}詩四言}

\section{停雲\hspace{1ex}{\footnotesize 并序}}

\begin{quoting}停雲,思親友也。罇湛新醪,園列初榮,願言不從,歎息彌襟。\end{quoting}

\textbf{靄靄停雲,濛濛時雨。八表\footnote{八表:即八方,蓋晉人常用語。}同昏,平路伊阻。靜寄東軒,春醪獨撫。良朋悠邈,搔首延佇。}

\textbf{停雲靄靄,時雨濛濛。八表同昏,平陸成江。有酒有酒,閒飲東窗。願言懷人,舟車靡從。}

\textbf{東園之樹,枝條再榮。競用新好,以怡余情。人亦有言,日月于征。安得促席,說彼平生\footnote{平生:平時。}。}

\textbf{翩翩飛鳥,息我庭柯。斂翮閒止,好聲相和。豈無他人,念子實多。願言不獲,抱恨如何。}

\section{時運\hspace{1ex}{\footnotesize 并序}}

\begin{quoting}時運,遊暮春也。春服既成,景物斯和,偶影獨遊,欣慨交心\footnote{晉書袁宏傳:遇之不能無欣,喪之不能無慨。案如詩所云,則欣遇遐景,慨喪黃唐也。}。\end{quoting}

\textbf{邁邁時運,穆穆良朝。襲我春服,薄言東郊。山滌餘靄,宇曖微霄。有風自南,翼彼新苗。}

\textbf{洋洋平澤,乃漱乃濯。邈邈遐景,載欣載矚。稱心而言,人亦易足。揮茲一觴,陶然自樂。}

\textbf{延目中流,悠想清沂。童冠齊業,閒詠以歸。我愛其靜,寤寐交揮。但恨殊世,邈不可追。}

\textbf{斯晨斯夕,言息其廬。花藥分列,林竹翳如。清琴橫床,濁酒半壺。黃唐莫逮,慨獨在余。}

\section{榮木\hspace{1ex}{\footnotesize 并序}}

\begin{quoting}榮木\footnote{古注:月令「仲夏之月木槿榮」與「日月推遷,已復九夏」應,說文「蕣,木菫,朝生暮落者」與「晨耀其華,夕已喪之」應。},念將老也。日月推遷,已復九夏\footnote{九夏:夏之季月。},總角聞道,白首無成。\end{quoting}

\textbf{采采榮木,結根于茲。晨耀其華,夕已喪之。人生若寄,顦顇有時。靜言孔念,中心悵而。}

\textbf{采采榮木,于茲托根。繁華朝起,慨暮不存。貞脆由人,禍福無門。匪道曷依,匪善奚敦\footnote{爾雅釋詁「敦,勉也」。}。}

\textbf{嗟予小子,禀茲固陋。徂年既流,業不增舊。志彼不舍,安此日富\footnote{安此日富:指安於飲酒,詩小宛「壹醉日富」。}。我之懷矣,怛焉內疚。}

\textbf{先師遺訓,余豈之墜。四十無聞,斯不足畏。脂我名車,策我名驥。千里雖遙,孰敢不至。}

\section{贈長沙公\hspace{1ex}{\footnotesize 并序}}

\begin{quoting}余於長沙公為族祖,同出大司馬,昭穆既遠,以為路人,經過潯陽,臨別贈此。\end{quoting}

\textbf{同源分流,人易世疏。慨然寤歎,念茲厥初。禮服遂悠,歲月眇徂。感彼行路,眷然躊躇。}

\textbf{於穆令族,允構斯堂。諧氣冬暄,映懷圭璋。爰采春花,載警秋霜。我曰欽哉,實宗之光。}

\textbf{伊余云遘,在長忘同。笑言未久,逝焉西東。遙遙三湘,滔滔九江。山川阻遠,行李\footnote{行李:使人也。}時通。}

\textbf{何以寫心,貽此話言。進簣雖微,終焉為山。敬哉離人,臨路悽然。款襟或遼,音問其先。}

\section{酬丁柴桑}

\textbf{有客有客,爰來宦止。秉直司聰,惠于百里\footnote{百里:一縣之所轄也,蜀志龐統傳「龐士元非百里才也」。}。飡勝如歸,聆善若始。}

\textbf{匪惟也諧,屢有良遊。載言載眺,以寫我憂。放歡一遇,既醉還休。實欣心期,方從我遊。}

\section{答龐參軍\hspace{1ex}{\footnotesize 并序}}

\begin{quoting}龐為衛軍參軍,從江陵使上都,過潯陽見贈。\end{quoting}

\textbf{衡門之下,有琴有書。載彈載詠,爰得我娛。豈無他好,樂是幽居。朝為灌園,夕偃蓬廬。}

\textbf{人之所寶,尚或未珍。不有同愛,云胡以親。我求良友,實覯懷人。歡心孔洽,棟宇惟鄰。}

\textbf{伊余懷人,欣德孜孜。我有旨酒,與汝樂之。乃陳好言,乃著新詩。一日不見,如何不思。}

\textbf{嘉遊未斁,誓將離分。送爾于路,銜觴無欣。依依舊楚,邈邈西雲。之子之遠,良話曷聞。}

\textbf{昔我云別,倉庚載鳴。今也遇之,霰雪飄零。大藩\footnote{大藩:謝晦也,時為衛軍將軍。}有命,作使上京。豈忘宴安,王事靡寧。}

\textbf{慘慘寒日,肅肅其風。翩彼方舟,容裔\footnote{容裔:容與,船行貌。}江中。勗哉征人,在始思終。敬茲良辰,以保爾躬。}

\section{勸農}

\textbf{悠悠上古,厥初生民。傲然自足,抱朴含真。智巧既萌,資待靡因。誰其贍之,實賴哲人。}

\textbf{哲人伊何,時惟后稷。贍之伊何,實曰播殖。舜既躬耕,禹亦稼穡。遠若周典,八政始食。}

\textbf{熙熙令音,猗猗原陸。卉木繁榮,和風清穆。紛紛士女,趨時競逐。桑婦宵興,農夫野宿。}

\textbf{氣節易過,和澤難久。冀缺攜儷,沮溺結耦。相彼賢達,猶勤壟畝。矧伊眾庶,曳裾拱手。}

\textbf{民生在勤,勤則不匱。宴安自逸,歲暮奚冀。擔石不儲,飢寒交至。顧余儔列,能不懷愧。}

\textbf{孔耽道德,樊須\footnote{樊須:樊遲,見論語子路。}是鄙。董樂琴書,田園弗履\footnote{漢書董仲舒傳:下帷讀書,三年不窺園。}。若能超然,投迹高軌。敢不斂衽,敬讚德美。}

\section{命子}

\textbf{悠悠我祖,爰自陶唐。邈為虞賓\footnote{虞賓:堯子丹朱也。},歷世重光。御龍勤夏,豕韋翼商。穆穆司徒,厥族以昌。}

\textbf{紛紛戰國,漠漠衰周。鳳隱於林,幽人在丘。逸虬遶雲,奔鯨駭流。天集有漢,眷予愍侯\footnote{史記高帝功臣表:開封愍侯陶舍,以右司馬從漢破代,封侯。}。}

\textbf{於赫愍侯,運當攀龍。撫劍風邁,顯茲武功。書誓山河,啟土開封。亹亹丞相\footnote{陶青於景帝時為相。},允迪前蹤。}

\textbf{渾渾長源,鬱鬱洪柯。群川載導,眾條載羅。時有語默,運因隆窊。在我中晉\footnote{漢季稱東漢為中漢,此中晉所本。},業融長沙\footnote{陶侃封長沙公,追贈大司馬,謚曰桓。}。}

\textbf{桓桓長沙,伊勳伊德。天子疇\footnote{漢書宣帝紀「疇其爵邑」注:律,非始封,十減二,疇者等也,言不復減也。}我,專征南國。功遂辭歸,臨寵不忒。孰謂斯心,而近可得。}

\textbf{肅矣我祖,慎終如始。直方二臺\footnote{漢官儀,刺史治所為外臺。},惠和千里\footnote{千里:一州之所轄也。}。於穆仁考,淡焉虛止。寄迹風雲,冥茲慍喜。}

\textbf{嗟余寡陋,瞻望弗及。顧慚華鬢,負影隻立。三千之罪,無後為急。我誠念哉,呱聞爾泣。}

\textbf{卜云嘉日,占亦良時。名汝曰儼,字汝求思\footnote{曲禮:勿不敬,儼若思。}。溫恭朝夕,念茲在茲。尚想孔伋,庶其企而。}

\textbf{厲夜生子,遽而求火\footnote{見莊子天地。}。凡百有心,奚特於我。既見其生,實欲其可。人亦有言,斯情無假。}

\textbf{日居月諸,漸免於孩。福不虛至,禍亦易來。夙興夜寐,願爾斯才。爾之不才,亦已焉哉。}

\section{歸鳥}

\textbf{翼翼歸鳥,晨去于林。遠之八表,近憩雲岑。和風不洽,翻翮求心。顧儔相鳴,景庇清陰。}

\textbf{翼翼歸鳥,載翔載飛。雖不懷遊,見林情依。遇雲頡頏,相鳴而歸。遐路誠悠,性愛無遺。}

\textbf{翼翼歸鳥,馴林\footnote{馴林:當作循林。}徘徊。豈思天路,欣反舊棲。雖無昔侶,眾聲每諧。日夕氣清,悠然其懷。}

\textbf{翼翼歸鳥,戢羽寒條。遊不曠林,宿則森標。晨風清興,好音時交。矰繳奚施,已卷安勞\footnote{言已卷藏在林,不勞弋者施矰繳 \texttt{zèng zhuó}。}。}

