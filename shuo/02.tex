\chapter{言語第二}

\subsection*{1}

\textbf{邊文禮見袁奉高,}{\footnotesize 閎也。}\textbf{失次序,}{\footnotesize \textbf{文士傳}曰邊讓,字文禮,陳留人,才儁辯逸,大將軍何進聞其名,召署令史,以禮見之,讓占對閑雅,聲氣如流,坐客皆慕之,讓出就曹,時孔融、王朗等並前為掾,共書刺從讓,讓平衡與交接,後為九江太守,為魏武帝所殺。}\textbf{奉高曰:「昔堯聘許由,面無怍色,}{\footnotesize \textbf{皇甫謐}曰由,字武仲,陽城槐里人也,堯舜皆師而學事焉,後隱於沛澤之中,堯乃致天下而讓焉,由為人據義履方,邪席不坐,邪饍不食,聞堯讓而去,其友巢父聞由為堯所讓,以為污己,乃臨池洗耳,池主怒曰「何以污我水」,由於是遁耕於中嶽潁水之陽、箕山之下,終身無經天下色,死葬箕山之巔,在陽城之南十里,堯因就其墓號曰箕山公神,以配食五嶽,世世奉祀,至今不絕也。}\textbf{先生何為顛倒衣裳?」文禮答曰:「明府初臨,堯德未彰,是以賤民顛倒衣裳耳。」}{\footnotesize \textbf{按}袁閎卒於太尉掾,未嘗為汝南,斯說謬矣。}

\subsection*{2}

\textbf{徐孺子}{\footnotesize 穉也。}\textbf{年九歲,嘗月下戲,人語之曰:「若令月中無物,當極明邪?」}{\footnotesize \textbf{五經通議}曰月中有兔、蟾蜍者何?月,陰也,蟾蜍,亦陰也,而與兔並明,陰繫於陽也。}\textbf{徐曰:「不然,譬如人眼中有瞳子,無此必不明。」}

\subsection*{3}

\textbf{孔文舉}{\footnotesize 融也。}\textbf{年十歲,隨父到洛,時李元禮有盛名,為司隸校尉,詣門者皆儁才清稱及中表親戚乃通,文舉至門,謂吏曰:「我是李府君親。」既通,前坐,元禮問曰:「君與僕有何親?」對曰:「昔先君仲尼與君先人伯陽有師資之尊,是僕與君奕世為通好也。」元禮及賓客莫不奇之,太中大夫陳韙後至,人以其語語之,韙曰:「小時了了,大未必佳。」文舉曰:「想君小時,必當了了。」韙大踧踖。}{\footnotesize \textbf{續漢書}曰孔融,字文舉,魯國人,孔子二十四世孫也,高祖父尚,鉅鹿太守,父宙,泰山都尉。\textbf{融別傳}曰融四歲,與兄食梨,輒引小者,人問其故,答曰「小兒法當取小者」,年十歲,隨父詣京師,河南尹李膺有重名,融欲觀其為人,遂造之,膺問「高明父祖,嘗與僕周旋乎」,融曰「然,先君孔子與君先人李老君同德比義,而相師友,則融與君累世通家也」,眾坐莫不歎息,僉曰「異童子也」,太中大夫陳韙後至,同坐以告,韙曰「人小時了了者,長大未必能奇」,融應聲曰「即如所言,君之幼時豈實慧乎」,膺大笑,顧謂融曰「長大必為偉器」。}

\subsection*{4}

\textbf{孔文舉有二子,大者六歲,小者五歲,晝日父眠,小者牀頭盜酒飲之,大兒謂曰:「何以不拜?」答曰:「偷,那得行禮?」}

\subsection*{5}

\textbf{孔融被收,中外惶怖,時融兒大者九歲,小者八歲,二兒故琢釘戲,了無遽容,融謂使者曰:「冀罪止於身,二兒可得全不?」兒徐進曰:「大人豈見覆巢之下,復有完卵乎?」尋亦收至。}{\footnotesize \textbf{魏氏春秋}曰融對孫權使有訕謗之言,坐棄市,二子方八歲九歲,融見收,弈棊端坐不起,左右曰「而父見執」,二子曰「安有巢毀而卵不破者哉」,遂俱見殺。\textbf{世語}曰魏太祖以歲儉禁酒,融謂酒以成禮,不宜禁,由是惑眾,太祖收寘法焉,二子齠齔見收,顧謂二子曰「何以不辟」,二子曰「父尚如此,復何所辟」。\textbf{裴松之}以為世語云融兒不辟,知必俱死,猶差可安,孫盛之言,誠所未譬,八歲小兒能懸了禍患,聰明特達,卓然既遠,則其憂樂之情固亦有過成人矣,安有見父被執而無變容,弈棊不起,若在暇豫者乎?昔申生就命,言不忘父,不以己之將死而廢念父之情也,父安尚猶若茲,而況顛沛哉,盛以此為美談,無乃賊夫人之子與?蓋由好奇情多,而不知言之傷理也。}

\subsection*{6}

\textbf{潁川太守髠陳仲弓,}{\footnotesize \textbf{按}寔之在鄉里,州郡有疑獄不能決者,皆將詣寔,或到而情首,或中途改辭,或託狂悸,皆曰「寧為刑戮所苦,不為陳君所非」,豈有盛德感人若斯之甚而不自衛,反招刑辟,殆不然乎?此所謂東野之言耳。}\textbf{客有問元方:「府君何如?」元方曰:「高明之君也。」「足下家君何如?」曰:「忠臣孝子也。」客曰:「易稱『二人同心,其利斷金,同心之言,其臭如蘭』,}{\footnotesize \textbf{王廙注繫辭}曰金,至堅矣,同心者,其利無不入,蘭,芳物也,無不樂者,言其同心者,物無不樂也。}\textbf{何有高明之君而刑忠臣孝子者乎?」元方曰:「足下言何其謬也,故不相答。」客曰:「足下但因傴為恭而不能答。」元方曰:「昔高宗放孝子孝己,}{\footnotesize \textbf{帝王世紀}曰殷高宗武丁有賢子孝己,其母蚤死,高宗惑後妻之言,放之而死,天下哀之。}\textbf{尹吉甫放孝子伯奇,}{\footnotesize \textbf{琴操}曰尹吉甫,周卿也,有子伯奇,母死更娶,後妻生子曰伯邽,乃譖伯奇於吉甫,於是放伯奇於野,宣王出遊,吉甫從,伯奇乃作歌,以言感之,宣王聞之曰「此孝子之辭也」,吉甫乃求伯奇於野,而射殺後妻。}\textbf{董仲舒放孝子符起,}{\footnotesize 未詳。}\textbf{唯此三君,高明之君,唯此三子,忠臣孝子。」客慚而退。}

\subsection*{7}

\textbf{荀慈明與汝南袁閬相見,}{\footnotesize 荀爽,一名諝。\textbf{漢南紀}曰諝文章典籍無不涉,時人諺曰「荀氏八龍,慈明無雙」,潛處篤志,徵聘無所就。\textbf{張璠漢紀}曰董卓秉政,復徵爽,爽欲遁去,吏持之急,起布衣,九十五日而至三公。}\textbf{問潁川人士,慈明先及諸兄,閬笑曰:「士但可因親舊而已乎?」慈明曰:「足下相難,依據者何經?」閬曰:「方問國士,而及諸兄,是以尤之耳。」慈明曰:「昔者祁奚內舉不失其子,外舉不失其讐,以為至公,}{\footnotesize \textbf{春秋傳}曰祁奚為中軍尉,請老,晉侯問嗣焉,稱解狐,其讐也,將立之而卒,又問焉,對曰「午也可」,其子也,君子謂祁奚可謂能舉善矣,稱其讐,不為諂,立其子,不為比。}\textbf{公旦文王之詩,不論堯舜之德而頌文武者,親親之義也,春秋之義,內其國而外諸夏,且不愛其親而愛他人者,不為悖德乎?」}

\subsection*{8}

\textbf{禰衡被魏武謫為鼓吏,正月半試鼓,衡揚枹為漁陽摻檛,淵淵有金石聲,四坐為之改容,}{\footnotesize \textbf{典略}曰衡,字正平,平原般人也。\textbf{文士傳}曰衡,不知先所出,逸才飄舉,少與孔融作爾汝之交,時衡未滿二十,融已五十,敬衡才秀,共結殷勤,不能相違,以建安初北游,或勸其詣京師貴游者,衡懷一刺,遂至漫滅,竟無所詣,融數與武帝牋,稱其才,帝傾心欲見,衡稱疾不肯往,而數有言論,帝甚忿之,以其才名不殺,圖欲辱之,乃令錄為鼓吏,後至八月朝會,大閱試鼓節,作三重閣,列坐賓客,以帛絹製衣,作一岑牟、一單絞及小褌,鼓吏度者,皆當脫其故衣,著此新衣,次傳衡,衡擊鼓為漁陽摻檛,蹋地來前,躡馺腳足,容態不常,鼓聲甚悲,音節殊妙,坐客莫不忼慨,知必衡也,既度,不肯易衣,吏呵之曰「鼓吏何獨不易服」,衡便止,當武帝前,先脫褌,次脫餘衣,裸身而立,徐徐乃著岑牟,次著單絞,後乃著褌,畢,復擊鼓摻槌而去,顏色無怍,武帝笑謂四坐曰「本欲辱衡,衡反辱孤」,至今有漁陽摻檛,自衡造也,為黃祖所殺。}\textbf{孔融曰:「禰衡罪同胥靡,不能發明王之夢。」}{\footnotesize \textbf{皇甫謐帝王世紀}曰武丁夢天賜己賢人,使百工寫其象,求諸天下,見築者胥靡,衣褐於傅巖之野,是謂傅說。\textbf{張晏}曰胥靡,刑名,胥,相也,靡,從也,謂相從坐輕刑也。}\textbf{魏武慚而赦之。}

\subsection*{9}

\textbf{南郡龐士元聞司馬德操在潁川,故二千里候之,至,遇德操采桑,士元從車中謂曰:「吾聞丈夫處世,當帶金佩紫,焉有屈洪流之量,而執絲婦之事?」}{\footnotesize \textbf{蜀志}曰龐統,字士元,襄陽人,少時樸鈍,未有識者,潁川司馬徽有知人之鑒,士元弱冠往見徽,徽采桑樹上,坐士元樹下,共語,自晝至夜,徽異之曰「生當為南州士人之冠冕」,由是漸顯。\textbf{襄陽記}曰士元,德公之從子也,年少未有識者,唯德公重之,年十八,使往見德操,與語,歎曰「德公誠知人,實盛德也」,後劉備訪世事於德操,德操曰「俗士豈識時務,此閒自有伏龍、鳳雛」,謂諸葛孔明與士元也。\textbf{華陽國志}曰劉備引士元為軍師中郎將,從攻洛,為流矢所中,卒,時年三十八。}\textbf{德操曰:}{\footnotesize \textbf{司馬徽別傳}曰徽,字德操,潁川陽翟人,有人倫鑒識,居荊州,知劉表性暗,必害善人,乃括囊不談議時人,有以人物問徽者,初不辨其高下,毎輒言佳,其婦諫曰「人質所疑,君宜辨論,而一皆言佳,豈人所以咨君之意乎」,徽曰「如君所言,亦復佳」,其婉約遜遁如此,嘗有妄認徽豬者,便推與之,後得其豬,叩頭來還,徽又厚辭謝之,劉表子琮往候徽,遣問在不,會徽自鋤園,琮左右問「司馬君在邪」,徽曰「我是也」,琮左右見其醜陋,罵曰「死傭!將軍諸郎欲求見司馬君,汝何等田奴,而自稱是邪」,徽歸,刈頭著幘出見,琮左右見徽故是向老翁,恐,向琮道之,琮起,叩頭辭謝,徽乃謂曰「卿真不可,然吾甚羞之,此自鋤園,唯卿知之耳」,有人臨蠶求簇箔者,徽自棄其蠶而與之,或曰「凡人損己以贍人者,謂彼急我緩也,今彼此正等,何為與人」,徽曰「人未嘗求己,求之不與將慚,何有以財物令人慚者」,人謂劉表曰「司馬德操,奇士也,但未遇耳」,表後見之曰「世閒人為妄語,此直小書生耳」,其智而能愚皆此類,荊州破,為曹操所得,操欲大用,會其病死。}\textbf{「子且下車!子適知邪徑之速,不慮失道之迷,昔伯成耦耕,不慕諸侯之榮,}{\footnotesize \textbf{莊子}曰堯治天下,伯成子高立為諸侯,禹為天子,伯成辭諸侯而耕於野,禹往見之,趨就下風而問焉,子高曰「昔堯治天下,不賞而民勸,不罰而民畏,今子賞罰而民且不仁,德自此衰,刑自此立,夫子盍行邪,毋落吾事」。}\textbf{原憲桑樞,不易有官之宅,}{\footnotesize \textbf{家語}曰原憲,字子思,宋人,孔子弟子,居魯,環堵之室,茨以生草,蓬戶不完,桑樞而瓮牖,上漏下濕,坐而弦歌,子貢軒車不容巷,往見之,曰「先生何病也」,憲曰「憲聞無財謂之貧,學而不能行謂之病,今憲貧也,非病也,夫希世而行,比周而友,學以為人,教以為己,仁義之慝,輿馬之飾,憲不忍為也」。}\textbf{何有坐則華屋,行則肥馬,侍女數十,然後為奇?此乃許父}{\footnotesize 許由、巢父。}\textbf{所以忼慨、夷齊所以長歎,}{\footnotesize \textbf{孟子}曰伯夷、叔齊目不視惡色,耳不聽惡聲,與鄉人居,若在塗炭,蓋聖人之清也。}\textbf{雖有竊秦之爵、千駟之富,}{\footnotesize \textbf{古史考}曰呂不韋為秦子楚行千金貨於華陽夫人,請立子楚為嗣,及子楚立,封不韋洛陽十萬戶,號文信侯。以詐獲爵,故曰竊也。\textbf{論語}曰齊景公有馬千駟,民無德而稱焉。\textbf{孔安國}曰千駟,四千匹。}\textbf{不足貴也。」士元曰:「僕生出邊垂,寡見大義,若不一叩洪鍾、伐雷鼓,則不識其音響也。」}

\subsection*{10}

\textbf{劉公幹以失敬罹罪,}{\footnotesize \textbf{典略}曰劉楨,字公幹,東平寧陽人,建安十六年,世子為五官中郎將,妙選文學,使楨隨侍世子,酒酣坐歡,乃使夫人甄氏出拜,坐上客多伏,而楨獨平視,他日公聞,乃收楨,減死輸作部。\textbf{文士傳}曰楨性辯捷,所問應聲而答,坐平視甄夫人,配輸作部,使磨石,武帝至尚方觀作者,見楨匡坐正色磨石,武帝問曰「石何如」,楨因得喻己自理,跪而對曰「石出荊山懸巖之巔,外有五色之章,內含卞氏之珍,磨之不加瑩,雕之不增文,稟氣堅貞,受之自然,顧其理枉屈紆繞而不得申」,帝顧左右大笑,即日赦之。}\textbf{文帝問曰:「卿何以不謹於文憲?」楨答曰:「臣誠庸短,亦由陛下綱目不疏。」}{\footnotesize \textbf{魏志}曰帝諱丕,字子桓,受漢禪。\textbf{按}諸書咸云楨被刑魏武之世,建安二十年病亡,後七年文帝乃即位,而謂楨得罪黃初之時,謬矣。}

\subsection*{11}

\textbf{鍾毓、鍾會少有令譽,}{\footnotesize \textbf{魏書}曰毓,字穉叔,潁川長社人,相國繇長子也,年十四,為散騎侍郎,機捷談笑有父風,仕至車騎將軍。}\textbf{年十三,魏文帝聞之,語其父鍾繇}{\footnotesize \textbf{魏志}曰繇,字元常,家貧好學,為周易、老子訓,歷大理、相國,遷太傅。}\textbf{曰:「可令二子來。」於是敕見,毓面有汗,帝曰:「卿面何以汗?」毓對曰:「戰戰惶惶,汗出如漿。」復問會:「卿何以不汗?」對曰:「戰戰慄慄,汗不敢出。」}

\subsection*{12}

\textbf{鍾毓兄弟小時,值父晝寢,因共偷服藥酒,其父時覺,且託寐以觀之,毓拜而後飲,會飲而不拜,}{\footnotesize \textbf{魏志}曰會,字士季,繇少子也,敏惠夙成,中護軍蔣濟著論,謂觀其眸子,足以知人,會年五歲,繇遣見濟,濟甚異之,曰「非常人也」,及壯,有才數,精練名理,累遷黃門侍郎,諸葛誕反,文王征之,會謀居多,時人謂之子房,拜鎮西將軍,伐蜀,蜀平,進位司徒,自謂功名蓋世,不可復為人下,謂所親曰「我淮南已來,畫無遺策,四海共知,將此欲安歸乎」,遂謀反見誅,時年四十。}\textbf{既而問毓何以拜,毓曰:「酒以成禮,不敢不拜。」又問會何以不拜,會曰:「偷本非禮,所以不拜。」}

\subsection*{13}

\textbf{魏明帝為外祖母築館於甄氏,}{\footnotesize \textbf{魏末傳}曰帝諱叡,字元仲,文帝太子,以其母廢,未立為嗣,文帝與俱獵,見子母鹿,文帝射其母,應弦而倒,復令帝射其子,帝置弓泣曰「陛下已殺其母,臣不忍復殺其子」,文帝曰「好語動人心」,遂定為嗣,是為明帝。\textbf{魏書}曰文昭甄皇后,明帝母也,父逸,上蔡令,烈宗即位,追封上蔡君,嫡孫象襲爵,象薨,子暢嗣,起大第,車駕親自臨之。}\textbf{既成,自行視,謂左右曰:「館當以何為名?」侍中繆襲曰:}{\footnotesize \textbf{文章敘錄}曰襲,字熙伯,東海蘭陵人,有才學,累遷侍中、光祿勳。}\textbf{「陛下聖思齊於哲王,罔極過於曾閔,此館之興,情鍾舅氏,宜以渭陽為名。」}{\footnotesize \textbf{秦詩}曰渭陽,康公念母也,康公之母,晉獻公之女,文公遭驪姬之難,未反而秦姬卒,穆公納文公,康公時為太子,贈送文公於渭之陽,念母之不見也,我見舅氏,如母存焉。\textbf{按}魏書「帝於後園為象母起觀,名其里曰渭陽」,然則象母即帝之舅母,非外祖母也,且渭陽為館名,亦乖舊史也。}

\subsection*{14}

\textbf{何平叔云:「服五石散,非唯治病,亦覺神明開朗。」}{\footnotesize \textbf{魏略}曰何晏,字平叔,南陽宛人,漢大將軍進孫也,或云何苗孫也,尚主,又好色,故黃初時無所事任,正始中,曹爽用為中書,主選舉,宿舊者多得濟拔,為司馬宣王所誅。\textbf{秦丞相寒食散論}曰寒食散之方雖出漢代,而用之者寡,靡有傳焉,魏尚書何晏首獲神效,由是大行於世,服者相尋也。}

\subsection*{15}

\textbf{嵇中散語趙景真:}{\footnotesize \textbf{嵇紹趙至敘}曰至,字景真,代郡人,漢末,其祖流宕客緱氏,令新之官,至年十二,與母共道傍看,母曰「汝先世非微賤家也,汝後能如此不」,至曰「可爾耳」,歸便求師誦書,蚤聞父耕叱牛聲,釋書而泣,師問之,答曰「自傷不能致榮華,而使老父不免勤苦」,年十四,入太學觀,時先君在學寫石經古文,事訖去,遂隨車問先君姓名,先君曰「年少何以問我」,至曰「觀君風器非常,故問耳」,先君具告之,至年十五,陽病,數數狂走五里三里,為家追得,又炙身體十數處,年十六,遂亡命,徑至洛陽,求索先君不得,至鄴,沛國史仲和是魏領軍史渙孫也,至便依之,遂名翼,字陽和,先君到鄴,至具道太學中事,便逐先君歸山陽經年,至長七尺三寸,潔白黑髮,赤脣明目,鬢鬚不多,閒詳安諦,體若不勝衣,先君嘗謂之曰「卿頭小而銳,瞳子白黑分明,視瞻停諦,有白起風」,至論議清辯,有從橫才,然亦不以自長也,孟元基辟為遼東從事,在郡斷九獄,見稱清當,自痛棄親遠遊,母亡不見,吐血發病,服未竟而亡。}\textbf{「卿瞳子白黑分明,有白起之風,}{\footnotesize \textbf{嚴尤三將敘}曰白起,平原君勸趙孝成王受馮亭,王曰「受之,秦兵必至,武安君必將,誰能當之者乎」,對曰「澠池之會,臣察武安君小頭而面銳,瞳子白黑分明,視瞻不轉,小頭而面銳者,敢斷決也,瞳子白黑分明者,見事明也,視瞻不轉者,執志強也,可與持久,難與爭鋒,廉頗為人,勇鷙而愛士,知難而忍恥,與之野戰則不如,持守足以當之」,王從其計。}\textbf{恨量小狹。」趙云:「尺表能審璣衡之度,}{\footnotesize \textbf{周髀}曰夏至,北方二萬六千里,冬至,南方十三萬五千里,日中樹表則無影矣,周髀長八尺,夏至日,晷尺六寸,髀,股也,晷,句也,正南千里,句尺五寸,正北千里,句尺七寸。周髀之書也。}\textbf{寸管能測往復之氣,}{\footnotesize \textbf{呂氏春秋}曰黃帝使伶倫自大夏之西、崑崙之陰取竹之嶰谷生、其竅厚薄均者,斷兩節間而吹之,以為黃鍾之管,制十二筩,以聽鳳凰之鳴,雄鳴六,雌亦六,以為律呂。\textbf{續漢書律曆志}曰十二律之變,至於六十,以律候氣,候氣之法,為室三重,戶閉,塗釁必周,密布緹幔,以木為案,加律其上,以葭莩灰抑其內,為氣所動者,其灰散也,以此候之。}\textbf{何必在大,但問識如何耳。」}

\subsection*{16}

\textbf{司馬景王東征,}{\footnotesize \textbf{魏書}曰司馬師,字子元,相國宣文侯長子也,以道德清粹,重於朝廷,為大將軍、錄尚書事,毋丘儉反,師自征之,薨,諡景王。}\textbf{取上黨李喜,以為從事中郎,因問喜曰:「昔先公辟君不就,今孤召君,何以來?」喜對曰:「先公以禮見待,故得以禮進退,明公以法見繩,喜畏法而至耳。」}{\footnotesize \textbf{晉諸公贊}曰喜,字季和,上黨銅鞮人也,少有高行,研精藝學,宣帝為相國,辟喜,喜固辭疾,景帝輔政,為從事中郎,累遷光祿大夫、特進,贈太保。}

\subsection*{17}

\textbf{鄧艾口吃,語稱艾艾,}{\footnotesize \textbf{魏志}曰艾,字士載,棘陽人,少為農人養犢,年十二,隨母至潁川,讀故太丘長碑文曰「言為世範,行為士則」,遂名範,字士則,後宗族有同者,故改焉,毎見高山大澤,輒規度指畫軍營處所,時人多笑焉,後見司馬宣王,辟為掾,累遷征西將軍,伐蜀,蜀平,進位太尉,為衛瓘所害。}\textbf{晉文王戲之曰:「卿云艾艾,定是幾艾?」對曰:「鳳兮鳳兮,故是一鳳。」}{\footnotesize \textbf{朱鳳晉紀}曰文王諱昭,字子上,宣帝次子也。\textbf{列仙傳}曰陸通者,楚狂接輿也,好養性,游諸名山,嘗遇孔子而歌曰「鳳兮鳳兮,何德之衰,往者不可諫,來者猶可追」,後入蜀,在峨嵋山中也。}

\subsection*{18}

\textbf{嵇中散既被誅,向子期舉郡計入洛,文王引進,問曰:「聞君有箕山之志,何以在此?」對曰:「巢、許狷介之士,不足多慕。」王大咨嗟。}{\footnotesize \textbf{向秀別傳}曰秀,字子期,河內人,少為同郡山濤所知,又與譙國嵇康、東平呂安友善,並有拔俗之韻,其進止無固必,而造事營生,業亦不異,常與嵇康偶鍛於洛邑,與呂安灌園於山陽,不慮家人有無,外物不足怫其心,弱冠,著儒道論,棄而不錄,好事者或存之,或云是其族人所作,困於不行,乃告秀,欲假其名,秀笑曰「可復爾耳」,後康被誅,秀遂失圖,乃應歲舉,到京師,詣大將軍司馬文王,文王問曰「聞君有箕山之志,何能自屈」,秀曰「常謂彼人不達堯意,本非所慕也」,一坐皆說,隨次轉至黃門侍郎、散騎常侍。}

\subsection*{19}

\textbf{晉武帝始登阼,探策得一,}{\footnotesize \textbf{晉世譜}曰世祖諱炎,字安世,咸熙二年受魏禪。}\textbf{王者世數,繫此多少,帝既不說,群臣失色,莫能有言者,侍中裴楷進曰:「臣聞天得一以清,地得一以寧,侯王得一以為天下貞。」帝說,群臣歎服。}{\footnotesize \textbf{王弼老子注}云一者,數之始、物之極也,各是一物,所以為主也,各以其一,致此清、寧、貞。}

\subsection*{20}

\textbf{滿奮畏風,在晉武帝坐,北窻作琉璃扇屏風,實密似疎,奮有難色,帝笑之,}{\footnotesize \textbf{荀綽冀州記}曰奮,字武秋,高平人,魏太尉寵之孫也,性清平有識,自吏部郎出為冀州刺史。\textbf{晉諸公贊}曰奮體量清雅,有曾祖寵之風,遷尚書令,為荀顗所害。}\textbf{奮答曰:「臣猶吳牛,見月而喘。」}{\footnotesize 今之水牛,惟生江淮間,故謂之吳牛也,南土多暑,而此牛畏熱,見月疑是日,所以見月則喘。}

\subsection*{21}

\textbf{諸葛靚在吳,於朝堂大會,}{\footnotesize \textbf{晉諸公贊}曰靚,字仲思,琅邪人,司空誕少子也,雅正有才望,誕以壽陽叛,遣靚入質於吳,以靚為右將軍、大司馬。}\textbf{孫皓問:「卿字仲思,為何所思?」對曰:「在家思孝,事君思忠,朋友思信,如斯而已。」}

\subsection*{22}

\textbf{蔡洪}{\footnotesize \textbf{洪集錄}曰洪,字叔開,吳郡人,有才辯,初仕吳朝,太康中,本州從事,舉秀才。\textbf{王隱晉書}曰洪仕至松滋令。}\textbf{赴洛,洛中人問曰:「幕府初開,群公辟命,求英奇於仄陋,採賢儁於巖穴,君吳楚之士,亡國之餘,有何異才而應斯舉?」蔡答曰:「夜光之珠,不必出於孟津之河,}{\footnotesize \textbf{舊說}云隋侯出行,有蛇斬而中斷者,侯連而續之,蛇遂得生而去,後銜明月珠以報其德,光明照夜同晝,因曰隋珠,左思蜀都賦所謂「隋侯鄙其夜光」也。}\textbf{盈握之璧,不必採於崑崙之山,}{\footnotesize \textbf{韓氏}曰和氏之璧,蓋出於井里之中。}\textbf{大禹生於東夷,文王生於西羌,}{\footnotesize \textbf{按}孟子曰「舜生於諸馮,東夷人也,文王生於岐周,西戎人也」,則東夷是舜非禹也。}\textbf{聖賢所出,何必常處?昔武王伐紂,遷頑民於洛邑,}{\footnotesize \textbf{尚書}曰成周既成,遷殷頑民,作多士。\textbf{孔安國}注曰殷大夫心不則德義之經,故徙於王都,邇教誨也。}\textbf{得無諸君是其苗裔乎?」}{\footnotesize \textbf{按}華令思舉秀才入洛,與王武子相酬對,皆與此言不異,無容二人同有此辭,疑世說穿鑿也。}

\subsection*{23}

\textbf{諸名士共至洛水戲,}{\footnotesize \textbf{竹林七賢論}曰王濟諸人嘗至洛水解禊事,明日,或問濟曰「昨遊,有何語議」,濟云云。}\textbf{還,樂令}{\footnotesize 廣也。}\textbf{問王夷甫曰:「今日戲樂乎?」}{\footnotesize \textbf{虞預晉書}曰王衍,字夷甫,琅邪臨沂人,司徒戎從弟,父乂,平北將軍,夷甫蚤知名,以清虛通理稱,仕至太尉,為石勒所害。}\textbf{王曰:「裴僕射善談名理,混混有雅致,}{\footnotesize \textbf{晉惠帝起居注}曰裴頠,字逸民,河東聞喜人,司空秀之少子也。\textbf{冀州記}曰頠弘濟有清識,稽古善言名理,履行高整,自少知名,歷侍中、尚書左僕射,為趙王倫所害。}\textbf{張茂先論史漢,靡靡可聽,}{\footnotesize \textbf{晉陽秋}曰華博覽洽聞,無不貫綜,世祖嘗問漢事及建章千門萬戶,華畫地成圖,應對如流,張安世不能過也。}\textbf{我與王安豐}{\footnotesize 戎也。}\textbf{說延陵、子房,亦超超玄著。」}{\footnotesize \textbf{晉諸公贊}曰夷甫好尚談稱,為時人物所宗。}

\subsection*{24}

\textbf{王武子、}{\footnotesize \textbf{晉諸公贊}曰王濟,字武子,太原晉陽人,司徒渾第二子也,有儁才,能清言,起家中書郎,終太僕。}\textbf{孫子荊}{\footnotesize \textbf{文士傳}曰孫楚,字子荊,太原中都人也。\textbf{晉陽秋}曰楚,驃騎將軍資之孫,南陽太守弘之子,鄉人王濟,豪俊公子,為本州大中正,訪問弘為鄉里品狀,濟曰「此人非鄉評所能名,吾自狀之,曰天才英特、亮拔不群」,仕至馮翊太守。}\textbf{各言其土地人物之美,王云:「其地坦而平,其水淡而清,其人廉且貞。」孫云:「其山㠑巍以嵯峨,其水㳌渫而揚波,其人磊砢而英多。」}{\footnotesize \textbf{按}三秦記、語林載蜀人伊籍稱吳土地人物,與此語同。}

\subsection*{25}

\textbf{樂令女適大將軍成都王穎,}{\footnotesize \textbf{虞預晉書}曰樂廣,字彥輔,南陽人,清夷沖曠,加有理識,累遷侍中、河南尹,在朝廷用心虛淡,時人重其貞貴,代王戎為尚書令。\textbf{八王故事}曰司馬穎,字叔度,世祖第十九子,封成都王、大將軍。}\textbf{王兄長沙王執權於洛,}{\footnotesize \textbf{晉百官名}曰司馬乂,字士度,封長沙王。\textbf{八王故事}曰世祖第十七子。}\textbf{遂構兵相圖,長沙王親近小人,遠外君子,凡在朝者,人懷危懼,樂令既處朝望,加有婚親,群小讒於長沙,長沙嘗問樂令,樂令神色自若,徐答曰:「豈以五男易一女?」}{\footnotesize \textbf{晉陽秋}曰成都王之起兵,長沙王猜廣,廣曰「寧以一女而易五男」,乂猶疑之,遂以憂卒。}\textbf{由是釋然,無復疑慮。}

\subsection*{26}

\textbf{陸機詣王武子,}{\footnotesize \textbf{晉陽秋}曰機,字士衡,吳郡人,祖遜,吳丞相,父抗,大司馬,機與弟雲並有儁才,司空張華見而說之,曰「平吳之利,在獲二儁」。\textbf{機別傳}曰博學善屬文,非禮不動,入晉,仕著作郎,至平原內史。}\textbf{武子前置數斛羊酪,指以示陸曰:「卿江東何以敵此?」陸云:「有千里蓴羹,但未下鹽豉耳。」}

\subsection*{27}

\textbf{中朝有小兒,父病,行乞藥,主人問病,曰:「患瘧也。」主人曰:「尊侯明德君子,何以病瘧?」}{\footnotesize 俗傳行瘧鬼小,多不病巨人,故光武皇帝嘗謂景丹曰「嘗聞壯士不病瘧,大將軍反病瘧耶」。}\textbf{答曰:「來病君子,所以為瘧耳。」}

\subsection*{28}

\textbf{崔正熊詣都郡,都郡將姓陳,問正熊:「君去崔杼幾世?」答曰:「民去崔杼,如明府之去陳恆。」}{\footnotesize \textbf{晉百官名}曰崔豹,字正熊,燕國人,惠帝時官至太傅丞。}

\subsection*{29}

\textbf{元帝始過江,}{\footnotesize \textbf{朱鳳晉書}曰帝諱叡,字景文,祖伷,封琅邪王,父恭王瑾嗣,帝襲爵為琅邪王,少而明惠,因亂過江起義,遂即皇帝位。\textbf{諡法}曰始建國都曰元。}\textbf{謂顧驃騎曰:「寄人國土,心常懷慚。」榮跪對曰:「臣聞王者以天下為家,是以耿、亳無定處,}{\footnotesize \textbf{帝王世紀}曰殷祖乙徙耿,為河所毀。今河東皮氏耿鄉是也。盤庚五遷,復南居亳。今景亳是也。}\textbf{九鼎遷洛邑,}{\footnotesize \textbf{春秋傳}曰武王克商,遷九鼎於洛邑。今之偃師是也。}\textbf{願陛下勿以遷都為念。」}

\subsection*{30}

\textbf{庾公造周伯仁,}{\footnotesize \textbf{虞預晉書}曰周顗,字伯仁,汝南安城人,揚州刺史浚長子也。\textbf{晉陽秋}曰顗有風流才氣,少知名,正體嶷然,儕輩不敢媟也,汝南賁泰,淵通清操之士,嘗歎曰「汝潁固多賢士,自頃陵遲,雅道殆衰,今復見周伯仁,伯仁將祛舊風,清我邦族矣」,舉寒素,累遷尚書僕射,為王敦所害。}\textbf{伯仁曰:「君子何欣說而忽肥?」庾曰:「君復何所憂慘而忽瘦?」伯仁曰:「吾無所憂,直是清虛日來,滓穢日去耳。」}

\subsection*{31}

\textbf{過江諸人,每至美日,輒相邀新亭,藉卉飲宴,}{\footnotesize \textbf{丹陽記}曰新亭,吳舊立,先基崩淪,隆安中,丹陽尹司馬恢之徙創今地。}\textbf{周侯}{\footnotesize 顗也。}\textbf{中坐而歎曰:「風景不殊,正自有山河之異。」皆相視流淚,唯王丞相}{\footnotesize 導也。}\textbf{愀然變色曰:「當共勠力王室,克復神州,何至作楚囚相對?」}{\footnotesize \textbf{春秋傳}曰楚伐鄭,諸侯救之,鄭執鄖公鍾儀獻晉,景公觀軍府,見而問之曰「南冠而縶者為誰」,有司對曰「楚囚也」,使脫之,問其族,對曰「伶人也」,「能為樂乎」,曰「先父之職,敢有二事」,與之琴,操南音,范文子曰「楚囚,君子也,樂操土風,不忘舊也,君盍歸之,以合晉楚之成」。}

\subsection*{32}

\textbf{衛洗馬初欲渡江,形神慘顇,語左右云:「見此芒芒,不覺百端交集,苟未免有情,亦復誰能遣此?」}{\footnotesize \textbf{晉諸公贊}曰衛玠,字叔寶,河東安邑人,祖父瓘,太尉,父恆,黃門侍郎。\textbf{玠別傳}曰玠穎識通達,天韻標令,陳郡謝幼輿敬以亞父之禮,論者以為出王眉子、平子、武子之右,世咸謂「諸王三子,不如衛家一兒」,娶樂廣女,裴叔道曰「妻父有冰清之姿,壻有璧潤之望,所謂秦晉之匹也」,為太子洗馬,永嘉四年,南至江夏,與兄別於梁里澗,語曰「在三之義,人之所重,今日忠臣致身之道,可不勉乎」,行至豫章,乃卒。}

\subsection*{33}

\textbf{顧司空未知名,詣王丞相,丞相小極,對之疲睡,顧思所以叩會之,}{\footnotesize \textbf{顧和別傳}曰和,字君孝,吳郡人,祖容,吳荊州刺史,父相,晉臨海太守,和總角知名,族人顧榮雅相器愛,曰「此吾家之騏驥也,必振衰族」,累遷尚書令。}\textbf{因謂同坐曰:「昔每聞元公}{\footnotesize 顧榮。}\textbf{道公協贊中宗,保全江表,}{\footnotesize \textbf{鄧粲晉紀}曰導與元帝有布衣之好,知中國將亂,勸帝渡江,求為安東司馬,政皆決之,號仲父,晉中興之功,導實居其首。}\textbf{體小不安,令人喘息。」丞相因覺,謂顧曰:「此子珪璋特達,機警有鋒。」}

\subsection*{34}

\textbf{會稽賀生,體識清遠,言行以禮,}{\footnotesize 賀循別見。}\textbf{不徒東南之美,}{\footnotesize \textbf{爾雅}曰東南之美者,有會稽之竹箭焉。}\textbf{實為海內之秀。}

\subsection*{35}

\textbf{劉琨雖隔閡寇戎,志存本朝,}{\footnotesize \textbf{王隱晉書}曰琨,字越石,中山魏昌人,祖邁,有經國之才,父璠,光祿大夫,琨少稱儁朗,累遷司徒長史、尚書右丞,迎大駕於長安,以有殊勳,封廣武侯,年三十五,出為并州刺史,為段日磾所害。}\textbf{謂溫嶠曰:「班彪識劉氏之復與,馬援知漢光之可輔,}{\footnotesize \textbf{漢書敘傳}曰彪,字叔皮,扶風人,客於天水,隴西隗囂有窺覦之志,彪作王命論以諷之。\textbf{東觀漢記}曰馬援,字文淵,茂陵人,從公孫述、隗囂游,後見光武,曰「天下反覆,盜名字者不可勝數,今見陛下寥廓大度,同符高祖,乃知帝王自有真也」,帝甚壯之。}\textbf{今晉阼雖衰,天命未改,吾欲立功於河北,使卿延譽於江南,子其行乎?」溫曰:「嶠雖不敏,才非昔人,明公以桓文之姿,建匡立之功,豈敢辭命?」}{\footnotesize \textbf{虞預晉書}曰嶠,字太真,太原祁人,少標俊清徹,英穎顯名,為司空劉琨左司馬,是時二都傾覆,天下大亂,琨聞元皇受命中興,忼慨幽朔,志存本朝,使嶠奉使,嶠喟然對曰「嶠雖乏管張之才,而明公有桓文之志,敢辭不敏,以違高旨」,以左長史奉使勸進,累遷驃騎大將軍。}

\subsection*{36}

\textbf{溫嶠初為劉琨使來過江,于時江左營建始爾,綱紀未舉,溫新至,深有諸慮,既詣王丞相,陳主上幽越,社稷焚滅,山陵夷毀之酷,有黍離之痛,溫忠慨深烈,言與泗俱,丞相亦與之對泣,敘情既畢,便深自陳結,丞相亦厚相酬納,既出,懽然言曰:「江左自有管夷吾,此復何憂?」}{\footnotesize \textbf{史記}曰管仲夷吾者,潁上人,相齊桓公,九合諸侯,一匡天下。\textbf{語林}曰初,溫奉使勸進,晉王大集賓客見之,溫公始入,姿形甚陋,合坐盡驚,既坐,陳說九服分崩,皇室弛絕,晉王君臣莫不歔欷,及言天下不可以無主,聞者莫不踴躍,植髮穿冠,王丞相深相付託,溫公既見丞相,便游樂不住,曰「既見管仲,天下事無復憂」。}

\subsection*{37}

\textbf{王敦兄含為光祿勳,}{\footnotesize \textbf{含別傳}曰含,字處弘,琅邪臨沂人,累遷徐州刺史、光祿勳,與弟敦作逆,伏誅。}\textbf{敦既逆謀,屯據南州,含委職奔姑孰,}{\footnotesize \textbf{鄧粲晉紀}曰初,王導協贊中興,敦有方面之功,敦以劉隗為閒己,舉兵討之,故含南奔武昌,朝廷始警備也。}\textbf{王丞相詣闕謝,}{\footnotesize \textbf{中興書}曰導從兄敦,舉兵討劉隗,導率子弟二十餘人,旦旦到公車,泥首謝罪。}\textbf{司徒、丞相、揚州官僚問訊,倉卒不知何辭,顧司空時為揚州別駕,援翰曰:「王光祿遠避流言,明公蒙塵路次,群下不寧,不審尊體起居何如?」}

\subsection*{38}

\textbf{郗太尉拜司空,語同坐曰:「平生意不在多,值世故紛紜,遂至台鼎,朱博翰音,實愧於懷。」}{\footnotesize \textbf{漢書}曰朱博,字子元,杜陵人,為丞相,臨拜,延登受策,有大聲如鍾鳴,上問揚雄、李尋,對曰「洪範所謂鼓妖者也,人君不聰,空名得進,則有無形之聲」,博後坐事自殺。故序傳曰「博之翰音,鼓妖先作」。\textbf{易中孚}曰上九,翰音登於天,貞凶。\textbf{王弼}注曰翰,高飛也,飛音者,音飛而實不從也。}

\subsection*{39}

\textbf{高坐道人不作漢語,或問此意,簡文曰:「以簡應對之煩。」}{\footnotesize \textbf{高坐別傳}曰和尚胡名尸黎密,西域人,傳云國王子,以國讓弟,遂為沙門,永嘉中始到此土,止於大市中,和尚天姿高朗,風韻遒邁,丞相王公一見奇之,以為吾之徒也,周僕射領選,撫其背而歎曰「若選得此賢,令人無恨」,俄而周侯遇害,和尚對其靈坐,作胡祝數千言,音聲高暢,既而揮涕收涙,其哀樂廢興皆此類,性高簡,不學晉語,諸公與之言,皆因傳譯,然神領意得,頓在言前。\textbf{塔寺記}曰尸黎密,宋曰高坐,在石子岡,常行頭陀,卒於梅岡,即葬焉,晉元帝於冢邊立寺,因名高坐。}

\subsection*{40}

\textbf{周僕射雍容好儀形,詣王公,初下車,隱數人,王公含笑看之,既坐,傲然嘯詠,王公曰:「卿欲希嵇阮邪?」答曰:「何敢近舍明公,遠希嵇阮?」}{\footnotesize \textbf{鄧粲晉紀}曰伯仁儀容弘偉,善於俛仰應答,精神足以蔭映數人,深自持,能致人,而未嘗往焉。}

\subsection*{41}

\textbf{庾公嘗入佛圖,見臥佛,}{\footnotesize \textbf{涅槃經}云如來背痛,於雙樹間北首而臥。故後之圖繪者為此象。}\textbf{曰:「此子疲於津梁。」于時以為名言。}

\subsection*{42}

\textbf{摯瞻曾作四郡太守、大將軍戶曹參軍,復出作內史,}{\footnotesize \textbf{摯氏世本}曰瞻,字景游,京兆長安人,太常虞兄子也,父育,涼州刺史,瞻少善屬文,起家著作郎,中朝亂,依王敦為戶曹參軍,歷安豐、新蔡、西陽內史,見敦以故壞裘賜老病外部都督,瞻諫曰「尊裘雖故,不宜與小吏」,敦曰「何為不可」,瞻時因醉,曰「若上服皆可用賜,貂蟬亦可賜下乎」,敦曰「非喻,所引如此,不堪二千石」,瞻曰「瞻視去西陽,如脫屣耳」,敦反,乃左遷隨郡內史。}\textbf{年始二十九,嘗別王敦,敦謂瞻曰:「卿年未三十,已為萬石,亦太蚤。」瞻曰:「方於將軍,少為太蚤,比之甘羅,已為太老。」}{\footnotesize \textbf{摯氏世本}曰瞻高亮有氣節,故以此答敦,後知敦有異志,建興四年,與第五猗據荊州以距敦,竟為所害。\textbf{史記}曰甘羅,秦相茂之孫也,年十二,而秦相呂不韋欲使張唐相燕,唐不肯行,甘羅說而行之,又請車五乘以使趙,還報秦,秦封甘羅為上卿,賜以甘茂田宅。}

\subsection*{43}

\textbf{梁國楊氏子,九歲,甚聰惠,孔君平}{\footnotesize \textbf{王隱晉書}曰孔坦,字君平,會稽山陰人,善春秋,有文辯,歷太子舍人,累遷廷尉卿。}\textbf{詣其父,父不在,乃呼兒出,為設果,果有楊梅,孔指以示兒曰:「此是君家果。」兒應聲答曰:「未聞孔雀是夫子家禽。」}

\subsection*{44}

\textbf{孔廷尉以裘與從弟沈,}{\footnotesize \textbf{孔氏譜}曰沈,字德度,會稽山陰人,祖父奕,全椒令,父群,鴻臚卿,沈至琅邪王文學。}\textbf{沈辭不受,廷尉曰:「晏平仲之儉,祠其先人,豚肩不掩豆,猶狐裘數十年,}{\footnotesize \textbf{劉向別錄}曰晏平仲,名嬰,東萊夷維人,事齊靈公、莊公,以節儉力行重於齊。\textbf{禮記}曰晏平仲祀其先人,豚肩不掩豆,君子以為儉也。又曰晏子一狐裘三十年,晏子焉知禮。\textbf{注}豚,俎實也,豆,徑尺,言併豚之兩肩不能掩豆,喻少也。}\textbf{卿復何辭此?」於是受而服之。}

\subsection*{45}

\textbf{佛圖澄與諸石遊,}{\footnotesize \textbf{澄別傳}曰道人佛圖澄,不知何許人,出於燉煌,好佛道,出家為沙門,永嘉中至洛陽,值京師有難,潛遁草澤間,石勒雄異好殺害,因勒大將軍郭默略見勒,以麻油塗掌,占見吉凶,數百里外聽浮圖鈴聲,逆知禍福,勒甚敬信之,虎即位,亦師澄,號大和尚,自知終日,開棺無屍,唯袈裟法服在焉。}\textbf{林公曰:「澄以石虎為海鷗鳥。」}{\footnotesize \textbf{趙書}曰虎,字季龍,勒從弟也,征伐毎斬將搴旗,勒死,誅勒諸兒,襲位。\textbf{莊子}曰海上之人好鷗者,毎旦之海上從鷗游,鷗之至者數百而不止,其父曰「吾聞鷗鳥從汝游,取來玩之」,明日之海上,鷗舞而不下。}

\subsection*{46}

\textbf{謝仁祖年八歲,謝豫章}{\footnotesize 鯤。子別見。}\textbf{將送客,爾時語已神悟,自參上流,諸人咸共歎之曰:「年少,一坐之顏回。」仁祖曰:「坐無尼父,焉別顏回?」}{\footnotesize \textbf{晉陽秋}曰謝尚,字仁祖,陳郡人,鯤之子也,齠齔喪兄,哀慟過人,及遭父喪,溫嶠唁之,尚號叫極哀,既而收涕告訴,有異常童,嶠奇之,由是知名,仕至鎮西將軍、豫州刺史。}

\subsection*{47}

\textbf{陶公病篤,都無獻替之言,朝士以為恨,}{\footnotesize \textbf{陶氏敘}曰侃,字士衡,其先鄱陽人,後徙尋陽,侃少有遠概綱維宇宙之志,察孝廉,入洛,司空張華見而謂曰「後來匡主寧民,君其人也」,劉弘鎮沔南,取為長史,謂侃曰「昔吾為羊太傅參佐,見語云『君後當居身處』,今相觀,亦復然矣」,累遷湘、廣、荊三州刺史,加羽葆鼓吹,封長沙郡公、大將軍,贊拜不名,劍履上殿,進太尉,贈大司馬,諡桓公。\textbf{按}王隱晉書載侃臨終表曰「臣少長孤寒,始願有限,過蒙先朝歷世異恩,臣年垂八十,位極人臣,啓手啓足,當復何恨,但以餘寇未誅,山陵未復,所以憤慨兼懷,唯此而已,猶冀犬馬之齒,尚可少延,欲為陛下北吞石虎,西誅李雄,勢遂不振,良圖永息,臨書扼腕,涕泗橫流,伏願遴選代人,使必得良才,足以奉宣王猷,遵成志業,則雖死之日,猶生之年」,有表若此,非無獻替。}\textbf{仁祖聞之曰:「時無豎刁,故不貽陶公話言。」}{\footnotesize \textbf{呂氏春秋}曰管仲病,桓公問曰「子如不諱,誰代子相者,豎刁何如」,管仲曰「自宮以事君,非人情,必不可用」,後果亂齊。}\textbf{時賢以為德音。}

\subsection*{48}

\textbf{竺法深在簡文坐,劉尹問:「道人何以游朱門?」答曰:「君自見其朱門,貧道如游蓬戶。」}{\footnotesize \textbf{高逸沙門傳}曰法師居會稽,皇帝重其風德,遣使迎焉,法師暫出應命,司徒會稽王天性虛澹,與法師結殷勤之歡,師雖升履丹墀,出入朱邸,泯然曠達,不異蓬宇也。}\textbf{或云卞令。}{\footnotesize 別見。}

\subsection*{49}

\textbf{孫盛為庾公記室參軍,}{\footnotesize \textbf{中興書}曰盛,字安國,太原中都人,博學強識,歷著作郎、瀏陽令,庾亮為荊州,以為征西主簿,累遷祕書監。}\textbf{從獵,將其二兒俱行,庾公不知,忽於獵場見齊莊,時年七八歲,庾謂曰:「君亦復來邪?」應聲答曰:「所謂無小無大,從公于邁。」}

\subsection*{50}

\textbf{孫齊由、齊莊二人少時詣庾公,公問齊由何字,答曰:「字齊由。」公曰:「欲何齊邪?」曰:「齊許由。」}{\footnotesize \textbf{晉百官名}曰孫潛,字齊由,太原人。\textbf{中興書}曰潛,盛長子也,豫章太守,殷仲堪下討王國寶,潛時在郡,逼為咨議參軍,固辭不就,遂以憂卒。}\textbf{齊莊何字,答曰:「字齊莊。」公曰:「欲何齊?」曰:「齊莊周。」公曰:「何不慕仲尼而慕莊周?」對曰:「聖人生知,故難企慕。」庾公大喜小兒對。}{\footnotesize \textbf{孫放別傳}曰放,字齊莊,監君次子也,年八歲,太尉庾公召見之,放清秀,欲觀試,乃授紙筆令書,放便自疏名字,公題後問之曰「為欲慕莊周邪」,放書答曰「意欲慕之」,公曰「何故不慕仲尼而慕莊周」,放曰「仲尼生而知之,非希企所及,至於莊周,是其次者,故慕耳」,公謂賓客曰「王輔嗣應答,恐不能勝之」,卒長沙王相。}

\subsection*{51}

\textbf{張玄之、顧敷是顧和中外孫,皆少而聰惠,和並知之,而常謂顧勝,親重偏至,張頗不懕。}{\footnotesize 敷別見。\textbf{續晉陽秋}曰張玄之,字祖希,吳郡太守澄之孫也,少以學顯,歷吏部尚書,出為冠軍將軍、吳興太守,會稽內史謝玄同時之郡,論者以為南北之望,玄之名亞謝玄,時亦稱南北二玄,卒於郡。}\textbf{於時張年九歲,顧年七歲,和與俱至寺中,見佛般泥洹像,弟子有泣者、有不泣者,和以問二孫,玄謂:「彼親故泣,彼不親故不泣。」敷曰:「不然,當由忘情故不泣,不能忘情故泣。」}{\footnotesize \textbf{大智度論}曰佛在陰庵羅雙樹間入般涅槃,臥北首,大地震動,諸三學人僉然不樂,郁伊交涕,諸無學人但念諸法一切無常。}

\subsection*{52}

\textbf{庾法暢造庾太尉,握麈尾至佳,公曰:「此至佳,那得在?」法暢曰:「廉者不求,貪者不與,故得在耳。」}{\footnotesize 法暢氏族所出未詳。法暢著人物論,自敘其美云「悟銳有神,才辭通辯」。}

\subsection*{53}

\textbf{庾穉恭為荊州,}{\footnotesize \textbf{庾翼別傳}曰翼,字穉恭,潁川鄢陵人也,少有大度,時論以經略許之,兄太尉亮薨,朝議推才,乃以翼都督七州,進征南將軍、荊州刺史。}\textbf{以毛扇上武帝,武帝疑是故物,}{\footnotesize \textbf{傅咸羽扇賦序}曰昔吳人直截鳥翼而搖之,風不減方圓二扇,而功無加,然中國莫有生意者,滅吳之後,翕然貴之,無人不用。\textbf{按}庾懌以白羽扇獻武帝,帝嫌其非新,反之,不聞翼也。}\textbf{侍中劉劭曰:}{\footnotesize \textbf{文字志}曰劭,字彥祖,彭城叢亭人,祖訥,司隸校尉,父松,成皋令,劭博識好學,多藝能,善草隸,初仕領軍參軍,太傅出東,劭謂京洛必危,乃單馬奔揚州,歷侍中、豫章太守。}\textbf{「柏梁雲構,工匠先居其下,管弦繁奏,鍾夔先聽其音,}{\footnotesize 鍾,鍾期也。夔,舜樂正。}\textbf{穉恭上扇,以好不以新。」庾後聞之曰:「此人宜在帝左右。」}

\subsection*{54}

\textbf{何驃騎亡後,}{\footnotesize 何充別見。}\textbf{徵褚公入,既至石頭,王長史、劉尹同詣褚,褚曰:「真長何以處我?」真長顧王曰:「此子能言。」褚因視王,王曰:「國自有周公。」}{\footnotesize \textbf{晉陽秋}曰充之卒,議者謂太后父裒宜秉朝政,裒自丹徒入朝,吏部尚書劉遐勸裒曰「會稽王令德,國之周公也,足下宜以大政付之」,裒長史王胡之亦勸歸藩,於是固辭歸京。}

\subsection*{55}

\textbf{桓公北征,經金城,見前為琅邪時種柳,皆已十圍,慨然曰:「木猶如此,人何以堪?」攀枝執條,泫然流涙。}{\footnotesize \textbf{桓溫別傳}曰溫,字元子,譙國龍亢人,漢五更桓榮後也,父彝,有識鑒,溫少有豪邁風氣,為溫嶠所知,累遷琅邪內史,進征西大將軍,鎮西夏,時逆胡未誅,餘燼假息,溫親勒郡卒,建旗致討,清蕩伊洛,展敬園陵,薨,諡宣武侯。}

\subsection*{56}

\textbf{簡文作撫軍時,嘗與桓宣武俱入朝,更相讓在前,宣武不得已而先之,因曰:「伯也執殳,為王前驅。」}{\footnotesize 衛詩也。殳長一丈二尺,無刃。}\textbf{簡文曰:「所謂無小無大,從公于邁。」}

\subsection*{57}

\textbf{顧悅與簡文同年,而髮蚤白,}{\footnotesize \textbf{中興書}曰悅,字君叔,晉陵人,初為殷浩揚州別駕,浩卒,上疏理浩,或諫以浩為太宗所廢,必不依許,悅固爭之,浩果得申,物論稱之,後至尚書左丞。}\textbf{簡文曰:「卿何以先白?」對曰:「蒲柳之姿,望秋而落,松柏之質,經霜彌茂。」}{\footnotesize \textbf{顧凱之}為父傳曰君以直道陵遲於世,入見王,王髮無二毛,而君已斑白,問君年,乃曰「卿何偏蚤白」,君曰「松柏之姿,經霜猶茂,臣蒲柳之質,望秋先零,受命之異也」,王稱善久之。}

\subsection*{58}

\textbf{桓公入峽,絕壁天懸,騰波迅急,}{\footnotesize \textbf{晉陽秋}曰溫以永和二年率所領七千餘人伐蜀,拜表輒行。}\textbf{迺歎曰:「既為忠臣,不得為孝子,如何?」}{\footnotesize \textbf{漢書}曰王陽為益州刺史,行部至邛僰九折阪,歎曰「奉先人遺體,奈何數乘此險」,以病去官,後王尊為刺史,至其阪,問吏曰「非王陽所畏之道邪」,吏曰「是」,叱其馭曰「驅之!王陽為孝子,王尊為忠臣」。}

\subsection*{59}

\textbf{初,熒惑入太微,尋廢海西,}{\footnotesize \textbf{晉陽秋}曰泰和六年閏十月,熒惑守太微端門,十一月,大司馬桓溫廢帝為海西公。\textbf{晉安帝紀}曰桓溫於枋頭奔敗,知民望之去也,乃屠袁真於壽陽,既而謂郗超曰「足以雪枋頭之恥乎」,超曰「未厭有識之情也,公六十之年,敗於大舉,不建高世之勳,未足以鎮厭民望」,因說溫以廢立之事,時溫夙有此謀,深納超言,遂廢海西。}\textbf{簡文登阼,復入太微,帝惡之,}{\footnotesize \textbf{徐廣晉紀}曰咸安元年十二月,熒惑逆行入太微,至二年七月猶在焉,帝懲海西之事,心甚憂之。}\textbf{時郗超為中書在直,}{\footnotesize \textbf{中興書}曰超,字景興,高平人,司空愔之子也,少而卓犖不羈,有曠世之度,累遷中書郎、司徒左長史。}\textbf{引超入曰:「天命脩短,故非所計,政當無復近日事不?」超曰:「大司馬方將外固封疆,內鎮社稷,必無若此之慮,臣為陛下以百口保之。」帝因誦庾仲初詩}{\footnotesize 庾闡從征詩也。}\textbf{曰:「志士痛朝危,忠臣哀主辱。」聲甚悽厲,郗受假還東,帝曰:「致意尊公,家國之事,遂至於此,由是身不能以道匡衛,思患預防,愧歎之深,言何能喻。」因泣下流襟。}{\footnotesize \textbf{續晉陽秋}曰帝外厭彊臣,憂憤不得志,在位二年而崩。}

\subsection*{60}

\textbf{簡文在暗室中坐,召宣武,宣武至,問:「上何在?」簡文曰:「某在斯。」時人以為能。}{\footnotesize \textbf{論語}曰師冕見,及階,子曰「階也」,及席,子曰「席也」,皆坐,子告之曰「某在斯,某在斯」。\textbf{注}歷告坐中人也。}

\subsection*{61}

\textbf{簡文入華林園,顧謂左右曰:「會心處不必在遠,翳然林水,便自有濠濮間想也,}{\footnotesize 濠濮,二水名也。\textbf{莊子}曰莊子與惠子游濠梁水上,莊子曰「鯈魚出游從容,是魚樂也」,惠子曰「子非魚,安知魚之樂邪」,莊子曰「子非我,安知我之不知魚之樂也」。莊周釣於濮水,楚王使二大夫造焉,曰「願以境內累莊子」,莊子持竿不顧,曰「吾聞楚有神龜者,死已三千年矣,巾笥而藏於廟,此寧曳尾於塗中,寧留骨而貴乎」,二大夫曰「寧曳尾於塗中」,莊子曰「往矣,吾亦寧曳尾塗中」。}\textbf{不覺鳥獸禽魚,自來親人。」}

\subsection*{62}

\textbf{謝太傅語王右軍曰:「中年傷於哀樂,與親友別,輒作數日惡。」王曰:}{\footnotesize \textbf{文字志}曰王羲之,字逸少,琅邪臨沂人,父曠,淮南太守,羲之少朗拔,為叔父廙所賞,善草隸,累遷江州刺史、右軍將軍、會稽內史。}\textbf{「年在桑榆,自然至此,正賴絲竹陶寫,恆恐兒輩覺,損欣樂之趣。」}

\subsection*{63}

\textbf{支道林常養數匹馬,或言道人畜馬不韻,支曰:「貧道重其神駿。」}{\footnotesize \textbf{高逸沙門傳}曰支遁,字道林,河內林慮人,或曰陳留人,本姓關氏,少而任心獨往,風期高亮,家世奉法,嘗於餘杭山沈思道行,泠然獨暢,年二十五始釋形入道,年五十三終於洛陽。}

\subsection*{64}

\textbf{劉尹與桓宣武共聽講禮記,桓云:「時有入心處,便覺咫尺玄門。」劉曰:「此未關至極,自是金華殿之語。」}{\footnotesize \textbf{漢書敘傳}曰班伯少受詩於師丹,大將軍王鳳薦伯於成帝,宜勸學,召見宴暱,拜為中常侍,時上方向學,鄭寬中、張禹朝夕入說尚書、論語於金華殿,詔伯受之。}

\subsection*{65}

\textbf{羊秉為撫軍參軍,少亡,有令譽,夏侯孝若為之敘,極相讚悼,}{\footnotesize \textbf{羊秉敘}曰秉,字長達,太山平陽人,漢南陽太守續曾孫,大父魏郡府君,即車騎掾元子也,府君夫人鄭氏無子,乃養秉,齠齔而佳,小心敬慎,十歲而鄭夫人薨,秉思容盡哀,俄而公府掾及夫人並卒,秉群從父率禮相承,人不閒其親,雍雍如也,仕參撫軍將軍事,將奮千里之足,揮沖天之翼,惜乎春秋三十有二而卒,昔罕虎死,子産以為無與為善,自夫子之沒,有子産之歎矣!亡後有子男又不育,是何行善而禍繁也,豈非司馬生之所惑歟?}\textbf{羊權為黃門侍郎,侍簡文坐,帝問曰:「夏侯湛}{\footnotesize 別見。}\textbf{作羊秉敘,絕可想,是卿何物,有後不?」}{\footnotesize \textbf{羊氏譜}曰權,字道輿,徐州刺史悅之子也,仕至尚書左丞。}\textbf{權潸然對曰:「亡伯令問夙彰,而無有繼嗣,雖名播天聽,然胤絕聖世。」帝嗟慨久之。}

\subsection*{66}

\textbf{王長史與劉真長別後相見,}{\footnotesize \textbf{王長史別傳}曰濛,字仲祖,太原晉陽人,其先出自周室,經漢魏,世為大族,祖父佐,北軍中候,父訥,葉令,濛神氣清韶,年十餘歲,放邁不群,弱冠檢尚,風流雅正,外絕榮競,內寡私欲,辟司徒掾、中書郎,以后父贈光祿大夫。}\textbf{王謂劉曰:「卿更長進。」答曰:「此若天之自高耳。」}{\footnotesize \textbf{語林}曰仲祖語真長曰「卿近大進」,劉曰「卿仰看邪」,王問何意,劉曰「不爾,何由測天之高也」。}

\subsection*{67}

\textbf{劉尹云:「人想王荊産佳,此想長松下當有清風耳。」}{\footnotesize 荊産,王徽小字也。\textbf{王氏譜}曰徽,字幼仁,琅邪人,祖父乂,平北將軍,父澄,荊州刺史,徽歷尚書郎、右軍司馬。}

\subsection*{68}

\textbf{王仲祖聞蠻語不解,茫然曰:「若使介葛盧來朝,故當不昧此語。」}{\footnotesize \textbf{春秋傳}曰介葛盧來朝魯,聞牛鳴,曰「是生三犧,皆用之矣,其音云」,問之而信。\textbf{杜預}注曰介,東夷國,葛盧,其君名也。}

\subsection*{69}

\textbf{劉真長為丹陽尹,許玄度出都就劉宿,}{\footnotesize \textbf{續晉陽秋}曰許詢,字玄度,高陽人,魏中領軍允玄孫,總角秀惠,眾稱神童,長而風情簡素,司徒掾辟,不就,蚤卒。}\textbf{牀帷新麗,飲食豐甘,許曰:「若保全此處,殊勝東山。」劉曰:「卿若知吉凶由人,吾安得不保此。」}{\footnotesize \textbf{春秋傳}曰吉凶無門,惟人所召。}\textbf{王逸少在坐曰:「令巢許遇稷契,當無此言。」二人並有愧色。}

\subsection*{70}

\textbf{王右軍與謝太傅共登冶城,}{\footnotesize \textbf{揚州記}曰冶城,吳時鼓鑄之所,吳平,猶不廢,王茂弘所治也。}\textbf{謝悠然遠想,有高世之志,王謂謝曰:「夏禹勤王,手足胼胝,}{\footnotesize \textbf{帝王世紀}曰禹治洪水,手足胼胝,世傳禹病偏枯,足不相過,今稱禹步是也。}\textbf{文王旰食,日不暇給,}{\footnotesize \textbf{尚書}曰文王自朝至于日昃,不遑暇食。}\textbf{今四郊多壘,}{\footnotesize \textbf{禮記}曰四郊多壘,卿大夫之辱也。}\textbf{宜人人自效,而虛談廢務,浮文妨要,恐非當今所宜。」謝答曰:「秦任商鞅,二世而亡,}{\footnotesize \textbf{戰國策}曰衛鞅,衛諸庶孽子也,名鞅,姓公孫氏,少好刑名學,為秦孝公相,封於商。}\textbf{豈清言致患邪?」}

\subsection*{71}

\textbf{謝太傅寒雪日內集,與兒女講論文義,俄而雪驟,公欣然曰:「白雪紛紛何所似?」兄子胡兒曰:}{\footnotesize 胡兒,謝朗小字也。\textbf{續晉陽秋}曰朗,字長度,安次兄據之長子,安蚤知之,文義艷發,名亞於玄,仕至東陽太守。}\textbf{「撒鹽空中差可擬。」兄女曰:「未若柳絮因風起。」公大笑樂,即公大兄無奕女,左將軍王凝之妻也。}{\footnotesize \textbf{王氏譜}曰凝之,字叔平,右將軍羲之第二子也,歷江州刺史、左將軍、會稽內史。\textbf{晉安帝紀}曰凝之事五斗米道,孫恩之攻會稽,凝之謂民吏曰「不須備防,吾已請大道,許遣鬼兵相助,賊自破矣」,既不設備,遂為恩所害。\textbf{婦人集}曰謝夫人名道蘊,有文才,所著詩賦誄頌傳於世。}

\subsection*{72}

\textbf{王中郎令伏玄度、習鑿齒}{\footnotesize \textbf{王中郎傳}曰坦之,字文度,太原晉陽人,祖東海太守承,清淡平遠,父述,貞貴簡正,坦之器度淳深,孝友天至,譽輯朝野,標的當時,累遷侍中、中書令,領北中郎將、徐兗二州刺史。\textbf{中興書}曰伏滔,字玄度,平昌安丘人,少有才學,舉秀才,大司馬桓溫參軍,領大著作、掌國史、遊擊將軍,卒。習鑿齒,字彥威,襄陽人,少以文稱,善尺牘,桓溫在荊州,辟為從事,歷治中、別駕,遷滎陽太守。}\textbf{論青楚人物,}{\footnotesize \textbf{滔集}載其論略曰滔以春秋時鮑叔、管仲、隰朋、召忽、輪扁、甯戚、麥丘人、逢丑父、晏嬰、涓子,戰國時公羊高、孟軻、鄒衍、田單、荀卿、鄒奭、莒大夫、田子方、檀子、魯連、淳于髡、盼子、田光、顏歜、黔子、於陵子仲、王叔、即墨大夫,前漢時伏徵君、終軍、東郭先生、叔孫通、萬石君、東方朔、安期先生,後漢時大司徒、伏三老、江革、逢萌、禽慶、承幼子、徐防、薛方、鄭康成、周孟玉、劉祖榮、臨孝存、侍其元矩、孫賓碩、劉仲謀、劉公山、王儀伯、郎宗、禰正平、劉成國,魏時管幼安、邴根矩、華子魚、徐偉長、任昭先、伏高陽,此皆青士有才德者也,鑿齒以神農生於黔中,邵南詠其美化,春秋稱其多才,漢廣之風,不同雞鳴之篇,子文、叔敖,羞與管仲比德,接輿之歌鳳兮,漁父之詠滄浪,漢陰丈人之折子貢,市南宜僚、屠羊說之不為利回,魯仲連不及老萊夫妻,田光之於屈原,鄧禹、卓茂無敵於天下,管幼安不勝龐公,龐士元不推華子魚,何、鄧二尚書獨步於魏朝,樂令無對於晉世,昔伏羲葬南郡,少昊葬長沙,舜葬零陵,比其人則準的如此,論其土則群聖之所葬,考其風則詩人之所歌,尋其事則未有赤眉、黃巾之賊,此何如青州邪?滔與相往反,鑿齒無以對也。}\textbf{臨成,以示韓康伯,康伯都無言,王曰:「何故不言?」韓曰:「無可無不可。」}{\footnotesize \textbf{馬融注論語}曰惟義所在。}

\subsection*{73}

\textbf{劉尹云:「清風朗月,輒思玄度。」}{\footnotesize \textbf{晉中興士人書}曰許珣能清言,于時士人皆欽慕仰愛之。}

\subsection*{74}

\textbf{荀中郎在京口,}{\footnotesize \textbf{晉陽秋}曰荀羡,字令則,潁川人,光祿大夫崧之子也,清和有識裁,少以主壻為駙馬都尉,是時殷浩參謀百揆,引羡為援,頻莅義興、吳郡,超授北中郎將、徐州刺史,以蕃屏焉。\textbf{中興書}曰羡年二十八,出為徐兗二州,中興方伯之少,未有若羡者也。}\textbf{登北固望海云:}{\footnotesize \textbf{南徐州記}曰城西北有別嶺入江,三面臨水,高數十丈,號曰北固。}\textbf{「雖未覩三山,便自使人有凌雲意,若秦漢之君,必當褰裳濡足。」}{\footnotesize \textbf{史記封禪書}曰蓬萊、方丈、瀛洲,此三山世傳在海中,去人不遠,嘗有至者,言諸仙人不死藥在焉,黃金白銀為宮闕,草物禽獸盡白,望之如雲,及至,反居水下,欲到,即風引船而去,終莫能至,秦始皇登會稽,並海上,冀遇三神山之奇藥,漢武帝既封泰山,無風雨變至,方士更言蓬萊諸藥可得,於是上欣然東至海,冀獲蓬萊者。}

\subsection*{75}

\textbf{謝公云:「賢聖去人,其間亦邇。」子姪未之許,公歎曰:「若郗超聞此語,必不至河漢。」}{\footnotesize \textbf{超別傳}曰超精於理義,沙門支道林以為一時之俊。\textbf{莊子}曰肩吾問於連叔曰「吾聞言於接輿,大而無當,往而不反,怪怖其言,猶河漢而無極也」。}

\subsection*{76}

\textbf{支公好鶴,住剡東卬山,}{\footnotesize \textbf{支公書}曰山去會稽二百里。}\textbf{有人遺其雙鶴,少時,翅長欲飛,支意惜之,乃鎩其翮,鶴軒翥不能復飛,乃反顧翅垂頭,視之如有懊喪意,林曰:「既有凌霄之姿,何肯為人作耳目近玩?」養令翮成,置使飛去。}

\subsection*{77}

\textbf{謝中郎經曲阿後湖,問左右:「此是何水?」}{\footnotesize \textbf{中興書}曰謝萬,字萬石,太傅安弟也,才氣高俊,蚤知名,歷吏部郎、西中郎將、豫州刺史、散騎常侍。}\textbf{答曰:「曲阿湖。」}{\footnotesize \textbf{太康地記}曰曲阿,本名雲陽,秦始皇以有王氣,鑿地阬山以敗其勢,截其直道,使其阿曲,故曰曲阿也,吳還為雲陽,今復名曲阿。}\textbf{謝曰:「故當淵注渟著,納而不流。」}

\subsection*{78}

\textbf{晉武帝毎餉山濤恆少,謝太傅}{\footnotesize 安也。}\textbf{以問子弟,車騎}{\footnotesize 玄也。}\textbf{答曰:「當由欲者不多,而使與者忘少。」}{\footnotesize \textbf{謝車騎家傳}曰玄,字幼度,鎮西奕第三子也,神理明俊,善微言,叔父太傅嘗與子姪燕集,問「武帝任山公以三事,任以官人,至於賜予,不過斤合,當有旨不」,玄答有辭致也。}

\subsection*{79}

\textbf{謝胡兒語庾道季:}{\footnotesize 道季,庾龢小字。\textbf{徐廣晉紀}曰龢,字道季,太尉亮子也,風情率悟,以文談致稱於時,歷仕至丹陽尹,兼中領軍。}\textbf{「諸人莫當就卿談,可堅城壘。」庾曰:「若文度來,我以偏師待之,康伯來,濟河焚舟。」}{\footnotesize \textbf{春秋傳}曰秦伯伐晉,濟河焚舟。\textbf{杜預}曰示必死。}

\subsection*{80}

\textbf{李弘度常歎不被遇,}{\footnotesize \textbf{中興書}曰李充,字弘度,江夏郢人也,祖康、父矩皆有美名,充初辟丞相掾、記室參軍,以貧,求剡縣,遷大著作、中書郎。}\textbf{殷揚州}{\footnotesize 殷浩別見。}\textbf{知其家貧,問:「君能屈志百里不?」李答曰:「北門之歎,久已上聞,}{\footnotesize 衛詩北門,刺仕不得志也。}\textbf{窮猿奔林,豈暇擇木?」遂授剡縣。}

\subsection*{81}

\textbf{王司州至吳興印渚中看,}{\footnotesize \textbf{王胡之別傳}曰胡之,字脩齡,琅邪臨沂人,王廙之子也,歷吳興太守,徵侍中、丹陽尹、祕書監,並不就,拜使持節、都督司州諸軍事、西中郎將、司州刺史。\textbf{吳興記}曰於潛縣東七十里有印渚,渚傍有白石山,峻壁四十丈,印渚蓋眾溪之下流也,印渚已上至縣,悉石瀨惡道,不可行船,印渚已下,水道無險,故行旅集焉。}\textbf{歎曰:「非惟使人情開滌,亦覺日月清朗。」}

\subsection*{82}

\textbf{謝萬作豫州都督,新拜,當西之都邑,相送累日,謝疲頓,於是高侍中往,}{\footnotesize \textbf{中興書}曰高崧,字茂琰,廣陵人,父悝,光祿大夫,崧少好學,善史傳,累遷吏部郎、侍中,以公累免官。}\textbf{徑就謝坐,因問:「卿今仗節方州,當疆理西蕃,何以為政?」謝粗道其意,高便為謝道形勢,作數百語,謝遂起坐,高去後,謝追曰:「阿酃故粗有才具。」}{\footnotesize 阿酃,崧小字也。}\textbf{謝因此得終坐。}

\subsection*{83}

\textbf{袁彥伯為謝安南司馬,}{\footnotesize 安南,謝奉,別見。}\textbf{都下諸人送至瀨鄉,將別,既自悽惘,歎曰:「江山遼落,居然有萬里之勢。」}{\footnotesize \textbf{續晉陽秋}曰袁宏,字彥伯,陳郡人,魏郎中令煥六世孫也,祖猷,侍中,父勖,臨汝令,宏起家建威參軍、安南司馬記室,太傅謝安賞宏機捷辯速,自吏部郎出為東陽郡,乃祖之於冶亭,時賢皆集,安欲卒迫試之,執手將別,顧左右取一扇而贈之,宏應聲答曰「輒當奉揚仁風,慰彼黎庶」,合坐歎其要捷,性直亮,故位不顯也,在郡卒。}

\subsection*{84}

\textbf{孫綽賦遂初,築室畎川,自言見止足之分,}{\footnotesize \textbf{中興書}曰綽,字興公,太原中都人,少以文稱,歷太學博士、大著作、散騎常侍。\textbf{遂初賦敘}曰余少慕老莊之道,仰其風流久矣,卻感於陵賢妻之言,悵然悟之,乃經始東山,建五畝之宅,帶長阜,倚茂林,孰與坐華幕、擊鍾鼓者同年而語其樂哉。}\textbf{齋前種一株松,恆自手壅治之,高世遠時亦鄰居,}{\footnotesize 世遠,高柔字也,別見。}\textbf{語孫曰:「松樹子非不楚楚可憐,但永無棟梁用耳。」孫曰:「楓柳雖合抱,亦何所施?」}

\subsection*{85}

\textbf{桓征西治江陵城甚麗,}{\footnotesize \textbf{盛弘之荊州記}曰荊州城臨漢江,臨江王所治,王被徵,出城北門而車軸折,父老泣曰「吾王去不還矣」,從此不開北門。}\textbf{會賓僚出江津望之,云:「若能目此城者有賞。」顧長康時為客,在坐,因曰:「遙望層城,丹樓如霞。」桓即賞以二婢。}

\subsection*{86}

\textbf{王子敬語王孝伯曰:「羊叔子自復佳耳,然亦何與人事?}{\footnotesize \textbf{晉諸公贊}曰羊祜,字叔子,太山平陽人也,世長吏二千石,至祜九世,以清德稱,為兒時,游汶濱,有行父止而觀焉,歎息曰「處士大好相,善為之,未六十,當有重功於天下,即富貴,無相忘」,遂去,莫知所在,累遷都督荊州諸軍事,自在南夏,吳人說服,稱曰羊公,莫敢名者,南州人聞公喪,號哭罷市。}\textbf{故不如銅雀臺上妓。」}{\footnotesize \textbf{魏武遺令}曰以吾妾與妓人皆著銅雀臺上,施六尺牀繐帷,月朝十五日,輒使向帳作伎。}

\subsection*{87}

\textbf{林公見東陽長山曰:「何其坦迤。」}{\footnotesize \textbf{會稽土地志}曰山靡迤而長,縣因山得名。}

\subsection*{88}

\textbf{顧長康從會稽還,人問山川之美,顧云:「千巖競秀,萬壑爭流,草木蒙籠其上,若雲興霞蔚。」}{\footnotesize \textbf{丘淵之文章錄}曰顧愷之,字長康,晉陵人,父說,尚書左丞,愷之義熙初為散騎常侍。}

\subsection*{89}

\textbf{簡文崩,孝武年十餘歲立,至暝不臨,}{\footnotesize \textbf{宋明帝文章志}曰孝武皇帝諱昌明,簡文第三子也,初,簡文觀讖書曰「晉氏阼盡昌明」,及帝誕育,東方始明,故因生時以為諱,而相與忘告簡文,問之,乃以諱對,簡文流涕曰「不意我家昌明便出」,帝聰惠,推賢任才,年三十五崩。}\textbf{左右啓:「依常應臨。」帝曰:「哀至則哭,何常之有?」}

\subsection*{90}

\textbf{孝武將講孝經,謝公兄弟與諸人私庭講習,}{\footnotesize \textbf{續晉陽秋}曰寧康三年九月九日,帝講孝經,僕射謝安侍坐,吏部尚書陸納、兼侍中卞耽讀,黃門侍郎謝石、吏部袁宏兼執經,中書郎車胤、丹陽尹王溫摘句。}\textbf{車武子難苦問謝,}{\footnotesize 車胤別見。}\textbf{謂袁羊曰:「不問則德音有遺,多問則重勞二謝。」}{\footnotesize 袁羊,喬小字也。\textbf{袁氏家傳}曰喬,字彥升,陳郡人,父瓌,光祿大夫,喬歷尚書郎、江夏相,從桓溫平蜀,封湘西伯、益州刺史。}\textbf{袁曰:「必無此嫌。」車曰:「何以知爾?」袁曰:「何嘗見明鏡疲於屢照、清流憚於惠風?」}

\subsection*{91}

\textbf{王子敬云:「從山陰道上行,}{\footnotesize \textbf{會稽土地志}曰邑在山陰,故以名焉。}\textbf{山川自相映發,使人應接不暇,若秋冬之際,尤難為懷。」}{\footnotesize \textbf{會稽郡記}曰會稽境特多名山水,峰崿隆峻,吐納雲霧,松栝楓柏,擢榦竦條,潭壑鏡徹,清流瀉注,王子敬見之曰「山水之美,使人應接不暇」。}

\subsection*{92}

\textbf{謝太傅問諸子姪:「子弟亦何預人事,而正欲使其佳?」諸人莫有言者,車騎答曰:}{\footnotesize 謝玄。}\textbf{「譬如芝蘭玉樹,欲使其生於階庭耳。」}

\subsection*{93}

\textbf{道壹道人好整飾音辭,}{\footnotesize \textbf{王珣遊嚴陵瀨詩敘}曰道壹,姓竺氏,名德。\textbf{沙門題目}曰道壹文鋒富贍,孫綽為之贊曰「馳騁遊說,言固不虛,惟茲壹公,綽然有餘,譬若春圃,載芬載敷,條柯猗蔚,枝榦扶疏」。}\textbf{從都下還東山,經吳中,已而會雪下,未甚寒,諸道人問在道所經,壹公曰:「風霜固所不論,乃先集其慘澹,郊邑正自飄瞥,林岫便已皓然。」}

\subsection*{94}

\textbf{張天錫為涼州刺史,稱制西隅,既為苻堅所禽,用為侍中,後於壽陽俱敗,至都,}{\footnotesize \textbf{張資涼州記}曰天錫,字純嘏,安定烏氏人,張耳後也,曾祖軌,永嘉中為涼州刺史,值京師大亂,遂據涼土,天錫簒位,自立為涼州牧,苻堅使將姚萇攻沒涼州,天錫歸長安,堅以為侍中、比部尚書、歸義侯,從堅至壽陽,堅軍敗,遂南歸,拜散騎常侍、西平公。\textbf{中興書}曰天錫後以貧拜廬江太守,薨,贈侍中。}\textbf{為孝武所器,毎入言論,無不竟日,頗有嫉己者,於坐問張:「北方何物可貴?」張曰:「桑椹甘香,鴟鴞革響,}{\footnotesize \textbf{詩魯頌}曰翩彼飛鴞,集于泮林,食我桑椹,懷我好音。}\textbf{淳酪養性,人無嫉心。」}{\footnotesize \textbf{西河舊事}曰河西牛羊肥,酪過精好,但寫酪置革上,都不解散也。}

\subsection*{95}

\textbf{顧長康拜桓宣武墓,作詩云:「山崩溟海竭,魚鳥將何依。」}{\footnotesize \textbf{宋明帝文章志}曰愷之為桓溫參軍,甚被親暱。}\textbf{人問之曰:「卿憑重桓乃爾,哭之狀其可見乎?」顧曰:「鼻如廣莫長風,眼如懸河決溜。」}{\footnotesize \textbf{春秋考異郵}曰距不周風四十五日,廣莫風至,廣莫者,精大備也,蓋北風也,一曰寒風。}\textbf{或曰:「聲如震雷破山,涙如傾河注海。」}

\subsection*{96}

\textbf{毛伯成既負其才氣,常稱:「寧為蘭摧玉折,不作蕭敷艾榮。」}{\footnotesize \textbf{征西寮屬名}曰毛玄,字伯成,潁川人,仕至征西行軍參軍。}

\subsection*{97}

\textbf{范甯作豫章,}{\footnotesize \textbf{中興書}曰甯,字武子,慎陽縣人,博學通覽,累遷中書郎、豫章太守。}\textbf{八日請佛有板,眾僧疑,或欲作答,有小沙彌在坐末曰:「世尊默然,則為許可。」眾從其義。}

\subsection*{98}

\textbf{司馬太傅齋中夜坐,}{\footnotesize \textbf{孝文王傳}曰王諱道子,簡文皇帝第五子也,封會稽王,領司徒、揚州刺史,進太傅,為桓玄所害,贈丞相。}\textbf{于時天月明淨,都無纖翳,太傅歎以為佳,謝景重在坐,}{\footnotesize \textbf{續晉陽秋}曰謝重,字景重,陳郡人,父朗,東陽太守,重明秀有才會,終驃騎長史。}\textbf{答曰:「意謂乃不如微雲點綴。」太傅因戲謝曰:「卿居心不淨,乃復強欲滓穢太清邪?」}

\subsection*{99}

\textbf{王中郎甚愛張天錫,問之曰:「卿觀過江諸人,經緯江左,軌轍有何偉異?後來之彥,復何如中原?」張曰:「研求幽邃,自王、何以還,因時脩制,荀、樂之風。」}{\footnotesize 荀顗、荀勖脩定法制,樂則未聞。}\textbf{王曰:「卿知見有餘,何故為苻堅所制?」}{\footnotesize \textbf{張資涼州記}曰天錫明鑒穎發,英聲少著。}\textbf{答曰:「陽消陰息,故天步屯蹇,否剝成象,豈足多譏?」}

\subsection*{100}

\textbf{謝景重女適王孝伯兒,二門公甚相愛美,}{\footnotesize \textbf{謝氏譜}曰重女月鏡,適王恭子愔之。}\textbf{謝為太傅長史,被彈,王即取作長史,帶晉陵郡,太傅已構嫌孝伯,不欲使其得謝,還取作咨議,外示縶維,而實以乖間之,及孝伯敗後,太傅繞東府城行散,}{\footnotesize \textbf{丹陽記}曰東府城西,有簡文為會稽王時第,東則孝文王道子府,道子領揚州,仍住先舍,故俗稱東府。}\textbf{僚屬悉在南門要望候拜,時謂謝曰:「王甯異謀,}{\footnotesize 阿甯,王恭小字也。}\textbf{云是卿為其計。」謝曾無懼色,斂笏對曰:「樂彥輔有言,豈以五男易一女?」太傅善其對,因舉酒勸之曰:「故自佳,故自佳。」}

\subsection*{101}

\textbf{桓玄義興還後,見司馬太傅,太傅已醉,坐上多客,問人云:「桓溫來欲作賊,如何?」}{\footnotesize \textbf{晉安帝紀}曰溫在姑孰,諷朝廷求九錫,謝安使吏部郎袁宏具其草,以示僕射王彪之,彪之作色曰「丈夫豈可以此事語人邪」,安徐問其計,彪之曰「聞其疾已篤,且可緩其事」,安從之,故不行。}\textbf{桓玄伏不得起,謝景重時為長史,舉板答曰:「故宣武公黜昏暗,登聖明,功超伊霍,紛紜之議,裁之聖鑒。」太傅曰:「我知,我知。」即舉酒云:「桓義興,勸卿酒。」桓出謝過。}{\footnotesize \textbf{檀道鸞}論之曰道子可謂易於由言,謝重能解紛紜矣。}

\subsection*{102}

\textbf{宣武移鎮南州,制街衢平直,人謂王東亭曰:}{\footnotesize \textbf{王司徒傳}曰王珣,字元琳,丞相導之孫、領軍洽之子也,少以清秀稱,大司馬桓溫辟為主簿,從討袁真,封交趾望海縣東亭侯,累遷尚書左僕射、領選,進尚書令。}\textbf{「丞相初營建康,無所因承,而制置紆曲,方此為劣。」}{\footnotesize \textbf{晉陽秋}曰蘇峻既誅,大事克平之後,都邑殘荒,溫嶠議徙都豫章,以即豐全,朝士及三吳豪傑謂可遷都會稽,王導獨謂「不宜遷都,建業,往之秣陵,古者既有帝王所治之表,又孫仲謀、劉玄德俱謂是王者之宅,今雖凋殘,宜修勞來旋定之道,鎮靜群情,且百堵皆作,何患不克復乎」,終至康寧,導之策也。}\textbf{東亭曰:「此丞相乃所以為巧,江左地促,不如中國,若使阡陌條暢,則一覽而盡,故紆餘委曲,若不可測。」}

\subsection*{103}

\textbf{桓玄詣殷荊州,殷在妾房晝眠,左右辭不之通,桓後言及此事,殷云:「初不眠,縱有此,豈不以賢賢易色也?」}{\footnotesize \textbf{孔安國注論語}曰言以好色之心好賢人則善。}

\subsection*{104}

\textbf{桓玄問羊孚:}{\footnotesize \textbf{羊氏譜}曰孚,字子道,泰山人,祖楷,尚書郎,父綏,中書郎,孚歷太學博士、州別駕、太尉參軍,年四十六卒。}\textbf{「何以共重吳聲?」羊曰:「當以其妖而浮。」}

\subsection*{105}

\textbf{謝混問羊孚:「何以器舉瑚璉?」}{\footnotesize \textbf{晉安帝紀}曰混,字叔源,陳郡人,司空琰少子也,文學砥礪立名,累遷中書令、尚書左僕射,坐黨劉毅伏誅。\textbf{論語}子貢問曰「賜也何如」,子曰「汝器也」,曰「何器也」,曰「瑚璉也」。\textbf{鄭玄}注曰黍稷器,夏曰瑚,殷曰璉。}\textbf{羊曰:「故當以為接神之器。」}

\subsection*{106}

\textbf{桓玄既簒位,後御牀微陷,群臣失色,侍中殷仲文進曰:}{\footnotesize \textbf{續晉陽秋}曰仲文,字仲文,陳郡人,祖融,太常,父康,吳興太守,仲文聞玄平京邑,棄郡投焉,玄甚說之,引為咨議參軍,時王謐見禮而不親,卞範之被親而少禮,其寵遇隆重,兼於王、卞矣,及玄簒位,以佐命親貴,厚自封崇,輿馬器服,窮極綺麗,後房妓妾數十,絲竹不絕音,性甚貪吝,多納賄賂,家累千金,常若不足,玄既敗,先投義軍,累遷侍中、尚書,以罪伏誅。}\textbf{「當由聖德淵重,厚地所以不能載。」時人善之。}

\subsection*{107}

\textbf{桓玄既簒位,將改置直館,問左右:「虎賁中郎省,應在何處?」有人答曰:「無省。」當時殊忤旨,問:「何以知無?」答曰:「潘岳秋興賦敘曰『余兼虎賁中郎將,寓直散騎之省』。」}{\footnotesize 岳別見。其\textbf{賦敘}曰晉十有四年,余年三十二,始見二毛,以太尉掾兼虎賁中郎將,寓直散騎之省,高閣連雲,陽景罕曜,僕野人也,猥廁朝列,譬猶池魚籠鳥,有江湖山藪之思,於是染翰操紙,慨然而賦,于時秋至,故以秋興命篇。}\textbf{玄咨嗟稱善。}{\footnotesize \textbf{劉謙之晉紀}曰玄欲復虎賁中郎將,疑應直與不,訪之僚佐,咸莫能定,參軍劉簡之對曰「昔潘岳秋興賦敘云『余兼虎賁中郎將,寓直於散騎之省』,以此言之,是應直也」,玄懽然從之。此語微異,又答者未知姓名,故詳載之。}

\subsection*{108}

\textbf{謝靈運好戴曲柄笠,}{\footnotesize \textbf{丘淵之新集錄}曰靈運,陳郡陽夏人,祖玄,車騎將軍,父渙,祕書郎,靈運歷祕書監、侍中、臨川內史,以罪伏誅。}\textbf{孔隱士謂曰:「卿欲希心高遠,何不能遺曲蓋之貌?」}{\footnotesize \textbf{宋書}曰孔淳之,字彥深,魯國人,少以辭榮就約,徵聘無所就,元嘉初,散騎郎徵,不到,隱上虞山。}\textbf{謝答曰:「將不畏影者未能忘懷。」}{\footnotesize \textbf{莊子}云漁父謂孔子曰「人有畏影惡跡而去之走者,舉足逾數而跡逾多,走逾疾而影不離,自以尚遲,疾走不休,絕力而死,不知處陰以休影,處靜以息跡,愚亦甚矣!子脩心守真,還以物與人,則無異矣,不脩身而求之人,不亦外事者乎」。}