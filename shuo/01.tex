\chapter{德行第一}

\subsection*{1}

\textbf{陳仲舉言為士則,行為世範,登車攬轡,有澄清天下之志,}{\footnotesize \textbf{汝南先賢傳}曰陳蕃,字仲舉,汝南平輿人,有室荒蕪不埽除,曰「大丈夫當為國家埽天下」,值漢桓之末,閹豎用事,外戚豪橫,及拜太傅,與大將軍竇武謀誅宦官,反為所害。}\textbf{為豫章太守,}{\footnotesize \textbf{海內先賢傳}曰蕃為尚書,以忠正忤貴戚,不得在臺,遷豫章太守。}\textbf{至,便問徐孺子所在,欲先看之,}{\footnotesize \textbf{謝承後漢書}曰徐穉,字孺子,豫章南昌人,清妙高跱,超世絕俗,前後為諸公所辟,雖不就,及其死,萬里赴弔,常預炙雞一隻,以綿漬酒中,暴乾以裹雞,徑到所赴冢隧外,以水漬綿,斗米飯,白茅為藉,以雞置前,酹酒畢,留謁即去,不見喪主。}\textbf{主簿曰:「群情欲府君先入廨。」陳曰:「武王式商容之閭,席不暇煗,}{\footnotesize \textbf{許叔重}曰商容,殷之賢人,老子師也。車上跽曰式。}\textbf{吾之禮賢,有何不可?」}{\footnotesize \textbf{袁宏漢紀}曰蕃在豫章,為穉獨設一榻,去則懸之,見禮如此。}

\subsection*{2}

\textbf{周子居常云:「吾時月不見黃叔度,則鄙吝之心已復生矣。」}{\footnotesize 子居別見。\textbf{典略}曰黃憲,字叔度,汝南慎陽人,時論者咸云顔子復生,而族出孤鄙,父為牛醫,潁川荀季和執憲手曰「足下,吾師範也」,後見袁奉高曰「卿國有顔子,寧知之乎」,奉高曰「卿見吾叔度邪」,戴良少所服下,見憲則自降薄,悵然若有所失,母問「汝何不樂乎,復從牛醫兒所來邪」,良曰「瞻之在前,忽焉在後,所謂良之師也」。}

\subsection*{3}

\textbf{郭林宗至汝南,造袁奉高,}{\footnotesize \textbf{續漢書}曰郭泰,字林宗,太原介休人,泰少孤,年二十,行學至城阜屈伯彥精廬,乏食,衣不蓋形,而處約味道,不改其樂,李元禮一見稱之,曰「吾見士多矣,無如林宗者也」,及卒,蔡伯喈為作碑,曰「吾為人作銘,未嘗不有慚容,唯為郭有道碑頌無愧耳」,初,以有道君子徵,泰曰「吾觀乾象人事,天之所廢,不可支也」,遂辭以疾。\textbf{汝南先賢傳}曰袁宏,字奉高,慎陽人,友黄叔度於童齒,薦陳仲舉於家巷,辟太尉掾,卒。}\textbf{車不停軌,鸞不輟軛,詣黃叔度,乃彌日信宿,人問其故,林宗曰:「叔度汪汪如萬頃之陂,澄之不清,擾之不濁,其器甚廣,難測量也。」}{\footnotesize \textbf{泰別傳}曰薛恭祖問之,泰曰「奉高之器,譬諸汎濫,雖清易挹耳」。}

\subsection*{4}

\textbf{李元禮風格秀整,高自標持,欲以天下名教是非為己任,}{\footnotesize \textbf{薛瑩後漢書}曰李膺,字元禮,潁川襄城人,抗志清妙,有文武儁才,遷司隸校尉,為黨事自殺。}\textbf{後進之士有升其堂者,皆以為登龍門。}{\footnotesize \textbf{三秦記}曰龍門,一名河津,去長安九百里,水懸絕,龜魚之屬莫能上,上則化為龍矣。}

\subsection*{5}

\textbf{李元禮嘗歎荀淑、鍾皓,}{\footnotesize \textbf{先賢行狀}曰荀淑,字季和,潁川潁陰人也,所拔韋褐芻牧之中、執案刀筆之吏,皆為英彥,舉方正,補朗陵侯相,所在流化。鍾皓,字季明,潁川長社人,父祖至德著名,皓高風承世,除林慮長,不之官,人位不足,天爵有餘。}\textbf{曰:「荀君清識難尚,鍾君至德可師。」}{\footnotesize \textbf{海內先賢傳}曰潁川先輩為海內所師者,定陵陳穉叔、潁陰荀淑、長社鍾皓,少府李膺宗此三君,常言「荀君清識難尚,陳鍾至德可師」。}

\subsection*{6}

\textbf{陳太丘詣荀朗陵,貧儉無僕役,}{\footnotesize \textbf{陳寔傳}曰寔,字仲弓,潁川許昌人,為聞喜令、太丘長,風化宣流。}\textbf{乃使元方將車,}{\footnotesize \textbf{先賢行狀}曰陳紀,字元方,寔長子也,至德絕俗,與寔高名並著,而弟諶又配之,每宰府辟召,羔雁成群,世號三君,百城皆圖畫。}\textbf{季方持杖從後,長文尚小,載著車中,既至,荀使叔慈應門,慈明行酒,餘六龍下食,}{\footnotesize \textbf{張璠漢紀}曰淑有八子,儉、緄、靖、燾、汪、爽、肅、敷,淑居西豪里,縣令范康曰「昔高陽氏有才子八人」,遂署其里為高陽里,時人號曰八龍。}\textbf{文若亦小,坐著厀前,于時太史奏「真人東行」。}{\footnotesize \textbf{檀道鸞續晉陽秋}曰陳仲弓從諸子姪造荀父子,于時德星聚,太史奏「五百里賢人聚」。}

\subsection*{7}

\textbf{客有問陳季方:}{\footnotesize \textbf{海內先賢傳}曰陳諶,字季方,寔少子也,才識博達,司空掾公車徵,不就。}\textbf{「足下家君太丘有何功德而荷天下重名?」季方曰:「吾家君譬如桂樹生泰山之阿,上有萬仞之高,下有不測之深,上為甘露所霑,下為淵泉所潤,當斯之時,桂樹焉知泰山之高、淵泉之深,不知有功德與無也。」}

\subsection*{8}

\textbf{陳元方子長文有英才,}{\footnotesize \textbf{魏書}曰陳群,字長文,祖寔,嘗謂宗人曰「此兒必興吾宗」,及長,有識度,其所善皆父黨。}\textbf{與季方子孝先}{\footnotesize \textbf{陳氏譜}曰諶子忠,字孝先,州辟不就。}\textbf{各論其父功德,爭之不能決,咨於太丘,太丘曰:「元方難為兄,季方難為弟。」}{\footnotesize 一作「元方難為弟,季方難為兄」。}

\subsection*{9}

\textbf{荀巨伯遠看友人疾,}{\footnotesize \textbf{荀氏家傳}曰巨伯,漢桓帝時人也,亦出潁川,未詳其始末。}\textbf{值胡賊攻郡,友人語巨伯曰:「吾今死矣,子可去。」巨伯曰:「遠來相視,子令吾去,敗義以求生,豈荀巨伯所行邪?」賊既至,謂巨伯曰:「大軍至,一郡盡空,汝何男子而敢獨止?」巨伯曰:「友人有疾,不忍委之,寧以我身代友人命。」賊相謂曰:「我輩無義之人,而入有義之國。」遂班軍而還,一郡並獲全。}

\subsection*{10}

\textbf{華歆遇子弟甚整,雖閒室之內,嚴若朝典。}{\footnotesize \textbf{魏志}曰歆,字子魚,平原高唐人。\textbf{魏略}曰靈帝時與北海邴原、管寧俱遊學相善,時號三人為一龍,謂歆為龍頭、寧為龍腹、原為龍尾。}\textbf{陳元方兄弟恣柔愛之道,而二門之裏,兩不失雍熙之軌焉。}

\subsection*{11}

\textbf{管寧、華歆共園中鋤菜,}{\footnotesize \textbf{傅子}曰寧,字幼安,北海朱虛人,齊相管仲之後也。}\textbf{見地有片金,管揮鋤與瓦石不異,華捉而擲去之,又嘗同席讀書,有乘軒冕過門者,寧讀如故,歆廢書出看,寧割席分坐曰:「子非吾友也。」}{\footnotesize \textbf{魏略}曰寧少恬静,常笑邴原、華子魚有仕宦意,及歆為司徒,上書讓寧,寧聞之,笑曰「子魚本欲作老吏,故榮之耳」。}

\subsection*{12}

\textbf{王朗每以識度推華歆,}{\footnotesize \textbf{魏書}曰朗,字景興,東海郯人,魏司徒。}\textbf{歆蜡日}{\footnotesize \textbf{禮記}曰天子大蜡八,伊耆氏始為蜡,蜡,索也,歲十二月合聚萬物而索饗之。\textbf{五經要義}曰三代名臘,夏曰嘉平,殷曰清祀,周曰大蜡,總謂之臘。\textbf{晉博士張亮}議曰蜡者,合聚百物索饗之,歲終休老息民也,臘者,祭宗廟五祀,傳曰「臘,接也」,祭則新故交接也,秦漢已來,臘之明日為初歲,古之遺語也。}\textbf{嘗集子姪燕飲,王亦學之,有人向張華說此事,張曰:「王之學華,皆是形骸之外,去之所以更遠。」}{\footnotesize \textbf{王隱晉書}曰張華,字茂先,范陽人也,累遷司空,而為趙王倫所害。}

\subsection*{13}

\textbf{華歆、王朗俱乘船避難,有一人欲依附,歆輒難之,朗曰:「幸尚寬,何為不可?」後賊追至,王欲舍所攜人,歆曰:「本所以疑,正為此耳,既已納其自託,寧可以急相棄邪?」遂攜拯如初,世以此定華、王之優劣。}{\footnotesize \textbf{華嶠譜叙}曰歆為下邽令,漢室方亂,乃與同志士鄭太等六七人避世,自武關出,道遇一丈夫獨行,願得與俱,皆哀許之,歆獨曰「不可,今在危險中,禍福患害,義猶一也,今無故受之,不知其義,若有進退,可中棄乎」,眾不忍,卒與俱行,此丈夫中道墮井,皆欲棄之,歆乃曰「已與俱矣,棄之不義」,卒共還,出之而後別。}

\subsection*{14}

\textbf{王祥事後母朱夫人甚謹,}{\footnotesize \textbf{晉諸公贊}曰祥,字休徵,琅邪臨沂人。\textbf{祥世家}曰祥父融,娶高平薛氏,生祥,繼室以廬江朱氏,生覽。\textbf{晉陽秋}曰後母數譖祥,屢以非理使祥,弟覽輒與祥俱,又虐使祥婦,覽妻亦趨而共之,母患,方盛寒冰凍,母欲生魚,祥解衣将剖冰求之,會有處冰小解,魚出。\textbf{蕭廣濟孝子傳}曰祥後母忽欲黄雀炙,祥念難卒致,須臾,有數十黄雀飛入其幕,母之所須,必自奔走,無不得焉,其誠至如此。}\textbf{家有一李樹,結子殊好,母恆使守之,時風雨忽至,祥抱樹而泣,}{\footnotesize \textbf{蕭廣濟孝子傳}曰祥後母庭中有李,始結子,使祥晝視鳥雀,夜則趍鼠,一夜,風雨大至,祥抱泣至曉,母見之惻然。}\textbf{祥嘗在別牀眠,母自往闇斫之,值祥私起,空斫得被,既還,知母憾之不已,因跪前請死,母於是感悟,愛之如己子。}{\footnotesize \textbf{虞預晉書}曰祥以後母故,陵遲不仕,年向六十,刺史呂虔檄為別駕,時人歌之曰「海沂之康,寔賴王祥,邦國不空,別駕之功」,累遷太保。}

\subsection*{15}

\textbf{晉文王稱阮嗣宗至慎,每與之言,言皆玄遠,未嘗臧否人物。}{\footnotesize \textbf{魏書}曰文王諱昭,字子上,宣帝第二子也。\textbf{魏氏春秋}曰阮籍,字嗣宗,陳留尉氏人,阮瑀子也,宏達不羈,不拘禮俗,兗州刺史王昶請與相見,終日不得與言,昶愧歎之,自以不能測也,口不論事,自然高邁。\textbf{李康家誡}曰昔嘗侍坐於先帝,時有三長史俱見,臨辭出,上曰「為官長當清、當慎、當勤,修此三者,何患不治乎」,並受詔,上顧謂吾等曰「必不得已而去,於斯三者何先」,或對曰「清固為本」,復問吾,吾對曰「清慎之道,相須而成,必不得已,慎乃為大」,上曰「卿言得之矣,可舉近世能慎者誰乎」,吾乃舉故太尉荀景倩、尚書董仲達、僕射王公仲,上曰「此諸人者,溫恭朝夕,執事有恪,亦各其慎也,然天下之至慎者,其唯阮嗣宗乎?每與之言,言及玄遠,而未嘗評論時事、臧否人物,可謂至慎乎」。}

\subsection*{16}

\textbf{王戎云:「與嵇康居二十年,未嘗見其喜慍之色。」}{\footnotesize \textbf{康集敘}曰康,字叔夜,譙國銍人。\textbf{王隱晉書}曰嵇本姓奚,其先避怨徙上虞,移譙國銍縣,以出自會稽,取國一支,音同本奚焉。\textbf{虞預晉書}曰銍有嵇山,家於其側,因氏焉。\textbf{康別傳}曰康性含垢藏瑕,愛惡不争於懷,喜怒不寄於顔,所知王濬沖在襄城,面數百,未嘗見其疾聲朱顔,此亦方中之美範、人倫之勝業也。\textbf{文章敘錄}曰康以魏長樂亭主壻遷郎中,拜中散大夫。}

\subsection*{17}

\textbf{王戎、和嶠同時遭大喪,俱以孝稱,王雞骨支牀,和哭泣備禮,}{\footnotesize \textbf{晉諸公贊}曰戎,字濬沖,琅邪人,太保祥宗族也,文皇帝輔政,鍾會薦之曰「裴楷清通,王戎簡要」,即俱辟為掾,晉踐阼,累遷荊州刺史,以平吳功封安豐侯。\textbf{晉陽秋}曰戎為豫州刺史,遭母憂,性至孝,不拘禮制,飲酒食肉,或觀棊弈,而容貌毁悴,杖而後起,時汝南和嶠亦名士也,以禮法自持,處大憂,量米而食,然顦顇哀毁不逮戎也。}\textbf{武帝謂劉仲雄曰:}{\footnotesize \textbf{王隱晉書}曰劉毅,字仲雄,東萊掖人,漢城陽景王後也,亮直清方,見有不善,必評論之,王公大人望風憚之,僑居陽平,太守杜恕致為功曹,沙汰郡吏三百餘人,三魏僉曰「但聞劉功曹,不聞杜府君」,累遷尚書、司隸校尉。}\textbf{「卿數省王、和不?聞和哀苦過禮,使人憂之。」仲雄曰:「和嶠雖備禮,神氣不損,王戎雖不備禮,而哀毀骨立,臣以和嶠生孝,王戎死孝,陛下不應憂嶠,而應憂戎。」}{\footnotesize \textbf{晉陽秋}曰世祖及時談以此貴戎也。}

\subsection*{18}

\textbf{梁王、趙王,}{\footnotesize \textbf{朱鳳晉書}曰宣帝張夫人生梁孝王彤,字子徽,位至太宰,桓夫人生趙王倫,字子彝,位至相國。}\textbf{國之近屬,貴重當時,裴令公}{\footnotesize \textbf{晉諸公贊}曰裴楷,字叔則,河東聞喜人,司空秀之從弟也,父徽,冀州刺史,有俊識,楷特精易義,累遷河南尹、中書令,卒。}\textbf{歲請二國租錢數百萬,以恤中表之貧者,或譏之曰:「何以乞物行惠?」裴曰:「損有餘、補不足,天之道也。」}{\footnotesize \textbf{名士傳}曰楷行己取與,任心而動,毁譽雖至,處之晏然,皆此類。}

\subsection*{19}

\textbf{王戎云:「太保居在正始中,不在能言之流,及與之言,理中清遠,將無以德掩其言。」}{\footnotesize \textbf{晉陽秋}曰祥少有美德行。}

\subsection*{20}

\textbf{王安豐遭艱,至性過人,裴令往弔之曰:「若使一慟果能傷人,濬沖必不免滅性之譏。」}{\footnotesize \textbf{曲禮}曰居喪之禮,毁瘠不形,視聽不衰,不勝喪,乃比於不慈不孝。\textbf{孝經}曰毁不滅性,聖人之教也。}

\subsection*{21}

\textbf{王戎父渾有令名,官至涼州刺史,}{\footnotesize \textbf{世語}曰渾,字長原,有才望,歷尚書、涼州刺史。}\textbf{渾薨,所歷州郡義故懷其德惠,相率致賻數百萬,戎悉不受。}{\footnotesize \textbf{虞預晉書}曰戎由是顯名。}

\subsection*{22}

\textbf{劉道真嘗為徒,}{\footnotesize \textbf{晉百官名}曰劉寶,字道真,高平人。徒,罪役作者。}\textbf{扶風王駿}{\footnotesize \textbf{虞預晉書}曰駿,字子臧,宣帝第十七子,好學至孝。\textbf{晉諸公贊}曰駿八歲為散騎常侍,侍魏齊王講,晉受禪,封扶風王,鎮關中,為政最美,薨,贈武王,西土思之,但見其碑贊者皆拜之而泣,其遺愛如此。}\textbf{以五百疋布贖之,既而用為從事中郎,當時以為美事。}

\subsection*{23}

\textbf{王平子、胡毋彥國諸人皆以任放為達,或有裸體者,}{\footnotesize \textbf{晉諸公贊}曰王澄,字平子,有達識,荊州刺史。\textbf{永嘉流人名}曰胡毋輔之,字彥國,泰山奉高人,湘州刺史。\textbf{王隱晉書}曰魏末,阮籍嗜酒荒放,露頭散髮,裸袒箕踞,其後貴游子弟阮瞻、王澄、謝鯤、胡毋輔之之徒皆祖述於籍,謂得大道之本,故去巾幘、脫衣服、露醜惡、同禽獸,甚者名之為通,次者名之為達也。}\textbf{樂廣笑曰:「名教中自有樂地,何為乃爾也。」}

\subsection*{24}

\textbf{郗公值永嘉喪亂,在鄉里甚窮餒,鄉人以公名德,傳共飴之,公常攜兄子邁及外生周翼二小兒往食,鄉人曰:「各自饑困,以君之賢,欲共濟君耳,恐不能兼有所存。」公於是獨往食,輒含飯著兩頰邊,還,吐與二兒,後並得存,同過江。}{\footnotesize \textbf{郗鑒別傳}曰鑒,字道徽,高平金鄉人,漢御史大夫郗慮後也,少有體正,躭思經籍,以儒雅著名,永嘉末,天下大亂,饑饉相望,冠帶以下皆割己之資供鑒,元皇徵為領軍,遷司空、太尉。\textbf{中興書}曰鑒兄子邁,字思遠,有幹世才略,累遷少府、中護軍。}\textbf{郗公亡,翼為剡縣,解職歸,席苫於公靈牀頭,心喪終三年。}{\footnotesize \textbf{周氏譜}曰翼,字子卿,陳郡人,祖奕,上谷太守,父優,車騎咨議,歷剡令、青州刺史、少府卿,六十四而卒。}

\subsection*{25}

\textbf{顧榮在洛陽,嘗應人請,覺行炙人有欲炙之色,因輟己施焉,同坐嗤之,榮曰:「豈有終日執之,而不知其味者乎?」後遭亂渡江,每經危急,常有一人左右己,問其所以,乃受炙人也。}{\footnotesize \textbf{文士傳}曰榮,字彥先,吳郡人,其先越王句踐之支庶,封於顧邑,子孫遂氏焉,世為吳著姓,大父雍,吳丞相,父穆,宜都太守,榮少朗俊機警,風穎標徹,歷廷尉正,曾在省與同僚共飲,見行炙者有異於常僕,乃割炙以噉之,後趙王倫篡位,其子為中領軍,逼用榮為長史,及倫誅,榮亦被執,凡受戮等輩十有餘人,或有救榮者,問其故,曰「某省中受炙臣也」,榮乃悟而歎曰「一餐之惠,恩今不忘,古人豈虛言哉」。}

\subsection*{26}

\textbf{祖光祿少孤貧,性至孝,常自為母炊爨作食,}{\footnotesize \textbf{王隱晉書}曰祖納,字士言,范陽遒人,九世孝廉,納諸母三兄,最治行操,能清言,歷太子中庶子、廷尉卿,避地江南,溫嶠薦為光祿大夫。}\textbf{王平北聞其佳名,以兩婢餉之,因取為中郎,}{\footnotesize \textbf{王乂別傳}曰乂,字叔元,琅邪臨沂人,時蜀新平,二將作亂,文帝西之長安,乃徵為相國司馬,遷大尚書,出督幽州諸軍事、平北將軍。}\textbf{有人戲之者曰:「奴價倍婢。」祖云:「百里奚亦何必輕於五羖之皮邪?」}{\footnotesize \textbf{楚國先賢傳}曰百里奚,字井伯,楚國人,少仕於虞,為大夫,晉欲假道於虞以伐虢,諫而不聽,奚乃去之。\textbf{說苑}曰秦穆公使賈人載鹽於虞,諸賈人買百里奚以五羊皮,穆公觀鹽,怪其牛肥,問其故,對曰「飲食以時,使之不暴,是以肥也」,公令有司沐浴衣冠之,公孫支讓其卿位,號曰五羖大夫。}

\subsection*{27}

\textbf{周鎮罷臨川郡,還都,未及上,住泊青溪渚,}{\footnotesize \textbf{永嘉流人名}曰鎮,字康時,陳留尉氏人也,祖父和,故安令,父震,司空長史。\textbf{中興書}曰鎮清約寡欲,所在有異績。}\textbf{王丞相往看之,}{\footnotesize \textbf{丞相別傳}曰王導,字茂弘,琅邪人,祖覽,以德行稱,父裁,侍御史,導少知名,家世貧約,恬暢樂道,未嘗以風塵經懷也。}\textbf{時夏月,暴雨卒至,舫至狹小,而又大漏,殆無復坐處,王曰:「胡威之清,何以過此?」即啓用為吳興郡。}{\footnotesize \textbf{晉陽秋}曰胡威,字伯虎,淮南人,父質,以忠清顯,質為荊州,威自京師往省之,及告歸,質賜威絹一匹,威跪曰「大人清高,於何得此」,質曰「是吾奉祿之餘,故以為汝糧耳」,威受而去,毎至客舍,自放驢取樵爨炊,食畢,復隨旅進道,質帳下都督陰齎糧要之,因與為伴,每事相助經營之,又進少飯,威疑之,密誘問之,乃知都督也,謝而遣之,後以白質,質杖都督一百,除其吏名,父子清慎如此,及威為徐州,世祖賜見,與論邊事及平生,帝歎其父清,因謂威曰「卿清孰與父」,對曰「臣清不如也」,帝曰「何以為勝汝邪」,對曰「臣父清畏人知,臣清畏人不知,是以不如遠矣」。}

\subsection*{28}

\textbf{鄧攸始避難,於道中棄己子,全弟子。}{\footnotesize \textbf{晉陽秋}曰攸,字伯道,平陽襄陵人,七歲喪父母及祖父母,持重九年,性清慎平簡。\textbf{鄧粲晉紀}曰永嘉中,攸為石勒所獲,召見,立幕下與語,說之,坐而飯焉,攸車所止,與胡人鄰轂,胡人失火燒車營,勒吏案問胡,胡誣攸,攸度不可與爭,乃曰「向為老姥作粥,失火延逸,罪應萬死」,勒知,遣之,所誣胡厚德攸,遺其驢馬,護送令得逸。\textbf{王隱晉書}曰攸以路遠,斫壞車,以牛馬負妻子以叛,賊又掠其牛馬,攸語妻曰「吾弟早亡,唯有遺民,今當步走,儋兩兒盡死,不如棄己兒,抱遺民,吾後猶當有兒」,婦從之。\textbf{中興書}曰攸棄兒於草中,兒啼呼追之,至莫復及,攸明日繋兒於樹而去,遂渡江,至尚書左僕射,卒,弟子綏服攸齊衰三年。}\textbf{既過江,取一妾,甚寵愛,歷年後訊其所由,妾具說是北人,遭亂,憶父母姓名,乃攸之甥也。攸素有德業,言行無玷,聞之哀恨終身,遂不復畜妾。}

\subsection*{29}

\textbf{王長豫為人謹順,事親盡色養之孝,}{\footnotesize \textbf{中興書}曰王悅,字長豫,丞相導長子也,仕至中書侍郎。}\textbf{丞相見長豫輒喜,見敬豫輒嗔,}{\footnotesize \textbf{文字志}曰王恬,字敬豫,導次子也,少卓犖不羈,疾學尚武,不為導所重,至中軍將軍,多才藝,善隸書,與濟陽江虨以善弈聞。}\textbf{長豫與丞相語,恆以慎密為端,丞相還臺,及行,未嘗不送至車後,恆與曹夫人併當箱篋。長豫亡後,丞相還臺,登車後,哭至臺門,曹夫人作簏,封而不忍開。}{\footnotesize \textbf{王氏譜}曰導娶彭城曹韶女,名淑。}

\subsection*{30}

\textbf{桓常侍聞人道深公者,輒曰:「此公既有宿名,加先達知稱,又與先人至交,不宜說之。」}{\footnotesize \textbf{桓彝別傳}曰彝,字茂倫,譙國龍亢人,漢五更桓榮十世孫也,父顥,有高名,彝少孤,識鑒明朗,避亂渡江,累遷散騎常侍。僧法深,不知其俗姓,蓋衣冠之胤也,道徽高扇,譽播山東,為中州劉公弟子,值永嘉亂,投迹楊土,居止京邑,內持法綱,外允具瞻,弘道之法師也,以業滋清淨,而不耐風塵,考室剡縣東二百里卬山中,同遊十餘人,高棲浩然,支道林宗其風範,與高麗道人書,稱其德行,年七十有九,終於山中也。}

\subsection*{31}

\textbf{庾公乘馬有的盧,}{\footnotesize \textbf{晉陽秋}曰庾亮,字元規,潁川鄢陵人,明穆皇后長兄也,淵雅有德量,時人方之夏侯太初、陳長文之倫,侍從父琛,避地會稽,端拱嶷然,郡人嚴憚之,覲接之者數人而已,累遷征西大將軍、荊州刺史。\textbf{伯樂相馬經}曰馬白頟入口至齒者,名曰榆雁,一名的盧,奴乘客死,主乘棄市,凶馬也。}\textbf{或語令賣去,}{\footnotesize \textbf{語林}曰殷浩勸公賣馬。}\textbf{庾云:「賣之必有買者,即當害其主,寧可不安己而移於他人哉?昔孫叔敖殺兩頭蛇以為後人,古之美談,}{\footnotesize \textbf{賈誼新書}曰孫叔敖為兒時,出道上,見兩頭蛇,殺而埋之,歸見其母,泣,問其故,對曰「夫見兩頭蛇者必死,今出見之,故爾」,母曰「蛇今安在」,對曰「恐後人見,殺而埋之矣」,母曰「夫有陰德,必有陽報,爾無憂也」,後遂興於楚朝,及長,為楚令尹。}\textbf{效之,不亦達乎?」}

\subsection*{32}

\textbf{阮光祿在剡,曾有好車,借者無不皆給,有人葬母,意欲借而不敢言,阮後聞之,歎曰:「吾有車而使人不敢借,何以車為?」遂焚之。}{\footnotesize \textbf{阮光祿別傳}曰裕,字思曠,陳留尉氏人,祖略,齊國內史,父顗,汝南太守,裕淹通有理識,累遷侍中,以疾築室會稽剡山,徵金紫光祿大夫,不就,年六十一卒。}

\subsection*{33}

\textbf{謝奕作剡令,}{\footnotesize \textbf{中興書}曰謝奕,字無奕,陳郡陽夏人,祖衡,太子少傅,父裒,吏部尚書,奕少有器鑒,辟太尉掾、剡令,累遷豫州刺史。}\textbf{有一老翁犯法,謝以醇酒罰之,乃至過醉而猶未已,太傅時年七八歲,著青布絝,在兄厀邊坐,諫曰:「阿兄!老翁可念,何可作此?」奕於是改容曰:「阿奴欲放去邪?」遂遣之。}

\subsection*{34}

\textbf{謝太傅絕重褚公,常稱:「褚季野雖不言,而四時之氣亦備。」}{\footnotesize \textbf{文字志}曰謝安,字安石,奕弟也,世有學行,安弘粹通遠,溫雅融暢,桓彝見其四歲時,稱之曰「此兒風神秀徹,當繼蹤王東海」,善行書,累遷太保、錄尚書事,贈太傅。\textbf{晉陽秋}曰禇裒,字季野,河南陽翟人,祖䂮,安東將軍,父洽,武昌太守,裒少有簡貴之風、沖默之稱,累遷江、兗二州刺史,贈侍中、太傅。}

\subsection*{35}

\textbf{劉尹在郡,臨終綿惙,聞閣下祠神鼓舞,正色曰:「莫得淫祀。」}{\footnotesize \textbf{劉尹別傳}曰惔,字真長,沛國蕭人也,漢氏之後,真長有雅裁,雖蓽門陋巷,晏如也,歷司徒左長史、侍中、丹陽尹,為政務鎮静信誠,風塵不能移也。}\textbf{外請殺車中牛祭神,真長答曰:「丘之禱久矣,勿復為煩。」}{\footnotesize \textbf{包氏論語}曰禱,請也。\textbf{孔安國}曰孔子素行合於神明,故曰「丘之禱久矣」。}

\subsection*{36}

\textbf{謝公夫人教兒,問太傅:「那得初不見君教兒?」答曰:「我常自教兒。」}{\footnotesize \textbf{謝氏譜}曰安娶沛國劉耽女。\textbf{按}太尉劉子真,清潔有志操,行己以禮,而二子不才,並黷貨致罪,子真坐免官,客曰「子奚不訓導之」,子真曰「吾之行事,是其耳目所聞見,而不放效,豈嚴訓所變邪」,安石之旨,同子真之意也。}

\subsection*{37}

\textbf{晉簡文為撫軍時,}{\footnotesize \textbf{續晉陽秋}曰帝諱昱,字道萬,中宗少子也,仁聞有智度,穆帝幼沖,以撫軍輔政,大司馬桓溫廢海西公而立帝,在位三年而崩。}\textbf{所坐牀上塵不聽拂,見鼠行跡,視以為佳,有參軍見鼠白日行,以手板批殺之,撫軍意色不說,門下起彈,教曰:「鼠被害,尚不能忘懷,今復以鼠損人,無乃不可乎?」}

\subsection*{38}

\textbf{范宣年八歲,後園挑菜,誤傷指,大啼,人問:「痛邪?」答曰:「非為痛!身體髮膚,不敢毀傷,是以啼耳。」}{\footnotesize \textbf{宣別傳}曰宣,字子宣,陳留人,漢萊蕪長范丹後也,年十歲,能誦詩書,兒童時,手傷改容,家人以其年幼,皆異之,徵太學博士、散騎常侍,一無所就,年五十四卒。}\textbf{宣潔行廉約,韓豫章遺絹百匹,不受,}{\footnotesize \textbf{中興書}曰宣家至貧,罕交人事,豫章太守殷羨見宣茅茨不完,欲為改室,宣固辭,羨愛之,以宣貧,加年饑疾疫,厚餉給之,宣又不受。\textbf{續晉陽秋}曰韓伯,字康伯,潁川人,好學,善言理,歷豫章太守、領軍將軍。}\textbf{減五十匹,復不受,如是減半,遂至一匹,既終不受,韓後與范同載,就車中裂二丈與范,云:「人寧可使婦無褌邪?」范笑而受之。}

\subsection*{39}

\textbf{王子敬病篤,道家上章,應首過,問子敬:「由來有何異同得失?」子敬云:「不覺有餘事,惟憶與郗家離婚。」}{\footnotesize \textbf{王氏譜}曰獻之娶高平郗曇女,名道茂,後離婚。\textbf{獻之別傳}曰祖父曠,淮南太守,父羲之,右將軍,咸寧中,詔尚餘姚公主,遷中書令,卒。}

\subsection*{40}

\textbf{殷仲堪既為荊州,值水,儉食,常五盌盤,外無餘肴,飯粒脫落盤席閒,輒拾以噉之,雖欲率物,亦緣其性真素,每語子弟云:「勿以我受任方州,云我豁平昔時意,今吾處之不易,貧者士之常,焉得登枝而捐其本,爾曹其存之。」}{\footnotesize \textbf{晉安帝紀}曰仲堪,陳郡人,太常融孫也,車騎將軍謝玄請為長史,孝武說之,俄為黃門侍郎,自殺袁悅之後,上深為晏駕後計,故先出王恭為北蕃,荊州刺史王忱死,乃中詔用仲堪代焉。}

\subsection*{41}

\textbf{初,桓南郡、楊廣共說殷荊州,宜奪殷覬南蠻以自樹,}{\footnotesize \textbf{桓玄別傳}曰玄,字敬道,譙國龍亢人,大司馬溫少子也,幼童中,溫甚愛之,臨終命以為嗣,年七歲,襲封南郡公,拜太子洗馬、義興太守,不得志,少時去職,歸其國,與荊州刺史殷仲堪素舊,情好甚隆。\textbf{周祗隆安記}曰廣,字德度,弘農人,楊震後也。\textbf{晉安帝紀}曰覬,字伯道,陳郡人,由中書郎出為南蠻校尉,覬亦以率易才悟著稱,與從弟仲堪俱知名。\textbf{中興書}曰初,仲堪欲起兵,密邀覬,覬不同,楊廣與弟佺期勸殺覬,仲堪不許。}\textbf{覬亦即曉其旨,嘗因行散,率爾去下舍,便不復還,內外無預知者,意色蕭然,遠同鬬生之無慍,時論以此多之。}{\footnotesize \textbf{春秋傳}曰楚令尹子文,鬬氏也。\textbf{論語}曰令尹子文三仕為令尹,無喜色,三已之,無慍色。}

\subsection*{42}

\textbf{王僕射在江州,為殷、桓所逐,奔竄豫章,存亡未測,}{\footnotesize \textbf{徐廣晉紀}曰王愉,字茂和,太原晉陽人,安北將軍坦之次子也,以輔國司馬出為江州刺史,愉始至鎮,而桓玄、楊佺期舉兵以應王恭,乘流奄至,愉無防,惶遽奔臨川,為玄所得,玄篡位,遷尚書左僕射。}\textbf{王綏在都,既憂戚在貌,居處飲食,每事有降,時人謂為「試守孝子」。}{\footnotesize \textbf{中興書}曰綏,字彥猷,愉子也,少有令譽,自王澤至坦之,六世盛德,綏又知名,于時冠冕,莫與為比,位至中書令、荊州刺史,桓玄敗後,與父愉謀反,伏誅。}

\subsection*{43}

\textbf{桓南郡}{\footnotesize 玄也。}\textbf{既破殷荊州,收殷將佐十許人,咨議羅企生亦在焉,}{\footnotesize \textbf{玄別傳}曰玄克荊州,殺殷道護及仲堪參軍羅企生、鮑季禮,皆仲堪所親仗也。}\textbf{桓素待企生厚,將有所戮,先遣人語云:「若謝我,當釋罪。」企生答曰:「為殷荊州吏,今荊州奔亡,存亡未判,我何顏謝桓公?」}{\footnotesize \textbf{中興書}曰企生,字宗伯,豫章人,殷仲堪初請為府功曹,桓玄來攻,轉咨議參軍,仲堪多疑少決,企生深憂之,謂其弟遵生曰「殷侯仁而無斷,事必無成,成敗天也,吾當死生以之」,及仲堪走,文武並無送者,惟企生從焉,路經家門,遵生紿之曰「作如此分別,何可不執手」,企生回馬授手,遵生便牽下之,謂曰「家有老母,將欲何行」,企生揮泣曰「今日之事,我必死之,汝等奉養,不失子道,一門之內,有忠與孝,亦復何恨」,遵生抱之愈急,仲堪於路待之,企生遙呼曰「今日死生是同,願少見待」,仲堪見其無脫理,策馬而去,俄而玄至,人士悉詣玄,企生獨不往,而營理仲堪家,或謂曰「玄性猜急,未能取卿誠節,若遂不詣,禍必至矣」,企生正色曰「我殷侯吏,見遇以國士,不能共殄醜逆,致此奔敗,何面目就桓求生乎」,玄聞,怒而收之,謂曰「相遇如此,何以見負」,企生曰「使君口血未乾,而生此奸計,自傷力劣,不能翦定凶逆,我死恨晚爾」,玄遂斬之,時年三十有七,眾咸悼之。}\textbf{既出市,桓又遣人問欲何言,答曰:「昔晉文王殺嵇康,而嵇紹為晉忠臣,}{\footnotesize \textbf{王隱晉書}曰紹,字延祖,譙國銍人,父康有奇才儁辯,紹十歲而孤,事母孝謹,累遷散騎常侍,惠帝敗於蕩陰,百官左右皆奔散,唯紹儼然端冕,以身衛帝,兵交御輦,飛箭雨集,遂以見害也。}\textbf{從公乞一弟以養老母。」桓亦如言宥之。桓先曾以一羔裘與企生母胡,胡時在豫章,企生問至,即日焚裘。}

\subsection*{44}

\textbf{王恭從會稽還,}{\footnotesize \textbf{周祗隆安記}曰恭,字孝伯,太原晉陽人,祖父濛,司徒左長史,風流標望,父蘊,鎮軍將軍,亦得世譽。\textbf{恭別傳}曰恭清廉貴峻,志存格正,起家著作郎,歷丹陽尹、中書令,出為五州都督、前將軍、青兗二州刺史。}\textbf{王大看之,}{\footnotesize 王忱小字佛大。\textbf{晉安帝紀}曰忱,字元達,平北將軍坦之第四子也,甚得名於當世,與族子恭少相善,齊聲見稱,仕至荊州刺史。}\textbf{見其坐六尺簟,因語恭:「卿東來,故應有此物,可以一領及我。」恭無言,大去後,即舉所坐者送之,既無餘席,便坐薦上,後大聞之甚驚,曰:「吾本謂卿多,故求耳。」對曰:「丈人不悉恭,恭作人無長物。」}

\subsection*{45}

\textbf{吳郡陳遺,}{\footnotesize 未詳。}\textbf{家至孝,母好食鐺底焦飯,遺作郡主簿,恆裝一囊,每煮食,輒貯錄焦飯,歸以遺母,後值孫恩賊出吳郡,}{\footnotesize \textbf{晉安帝紀}曰孫恩,一名靈秀,琅邪人,叔父泰事五斗米道,以謀反誅,恩逸逃於海上,聚眾十萬人,攻沒郡縣,後為臨海太守辛昺斬首送之。}\textbf{袁府君}{\footnotesize 山松別見。}\textbf{即日便征,遺已聚斂得數斗焦飯,未展歸家,遂帶以從軍,戰於滬瀆,敗,軍人潰散,逃走山澤,皆多饑死,遺獨以焦飯得活,時人以為純孝之報也。}

\subsection*{46}

\textbf{孔僕射為孝武侍中,豫蒙眷接,烈宗山陵,孔時為太常,形素羸瘦,著重服,竟日涕泗流漣,見者以為真孝子。}{\footnotesize \textbf{續晉陽秋}曰孔安國,字安國,會稽山陰人,車騎愉第六子也,少而孤貧,能善樹節,以儒素見稱,歷侍中、太常、尚書,遷左僕射、特進,卒。}

\subsection*{47}

\textbf{吳道助、附子兄弟居在丹陽郡後,遭母童夫人艱,}{\footnotesize 道助,坦之小字,附子,隱之小字也。\textbf{吳氏譜}曰坦之,字處靖,濮陽人,仕至西中郎將功曹,父堅,取東苑童儈女,名秦姬。}\textbf{朝夕哭臨,及思至,賓客弔省,號踊哀絕,路人為之落淚,韓康伯時為丹陽尹,母殷在郡,每聞二吳之哭,輒為悽惻,語康伯曰:「汝若為選官,當好料理此人。」康伯亦甚相知。韓後果為吏部尚書,大吳不免哀制,小吳遂大貴達。}{\footnotesize \textbf{鄭緝孝子傳}曰隱之,字處默,少有孝行,遭母喪,哀毁過禮,時與太常韓康伯鄰居,康伯母,揚州刺史殷浩之妹,聰明婦人也,隱之毎哭,康伯母輒輟事流涕,悲不自勝,終其喪如此,謂康伯曰「汝後若居銓衡,當用此輩人」,後康伯為吏部尚書,乃進用之。\textbf{晉安帝紀}曰隱之既有至性,加以廉潔,奉祿頒九族,冬月無被,桓玄欲革嶺南之弊,以為廣州刺史,去州二十里有貪泉,世傳飲之者其心無厭,隱之乃至水上,酌而飲之,因賦詩曰「石門有貪泉,一歃重千金,試使夷齊飲,終當不易心」,為盧循所攻,還京師,歷尚書、領軍將軍。\textbf{晉中興書}曰舊云,往廣州飲貪泉,失廉潔之性,吳隱之為刺史,自酌貪泉飲之,題石門為詩云云。}