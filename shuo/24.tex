\chapter{簡傲第二十四}

\subsection*{1}

\textbf{晉文王功德盛大,坐席嚴敬,擬於王者,}{\footnotesize \textbf{漢晉春秋}曰文王進爵為王,司徒何曾與朝臣皆盡禮,唯王祥長揖不拜。}\textbf{唯阮籍在坐,箕踞嘯歌,酣放自若。}

\subsection*{2}

\textbf{王戎弱冠詣阮籍,時劉公榮在坐,阮謂王曰:「偶有二斗美酒,當與君共飲,彼公榮者,無預焉。」二人交觴酬酢,公榮遂不得一桮,而言語談戲,三人無異,或有問之者,阮答曰:「勝公榮者,不得不與飲酒,不如公榮者,不可不與飲酒,唯公榮,可不與飲酒。」}{\footnotesize \textbf{晉陽秋}曰戎年十五,隨父渾在郎舍,阮籍見而說焉,每適渾俄頃,輒在戎室久之,乃謂渾「濬沖清尚,非卿倫也」,戎嘗詣籍共飲,而劉昶在坐不與焉,昶無恨色,既而戎問籍曰「彼為誰也」,曰「劉公榮也」,濬沖曰「勝公榮,故與酒,不如公榮,不可不與酒,唯公榮者,可不與酒」。\textbf{竹林七賢論}曰初,籍與戎父渾俱為尚書郎,每造渾,坐未安,輒曰「與卿語,不如與阿戎語」,就戎,必日夕而返,籍長戎二十歲,相得如時輩,劉公榮通士,性尤好酒,籍與戎酬酢終日,而公榮不蒙一桮,三人各自得也,戎為物論所先,皆此類。}

\subsection*{3}

\textbf{鍾士季精有才理,先不識嵇康,鍾要于時賢儁之士,俱往尋康,康方大樹下鍛,向子期為佐鼓排,康揚槌不輟,傍若無人,移時不交一言,鍾起去,康曰:「何所聞而來,何所見而去?」鍾曰:「聞所聞而來,見所見而去。」}{\footnotesize \textbf{文士傳}曰康性絕巧,能鍛鐵,家有盛柳樹,乃激水以圜之,夏天甚清涼,恆居其下傲戲,乃身自鍛,家雖貧,有人就鍛者,康不受直,雖親舊以雞酒往,與共飲噉,清言而已。\textbf{魏氏春秋}曰鍾會為大將軍兄弟所暱,聞康名而造焉,會名公子,以才能貴幸,乘肥衣輕,賓從如雲,康方箕踞而鍛,會至不為之禮,會深銜之,後因呂安事而遂譖康焉。}

\subsection*{4}

\textbf{嵇康與呂安善,每一相思,千里命駕,}{\footnotesize \textbf{晉陽秋}曰安,字仲悌,東平人,冀州刺史招之第二子,志量開曠,有拔俗風氣。\textbf{干寶晉紀}曰初,安之交康也,其相思則率爾命駕。}\textbf{安後來,值康不在,喜出戶延之,不入,}{\footnotesize \textbf{晉百官名}曰嵇喜,字公穆,歷揚州刺史,康兄也,阮籍遭喪,往弔之,籍能為青白眼,見凡俗之士,以白眼對之,及喜往,籍不哭,見其白眼,喜不懌而退,康聞之,乃齎酒挾琴而造之,遂相與善。\textbf{干寶晉紀}曰安嘗從康,或遇其行,康兄喜拭席而待之,弗顧,獨坐車中,康母就設酒食,求康兒共與戲,良久則去,其輕貴如此。}\textbf{題門上作鳳字而去,喜不覺,猶以為欣故作,鳳字,凡鳥也。}{\footnotesize \textbf{許慎說文}曰鳳,神鳥也,從鳥,凡聲。}

\subsection*{5}

\textbf{陸士衡初入洛,咨張公所宜詣,劉道真是其一,陸既往,劉尚在哀制中,性嗜酒,禮畢,初無他言,唯問:「東吳有長柄壺盧,卿得種來不?」陸兄弟殊失望,乃悔往。}

\subsection*{6}

\textbf{王平子出為荊州,}{\footnotesize \textbf{晉陽秋}曰惠帝時,太尉王夷甫言於選者,以弟澄為荊州刺史,從弟敦為青州刺史,澄、敦俱詣太尉辭,太尉謂曰「今王室將卑,故使弟等居齊、楚之地,外可以建霸業,內足以匡帝室,所望於二弟也」。}\textbf{王太尉及時賢送者傾路,時庭中有大樹,上有鵲巢,平子脫衣巾,徑上樹取鵲子,涼衣拘閡樹枝,便復脫去,得鵲子還,下弄,神色自若,傍若無人。}{\footnotesize \textbf{鄧粲晉紀}曰澄放蕩不拘,時謂之達。}

\subsection*{7}

\textbf{高坐道人於丞相坐,恆偃臥其側,見卞令,肅然改容云:「彼是禮法人。」}{\footnotesize \textbf{高坐傳}曰王公曾詣和上,和上解帶偃伏,悟言神解,見尚書令卞望之,便斂衿飾容,時歎皆得其所。}

\subsection*{8}

\textbf{桓宣武作徐州,時謝奕為晉陵,}{\footnotesize \textbf{中興書}曰奕自吏部郎出為晉陵太守。}\textbf{先粗經虛懷,而乃無異常,及桓還荊州,將西之間,意氣甚篤,奕弗之疑,唯謝虎子婦王悟其旨,}{\footnotesize 虎子,謝據小字,奕弟也。其妻王氏,已見。}\textbf{每曰:「桓荊州用意殊異,必與晉陵俱西矣。」俄而引奕為司馬,奕既上,猶推布衣交,在溫坐,岸幘嘯詠,無異常日,宣武每曰:「我方外司馬。」遂因酒,轉無朝夕禮,桓舍入內,奕輒復隨去,後至奕醉,溫往主許避之,主曰:「君無狂司馬,我何由得相見?」}

\subsection*{9}

\textbf{謝萬在兄前,欲起索便器,于時阮思曠在坐,曰:「新出門戶,篤而無禮。」}

\subsection*{10}

\textbf{謝中郎是王藍田女壻,}{\footnotesize \textbf{謝氏譜}曰萬取太原王述女,名荃。}\textbf{嘗著白綸巾,肩輿徑至揚州聽事見王,直言曰:「人言君侯癡,君侯信自癡。」藍田曰:「非無此論,但晚令耳。」}{\footnotesize \textbf{述別傳}曰述少真獨退靜,人未嘗知,故有晚令之言。}

\subsection*{11}

\textbf{王子猷作桓車騎騎兵參軍,桓問曰:「卿何署?」答曰:「不知何署,時見牽馬來,似是馬曹。」}{\footnotesize \textbf{中興書}曰桓沖引徽之為參軍,蓬首散帶,不綜知其府事。}\textbf{桓又問:「官有幾馬?」答曰:「不問馬,何由知其數?」}{\footnotesize \textbf{論語}曰廐焚,孔子退朝曰「傷人乎」,不問馬。\textbf{注}貴人賤畜,故不問也。}\textbf{又問:「馬比死多少?」答曰:「未知生,焉知死?」}{\footnotesize \textbf{論語}曰子路問死,孔子曰「未知生,焉知死」。\textbf{馬融}注曰死事難明,語之無益,故不答。}

\subsection*{12}

\textbf{謝公嘗與謝萬共出西,過吳郡,阿萬欲相與共萃王恬許,}{\footnotesize 恬已見。時為吳郡太守。}\textbf{太傅云:「恐伊不必酬汝意,不足爾。」萬猶苦要,太傅堅不回,萬乃獨往,坐少時,王便入門內,謝殊有欣色,以為厚待己,良久,乃沐頭散髮而出,亦不坐,仍據胡牀,在中庭曬頭,神氣傲邁,了無相酬對意,謝於是乃還,未至船,逆呼太傅,安曰:「阿螭不作爾。」}{\footnotesize 王恬,小字螭虎。}

\subsection*{13}

\textbf{王子猷作桓車騎參軍,桓謂王曰:「卿在府久,比當相料理。」初不答,直高視,以手版拄頰云:「西山朝來,致有爽氣。」}

\subsection*{14}

\textbf{謝萬北征,常以嘯詠自高,未嘗撫慰眾士,謝公甚器愛萬,而審其必敗,乃俱行,從容謂萬曰:「汝為元帥,宜數喚諸將宴會,以說眾心。」萬從之,因召集諸將,都無所說,直以如意指四坐云:「諸君皆是勁卒。」諸將甚忿恨之,謝公欲深著恩信,自隊主將帥以下,無不身造,厚相遜謝,及萬事敗,軍中因欲除之,復云:「當為隱士。」故幸而得免。}{\footnotesize 萬敗事已見上。}

\subsection*{15}

\textbf{王子敬兄弟見郗公,躡履問訊,甚修外生禮,及嘉賓死,皆著高屐,儀容輕慢,命坐,皆云「有事,不暇坐」,既去,郗公慨然曰:「使嘉賓不死,鼠輩敢爾。」}{\footnotesize 愔子超,有盛名,且獲寵於桓溫,故為超敬愔。}

\subsection*{16}

\textbf{王子猷嘗行過吳中,見一士大夫家,極有好竹,主已知子猷當往,乃灑埽施設,在聽事坐相待,王肩輿徑造竹下,諷嘯良久,主已失望,猶冀還當通,遂直欲出門,主人大不堪,便令左右閉門不聽出,王更以此賞主人,乃留坐,盡歡而去。}

\subsection*{17}

\textbf{王子敬自會稽經吳,聞顧辟疆}{\footnotesize \textbf{顧氏譜}曰辟疆,吳郡人,歷郡功曹、平北參軍。}\textbf{有名園,先不識主人,徑往其家,值顧方集賓友酣燕,而王遊歷既畢,指麾好惡,傍若無人,顧勃然不堪曰:「傲主人,非禮也,以貴驕人,非道也,失此二者,不足齒人,傖耳。」便驅其左右出門,王獨在輿上回轉顧望,左右移時不至,然後令送著門外,怡然不屑。}