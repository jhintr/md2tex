\chapter{彼岸道品第五}

\section{序颂}


从㤭萨罗怡人的城市来到了南方之地,\hfill\textcolor{gray}{\footnotesize \textbf{983}} \\
婆罗门通晓颂诗,愿求着无所有。


他在阿萨迦的境内、阿罗迦的接壤处\hfill\textcolor{gray}{\footnotesize \textbf{984}} \\
居住,在乔陀婆利岸边,靠着采集与果实。


就在这附近,有一个大村庄,\hfill\textcolor{gray}{\footnotesize \textbf{985}} \\
随后,他以所得的收入举行了大祭祀。


举行完大祭祀,便又进入了草庵,\hfill\textcolor{gray}{\footnotesize \textbf{986}} \\
当他回去时,另一个婆罗门来到。


脚已磨破,口干齿污,头面蒙尘,\hfill\textcolor{gray}{\footnotesize \textbf{987}} \\
且他走向他后,乞求五百钱。


波婆利看到他后,请他坐下,\hfill\textcolor{gray}{\footnotesize \textbf{988}} \\
问罢安乐与健康,说了这样的话:


「我所能布施之物,我已全部派发,\hfill\textcolor{gray}{\footnotesize \textbf{989}} \\
「许可我!婆罗门!我没有五百钱。」


「如果对我的乞求,您不作施舍,\hfill\textcolor{gray}{\footnotesize \textbf{990}} \\
「在第七天,让你的头裂成七分!」

诡诈者造作后,他扬言恐吓,\hfill\textcolor{gray}{\footnotesize \textbf{991}} \\
听到他的这话,波婆利生起苦恼。


他焦灼、不食,被忧箭射中,\hfill\textcolor{gray}{\footnotesize \textbf{992}} \\
然后,对这样的心,意便不喜于禅那。


看到惧怕、苦恼,天人怀着善意,\hfill\textcolor{gray}{\footnotesize \textbf{993}} \\
走向波婆利,说了这样的话:


「他不了知头,他是希求财产的诡诈者,\hfill\textcolor{gray}{\footnotesize \textbf{994}} \\
「对于头或头落,他并没有智。」

「那么您知道,请告诉我!当被问及\hfill\textcolor{gray}{\footnotesize \textbf{995}} \\
「头与头裂,让我们听你的话!」


「我对此也不知,我于此没有智,\hfill\textcolor{gray}{\footnotesize \textbf{996}} \\
「对于头与头裂,于此唯是胜者们的知见。」

「那么在这地轮上,谁会知道\hfill\textcolor{gray}{\footnotesize \textbf{997}} \\
「头与头裂?请告诉我!天人!」

「先前从迦毗罗卫出离,世间的导师,\hfill\textcolor{gray}{\footnotesize \textbf{998}} \\
「甘蔗王的后裔,释迦之子,放光者。


「婆罗门!他实为等觉者,通晓一切法,\hfill\textcolor{gray}{\footnotesize \textbf{999}} \\
「得证一切神通力,于一切法具眼,\\
「得证一切业的灭尽,在依持的灭尽处解脱。


「他是佛陀、世尊、具眼者,在世间开示法,\hfill\textcolor{gray}{\footnotesize \textbf{1000}} \\
「你去后若是问他,他会为你解答。」

听到「等觉者」之语,波婆利起了踊跃,\hfill\textcolor{gray}{\footnotesize \textbf{1001}} \\
他的忧伤变薄了,并得了广大的喜。

这波婆利心满意足、踊跃、充满喜悦,问那天人:\hfill\textcolor{gray}{\footnotesize \textbf{1002}} \\
「世间之依怙在何村何镇,或在何国土?\\
「去到那里后,我们好看看等觉者、两足尊。」

「广慧、最上宏慧的胜者在舍卫国、㤭萨罗的宫城,\hfill\textcolor{gray}{\footnotesize \textbf{1003}} \\
「这释迦之子离于重担、无漏,人中公牛知道头裂。」


随后,他便招呼学生,通晓颂诗的众婆罗门:\hfill\textcolor{gray}{\footnotesize \textbf{1004}} \\
「来!学童们!我将宣告,请听我的话!


「这在世间总是难得出现、\hfill\textcolor{gray}{\footnotesize \textbf{1005}} \\
「以等觉者著名者,他现在已在世间出现,\\
「速速去到舍卫国,看看两足尊!」

「看到后,我们怎么就知道是『佛陀』?婆罗门!\hfill\textcolor{gray}{\footnotesize \textbf{1006}} \\
「请告诉无知的我们,好让我们认识他!」

「因为在颂诗中流传有大人相,\hfill\textcolor{gray}{\footnotesize \textbf{1007}} \\
「且三十二种被完整、依次地解释。


「若他身上存在这些大人相,\hfill\textcolor{gray}{\footnotesize \textbf{1008}} \\
「他唯有两种趣向,实无第三种。

「如果他居家,则能征服这大地,\hfill\textcolor{gray}{\footnotesize \textbf{1009}} \\
「不以杖,不以刀,而以法教训。

「而如果他出家,从家至于非家,\hfill\textcolor{gray}{\footnotesize \textbf{1010}} \\
「便是去蔽的等觉者,无上的阿罗汉。

「出身、族姓与相,颂诗、学生,还有\hfill\textcolor{gray}{\footnotesize \textbf{1011}} \\
「头与头裂,你们唯当以意去问。


「如果他是佛陀、所见无碍者,\hfill\textcolor{gray}{\footnotesize \textbf{1012}} \\
「他会用语言回答以意所提的问题。」

听到波婆利的话后,十六个婆罗门学生:\hfill\textcolor{gray}{\footnotesize \textbf{1013}} \\
阿耆多、低舍弥勒、富楼那,还有慈达,


净洗、优波湿婆与难陀,还有黄金,\hfill\textcolor{gray}{\footnotesize \textbf{1014}} \\
可教、劫波二人,与智者胶耳,

善器与生起,和布萨罗婆罗门,\hfill\textcolor{gray}{\footnotesize \textbf{1015}} \\
有智的空王,与大仙褐者,

全都有各自的徒众,闻名于一切世间,\hfill\textcolor{gray}{\footnotesize \textbf{1016}} \\
是禅者,乐于禅那,智者,受过去习气的熏习,


礼拜了波婆利,并且右绕了他,\hfill\textcolor{gray}{\footnotesize \textbf{1017}} \\
全都萦发、持羚羊皮,朝北方出发。

(途经)阿罗迦的波提吒那,然后是大箭城、\hfill\textcolor{gray}{\footnotesize \textbf{1018}} \\
优禅尼以及乔那陀、吠提舍、名为婆那萨者,


㤭赏弥,沙计多,与最上之城舍卫国,\hfill\textcolor{gray}{\footnotesize \textbf{1019}} \\
私多毗,迦毗罗卫,与拘尸那罗宫城,


财富城波婆,毗舍离,摩竭陀城,\hfill\textcolor{gray}{\footnotesize \textbf{1020}} \\
与怡人、悦意的石支提。


如口干者之于寒泉,如商人之于大利,\hfill\textcolor{gray}{\footnotesize \textbf{1021}} \\
如为暑热炙灼者之于凉荫,他们匆匆登山。


此时,世尊置身比丘僧团之前,\hfill\textcolor{gray}{\footnotesize \textbf{1022}} \\
对诸比丘开示法,如林中狮吼。

阿耆多见到了佛陀,如百道光芒的太阳,\hfill\textcolor{gray}{\footnotesize \textbf{1023}} \\
好比十五之夜趋近圆满的月亮。

然后,看到他的身体和圆满的特相,\hfill\textcolor{gray}{\footnotesize \textbf{1024}} \\
站在一边,振奋,在意中问了问题:


「关于出身,请说!请说族姓及其相!\hfill\textcolor{gray}{\footnotesize \textbf{1025}} \\
「请说颂诗的成就!婆罗门教授几多?」


「年寿一百二十,他族姓波婆利,\hfill\textcolor{gray}{\footnotesize \textbf{1026}} \\
「他身上有三相,通晓三种吠陀,

「于相、传承及其词汇、仪轨,\hfill\textcolor{gray}{\footnotesize \textbf{1027}} \\
「教授五百,于自法已达成就。」


「波婆利之相的简择,最上之人!\hfill\textcolor{gray}{\footnotesize \textbf{1028}} \\
「断除疑惑者!请阐明!莫使我们有疑惑!」


「以舌覆面,眉间有毫,\hfill\textcolor{gray}{\footnotesize \textbf{1029}} \\
「阴马藏,当知如是!学童!」

尚未听到任何所问,却听到问题被解答,\hfill\textcolor{gray}{\footnotesize \textbf{1030}} \\
所有人都充满喜悦、合掌,思忖道:

「到底是谁,是天、梵、还是善生之主因陀\hfill\textcolor{gray}{\footnotesize \textbf{1031}} \\
「以意问了这些问题?他在应答谁?」


「波婆利遍问头与头裂,\hfill\textcolor{gray}{\footnotesize \textbf{1032}} \\
「请解答!世尊!调伏我们的疑惑!仙人!」


「当知无明是头,而明是头裂,\hfill\textcolor{gray}{\footnotesize \textbf{1033}} \\
「伴以信、念、定、欲与精进。」


随后,学童为大喜悦裹挟,\hfill\textcolor{gray}{\footnotesize \textbf{1034}} \\
把羚羊皮偏覆一肩,以头顶礼双足:


「波婆利婆罗门,与其学生一起,先生!\hfill\textcolor{gray}{\footnotesize \textbf{1035}} \\
「心怀踊跃而欢喜,礼拜双足,具眼者!」


「愿波婆利婆罗门与诸学生幸福!\hfill\textcolor{gray}{\footnotesize \textbf{1036}} \\
「也愿你幸福!愿你长命!学童!


「波婆利的与你的,或所有人的一切疑虑,\hfill\textcolor{gray}{\footnotesize \textbf{1037}} \\
「都有机会,请问任何心中所希望的!」


得到等觉者的许可,坐下后合了掌,\hfill\textcolor{gray}{\footnotesize \textbf{1038}} \\
阿耆多于此便问了如来第一个问题。


\section{阿耆多学童问}

「世间被什么覆蔽?」尊者阿耆多说,「为何它不显露?\hfill\textcolor{gray}{\footnotesize \textbf{1039}} \\
「你说什么是它的胶合?什么是它的大怖畏?」


「世间被无明覆蔽,阿耆多!」世尊说,「由悭贪、放逸而不显露,\hfill\textcolor{gray}{\footnotesize \textbf{1040}} \\
「我说渴望是胶合,苦是它的大怖畏。」


「众流四处流淌,」尊者阿耆多说,「什么是众流的遮止?\hfill\textcolor{gray}{\footnotesize \textbf{1041}} \\
「请说众流的防护!众流应以什么来阻碍?」


「世间的这些众流,阿耆多!」世尊说,「念是它们的遮止,\hfill\textcolor{gray}{\footnotesize \textbf{1042}} \\
「我说众流的防护,它们应以慧来阻碍。」


「即此慧与念,」尊者阿耆多说,「以及名色,先生!\hfill\textcolor{gray}{\footnotesize \textbf{1043}} \\
「既然问到,请告诉我!它在何处灭去?」


「你所问的这问题,阿耆多!我对你说,\hfill\textcolor{gray}{\footnotesize \textbf{1044}} \\
「于此,名与色无余地灭去:\\
「因识的灭,它即在此灭去。」


「那些已察知法者,与此处的种种有学,\hfill\textcolor{gray}{\footnotesize \textbf{1045}} \\
「既然问到,请贤者告诉我他们的威仪!先生!」


「他不应贪求于爱欲,他不应污浊其意,\hfill\textcolor{gray}{\footnotesize \textbf{1046}} \\
「善巧于一切法,具念的比丘便能游行。」


\section{低舍弥勒学童问}


「谁在此对世间知足?」尊者低舍弥勒说,「谁没有动摇?\hfill\textcolor{gray}{\footnotesize \textbf{1047}} \\
「谁证知了两端,以智慧不染于中间?\\
「你说谁为『大人』?谁在此超越了缝合?」


「于爱欲中具梵行,弥勒!」世尊说,「离爱,始终具念,\hfill\textcolor{gray}{\footnotesize \textbf{1048}} \\
「经省思而寂灭的比丘,他没有动摇。


「他证知了两端,以智慧不染于中间,\hfill\textcolor{gray}{\footnotesize \textbf{1049}} \\
「我说他为『大人』,他在此超越了缝合。」


\section{富楼那学童问}

「不动者、得见根本者,」尊者富楼那说,「我带着问题前来,\hfill\textcolor{gray}{\footnotesize \textbf{1050}} \\
「依据什么,仙人、人类、刹帝利、婆罗门向诸天\\
「在此世间举行种种祭祀?我问你,世尊!请对我说说这个!」


「富楼那!凡仙人、人类、」世尊说,「刹帝利、婆罗门向诸天\hfill\textcolor{gray}{\footnotesize \textbf{1051}} \\
「在此世间举行种种祭祀,\\
「富楼那!他们希求着这样的状态、束缚于老而举行祭祀。」


「凡仙人、人类、」尊者富楼那说,「刹帝利、婆罗门向诸天\hfill\textcolor{gray}{\footnotesize \textbf{1052}} \\
「在此世间举行种种祭祀,世尊!他们不放逸于祭祀之路,是否能\\
「度脱生与老?先生!我问你,世尊!请对我说说这个!」


「他们希求、赞美、渴望、供奉,富楼那!」世尊说,「出于利养而渴望爱欲,\hfill\textcolor{gray}{\footnotesize \textbf{1053}} \\
「他们从事祭祀,染著有贪,我说不能度脱生老。」


「如果他们从事祭祀,」尊者富楼那说,「不能以祭祀度脱生与老,先生!\hfill\textcolor{gray}{\footnotesize \textbf{1054}} \\
「那么,谁能在天人的世间度脱生与老?先生!\\
「我问你,世尊!请对我说说这个!」

「省思了世间种种,富楼那!」世尊说,「他在世间没有任何动摇,\hfill\textcolor{gray}{\footnotesize \textbf{1055}} \\
「寂静、无烟、无患、无待,我说他度脱了生老。」


\section{慈达学童问}

「我问你,世尊!请对我说说这个!」尊者慈达说,「我认为你通达诸明、修己,\hfill\textcolor{gray}{\footnotesize \textbf{1056}} \\
「世间这些种种形相的苦,它们都从哪里产生?」

「你问我苦的根源,慈达!」世尊说,「我将对你说,如同了知者,\hfill\textcolor{gray}{\footnotesize \textbf{1057}} \\
「世间这些种种形相的苦,由依持为因而产生。


「若愚钝的无知者造作依持,则再再地经历苦,\hfill\textcolor{gray}{\footnotesize \textbf{1058}} \\
「所以,知晓者、随观苦的生与源者不应造作依持。」


「我们所问的,你已向我们宣说,我们另有所问,请你快说!\hfill\textcolor{gray}{\footnotesize \textbf{1059}} \\
「智者们如何度过暴流,及生、老、忧悲?\\
「牟尼!请对我善加解释!因为这法已如是为你所知。」


「我将对你宣说法,慈达!」世尊说,「所见之法,而非传闻,\hfill\textcolor{gray}{\footnotesize \textbf{1060}} \\
「了知此已,具念而行,便能度过世间的爱著。」


「而我欢喜这无上之法,大仙!\hfill\textcolor{gray}{\footnotesize \textbf{1061}} \\
「了知此已,具念而行,便能度过世间的爱著。」


「凡是你所知的,慈达!」世尊说,「上方、下方、四旁及中间,\hfill\textcolor{gray}{\footnotesize \textbf{1062}} \\
「除去了其中的欢喜与住著,识便不住于有。


「如是而住、具念、不放逸的比丘,舍弃了执为我者,\hfill\textcolor{gray}{\footnotesize \textbf{1063}} \\
「知者便能于此舍弃生、老、忧悲之苦。」


「我欢喜大仙的这番话语,乔达摩!无依持是善说,\hfill\textcolor{gray}{\footnotesize \textbf{1064}} \\
「因为世尊确实已舍弃了苦,因为这法已如是为你所知。


「而且,你不停教诫之人,牟尼!他们也能舍弃苦,\hfill\textcolor{gray}{\footnotesize \textbf{1065}} \\
「遇到了你,我将礼敬,龙象!愿世尊也能不停教诫我!」


「你所证知的通达诸明、无所牵绊、不取著爱欲与有的婆罗门,\hfill\textcolor{gray}{\footnotesize \textbf{1066}} \\
「他确实已度过了这暴流,且已度彼岸、无荒秽、无疑惑。


「而且于此,这知者、通达诸明之人舍遣了对有与无有的染著,\hfill\textcolor{gray}{\footnotesize \textbf{1067}} \\
「他离爱、无患、无待,我说他已度脱了生老。」


\section{净洗学童问}

「我问你,世尊!请对我说说这个!」尊者净洗说,「我期待你的话语,大仙!\hfill\textcolor{gray}{\footnotesize \textbf{1068}} \\
「听了你的教训,我将为自己修学涅槃。」


「那么,请提起热忱!净洗!」世尊说,「于此贤明、具念,\hfill\textcolor{gray}{\footnotesize \textbf{1069}} \\
「从此听取声明,你将为自己修学涅槃。」


「在天与人的世间,我看见无所牵绊而行止的婆罗门,\hfill\textcolor{gray}{\footnotesize \textbf{1070}} \\
「我礼敬您,一切眼者!请脱我出诸疑惑!释迦!」


「我不能让世间任何有疑惑者解脱,净洗!\hfill\textcolor{gray}{\footnotesize \textbf{1071}} \\
「然而,证知殊胜的法,如是,你便能度过这暴流。」


「带着悲悯,梵天!请教训我所能了知的远离之法!\hfill\textcolor{gray}{\footnotesize \textbf{1072}} \\
「犹如虚空不被妨碍,我将于此寂静、无所依而行。」


「我将对你宣说寂静,净洗!」世尊说,「所见之法,而非传闻,\hfill\textcolor{gray}{\footnotesize \textbf{1073}} \\
「了知此已,具念而行,便能度过世间的爱著。」


「而我欢喜这无上寂静,大仙!\hfill\textcolor{gray}{\footnotesize \textbf{1074}} \\
「了知此已,具念而行,便能度过世间的爱著。」

「凡是你所知的,净洗!」世尊说,「上方、下方、四旁及中间,\hfill\textcolor{gray}{\footnotesize \textbf{1075}} \\
「了知了这在世间是染著,切莫对有与无有起渴爱!」


\section{优波湿婆学童问}

「我独自、无依,释迦!」尊者优波湿婆说,「不能度过洪大的暴流,\hfill\textcolor{gray}{\footnotesize \textbf{1076}} \\
「请说说所缘!一切眼者!依于此,我好度过这暴流。」


「觉察着无所有,具念,优波湿婆!」世尊说,「依于『这不存在』,你能度过暴流,\hfill\textcolor{gray}{\footnotesize \textbf{1077}} \\
「舍弃了爱欲,戒离疑惑,昼夜寻求渴爱之灭尽!」


「若于一切爱欲离贪,」尊者优波湿婆说,「依于无所有,舍弃了其它,\hfill\textcolor{gray}{\footnotesize \textbf{1078}} \\
「解脱于最高的想之解脱,他能否住立于此,不再随行?」


「若于一切爱欲离贪,优波湿婆!」世尊说,「依于无所有,舍弃了其它,\hfill\textcolor{gray}{\footnotesize \textbf{1079}} \\
「解脱于最高的想之解脱,他能住立于此,不再随行。」


「如果他能住立于此若许年,不再随行,一切眼者!\hfill\textcolor{gray}{\footnotesize \textbf{1080}} \\
「他即于此处得清凉而解脱,对这样的人,识是否消灭?」


「好比火焰为疾风所扰乱,优波湿婆!」世尊说,「消逝而不可得名,\hfill\textcolor{gray}{\footnotesize \textbf{1081}} \\
「如是,牟尼解脱于名身,消逝而不可得名。」


「这消逝,是说他不存在,还是说永远无病?\hfill\textcolor{gray}{\footnotesize \textbf{1082}} \\
「牟尼!请对我善加解释!因为这法已如是为你所知。」


「消逝者无法度量,优波湿婆!」世尊说,「对他无法有所言说,\hfill\textcolor{gray}{\footnotesize \textbf{1083}} \\
「当一切法都已铲除,一切言路也被铲除。」


\section{难陀学童问}

「人们说『世间有牟尼』,」尊者难陀说,「这是如何?\hfill\textcolor{gray}{\footnotesize \textbf{1084}} \\
「他们是说具足智者是牟尼,还是说具足活命者?」


「善巧者不以见、不以闻、不以智而说是此处的牟尼,难陀!\hfill\textcolor{gray}{\footnotesize \textbf{1085}} \\
「消灭了敌军,无患、无待而行者,我说他们是牟尼。」


「凡是这些沙门、婆罗门,」尊者难陀说,「以见闻而说清净,\hfill\textcolor{gray}{\footnotesize \textbf{1086}} \\
「以戒禁而说清净,以多种方式而说清净,\\
「世尊!他们于此自制而行,是否得度生与老?先生!\\
「我问你,世尊!请对我说说这个!」


「凡是这些沙门、婆罗门,难陀!」世尊说,「以见闻而说清净,\hfill\textcolor{gray}{\footnotesize \textbf{1087}} \\
「以戒禁而说清净,以多种方式而说清净,\\
「即便他们于此自制而行,我说无法得度生老。」


「凡是这些沙门、婆罗门,」尊者难陀说,「以见闻而说清净,\hfill\textcolor{gray}{\footnotesize \textbf{1088}} \\
「以戒禁而说清净,以多种方式而说清净,\\
「牟尼!如果你说无法度过暴流,那么在天与人的世间,谁能\\
「得度生与老?先生!我问你,世尊!请对我说说这个!」


「我不说所有的沙门、婆罗门,难陀!」世尊说,「为生老覆蔽,\hfill\textcolor{gray}{\footnotesize \textbf{1089}} \\
「那些于此舍弃了一切所见、所闻、所觉或戒禁,\\
「舍弃了种种一切方式,遍知了渴爱的无漏者们,\\
「我说这些人已度过暴流。」


「我欢喜大仙的这番话语,乔达摩!无依持是善说,\hfill\textcolor{gray}{\footnotesize \textbf{1090}} \\
「那些于此舍弃了一切所见、所闻、所觉或戒禁,\\
「舍弃了种种一切方式,遍知了渴爱的无漏者们,\\
「我也说他们已度过暴流。」


\section{黄金学童问}

「在乔达摩的教法之前,」尊者黄金说,「他们先前对我所说的,\hfill\textcolor{gray}{\footnotesize \textbf{1091}} \\
「『这曾是如此、这将是如此』,这一切都是传闻,\\
「这一切都是寻的增长,我对此不喜乐。


「请你对我宣说根除渴爱之法!牟尼!\hfill\textcolor{gray}{\footnotesize \textbf{1092}} \\
「了知此已,具念而行,便能度过世间的爱著。」

「于此,对所见、所闻、所觉、所知的可喜之色,黄金!\hfill\textcolor{gray}{\footnotesize \textbf{1093}} \\
「驱除欲贪,即是不殁的涅槃境地。


「知晓此已,具念的现法寂灭者\hfill\textcolor{gray}{\footnotesize \textbf{1094}} \\
「便始终寂静,得度世间的爱著。」


\section{可教学童问}

「若爱欲不居于他,」尊者可教说,「他也没有渴爱,\hfill\textcolor{gray}{\footnotesize \textbf{1095}} \\
「且已度脱疑惑,那他的解脱是怎样的?」


「若爱欲不居于他,可教!」世尊说,「他也没有渴爱,\hfill\textcolor{gray}{\footnotesize \textbf{1096}} \\
「且已度脱疑惑,那他已无更多的解脱。」


「他是离欲还是仍在希求?他具有智慧,还是作智慧想?\hfill\textcolor{gray}{\footnotesize \textbf{1097}} \\
「释迦!请对我说明他!好让我能了知牟尼,一切眼者!」


「他离欲,不再希求,他具有智慧,不作智慧想,\hfill\textcolor{gray}{\footnotesize \textbf{1098}} \\
「如是,可教!应知牟尼无所牵绊、不取著爱欲与有!」


\section{劫波学童问}

「当大怖畏的暴流生起,」尊者劫波说,「对佇立于中流者,\hfill\textcolor{gray}{\footnotesize \textbf{1099}} \\
「对被老死征服者,请告知洲渚!先生!\\
「请您对我宣说洲渚!好让这没有更多。」


「当大怖畏的暴流生起,劫波!」世尊说,「对佇立于中流者,\hfill\textcolor{gray}{\footnotesize \textbf{1100}} \\
「对被老死征服者,我告知你洲渚,劫波!\\


「无牵绊,无执取,这即是没有更多的洲渚,\hfill\textcolor{gray}{\footnotesize \textbf{1101}} \\
「我说这即是涅槃,老死的灭尽。


「知晓此已,那些具念的现法寂灭者,\hfill\textcolor{gray}{\footnotesize \textbf{1102}} \\
「他们不受制于魔罗,他们不侍奉于魔罗。」


\section{胶耳学童问}

「听说英雄不欲求爱欲,」尊者胶耳说,「为问度过暴流,我来至无爱欲者,\hfill\textcolor{gray}{\footnotesize \textbf{1103}} \\
「请说寂静的境地!俱生眼者!请如实地对我说!世尊!


「因为世尊征服了爱欲而行止,如同光辉的太阳以光芒之于大地,\hfill\textcolor{gray}{\footnotesize \textbf{1104}} \\
「宏慧者!请对小慧的我说法!我好了知,\\
「于此舍弃生老。」


「调伏对爱欲的贪求!胶耳!」世尊说,「视出离为安稳,\hfill\textcolor{gray}{\footnotesize \textbf{1105}} \\
「你莫存有任何的执取或是扬弃。


「让先前的凋萎,你切莫有任何后来,\hfill\textcolor{gray}{\footnotesize \textbf{1106}} \\
「如果你不执取中间,你将寂静而行。


「于一切名色离贪者,婆罗门!\hfill\textcolor{gray}{\footnotesize \textbf{1107}} \\
「则无有能受制于死亡的诸漏。」


\section{善器学童问}

「舍弃家、断爱、不动、」尊者善器说,「舍弃欢喜、已度暴流、解脱、\hfill\textcolor{gray}{\footnotesize \textbf{1108}} \\
「舍弃想的善慧者,我恳求!从龙象处听闻后,人们将从此处离去。


「种种人们已从诸多国土聚集,英雄!期待着您的言语,\hfill\textcolor{gray}{\footnotesize \textbf{1109}} \\
「请您对他们善加解释!因为这法已如是为你所知。」


「应调伏一切对取著的渴爱!善器!」世尊说,「上方、下方、四旁及中间,\hfill\textcolor{gray}{\footnotesize \textbf{1110}} \\
「因为任何他们在世间所执取者,魔罗便以此追随造物。


「所以,知晓着,具念的比丘不应执取一切世间中的任何,\hfill\textcolor{gray}{\footnotesize \textbf{1111}} \\
「觉察着『执取的有情,这人类纠缠于死亡的境域』。」\\


\section{生起学童问}

「离尘、安坐的禅修者,」尊者生起说,「应作已作的无漏者,\hfill\textcolor{gray}{\footnotesize \textbf{1112}} \\
「已度一切法者,我带着问题前来,\\
「请告知破碎无明的了知解脱!」


「舍弃欲贪,生起!」世尊说,「与忧虑两者,\hfill\textcolor{gray}{\footnotesize \textbf{1113}} \\
「除去昏沉,防止恶作,


「舍念清净,以法寻为前行,\hfill\textcolor{gray}{\footnotesize \textbf{1114}} \\
「我说即是破碎无明的了知解脱。」


「什么是世间的结缚?什么是它的运作?\hfill\textcolor{gray}{\footnotesize \textbf{1115}} \\
「因舍弃什么,而被称为涅槃?」


「欢喜是世间的结缚,寻是它的运作,\hfill\textcolor{gray}{\footnotesize \textbf{1116}} \\
「因舍弃渴爱,而被称为涅槃。」


「对具念而行者,识如何灭去?\hfill\textcolor{gray}{\footnotesize \textbf{1117}} \\
「我们前来问世尊,愿听你的话语!」


「对于不欢喜内与外之受、\hfill\textcolor{gray}{\footnotesize \textbf{1118}} \\
「如是具念而行者,识即灭去。」


\section{布萨罗学童问}

「能宣示过去者,」尊者布萨罗说,「不动者,切断疑虑者,\hfill\textcolor{gray}{\footnotesize \textbf{1119}} \\
「已度一切法者,我带着问题前来。


「无有色想者,舍弃一切身者,\hfill\textcolor{gray}{\footnotesize \textbf{1120}} \\
「于内外见『不存在任何』者,\\
「释迦!我问其智,这样的人应如何被引领?」


「如来证知着,布萨罗!」世尊说,「一切识住,\hfill\textcolor{gray}{\footnotesize \textbf{1121}} \\
「知晓他住立、解脱或是志在于彼。


「了知了无所有的生起,即『欢喜是结缚』,\hfill\textcolor{gray}{\footnotesize \textbf{1122}} \\
「如是证知此已,随后他对此修观,\\
「这便是那已立婆罗门的如实之智。」


\section{空王学童问}

「我两次问了释迦,」尊者空王说,「具眼者没有对我解答,\hfill\textcolor{gray}{\footnotesize \textbf{1123}} \\
「而我听说:『要到第三次,天仙便会解答。』


「此世间、他世间、俱有天的梵世间,\hfill\textcolor{gray}{\footnotesize \textbf{1124}} \\
「不知道您,享有名望的乔达摩的见。


「如是具殊胜之见者,我带着问题前来,\hfill\textcolor{gray}{\footnotesize \textbf{1125}} \\
「如何观察世间,死王便不得见他?」


「你应从空观察世间!空王!始终具念,\hfill\textcolor{gray}{\footnotesize \textbf{1126}} \\
「除去了我随见,如是便越过死亡,\\
「如是观察世间者,死王不得见他。」


\section{褐者学童问}

「我已年迈、无力、失去光泽,」尊者褐者说,「眼不明净,听不安顺,\hfill\textcolor{gray}{\footnotesize \textbf{1127}} \\
「莫让我在愚痴之间消逝!\\
「于此,请说我能了知的舍弃生老之法!」


「看到在色中遘难,褐者!」世尊说,「放逸的人们被色恼害,\hfill\textcolor{gray}{\footnotesize \textbf{1128}} \\
「所以,褐者!你应不放逸,为了不再有,应舍弃色!」


「四方、四维、上方、下方这十方,\hfill\textcolor{gray}{\footnotesize \textbf{1129}} \\
「世间没有任何是你所未见、未闻、未觉、未知的,\\
「于此,请说我能了知的舍弃生老之法!」


「觉察着陷入渴爱的人们,褐者!」世尊说,「种种热恼,被老击溃,\hfill\textcolor{gray}{\footnotesize \textbf{1130}} \\
「所以,褐者!你应不放逸,为了不再有,应舍弃渴爱!」


\section{彼岸道赞颂}

当住于摩竭陀的石支提时,世尊说了这些,对随侍的十六个婆罗门请求的问题一一作了解答。如果对每个问题知晓其义、知晓其法,能修行法与随法,便能去到老死的彼岸,以这些法能趣向彼岸,所以这法门即名「彼岸道」。


阿耆多、低舍弥勒、富楼那,还有慈达,\hfill\textcolor{gray}{\footnotesize \textbf{1131}} \\
净洗、优波湿婆与难陀,还有黄金,

可教、劫波二人,与智者胶耳,\hfill\textcolor{gray}{\footnotesize \textbf{1132}} \\
善器与生起,和布萨罗婆罗门,\\
有智的空王,与大仙褐者,

他们去到佛陀、具足行的仙人处,\hfill\textcolor{gray}{\footnotesize \textbf{1133}} \\
去到最胜的佛陀处,问着微妙的问题。


佛陀如实地解答了他们提出的问题,\hfill\textcolor{gray}{\footnotesize \textbf{1134}} \\
以对问题的解释,牟尼令众婆罗门满足。

他们满足于具眼者、佛陀、日种,\hfill\textcolor{gray}{\footnotesize \textbf{1135}} \\
便在胜慧者的跟前行梵行。


对每个问题,能如佛陀所开示的\hfill\textcolor{gray}{\footnotesize \textbf{1136}} \\
那样去行道,他便能从此岸去到彼岸。

修习着最上之道,他便能从此岸去到彼岸,\hfill\textcolor{gray}{\footnotesize \textbf{1137}} \\
这道趣向彼岸,所以名为「彼岸道」。

\section{诵彼岸道颂}

「我将诵出彼岸道,」尊者褐者说,\hfill\textcolor{gray}{\footnotesize \textbf{1138}} \\
「无垢的宏慧者如是见便如是宣说,\\
「离欲、消尽的龙象,为何会妄语?


「舍弃了尘垢与愚痴者、舍弃了慢与覆藏者的\hfill\textcolor{gray}{\footnotesize \textbf{1139}} \\
「具有德泽的话语,噫!我将宣扬!


「除去暗冥的佛陀具一切眼,已到世间的边际,超越一切有,\hfill\textcolor{gray}{\footnotesize \textbf{1140}} \\
「无漏,舍弃了一切苦,以真实得名者,梵天!为我所侍奉。


「好比鸟儿舍弃了灌木丛,居于果实丰硕的森林,\hfill\textcolor{gray}{\footnotesize \textbf{1141}} \\
「如是,我舍弃了短视者,如同天鹅到达了大湖。


「在乔达摩的教法之前,他们先前所说的这些,\hfill\textcolor{gray}{\footnotesize \textbf{1142}} \\
「『这曾是如此、这将是如此』,这一切都是传闻,\\
「这一切都是寻的增长。

「除去暗冥者独自而坐,具有光,他是放光者,\hfill\textcolor{gray}{\footnotesize \textbf{1143}} \\
「乔达摩是宏慧者,乔达摩是宏智者,


「他对我开示的法,自见、无时、\hfill\textcolor{gray}{\footnotesize \textbf{1144}} \\
「爱尽、无灾,无处有其雷同者。」


「褐者!你是否须臾间离开过这\hfill\textcolor{gray}{\footnotesize \textbf{1145}} \\
「宏慧的乔达摩,宏智的乔达摩?


「他对你开示的法,自见、无时、\hfill\textcolor{gray}{\footnotesize \textbf{1146}} \\
「爱尽、无灾,无处有其雷同者。」

「婆罗门!我未须臾间离开过这\hfill\textcolor{gray}{\footnotesize \textbf{1147}} \\
「宏慧的乔达摩,宏智的乔达摩,


「他对我开示的法,自见、无时、\hfill\textcolor{gray}{\footnotesize \textbf{1148}} \\
「爱尽、无灾,无处有其雷同者。

「我用意去看他,如同用眼,日夜不放逸,婆罗门!\hfill\textcolor{gray}{\footnotesize \textbf{1149}} \\
「我礼敬着度夜,因此,我认为未曾离开。


「我的信、喜、意、念都不曾离开乔达摩的教法,\hfill\textcolor{gray}{\footnotesize \textbf{1150}} \\
「无论宏慧者去向何方,我都向之倾身。


「我老迈、力气弱,因此身体不能奔赴那里,\hfill\textcolor{gray}{\footnotesize \textbf{1151}} \\
「我始终以思惟之行而去,婆罗门!因为我的意与之相应。


「在淤泥中躺着颤栗,从洲渚漂流到洲渚,\hfill\textcolor{gray}{\footnotesize \textbf{1152}} \\
「然后,我看到了等正觉,已度过暴流、无漏。」


「好比婆迦利曾信解于信,以及善器、旷野乔达摩,\hfill\textcolor{gray}{\footnotesize \textbf{1153}} \\
「如是,你也应信解于信!褐者!你将去到死境的彼岸。」


「听了牟尼的话,我更加净喜,这\hfill\textcolor{gray}{\footnotesize \textbf{1154}} \\
「去蔽的等觉、无荒秽、具辩才者,


「证知了上天,了知了一切上下,\hfill\textcolor{gray}{\footnotesize \textbf{1155}} \\
「大师彻底解决了自称有疑者的问题。


「不可摧伏、不可动摇、无有雷同之处,\hfill\textcolor{gray}{\footnotesize \textbf{1156}} \\
「我定将到达,我对此没有疑惑,如是,请受持具信解之心的我!」
