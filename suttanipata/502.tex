\section{阿耆多学童问}

\subsection\*{\textbf{1039} {\footnotesize 〔PTS 1032〕}}

\textbf{「世间被什么覆蔽?」尊者阿耆多说,「为何它不显露?\\}
\textbf{「你说什么是它的胶合?什么是它的大怖畏?」}

\begin{enumerate}\item 在此问中,\textbf{覆蔽},即被遮蔽。\textbf{你说什么是它的胶合},即你说什么是世间的胶合。\end{enumerate}

\subsection\*{\textbf{1040} {\footnotesize 〔PTS 1033〕}}

\textbf{「世间被无明覆蔽,阿耆多!」世尊说,「由悭贪、放逸而不显露,\\}
\textbf{「我说渴望是胶合,苦是它的大怖畏。」}

\begin{enumerate}\item \textbf{由悭贪、放逸而不显露},即由悭吝之因、放逸之因而不显露。因为悭吝使其不以布施等功德显露,放逸使其不以戒等。\textbf{渴望是胶合},即渴爱是世间的胶合,好比猿胶之于猿。\textbf{苦},即生等苦。\end{enumerate}

\subsection\*{\textbf{1041} {\footnotesize 〔PTS 1034〕}}

\textbf{「众流四处流淌,」尊者阿耆多说,「什么是众流的遮止?\\}
\textbf{「请说众流的防护!众流应以什么来阻碍?」}

\begin{enumerate}\item \textbf{众流四处流淌},即渴爱等的众流流入一切色等入处。\textbf{什么是遮止},即它们的障碍为何、守护为何?\textbf{请说防护},即请说它们被称为遮止的防护,以此而问有余之舍弃。\textbf{众流应以什么来阻碍},即这些众流应以何法来阻碍、截断,以此而问无余之舍弃。\end{enumerate}

\subsection\*{\textbf{1042} {\footnotesize 〔PTS 1035〕}}

\textbf{「世间的这些众流,阿耆多!」世尊说,「念是它们的遮止,\\}
\textbf{「我说众流的防护,它们应以慧来阻碍。」}

\begin{enumerate}\item \textbf{念是它们的遮止},即与毗婆舍那相应、探求善法之趣的念,是这些众流的遮止。\textbf{我说众流的防护},意即我说此念即是众流的防护。\textbf{应以慧来阻碍},即这些众流应以超过通达无常等的道慧来完全地阻碍于色等。\end{enumerate}

\subsection\*{\textbf{1043} {\footnotesize 〔PTS 1036〕}}

\textbf{「即此慧与念,」尊者阿耆多说,「以及名色,先生!\\}
\textbf{「既然问到,请告诉我!它在何处灭去?」}

\begin{enumerate}\item 在此问颂中,你所说的\textbf{慧与念},和余下的\textbf{名色},这些全都在\textbf{何处}灭去,\textbf{既然问到}这问题,\textbf{请告诉我}!当知略义如是。\end{enumerate}

\subsection\*{\textbf{1044} {\footnotesize 〔PTS 1037〕}}

\textbf{「你所问的这问题,阿耆多!我对你说,\\}
\textbf{「于此,名与色无余地灭去:\\}
\textbf{「因识的灭,它即在此灭去。」}

\begin{enumerate}\item 而在其答颂中,因为慧与念都归于名,所以不再个别叙述。此中的略义为:你,\textbf{阿耆多!所问}我\textbf{的这问题}「它在何处灭去」,\textbf{我对你说,于此,名与色无余地灭去}:\textbf{因}彼彼\textbf{识的灭},与之相伴,不先不后,\textbf{它即在此灭去},它即在识灭处灭去,由识灭而彼灭,即是说不得越此。\end{enumerate}

\subsection\*{\textbf{1045} {\footnotesize 〔PTS 1038〕}}

\textbf{「那些已察知法者,与此处的种种有学,\\}
\textbf{「既然问到,请贤者告诉我他们的威仪!先生!」}

\begin{enumerate}\item 至此,已由「苦是它的大怖畏」阐明了苦谛,由「这些众流」阐明了集谛,由「它们应以慧来阻碍」阐明了道谛,由「无余地灭去」阐明了灭谛,如是听闻了四谛后,仍未证得圣地,为问有学与无学的行道,说了此颂。
\item 这里,\textbf{已察知法者},即以无常等思量诸法者,即阿罗汉的同义语。\textbf{有学},即修学戒等的其余圣补特伽罗。\textbf{种种},即众多的七类人。\textbf{请贤者告诉我他们的威仪},即请贤明的智者您告诉我这些有学、无学的行道。\end{enumerate}

\subsection\*{\textbf{1046} {\footnotesize 〔PTS 1039〕}}

\textbf{「他不应贪求于爱欲,他不应污浊其意,\\}
\textbf{「善巧于一切法,具念的比丘便能游行。」}

\begin{enumerate}\item 于是,因为有学应以欲贪盖为首舍弃一切烦恼,所以世尊以前半颂对其显示有学的行道。其义为:\textbf{他不应}以烦恼欲\textbf{贪求}物\textbf{欲},且为舍弃身恶行等令意污浊的法,\textbf{他不应污浊其意}。
\item 而因为无学由以无常等衡量一切诸行等故,\textbf{善巧于一切法},且由身随念等\textbf{具念},由破碎了有身见等故,得证\textbf{比丘}相已,便能\textbf{游行}于一切威仪路,所以,以后半颂显示无学的行道。其余一切处皆自明。
\item 如是,世尊以阿罗汉为顶点完成了开示。当开示终了,阿耆多与一千弟子即住于阿罗汉,而其他数千人生起了法眼。且伴随着阿罗汉的证得,尊者阿耆多及千名弟子的羚羊皮、萦发、树皮衣等都消失了,而全都持着神变所成的衣钵,发长二指,成为「来!比丘」,礼敬着世尊,合掌落坐。\end{enumerate}

