\section{诵彼岸道颂}

\begin{enumerate}\item 此处的连结为:当世尊开示「彼岸道」时,一万六千萦发者证得了阿罗汉,其余十四俱胝之数的人天得了法的现观。这便如古人所说:\begin{quoting}随后,在惬意的石上,在彼岸道的集会中,\\佛陀令十四俱胝的生类得至不死。\end{quoting}而当法的开示结束,从各处来集的人们便以世尊的威力现身于各自的村镇。世尊也为十六随侍等数千比丘围绕,去了舍卫国。于此,褐者礼拜了世尊,说:「尊者!我前去告知波婆利佛陀的出世,因为我已应允他。」然后,在得到世尊的许可后,唯以智行去到乔陀婆利岸边,以步行朝草庵走去。
\item 婆罗门波婆利坐而观察道路,远远看到他没了篮子、萦发等,以比丘的衣装而来,便得出结论「佛陀出世」。到后便问他:「褐者!佛陀是否出世?」「唯!婆罗门!已经出世,坐在石支提上对我们开示了法,我将对你说。」随后,波婆利便以极大的恭敬,携随众供养了他,备好坐具。褐者于此落坐,便说了「我将诵出彼岸道……」等等。\end{enumerate}

\subsection\*{\textbf{1138} {\footnotesize 〔PTS 1131〕}}

\textbf{「我将诵出彼岸道,」尊者褐者说,\\}
\textbf{「无垢的宏慧者如是见便如是宣说,\\}
\textbf{「离欲、消尽的龙象,为何会妄语?}

\begin{enumerate}\item 这里,\textbf{我将诵出},即我将诵出世尊之所诵。\textbf{如是见},即如自身以对谛的等觉、以不共智而见。\textbf{离欲},即舍弃了欲。文本也作 nikkamo,即具有精进或从不善分出离之义。\textbf{消尽},即无有烦恼丛林,或即无有渴爱。\textbf{为何会妄语},即会以何烦恼而妄语?显示其已舍弃了这些。以此令婆罗门对听闻生起热情。\end{enumerate}

\subsection\*{\textbf{1139} {\footnotesize 〔PTS 1132〕}}

\textbf{「舍弃了尘垢与愚痴者、舍弃了慢与覆藏者的\\}
\textbf{「具有德泽的话语,噫!我将宣扬!}

\begin{enumerate}\item \textbf{具有德泽},即具有功德。\end{enumerate}

\subsection\*{\textbf{1140} {\footnotesize 〔PTS 1133〕}}

\textbf{「除去暗冥的佛陀具一切眼,已到世间的边际,超越一切有,\\}
\textbf{「无漏,舍弃了一切苦,以真实得名者,梵天!为我所侍奉。}

\begin{enumerate}\item \textbf{以真实得名者},即「佛陀」与以真实所得之名号相称。\textbf{梵天},即称呼此婆罗门。\end{enumerate}

\subsection\*{\textbf{1141} {\footnotesize 〔PTS 1134〕}}

\textbf{「好比鸟儿舍弃了灌木丛,居于果实丰硕的森林,\\}
\textbf{「如是,我舍弃了短视者,如同天鹅到达了大湖。}

\begin{enumerate}\item \textbf{灌木丛},即小树林。\textbf{居于果实丰硕的森林},即来到充满各种果实的森林而住。\textbf{短视者},即波婆利以降的小慧者。\textbf{大湖},即阿耨达等的大水聚。\end{enumerate}

\subsection\*{\textbf{1142} {\footnotesize 〔PTS 1135〕}}

\textbf{「在乔达摩的教法之前,他们先前所说的这些,\\}
\textbf{「『这曾是如此、这将是如此』,这一切都是传闻,\\}
\textbf{「这一切都是寻的增长。\footnote{此颂与\textbf{黄金学童问}第 1091 颂的前五句大致相同。}}

\subsection\*{\textbf{1143} {\footnotesize 〔PTS 1136〕}}

\textbf{「除去暗冥者独自而坐,具有光,他是放光者,\\}
\textbf{「乔达摩是宏慧者,乔达摩是宏智者,}

\begin{enumerate}\item \textbf{宏慧者},即智的旗帜\footnote{宏慧者 \textit{bhūripaññāṇa}:义注释作「旗帜 \textit{dhaja}」是因为 paññāṇa 兼有智慧和标识之义。}。\textbf{宏智者},即广慧者。\end{enumerate}

\subsection\*{\textbf{1144} {\footnotesize 〔PTS 1137〕}}

\textbf{「他对我开示的法,自见、无时、\\}
\textbf{「爱尽、无灾,无处有其雷同者。」}

\begin{enumerate}\item \textbf{自见、无时},即其果亲自可见,且其果无需时间的间隔即可证。\textbf{无灾},即无有烦恼之灾。\end{enumerate}

\subsection\*{\textbf{1145} {\footnotesize 〔PTS 1138〕}}

\textbf{「褐者!你是否须臾间离开过这\\}
\textbf{「宏慧的乔达摩,宏智的乔达摩?}

\begin{enumerate}\item 于是,波婆利对他说了二颂。\end{enumerate}

\subsection\*{\textbf{1146} {\footnotesize 〔PTS 1139〕}}

\textbf{「他对你开示的法,自见、无时、\\}
\textbf{「爱尽、无灾,无处有其雷同者。」}

\subsection\*{\textbf{1147} {\footnotesize 〔PTS 1140〕}}

\textbf{「婆罗门!我未须臾间离开过这\\}
\textbf{「宏慧的乔达摩,宏智的乔达摩,}

\begin{enumerate}\item 随后,褐者为显明未曾离开世尊跟前,说了以下几颂。\end{enumerate}

\subsection\*{\textbf{1148} {\footnotesize 〔PTS 1141〕}}

\textbf{「他对我开示的法,自见、无时、\\}
\textbf{「爱尽、无灾,无处有其雷同者。}

\subsection\*{\textbf{1149} {\footnotesize 〔PTS 1142〕}}

\textbf{「我用意去看他,如同用眼,日夜不放逸,婆罗门!\\}
\textbf{「我礼敬着度夜,因此,我认为未曾离开。}

\begin{enumerate}\item \textbf{我用意去看他,如同用眼},即我好像用眼看那佛陀一般,用意去看。\end{enumerate}

\subsection\*{\textbf{1150} {\footnotesize 〔PTS 1143〕}}

\textbf{「我的信、喜、意、念都不曾离开乔达摩的教法,\\}
\textbf{「无论宏慧者去向何方,我都向之倾身。}

\begin{enumerate}\item \textbf{我都向之倾身},即显示我都朝着佛陀所在的方向倾身,趋附于彼、倾向于彼。\end{enumerate}

\subsection\*{\textbf{1151} {\footnotesize 〔PTS 1144〕}}

\textbf{「我老迈、力气弱,因此身体不能奔赴那里,\\}
\textbf{「我始终以思惟之行而去,婆罗门!因为我的意与之相应。}

\begin{enumerate}\item \textbf{力气弱},即力气少,或即是说力少、气弱且精力不足。\textbf{因此身体不能奔赴},即因这力气弱,身体不能前行,或不能去到佛陀处。文本也作 na pareti,义同。\textbf{那里},即佛陀跟前。\textbf{与之相应},即显示与佛陀相应。\end{enumerate}

\subsection\*{\textbf{1152} {\footnotesize 〔PTS 1145〕}}

\textbf{「在淤泥中躺着颤栗,从洲渚漂流到洲渚,\\}
\textbf{「然后,我看到了等正觉,已度过暴流、无漏。」}

\begin{enumerate}\item \textbf{在淤泥中躺着},即在爱欲的泥淖躺着。\textbf{从洲渚漂流到洲渚},即从大师等到大师等。\textbf{然后,我看到了等正觉},即当我如是执取恶见而彷徨时,在石支提看到了佛陀。\end{enumerate}

\subsection\*{\textbf{1153} {\footnotesize 〔PTS 1146〕}}

\textbf{「好比婆迦利曾信解于信,以及善器、旷野乔达摩,\\}
\textbf{「如是,你也应信解于信!褐者!你将去到死境的彼岸。」}

\begin{enumerate}\item 在上颂的最后,世尊在了知褐者与波婆利的根已成熟后,仍立于舍卫国,放出金色的光。褐者正坐着对波婆利解说佛陀的功德,看到了这光,观察「这是什么」,看到世尊像是站在自己前面一样,便告诉婆罗门波婆利「佛陀来了」。婆罗门从坐起,合掌而立。世尊将光遍满,向婆罗门显示自身,在了知二人的适宜后,唯对褐者说了此颂。
\item 其义为:\textbf{好比婆迦利}长老曾信解于信,且以信之轭得证阿罗汉,又好比十六人之一、名为\textbf{善器}者,又好比\textbf{旷野乔达摩},\textbf{如是,你也应信解于信},从此信解于信起,以「一切行无常」等方法开始修观,\textbf{你将去到死境的彼岸}、涅槃,即以阿罗汉为顶点完成了开示。当开示终了,褐者住于阿罗汉,波婆利住于阿那含果,而波婆利婆罗门的五百学生则成了须陀洹。\end{enumerate}

\subsection\*{\textbf{1154} {\footnotesize 〔PTS 1147〕}}

\textbf{「听了牟尼的话,我更加净喜,这\\}
\textbf{「去蔽的等觉、无荒秽、具辩才者,}

\begin{enumerate}\item 现在,褐者为宣告自己的净喜,说了以下几颂。这里,\textbf{具辩才者},即具辩无碍解者。\end{enumerate}

\subsection\*{\textbf{1155} {\footnotesize 〔PTS 1148〕}}

\textbf{「证知了上天,了知了一切上下,\\}
\textbf{「大师彻底解决了自称有疑者的问题。}

\begin{enumerate}\item \textbf{证知了上天},即证知了可致上天的法。\textbf{上下},即尊卑,即是说了知了对自己和他人可致上天性的所有种种法。\textbf{自称有疑者},即在仍有疑惑时,自称「我们没有疑惑」者\footnote{义注此处费解。}。\end{enumerate}

\subsection\*{\textbf{1156} {\footnotesize 〔PTS 1149〕}}

\textbf{「不可摧伏、不可动摇、无有雷同之处,\\}
\textbf{「我定将到达,我对此没有疑惑,如是,请受持具信解之心的我!」}

\begin{enumerate}\item \textbf{不可摧伏},即不可被贪等摧伏。\textbf{不可动摇},即非变易法。以两词说涅槃。\textbf{我定将到达},即我肯定会到达这无余涅槃界。\textbf{我对此没有疑惑},即我对此涅槃没有疑惑。\textbf{如是,请受持具信解之心的我},即褐者以世尊的教诫「如是,你也应信解于信」生起自己的信已,以信之轭解脱,为阐明其信解于信,便对世尊说:「如是,请受持具信解之心的我!」此中之意即:「好比你对我说的,如是请受持信解!」\end{enumerate}