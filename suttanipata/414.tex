\section{迅速经}

\begin{enumerate}\item 缘起为何?仍在此大集会中,有些天人生起「那么,什么是证得阿罗汉的行道」之心,为阐明此义,仍以如前的方法,让相佛问自己问题后而说。\end{enumerate}

\subsection\*{\textbf{922} {\footnotesize 〔PTS 915〕}}

\textbf{「我问你,日种!远离以及寂静的境地,大仙!\\}
\textbf{「如何见后,比丘便涅槃,于此世间都无所取?」}

\begin{enumerate}\item 这里,首先在初颂中,在「\textbf{我问}」中,问题被分解为解释未见等。\textbf{日种},即太阳的族属。\textbf{如何见后},即以何原因见,即是说如何转起知见。\end{enumerate}

\subsection\*{\textbf{923} {\footnotesize 〔PTS 916〕}}

\textbf{「名为戏论的根本,」世尊说,「『我是』等一切,他应以智慧止息,\\}
\textbf{「凡是内在的渴爱,为调伏彼等,应始终具念修学。}

\begin{enumerate}\item 于是,因为如是见者能止息烦恼,如是转起知见,便即涅槃,所以世尊为揭示此义,为以种种品类敦促此天众舍弃烦恼,说了以下的五颂。这里,先说初颂的略义为:由被称为戏论故,戏论即名为戏论,其根本即无明等烦恼,\textbf{他应以智慧止息}这\textbf{名为戏论的根本}及以「\textbf{我是}」转起的慢\textbf{等一切}。\textbf{凡是内在}生起\textbf{的渴爱,为调伏彼等,应始终具念修学},应建立念而修学。\end{enumerate}

\subsection\*{\textbf{924} {\footnotesize 〔PTS 917〕}}

\textbf{「任何他所证知的法,内在的或外在的,\\}
\textbf{「不应以此强势,因为这不是善人们所说的寂灭。}

\begin{enumerate}\item 如是,先在初颂中,只以阿罗汉为顶点开示了与三学有关的开示后,又为以舍弃慢作开示,说了此颂。这里,\textbf{任何他所证知的法,内在的},即任何他所了知的家族高贵等自身的功德,\textbf{或外在的},即他所了知的外在的阿阇黎、亲教师们的功德。\textbf{不应以此强势},即不应以此功德强势。\end{enumerate}

\subsection\*{\textbf{925} {\footnotesize 〔PTS 918〕}}

\textbf{「他不应以此认为更优、更劣,或是相同,\\}
\textbf{「为种种形相所触时,他不应将自己区分。}

\begin{enumerate}\item 现在,为显示此不应作的规则,说了此颂。其义为:\textbf{他不应以此}慢\textbf{认为}「我\textbf{更优}、我\textbf{更劣}或我\textbf{相同}」,且\textbf{为}这些家族高贵等\textbf{种种形相所触时},\textbf{他不应}以「我从高贵的家族出家」等方法\textbf{将自己区分}。\end{enumerate}

\subsection\*{\textbf{926} {\footnotesize 〔PTS 919〕}}

\textbf{「他唯应寂止其内在,比丘不应从别处寻求寂静,\\}
\textbf{「对于内在寂静者,没有执取,又何来扬弃?}

\begin{enumerate}\item 如是以舍弃慢作了开示后,现在,仍以止息一切烦恼作开示,说了此颂。这里,\textbf{他唯应寂止其内在},即他唯应寂止自身中的贪等一切烦恼。\textbf{比丘不应从别处寻求寂静},即除了念处等,不应以别的方法寻求寂静。\end{enumerate}

\subsection\*{\textbf{927} {\footnotesize 〔PTS 920〕}}

\textbf{「好比在大海中间,不起波浪而住立,\\}
\textbf{「如是他应住立不动,比丘不应生起任何增盛。」}

\begin{enumerate}\item 现在,为显明内在寂静的漏尽者的如性,说了此颂。其义为:\textbf{好比在大海中间},在四千由旬之量的被称为上下分的中间,或住立于山间的大海的中间,\textbf{不起波浪而住立}、不动摇,\textbf{如是},\textbf{不动}的漏尽者于利养等\textbf{应住立}不动摇,如如的\textbf{比丘不应生起任何}贪等的\textbf{增盛}。\end{enumerate}

\subsection\*{\textbf{928} {\footnotesize 〔PTS 921〕}}

\textbf{「开眼者已解说了调伏危难的亲证之法,\\}
\textbf{「大德!请宣说行道、波罗提木叉和三摩地!」}

\begin{enumerate}\item 现在,相佛为随喜以阿罗汉为顶点所开示的法的开示,并为问这阿罗汉的初行道,说了此颂。这里,\textbf{开眼者},即具足开敞、无碍的五眼者。\textbf{亲证之法},即自己证知、自己现量的法。\textbf{大德},即以「愿您贤善」称呼世尊而说,或者即是说「请宣说您贤善、善妙的行道」。\textbf{波罗提木叉和三摩地},即区别此行道而问,或者,他以此「行道」问道,而以另二者问戒与定。\end{enumerate}

\subsection\*{\textbf{929} {\footnotesize 〔PTS 922〕}}

\textbf{「切勿以眼摇曳,于村谈应遮耳,\\}
\textbf{「不应贪求众味,也不应执取世间任何为我所。}

\begin{enumerate}\item 然后,因为根律仪是戒的守护,或者因为以此渐进开示所作的开示适宜彼等天人,所以,世尊为从根律仪开始显明行道,发起此初颂。这里,\textbf{切勿以眼摇曳},即切勿因想见所未见等,以眼摇曳。\textbf{于村谈应遮耳},即应遮耳于旁论。\end{enumerate}

\subsection\*{\textbf{930} {\footnotesize 〔PTS 923〕}}

\textbf{「当他为触所触时,比丘不应起任何悲伤,\\}
\textbf{「不应渴望有,于诸恐怖也不应震颤。}

\begin{enumerate}\item \textbf{触},即疾病之触。\textbf{不应渴望有},即不应为了驱除此触而愿求欲有等的有。\textbf{于诸恐怖也不应震颤},即于作为此触之缘的狮虎等诸恐怖也不应震颤,或者,不应震颤于其余的鼻根、意根的境域。如是即说了圆满的根律仪。或者,在显示了根律仪后,以此显示「住于林野者见、闻恐怖已,不应震颤」。\end{enumerate}

\subsection\*{\textbf{931} {\footnotesize 〔PTS 924〕}}

\textbf{「然后,食物、饮品、硬食、衣服等,\\}
\textbf{「获得后不应积贮,未获得这些时也不应焦渴。}

\begin{enumerate}\item \textbf{获得后不应积贮},即如法地获得这些食物等中的任何时,应思「林野住者总是难以获得(这些)」,便不积贮。\end{enumerate}

\subsection\*{\textbf{932} {\footnotesize 〔PTS 925〕}}

\textbf{「应禅修,不应游步,应戒离恶作,不应放逸,\\}
\textbf{「然后,比丘应居于安静的坐卧处。}

\begin{enumerate}\item \textbf{应禅修,不应游步},即应乐于禅修,不应游步。\textbf{应戒离恶作,不应放逸},即应去除手的不安等的恶作,且应以常恒行而不放逸。\end{enumerate}

\subsection\*{\textbf{933} {\footnotesize 〔PTS 926〕}}

\textbf{「不应多睡眠,应保持警觉,热忱,\\}
\textbf{「应舍弃倦怠、伪善、戏笑、嬉戏、淫欲以及严饰。}

\begin{enumerate}\item \textbf{倦怠、伪善、戏笑、嬉戏},即怠惰、伪善、戏笑,以及身心的嬉戏。\end{enumerate}

\subsection\*{\textbf{934} {\footnotesize 〔PTS 927〕}}

\textbf{「不应使用阿闼婆、梦、相,以及星占,\\}
\textbf{「我的弟子不应从事鸣声、助孕、医疗。}

\begin{enumerate}\item \textbf{阿闼婆},即阿闼婆咒语之加行。\textbf{梦},即占梦术。\textbf{相},即摩尼之相等。\textbf{鸣声},即鹿等的叫声。\end{enumerate}

\subsection\*{\textbf{935} {\footnotesize 〔PTS 928〕}}

\textbf{「比丘不应震颤于责备,被赞赏时不应高举,\\}
\textbf{「应去除贪以及悭吝、忿怒与诽谤。}

\subsection\*{\textbf{936} {\footnotesize 〔PTS 929〕}}

\textbf{「不应从事买卖,比丘在任何处不应引来指责,\\}
\textbf{「且在村中不应冒犯,不应为好乐利养而与人闲谈。}

\begin{enumerate}\item \textbf{买卖},即\textbf{不应}与五同法者\footnote{五同法者 \textit{sahadhammika}:即比丘、比丘尼、式叉摩那、沙弥、沙弥尼。}一起,以欺诈或以愿求盈利来\textbf{从事}。\textbf{比丘不应引来指责},即为不生起引来指责的烦恼,他不应从其他沙门婆罗门处引来对自己的指责。\textbf{且在村中不应冒犯},即且在村中,不应以俗家的交际等冒犯。\end{enumerate}

\subsection\*{\textbf{937} {\footnotesize 〔PTS 930〕}}

\textbf{「比丘既不应夸耀,也不应说有企图的话,\\}
\textbf{「不应变得鲁莽,不应谈论争议之论。}

\begin{enumerate}\item \textbf{企图},即与衣等相关者,或为此目的所从事者。\end{enumerate}

\subsection\*{\textbf{938} {\footnotesize 〔PTS 931〕}}

\textbf{「不应堕入妄语,不应知而行狡诈,\\}
\textbf{「不应以活命、慧、戒禁蔑视他人。}

\subsection\*{\textbf{939} {\footnotesize 〔PTS 932〕}}

\textbf{「当被激怒时,听到沙门或凡夫们的众多话语后,\\}
\textbf{「不应以恶口回答他们,因为善人们不制造敌对。}

\begin{enumerate}\item \textbf{当被激怒时,听到沙门或凡夫们的众多话语后},即当被激怒、被他人刺激时,听到那些沙门,或刹帝利等类,或其他凡夫的众多不可意的话语后,\textbf{不应回答}。什么原因?\textbf{因为善人们不制造敌对}。\end{enumerate}

\subsection\*{\textbf{940} {\footnotesize 〔PTS 933〕}}

\textbf{「且知晓了这法,比丘审视着,应始终具念而修学,\\}
\textbf{「了知了寂灭为寂静,在乔达摩的教法中不应放逸。}

\begin{enumerate}\item \textbf{且知晓了这法},即知晓了这如上所述的一切法。\textbf{了知了寂灭为寂静},即了知了寂灭为贪等的寂静。\end{enumerate}

\subsection\*{\textbf{941} {\footnotesize 〔PTS 934〕}}

\textbf{「因为他是征服者,不被征服,得见亲证之法,而非传闻,\\}
\textbf{「所以,应当在彼世尊的教法内不放逸,始终礼敬而随学。」}

\begin{enumerate}\item 设问:什么原因不应放逸?「因为他是征服者……」。这里,\textbf{征服},即征服了色等。\textbf{不被征服},即不被彼等征服。\textbf{得见亲证之法,而非传闻},即唯得见现量,而非传闻之法。\textbf{始终礼敬而随学},即应始终礼敬,修学三学。其余一切处皆自明。
\item 而总的来说,此处以「切勿以眼摇曳」等说根律仪,以拒绝积贮「食物、饮品」等为首说资具受用戒,以淫欲、妄语、诽谤等说别解脱律仪戒,以「阿闼婆、梦、相」等说活命遍净戒,以「应禅修」说定,以「比丘审视着」说慧,以「应始终具念而修学」再次略说三学,以「比丘应居于安静的坐卧处,不应多睡眠」等说戒定慧之资助、损害的摄受与除遣。
\item 如是,世尊对相(佛)说了圆满的行道,以阿罗汉为顶点完成了开示。当开示终了,仍与「前分离经」所说的一样,而有现观。\end{enumerate}

