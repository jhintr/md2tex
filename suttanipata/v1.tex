\chapter{蛇品第一}

\section{蛇经}

若调伏生起的忿怒,如同用众药(调伏)蔓延的蛇毒,\hfill\textcolor{gray}{\footnotesize \textbf{1}} \\
那比丘舍弃此岸彼岸,如蛇(舍弃)先前的老皮。


若无余地断除了贪染,如拔擢了临于水面的莲花,\hfill\textcolor{gray}{\footnotesize \textbf{2}} \\
那比丘舍弃此岸彼岸,如蛇(舍弃)先前的老皮。


若无余地断除了渴爱,(如)竭涸了疾驶的川流,\hfill\textcolor{gray}{\footnotesize \textbf{3}} \\
那比丘舍弃此岸彼岸,如蛇(舍弃)先前的老皮。


若无余地扫除了慢,如同洪流(扫除了)危脆的苇堤,\hfill\textcolor{gray}{\footnotesize \textbf{4}} \\
那比丘舍弃此岸彼岸,如蛇(舍弃)先前的老皮。


若于诸有中不得坚实,如寻觅于无花果林(不得)花,\hfill\textcolor{gray}{\footnotesize \textbf{5}} \\
那比丘舍弃此岸彼岸,如蛇(舍弃)先前的老皮。


若其中间无诸忿恨,且已超越这有与无有的状态,\hfill\textcolor{gray}{\footnotesize \textbf{6}} \\
那比丘舍弃此岸彼岸,如蛇(舍弃)先前的老皮。


若其诸寻业已熏散,内在已善加廓清而无余,\hfill\textcolor{gray}{\footnotesize \textbf{7}} \\
那比丘舍弃此岸彼岸,如蛇(舍弃)先前的老皮。


若既不超前,也不折返,超越了这一切戏论,\hfill\textcolor{gray}{\footnotesize \textbf{8}} \\
那比丘舍弃此岸彼岸,如蛇(舍弃)先前的老皮。


若既不超前,也不折返,了知了世间「这一切虚妄」,\hfill\textcolor{gray}{\footnotesize \textbf{9}} \\
那比丘舍弃此岸彼岸,如蛇(舍弃)先前的老皮。


若既不超前,也不折返,以「这一切虚妄」离贪,\hfill\textcolor{gray}{\footnotesize \textbf{10}} \\
那比丘舍弃此岸彼岸,如蛇(舍弃)先前的老皮。


若既不超前,也不折返,以「这一切虚妄」离染,\hfill\textcolor{gray}{\footnotesize \textbf{11}} \\
那比丘舍弃此岸彼岸,如蛇(舍弃)先前的老皮。

若既不超前,也不折返,以「这一切虚妄」离嗔,\hfill\textcolor{gray}{\footnotesize \textbf{12}} \\
那比丘舍弃此岸彼岸,如蛇(舍弃)先前的老皮。

若既不超前,也不折返,以「这一切虚妄」离痴,\hfill\textcolor{gray}{\footnotesize \textbf{13}} \\
那比丘舍弃此岸彼岸,如蛇(舍弃)先前的老皮。

若其已无任何随眠,并且诸不善根已被铲除,\hfill\textcolor{gray}{\footnotesize \textbf{14}} \\
那比丘舍弃此岸彼岸,如蛇(舍弃)先前的老皮。


若其已无任何恼患所生的、回到此岸的众缘,\hfill\textcolor{gray}{\footnotesize \textbf{15}} \\
那比丘舍弃此岸彼岸,如蛇(舍弃)先前的老皮。


若其已无任何欲念所生的、系缚于有的众因,\hfill\textcolor{gray}{\footnotesize \textbf{16}} \\
那比丘舍弃此岸彼岸,如蛇(舍弃)先前的老皮。


若已舍弃五盖,无患,度诸犹疑,离于箭刺,\hfill\textcolor{gray}{\footnotesize \textbf{17}} \\
那比丘舍弃此岸彼岸,如蛇(舍弃)先前的老皮。


\section{有财者经}

「我已煮好米饭,挤好牛奶,」有财者牛场主说,「在摩醯沿岸聚居,\hfill\textcolor{gray}{\footnotesize \textbf{18}} \\
「屋已覆蔽,火已燃起,那么,若您愿意,下雨吧!天!」


「我已无有忿怒,离于荒秽,」世尊说,「在摩醯沿岸住一夜,\hfill\textcolor{gray}{\footnotesize \textbf{19}} \\
「屋已敞开,火已熄灭,那么,若您愿意,下雨吧!天!」


「没有苍蝇、蚊子,」有财者牛场主说,「牛群在丰草的泽边游荡,\hfill\textcolor{gray}{\footnotesize \textbf{20}} \\
「它们堪能忍受来临的雨,那么,若您愿意,下雨吧!天!」


「筏已扎结实,」世尊说,「已度,已到彼岸,堪能调伏暴流,\hfill\textcolor{gray}{\footnotesize \textbf{21}} \\
「筏已没有意义,那么,若您愿意,下雨吧!天!」


「我的妻子顺从、不动摇,」有财者牛场主说,「长期同居,适意可人,\hfill\textcolor{gray}{\footnotesize \textbf{22}} \\
「我未听到任何关于她的过恶,那么,若您愿意,下雨吧!天!」


「我的心顺从、解脱,」世尊说,「长期遍修,已善调御,\hfill\textcolor{gray}{\footnotesize \textbf{23}} \\
「我已没有过恶,那么,若您愿意,下雨吧!天!」


「我以自己的酬劳养活,」有财者牛场主说,「儿女与我一起,无病,\hfill\textcolor{gray}{\footnotesize \textbf{24}} \\
「我未听到任何关于他们的过恶,那么,若您愿意,下雨吧!天!」


「我不是任何人的雇工,」世尊说,「我凭所得在一切世间游行,\hfill\textcolor{gray}{\footnotesize \textbf{25}} \\
「酬劳已没有意义,那么,若您愿意,下雨吧!天!」


「有母牛,有奶牛,」有财者牛场主说,「也有孕牛和待配的牛,\hfill\textcolor{gray}{\footnotesize \textbf{26}} \\
「这里还有作为头牛的公牛,那么,若您愿意,下雨吧!天!」


「没有母牛,没有奶牛,」世尊说,「也没有孕牛和待配的牛,\hfill\textcolor{gray}{\footnotesize \textbf{27}} \\
「这里也没有作为头牛的公牛,那么,若您愿意,下雨吧!天!」


「埋好的柱子无法撼动,」有财者牛场主说,「文阇草编的绳子崭新、整齐,\hfill\textcolor{gray}{\footnotesize \textbf{28}} \\
「任凭几头牛犊也扯不断,那么,若您愿意,下雨吧!天!」


「如公牛扯断了束缚,」世尊说,「如大象撕裂了腐蔓,\hfill\textcolor{gray}{\footnotesize \textbf{29}} \\
「我绝不再入胎室,那么,若您愿意,下雨吧!天!」


顷刻之间,大云降雨,遍满低洼与高地,\hfill\textcolor{gray}{\footnotesize \textbf{30}} \\
听到天降大雨,有财者便说了此义:


「我们的所得确实匪浅,我们得见世尊,\hfill\textcolor{gray}{\footnotesize \textbf{31}} \\
「我们皈依您,具眼者!请您作我们的大师,大牟尼!


「妻子与我都顺从,愿在善逝处修行梵行,\hfill\textcolor{gray}{\footnotesize \textbf{32}} \\
「我们愿得达生死的彼岸,得尽苦的边际。」


「有孩子的因孩子而欢喜,」恶者魔罗说,「同样,有牛的因牛而欢喜,\hfill\textcolor{gray}{\footnotesize \textbf{33}} \\
「因为依持是人的欢喜,若离开依持,他就不会欢喜。」


「有孩子的因孩子而忧伤,」世尊说,「同样,有牛的因牛而忧伤,\hfill\textcolor{gray}{\footnotesize \textbf{34}} \\
「因为依持是人的忧伤,若离开依持,他就不会忧伤。」


\section{犀牛角经}


对一切生物放下了棍杖,也不恼害其中的某个,\hfill\textcolor{gray}{\footnotesize \textbf{35}} \\
不会希求孩子,遑论朋友?他应当独自游行,像犀牛角一样。


从事交际便有诸多爱执,这苦追随爱执而生,\hfill\textcolor{gray}{\footnotesize \textbf{36}} \\
觉察着爱执所生的过患,他应当独自游行,像犀牛角一样。


同情着朋友和知心,被牵绊的心便忽视义利,\hfill\textcolor{gray}{\footnotesize \textbf{37}} \\
觉察着这亲密中的怖畏,他应当独自游行,像犀牛角一样。


关切于妻子与儿女,好比修竹交织牵扯,\hfill\textcolor{gray}{\footnotesize \textbf{38}} \\
如竹笋般不受羁绊,他应当独自游行,像犀牛角一样。


就像林野中不羁的鹿,随意地漫步于行处,\hfill\textcolor{gray}{\footnotesize \textbf{39}} \\
有智之士觉察着自由,他应当独自游行,像犀牛角一样。


在朋友中,于居止行游,总有应对商谈,\hfill\textcolor{gray}{\footnotesize \textbf{40}} \\
觉察着不被渴求的自由,他应当独自游行,像犀牛角一样。


在朋友中有嬉戏、喜乐,在孩子中有深厚的爱怜,\hfill\textcolor{gray}{\footnotesize \textbf{41}} \\
厌烦着爱别离,他应当独自游行,像犀牛角一样。


游行四方者无有障碍,随所遇而知足,\hfill\textcolor{gray}{\footnotesize \textbf{42}} \\
忍受危难,而不惊惧,他应当独自游行,像犀牛角一样。


即便有些出家人也难以摄受,还有居家的在家人,\hfill\textcolor{gray}{\footnotesize \textbf{43}} \\
不再操心别人的孩子们,他应当独自游行,像犀牛角一样。


除去了俗家相,好比树叶落尽的黑檀,\hfill\textcolor{gray}{\footnotesize \textbf{44}} \\
英雄斩断了俗家束缚,他应当独自游行,像犀牛角一样。


若可得贤明、同行、善住、坚定的朋友,\hfill\textcolor{gray}{\footnotesize \textbf{45}} \\
征服了一切危难,他满意、具念,应当与其同行。


若不可得贤明、同行、善住、坚定的朋友,\hfill\textcolor{gray}{\footnotesize \textbf{46}} \\
如国王舍弃了征服的国土,他应当独自游行,如林中的大象。


当然,我们赞叹成就的朋友,应亲近更胜或同等的朋友,\hfill\textcolor{gray}{\footnotesize \textbf{47}} \\
得不到这些,无过地受用,他应当独自游行,像犀牛角一样。


看到锻工之子善加打造的光彩的金(钏)\hfill\textcolor{gray}{\footnotesize \textbf{48}} \\
双双在臂上哐当作响,他应当独自游行,像犀牛角一样。


如是,与伴侣一起,我也会有言谈或执著,\hfill\textcolor{gray}{\footnotesize \textbf{49}} \\
觉察着这未来的怖畏,他应当独自游行,像犀牛角一样。


爱欲实在多彩、甜蜜而悦意,以各色形相搅乱着心,\hfill\textcolor{gray}{\footnotesize \textbf{50}} \\
看到了种种爱欲中的过患,他应当独自游行,像犀牛角一样。


「这对于我是灾、痈、祸害,是病、箭、怖畏」,\hfill\textcolor{gray}{\footnotesize \textbf{51}} \\
看到了这种种爱欲中的怖畏,他应当独自游行,像犀牛角一样。


寒热饥渴,以及风晒虻蛇,\hfill\textcolor{gray}{\footnotesize \textbf{52}} \\
克服了这一切,他应当独自游行,像犀牛角一样。


如象离了群,肩背宽阔、莲色而伟岸,\hfill\textcolor{gray}{\footnotesize \textbf{53}} \\
随所欢喜地住于林野,他应当独自游行,像犀牛角一样。


「对耽乐聚会者,不可能证得暂时的解脱」,\hfill\textcolor{gray}{\footnotesize \textbf{54}} \\
留意到日种的话语,他应当独自游行,像犀牛角一样。


「已超越见的蠢动,已达决定,已获得道,\hfill\textcolor{gray}{\footnotesize \textbf{55}} \\
「我生起智,不由他人引导」,他应当独自游行,像犀牛角一样。


离贪求、离诡诈、离渴望,离覆藏,除去恶浊、愚痴,\hfill\textcolor{gray}{\footnotesize \textbf{56}} \\
于一切世间无意乐,他应当独自游行,像犀牛角一样。


他应回避示以非义、住于不正的恶友,\hfill\textcolor{gray}{\footnotesize \textbf{57}} \\
莫自己亲近执著而放逸者,他应当独自游行,像犀牛角一样。


他应结交多闻、持法、高尚、富有辩才的朋友,\hfill\textcolor{gray}{\footnotesize \textbf{58}} \\
知晓了义利,调伏了疑惑,他应当独自游行,像犀牛角一样。


不满于世间的嬉戏、喜乐和欲乐,便不再关切,\hfill\textcolor{gray}{\footnotesize \textbf{59}} \\
戒离严饰,言语真实,他应当独自游行,像犀牛角一样。


孩子、妻子、父亲、母亲,财富、谷物及众眷属,\hfill\textcolor{gray}{\footnotesize \textbf{60}} \\
舍弃了如其限度的爱欲,他应当独自游行,像犀牛角一样。


「这是染著,于此幸福有限,乐味些许,而苦于此更多,\hfill\textcolor{gray}{\footnotesize \textbf{61}} \\
「这是钓钩」,具慧者如此了知已,他应当独自游行,像犀牛角一样。


摆脱了结缚,如水中游鱼冲破了网,\hfill\textcolor{gray}{\footnotesize \textbf{62}} \\
如火不返余烬,他应当独自游行,像犀牛角一样。


目光下视,且不游步,防护诸根,守护于意,\hfill\textcolor{gray}{\footnotesize \textbf{63}} \\
不漏泄,不为所烧,他应当独自游行,像犀牛角一样。


除去了俗家相,好比树叶密覆,\hfill\textcolor{gray}{\footnotesize \textbf{64}} \\
著袈裟衣而出家后,他应当独自游行,像犀牛角一样。


于众味不贪图、不动摇,不养育他人,次第行乞,\hfill\textcolor{gray}{\footnotesize \textbf{65}} \\
于各家不牵绊其心,他应当独自游行,像犀牛角一样。


舍弃了心之五盖,除去了一切随烦恼,\hfill\textcolor{gray}{\footnotesize \textbf{66}} \\
无依止者斩断了爱执之过,他应当独自游行,像犀牛角一样。


背离了先前的苦乐与喜忧,\hfill\textcolor{gray}{\footnotesize \textbf{67}} \\
获得了舍、止息、清净,他应当独自游行,像犀牛角一样。


为得第一义而勇猛精进,心不沉滞,行不怠堕,\hfill\textcolor{gray}{\footnotesize \textbf{68}} \\
坚持不懈,具足强力,他应当独自游行,像犀牛角一样。


不疏忽于宴坐与禅那,于诸法始终随法行,\hfill\textcolor{gray}{\footnotesize \textbf{69}} \\
于诸有思惟过患,他应当独自游行,像犀牛角一样。


希求着渴爱的灭尽,不放逸,聪明,多闻,具念,\hfill\textcolor{gray}{\footnotesize \textbf{70}} \\
已察知法,已入决定,具足精勤,他应当独自游行,像犀牛角一样。


好比狮子不惊怖于声响,好比清风不羁绊于罗网,\hfill\textcolor{gray}{\footnotesize \textbf{71}} \\
好比莲花不著于水,他应当独自游行,像犀牛角一样。


好比狮子齿牙有力,众兽之王制胜、征服而行,\hfill\textcolor{gray}{\footnotesize \textbf{72}} \\
应亲近边鄙的坐卧处,他应当独自游行,像犀牛角一样。


适时习行慈、舍、悲、喜解脱,\hfill\textcolor{gray}{\footnotesize \textbf{73}} \\
不被一切世间妨碍,他应当独自游行,像犀牛角一样。


舍弃了贪、嗔、痴,摆脱了结缚,\hfill\textcolor{gray}{\footnotesize \textbf{74}} \\
不惊怖于生命的灭尽,他应当独自游行,像犀牛角一样。


他们结交、亲近缘于利益,当今难得没有缘由的朋友,\hfill\textcolor{gray}{\footnotesize \textbf{75}} \\
不纯洁的人们谋图自利,他应当独自游行,像犀牛角一样。


\section{耕田婆罗豆婆遮经}

如是我闻。一时世尊住摩竭陀南山的一那罗婆罗门村。尔时,耕田婆罗豆婆遮婆罗门的五百张犁已经上轭,正值播种时节。于是,世尊晨朝著了下衣,持了衣钵,便往耕田婆罗豆婆遮婆罗门的农场走去。尔时,耕田婆罗豆婆遮婆罗门的食物分发正在进行。于是,世尊便往食物分发处走去,走到后,站在一边。


耕田婆罗豆婆遮婆罗门看到世尊站着乞食。看到后,对世尊说:「沙门!我耕作、播种,耕作、播种后我才吃饭,沙门!你也该(为自己)耕作、播种,耕作、播种后你才该吃饭。」「婆罗门!我也耕作、播种,耕作、播种后我才吃饭。」


「但我们没看见乔达摩君的轭、犁、犁铧、刺棒或是耕牛,然而乔达摩君却如是说『婆罗门!我也耕作、播种,耕作、播种后我才吃饭』。」于是,耕田婆罗豆婆遮婆罗门以偈颂对世尊说:


「您自称是耕者,但我们却没看见您的耕作,\hfill\textcolor{gray}{\footnotesize \textbf{76}} \\
「既然问到,请对我们说说耕作,好让我们知道您的耕作。」

「信是种子,苦行是雨水,慧是我的轭与犁,\hfill\textcolor{gray}{\footnotesize \textbf{77}} \\
「惭是辕,意是轭带,念是我的犁铧与刺棒。


「身防护,语防护,节制于食与腹,\hfill\textcolor{gray}{\footnotesize \textbf{78}} \\
「我用真实芟夷,调柔是我的解脱。


「精进是我负重的牛,运载至离轭安稳,\hfill\textcolor{gray}{\footnotesize \textbf{79}} \\
「它前行而不退转,所到之处即不忧伤。


「如是,这耕作已作,那是不死之果,\hfill\textcolor{gray}{\footnotesize \textbf{80}} \\
「他耕作这耕作已,从一切苦中解脱。」


于是,耕田婆罗豆婆遮婆罗门用大铜钵盛了粥,授予世尊:「请乔达摩君吃粥!您是耕者,因为乔达摩君耕作不死之果。」


「我不应受用吟颂之物,对诸正观者,婆罗门!此即非法,\hfill\textcolor{gray}{\footnotesize \textbf{81}} \\
「诸佛拒绝吟颂之物,法既存在,婆罗门!此即行事之道。


「对整全者、大仙、漏尽者、恶作止息者,应以其它\hfill\textcolor{gray}{\footnotesize \textbf{82}} \\
「饮食给侍,因为他是希求福德者的良田。」


「那么,乔达摩君!我应把这粥给谁呢?」「我实不见,婆罗门!在这俱有天、魔、梵、沙门婆罗门、天人的人世间,有谁受用此粥而能正常消化的,除了如来或如来弟子,因此你,婆罗门!应把这粥丢到少草处,或投入无生类的水中。」于是,耕田婆罗豆婆遮婆罗门便把这粥投入无生类的水中。被投入水中的粥唧唧啾啾,冒着烟雾和水汽。好比白天晒热的犁铧被投入水中,唧唧啾啾,冒着烟雾和水汽,如是被投入水中的粥唧唧啾啾,冒着烟雾和水汽。


于是,耕田婆罗豆婆遮婆罗门惊恐万状,身毛竖立,往世尊处走去,走到后,以头投于世尊的双足,对世尊说:「希有!乔达摩君!希有!乔达摩君!好比,乔达摩君!能扶正被倾倒的,能揭示被遮蔽的,能给迷者指路,能在黑暗中持油灯,以使『具眼者能见色』,如是乔达摩君以种种方法阐明法。我皈依乔达摩君、法与比丘僧,愿我能在乔达摩君跟前出家,愿我能受具足!」


耕田婆罗豆婆遮婆罗门便在世尊跟前出了家,受了具足。受具后不久,尊者婆罗豆婆遮独一、远离、不放逸、热忱、自励而住,此后不久,就在今生,他以自身的证智证得并具足而住于无上梵行的终了,正是为此义利,族姓子们从家出至非家。他证知:「生已灭尽,梵行已立,应作已作,不更为此。」尊者婆罗豆婆遮便成了众阿罗汉中的某个。


\section{纯陀经}

「我问牟尼、广慧者、」锻工之子纯陀说,「佛陀、法主、离爱者、\hfill\textcolor{gray}{\footnotesize \textbf{83}} \\
「两足尊、御者之最胜,世间有几种沙门?请快说说这个!」


「有四种沙门,没有第五种,纯陀!」世尊说,「作为见证,我向你解释这些,\hfill\textcolor{gray}{\footnotesize \textbf{84}} \\
「胜道者,示道者,依道生活者,以及污道者。」


「诸佛说谁是胜道者?」锻工之子纯陀说,「宣道者如何无等伦?\hfill\textcolor{gray}{\footnotesize \textbf{85}} \\
「既然问到,请对我说说依道生活者,然后,请对我解释污道者!」


「度诸犹疑,离于箭刺,喜于涅槃,无有贪求,\hfill\textcolor{gray}{\footnotesize \textbf{86}} \\
「俱有天的世间的导师,诸佛说此等是胜道者。


「于此了知了最上为最上,即于此宣说、分别法,\hfill\textcolor{gray}{\footnotesize \textbf{87}} \\
「他们说这断疑者、牟尼、不动者是第二种比丘、示道者。


「于善开示的法句依道生活,自制,具念,\hfill\textcolor{gray}{\footnotesize \textbf{88}} \\
「从事着无过的行迹,他们说第三种比丘是活道者。


「披了善行者的外衣,唐突,污家,鲁莽,\hfill\textcolor{gray}{\footnotesize \textbf{89}} \\
「欺瞒、不自制、伪装,以模仿而行,那是污道者。


「若在家人通达了这些,便为多闻、圣人弟子、有慧,\hfill\textcolor{gray}{\footnotesize \textbf{90}} \\
「了知到一切并非这般,见到如此,他的信便不退失,\\
「因为他如何能把败坏与未败坏、清净与不清净等同起来?」


\section{衰败经}

如是我闻。一时世尊住舍卫国祇树给孤独园。于是,有一容貌殊胜的天人在深夜中照亮了整座祇园,往世尊处走去,走到后,礼敬了世尊,站在一边。然后,这位站在一边的天人以偈颂对世尊说:


「我们问问衰败之人!乔达摩!\hfill\textcolor{gray}{\footnotesize \textbf{91}} \\
「前来问您,何为衰败者之因?」


「兴盛易知,衰败易知,\hfill\textcolor{gray}{\footnotesize \textbf{92}} \\
「爱法者兴盛,厌法者衰败。」


「我们了知了这点,这是第一种衰败,\hfill\textcolor{gray}{\footnotesize \textbf{93}} \\
「世尊!请说说何为衰败者的第二因?」


「他喜爱不善人,不去喜爱善人,\hfill\textcolor{gray}{\footnotesize \textbf{94}} \\
「赞许不善人的法,这是衰败者之因。」


「我们了知了这点,这是第二种衰败,\hfill\textcolor{gray}{\footnotesize \textbf{95}} \\
「世尊!请说说何为衰败者的第三因?」

「若人惯于睡眠、惯于集会而不奋起,\hfill\textcolor{gray}{\footnotesize \textbf{96}} \\
「懒惰,现忿怒相,这是衰败者之因。」


「我们了知了这点,这是第三种衰败,\hfill\textcolor{gray}{\footnotesize \textbf{97}} \\
「世尊!请说说何为衰败者的第四因?」

「若对年老、青春已逝的父母,\hfill\textcolor{gray}{\footnotesize \textbf{98}} \\
「堪能却不赡养,这是衰败者之因。」


「我们了知了这点,这是第四种衰败,\hfill\textcolor{gray}{\footnotesize \textbf{99}} \\
「世尊!请说说何为衰败者的第五因?」

「若对婆罗门、沙门或其他的乞食者\hfill\textcolor{gray}{\footnotesize \textbf{100}} \\
「以妄语欺骗,这是衰败者之因。」


「我们了知了这点,这是第五种衰败,\hfill\textcolor{gray}{\footnotesize \textbf{101}} \\
「世尊!请说说何为衰败者的第六因?」

「那人饶有财富,有货币、有食物,\hfill\textcolor{gray}{\footnotesize \textbf{102}} \\
「独自享用美味,这是衰败者之因。」


「我们了知了这点,这是第六种衰败,\hfill\textcolor{gray}{\footnotesize \textbf{103}} \\
「世尊!请说说何为衰败者的第七因?」

「若人以出身为傲,以财产为傲,以种姓为傲,\hfill\textcolor{gray}{\footnotesize \textbf{104}} \\
「鄙视自己的亲戚,这是衰败者之因。」


「我们了知了这点,这是第七种衰败,\hfill\textcolor{gray}{\footnotesize \textbf{105}} \\
「世尊!请说说何为衰败者的第八因?」

「若人沉湎女人,嗜酒,嗜赌,\hfill\textcolor{gray}{\footnotesize \textbf{106}} \\
「倾尽所有,这是衰败者之因。」


「我们了知了这点,这是第八种衰败,\hfill\textcolor{gray}{\footnotesize \textbf{107}} \\
「世尊!请说说何为衰败者的第九因?」

「不满于自身的妻妾,混迹于娼妓,\hfill\textcolor{gray}{\footnotesize \textbf{108}} \\
「混迹于他人的妻妾,这是衰败者之因。」


「我们了知了这点,这是第九种衰败,\hfill\textcolor{gray}{\footnotesize \textbf{109}} \\
「世尊!请说说何为衰败者的第十因?」

「青春已过的男人,带回乳房如小果者,\hfill\textcolor{gray}{\footnotesize \textbf{110}} \\
「出于对她的嫉妒而无法入睡,这是衰败者之因。」


「我们了知了这点,这是第十种衰败,\hfill\textcolor{gray}{\footnotesize \textbf{111}} \\
「世尊!请说说何为衰败者的第十一因?」

「对上瘾、挥霍的女人,或对这样的男人,\hfill\textcolor{gray}{\footnotesize \textbf{112}} \\
「他委以权柄,这是衰败者之因。」


「我们了知了这点,这是第十一种衰败,\hfill\textcolor{gray}{\footnotesize \textbf{113}} \\
「世尊!请说说何为衰败者的第十二因?」

「薄财而多爱,生于刹帝利家族,\hfill\textcolor{gray}{\footnotesize \textbf{114}} \\
「且希求王位,这是衰败者之因。


「智者省察了世间的这些衰败,\hfill\textcolor{gray}{\footnotesize \textbf{115}} \\
「圣者具足知见,追随吉祥的世间。」


\section{贱民经}

如是我闻。一时世尊住舍卫国祇树给孤独园。于是,世尊晨朝著了下衣,持了衣钵,便入舍卫国乞食。尔时,在事火婆罗豆婆遮婆罗门的住处,火已燃起,祭品已备好。于是,世尊在舍卫国次第行乞时,往事火婆罗豆婆遮婆罗门的住处走去。


事火婆罗豆婆遮婆罗门看见世尊从远处走来,看见后,对世尊说:「就那里,秃头!就那里,沙门!就那里,贱民!站住!」如是说已,世尊对事火婆罗豆婆遮婆罗门说:「婆罗门!那你知道贱民或成为贱民之法吗?」「乔达摩君!我并不知道贱民或成为贱民之法,善哉!请乔达摩君对我开示这样的法!好让我知道贱民或成为贱民之法。」「那么,婆罗门!谛听!善加作意!我将要说。」「如是,先生!」事火婆罗豆婆遮婆罗门答世尊。世尊说:


若人忿怒,怨恨,恶且覆藏,\hfill\textcolor{gray}{\footnotesize \textbf{116}} \\
破见,欺瞒,当知他是贱民。


无论对一生者或对再生者,若于此杀害生命,\hfill\textcolor{gray}{\footnotesize \textbf{117}} \\
若于生命无有怜悯,当知他是贱民。


若摧毁、围攻村、镇等,\hfill\textcolor{gray}{\footnotesize \textbf{118}} \\
被认为是压迫者,当知他是贱民。


于村或若林野,凡他人的所属,\hfill\textcolor{gray}{\footnotesize \textbf{119}} \\
出于盗窃而取未给予物,当知他是贱民。


若确实借了债,当被催促时却逃赖:\hfill\textcolor{gray}{\footnotesize \textbf{120}} \\
「没有欠你的债」,当知他是贱民。


若出于对某物的欲求,对路上的行人\hfill\textcolor{gray}{\footnotesize \textbf{121}} \\
加以伤害,取走某物,当知他是贱民。


若人因自、因他以及因财,\hfill\textcolor{gray}{\footnotesize \textbf{122}} \\
作为见证而说妄语,当知他是贱民。


若现身于亲戚或者朋友的妻妾中,\hfill\textcolor{gray}{\footnotesize \textbf{123}} \\
以暴力或以亲昵,当知他是贱民。


若对年老、青春已逝的父母,\hfill\textcolor{gray}{\footnotesize \textbf{124}} \\
堪能却不赡养,当知他是贱民。


若对父母或兄弟、姐妹、岳母\hfill\textcolor{gray}{\footnotesize \textbf{125}} \\
加以伤害,以言语恼害,当知他是贱民。


若被问及义利,却教授非义,\hfill\textcolor{gray}{\footnotesize \textbf{126}} \\
以隐语商讨,当知他是贱民。


若作恶业后,希望「他莫发现我」,\hfill\textcolor{gray}{\footnotesize \textbf{127}} \\
行事隐密,当知他是贱民。


若到了别人家,享用了净妙的食物,\hfill\textcolor{gray}{\footnotesize \textbf{128}} \\
却不敬待来者,当知他是贱民。


若对婆罗门、沙门或其他的乞食者\hfill\textcolor{gray}{\footnotesize \textbf{129}} \\
以妄语欺骗,当知他是贱民。


若食时已到,对婆罗门或沙门\hfill\textcolor{gray}{\footnotesize \textbf{130}} \\
以言语恼害,且不布施,当知他是贱民。


若于此出言不善,以愚弄裹挟,\hfill\textcolor{gray}{\footnotesize \textbf{131}} \\
企求某物,当知他是贱民。


若赞叹自己,且蔑视他人,\hfill\textcolor{gray}{\footnotesize \textbf{132}} \\
以自身的慢而下劣,当知他是贱民。


恼害、贪婪,恶欲、悭吝、狡诈,\hfill\textcolor{gray}{\footnotesize \textbf{133}} \\
无惭、无愧,当知他是贱民。


若谤骂佛陀或他的弟子、\hfill\textcolor{gray}{\footnotesize \textbf{134}} \\
游行者或在家人,当知他是贱民。


若实非阿罗汉,却自称阿罗汉,\hfill\textcolor{gray}{\footnotesize \textbf{135}} \\
在俱梵的世间作贼,他是最下劣的贱民,\\
上述这些贱民,我已向你阐明。


不由出生而成贱民,不由出生而成婆罗门,\hfill\textcolor{gray}{\footnotesize \textbf{136}} \\
由业而成贱民,由业而成婆罗门。


你们也可以此了知,好比我的这例子:\hfill\textcolor{gray}{\footnotesize \textbf{137}} \\
旃陀罗之子、贱民,以摩登伽著名者。


这摩登伽获得了极难得的至高的声誉,\hfill\textcolor{gray}{\footnotesize \textbf{138}} \\
许多刹帝利、婆罗门前往侍奉他。


他登上了天乘、离尘的大路,\hfill\textcolor{gray}{\footnotesize \textbf{139}} \\
弃绝了对爱欲的贪染,便至梵界,\\
出身不能遮止他投生到梵界。


生于唱诵者之家、精通经典的婆罗门,\hfill\textcolor{gray}{\footnotesize \textbf{140}} \\
他们却常在众多恶业中现身。


在现法即遭谴责,且在来世堕恶趣,\hfill\textcolor{gray}{\footnotesize \textbf{141}} \\
出身不能遮止他们堕恶趣或遭谴责。


不由出生而成贱民,不由出生而成婆罗门,\hfill\textcolor{gray}{\footnotesize \textbf{142}} \\
由业而成贱民,由业而成婆罗门。


如是说已,事火婆罗豆婆遮婆罗门对世尊说:「希有!乔达摩君!……从今起,尽寿命,请乔达摩君受持我皈依为优婆塞!」


\section{慈经}


在证得寂静境地后,善巧于义利者应当\hfill\textcolor{gray}{\footnotesize \textbf{143}} \\
堪能、正直、极正直,且应易语、柔和、不傲慢,


知足,易养,少事务,生活简朴,\hfill\textcolor{gray}{\footnotesize \textbf{144}} \\
诸根寂静,贤明,不鲁莽,不贪求于俗家。


且不应做其他智者会谴责的任何小事!\hfill\textcolor{gray}{\footnotesize \textbf{145}} \\
愿他们快乐、安稳!愿一切有情幸福!


举凡是呼吸的生命,弱者或强者,皆无遗漏,\hfill\textcolor{gray}{\footnotesize \textbf{146}} \\
长者或大者、中者、短者、细者、粗者,


可见者或不可见者,住于远方或非远方,\hfill\textcolor{gray}{\footnotesize \textbf{147}} \\
已出生者、将出生者,愿一切有情幸福!


愿无人欺骗他人!愿不在任何场合轻贱任何人!\hfill\textcolor{gray}{\footnotesize \textbf{148}} \\
愿不以忿怒、嗔恚想而希望彼此受苦!


好比母亲对自己的孩子,会以生命保护独子,\hfill\textcolor{gray}{\footnotesize \textbf{149}} \\
如是,也应对一切生命培育无量的心意!


且在所有世间,应培育无量的慈意!\hfill\textcolor{gray}{\footnotesize \textbf{150}} \\
对上方、下方及四旁,无障碍、无怨恨、无敌对。


站着、走着、坐着或躺着,只要离于睡眠,\hfill\textcolor{gray}{\footnotesize \textbf{151}} \\
就应决意此念,他们说这就是此世的梵住。


且不再执取见,具戒,具足知见,\hfill\textcolor{gray}{\footnotesize \textbf{152}} \\
调伏了对爱欲的贪求,他就决不再入母胎。


\section{雪山经}

「今天是十五布萨日,」七岳夜叉说,「圣洁的夜晚降临,\hfill\textcolor{gray}{\footnotesize \textbf{153}} \\
「噫!我们去见乔达摩,享有盛名的大师!」


「如此之人的心意,」雪山夜叉说,「是否善待一切生命?\hfill\textcolor{gray}{\footnotesize \textbf{154}} \\
「于诸可意和不可意,他的思惟是否受控?」


「如此之人的心意,」七岳夜叉说,「善待一切生命,\hfill\textcolor{gray}{\footnotesize \textbf{155}} \\
「并且于诸可意和不可意,他的思惟受控。」


「是否不取不与物?」雪山夜叉说,「是否于生命自制?\hfill\textcolor{gray}{\footnotesize \textbf{156}} \\
「是否离于放逸?是否不疏忽禅那?」


「他不取不与物,」七岳夜叉说,「并于生命自制,\hfill\textcolor{gray}{\footnotesize \textbf{157}} \\
「并且离于放逸,佛陀不疏忽禅那。」


「是否不妄语?」雪山夜叉说,「是否言路不粗鲁?\hfill\textcolor{gray}{\footnotesize \textbf{158}} \\
「是否不语中伤?是否不说绮语?」


「他不妄语,」七岳夜叉说,「且言路不粗鲁,\hfill\textcolor{gray}{\footnotesize \textbf{159}} \\
「并且不语中伤,以智慧讲说义利。」


「是否不味著爱欲?」雪山夜叉说,「是否心不污浊?\hfill\textcolor{gray}{\footnotesize \textbf{160}} \\
「是否超越愚痴?是否于法具眼?」


「他不味著爱欲,」七岳夜叉说,「且心不污浊,\hfill\textcolor{gray}{\footnotesize \textbf{161}} \\
「超越一切愚痴,佛陀于法具眼。」


「是否具足明?」雪山夜叉说,「是否行为清净?\hfill\textcolor{gray}{\footnotesize \textbf{162}} \\
「是否诸漏已尽?是否无有再有?」


「既具足明,」七岳夜叉说,「且行为清净,\hfill\textcolor{gray}{\footnotesize \textbf{163}} \\
「一切漏已尽,他无有再有。」


「对具足业与言路的牟尼的心,以及\hfill\textcolor{gray}{\footnotesize \textbf{164}} \\
「明行足,你如法地赞叹他。」


「对具足业与言路的牟尼的心,以及\hfill\textcolor{gray}{\footnotesize \textbf{165}} \\
「明行足,你如法地随喜。


「对具足业与言路的牟尼的心,以及\hfill\textcolor{gray}{\footnotesize \textbf{166}} \\
「明行足,噫!我们去见乔达摩!」


「具羚羊之腿肚者,瘦削的英雄,少食而无贪求,\hfill\textcolor{gray}{\footnotesize \textbf{167}} \\
「在林中禅修的牟尼,来!我们去见乔达摩!


「如同狮子,龙象独行,不关切爱欲,\hfill\textcolor{gray}{\footnotesize \textbf{168}} \\
「去到后,我们问问解脱死亡之网!」


「宣说者,转起者,已度一切法者,\hfill\textcolor{gray}{\footnotesize \textbf{169}} \\
「佛陀,超越敌对怖畏者,我们问问乔达摩!」


「于何世间生起?」雪山夜叉说,「于何产生亲密?\hfill\textcolor{gray}{\footnotesize \textbf{170}} \\
「取著何者而有世间?于何世间遘难?」


「于六世间生起,雪山!」世尊说,「于六产生亲密,\hfill\textcolor{gray}{\footnotesize \textbf{171}} \\
「唯取著六(而有世间),于六世间遘难。」


「什么是这世间在其中遘难的取著?\hfill\textcolor{gray}{\footnotesize \textbf{172}} \\
「问及出离,请说如何从苦解脱!」


「种种五欲被宣告于世间,意为第六,\hfill\textcolor{gray}{\footnotesize \textbf{173}} \\
「除去此中的欲,如是从苦解脱。


「此即世间的出离,已如实对你们宣说,\hfill\textcolor{gray}{\footnotesize \textbf{174}} \\
「我将对你们宣说此,如是从苦解脱。」


「谁于此度过暴流?谁于此度过海洋?\hfill\textcolor{gray}{\footnotesize \textbf{175}} \\
「于无落足、无攀援之深,谁不沉没?」


「始终具足戒,具慧,善等持,\hfill\textcolor{gray}{\footnotesize \textbf{176}} \\
「内省,具念,他度过难度的暴流。


「戒离爱欲想,越过一切结缚,\hfill\textcolor{gray}{\footnotesize \textbf{177}} \\
「灭尽喜与有,他不沉没于深。」


「深慧,见微妙义,无所牵绊,不取著爱欲与有,\hfill\textcolor{gray}{\footnotesize \textbf{178}} \\
「你们看这解脱于一切处、行走在天路上的大仙!


「享有盛名,见微妙义,给予智慧,不取著于欲执,\hfill\textcolor{gray}{\footnotesize \textbf{179}} \\
「你们看这知晓一切、善慧、行走在圣路上的大仙!


「我们今天确实有好的所见、好的早晨、好的起身,\hfill\textcolor{gray}{\footnotesize \textbf{180}} \\
「因为我们见到了已度过暴流、无漏的等正觉。


「这一千个夜叉具有神变、具有名望,\hfill\textcolor{gray}{\footnotesize \textbf{181}} \\
「全都皈依你,你是我们无上的大师。


「我们将从村到村、从山到山地游行,\hfill\textcolor{gray}{\footnotesize \textbf{182}} \\
「礼敬着等正觉,以及法的善法性。」


\section{旷野经}

如是我闻。一时世尊住在旷野中旷野夜叉的居处。于是,旷野夜叉往世尊处走去,走到后,对世尊说:「出来!沙门!」「善哉!朋友!」世尊便出来。「进去!沙门!」「善哉!朋友!」世尊便进去。第二次……第三次,旷野夜叉对世尊说:「出来!沙门!」「善哉!朋友!」世尊便出来。「进去!沙门!」「善哉!朋友!」世尊便进去。第四次,旷野夜叉对世尊说:「出来!沙门!」「朋友!我不会再出来,你该做什么就做吧!」


「沙门!我将问你问题,如果你不能向我解答,我就扰乱你的心识,撕碎你的心脏,捉住脚抛到恒河对岸去。」「朋友!我实不见在这俱有天、魔、梵、沙门婆罗门、天人的人世间,有人能扰乱我的心识、撕碎心脏、捉住脚抛到恒河对岸去的,但是,朋友!问你所愿吧!」于是,旷野夜叉以偈颂对世尊说:


「什么是人在此世最上的财富?什么善加习行能带来乐?\hfill\textcolor{gray}{\footnotesize \textbf{183}} \\
「味中究竟什么更甜?他们说如何生活才是最上的生活?」


「信是人在此世最上的财富,法的善加习行能带来乐,\hfill\textcolor{gray}{\footnotesize \textbf{184}} \\
「味中真实更甜,他们说有慧的生活才是最上的生活。」


「如何度过暴流?如何度过大海?\hfill\textcolor{gray}{\footnotesize \textbf{185}} \\
「如何克服苦?如何得净化?」


「以信度过暴流,以不放逸于大海,\hfill\textcolor{gray}{\footnotesize \textbf{186}} \\
「以精进克服苦,以慧得净化。」


「如何获得智慧?如何得到财产?\hfill\textcolor{gray}{\footnotesize \textbf{187}} \\
「如何成就名声?如何交结朋友?\\
「从此世到他世,如何死后无忧?」


「置信于诸阿罗汉的得达涅槃之法,\hfill\textcolor{gray}{\footnotesize \textbf{188}} \\
「愿欲听闻、不放逸的明眼人获得智慧。


「行事得体、负责的奋起者得到财产,\hfill\textcolor{gray}{\footnotesize \textbf{189}} \\
「以真实成就名声,施予者交结朋友。


「对于有信的寻求居家者,存在这四种法:\hfill\textcolor{gray}{\footnotesize \textbf{190}} \\
「真实、如法、坚定、舍,他必死后无忧。


「去!你也可以问问其他各个沙门婆罗门,\hfill\textcolor{gray}{\footnotesize \textbf{191}} \\
「是否于此有比真实、调御、舍、忍辱更好的?」


「为何现在我还要去问其他各个沙门婆罗门?\hfill\textcolor{gray}{\footnotesize \textbf{192}} \\
「今天,我了知了来世的义利。


「佛陀确是为了我的义利,才来旷野居住,\hfill\textcolor{gray}{\footnotesize \textbf{193}} \\
「今天,我了知了何处布施有大果报。


「我将从村到村、从城到城地游行,\hfill\textcolor{gray}{\footnotesize \textbf{194}} \\
「礼敬着等正觉,以及法的善法性。」


\section{胜利经}


走着或是站着,坐着或是躺着,\hfill\textcolor{gray}{\footnotesize \textbf{195}} \\
弯曲或伸展,这是身体的运动。


骨腱相连,涂以皮肉,\hfill\textcolor{gray}{\footnotesize \textbf{196}} \\
身体为表皮所遮蔽,不能如实得见。


充以小肠,充以胃,肝脏、膀胱,\hfill\textcolor{gray}{\footnotesize \textbf{197}} \\
与心、肺、肾、脾,


以及涕、唾、汗、脂肪,\hfill\textcolor{gray}{\footnotesize \textbf{198}} \\
与血、关节滑液、胆汁、膏。

另外,从其九孔总有不净流出,\hfill\textcolor{gray}{\footnotesize \textbf{199}} \\
眼眵从眼中,耳垢从耳中,


涕从鼻中,有时从口吐出\hfill\textcolor{gray}{\footnotesize \textbf{200}} \\
胆汁和痰,汗污则从身上。


另外,其头颅充满了脑,\hfill\textcolor{gray}{\footnotesize \textbf{201}} \\
出于无明,愚人以净思量它。


而当其死去,躺着,膨胀,青瘀,\hfill\textcolor{gray}{\footnotesize \textbf{202}} \\
被丢弃在塚间,亲戚们不再关切。


狗、豺、狼、蛆都来啖食他,\hfill\textcolor{gray}{\footnotesize \textbf{203}} \\
乌鸦、秃鹫也来啄食,还有其它生类。


听闻了佛语,俱智的比丘于此\hfill\textcolor{gray}{\footnotesize \textbf{204}} \\
遍知了它,因为他如实地得见。


此即如彼,彼即如此,\hfill\textcolor{gray}{\footnotesize \textbf{205}} \\
于内在及外在,他能于身离欲。


这俱智的比丘于此已离欲贪,\hfill\textcolor{gray}{\footnotesize \textbf{206}} \\
得证不死、寂静、涅槃、不殁的境地。


这两足者不净,恶臭,备受爱护,\hfill\textcolor{gray}{\footnotesize \textbf{207}} \\
充满种种腐臭,散发于处处。


若以这样的身体自视高标,\hfill\textcolor{gray}{\footnotesize \textbf{208}} \\
或鄙视他人,除无知见,还能为何?


\section{牟尼经}

从亲密生出怖畏,从居所生出尘垢,\hfill\textcolor{gray}{\footnotesize \textbf{209}} \\
无居所,无亲密,这才是牟尼知见。


若切断了已生者,不再培植正生者,不再滋益它,\hfill\textcolor{gray}{\footnotesize \textbf{210}} \\
他们说他是独行的牟尼,这大仙得见寂静的境地。


省察了依处,碾碎了种子,不再以爱执滋益它,\hfill\textcolor{gray}{\footnotesize \textbf{211}} \\
他实是见生之尽头的牟尼,舍弃了寻思,不可得名。


了知了一切住处,也不欲求其中的某个,\hfill\textcolor{gray}{\footnotesize \textbf{212}} \\
他实是牟尼,离贪而无求,不再追逐,因为已到彼岸。


征服一切,知晓一切,善慧,于一切法不染,\hfill\textcolor{gray}{\footnotesize \textbf{213}} \\
舍弃一切,于渴爱尽处解脱,智者们知晓他实是牟尼。


有慧力,具足戒行,等持,乐于禅那,具念,\hfill\textcolor{gray}{\footnotesize \textbf{214}} \\
解脱于执著,无荒秽,无漏,智者们知晓他实是牟尼。


独行的牟尼不放逸,不为毁誉所动,\hfill\textcolor{gray}{\footnotesize \textbf{215}} \\
好比狮子不惊怖于声响,好比清风不羁绊于罗网,好比莲花不著于水,\\
引领他人,而非被他人引领,智者们知晓他实是牟尼。


当别人极端地说话时,如浴场的柱子般不动,\hfill\textcolor{gray}{\footnotesize \textbf{216}} \\
离贪,善等持诸根,智者们知晓他实是牟尼。


坚定,如梭子般正直,嫌厌于恶业,\hfill\textcolor{gray}{\footnotesize \textbf{217}} \\
审视邪正,智者们知晓他实是牟尼。


自制,当少年及中年时不作恶,克己,牟尼\hfill\textcolor{gray}{\footnotesize \textbf{218}} \\
不可使怒,他不激怒任何人,智者们知晓他实是牟尼。


靠他人布施的活命者,无论从上、中或从余处得到食物,\hfill\textcolor{gray}{\footnotesize \textbf{219}} \\
不去赞美,也不贬低,智者们知晓他实是牟尼。


牟尼游行,戒离交媾,青春之时不束缚于任何处,\hfill\textcolor{gray}{\footnotesize \textbf{220}} \\
戒离㤭慢与放逸,解脱,智者们知晓他实是牟尼。


了知世间,见第一义,度过暴流、大海而如如,\hfill\textcolor{gray}{\footnotesize \textbf{221}} \\
切断系缚,无所依,无漏,智者们知晓他实是牟尼。


两者不同,住处与行为差远,在家人养育妻子,善行者无我所,\hfill\textcolor{gray}{\footnotesize \textbf{222}} \\
在家人伤害别的生命而不自制,克己的牟尼总是保护生命。


好比青颈的孔雀飞在空中,永远也赶不上天鹅的速度,\hfill\textcolor{gray}{\footnotesize \textbf{223}} \\
如是,在家人也无法效仿比丘,那在林中独处禅修的牟尼。
