\section{空王学童问}

\subsection\*{\textbf{1123} {\footnotesize 〔PTS 1116〕}}

\textbf{「我两次问了释迦,」尊者空王说,「具眼者没有对我解答,\\}
\textbf{「而我听说:『要到第三次,天仙便会解答。』}

\begin{enumerate}\item 这里,\textbf{我两次},因为他先前在「阿耆多经」及「低舍弥勒经」的最后,两次问了世尊,而世尊为等待他的根成熟,便未回答,因此说「我两次问了释迦」。\textbf{而我听说:「要到第三次,天仙便会解答」},即要到第三次被如法问及,作为清净天的仙人、世尊、正等正觉便会解答,我如是听说。据说,他在乔陀婆利岸边便如是听说,因此说「而我听说便会解答」。\end{enumerate}

\subsection\*{\textbf{1124} {\footnotesize 〔PTS 1117〕}}

\textbf{「此世间、他世间、俱有天的梵世间,\\}
\textbf{「不知道您,享有名望的乔达摩的见。\footnote{不知道 \textit{nābhijānāti} 的主语应是上半颂中作主格的「此世、他世、俱有天的梵世间」,而PTS 本作「我不知道 \textit{nābhijānāmi}」,且将上半颂纳入引号,则意思变为「我不知道您对这些世间的见」,二英译本正是如此,义注于此无文,若结合下二颂,似以 PTS 本为佳,兹存缅文本之旧,聊备一说。}}

\begin{enumerate}\item \textbf{此世间},即人世间。\textbf{他世间},即除此以外的其它。\textbf{俱有天},即除了梵世间以外的其它投生之天、共许之天相关者,或者,这「俱有天的梵世间」只是「俱有天的世间」等方法的例子,以此当知一切如上所说品类的世间。\end{enumerate}

\subsection\*{\textbf{1125} {\footnotesize 〔PTS 1118〕}}

\textbf{「如是具殊胜之见者,我带着问题前来,\\}
\textbf{「如何观察世间,死王便不得见他?」}

\begin{enumerate}\item \textbf{如是具殊胜之见者},即如是具最上之见者,显示他能看见俱有天的世间的意乐、胜解、趣向、归宿等。\end{enumerate}

\subsection\*{\textbf{1126} {\footnotesize 〔PTS 1119〕}}

\textbf{「你应从空观察世间!空王!始终具念,\\}
\textbf{「除去了我随见,如是便越过死亡,\\}
\textbf{「如是观察世间者,死王不得见他。」}

\begin{enumerate}\item \textbf{你应从空观察世间},即你应以了解转起之无主,或以随观诸行空无等两种方式,从空看世间!\textbf{除去了我随见},即拔除了有身见。其余一切处皆自明。
\item 如是,世尊仍以阿罗汉为顶点开示了此经。当开示终了,如前所述,而有法的现观。\end{enumerate}

