\section{孙陀利迦婆罗豆婆遮经}

\textbf{如是我闻。一时世尊住㤭萨罗孙陀利迦河的岸边。尔时,孙陀利迦婆罗豆婆遮婆罗门在孙陀利迦河的岸边献火供、事火祭。然后,孙陀利迦婆罗豆婆遮婆罗门献了火供、事了火祭,便从坐起,观察周围四方:「谁当享用这祭品的残留?」孙陀利迦婆罗豆婆遮婆罗门看到世尊在不远处的某棵树下蒙头而坐,看到后,左手拿了祭品的残留,右手拿了水壶,往世尊处走去。}

\begin{enumerate}\item \textbf{祭饼经} \textit{Pūraḷāsasutta}\footnote{这是义注中的经题。}。缘起为何?世尊在饭后义务的终了,以佛眼观察世间,便见到孙陀利迦婆罗豆婆遮婆罗门具足阿罗汉的近依,且了知到「当我到达那里,将发生谈话,随后,在谈话终了,听闻了法的开示,这婆罗门将出家而圆满阿罗汉」,便去到那里,引起谈话,说了此经。
\item 这里,\textbf{如是我闻}等是结集者的话,\textbf{先生是何出身}等是这婆罗门的,\textbf{我不是婆罗门}等是世尊的,这一切汇集后,被称为「祭饼经」。这里,当知与已述相同的仍如所述之法,我们只解释未述者,且不涉及语义自明的词句。
\item \textbf{㤭萨罗},即名为㤭萨罗的国中众王子,他们所居的一方国土便俗称作「㤭萨罗」——即于此㤭萨罗国。而有些人解释说,因为先前「大喝王子」见到种种舞蹈都未露丝毫微笑,国王听闻后,便命令「若有人能让我的孩子笑出来,我就用一切璎珞装饰他」。随后,众人都丢开了犁,聚集起来。但七年多来,这些人表演了种种戏耍等,都未能让他笑出来。随后,帝释便派遣天舞者,他表演天舞后,便让王子笑了出来。于是,这些人便朝各自的住处离开,朋友亲人等在路上看见他们,就欢迎说:「您还好吧,先生!您还好吧,先生!」所以由「好 \textit{kusala}」字的发音,这地方便被称为「㤭萨罗 \textit{Kosala}」。\textbf{孙陀利迦河的岸边},即名为孙陀利迦之河的岸边。
\item \textbf{尔时},即世尊欲调伏此婆罗门而前往,在此岸边蒙头,在树下以被称为坐的威仪住而住之时。\textbf{孙陀利迦婆罗豆婆遮婆罗门},此婆罗门在此河的岸边居住,并献火供,而其族为「婆罗豆婆遮」,故而如是得称。\textbf{献火供},即将祭品投入火中燃烧。\textbf{事火祭},即以洒扫、涂膏、供奉等承事火处。
\item \textbf{谁当享用这祭品的残留},据说,这婆罗门献了火供,看到残留的粥,便想:「在火中投入的粥先由大梵享用,却仍有残留,若我布施给从梵天的口中出生的婆罗门,则我的孩子和我父亲也会高兴,且趣向梵界之路会极清净,噫!我去寻找婆罗门!」随后,为遇见婆罗门,\textbf{便从坐起,观察四方}:「谁当享用这祭品的残留?」
\item \textbf{在某棵树下},即在此密林中最胜的树下。\textbf{蒙头},即连头一起披覆身体。但世尊为什么这样做?难道有着被称为那罗衍那之力\footnote{那罗衍那之力 \textit{Nārāyana-bala}:菩提比丘注 1372 云,据\textbf{分别论义注},那罗衍那之力等于一千俱胝大象和一万俱胝人的力气。},尚不能抵御落雪和寒风吗?这自有原因。因为诸佛不全是为了料理身体而这么做,而是世尊想到「当婆罗门来时,我将揭开头,婆罗门见了我,将发起谈话,然后随着谈话,我将开示法」,为了发起谈话而这么做。
\item \textbf{看到后,左手……走去},据说,这婆罗门见到世尊,便作婆罗门想「他蒙了头,整夜精勤,施了这供水后,我再施祭品的残留」,便走去。\end{enumerate}

\textbf{于是,随着孙陀利迦婆罗豆婆遮婆罗门的脚步声,世尊便揭开了头。然后,孙陀利迦婆罗豆婆遮婆罗门想「这先生是光头,这先生是秃头」,便想从此回去。然后,孙陀利迦婆罗豆婆遮婆罗门便想:「于此,有些婆罗门也是光头,我何不前去问问出身?」然后,孙陀利迦婆罗豆婆遮婆罗门往世尊处走去,走到后,对世尊说:「先生是何出身?」}

\begin{enumerate}\item \textbf{这先生是光头,这先生是秃头},甫一揭开头,他便看见发尖而说「光头」。随后,经仔细观察,连分毫也不见,便轻蔑地说「秃头」。因为婆罗门的见便是如此。\textbf{从此},即从所站立而见之处。\textbf{也是光头},即以某些原因也剃了光头。\end{enumerate}

\textbf{于是,世尊以偈颂对孙陀利迦婆罗豆婆遮婆罗门说:}

\subsection\*{\textbf{458} {\footnotesize 〔PTS 455〕}}

\textbf{「我既非婆罗门,亦非王子,不是吠舍或是任何其他,\\}
\textbf{「遍知了凡夫们的种姓,无所牵绊,我以考量在世间游行。}

\begin{enumerate}\item \textbf{我既非婆罗门},此中的「非 \textit{na}」字为遮止,「既 \textit{no}」字为强调,如同\begin{quoting}都不能 \textit{na no}(与如来)等同。(经集第 226 颂)\end{quoting}等处,以此显示我绝非婆罗门。\textbf{亦非王子},即不是刹帝利。\textbf{或是任何其他},即我不是任何其他首陀罗或旃陀罗。如是便完整地拒斥了出身论的攻击。为什么?因为如同众流汇入大海,众族姓子既已出家,便舍弃了先前的姓名、种姓。且此处有「般诃罗陀经\footnote{般诃罗陀经 \textit{Pahārādasutta}:即\textbf{增支部}第 8:19 经。}」为证。
\item 如是拒斥了出身论后,为了如实揭露自身,便说「\textbf{遍知了凡夫们的种姓,无所牵绊,我以考量在世间游行}」。设问:如何遍知种姓?因为世尊以三遍知而遍知五蕴,而当此等遍知时,即遍知了种姓。以无任何贪等,他无所牵绊。考量、知晓已,以与智随转的身业等而行,因此说「遍知了……在世间游行」。考量即是慧,且他以此而行,因此说「我以考量在世间游行」,因韵律而作短音\footnote{这是说颂中的「考量 \textit{manta}」原本应作 mantā。关于 mantā,义注给出了两种解释,一作动词,一作名词,在此不详述。}。\end{enumerate}

\subsection\*{\textbf{459} {\footnotesize 〔PTS 456〕}}

\textbf{「穿著僧伽梨,我无家而行,剃去头发,内在寂静,\\}
\textbf{「于此不著于世人,婆罗门!你问我种姓的问题不合适。」}

\begin{enumerate}\item 如是揭露了自身,现在,为向婆罗门提出诘难「都已见到如是明显的特相,你还不知晓什么该问、什么不该问」,说了此颂。且此中,以连缀截断之义,「僧伽梨」意指三衣,以穿著彼等为\textbf{穿著僧伽梨}。\textbf{无家},意即无渴爱。虽然世尊所住之家有祇园的大香房、树圆亭、㤭赏弥寮、旃檀亭等多种,但与此无涉。\textbf{剃去头发},即脱去头发,即是说翦除须发。\textbf{内在寂静},即心极止息了热恼,或心有守护。
\item \textbf{于此不著于世人},由舍弃对资助的爱执,不著于众人、不交际、完全地独处。\textbf{婆罗门……不合适},即我如是穿著僧伽梨……于此不著于世人,婆罗门!我于过去既已出家,你问这原本的姓名、种姓之问便不恰当。\end{enumerate}

\subsection\*{\textbf{460} {\footnotesize 〔PTS 457\textit{a-b}〕}}

\textbf{「先生!婆罗门与婆罗门一起,总是问:『您是婆罗门吗?』」}

\begin{enumerate}\item 如是说已,婆罗门为摆脱诘难,说了此句。这即是说:先生!我问的并非不合适。因为在我们婆罗门的集会\footnote{集会 \textit{samaya}:或可译作「教义」。}中,婆罗门与婆罗门碰到一起,总这样问出身和种姓:「您是婆罗门吗?您是婆罗豆婆遮吗?」\end{enumerate}

\subsection\*{\textbf{461} {\footnotesize 〔PTS 457\textit{c-f}〕}}

\textbf{「因为若你说是婆罗门,而说我非婆罗门,\\}
\textbf{「我就来问问这三句、二十四音节的颂诗。」}

\begin{enumerate}\item 如是说已,世尊为令婆罗门的心柔和,阐明自己熟稔颂诗,说了此颂。其义为:如果你说「我是婆罗门」,\textbf{而说我非婆罗门},那么,\textbf{我就来问问}您,\textbf{这三句、二十四音节的颂诗},请对我说!
\item 且此中,世尊是就作为第一义吠陀的三藏的开卷,就作为第一义婆罗门的一切诸佛所阐明的具足义、具足文的「我皈依佛、我皈依法、我皈依僧」的圣颂诗\footnote{颂诗 \textit{sāvitti/sāvitrī}:本是梨俱吠陀中颂诗的名称,因致敬太阳 \textit{savitṛ} 而得名,也称 gāyatrī,不过这里,义注说指的是「我皈依佛、我皈依法、我皈依僧」这三句,在巴利中正好是二十四音节。}而问的。因为要是婆罗门说其它的,世尊便会对他说「婆罗门!这在圣律中不被称为颂诗」,在显示其非坚实后,唯令他住立于此(教法)。\end{enumerate}

\subsection\*{\textbf{462} {\footnotesize 〔PTS 458\textit{a-c}〕}}

\textbf{「以何依据,仙人、世人、刹帝利、婆罗门向诸天\\}
\textbf{「举行各种献牲,在此世间?」}

\begin{enumerate}\item 然而,婆罗门在听到「我就来问问三句、二十四音节的颂诗」,这成就了自身的教法、有着颂诗的相与文、以梵音说出的话后,得出结论「确实,这沙门在婆罗门教中已得究竟,而我以无智轻蔑道『他非婆罗门』,他实是好样的、通晓颂诗的婆罗门,噫!我来问他献牲的方法和供养的方法」,为问此义,说了这三句不等的偈颂\footnote{三句不等的偈颂:指此颂的三句音节数不等。}。
\item 其义为:\textbf{以何依据}、以何意趣、愿求什么,\textbf{仙人、刹帝利、婆罗门}与其他\textbf{世人}为了\textbf{诸天}的义利举行献牲?Yañña-m-akappayiṃsu 中的 m 字作词的连接。\textbf{举行},即安排、实行。\textbf{各种},即众多,以饮食布施等类而有许多品类,或者,各种仙人、世人、刹帝利、婆罗门以何依据举行献牲?他们的业如何成功?他以此意趣而问。\end{enumerate}

\subsection\*{\textbf{463} {\footnotesize 〔PTS 458\textit{d-e}〕}}

\textbf{「若到达边际者、通达诸明者在献牲时,能从中得到祭品,我说,他便成功。」}

\begin{enumerate}\item 于是,世尊为向其解释此义,说了这剩余的两句。这里,\textbf{Ya-d-antagū},即 yo antagū,a 字代替 o 字,而 d 字作词的连接,如同 asādhāraṇa-m-aññesan 等处的 m 字一般。
\item 而其义为:\textbf{若}以三遍知到达流转之苦的边际的\textbf{到达边际者},与以四道智之明穿透烦恼而通达的\textbf{通达诸明者},他\textbf{在}仙人、世人、刹帝利、婆罗门中任一的\textbf{献牲时},\textbf{能从}任何被供奉的食物——乃至林中的叶、根、果等——\textbf{中得到祭品},\textbf{我说,他}的这献牲之业\textbf{便成功}、兴盛、得大果报。\end{enumerate}

\subsection\*{\textbf{464} {\footnotesize 〔PTS 459〕}}

\textbf{「确实,这献祭成功,」婆罗门说,「当我们见了这样的通达诸明者,\\}
\textbf{「因为没有得见像你这样的人,其他人便享用了祭饼。」}

\begin{enumerate}\item 于是,婆罗门听闻了世尊这与第一义相应而甚深、具足极甜美的发音与淡定之声的开示,为尊敬其身成就之清净与一切功德的成就,生起喜乐,说了此颂。这里,「\textbf{婆罗门说}」是结集者的话,其余则是婆罗门的。
\item 其义为:\textbf{确实},我的\textbf{这献祭成功},今天,这所施会成功、兴盛、得大果报,\textbf{当我们见了这样的通达诸明者},因为我们见了像您这样的通达诸明者。因为你便是这通达诸明者,而非他人。而此前\textbf{因为没有得见像你这样}通达诸明、到达边际的人,\textbf{其他人便享用了}在我们这样的献牲中准备的\textbf{祭饼}、供品和饼。\end{enumerate}

\subsection\*{\textbf{465} {\footnotesize 〔PTS 460〕}}

\textbf{「所以,婆罗门!你在此希求义利,上前来问!\\}
\textbf{「兴许于此,能发现寂静、无烟、无患、无待的善慧者。」}

\begin{enumerate}\item 随后,世尊在确认了婆罗门已对自己净喜、准备接受话语之后,为让他完全明了,而欲以种种方式阐明应供者,说了此颂。其义为:因为你对我净喜,\textbf{所以},为显示自身,而说:\textbf{婆罗门!你在此,上前来问}!现在,在此之前的「希求义利」一词应与后句相连:\textbf{希求义利,兴许于此},即很可能就立于此处,或于此教法,你\textbf{能发现}、能得到、证得随适于此义利希求之相的以止息烦恼之火而\textbf{寂静}、以消逝忿怒之烟而\textbf{无烟}、以无有苦而\textbf{无患}、以无各种希求而\textbf{无待}的最上慧、漏尽应供的\textbf{善慧者}。
\item 或者,当知此中也可如是连结:因为你对我净喜,所以,你,婆罗门!当在此希求义利时,请上前来问寂静、无烟、无患、无待——为显示自身而说。当如是发问时,兴许于此,能发现漏尽应供的善慧者。\end{enumerate}

\subsection\*{\textbf{466} {\footnotesize 〔PTS 461〕}}

\textbf{「我乐于献牲,乔达摩君!欲行献牲,却不知晓,\\}
\textbf{「请您教授我!何处献祭能成功?请对我说!」}

\begin{enumerate}\item 于是,婆罗门如所教授而行,对世尊说了此颂。这里,\textbf{献牲}、祭祀、布施的之义相同。所以,我乐于布施,正因此乐于布施而欲行布施,但我却不知晓,\textbf{请您教授}如是无知的\textbf{我}!且教授时,\textbf{请}以明显的方法\textbf{对我说}:\textbf{何处献祭能成功}?当知如是连结此中之义。文本(何处献祭)也作「如何献祭 \textit{yathāhutaṃ}」。\end{enumerate}

\subsection\*{\textbf{467} {\footnotesize 〔PTS 462〕}}

\textbf{「既然如此,你,婆罗门!请注意听!我将对你说法:}

\begin{enumerate}\item 于是,世尊欲对他说,便说了「既然如此……我将对你说法」。且为教授已注意倾听的他,先说了此颂。\end{enumerate}

\textbf{「莫问出身,当问行为,从薪实能生火,\\}
\textbf{「卑贱之家者也可成为坚定、高贵、以惭禁止的牟尼。}

\begin{enumerate}\item 这里,\textbf{莫问出身},即如果你期望祭祀成功、布施得大果报,不应问出身。因为出身不是检视应供者的原因。\textbf{当问行为},而当问戒等德的行为。因为这才是检视应供者的原因。
\item 现在,为向他阐明此义,便举例说「\textbf{从薪实能生火}」。此处的意趣为:于此,从薪生火,不是只从沙罗树等薪木所生者起火的作用,从狗槽等薪木所生者不起,而是由自身具足火焰等德而起。如是,不是只从婆罗门家族等所生者便成应供者,从旃陀罗家族等所生者不成,而是不论\textbf{卑贱之家者}或上等之家者,都\textbf{可成为坚定、以惭禁止、高贵}的漏尽\textbf{牟尼},以成就此坚定与惭为首的德而成具有出身的最上应供者。因为他以坚定保持诸德,以惭禁止过失。如说:\begin{quoting}因为惭,诸善人不作恶。\end{quoting}因此我对你说了此颂。这是略说,而详说当依「中部·马刈经」而知。\end{enumerate}

\subsection\*{\textbf{468} {\footnotesize 〔PTS 463〕}}

\textbf{「以真实而调御,具足调御,通达诸明,梵行已立,\\}
\textbf{「希求福德的婆罗门若欲祭祀,应适时给予他祭品。}

\begin{enumerate}\item 如是,世尊在教授了四种姓的清净后,现在,为显示在何处献祭成功及如何献祭成功之义,说了这初颂。这里,\textbf{以真实},即以第一义真实,因为证得此而为\textbf{调御},故说「以真实而调御」。\textbf{具足调御},即具足根的调御。\textbf{通达诸明},即或以诸明到达烦恼的边际,或到达诸明边际的第四道智。\textbf{梵行已立},即由不需再住而已住于道梵行\footnote{道梵行,见\textbf{吉祥经}第 270 颂注。}。\textbf{应适时给予他祭品},即注意到自己有所施之时和他人现前之时后,当在此时给予\footnote{给予 \textit{pavecchati}:一般认为 payacchati > payecchati > pavecchati,不同于义注给出的「引介 \textit{paveseti}」,详见 Norman 及 PED。}、引介、供与这样的应供者以所施。\end{enumerate}

\subsection\*{\textbf{469} {\footnotesize 〔PTS 464〕}}

\textbf{「舍弃了爱欲,无家而行,善加自制,如梭子般正直,\\}
\textbf{「希求福德的婆罗门若欲祭祀,应适时给予他们祭品。}

\begin{enumerate}\item \textbf{爱欲},即物欲和烦恼欲\footnote{两种爱欲,见\textbf{犀牛角经}第 50 颂注。}。\end{enumerate}

\subsection\*{\textbf{470} {\footnotesize 〔PTS 465〕}}

\textbf{「离于贪染,善等持诸根,如月亮解脱于罗睺的束缚,\\}
\textbf{「希求福德的婆罗门若欲祭祀,应适时给予他们祭品。}

\begin{enumerate}\item \textbf{善等持诸根},即善加等持诸根,即是说诸根不散乱。\textbf{如月亮解脱于罗睺的束缚},即好比月亮从罗睺的束缚中解脱,如是从烦恼的束缚中解脱,极闪耀和光辉。\end{enumerate}

\subsection\*{\textbf{471} {\footnotesize 〔PTS 466〕}}

\textbf{「无所羁绊地行于世间,始终具念,舍弃了执为我者,\\}
\textbf{「希求福德的婆罗门若欲祭祀,应适时给予他们祭品。}

\begin{enumerate}\item \textbf{执为我者},即以爱、见而执为我者。\end{enumerate}

\subsection\*{\textbf{472} {\footnotesize 〔PTS 467〕}}

\textbf{「舍弃了爱欲,征服而行,他知道生死的边际,\\}
\textbf{「已止息,如池水般清凉,如来应得祭饼。}

\begin{enumerate}\item 从此开始,(世尊)就自身而说。这里,\textbf{舍弃了爱欲},即舍弃了烦恼欲。\textbf{征服而行},即由舍弃了彼等而于物欲征服而行。\textbf{他知道}名为涅槃的\textbf{生死的边际},即以自身的慧力而知。\textbf{如池水般清凉},即如阿耨达池、耳秃池、造车池、六牙池、杜鹃池、曼陀吉尼池、狮崖池等雪山中的七大池,由不为火、日的炎热所触而恒久清凉,由\textbf{已止息}烦恼的热恼故,如其中任一池水般清凉。\end{enumerate}

\subsection\*{\textbf{473} {\footnotesize 〔PTS 468〕}}

\textbf{「与相同者相同,与不同者差远,如来是无尽慧者,\\}
\textbf{「不染于此世或他世,如来应得祭饼。}

\begin{enumerate}\item \textbf{相同},即相等。\textbf{相同者},即毗婆尸等佛。由通达相同故,他们被称为「相同者」。他们在以通达应证的功德上,或应舍弃的过失上没有差别,但他们有时量、寿量、家族、身量、出离、精勤、菩提树与光上的差别。
\item 因为他们最少以四阿僧祇又十万劫来圆满波罗蜜,最多以十六阿僧祇又十万劫,这是他们的\textbf{时量差别}。最少投生在寿量百年之时,最多在寿量十万年之时,这是他们的\textbf{寿量差别}。投生在刹帝利家族或婆罗门家族,这是\textbf{家族差别}。高者有八十八肘量,矮者为十五、十八肘量\footnote{肘 \textit{hattha}:为长度单位,一肘为廿四指节 \textit{aṅgulapabba} 或四分之一寻 \textit{vyāma}。},这是\textbf{身量差别}。以象、马、车、轿等出离,或以空中,因为毗婆尸、拘留孙以马车出离,尸弃、拘那含以象背,毗舍婆以轿,迦叶以空中,释迦牟尼以马背,这是\textbf{出离差别}。从事精勤或七天、半月、一月、二月、三月、四月、五月、六月、一年、二三四五六年,这是\textbf{精勤差别}。或无花果树为菩提树,或榕树等中的某种,这是\textbf{菩提树差别}。他们与一寻、八十寻或无量光相应。这里,一寻光或八十寻光对一切相同,而无量光可至远处或近处,一牛呼、二牛呼、一由旬、数由旬乃至轮围的边界,吉祥佛的身光可至一万轮围。即便如此,对一切佛,唯依于意的思量,希望多远便能到达多远,这是\textbf{光差别}。除了这八种差别,他们在其余以通达应证的功德上,或应舍弃的过失上没有特殊之处,所以说是「相同者」。如是,与这些相同者相同。
\item \textbf{与不同者差远},非相同者为不同者,即辟支佛等其他一切有情,以不等同性而与这些不同者差远。因为众辟支佛即便以跏趺抵跏趺而坐,遍满整个阎浮提,其功德不及一正等正觉者的十六分之一,遑论声闻等?因此说「与不同者差远」,应以「\textbf{如来是}」两词与「差远」相连。\textbf{无尽慧者},即无量慧者。因为与世人的智慧相较,第八者\footnote{第八者 \textit{aṭṭhamaka}:即四双八辈中的须陀洹向。}的智慧更上,与其智慧相较,须陀洹的更上,如是乃至与阿罗汉的智慧相较,辟支佛的智慧更上,而与辟支佛的智慧相较,如来的智慧不应说为「更上」,而应说为「无尽」,因此说「无尽慧者」。
\item \textbf{不染},即不为爱、见之胶合而染。\textbf{此世或他世},即在此世间或在他世间。而此中的连结为:如来与相同者相同,与不同者差远,为什么?因为他是无尽慧者,不染于此世或他世,因此,如来应得祭饼。\end{enumerate}

\subsection\*{\textbf{474} {\footnotesize 〔PTS 469〕}}

\textbf{「伪善、慢不住于他,他离贪,无我所,无待,\\}
\textbf{「去除忿怒,内在寂静,这婆罗门舍弃了忧尘,\\}
\textbf{「如来应得祭饼。}

\begin{enumerate}\item 此颂及其它类此者,当知是为舍弃对于与伪善等过失相应的婆罗门的应供之想而说。这里,\textbf{无我所},即于有情、诸行等已舍弃「这是我的」的执为我之相。\end{enumerate}

\subsection\*{\textbf{475} {\footnotesize 〔PTS 470〕}}

\textbf{「舍弃了意的住处,他没有任何执取,\\}
\textbf{「无取于此世或他世,如来应得祭饼。}

\begin{enumerate}\item \textbf{住处},即爱、见的住处。因为意以此而住于三有,因此称为「意的住处」。或者,因为唯住于此处,不能舍此而行,因此称为「住处」。\textbf{执取},即爱、见,或以两者所执取之法。\textbf{任何},即哪怕少许之量。\textbf{无取},即由无有此等住处、执取而无取于任何法。\end{enumerate}

\subsection\*{\textbf{476} {\footnotesize 〔PTS 471〕}}

\textbf{「等持,他度过暴流,以最高的见了知了法,\\}
\textbf{「漏尽,持最后身,如来应得祭饼。}

\begin{enumerate}\item \textbf{等持},即以道定。\textbf{了知了法},即了知了一切应知之法。\textbf{以最高的见},即以一切知智。\end{enumerate}

\subsection\*{\textbf{477} {\footnotesize 〔PTS 472〕}}

\textbf{「他的有漏与粗砺之语,已熏散、消尽而无存,\\}
\textbf{「他通达诸明,于一切处解脱,如来应得祭饼。}

\begin{enumerate}\item \textbf{有漏},即伴随常见的对有、贪、禅那、欣求的贪染。\textbf{粗砺},即酷虐、粗恶。\textbf{熏散},即烧尽。\textbf{无存},即由熏散与消尽故,而两者应与两者相连\footnote{即「熏散、消尽」应分别与「有漏、粗砺之语」相连。}。\textbf{于一切处},即于一切蕴、处等。\end{enumerate}

\subsection\*{\textbf{478} {\footnotesize 〔PTS 473〕}}

\textbf{「超越执著,他已没有执著,在有慢的有情中,为无慢的有情,\\}
\textbf{「遍知了有田与物之苦,如来应得祭饼。}

\begin{enumerate}\item \textbf{有慢的有情},即以慢而固著者。\textbf{遍知了苦},以三遍知遍知了流转之苦。\textbf{有田与物},即有因与缘,即是说与业、烦恼俱。\end{enumerate}

\subsection\*{\textbf{479} {\footnotesize 〔PTS 474〕}}

\textbf{「不依希望,得见远离,超越他人所知的见,\\}
\textbf{「他没有任何所缘,如来应得祭饼。}

\begin{enumerate}\item \textbf{不依希望},即不跟随渴爱。\textbf{得见远离},即得见涅槃。\textbf{他人所知的},即由他人令知者。\textbf{超越见},即越过六十二种邪见。\textbf{所缘},即缘,即是说再有的原因。\end{enumerate}

\subsection\*{\textbf{480} {\footnotesize 〔PTS 475〕}}

\textbf{「他所体认的上下诸法,已熏散、消尽而无存,\\}
\textbf{「寂静,于取的灭尽解脱,如来应得祭饼。}

\begin{enumerate}\item \textbf{上下},即尊卑、善妙不善妙,或者以外为上,以内为下。\textbf{体认},即以智通达。\textbf{诸法},即蕴、处等法。\textbf{于取的灭尽解脱},即于涅槃,由涅槃为所缘而解脱,以涅槃为所缘而得解脱之义。\end{enumerate}

\subsection\*{\textbf{481} {\footnotesize 〔PTS 476〕}}

\textbf{「得见结缚与生的尽头,他无余除去了贪路,\\}
\textbf{「清净、无过、无垢、无瑕,如来应得祭饼。}

\begin{enumerate}\item \textbf{得见结缚与生的尽头},即得见结缚的尽头与得见生的尽头。且此中,以结缚的尽头来说有余依涅槃界,以生的尽头来说无余依。因为「尽头」是究竟灭尽的正断断的同义语。且此中的鼻音,如 vivekajaṃ pītisukhaṃ 等处一般,未予省略。\textbf{贪路},即贪的所缘,或即是贪。由为恶趣之路故,贪被称为「贪路」,如说「业道 \textit{kammapatha}」。
\item 清净、无过、无垢、无瑕等,由遍净的身正行等为\textbf{清净},由无有被称为「这人类有贪的过失、嗔的过失、痴的过失」者为\textbf{无过},以离八种人的垢秽为\textbf{无垢}\footnote{八种人的垢秽,见\textbf{法句}·垢秽品第 241~243 颂。},以无随烦恼为\textbf{无瑕}。因为被随烦恼所染污者被称为有瑕。或者,由清净而无过,由无过而无垢,由无有外尘而无垢故无瑕。因为有垢即是有瑕。或者,由无垢故不造作罪恶,因此无瑕。因为造作罪恶,由引起伤害故,被称为瑕。\end{enumerate}

\subsection\*{\textbf{482} {\footnotesize 〔PTS 477〕}}

\textbf{「他不随观自身为我,等持、正直、坚定,\\}
\textbf{「他确实无动摇、无荒秽、无疑惑,如来应得祭饼。}

\begin{enumerate}\item \textbf{不随观自身为我},即以智相应的心对自身的诸蕴作毗婆舍那,不见另有名为我者,唯见蕴为量。因无有从真实、坚牢生起的见「我唯以自身觉知我」,故不随观自身为我,而是以慧见诸蕴。以道定\textbf{等持},无有身邪曲等为\textbf{正直},不为世间法所动摇为\textbf{坚定},以无有被称为渴爱的动摇、五种心的荒秽\footnote{五种心的荒秽,见\textbf{有财者经}第 19 颂的译注。}、八事的疑惑\footnote{八事的疑惑,见\textbf{宝经}第 233 颂的译注。},故\textbf{无动摇、无荒秽、无疑惑}。\end{enumerate}

\subsection\*{\textbf{483} {\footnotesize 〔PTS 478〕}}

\textbf{「他没有任何内在愚痴,以智见一切法,\\}
\textbf{「持最后身,且已证得无上吉祥的等觉,\\}
\textbf{「至此而成夜叉的清净,如来应得祭饼。」}

\begin{enumerate}\item \textbf{内在愚痴},即愚痴之因、愚痴之缘,即一切烦恼的同义语。\textbf{以智见一切法},即证得一切知智。因为此智通一切法,而世尊已得见此,以「我已得证」证得而住,因此说「以智见一切法」。\textbf{等觉},即阿罗汉性。\textbf{无上},即不与辟支佛、声闻共。\textbf{吉祥},即安稳、无祸害或祥瑞。\textbf{夜叉},即人。\textbf{清净},即洁净。因为于此,以无内在愚痴而无一切过失,因此断了轮回之因而持最后之身,以智见得生一切功德,因此得证无上等觉,从此更无别的应断与应证,故说「至此而成夜叉的清净」。\end{enumerate}

\subsection\*{\textbf{484} {\footnotesize 〔PTS 479〕}}

\textbf{「让我的献祭成为真实的献祭!当我得到这样的通达诸明者,\\}
\textbf{「因为梵天作证,请世尊接受我!请世尊享用我的祭饼!」}

\begin{enumerate}\item 如是说已,婆罗门对世尊更加净喜,作净喜状,说了此颂。其义为:当我此前对梵天献了火供,我不知道我的献祭是真实还是虚妄,而现在,\textbf{让我的}这\textbf{献祭成为真实的献祭}!他请求着说「让它成为真实的献祭」。\textbf{当我得到这样的通达诸明者},因为站立于此,我得到您这样的通达诸明者。\textbf{因为梵天作证},因为你就是梵天现前,所以,\textbf{请世尊接受我}!且接受后,\textbf{请世尊享用我的祭饼}!他手授以祭品的残留说道。\end{enumerate}

\subsection\*{\textbf{485} {\footnotesize 〔PTS 480〕}}

\textbf{「我不应受用吟颂之物,对诸正观者,婆罗门!此即非法,\\}
\textbf{「诸佛拒绝吟颂之物,法既存在,婆罗门!此即行事之道。}

\begin{enumerate}\item 于是,世尊以耕田婆罗豆婆遮经(第 81~82 颂)中所说的方法,说了两颂。\end{enumerate}

\subsection\*{\textbf{486} {\footnotesize 〔PTS 481〕}}

\textbf{「对整全者、大仙、漏尽者、恶作止息者,应以其它\\}
\textbf{「饮食给侍,因为他是希求福德者的良田。」}

\subsection\*{\textbf{487} {\footnotesize 〔PTS 482〕}}

\textbf{「善哉!世尊!我应如是了知当享用如我等者的供品者,\\}
\textbf{「在献牲时寻求他,遵从你的教法。」}

\begin{enumerate}\item 随后,婆罗门想「他自己不想要,那所说的『对整全者、大仙、漏尽者、恶作止息者,应以饮食给侍』的其他人是谁呢」,如是未解颂义而欲知此,说了此颂。
\item 这里,\textbf{善哉},即请求之义的不变词。\textbf{如是},即以你说的方式。\textbf{他},即此应供者。\textbf{在献牲时寻求},文本省略了「我应给侍 \textit{upaṭṭhaheyyaṃ}」。\textbf{你的教法},即你的教诫。
\item 这即是说:善哉!世尊!我应按照你的教诫如是了知,请告知我这整全者,他当享用如我等者的供品,且我应在献牲时寻求、给侍他,请示予我这样的应供者,如果你不享用的话。\end{enumerate}

\subsection\*{\textbf{488} {\footnotesize 〔PTS 483〕}}

\textbf{「他已离于愤激,他的心不污浊,\\}
\textbf{「且已解脱爱欲,他已除去昏沉。}

\begin{enumerate}\item 于是,世尊为以明了的方法显示这样的应供者,说了以下三颂。\end{enumerate}

\subsection\*{\textbf{489} {\footnotesize 〔PTS 484〕}}

\textbf{「界限的去除者,熟知生死者,\\}
\textbf{「具足寂默的牟尼,像这样前来献牲者,\footnote{此颂的四句都是下颂「敬礼、供养」的宾语。}}

\begin{enumerate}\item 这里,\textbf{界限的去除者},界即边界、善人的行为,以其限、其终为其他部分,故烦恼被称为界限,即去除彼等之义。也有人说,界限是佛陀所能调伏的有学及凡夫,即彼等的调伏者\footnote{vinetā 兼有「去除者」与「调伏者」的意思,义注两释之。}。\textbf{熟知生死者},即于此善巧于「如是生、如是死」。\textbf{具足寂默},即具足慧,或具足身寂默等。\end{enumerate}

\subsection\*{\textbf{490} {\footnotesize 〔PTS 485〕}}

\textbf{「调伏了高傲,你应合掌敬礼,\\}
\textbf{「应供养饮食,供品如是成功。」}

\begin{enumerate}\item \textbf{调伏了高傲},即调伏了某些恶觉者见到乞求者所现起的高傲,面露净喜之义。\end{enumerate}

\subsection\*{\textbf{491} {\footnotesize 〔PTS 486〕}}

\textbf{「佛陀您应得祭饼、无上的福田,\\}
\textbf{「一切世间的受献,对您的布施有大果报。」}

\begin{enumerate}\item 于是,婆罗门为赞叹世尊,说了此颂。这里,\textbf{受献},即应献,或者,以「从彼彼而来,应在此献祭」为受献,即是说作为所施的基础。
\item 此中其余及此前诸颂中所未解释者,由意义自明故,虽未解释亦能知晓,故不作解释。而此后则如耕田婆罗豆婆遮经中所述。\end{enumerate}

\textbf{于是,孙陀利迦婆罗豆婆遮婆罗门对世尊说:「希有!乔达摩君!希有!乔达摩君!好比,乔达摩君!能扶正被倾倒的,能揭示被遮蔽的,能给迷者指路,能在黑暗中持油灯,以使『具眼者能见色』,如是乔达摩君以种种方法阐明法。我皈依乔达摩君、法与比丘僧,愿我能在乔达摩君跟前出家,愿我能受具足!」}

\textbf{于是,孙陀利迦婆罗豆婆遮婆罗门……便成了众阿罗汉中的某个。}

