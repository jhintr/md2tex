\section{善器学童问}

\subsection\*{\textbf{1108} {\footnotesize 〔PTS 1101〕}}

\textbf{「舍弃家、断爱、不动、」尊者善器说,「舍弃欢喜、已度暴流、解脱、\\}
\textbf{「舍弃想的善慧者,我恳求!从龙象处听闻后,人们将从此处离去。}

\begin{enumerate}\item 这里,\textbf{舍弃家},即舍弃执著。\textbf{断爱},即断除六爱身。\textbf{不动},即不为世间法所震动。\textbf{舍弃欢喜},即舍弃对未来的色等的愿求。以上即同一渴爱,以称赞而在此从种种品类来说。\textbf{舍弃想},即舍弃两种想。\textbf{我恳求},即我极度请求。\textbf{从龙象处听闻后,人们将从此处离去},意即世尊!听闻了龙象你的话语,众人将从此石支提离去。\end{enumerate}

\subsection\*{\textbf{1109} {\footnotesize 〔PTS 1102〕}}

\textbf{「种种人们已从诸多国土聚集,英雄!期待着您的言语,\\}
\textbf{「请您对他们善加解释!因为这法已如是为你所知。」}

\begin{enumerate}\item \textbf{从诸多国土聚集},即从鸯伽等诸多国土聚集在此。\textbf{解释},即开示法。\end{enumerate}

\subsection\*{\textbf{1110} {\footnotesize 〔PTS 1103〕}}

\textbf{「应调伏一切对取著的渴爱!善器!」世尊说,「上方、下方、四旁及中间,\\}
\textbf{「因为任何他们在世间所执取者,魔罗便以此追随造物。}

\begin{enumerate}\item 于是,世尊为以随顺其意乐而开示法,说了二颂。这里,\textbf{对取著的渴爱},即取著色等的执取之爱,即是说爱取。\textbf{因为任何他们在世间所执取者},即任何在这些上方等类中所执取者。\textbf{魔罗便以此追随造物},即以取为缘所转起的业之行作所转起的结生蕴魔罗,便以此追随这有情。\end{enumerate}

\subsection\*{\textbf{1111} {\footnotesize 〔PTS 1104〕}}

\textbf{「所以,知晓着,具念的比丘不应执取一切世间中的任何,\\}
\textbf{「觉察着『执取的有情,这人类纠缠于死亡的境域』。」\\}

\begin{enumerate}\item \textbf{所以,知晓着},即所以,了知着这过患,或以无常等了知着诸行。\textbf{觉察着「执取的有情,这人类纠缠于死亡的境域」},即在以可执取之义而为执取的色等一切取著的世间中,觉察着这人类固著于死亡的境域。或者,觉察着执取的有情、执著于执取的补特伽罗,以及因为染著于执取,这人类固著于死亡的境域,无能超越于此,不应执取一切世间中的任何\footnote{原文 ādānasatte 一词,义注给出两种解释,一是将 satte 作「取著」解,再将其与上句的「一切世间」并列,义注中译作「一切取著的世间」,二则将 satte 作「补特伽罗」解,颂中的译文从第二种解释。}。其余一切处皆自明。
\item 如是,世尊仍以阿罗汉为顶点开示了此经。当开示终了,与先前一样,而有法的现观。\end{enumerate}

