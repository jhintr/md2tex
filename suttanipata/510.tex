\section{可教学童问}

\subsection\*{\textbf{1095} {\footnotesize 〔PTS 1088〕}}

\textbf{「若爱欲不居于他,」尊者可教说,「他也没有渴爱,\\}
\textbf{「且已度脱疑惑,那他的解脱是怎样的?」}

\begin{enumerate}\item 这里,\textbf{他的解脱是怎样的},即是问:他的解脱被希望是怎样的。\end{enumerate}

\subsection\*{\textbf{1096} {\footnotesize 〔PTS 1089〕}}

\textbf{「若爱欲不居于他,可教!」世尊说,「他也没有渴爱,\\}
\textbf{「且已度脱疑惑,那他已无更多的解脱。」}

\begin{enumerate}\item 现在,世尊为显示其无有其它的解脱,说了第二颂。这里,\textbf{他已无更多的解脱},即他没有其它的解脱。\end{enumerate}

\subsection\*{\textbf{1097} {\footnotesize 〔PTS 1090〕}}

\textbf{「他是离欲还是仍在希求?他具有智慧,还是作智慧想?\\}
\textbf{「释迦!请对我说明他!好让我能了知牟尼,一切眼者!」}

\begin{enumerate}\item 如是,当说「爱尽即是解脱」时,未能了解其义,以此颂再次发问。这里,\textbf{还是作智慧想},即还是以等至之智等的智作爱想或见想。\end{enumerate}

\subsection\*{\textbf{1098} {\footnotesize 〔PTS 1091〕}}

\textbf{「他离欲,不再希求,他具有智慧,不作智慧想,\\}
\textbf{「如是,可教!应知牟尼无所牵绊、不取著爱欲与有!」}

\begin{enumerate}\item 于是,世尊为对其宣说,说了第二颂。这里,\textbf{爱欲与有},即爱欲和有。其余一切处皆自明。
\item 如是,世尊仍以阿罗汉为顶点开示了此经。当开示终了,与先前一样,而有法的现观。\end{enumerate}

