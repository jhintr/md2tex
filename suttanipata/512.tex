\section{胶耳学童问}

\subsection\*{\textbf{1103} {\footnotesize 〔PTS 1096〕}}

\textbf{「听说英雄不欲求爱欲,」尊者胶耳说,「为问度过暴流,我来至无爱欲者,\\}
\textbf{「请说寂静的境地!俱生眼者!请如实地对我说!世尊!}

\begin{enumerate}\item 这里,\textbf{听说英雄不欲求爱欲},即我以「彼世尊亦即是……」等方法听说了英雄、不欲求爱欲者、佛陀。\textbf{我来至无爱欲者},即我前来问离爱欲的世尊。\textbf{俱生眼者},即与生俱来的一切知性之智眼者。\textbf{请对我说},即为再次请求而说。因为当请求时,应说千次,何况两次?\end{enumerate}

\subsection\*{\textbf{1104} {\footnotesize 〔PTS 1097〕}}

\textbf{「因为世尊征服了爱欲而行止,如同光辉的太阳以光芒之于大地,\\}
\textbf{「宏慧者!请对小慧的我说法!我好了知,\\}
\textbf{「于此舍弃生老。」}

\begin{enumerate}\item \textbf{光辉(的太阳)以光芒},即具足光辉(的太阳)以光芒征服。\textbf{我好了知,于此舍弃生老},即我能了知于此作为舍弃生老之法。\end{enumerate}

\subsection\*{\textbf{1105} {\footnotesize 〔PTS 1098〕}}

\textbf{「调伏对爱欲的贪求!胶耳!」世尊说,「视出离为安稳,\\}
\textbf{「你莫存有任何的执取或是扬弃。}

\begin{enumerate}\item 于是,世尊为对其宣说此法,说了以下三颂。这里,\textbf{视出离为安稳},即视涅槃及趣向涅槃的行道为安稳。\textbf{执取},即以爱、见摄取。\textbf{扬弃},即应扬弃者,即是说应释手者。\textbf{你莫存有},即你莫成为。\textbf{任何的},即你莫存有任何的贪等。\end{enumerate}

\subsection\*{\textbf{1106} {\footnotesize 〔PTS 1099〕}}

\textbf{「让先前的凋萎,你切莫有任何后来,\\}
\textbf{「如果你不执取中间,你将寂静而行。\footnote{此颂全同\textbf{执杖经}第 956 颂。}}

\begin{enumerate}\item \textbf{先前的},即与过去诸行相关的已生起的烦恼。\end{enumerate}

\subsection\*{\textbf{1107} {\footnotesize 〔PTS 1100〕}}

\textbf{「于一切名色离贪者,婆罗门!\\}
\textbf{「则无有能受制于死亡的诸漏。」}

\begin{enumerate}\item \textbf{婆罗门},即世尊称呼胶耳。
\item 如是,世尊仍以阿罗汉为顶点开示了此经。当开示终了,与先前一样,而有法的现观。\end{enumerate}

